\documentclass[12pt]{article}
\usepackage{pmmeta}
\pmcanonicalname{GiugaNumber}
\pmcreated{2013-03-22 15:50:22}
\pmmodified{2013-03-22 15:50:22}
\pmowner{Mravinci}{12996}
\pmmodifier{Mravinci}{12996}
\pmtitle{Giuga number}
\pmrecord{7}{37815}
\pmprivacy{1}
\pmauthor{Mravinci}{12996}
\pmtype{Definition}
\pmcomment{trigger rebuild}
\pmclassification{msc}{11D85}
\pmrelated{PrimaryPseudoperfectNumber}

\endmetadata

% this is the default PlanetMath preamble.  as your knowledge
% of TeX increases, you will probably want to edit this, but
% it should be fine as is for beginners.

% almost certainly you want these
\usepackage{amssymb}
\usepackage{amsmath}
\usepackage{amsfonts}

% used for TeXing text within eps files
%\usepackage{psfrag}
% need this for including graphics (\includegraphics)
%\usepackage{graphicx}
% for neatly defining theorems and propositions
%\usepackage{amsthm}
% making logically defined graphics
%%%\usepackage{xypic}

% there are many more packages, add them here as you need them

% define commands here
\begin{document}
A \emph{Giuga number} is a squarefree composite number $n$ such that each prime factor $p_i|({n \over {p_i}} - 1)$. For these numbers it then follows that $nB_{\phi(n)} \equiv -1 \bmod n$, (where $B_x$ is a Bernoulli number).

The first few Giuga numbers are 30, 858, 1722, 66198, 2214408306, 24423128562 (listed in sequence A007850 of Sloane's OEIS).

All known Giuga numbers are even and have at least three factors. An odd Giuga number would have to have at least twelve factors.
%%%%%
%%%%%
\end{document}
