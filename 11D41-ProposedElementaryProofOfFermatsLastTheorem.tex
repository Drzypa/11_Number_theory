\documentclass[12pt]{article}
\usepackage{pmmeta}
\pmcanonicalname{ProposedElementaryProofOfFermatsLastTheorem}
\pmcreated{2013-03-22 17:36:55}
\pmmodified{2013-03-22 17:36:55}
\pmowner{Mathprof}{13753}
\pmmodifier{Mathprof}{13753}
\pmtitle{proposed elementary proof of Fermat's last theorem}
\pmrecord{78}{40033}
\pmprivacy{1}
\pmauthor{Mathprof}{13753}
\pmtype{Proof}
\pmcomment{trigger rebuild}
\pmclassification{msc}{11D41}

% this is the default PlanetMath preamble.  as your knowledge
% of TeX increases, you will probably want to edit this, but
% it should be fine as is for beginners.

% almost certainly you want these
\usepackage{amssymb}
\usepackage{amsmath}
\usepackage{amsfonts}

% used for TeXing text within eps files
%\usepackage{psfrag}
% need this for including graphics (\includegraphics)
%\usepackage{graphicx}
% for neatly defining theorems and propositions
%\usepackage{amsthm}
% making logically defined graphics
%%%\usepackage{xypic}

% there are many more packages, add them here as you need them

% define commands here

\begin{document}
Michael Pogorsky has offered what is said to be an elementary proof of Fermat's
last theorem. Is the proof correct? The intent
of this entry is to show the proof up to the point at which it fails,
if there is such a point. New equation numbers will be used. 

\textbf{proof}
We assume that there are positive integers $a,b$ and $c$ such that

$$
a^n + b^n = c^n \quad \quad \quad (1)
$$.

We can assume without loss of generality that $a,b$ and $c$ are mutually coprime, so that in fact
they are also pairwise coprime. The proof is split into 3 major cases:
(1) $n$ is a prime greater than 2, (2) $n$ is divisible by a prime greater than 2,
and (3) $n$ is a power of 2.

\section{n is a prime greater than 2}


Write $c$ as
$$
c = a + k = b+f
$$
for some integers $k$ and $f$.

Then
$$
a^n + b^n =  (a+k)^n = (b+f)^n .
$$
Using the binomial theorem we can write 
$$
a^n = f(nb^{n-1} + \frac{1}{2}n(n-1)b^{n-2}f+ \cdots + f^{n-1})  \quad \quad \quad (2)
$$
and
$$
b^n = k(na^{n-1} + \frac{1}{2}n(n-1)a^{n-2}k + \cdots + k^{n-1}). \quad \quad \quad (3)
$$

Lemma 1. If $n$ is prime number then $n$ divides $\binom{n}{k}$ for $0 < k < n$.\\
The proof is easy.

Claim: $gcd(f,k) = 1$.  \\
Proof. A factor of $f$ and $k$ will also divide $a$ and $b$ by equations (2) 
and (3). But $a$ and $b$ are coprime, so the gcd of $f$ and $k$ must be 1.

Now write (2) as $a^n = fs$ for some integer $s$. 

From that point in the proof the exposition is somewhat unclear 
so I will attempt to rearrange the steps in what seems to be a better order. 
First, I introduce a lemma of my own. 
Write (3) as $b^n = kt$ for some integer $t$. 

Lemma 2. $gcd(f,s) = n^{\alpha}$ and   $gcd(k,t) = n^{\beta}$ for
some nonnegative integers  $\alpha$ and $\beta$. \\
Proof. Suppose $q$ divides $\gcd(f,s)$ where $q$ is a prime. Then $q$ divides $a$ and $s$. 
We can write $s=nb^{n-1} + fT$ for some integer $T$, so that $q$ divides $s-fT$ and therefore
$q$ divides $nb^{n-1}$. Hence $q$ divides $n$ or $q$ divides $b^{n-1}$. But if $q$ divides
$b^{n-1}$ then $q$ divides $b$, a contradiction. Hence $q$ divides $n$. But $n$ is a prime, so
$q=n$. From this we get that $gcd(f,s) = n^{\alpha}$ for some nonnegative integer $\alpha$.
Similarly, $gcd(k,t) = n^{\beta}$ for
some nonnegative integer $\beta$. \\

It is clear that at least one of $\alpha$ and $\beta$ is zero, otherwise $n$ divides $f$
and $k$. Without loss of generality, we can assume that $\alpha=0$.

The author now introduces what he calls version A and version B. I would prefer to call these
Case A and Case B. But there is no claim outstanding yet, so I have to defer the case split.
What seems to be the next main result is stated in the following lemma. \\

Lemma 3. There exist positive integers $p,u,v,w$ such  \\
1) $a=vp$, \\
2) if $\beta = 0$ then  \\
$$
a = uwv + v^n 
$$
$$
b= uwv + w^n 
$$
$$
c = uwv + v^n + w^n.
$$ 
and \\
3) if $\beta > 0$ then there is a  positive integers $g$ such that
$$ 
a= n^guwv + v^ n
$$
$$
b= n^guwv + n^{gn-1}w^n
$$
and
$$
c= n^guwv + v^n + n^{gn-1}w^n.
$$ \\
Proof. (1) We have $a^n=fs$,where $f$ and $s$ are coprime. By unique factorization of integers
it must be that $f=v^n$ and $s=p^n$ for some positive integers $p$ and $v$. It follows that
$a=pv$. \\
(2) Similarly, there are positive integers $w$ and $q$ such that $b=wq$, where $w^n=k$.
From $a+k=b+f$ we get
$$
vp + w^n = wq + v^n
$$
and after regrouping we have
$$
v(p-v^{n-1}) = w(q-w^{n-1})
$$
Since $gcd(f,k)=1$ it follows that $gcd(v,w) = 1$, so that 
$$
v | (q-w^{n-1})
$$
and 
$$
w | (p-v^{n-1}). 
$$
Hence 
$$
u:= \frac{p-v^{n-1}}{w} = \frac{q-w^{n-1}}{v} \quad \quad \quad (4)
$$
is an integer. Using $u$ we can now write
$a=uwv+v^n$, $b=uwv+w^n$, and $c=uwv+v^n+w^n$.  \\
(3) Since  $\alpha=0$, we have  $gcd(f,s)=1$. 
Since $\beta>0$, we can write
$$
k = k_1 n^{\tau}
$$
where $\tau>0$ and $gcd(k_1, n) = 1$. By Lemma 1 there is a positive  integer $c_i$ such that
$\binom{n}{i} = nc_i$ for $0 < i < n$. We can write
$$
t = \sum_{i=1}^n \binom{n}{i}a^{n-i}k^{i-1} = \sum_{i=1}^{n-1}nc_i a^{n-i}k^{i-1} + k^{n-1} 
$$ 
$$
 = \sum_{i=1}^{n-1}nc_i a^{n-i}k^{i-1} + nk_1n^{\tau-1}k^{n-2} = nT
$$
where $T=a^{n-1} + \frac{1}{2}(n-1)a^{n-2}k + \cdots + k_1 n^{\tau-1}k^{n-2}$.
Hence,
$$
b^n = kt = knT = k_1n^{\tau+1}T.
$$
Claim: $gcd(T,n) = 1 $. \\
This follows from the fact that $n$ divides all the terms of $T$ except the first term.
The first term is not divisible by $n$ because $k$ divides $n$ and therefore $n$ divides $b$
and $a$ and $b$ are coprime. \\
Claim: $gcd(T,k_1) = 1$. \\
This follows from the fact that $k$ divides $b$, so $k_1$ divides $b$, and $a$ and
$b$ are coprime.
By unique factorization of integers, then, it must be that there are positive integers
$q$, $w$ and $\lambda$ such that
$T = q^n$, $k_1 = w^n$ and $n^{\tau+1} = \lambda^n$. Since $n$ is a prime, it follows
that $\lambda = n^g$ for some positive integer $g$. Hence $gn= \tau+1$. 

It follows that
$b^n = w^nn^{gn}q^n$, so that $b= n^gwq$. From $a+k = b+ f $ we get
$$
vp + n^{gn-1}w^n = n^gwq + v^n
$$
which we can regroup to get
$$
v(p-v^{n-1}) = n^gw(q-n^{g(n-1)-1}w^{n-1}).
$$
Since $a$ and $b$ are coprime, it follows that $v$ and $n^gw$ are coprime.
Hence
$$
u:=\frac{p-v^{n-1}}{n^gw} = \frac{q-n^{g(n-1)-1}w^{n-1}}{v}
$$ 
is an integer. 
It follows that
$$
vp - v^n = n^gwq-n^{gn-1}w^n = n^guwv
$$
and one can now express $a$,$b$ and $c$ in terms of $u$,$v$,$w$.


Lemma 4. Let $u$ be the integer of Lemma 3. There is a monic polynomial $P$ 
with integer coefficients such that  \\
(a) $P(u)=0$,  \\
(b) the sum of the roots of $P$ is 0, and \\
(c) all coefficients of $P$ are divisible by $n$ except that when 
$\beta$ is positive  the last coefficient is not divisible by $n$. \\
Proof. We use the same cases as in Lemma 3. (1) In this case we have
$$
(uwv+v^n)^n + (uwv+w^n )^n - (uwv+v^n + w^n)^n = 0.
$$
The left hand side can be  expanded using the binomial theorem to get a 
polynomial $Q$ 
with coefficients that
depend on $v$ and $w$.  For the coefficient of $u^{n-i}$ we have to combine
$$
\binom{n}{i}(uwv)^{n-i}(v^n)^i + \binom{n}{i}(uwv)^{n-i}(w^n)^i - \binom{n}{i}(uwv)^{n-i}(v^n+w^n)^i
$$ 
$$
= u^{n-i}\binom{n}{i}(wv)^{n-i}(-(v^n+w^n)^i + (v^n)^i+(w^n)^i)  \quad \quad \quad (5)
$$
Clearly if $i=1$ this coefficient is 0. 
If $i=0$ the coefficient is $(wv)^n$. 
For the other terms we can write
them as
$$
\binom{n}{i}(wv)^{n-i}v^nw^n\sum_{j=1}^{i-1}\binom{i}{j}v^{n(j-1)}w^{n(i-j-1)}
$$ 
so that the coefficient is divisible by $(wv)^n$.
The coefficient is also divisible by $n$ if $1 \le i \le n$ by Lemma 1. 
So we set $P := Q/(wv)^n$ to get the conclusion for case (1). \\
(2) We proceed as in case (1). The left side of the equation 
$$
(n^guwv+v^n)^n + (n^guwv+n^{gn-1}w^n)^n - (n^guwv+v^n+n^{gn-1}w^n)^n =0
$$ 
can be expanded by the binomial theorem to get a polynomial
$Q$ with coefficients that depend on $v$ and $w$. 
It is clear that the leading term is
$(n^guwv)^n$. For the coefficient of $u^{n-i}$ we have to combine
$$
\binom{n}{i}(n^guwv)^{n-i}(v^n)^i+\binom{n}{i}(n^guwv)^{n-i}(n^{gn-1}w^n)^i-\binom{n}{i}(n^guwv)^{n-i}(v^n+n^{gn-1}w^n)^i
$$
$$
=u^{n-i}\binom{n}{i}(n^gwv)^{n-i}((v^n)^i+(n^{gn-1}w^n)^i-(v^n+n^{gn-1}w^n)^i . \quad \quad \quad (6)
$$
This form makes it clear that the coefficient is 0 if $i=1$ and divisible by $n$ if
$1 \le i < n$. 
If $i=0$, the coefficient is $(n^gwv)^n$. Equation (6) is equal to 
$$
u^{n-i}\binom{n}{i}(n^gwv)^{n-i} \sum_{j=1}^{i-1}v^{n(j-1)}(n^{gn-1}w^n)^{i-j-1}v^nn^{gn-1}w^n
$$
which shows that $n^{gn}w^nv^n$ divides each coefficient. Set
$P:=Q/(n^gwv)^n$ to get the conclusion for case (2).

Lemma 5. The polynomial $P$ of Lemma 4 has exactly one positive root. \\
Proof. By (5) and (6) the coefficients of $P$ are negative except for the leading 
coefficient. So there is exactly one sign change and by Descartes's rule of signs
there is exactly one positive root.

Definition. For each real root $u_i$ of $P$ we can define $a$, $b$ and $c$. 
(For example, $a=u_iwv + v^n$ and so on.) We say that a root $u_i$ is
\emph{acceptable} if the resulting $a,b,c$ are all positive integers.

Lemma 6.   The only acceptable root
of $P$ is $u$ and $u>0.$ \\
Proof. Suppose that $u_i$ is a nonpositive acceptable root. Then $a,b,c$ are all
positive and in case (1) we have
$$
a+b = 2u_iwv + v^n + w^n = c + u_iwv \le c,
$$
while in case (2) we have
$$
a+b= 2n^gu_iwv + v^n + n^{gn-1}w^n = c + n^gu_iwv \le c.
$$
But
$$
a^n + b^n < (a+b)^n \le c^n 
$$
which is a contradiction. Since $u$ is acceptable, it must be that
$u>0$.\\

The following lemma 7 is incorrect. \\
Lemma 7.  $n$ does not divide $a+b$. \\
Proof.  We use the cases of Lemma 3. (1)  We write \\
$$
a^n + b^n = (a+b)Q
$$
where 
$$
Q=\sum_{j=1}^na^{n-j}b^{j-1}(-1)^{j-1}.
$$
It is known that the common divisor if $a+b$ and $Q$ is $n$ and that
if $n^s || a+b$ then $n^{s+1} || a^n + b^n$. 
Hence, we can write
$$
a+b = n^s \delta
$$
and 
$$
Q=n \gamma
$$
where $gcd(n,\delta) = 1$, $gcd(n, \gamma) =1$ and $gcd(\delta, \gamma) =1$.
From $c^n = a^n+b^n = (a+b)Q = n^{s+1}\delta \gamma$
we get $s=n-1$, $n||c$ and 
$$
a+b=n^{n-1}\delta . \quad \quad \quad (7)
$$
Since $n$ divides $a+b$ and $c$ we have $n$ divides $2c-(a+b) = v^n + w^n$.
It is also known that
$$
v^n + w^n = (v+w)[(v+w)^{n-1}-nvw(v^{n-1}+ \cdots + w^{n-1})] \quad \quad \quad (8)
$$
so that $n$ divides $v+w$. But then from (8) again, $n^2$ divides $v^n+w^n$.
Now from (7) we have $n^2$ divides $a+b$.
Hence  
$$
2c = (a+b)+ v^n + w^n
$$
is divisible by $n^2$ and this is a contradiction. \\
(2) In this case 
$$
a+b = 2n^guwv + v^n + n^{gn-1}w^n
$$
so that if $n$ divides $a+b$ then $n$ divides $v^n$ and therefore $n$ divides $v$. 
From $a=vp$ we get then $n$ divides $a$ and therefore $n$ divides $b=a+b-a$.
But $a$ and $b$ are coprime, so we have a contradiction.

Since Lemma 7 is incorrect, Lemma 8 is also incorrect. \\
Lemma 8.  There are positive integers
$u_p$ and $c_p$ such that
$a+b = {u_p}^n$ and $c=u_pc_p$. \\
Proof.  We can write
$$
x^z+y^z = (x+y)Q
$$
where
$$
Q=\sum_{j=1}^zx^{z-j}y^{j-1}(-1)^{j-1}.
$$
It is an old result first attributed to Nicolas Malebranche (1638-1715) 
that if $x$ and $y$ are coprime and $d$ is a prime divisor of 
$x+y$ and $Q$ then $d$ divides $z$. I will give a proof of this here.
Define 
$$
Q_1 = Q = x^{z-1} + x^{z-2}y + \sum_{j=3}^zx^{z-j}y^{j-1}(-1)^{j-1}.
$$
Let
$$
Q_2 = Q_1 - x^{z-2}(x+y) = -2x^{z-2}y + x^{z-3}y^2+ \sum_{j=4}^zx^{z-j}y^{j-1}(-1)^{j-1}.
$$
Then $d$ divides $Q_2$.
Define 
$$
Q_3 = Q_2 + 2x^{z-3}y(x+y) = 3x^{z-3}y^2-x^{z-4}y^3 + \sum_{j=5}^zx^{z-j}y^{j-1}(-1)^{j-1}.
$$
In general 
$$
Q_n = (-1)^{n+1}nx^{z-n}y^{n-1}+x^{z-n-1}y^n + \sum_{j=n+2}^zx^{z-j}y^{j-1}(-1)^{j-1}
$$
and each $Q_n$ is divisible by $d$. Hence,
$$
Q_z = \pm zy^{z-1}
$$
will be divisible by $d$. But $d$ does not divide $y$ (since otherwise it would also divide
$x$, which would contradict that $x$ and $y$ are coprime). Hence, $d$ divides $z$. 
Using this result, we can say that
if
$$
a^n+b^n = (a+b)(a^{n-1}-a^{n-2}b+\cdots-ab^{n-2}+b^{n-1}) = (a+b)Q
$$
then $a+b$ and $Q$ are coprime. Because if $d$ is a prime divisor of each
then $d$ divides $n$, so $d=n$ and then $n$ divides $a+b$, which contradicts Lemma 7.
By unique factorization there  are positive integers $u_p$ and $c_p$
such that
$$
a+b = {u_p}^n
$$
and
$$
Q = {c_p}^n .
$$
Then $c^n = (a+b)Q = {u_p}^n {c_p}^n$.
Hence $c = u_p c_p$.

Lemma 9. Let $p,u,v,w$ be as in Lemma 3. Let $c_p$ be as in 
Lemma 8.  Suppose there are positive integers $h$and $q$  such that
$$
ah + bq = cp .
$$
Then one of the following possibilities holds: \\
(a) $h = h_k c, q = q_k c$ for some integers $h_k$ and $q_k$; \\
(b) $h = q = jc_p$ for some integer $j$; \\
(c) $h = jw^{n(n-1)}, q = j v^{n(n-1)}$ for some integer $j$;  \\
(d) $h = jb, q = jw^n$ for some integer $j$; \\
(e) $h = jv^n, q = -jw^n$ for some integer $j$; \\
(f) $h = jv^n, q = j(2uwv+w^n+2v^n)$ for some integer $j$; \\
(g) $h=j(2uwv+2w^n+v^n), q=jw^n$ for some integer $j$. \\
Proof. At this point I think the proof is incomplete since he does not prove the result, but rather
verifies that each of the possible solutions is indeed a solution. Later on,he needs to know that these
are the only solutions.


\section{n is divisible by a prime greater than 2}
If $n = mz$ where $z$ is a prime greater than 2, then
$$
(a^m)^z + (b^m)^z = (c^m)^z
$$ 
and we can apply the results of section 1 to conclude that no such $z$ can exist. 


\section{n is a power of 2}

It is known that if $n=4$ then Fermat's Last Theorem is true. For example,
see \cite{HW}. So if $n=2^t, t\ge 3$, then we can write
$$
(a^{2^{t-2}})^4 + (b^{2^{t-2}})^4 = (c^{2^{t-2}})^4
$$
which contradicts the theorem for $n=4$. 

\begin{thebibliography}{1}
\bibitem{HW} G.H. Hardy, E.M. Wright, \emph{An Introduction to the Theory of Numbers}, 5th ed., Oxford
University Press, page 191. 
\end{thebibliography}
 
%%%%%
%%%%%
\end{document}
