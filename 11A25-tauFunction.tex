\documentclass[12pt]{article}
\usepackage{pmmeta}
\pmcanonicalname{tauFunction}
\pmcreated{2013-03-22 13:30:16}
\pmmodified{2013-03-22 13:30:16}
\pmowner{Wkbj79}{1863}
\pmmodifier{Wkbj79}{1863}
\pmtitle{$\tau$ function}
\pmrecord{21}{34085}
\pmprivacy{1}
\pmauthor{Wkbj79}{1863}
\pmtype{Definition}
\pmcomment{trigger rebuild}
\pmclassification{msc}{11A25}
\pmsynonym{divisor function}{tauFunction}
\pmrelated{Divisor}
\pmrelated{DirichletHyperbolaMethod}
\pmrelated{2omeganLeTaunLe2Omegan}
\pmrelated{Divisibility}
\pmrelated{ValuesOfNForWhichVarphintaun}
\pmrelated{LambertSeries}
\pmrelated{ParityOfTauFunction}

\endmetadata

% this is the default PlanetMath preamble.  as your knowledge
% of TeX increases, you will probably want to edit this, but
% it should be fine as is for beginners.

% almost certainly you want these
\usepackage{amssymb}
\usepackage{amsmath}
\usepackage{amsfonts}

% used for TeXing text within eps files
%\usepackage{psfrag}
% need this for including graphics (\includegraphics)
%\usepackage{graphicx}
% for neatly defining theorems and propositions
%\usepackage{amsthm}
% making logically defined graphics
%%%\usepackage{xypic}

% there are many more packages, add them here as you need them

% define commands here
\begin{document}
\PMlinkescapeword{function}

The $\tau$ \emph{function}, also called the \emph{divisor function}, takes a positive integer as its input and gives the number of positive divisors of its input as its output.  For example, since $1$, $2$, and $4$ are all of the positive divisors of $4$, we have $\tau (4)=3$.  As another example, since $1$, $2$, $5$, and $10$ are all of the positive divisors of $10$, we have $\tau (10)=4$.

The $\tau$ function behaves according to the following two rules:

1.   If $p$ is a prime and $k$ is a nonnegative integer, then $\tau(p^k)=k+1$.

2.   If $\gcd(a,b)=1$, then $\tau(ab)=\tau(a)\tau(b)$.

Because these two rules hold for the $\tau$ function, it is a multiplicative function.

Note that these rules work for the previous two examples.  Since $2$ is prime, we have $\tau(4)=\tau(2^2)=2+1=3$.  Since $2$ and $5$ are distinct primes, we have $\tau(10)=\tau(2\cdot 5)=\tau(2)\tau(5)=(1+1)(1+1)=4$.

If $n$ is a positive integer, the number of \PMlinkname{prime factors}{UFD} of $x^n-1$ over $\mathbb{Q}[x]$ is $\tau(n)$.  For example, $x^9-1=(x^3-1)(x^6+x^3+1)=(x-1)(x^2+x+1)(x^6+x^3+1)$ and $\tau(9)=3$.

The $\tau$ function is extremely useful for studying cyclic rings.

The sequence $\{\tau(n)\}$ appears in the OEIS as sequence \PMlinkexternal{A000005}{http://www.research.att.com/~njas/sequences/A000005}.
%%%%%
%%%%%
\end{document}
