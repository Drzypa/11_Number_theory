\documentclass[12pt]{article}
\usepackage{pmmeta}
\pmcanonicalname{PropertiesOfRiemannXiFunction}
\pmcreated{2013-03-22 19:35:25}
\pmmodified{2013-03-22 19:35:25}
\pmowner{pahio}{2872}
\pmmodifier{pahio}{2872}
\pmtitle{properties of Riemann xi function}
\pmrecord{7}{42579}
\pmprivacy{1}
\pmauthor{pahio}{2872}
\pmtype{Result}
\pmcomment{trigger rebuild}
\pmclassification{msc}{11M06}
\pmrelated{RobinsTheorem}
\pmrelated{ExtraordinaryNumber}

\endmetadata

% this is the default PlanetMath preamble.  as your knowledge
% of TeX increases, you will probably want to edit this, but
% it should be fine as is for beginners.

% almost certainly you want these
\usepackage{amssymb}
\usepackage{amsmath}
\usepackage{amsfonts}

% used for TeXing text within eps files
%\usepackage{psfrag}
% need this for including graphics (\includegraphics)
%\usepackage{graphicx}
% for neatly defining theorems and propositions
 \usepackage{amsthm}
% making logically defined graphics
%%%\usepackage{xypic}

% there are many more packages, add them here as you need them

% define commands here

\theoremstyle{definition}
\newtheorem*{thmplain}{Theorem}

\begin{document}
The Riemann xi function, defined by 
\begin{align*}
\xi(s) \;:=\; \frac{s}{2}(s\!-\!1)\pi^{-\frac{s}{2}}\Gamma\!\left(\frac{s}{2}\right)\zeta(s),
\end{align*}
is an entire function having as zeros the nonreal zeros of the Riemann zeta function $\zeta$ and only them.

The modulus of the xi function is strictly increasing along every horizontal half-line lying
in any open right half-plane that contains no xi zeros.\, As well, the modulus decreases strictly along
every horizontal half-line in any zero-free, open left half-plane.

Taking into account the functional equation
$$\xi(1\!-\!s) \;=\; \xi(s)$$
it follows the reformulation of the Riemann hypothesis:\\

\textbf{Theorem.}\, The following three statements are equivalent.

(i). If $t$ is any fixed real number, then $|\xi(\sigma\!+\!it)|$ is increasing for\, $\frac{1}{2} < \sigma < \infty$.

(ii). If $t$ is any fixed real number, then $|\xi(\sigma\!+\!it)|$ is decreasing for\, $-\infty < \sigma < \frac{1}{2}$.

(iii). The Riemann hypothesis is true.

\begin{thebibliography}{8}
\bibitem{SD}{\sc Jonathan Sondow \& Christian Dumitrescu}: A monotonicity property Riemann's xi function and a reformulation of the Riemann Hypothesis. -- {\it Periodica Mathematica Hungarica} \textbf{60} (2010) 37--40.  Also available \PMlinkexternal{here}{http://arxiv.org/ftp/arxiv/papers/1005/1005.1104.pdf}.
\end{thebibliography}

%%%%%
%%%%%
\end{document}
