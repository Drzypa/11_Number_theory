\documentclass[12pt]{article}
\usepackage{pmmeta}
\pmcanonicalname{AsymptoticsOfCentralBinomialCoefficient}
\pmcreated{2013-03-22 17:40:50}
\pmmodified{2013-03-22 17:40:50}
\pmowner{rspuzio}{6075}
\pmmodifier{rspuzio}{6075}
\pmtitle{asymptotics of central binomial coefficient}
\pmrecord{18}{40120}
\pmprivacy{1}
\pmauthor{rspuzio}{6075}
\pmtype{Result}
\pmcomment{trigger rebuild}
\pmclassification{msc}{11B65}
\pmclassification{msc}{05A10}
\pmrelated{CatalanNumbers}

% this is the default PlanetMath preamble.  as your knowledge
% of TeX increases, you will probably want to edit this, but
% it should be fine as is for beginners.

% almost certainly you want these
\usepackage{amssymb}
\usepackage{amsmath}
\usepackage{amsfonts}

% used for TeXing text within eps files
%\usepackage{psfrag}
% need this for including graphics (\includegraphics)
%\usepackage{graphicx}
% for neatly defining theorems and propositions
%\usepackage{amsthm}
% making logically defined graphics
%%%\usepackage{xypic}

% there are many more packages, add them here as you need them

% define commands here

\begin{document}
By making use of the expression 
\[
 4^{-n} {2n \choose n} = \prod_{m=1}^n {2m - 1 \over 2m}  ,
\]
we may obtain estimates of the central binomial 
coefficient ${2n \choose n}$ for large values of $n$.
We begin by making some elementary algebraic 
manipulations of this product:
\[
 4^{-n} (2n+1) {2n \choose n} = {(2n+1) \prod_{m=1}^n (2m-1) \over
\prod_{m=1}^n 2m} = {\prod_{m=1}^n (2m+1) \over \prod_{m=1}^n 2m} =
\prod_{m=1}^n {2m+1 \over 2m}
\]
Then we multiply this by our previous expression, factor-by-factor:
\[
 16^{-n} (2n+1) {2n \choose n}^2 = \prod_{m=1}^n {(2m+1) (2m-1)
\over (2m)^2} .
\]
With a bit of rearrangement and manipulation, this gives
us the following formula for the central binomial coefficient,
\[
 {2n \choose n} = {4^n \over \sqrt{2n+1}} \sqrt{ \prod_{m=1}^n 
 \left( 1 - {1 \over 4 m^2} \right)} ,
\]
which we shall examine to obtain estimates of the central
binomial coefficient for large $n$.

To make use of this formula, we make two key observations
about the product which appears on the right-hand side.
First, since each of the terms in the product lies between
$0$ and $1$, the product is an decreasing function of $n$,
Thus, we have            
\[
 4^{-a} \sqrt{2a+1} {2a \choose a} > 
 4^{-b} \sqrt{2b+1} {2b \choose b}
\]
when $a < b$.  Secondly, the product converges in the limit
$n \to \infty$.  In fact, it turns out to be Wallis' product
for $\pi$, so we have
\[
 \lim_{n \to \infty}
 4^{-n} \sqrt{2n+1} {2n \choose n} =
 \sqrt{2 \over \pi} .
\]

This limit may be reread as an approximate formula when
$n$ is large:
\[
 {2n \choose n} \approx 
 \sqrt{2 \over \pi} 
 {4^n \over \sqrt{2n+1}}
\]
As an example, let us consider the case $n=10$.  The
exact answer is ${20 \choose 10} = 184756$.  The approximate
answer is $(4^{10} \sqrt{2}/\sqrt{21 \pi}) = 182570.38\ldots$,
which agrees with the exact answer to a percent.  Also
note that the estimate is smaller than the exact answer ---
this is a general feature which is due to the observation
made above that the product is an decreasing function of $n$.
Moreover, this observation also implies that the percentage 
error decreases as $n$ increases; in particular, the approximation
is good within a percent when $n > 10$.

It is worth noting that same result can be obtained from Stirling's 
formula.  In fact, one can deduce Stirling's formula by a similar 
line of reasoning.

With a little more work, wee can improve our approximation.
We begin by considering the product
\[
 \prod_{m=1}^n  {64 m^2 - 9 \over 64 m^2 - 25} .
\]
On the one hand, we can evaluate this product by factoring 
the numerator and denominator and cancelling terms:
\begin{align*}
 \prod_{m=1}^n {64 m^2 - 9 \over 64 m^2 - 25} &=
 \prod_{m=1}^n {(8m-3)(8m+3) \over (8m-5)(8m+5)} \\ &=
 {5 \cdot 11 \over 3 \cdot 13} \cdot
 {13 \cdot 19 \over 11 \cdot 21} \cdot
 {21 \cdot 27 \over 19 \cdot 29}
 \cdots \cdots
 {(8n-11)(8n+5) \over (8n-13)(8n-3)} \cdot
 {(8n-3)(8n+3) \over (nm-5)(8n+5)} \\ &=
 {5 \over 3} \cdot {8n+3 \over 8n+5}
\end{align*}
From this expression, it follows that the product converges
to $5/3$ as $n \to \infty$.
On the other hand, we can multiply this product by the product
we obtained earlier termwise:
\begin{align*}
 \left( 
  \prod_{m=1}^n {64 m^2 - 9 \over 64 m^2 - 25} 
 \right)
 \left( 
  \prod_{m=1}^n 
    \left( 1 - {1 \over 4 m^2} \right)
 \right) &=
 \prod_{m=1}^n
  {64 m^2 - 9 \over 64 m^2 - 25} \cdot
  {4 m^2 - 1 \over 4 m^2} \\ &=
 \prod_{m=1}^n
  {256 m^4 - 100 m^2 + 9 \over 256 m^4 - 100 m^2} \\ &=
 \prod_{m=1}^n
  \left(
   1 + {9 \over 256 m^4 - 100 m^2}
  \right)
\end{align*}
By taking limits on both sides, we conclude that
\[
  \prod_{m=1}^\infty
  \left(
   1 + {9 \over 256 m^4 - 100 m^2}
  \right) =
 {10 \over 3 \pi} .
\]
Embracing our earlier formula for the central binomial
coefficient with both hands, we obtain
\[
 16^{-n} \cdot
  {80 n^2 + 70 n + 15 \over 24 n + 15} 
  {2n \choose n}^2 =
 \prod_{m=1}^n
  \left(
   1 + {9 \over 256 m^4 - 100 m^2}
  \right) .
\]
Juggling terms from one hand to the other, we obtain
a new formula for the central binomial coefficient:
\[
 {2n \choose n} =
 4^n 
  \sqrt{24 n + 15 \over 80 n^2 + 70 n + 15} \,
  \sqrt{
   \prod_{m=1}^n
    \left(
     1 + {9 \over 256 m^4 - 100 m^2}
    \right)}
\]
Despite the increase in complexity, this formula is
actually an improvement over the old formula.  The reason
for this is that the product converges more rapidly
because the polynomial in the denominator is now
of the fourth order.

As before, we may take the limit $n \to \infty$:
\[
 \lim_{n \to \infty}
  4^{-n} \,
  \sqrt{16 n^2 + 14 n + 3 \over 8 n + 5} 
  {2n \choose n} =
 \sqrt{2 \over \pi}
\]
This gives us a new, improved asymptotic formula:
\[
 {2n \choose n} \approx
 \sqrt{2 \over \pi} \,
  4^n \,
  \sqrt{8 n + 5 \over 16 n^2 + 14 n + 3}
\]

To see just how good this formula is, let us revisit 
the case $n=10$.  Now, we get the approximation 
$(4^{10} \sqrt{170/1743\pi}) = 184756.93\ldots$.
Because the terms in the new product are greater
than unity, the approximation is an overestimate,
so we round it down to $184756$, which just so
happens to be the exact answer.  The approximation
is that good!
%%%%%
%%%%%
\end{document}
