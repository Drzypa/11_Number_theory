\documentclass[12pt]{article}
\usepackage{pmmeta}
\pmcanonicalname{OneHundredSixtythree}
\pmcreated{2013-03-22 16:54:16}
\pmmodified{2013-03-22 16:54:16}
\pmowner{PrimeFan}{13766}
\pmmodifier{PrimeFan}{13766}
\pmtitle{one hundred sixty-three}
\pmrecord{5}{39163}
\pmprivacy{1}
\pmauthor{PrimeFan}{13766}
\pmtype{Feature}
\pmcomment{trigger rebuild}
\pmclassification{msc}{11A99}
\pmsynonym{one hundred and sixty-three}{OneHundredSixtythree}

% this is the default PlanetMath preamble.  as your knowledge
% of TeX increases, you will probably want to edit this, but
% it should be fine as is for beginners.

% almost certainly you want these
\usepackage{amssymb}
\usepackage{amsmath}
\usepackage{amsfonts}

% used for TeXing text within eps files
%\usepackage{psfrag}
% need this for including graphics (\includegraphics)
%\usepackage{graphicx}
% for neatly defining theorems and propositions
%\usepackage{amsthm}
% making logically defined graphics
%%%\usepackage{xypic}

% there are many more packages, add them here as you need them

% define commands here

\begin{document}
Of Martin Gardner's April Fool's hoaxes, perhaps the most famous comes from the April 1975 issue of {\it Scientific American}, in which he claimed that Srinivasa Ramanujan had proven that $e^{\pi\sqrt{163}}$ is exactly equal to an integer. The truth is that it is not, but it comes surprisingly close, being approximately .0000000000007499274028018143 short of the nearest integer.

{\em One hundred sixty-three} also appears in an approximation of $\pi$, namely, $$\frac{2^9}{163} \approx \pi,$$ but this is barely correct to three decimal digits.

There are other qualities of 163 that are somewhat more exact, such as the fact that $$\sum_{i = 0}^4 {8 \choose i} = 163.$$

Kurt Heegner proved that $n = 163$ is the largest value for which the imaginary quadratic field $\mathbb{Q}(\sqrt{-n})$ has a unique factorization (thus 163 is the largest Heegner number).

Among the real integers, 163 is a prime number. As $163 + 0i$, it is also a prime on the complex plane, that is, a Gaussian prime.

163 is the eighth prime that is not a Chen prime. Nor is it a palindromic prime in any base from binary to base 161 (hence it's a strictly non-palindromic number).


%%%%%
%%%%%
\end{document}
