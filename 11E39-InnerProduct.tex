\documentclass[12pt]{article}
\usepackage{pmmeta}
\pmcanonicalname{InnerProduct}
\pmcreated{2013-03-22 12:13:39}
\pmmodified{2013-03-22 12:13:39}
\pmowner{djao}{24}
\pmmodifier{djao}{24}
\pmtitle{inner product}
\pmrecord{15}{31601}
\pmprivacy{1}
\pmauthor{djao}{24}
\pmtype{Definition}
\pmcomment{trigger rebuild}
\pmclassification{msc}{11E39}
\pmclassification{msc}{15A63}
\pmsynonym{Hermitian inner product}{InnerProduct}
\pmrelated{InnerProductSpace}
\pmrelated{HermitianForm}
\pmrelated{EuclideanVectorSpace}

% this is the default PlanetMath preamble.  as your knowledge
% of TeX increases, you will probably want to edit this, but
% it should be fine as is for beginners.

% almost certainly you want these
\usepackage{amssymb}
\usepackage{amsmath}
\usepackage{amsfonts}

% used for TeXing text within eps files
%\usepackage{psfrag}
% need this for including graphics (\includegraphics)
%\usepackage{graphicx}
% for neatly defining theorems and propositions
%\usepackage{amsthm}
% making logically defined graphics
%%%%\usepackage{xypic} 

% there are many more packages, add them here as you need them

% define commands here
\renewcommand{\v}{{{\bf v}}}
\newcommand{\w}{{{\bf w}}}
\newcommand{\0}{{{\bf 0}}}
\begin{document}
An \emph{inner product} on a vector space $V$ over a field $K$ (which must be either the field $\mathbb{R}$ of real numbers or the field $\mathbb{C}$ of complex numbers) is a function $(\ ,\ ): V \times V \longrightarrow K$ such that, for all $k_1,k_2 \in K$ and $\v_1, \v_2, \v, \w \in V$, the following properties hold:
\begin{enumerate}
\item $(k_1 \v_1 + k_2 \v_2, \w) = k_1 (\v_1, \w) + k_2 (\v_2, \w)$ (linearity\footnote{A small minority of authors impose linearity on the second coordinate instead of the first coordinate.})
\item $(\v, \w) = \overline{(\w, \v)}$, where $\overline{\ \ \ \ }$ denotes complex conjugation (conjugate symmetry)
\item $(\v, \v) \geq 0$, and $(\v, \v) = 0$ if and only if $\v = \0$ (positive definite)
\end{enumerate}

(Note: Rule 2 guarantees that $(\v,\v) \in \mathbb{R}$, so the inequality $(\v,\v) \geq 0$ in rule 3 makes sense even when $K=\mathbb{C}$.)

The standard example of an inner product is the dot product on $K^n$:
$$
((x_1,\dots,x_n), (y_1,\dots,y_n)) := \sum_{i=1}^n x_i \overline{y_i}
$$

Every inner product space is a normed vector space, with the norm being defined by $||\v|| := \sqrt{(\v,\v)}$.
%%%%%
%%%%%
%%%%%
\end{document}
