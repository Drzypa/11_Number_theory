\documentclass[12pt]{article}
\usepackage{pmmeta}
\pmcanonicalname{PadovanSequence}
\pmcreated{2013-03-22 16:37:21}
\pmmodified{2013-03-22 16:37:21}
\pmowner{PrimeFan}{13766}
\pmmodifier{PrimeFan}{13766}
\pmtitle{Padovan sequence}
\pmrecord{4}{38821}
\pmprivacy{1}
\pmauthor{PrimeFan}{13766}
\pmtype{Definition}
\pmcomment{trigger rebuild}
\pmclassification{msc}{11B39}

\endmetadata

% this is the default PlanetMath preamble.  as your knowledge
% of TeX increases, you will probably want to edit this, but
% it should be fine as is for beginners.

% almost certainly you want these
\usepackage{amssymb}
\usepackage{amsmath}
\usepackage{amsfonts}

% used for TeXing text within eps files
%\usepackage{psfrag}
% need this for including graphics (\includegraphics)
%\usepackage{graphicx}
% for neatly defining theorems and propositions
%\usepackage{amsthm}
% making logically defined graphics
%%%\usepackage{xypic}

% there are many more packages, add them here as you need them

% define commands here

\begin{document}
Construct a recurrence relation with initial terms $a_0 = 1$, $a_1 = 0$, $a_2 = 0$ and $a_n = a_{n - 3} + a_{n - 2}$ for $n > 2$. The first few terms of the sequence defined by this recurrence relation are: 1, 0, 0, 1, 0, 1, 1, 1, 2, 2, 3, 4, 5, 7, 9, 12, 16, 21, 28, 37, 49, 65, 86, 114, 151 (listed in A000931 of Sloane's OEIS). This is the {\em Padovan sequence}, named after mathematician Richard Padovan. Its generating function is $$G(a(n); x) = \frac{1 - x^2}{1 - x^2 - x^3}$$.

It has been observed that in taking seven consecutive terms of this sequence, the sum of the squares of the first, third and seventh terms is equal to the sum of the squares of the second, fourth, fifth and sixth terms.

The $n$th Padovan number asymptotically matches the $n$th power of the plastic constant.
%%%%%
%%%%%
\end{document}
