\documentclass[12pt]{article}
\usepackage{pmmeta}
\pmcanonicalname{HeckeAlgebra}
\pmcreated{2013-03-22 14:08:13}
\pmmodified{2013-03-22 14:08:13}
\pmowner{olivierfouquetx}{2421}
\pmmodifier{olivierfouquetx}{2421}
\pmtitle{Hecke algebra}
\pmrecord{13}{35551}
\pmprivacy{1}
\pmauthor{olivierfouquetx}{2421}
\pmtype{Definition}
\pmcomment{trigger rebuild}
\pmclassification{msc}{11F11}
\pmclassification{msc}{20C08}
\pmrelated{ModularForms}
\pmrelated{AlgebraicNumberTheory}
\pmdefines{Hecke operator}
\pmdefines{Hecke algebra ${H}(G)$ of the group G}

% this is the default PlanetMath preamble.  as your knowledge
% of TeX increases, you will probably want to edit this, but
% it should be fine as is for beginners.

% almost certainly you want these
\usepackage{amssymb}
\usepackage{amsmath}
\usepackage{amsfonts}

% used for TeXing text within eps files
%\usepackage{psfrag}
% need this for including graphics (\includegraphics)
%\usepackage{graphicx}
% for neatly defining theorems and propositions
%\usepackage{amsthm}
% making logically defined graphics
%%%\usepackage{xypic}

% there are many more packages, add them here as you need them

\newcommand{\sldeuxz}{\textrm{SL}_{2}(\mathbb Z)}
\newcommand{\somme}[2]{\underset{#1}{\overset{#2}{\sum}}}
\begin{document}
Let $f$ be a modular form for $\Gamma$ a congruence subgroup of $\sldeuxz$.
\begin{equation}
f(z)=\somme{n=0}{\infty}a_{n}q^n
\end{equation}
where $q=e^{2i\pi \tau}$. 

For $m\in \mathbb N$, let $T_{m}f(z)=\somme{n=0}{\infty}b_{n}q^n$ with :
\begin{equation}
b_{n}=\somme{d|\gcd(m,n)}{}d^{k-1}a_{mn/d^2}
\end{equation}
In particular, for $p$ a prime, $T_{p}f(z)=\somme{n=0}{\infty}b_{n}q^n$ with;
\begin{equation}
b_{n}=a_{pn}+p^{k-1}a_{n/p}
\end{equation}
where $a_{n/p}=0$ if $n$ is not divisible by $p$.

The operator $T_n$ is a linear operator on the space of modular forms called a \emph{Hecke operator}.


The Hecke operators leave the space of modular forms and cusp forms invariant
and turn out to be self-adjoint for a scalar product called the Petersson 
scalar product. In particular they have real eigenvalues. Hecke operators
also satisfy multiplicative properties that are best summarized by the formal
identity:
\begin{equation}
\underset{n=1}{\overset{\infty}{\sum}}T_{n}n^{-s}=\underset{p}{\prod}(1-
T_{p}p^{-s}+p^{k-1-2s})^{-1}
\end{equation}
That equation in particular implies that $T_mT_n=T_nT_m$ whenever $\gcd(n,m)=1$.

The set of all Hecke operators is usually denoted $\mathbb T$ and is called the \emph{Hecke algebra}.

\subsection{Group algebra example}
\textbf{Definition 0.1}
Let $G_{lcd}$ be a locally compact totally disconnected group; then the 
\emph{Hecke algebra $\mathcal{H}(G_{lcd})$ of the group} $G_{lcd}$ is defined as the convolution algebra of 
locally constant complex-valued functions on $G_{lcd}$ with compact support. 


Such $\mathcal{H}(G)$ algebras play an important role in the theory of 
decomposition of group representations into tensor products.



%%%%%
%%%%%
\end{document}
