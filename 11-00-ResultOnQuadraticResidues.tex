\documentclass[12pt]{article}
\usepackage{pmmeta}
\pmcanonicalname{ResultOnQuadraticResidues}
\pmcreated{2013-03-22 16:08:09}
\pmmodified{2013-03-22 16:08:09}
\pmowner{gilbert_51126}{14238}
\pmmodifier{gilbert_51126}{14238}
\pmtitle{result on quadratic residues}
\pmrecord{25}{38207}
\pmprivacy{1}
\pmauthor{gilbert_51126}{14238}
\pmtype{Theorem}
\pmcomment{trigger rebuild}
\pmclassification{msc}{11-00}

% this is the default PlanetMath preamble.  as your knowledge
% of TeX increases, you will probably want to edit this, but
% it should be fine as is for beginners.

% almost certainly you want these
\usepackage{amssymb}
\usepackage{amsmath}
\usepackage{amsfonts}

% used for TeXing text within eps files
%\usepackage{psfrag}
% need this for including graphics (\includegraphics)
%\usepackage{graphicx}
% for neatly defining theorems and propositions
%\usepackage{amsthm}
% making logically defined graphics
%%%\usepackage{xypic}

% there are many more packages, add them here as you need them

% define commands here

\begin{document}
\textbf{Theorem.}
Let $p$ be an odd prime. Then $-3$ is a quadratic residue modulo $p$ if and only if $p \equiv 1\pmod{3}$.

\emph{Proof.}
Preliminary to the proof, we remark first that $-1$ is a quadratic residue modulo $p$, where $p$ is an odd prime, if and only if $p \equiv 1\pmod{4}$.

If $p \equiv 1\pmod{4}$ then
\[\left(\frac{-3}{p}\right) = \left(\frac{-1}{p}\right)\left(\frac{3}{p}\right) = \left(\frac{p}{3}\right).\]
Now if $p \equiv 3\pmod{4}$, then 
\[ \left(\frac{-3}{p}\right) = \left(\frac{-1}{p}\right)\left(\frac{3}{p}\right) = (-1)(-1)\left(\frac{p}{3}\right) = \left(\frac{p}{3}\right).\] 
Thus, $\displaystyle \left(\frac{-3}{p}\right) = \left(\frac{p}{3}\right)$, and $\displaystyle \left(\frac{p}{3}\right)=1$ if and only if $p \equiv 1\pmod{3}$. $\Box$


 
%%%%%
%%%%%
\end{document}
