\documentclass[12pt]{article}
\usepackage{pmmeta}
\pmcanonicalname{SalemNumber}
\pmcreated{2013-03-22 13:38:48}
\pmmodified{2013-03-22 13:38:48}
\pmowner{bbukh}{348}
\pmmodifier{bbukh}{348}
\pmtitle{Salem number}
\pmrecord{6}{34298}
\pmprivacy{1}
\pmauthor{bbukh}{348}
\pmtype{Definition}
\pmcomment{trigger rebuild}
\pmclassification{msc}{11R06}
\pmclassification{msc}{11J71}

\usepackage{amssymb}
\usepackage{amsmath}
\usepackage{amsfonts}

% used for TeXing text within eps files
%\usepackage{psfrag}
% need this for including graphics (\includegraphics)
%\usepackage{graphicx}
% for neatly defining theorems and propositions
%\usepackage{amsthm}
% making logically defined graphics
%%%\usepackage{xypic}
\begin{document}
Salem number is a real algebraic integer $\alpha>1$ whose algebraic conjugates all lie in the  unit disk $\{\,z\in\mathbb{C} \,\big|\, |z|\leq 1\,\}$ with at least one on the unit circle $\{\,z\in\mathbb{C}\,\big|\,|z|= 1\,\}$.

Powers of a Salem number $\alpha^n\ (n=1,2,\dotsc)$ are everywhere dense modulo $1$, but are not uniformly distributed modulo $1$.

The smallest known Salem number is the largest positive root of
\begin{equation*}
\alpha^{10}+\alpha^9-\alpha^7-\alpha^6-\alpha^5-\alpha^4-\alpha^3+\alpha+1=0.
\end{equation*}
%%%%%
%%%%%
\end{document}
