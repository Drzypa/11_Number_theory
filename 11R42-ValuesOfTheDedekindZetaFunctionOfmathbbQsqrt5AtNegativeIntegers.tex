\documentclass[12pt]{article}
\usepackage{pmmeta}
\pmcanonicalname{ValuesOfTheDedekindZetaFunctionOfmathbbQsqrt5AtNegativeIntegers}
\pmcreated{2013-03-22 16:01:29}
\pmmodified{2013-03-22 16:01:29}
\pmowner{alozano}{2414}
\pmmodifier{alozano}{2414}
\pmtitle{values of the Dedekind zeta function of $\mathbb{Q}(\sqrt{5})$ at negative integers}
\pmrecord{5}{38065}
\pmprivacy{1}
\pmauthor{alozano}{2414}
\pmtype{Example}
\pmcomment{trigger rebuild}
\pmclassification{msc}{11R42}
\pmclassification{msc}{11M06}

% this is the default PlanetMath preamble.  as your knowledge
% of TeX increases, you will probably want to edit this, but
% it should be fine as is for beginners.

% almost certainly you want these
\usepackage{amssymb}
\usepackage{amsmath}
\usepackage{amsthm}
\usepackage{amsfonts}

% used for TeXing text within eps files
%\usepackage{psfrag}
% need this for including graphics (\includegraphics)
%\usepackage{graphicx}
% for neatly defining theorems and propositions
%\usepackage{amsthm}
% making logically defined graphics
%%%\usepackage{xypic}

% there are many more packages, add them here as you need them

% define commands here

\newtheorem{thm}{Theorem}
\newtheorem{defn}{Definition}
\newtheorem{prop}{Proposition}
\newtheorem{lemma}{Lemma}
\newtheorem{cor}{Corollary}

\theoremstyle{definition}
\newtheorem{exa}{Example}

% Some sets
\newcommand{\Nats}{\mathbb{N}}
\newcommand{\Ints}{\mathbb{Z}}
\newcommand{\Reals}{\mathbb{R}}
\newcommand{\Complex}{\mathbb{C}}
\newcommand{\Rats}{\mathbb{Q}}
\newcommand{\Gal}{\operatorname{Gal}}
\newcommand{\Cl}{\operatorname{Cl}}
\begin{document}
As an example of the \PMlinkid{parent entry}{8064}, we exhibit a few values of the Dedekind zeta function $\zeta(s,K)$ associated to the number field $K=\Rats(\sqrt{5})$. The following are the first few values of $\zeta(1-k,K)$ for $k\geq 2$ even ($\zeta(1-k,K)=0$ for odd $k$): 

\small

\begin{center}
\begin{tabular}{|c|c|}
  \hline
  % after \\: \hline or \cline{col1-col2} \cline{col3-col4} ...
  $k$ & $\zeta(1-k,K)$ \\
  \hline
  $2$ & $1/30$ \\
  $4$ & $1/60$ \\
  $6$ & $67/630$ \\
  $8$ & $361/120$ \\
  $10$ & $412751/1650$ \\
  $12$ & $795286411/16380$ \\
  $14$ & $568591843/30$ \\
$16$ & $54701427071177/4080$\\
$18$ & $571363169189645713/35910$\\
$20$ & $98510726938027364651/3300$\\
$22$ & $58282448789678207092153/690$\\
$24$ & $11364600197977872303826339891/32760$\\
$26$ & $60097244486962154421889002337/30$\\
$28$ & $27553534229181632149212403498558667/1740$\\ 
$30$ & $179897691732312705009829008469165679351567/1074150$\\
$32$ & $18959952687425212997198188226124948760804457/8160$\\
$34$ & $1246857198273009466623646468891606914655865453/30$\\
$36$ & $32613724432434114882079789497713006473885167100345949413/34545420$\\
$38$ & $806046617423971060143933083079087480165364733020665319/30$\\
$40$ & $256143677142597053855778958609943339670281438740737609308546491/270600$\\
$42$ & $7744103059592598379932370402622990883812122603998003590194837592521/189630$\\
$44$ & $2948486831915768046445018911155037347932306554653422623861453575795893/1380$\\
$46$ & $189448047879825594335580602295205397993735968958688272066548049250586050109/1410$\\
$48$ & $11220849734737403891785284646197712257525539376899788451604867938805644674521295667/1113840$\\
$50$ & $7374830701193124787713243044624034212252297046043499899645907770420650321652018100351/8250$\\
 \hline
\end{tabular}
\end{center}
%%%%%
%%%%%
\end{document}
