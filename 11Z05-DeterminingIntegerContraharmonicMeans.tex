\documentclass[12pt]{article}
\usepackage{pmmeta}
\pmcanonicalname{DeterminingIntegerContraharmonicMeans}
\pmcreated{2013-11-19 18:13:25}
\pmmodified{2013-11-19 18:13:25}
\pmowner{pahio}{2872}
\pmmodifier{pahio}{2872}
\pmtitle{determining integer contraharmonic means}
\pmrecord{15}{42183}
\pmprivacy{1}
\pmauthor{pahio}{2872}
\pmtype{Algorithm}
\pmcomment{trigger rebuild}
\pmclassification{msc}{11Z05}
\pmclassification{msc}{11A05}
\pmclassification{msc}{11D09}
\pmclassification{msc}{11D45}
\pmrelated{LinearFormulasForPythagoreanTriples}

\endmetadata

% this is the default PlanetMath preamble.  as your knowledge
% of TeX increases, you will probably want to edit this, but
% it should be fine as is for beginners.

% almost certainly you want these
\usepackage{amssymb}
\usepackage{amsmath}
\usepackage{amsfonts}

% used for TeXing text within eps files
%\usepackage{psfrag}
% need this for including graphics (\includegraphics)
%\usepackage{graphicx}
% for neatly defining theorems and propositions
 \usepackage{amsthm}
% making logically defined graphics
%%%\usepackage{xypic}

% there are many more packages, add them here as you need them

% define commands here

\theoremstyle{definition}
\newtheorem*{thmplain}{Theorem}

\begin{document}
For determining effectively values $c$ of integer contraharmonic means of two positive integers $u$ and $v$ 
($1\,<\,u\,<\,v$), it's convenient to start from the 
\PMlinkescapetext{formula} (7) in the 
\PMlinkname{parent entry}{IntegerContraharmonicMeans}:
\begin{align}
v \;=\; \frac{2u^2}{w}\!-\!u
\end{align}
where $w$ is any positive factor of $2u^2$ less than $u$.\, Substituting the above expression of $v$ to the defining expression 
$$c \;=\; \frac{u^2\!+\!v^2}{u\!+\!v}$$
of $c$, this gets the form
\begin{align}
c \;=\; \frac{2u^2}{w}\!-\!2u\!+\!w.
\end{align}
Hence one can use the formulae (1) and (2), giving in them for each desired $u$ the values $w$ of the positive factors of $2u^2$, beginning from\, $w := 1$\, and stopping before\, $w = u$.\\

The \PMlinkescapetext{formula} for the integer harmonic mean, corresponding (2), is simply
\begin{align}
h \;=\; 2u\!-\!w.
\end{align}


\textbf{Example.}\, In the following table one sees for\, $u = 36$\, all possible values of the parametre $w$ and the corresponding
values of $c$ and $h$; the pertinent values of $v$ are given, too.
\begin{center}
\begin{tabular}{||c||c|c|c|c|c|c|c|c|c|c|c|c|c||}
\hline
$w$ & $1$ & $2$ & $3$ & $4$ & $6$ & $8$ & $9$ & $12$ & $16$
& $18$ & $24$ & $27$ & $32$\\
\hline
$v$ & $2556$ & $1260$ & $828$ & $612$ & $396$ & $288$ & $252$ & $180$ & $126$
& $108$ & $72$ & $60$ & $45$\\
\hline
$c$ & $2521$ & $1226$ & $795$ & $580$ & $366$ & $260$ & $225$ & $156$ & $106$
& $90$ & $60$ & $51$ & $41$\\
\hline
$h$ & $71$ & $70$ & $69$ & $68$ & $66$ & $64$ & $63$ & $60$ & $56$
& $54$ & $48$ & $45$ & $40$\\
\hline
\end{tabular}
\end{center}
As one sees, the contraharmonic and the harmonic mean may differ considerably, but also the difference 1 is possible.


\begin{thebibliography}{8}
\bibitem{K}{\sc J. Pahikkala}: ``On contraharmonic mean and Pythagorean triples''.\, -- \emph{Elemente der Mathematik} \textbf{65}:2 (2010).
\end{thebibliography}

%%%%%
%%%%%
\end{document}
