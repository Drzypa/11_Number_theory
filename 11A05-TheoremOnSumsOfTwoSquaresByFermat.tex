\documentclass[12pt]{article}
\usepackage{pmmeta}
\pmcanonicalname{TheoremOnSumsOfTwoSquaresByFermat}
\pmcreated{2014-10-25 17:44:02}
\pmmodified{2014-10-25 17:44:02}
\pmowner{pahio}{2872}
\pmmodifier{pahio}{2872}
\pmtitle{theorem on sums of two squares by Fermat}
\pmrecord{13}{87625}
\pmprivacy{1}
\pmauthor{pahio}{2872}
\pmtype{Theorem}
\pmclassification{msc}{11A05}
\pmclassification{msc}{11A41}
\pmclassification{msc}{11A67}
\pmclassification{msc}{11E25}

% this is the default PlanetMath preamble.  as your knowledge
% of TeX increases, you will probably want to edit this, but
% it should be fine as is for beginners.

% almost certainly you want these
\usepackage{amssymb}
\usepackage{amsmath}
\usepackage{amsfonts}

% need this for including graphics (\includegraphics)
\usepackage{graphicx}
% for neatly defining theorems and propositions
\usepackage{amsthm}

% making logically defined graphics
%\usepackage{xypic}
% used for TeXing text within eps files
%\usepackage{psfrag}

% there are many more packages, add them here as you need them

% define commands here

\begin{document}
Suppose that an odd prime number $p$ can be written as the sum
$$p \;=\; a^2\!+\!b^2$$
where $a$ and $b$ are integers.\, Then they have to be coprime.\, 
We will show that $p$ is of the form $4n\!+\!1$.

Since\, $p \nmid b$,\, the congruence
$$bb_1 \;\equiv\; 1 \pmod p$$
has a solution $b_1$, whence
$$0 \;\equiv\; pb_1^2 \;=\; (ab_1)^2\!+\!(bb_1)^2 \;\equiv\; (ab_1)^2\!+\!1 \pmod p,$$
and thus
$$(ab_1)^2 \;\equiv\; -1 \pmod p.$$
Consequently, the Legendre symbol $\left(\frac{-1}{p}\right)$ is $+1$, i.e.
$$(-1)^{\frac{p-1}{2}} \;=\; 1.$$
Therefore, we must have
\begin{align}
p \;=\; 4n\!+\!1
\end{align}
where $n$ is a positive integer.  \\


Euler has first proved the following theorem presented by 
Fermat and containing also the converse of the above claim.

\textbf{Theorem 
(\PMlinkname{Thue's lemma}{ThuesLemma}).}\, An odd prime $p$ is 
uniquely expressible as sum of two squares of integers if and 
only if it satisfies (1) with an integer value of $n$.\\

The theorem implies easily the 

\textbf{Corollary.}\, If all odd prime factors of a positive 
integer are congruent to 1 modulo 4 then the integer is a sum
of two squares. (Cf. the proof of the parent article and the article 
``\PMlinkname{prime factors of Pythagorean hypotenuses}{primefactorsofpythagoreanhypotenuses}''.)



\end{document}
