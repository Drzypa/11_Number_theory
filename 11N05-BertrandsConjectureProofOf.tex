\documentclass[12pt]{article}
\usepackage{pmmeta}
\pmcanonicalname{BertrandsConjectureProofOf}
\pmcreated{2013-03-22 13:18:55}
\pmmodified{2013-03-22 13:18:55}
\pmowner{CWoo}{3771}
\pmmodifier{CWoo}{3771}
\pmtitle{Bertrand's conjecture, proof of}
\pmrecord{15}{33822}
\pmprivacy{1}
\pmauthor{CWoo}{3771}
\pmtype{Proof}
\pmcomment{trigger rebuild}
\pmclassification{msc}{11N05}

\endmetadata

% this is the default PlanetMath preamble.  as your knowledge
% of TeX increases, you will probably want to edit this, but
% it should be fine as is for beginners.

% almost certainly you want these
\usepackage{amssymb}
\usepackage{amsmath}
\usepackage{amsfonts}

% used for TeXing text within eps files
%\usepackage{psfrag}
% need this for including graphics (\includegraphics)
%\usepackage{graphicx}
% for neatly defining theorems and propositions
\usepackage{amsthm}
% making logically defined graphics
%%%\usepackage{xypic}

% there are many more packages, add them here as you need them

% define commands here
\newtheorem*{lemma*}{Lemma}
\begin{document}
\PMlinkescapeword{term}
\PMlinkescapeword{terms}
\PMlinkescapeword{conjecture}
\PMlinkescapeword{minimal}
\PMlinkescapeword{formula}
\PMlinkescapeword{information}

This is a version of Erd\H{o}s's proof as it appears in Hardy and Wright.

We begin with the following lemma.

\begin{lemma*}
Let $n$ be a positive integer and $p$ be a prime.  The highest power of $p$ dividing $n!$ is $\displaystyle\sum_j\left\lfloor\frac{n}{p^j}\right\rfloor$, where $\lfloor x\rfloor$ is the floor of $x$.
\end{lemma*}

\begin{proof}
Let $p^k$ divide $n!$ with $k$ as large as possible.  Every $p$th term of the sequence $1,\dots,n$ is divisible by $p$, so contributes a factor to $p^k$.  There are $\displaystyle\left\lfloor\frac{n}{p}\right\rfloor$ such factors.  Every $p^2$th term contributes an extra factor above that, providing $\displaystyle\left\lfloor\frac{n}{p^2}\right\rfloor$ new factors.  In general, the $p^j$th terms contribute $\displaystyle\left\lfloor\frac{n}{p^j}\right\rfloor$ extra factors to $p^k$.  So the highest power of $p$ dividing $n!$ is $\displaystyle\sum_j\left\lfloor\frac{n}{p^j}\right\rfloor$.
\end{proof}

We now prove the theorem.

\begin{proof}[Bertrand's conjecture]
Let $n>2$ be a minimal counterexample to the claim. Thus there is no prime $p$ such that $n<p<2n$.

In the sequence of primes
\[
2, 3, 5, 7, 13, 23, 43, 83, 163, 317, 631, 1259, 2503,
\]
each succeeding term is smaller than the double of its predecessor.  This shows
that $n\ge 2503$.

The \PMlinkname{binomial expansion}{BinomialTheorem} of $(1+1)^{2n}$ has $2n+1$ terms and has largest term $\binom{2n}{n}$.  Hence 
\[
\binom{2n}{n}\ge\frac{4^n}{2n+1}\ge\frac{4^n}{(2n)^2}.
\]

For a prime $p$ define $r(p,n)$ to be the highest power of $p$ dividing $\binom{2n}{n}$.  To compute $r(p,n)$, we apply the lemma to $(2n)!$ and $n!$.  We get that
\[
r(p,n)=\sum_{j\ge 1}\left\lfloor\frac{2n}{p^j}\right\rfloor
-2\sum_{j\ge 1}\left\lfloor\frac{n}{p^j}\right\rfloor
=\sum_{j\ge 1}\left(\left\lfloor\frac{2n}{p^j}\right\rfloor
-2\left\lfloor\frac{n}{p^j}\right\rfloor\right).
\]
The terms of the sum are all 0 or 1 and vanish when
$\displaystyle j>\left\lfloor\frac{\log(2n)}{\log p}\right\rfloor$, so
$\displaystyle r(p,n)\le\left\lfloor\frac{\log(2n)}{\log p}\right\rfloor$, that is,
$p^{r(p,n)}\le 2n$.

Now $\displaystyle\binom{2n}{n}=\prod_p p^{r(p,n)}$. By the inequality just proved, primes larger than $2n$ do not contribute to this product, and by assumption there are no primes between $n$ and $2n$. So
\[
\binom{2n}{n}=\prod_{\substack{1\le p \le n\\ p\text{\ prime}}}p^{r(p,n)}.
\]

For $n>p>\frac{2n}{3}$, $\frac{3}{2}>\frac{n}{p}>1$ and so for $p>\frac{2n}{3}>\sqrt{2n}$ we can apply the previous formula for $r(p,n)$ and find that it is zero. So for all $n>4$, the contribution of the primes larger than $\frac{2n}{3}$ is zero.

If $p>\sqrt{2n}$, all the terms for higher powers of $p$ vanish and $\displaystyle r(p,n)=\left\lfloor\frac{2n}{p}\right\rfloor-2\left\lfloor\frac{n}{p}\right\rfloor$.
Since $r(p,n)$ is at most 1, an upper bound for the
contribution for the primes between $\sqrt{2n}$ and $\frac{2n}{3}$ is the product of all primes smaller than $\frac{2n}{3}$.  This product is
$\exp\left(\vartheta(\frac{2n}{3})\right)$, where $\vartheta(n)$ is the Chebyshev function
\[
\vartheta(n)=\sum_{\substack{p\le n\\p\text{\ prime}}}\log p.
\]
There are at most $\sqrt{2n}$ primes smaller than $\sqrt{2n}$ and by the inequality $p^{r(p,n)}\le 2n$ their product is less than
$(2n)^{\sqrt{2n}}$.  Combining this information, we get the inequality
\[
\frac{4^n}{(2n)^2}\le\binom{2n}{n}\le(2n)^{\sqrt{2n}}\exp\left({\textstyle\vartheta(\frac{2n}{3})}\right).
\]
Taking logarithms and applying the \PMlinkname{upper bound of $n\log 4$ for $\vartheta(n)$}{UpperBoundOnVarthetan}, we obtain the inequality $\frac{n}{3}\log 4\le(\sqrt{2n}+2)\log(2n)$, which is false for sufficiently large $n$, say $n=2^{11}$.  This shows that $n<2^{11}$.

Since the conditions $n\ge 2503$ and $n<2^{11}$ are incompatible, there are no counterexamples to the claim.
\end{proof}

\begin{thebibliography}{9}
\bibitem{HW}
G.H. Hardy, E.M. Wright, \emph{An Introduction to the Theory of Numbers}, Oxford University Press, 1938.
\end{thebibliography}

%%%%%
%%%%%
\end{document}
