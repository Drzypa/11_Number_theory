\documentclass[12pt]{article}
\usepackage{pmmeta}
\pmcanonicalname{PrimitiveRoot}
\pmcreated{2013-03-22 16:04:33}
\pmmodified{2013-03-22 16:04:33}
\pmowner{CWoo}{3771}
\pmmodifier{CWoo}{3771}
\pmtitle{primitive root}
\pmrecord{12}{38133}
\pmprivacy{1}
\pmauthor{CWoo}{3771}
\pmtype{Definition}
\pmcomment{trigger rebuild}
\pmclassification{msc}{11-00}
\pmsynonym{primitive root modulo n}{PrimitiveRoot}
\pmsynonym{primitive element}{PrimitiveRoot}
\pmrelated{MultiplicativeOrderOfAnIntegerModuloM}
\pmrelated{PrimeResidueClass}
\pmrelated{UsingPrimitiveRootsAndIndexToSolveCongruences}

% this is the default PlanetMath preamble.  as your knowledge
% of TeX increases, you will probably want to edit this, but
% it should be fine as is for beginners.

% almost certainly you want these
\usepackage{amssymb}
\usepackage{amsmath}
\usepackage{amsfonts}

% used for TeXing text within eps files
%\usepackage{psfrag}
% need this for including graphics (\includegraphics)
%\usepackage{graphicx}
% for neatly defining theorems and propositions
%\usepackage{amsthm}
% making logically defined graphics
%%%\usepackage{xypic}

% there are many more packages, add them here as you need them

% define commands here

\begin{document}
Given any positive integer $n$, the group of units $U(\mathbb{Z}/n\mathbb{Z})$ of the ring $\mathbb{Z}/n\mathbb{Z}$ is a cyclic group iff $n$ is 4, $p^m$ or $2p^m$ for any odd positive prime $p$ and any non-negative integer $m$.\, A \emph{primitive root} is a generator of this group of units when it is cyclic.

Equivalently, one can define the integer $r$ to be a {\em primitive root modulo} $n$, if the numbers $r^0,\,r^1,\,\ldots,\,r^{n-2}$ form a reduced residue system modulo $n$.

For example, 2 is a primitive root modulo 5, since
$1,\; 2,\; 2^2 = 4,\; 2^3 = 8 \equiv 3 \pmod{5}$
are all with 5 coprime positive integers less than 5.\\

The generalized Riemann hypothesis implies that every prime number $p$ has a primitive root below $70(\ln p)^2$.

\begin{thebibliography}{8}
Wikipedia, ``Primitive root modulo n''
\end{thebibliography}
%%%%%
%%%%%
\end{document}
