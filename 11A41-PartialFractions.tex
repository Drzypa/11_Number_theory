\documentclass[12pt]{article}
\usepackage{pmmeta}
\pmcanonicalname{PartialFractions}
\pmcreated{2013-03-22 14:18:10}
\pmmodified{2013-03-22 14:18:10}
\pmowner{pahio}{2872}
\pmmodifier{pahio}{2872}
\pmtitle{partial fractions}
\pmrecord{34}{35761}
\pmprivacy{1}
\pmauthor{pahio}{2872}
\pmtype{Definition}
\pmcomment{trigger rebuild}
\pmclassification{msc}{11A41}
\pmsynonym{partial fractions of fractional numbers}{PartialFractions}
\pmrelated{CategoryOfAdditiveFractions}
\pmdefines{fractional number}

\endmetadata

% this is the default PlanetMath preamble.  as your knowledge
% of TeX increases, you will probably want to edit this, but
% it should be fine as is for beginners.

% almost certainly you want these
\usepackage{amssymb}
\usepackage{amsmath}
\usepackage{amsfonts}

% used for TeXing text within eps files
%\usepackage{psfrag}
% need this for including graphics (\includegraphics)
%\usepackage{graphicx}
% for neatly defining theorems and propositions
%\usepackage{amsthm}
% making logically defined graphics
%%%\usepackage{xypic}

% there are many more packages, add them here as you need them

% define commands here
\begin{document}
Every {\em fractional number}, i. e. such a rational number $\frac {m}{n}$ that the integer $m$ is not divisible by the integer $n$, can be decomposed to a sum of {\em partial fractions} as follows:
$$\frac{m}{n} \;=\; \frac{m_1}{p_1^{\nu_1}}+\frac{m_2}{p_2^{\nu_2}}+\cdots+\frac{m_t}{p_t^{\nu_t}}$$
Here, the $p_i$'s are distinct positive prime numbers, the $\nu_i$'s positive integers and the $m_i$'s some integers.\, Cf. the partial fractions of expressions.

\textbf{Examples:} 
 $$\frac{6}{289} \;=\; \frac{6}{17^2}$$ 
 $$-\frac{1}{24} \;=\; -\frac{3}{2^3}+\frac{1}{3^1}$$
 $$\frac{1}{504} \;=\; -\frac{1}{2^3}+\frac{32}{3^2}-\frac{24}{7^1}$$\\

How to get the numerators $m_i$ for decomposing a fractional number $\frac{1}{n}$ to partial fractions?\, First one can take the highest power $p^{\nu}$ of a prime $p$ which divides the denominator $n$.\, Then\, $n = p^{\nu}u$,\, where\, $\gcd{(u,\,p^{\nu})} = 1$.\, Euclid's algorithm gives some integers $x$ and $y$ such that 
                     $$1 \;=\; xu\!+\!yp^{\nu}.$$
Dividing this equation by $p^{\nu}u$ gives the \PMlinkescapetext{decomposition}
  $$\frac{1}{n} \;=\; \frac{1}{p^{\nu}u} \;=\; \frac{x}{p^{\nu}}\!+\!\frac{y}{u}.$$
If $u$ has more than one distinct prime factors, a similar procedure can be made for the fraction $\frac{y}{u}$, and so on.

\textbf{Note.}\, The numerators\, $m_1$, $m_2$, \ldots, $m_t$\, in the decomposition are not unique.\, E. g., we have also
          $$-\frac{1}{24} \;=\; -\frac{11}{2^3}+\frac{4}{3^1}.$$\\

Cf. the programme ``Murto'' (in Finnish) or ``Murd'' (in Estonian) or ``Bruch'' (in German) or ``Br\aa k'' (in Swedish) or ``Fraction''(in French) \PMlinkexternal{here}{http://www.wakkanet.fi/~pahio/ohjelmi.html}. 

%%%%%
%%%%%
\end{document}
