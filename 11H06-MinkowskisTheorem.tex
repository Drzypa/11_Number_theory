\documentclass[12pt]{article}
\usepackage{pmmeta}
\pmcanonicalname{MinkowskisTheorem}
\pmcreated{2013-03-22 13:51:42}
\pmmodified{2013-03-22 13:51:42}
\pmowner{alozano}{2414}
\pmmodifier{alozano}{2414}
\pmtitle{Minkowski's theorem}
\pmrecord{8}{34601}
\pmprivacy{1}
\pmauthor{alozano}{2414}
\pmtype{Theorem}
\pmcomment{trigger rebuild}
\pmclassification{msc}{11H06}
\pmsynonym{Minkowski's theorem on convex bodies}{MinkowskisTheorem}
%\pmkeywords{Minkowski}
%\pmkeywords{convex}
\pmrelated{LatticeInMathbbRn}
\pmrelated{ProofOfMinkowskisBound}
\pmdefines{Minkowski's theorem}
\pmdefines{fundamental parallelogram}

\endmetadata

% this is the default PlanetMath preamble.  as your knowledge
% of TeX increases, you will probably want to edit this, but
% it should be fine as is for beginners.

% almost certainly you want these
\usepackage{amssymb}
\usepackage{amsmath}
\usepackage{amsthm}
\usepackage{amsfonts}

% used for TeXing text within eps files
%\usepackage{psfrag}
% need this for including graphics (\includegraphics)
%\usepackage{graphicx}
% for neatly defining theorems and propositions
%\usepackage{amsthm}
% making logically defined graphics
%%%\usepackage{xypic}

% there are many more packages, add them here as you need them

% define commands here

\newtheorem{thm}{Theorem}
\newtheorem{defn}{Definition}
\newtheorem{prop}{Proposition}
\newtheorem{lemma}{Lemma}
\newtheorem{cor}{Corollary}
\begin{document}
Let $\mathcal{L} \in \mathbb{R}^2$ be a lattice in the sense of
number theory, i.e. a 2-dimensional free group over ${\mathbb{Z}}$
which generates $\mathbb{R}^2$ over $\mathbb{R}$. Let $w_1,w_2$ be
generators of the lattice $\mathcal{L}$. A set $\mathcal{F}$ of
the form
$$\mathcal{F}=\{(x,y)\in\mathbb{R}^2: (x,y)=\alpha w_1+\beta w_2,\quad 0\leq \alpha < 1,\quad 0\leq \beta <1 \}$$
is usually called a \emph{fundamental domain} or \emph{fundamental parallelogram} for the lattice $\mathcal{L}$.

\begin{thm}[Minkowski's Theorem]
Let $\mathcal{L}$ be an arbitrary lattice in $\mathbb{R}^2$ and
let $\Delta$ be the area of a fundamental parallelogram. Any
convex region $\mathfrak{K}$ symmetrical about the origin and of
area greater than $4\Delta$ contains points of the lattice
$\mathcal{L}$ other than the origin.
\end{thm}

More generally, there is the following $n$-dimensional analogue.

\begin{thm}
Let $\mathcal{L}$ be an arbitrary lattice in $\mathbb{R}^n$ and
let $\Delta$ be the area of a fundamental parallelopiped. Any
convex region $\mathfrak{K}$ symmetrical about the origin and of
volume greater than $2^n\Delta$ contains points of the lattice
$\mathcal{L}$ other than the origin.
\end{thm}
%%%%%
%%%%%
\end{document}
