\documentclass[12pt]{article}
\usepackage{pmmeta}
\pmcanonicalname{DoubleMersenneNumber}
\pmcreated{2013-03-22 17:38:50}
\pmmodified{2013-03-22 17:38:50}
\pmowner{PrimeFan}{13766}
\pmmodifier{PrimeFan}{13766}
\pmtitle{double Mersenne number}
\pmrecord{4}{40078}
\pmprivacy{1}
\pmauthor{PrimeFan}{13766}
\pmtype{Definition}
\pmcomment{trigger rebuild}
\pmclassification{msc}{11A41}

\endmetadata

% this is the default PlanetMath preamble.  as your knowledge
% of TeX increases, you will probably want to edit this, but
% it should be fine as is for beginners.

% almost certainly you want these
\usepackage{amssymb}
\usepackage{amsmath}
\usepackage{amsfonts}

% used for TeXing text within eps files
%\usepackage{psfrag}
% need this for including graphics (\includegraphics)
%\usepackage{graphicx}
% for neatly defining theorems and propositions
%\usepackage{amsthm}
% making logically defined graphics
%%%\usepackage{xypic}

% there are many more packages, add them here as you need them

% define commands here

\begin{document}
A {\em double Mersenne number} is a number of the form $2^{2^p - 1} - 1$. Put another way, negative one plus 2 raised to the power of a Mersenne number. The first few double Mersenne numbers (and the ones small enough to show here) are: 3, 7, 127, 170141183460469231731687303715884105727.

If a double Mersenne number is itself prime, then it is called a {\em double Mersenne~prime}. Obviously its index is then a Mersenne prime. The four double Mersenne numbers listed above are all primes, but as of today, $2^{170141183460469231731687303715884105727} - 1$ is a probable prime, despite an intense effort to find factors. (According to the Prime Pages, if it's composite, its least prime factor must be at least $5 \times 10^{51}$.
%%%%%
%%%%%
\end{document}
