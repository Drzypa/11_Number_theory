\documentclass[12pt]{article}
\usepackage{pmmeta}
\pmcanonicalname{IteratedTotientFunction}
\pmcreated{2013-03-22 16:33:09}
\pmmodified{2013-03-22 16:33:09}
\pmowner{CompositeFan}{12809}
\pmmodifier{CompositeFan}{12809}
\pmtitle{iterated totient function}
\pmrecord{5}{38737}
\pmprivacy{1}
\pmauthor{CompositeFan}{12809}
\pmtype{Definition}
\pmcomment{trigger rebuild}
\pmclassification{msc}{11A25}
\pmrelated{PerfectTotientNumber}

\endmetadata

% this is the default PlanetMath preamble.  as your knowledge
% of TeX increases, you will probably want to edit this, but
% it should be fine as is for beginners.

% almost certainly you want these
\usepackage{amssymb}
\usepackage{amsmath}
\usepackage{amsfonts}

% used for TeXing text within eps files
%\usepackage{psfrag}
% need this for including graphics (\includegraphics)
%\usepackage{graphicx}
% for neatly defining theorems and propositions
%\usepackage{amsthm}
% making logically defined graphics
%%%\usepackage{xypic}

% there are many more packages, add them here as you need them

% define commands here

\begin{document}
The {\em iterated totient function} $\phi^k(n)$ is $a_k$ in the recurrence relation $a_0 = n$ and $a_i = \phi(a_{i - 1})$ for $i > 0$, where $\phi(x)$ is Euler's totient function.

After enough iterations, the function eventually hits 2 followed by an infinite trail of ones. Ianucci et al define the ``class'' $c$ of $n$ as the integer such that $\phi^c(n) = 2$.

When the iterated totient function is summed thus: $$\sum_{i = 1}^{c + 1} \phi^i(n)$$ it can be observed that just as $2^x$  is a quasiperfect number when it comes to adding up proper divisors, it is also ``quasiperfect'' when adding up the iterated totient function. Quite unlike regular perfect numbers, $3^x$ (which are obviously odd) are ``perfect'' when adding up the iterated totient.

\begin{thebibliography}{2}
\bibitem{di} D. E. Ianucci, D. Moujie \& G. L. Cohen, ``On Perfect Totient Numbers'' {\it Journal of Integer Sequences}, {\bf 6}, 2003: 03.4.5
\bibitem{rg} R. K. Guy, {\it Unsolved Problems in Number Theory} New York: Springer-Verlag 2004: B42
\end{thebibliography}
%%%%%
%%%%%
\end{document}
