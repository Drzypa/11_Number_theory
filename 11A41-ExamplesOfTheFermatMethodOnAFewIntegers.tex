\documentclass[12pt]{article}
\usepackage{pmmeta}
\pmcanonicalname{ExamplesOfTheFermatMethodOnAFewIntegers}
\pmcreated{2013-03-22 16:39:28}
\pmmodified{2013-03-22 16:39:28}
\pmowner{PrimeFan}{13766}
\pmmodifier{PrimeFan}{13766}
\pmtitle{examples of the Fermat method on a few integers}
\pmrecord{5}{38863}
\pmprivacy{1}
\pmauthor{PrimeFan}{13766}
\pmtype{Example}
\pmcomment{trigger rebuild}
\pmclassification{msc}{11A41}

\endmetadata

% this is the default PlanetMath preamble.  as your knowledge
% of TeX increases, you will probably want to edit this, but
% it should be fine as is for beginners.

% almost certainly you want these
\usepackage{amssymb}
\usepackage{amsmath}
\usepackage{amsfonts}

% used for TeXing text within eps files
%\usepackage{psfrag}
% need this for including graphics (\includegraphics)
%\usepackage{graphicx}
% for neatly defining theorems and propositions
%\usepackage{amsthm}
% making logically defined graphics
%%%\usepackage{xypic}

% there are many more packages, add them here as you need them

% define commands here

\begin{document}
Take $n = 2039183$. The square root is approximately 1428, so that's what we set our iterator's initial state to. The test cap is 339865.

At $i = 1428$, we find that $\sqrt{1428^2 - 2039183} = 1$, clearly an integer.

Then, $1428 - 1 = 1427$ and $1428 + 1 = 1429$. By multiplication, we verify that $2039183 = 1427 \cdot 1429$, indeed. It is the product of a twin prime. By trial division, this would have taken 225 test divisions.

However, there are integers for which trial division performs much better than the Fermat method. For example, take $n = 1411041$. Our iterator starts at 1188 and the test cap is 235175.

At $i = 1188$, we find that $\sqrt{1188^2 - 1411041} \approx 17.4068952$. Similar frustration at 1189, 1190, etc.

It's not until we get to $i = 235175$ that we finally get $\sqrt{235175^2 - 1411041} = 235172$.

Then, $235175 - 235172 = 3$, and $235175 + 235172 = 470347$, and we verify that indeed $3 \cdot 470347 = 2039183$. This took 233987 instances of squaring, subtraction and square root extraction, which would have taken just 196 test divisions in trial division.

How about 4393547637856664251490043044051018234292171475232959? The square root is approximately 6628384145368058140794809 and the test cap is 732257939642777375248340507341836372382028579205495. At worst, the Fermat method could take as many as 732257939642777375248340500713452227013970438410686 instances of squaring, subtraction and square root extraction to give a result. Clearly a more sophisticated method is needed to factorize an integer of this magnitude.
%%%%%
%%%%%
\end{document}
