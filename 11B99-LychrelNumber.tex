\documentclass[12pt]{article}
\usepackage{pmmeta}
\pmcanonicalname{LychrelNumber}
\pmcreated{2013-03-22 12:57:09}
\pmmodified{2013-03-22 12:57:09}
\pmowner{akrowne}{2}
\pmmodifier{akrowne}{2}
\pmtitle{Lychrel number}
\pmrecord{10}{33312}
\pmprivacy{1}
\pmauthor{akrowne}{2}
\pmtype{Definition}
\pmcomment{trigger rebuild}
\pmclassification{msc}{11B99}

\usepackage{amssymb}
\usepackage{amsmath}
\usepackage{amsfonts}

%\usepackage{psfrag}
%\usepackage{graphicx}
%%%\usepackage{xypic}
\begin{document}
\PMlinkescapeword{range} \PMlinkescapeword{ranges}
A \emph{Lychrel number} is a number which never yields a palindrome in the iterative process of adding to itself a copy of itself with digits reversed.  For example, if we start with the number 983 we get:

\begin{itemize}
\item $983+389 = 1372$
\item $1372+2731 = 4103$
\item $4103+3014 = 7117$
\end{itemize}

So in 3 steps we get a palindrome, hence 983 is not a Lychrel number.  

In fact, it is not known if there exist any Lychrel numbers in base 10 (numbers colloquially called ``Lychrel numbers'' in base 10 are in fact just \emph{Lychrel candidates}).  However, in base 2 for example, there have been numbers proven to be Lychrel numbers\footnote{\cite{walker} informs us that Ronald Sprague has proved that the number 10110, for example, is a Lychrel number is base 2.}.  The first Lychrel candidate is 196:

\begin{itemize}
\item $196+691 = 887$
\item $887+788 = 1675$
\item $1675+5761 = 7436$
\item $7436+6347 = 13783$
\item $13783+38731 = 52514$
\item $52514+41525 = 94039$
\item $94039+93049 = 187088$
\item $187088+880781 = 1067869$
\item $\ldots$
\end{itemize}

This has been followed out to millions of digits, with no palindrome found in the sequence.

The following table gives the number of Lychrel candidates found within ascending ranges:

\begin{center}
\begin{tabular}{cc}
Range & Possible Lychrels \\
\hline
0 - 100 & 0 \\
100 - 1,000 & 2 \\
1,000 - 10,000 & 3 \\
10,000 - 100,000 & 69 \\
100,000 - 1,000,000 & 99 \\
10,000,000 - 100,000,000 & 1728 \\
100,000,000 - 1,000,000,000 & 29,813 \\
\end{tabular}
\end{center}

\begin{thebibliography}{9}

\bibitem{vl} Wade VanLandingham, \PMlinkexternal{196 And Other Lychrel Numbers}{http://www.p196.org/}
\bibitem{walker} John Walker, \PMlinkexternal{Three Years of Computing}{http://www.fourmilab.ch/documents/threeyears/threeyears.html}

\end{thebibliography}
%%%%%
%%%%%
\end{document}
