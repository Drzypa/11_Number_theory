\documentclass[12pt]{article}
\usepackage{pmmeta}
\pmcanonicalname{MillerRabinPrimeTest}
\pmcreated{2013-03-22 14:54:18}
\pmmodified{2013-03-22 14:54:18}
\pmowner{mathwizard}{128}
\pmmodifier{mathwizard}{128}
\pmtitle{Miller-Rabin prime test}
\pmrecord{9}{36589}
\pmprivacy{1}
\pmauthor{mathwizard}{128}
\pmtype{Algorithm}
\pmcomment{trigger rebuild}
\pmclassification{msc}{11Y11}
\pmsynonym{Rabin prime test}{MillerRabinPrimeTest}
\pmrelated{SolovayStrassenTest}
\pmrelated{FermatCompositenessTest}
\pmdefines{strong pseudoprime}

% this is the default PlanetMath preamble.  as your knowledge
% of TeX increases, you will probably want to edit this, but
% it should be fine as is for beginners.

% almost certainly you want these
\usepackage{amssymb}
\usepackage{amsmath}
\usepackage{amsfonts}

% used for TeXing text within eps files
%\usepackage{psfrag}
% need this for including graphics (\includegraphics)
%\usepackage{graphicx}
% for neatly defining theorems and propositions
\usepackage{amsthm}
% making logically defined graphics
%%%\usepackage{xypic}

% there are many more packages, add them here as you need them

% define commands here
\newtheorem{theorem}{Theorem}
\begin{document}
The Miller-Rabin prime test is a probabilistic test based on the following
\begin{theorem}
Let $N\geq 3$ be an odd number and $N-1=2^tn$, $n$ odd. If and only if for every $a\in\mathbb{Z}$ with $\gcd(a,N)=1$ we have
\begin{subequations}\label{eq:cond}
\begin{gather}
a^n\equiv 1\quad\text{mod }N\qquad\text{or}\\
a^{2^sn}\equiv-1\quad\text{mod }N
\end{gather}
\end{subequations}
for some $s\in\{0,\dots,t-1\}$, then $N$ is prime. If $N$ is not prime, then let $A$ be the set of all $a$ satisfying the above conditions. We have
$$\text{Card}(A)\leq\frac{1}{4}\varphi(N),$$
where $\varphi$ is Euler's Phi-function.
\end{theorem}
A composite number $N$ satisfying conditions (\ref{eq:cond}) for some $a\in\mathbb{Z}$ is called a \textit{strong \PMlinkescapetext{pseudoprime}} in the basis $a$. Note that this theorem states that there are no such things as Carmichael numbers for strong pseudoprimes (i.e. composite numbers that are strong pseudoprimes for all values of $a$).
From this theorem we can construct the following algorithm:
\begin{enumerate}
\item Choose a random $2\leq a\leq N-1$.
\item If $\gcd(a,N)\neq 1$, we have a non-trivial divisor of $N$, so $N$ is not prime.
\item If $\gcd(a,N)=1$, check if the conditions (\ref{eq:cond}) are satisfied. If not, $N$ is not prime, otherwise the probability that $N$ is not prime is less than $\frac{1}{4}$.
\end{enumerate}
In general the probability, that the test falsely reports $N$ as a prime is far less than $\frac{1}{4}$. Of course the algorithm can be applied several times in order to reduce the probability, that $N$ is not a prime but reported as one.

When comparing this test with the related Solovay-Strassen test one sees that this test is superior is several ways:
\begin{itemize}
\item There is no need to calculate the Jacobi symbol.
\item The amount of \textit{false witnesses}, i.e. numbers $a$ that report $N$ as a prime when it is not, is at most $\frac{N}{4}$ instead of $\frac{N}{2}$.
\item One can even show that $N$ which passes Miller-Rabin for some $a$ also passes Solovay-Strassen for that $a$, so Miller-Rabin is always better.
\end{itemize}
Note that when the test returns that $N$ is composite, then this is indeed the case, so it is really a compositeness test rather than a test for primality.
%%%%%
%%%%%
\end{document}
