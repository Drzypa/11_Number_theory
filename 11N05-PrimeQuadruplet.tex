\documentclass[12pt]{article}
\usepackage{pmmeta}
\pmcanonicalname{PrimeQuadruplet}
\pmcreated{2013-03-22 16:03:02}
\pmmodified{2013-03-22 16:03:02}
\pmowner{PrimeFan}{13766}
\pmmodifier{PrimeFan}{13766}
\pmtitle{prime quadruplet}
\pmrecord{7}{38101}
\pmprivacy{1}
\pmauthor{PrimeFan}{13766}
\pmtype{Definition}
\pmcomment{trigger rebuild}
\pmclassification{msc}{11N05}
\pmsynonym{prime quadruple}{PrimeQuadruplet}
\pmsynonym{prime quartet}{PrimeQuadruplet}

\endmetadata

% this is the default PlanetMath preamble.  as your knowledge
% of TeX increases, you will probably want to edit this, but
% it should be fine as is for beginners.

% almost certainly you want these
\usepackage{amssymb}
\usepackage{amsmath}
\usepackage{amsfonts}

% used for TeXing text within eps files
%\usepackage{psfrag}
% need this for including graphics (\includegraphics)
%\usepackage{graphicx}
% for neatly defining theorems and propositions
%\usepackage{amsthm}
% making logically defined graphics
%%%\usepackage{xypic}

% there are many more packages, add them here as you need them

% define commands here

\begin{document}
A \emph{prime quadruplet} is a set of four prime numbers, $p, p + 2, p + 6, p + 8$. In most cases, $p + 4$ is a multiple of 15. The only quadruplet for which this is not the case is 5, 7, 11, 13, which overlaps with the quadruplet 11, 13, 17, 19. Sometimes 2, 3, 5, 7 is referred to as a prime quadruplet.

If the twin prime conjecture is ever proven it is possible that it might neither prove nor disprove the prime quadruplet conjecture.

The sum of the reciprocals of the members of the prime quadruplets is Brun's constant for prime quadruplets, $B_4 \approx  0.87058838$.
%%%%%
%%%%%
\end{document}
