\documentclass[12pt]{article}
\usepackage{pmmeta}
\pmcanonicalname{FactorialBaseRepresentationOfFractions}
\pmcreated{2013-03-22 16:46:01}
\pmmodified{2013-03-22 16:46:01}
\pmowner{rspuzio}{6075}
\pmmodifier{rspuzio}{6075}
\pmtitle{factorial base representation of fractions}
\pmrecord{13}{38994}
\pmprivacy{1}
\pmauthor{rspuzio}{6075}
\pmtype{Definition}
\pmcomment{trigger rebuild}
\pmclassification{msc}{11A63}

% this is the default PlanetMath preamble.  as your knowledge
% of TeX increases, you will probably want to edit this, but
% it should be fine as is for beginners.

% almost certainly you want these
\usepackage{amssymb}
\usepackage{amsmath}
\usepackage{amsfonts}

% used for TeXing text within eps files
%\usepackage{psfrag}
% need this for including graphics (\includegraphics)
%\usepackage{graphicx}
% for neatly defining theorems and propositions
%\usepackage{amsthm}
% making logically defined graphics
%%%\usepackage{xypic}

% there are many more packages, add them here as you need them

% define commands here

\begin{document}
One can represent fractions as well as whole numbers using factorials
much in the same way that one has, say, a decimal representation of
both whole numbers and fractions.

Suppose that $x$ is a rational number.  For simplicity, let us assume
that $0 < x < 1$.  Then we can write
 \[x = \sum_{k=2}^N {d_k \over k!}\]
where $0 \le d_k < k$ for some integer $N$.  Unlike decimal representations
of fractions and, more generally representations with any fixed base,
factorial base representations of rational numbers all terminate.

Let us illustrate with some simple examples:
\begin{eqnarray*} 
\frac{1}{2} &=& \frac{1}{2!} \\
\frac{1}{3} &=& \frac{2}{3!} \\
\frac{2}{3} &=& \frac{1}{2!} + \frac{1}{3!} \\
\frac{1}{4} &=& \frac{1}{3!} + \frac{2}{4!} \\
\frac{3}{4} &=& \frac{1}{2!} + \frac{1}{3!} + \frac{2}{4!} \\
\frac{1}{5} &=& \frac{1}{3!} + \frac{1}{4!} + \frac{1}{5!} \\
\frac{2}{5} &=& \frac{2}{3!} + \frac{1}{4!} + \frac{3}{5!} \\
\frac{3}{5} &=& \frac{1}{2!} + \frac{2}{4!} + \frac{2}{5!} \\
\frac{4}{5} &=& \frac{1}{2!} + \frac{1}{3!} + \frac{3}{4!} +
\frac{1}{5!}
\end{eqnarray*}

We can also employ a more concise notation as is used in 
representing fractions in other bases and simply list
digits after a point.  Since we would need an infinite
supply of digits, we make the same compromise as when
writing factorial base representations of integers.  
With this convention, we than have the following table
of factorial base representations of fractions.

\begin{tabular}
{| c | l |}
1/2 & 0 . 1 \\ 
1/3 & 0 . 0 2 \\ 
2/3 & 0 . 1 1 \\ 
1/4 & 0 . 0 1 2 \\ 
3/4 & 0 . 1 1 2 \\ 
1/5 & 0 . 0 1 0 4 \\ 
2/5 & 0 . 0 2 1 3 \\ 
3/5 & 0 . 1 0 2 2 \\ 
4/5 & 0 . 1 1 3 1 \\ 
1/6 & 0 . 0 1 \\ 
5/6 & 0 . 1 2 \\
1/7 & 0 . 0 0 3 2 0 6 \\
2/7 & 0 . 0 1 2 4 1 5 \\
3/7 & 0 . 0 2 2 1 2 4 \\
4/7 & 0 . 1 0 1 3 3 3 \\
5/7 & 0 . 1 1 1 0 4 2 \\
6/7 & 0 . 1 2 0 2 5 1 \\
1/8 & 0 . 0 0 3 \\
3/8 & 0 . 0 2 1 \\
5/8 & 0 . 1 0 3 \\
7/8 & 0 . 1 2 1 \\
1/9 & 0 . 0 0 2 3 2 \\
2/9 & 0 . 0 1 1 1 4 \\
4/9 & 0 . 0 2 2 3 2 \\
5/9 & 0 . 1 0 1 1 4 \\
7/9 & 0 . 1 1 2 3 2 \\
8/9 & 0 . 1 2 1 1 4 \\
1/10 & 0 . 0 0 2 2 \\
3/10 & 0 . 0 1 3 1 \\
7/10 & 0 . 1 1 0 4 \\
9/10 & 0 . 1 2 1 3 \\
1/11 & 0 . 0 0 2 0 5 3 1 4 0 10 \\
2/11 & 0 . 0 1 0 1 4 6 2 8 1 9 \\
3/11 & 0 . 0 1 2 2 4 2 4 3 2 8 \\
4/11 & 0 . 0 2 0 3 3 5 5 7 3 7 \\
5/11 & 0 . 0 2 2 4 3 1 7 2 4 6 \\
6/11 & 0 . 1 0 1 0 2 5 0 6 5 5 \\
7/11 & 0 . 1 0 3 1 2 1 2 1 6 4 \\
8/11 & 0 . 1 1 1 2 1 4 3 5 7 3 \\
9/11 & 0 . 1 1 3 3 1 0 5 0 8 2 \\
10/11 & 0 . 1 2 1 4 0 3 6 4 9 1 \\
1/12 & 0 . 0 0 2 \\
5/12 & 0 . 0 2 2 \\
7/12 & 0 . 1 0 2 \\
11/12 & 0 . 1 2 2 
\end{tabular}
%%%%%
%%%%%
\end{document}
