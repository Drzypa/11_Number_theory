\documentclass[12pt]{article}
\usepackage{pmmeta}
\pmcanonicalname{ProofOfTranscendentalRootTheorem}
\pmcreated{2013-03-22 14:11:41}
\pmmodified{2013-03-22 14:11:41}
\pmowner{alozano}{2414}
\pmmodifier{alozano}{2414}
\pmtitle{proof of transcendental root theorem}
\pmrecord{6}{35625}
\pmprivacy{1}
\pmauthor{alozano}{2414}
\pmtype{Proof}
\pmcomment{trigger rebuild}
\pmclassification{msc}{11R04}
\pmrelated{AlgebraicElement}
\pmrelated{AlgebraicClosure}
\pmrelated{Algebraic}
\pmrelated{AlgebraicExtension}
\pmrelated{AFiniteExtensionOfFieldsIsAnAlgebraicExtension}

% this is the default PlanetMath preamble.  as your knowledge
% of TeX increases, you will probably want to edit this, but
% it should be fine as is for beginners.

% almost certainly you want these
\usepackage{amssymb}
\usepackage{amsmath}
\usepackage{amsthm}
\usepackage{amsfonts}

% used for TeXing text within eps files
%\usepackage{psfrag}
% need this for including graphics (\includegraphics)
%\usepackage{graphicx}
% for neatly defining theorems and propositions
%\usepackage{amsthm}
% making logically defined graphics
%%%\usepackage{xypic}

% there are many more packages, add them here as you need them

% define commands here

\newtheorem{thm}{Theorem}
\newtheorem{defn}{Definition}
\newtheorem{prop}{Proposition}
\newtheorem{lemma}{Lemma}
\newtheorem{cor}{Corollary}

% Some sets
\newcommand{\Nats}{\mathbb{N}}
\newcommand{\Ints}{\mathbb{Z}}
\newcommand{\Reals}{\mathbb{R}}
\newcommand{\Complex}{\mathbb{C}}
\newcommand{\Rats}{\mathbb{Q}}
\begin{document}
\begin{prop}
Let $F\subset K$ be a field extension with $K$ an algebraically closed field. Let $x\in K$ be transcendental over $F$. Then for any natural number $n\geq 1$, the element $x^{1/n}\in K$ is also transcendental over $F$. 
\end{prop}
\begin{proof}
Suppose $x$ is transcendental over a field $F$, and assume for a contradiction that $x^{1/n}$ is algebraic over $F$. Thus, there is a polynomial $P(y)\in F[y]$ such that $P(x^{1/n})=0$ (note that the polynomial $y^n-x$ is {\it not} a polynomial with coefficients in $F$, so $P(y)$ might be more involved). Then the field $H=F(x^{1/n})\subseteq K$ is a finite algebraic extension of $F$, and every element of $H$ is algebraic over $K$. However $x\in H$, so $x$ is algebraic over $F$ which is a contradiction.
\end{proof}
%%%%%
%%%%%
\end{document}
