\documentclass[12pt]{article}
\usepackage{pmmeta}
\pmcanonicalname{FormulaForSequencesSatisfyingSecondOrderRecurrenceRelations}
\pmcreated{2013-03-22 17:51:43}
\pmmodified{2013-03-22 17:51:43}
\pmowner{Wkbj79}{1863}
\pmmodifier{Wkbj79}{1863}
\pmtitle{formula for sequences satisfying second order recurrence relations}
\pmrecord{7}{40339}
\pmprivacy{1}
\pmauthor{Wkbj79}{1863}
\pmtype{Theorem}
\pmcomment{trigger rebuild}
\pmclassification{msc}{11B37}
\pmclassification{msc}{03D20}

\endmetadata

\usepackage{amssymb}
\usepackage{amsmath}
\usepackage{amsfonts}
\usepackage{pstricks}
\usepackage{psfrag}
\usepackage{graphicx}
\usepackage{amsthm}
%%\usepackage{xypic}

\newtheorem*{thm*}{Theorem}
\begin{document}
\PMlinkescapeword{term}
\PMlinkescapeword{terms}

\begin{thm*}
Let $\{s_n\}$ be a sequence such that there exist constants $A,B\in\mathbb{C}$ with $s_1=As_0$ and, for all positive integers $n$,
\[
s_{n+1}=As_n+Bs_{n-1}.
\]
Then for all nonnegative integers $n$, we have
\[
s_n=s_0\sum_{k=0}^{\lfloor\frac{n}{2}\rfloor} \binom{n-k}{k} B^k A^{n-2k},
\]
where $\lfloor\cdot\rfloor$ denotes the floor function and $\binom{a}{r}$ denotes the binomial coefficient.
\end{thm*}

\begin{proof}
This will be proven by induction.  Due to the presence of $\lfloor\frac{n}{2}\rfloor$, odd $n$ and even $n$ should be considered separately.  Thus, there are two base steps:
\begin{itemize}
\item If $n=0$, then
\[
s_0=s_0\sum_{k=0}^0 \binom{0-k}{k} B^k A^{0-2k}.
\]
\item If $n=1$, then
\[
s_1=As_0=s_0\sum_{k=0}^0 \binom{1-k}{k} B^k A^{1-2k}.
\]
\end{itemize}

Now assume that the theorem holds for all nonnegative integers less than or equal to some positive integer $n$.  We must show that the theorem holds for $n+1$.

Recall that
\[
s_{n+1}=As_n+Bs_{n-1}.
\]
We can use the induction hypothesis on $s_n$ and $s_{n-1}$:
\begin{align*}
s_{n+1} & =As_0\sum_{k=0}^{\lfloor\frac{n}{2}\rfloor} \binom{n-k}{k} B^k A^{n-2k} +Bs_0\sum_{k=0}^{\lfloor\frac{n-1}{2}\rfloor} \binom{n-1-k}{k} B^k A^{n-1-2k} \\
& =s_0\left( \sum_{k=0}^{\lfloor\frac{n}{2}\rfloor} \binom{n-k}{k} B^k A^{n+1-2k}
+\sum_{k=0}^{\lfloor\frac{n-1}{2}\rfloor} \binom{n-1-k}{k} B^{k+1} A^{n-1-2k} \right) \\
& =s_0\left( A^{n+1}+\sum_{k=1}^{\lfloor\frac{n}{2}\rfloor} \binom{n-k}{k} B^k A^{n+1-2k}
+\sum_{k=1}^{\lfloor\frac{n+1}{2}\rfloor} \binom{n-k}{k-1} B^k A^{n+1-2k} \right)
\end{align*}

First assume that $n$ is odd.  Then the first of the two summations has $\lfloor\frac{n}{2}\rfloor=\frac{n-1}{2}$ terms and the second summation has $\lfloor\frac{n+1}{2}\rfloor=\frac{n+1}{2}$ terms.  Therefore, we must split off the $k=\frac{n+1}{2}$ term from the second summation before combining the two summations:
\begin{align*}
s_{n+1} & =s_0\left( A^{n+1}+\sum_{k=1}^{\lfloor\frac{n}{2}\rfloor} \binom{n-k}{k} B^k A^{n+1-2k}
+\sum_{k=1}^{\lfloor\frac{n}{2}\rfloor} \binom{n-k}{k-1} B^k A^{n+1-2k} +B^{\frac{n+1}{2}} \right) \\
& =s_0\left( A^{n+1}+\sum_{k=1}^{\lfloor\frac{n}{2}\rfloor} \left[ \binom{n-k}{k}+\binom{n-k}{k-1} \right] B^k A^{n+1-2k} +B^{\frac{n+1}{2}} \right) \\
& =s_0\left( A^{n+1}+\sum_{k=1}^{\lfloor\frac{n}{2}\rfloor} \binom{n+1-k}{k} B^k A^{n+1-2k} +B^{\frac{n+1}{2}} \right) \\
& =s_0\sum_{k=0}^{\lfloor\frac{n+1}{2}\rfloor} \binom{n+1-k}{k} B^k A^{n+1-2k}
\end{align*}

Now assume that $n$ is even.  Then the two summations have the same number of terms and can be combined as is:
\begin{align*}
s_{n+1} & =s_0\left( A^{n+1}+\sum_{k=1}^{\lfloor\frac{n+1}{2}\rfloor} \left[ \binom{n-k}{k}+\binom{n-k}{k-1} \right] B^k A^{n+1-2k} \right) \\
& =s_0\sum_{k=0}^{\lfloor\frac{n+1}{2}\rfloor} \binom{n+1-k}{k} B^k A^{n+1-2k}
\end{align*}

Hence, the theorem holds for all nonnegative integers $n$.
\end{proof}
%%%%%
%%%%%
\end{document}
