\documentclass[12pt]{article}
\usepackage{pmmeta}
\pmcanonicalname{FourierCoefficients}
\pmcreated{2013-03-22 13:57:07}
\pmmodified{2013-03-22 13:57:07}
\pmowner{mathcam}{2727}
\pmmodifier{mathcam}{2727}
\pmtitle{Fourier coefficients}
\pmrecord{19}{34716}
\pmprivacy{1}
\pmauthor{mathcam}{2727}
\pmtype{Definition}
\pmcomment{trigger rebuild}
\pmclassification{msc}{11F30}
\pmrelated{GeneralizedRiemannLebesgueLemma}
\pmrelated{FourierSeriesOfFunctionOfBoundedVariation}
\pmrelated{DirichletConditions}
\pmdefines{Fourier series}
\pmdefines{trigonometric series}

\endmetadata

% this is the default PlanetMath preamble.  as your knowledge
% of TeX increases, you will probably want to edit this, but
% it should be fine as is for beginners.

% almost certainly you want these
\usepackage{amssymb}
\usepackage{amsmath}
\usepackage{amsfonts}

% used for TeXing text within eps files
%\usepackage{psfrag}
% need this for including graphics (\includegraphics)
%\usepackage{graphicx}
% for neatly defining theorems and propositions
%\usepackage{amsthm}
% making logically defined graphics
%%%\usepackage{xypic}

% there are many more packages, add them here as you need them

% define commands here
\begin{document}
Let $\mathbb{T}^n=\mathbb{R}^n/(2\pi\mathbb{Z})^n$ be the $n$-dimensional torus, let $\{\phi_k(x)\}_{k\in\mathbb{Z}^n}$ be an orthonormal basis for $L^2(\mathbb{T}^n)$, and suppose that $f(x)\in L^2(\mathbb{T}^n)$.

We can expand $f$ as a Fourier series

\begin{align*}
\sum_{k\in\mathbb{Z}^n}\hat{f}(k)\phi_k,
\end{align*}

and we call the numbers $\hat{f}(k)$ the \emph{Fourier coefficients} of $f$ with respect to the given basis.  In particular, the Fourier series for $f$ converges to $f$ in the $L^2$ norm.

The most basic incarnation of this is finding the Fourier coefficients of a Riemann integrable function with respect to the orthonormal basis given by the trigonometric functions:

Let $f$ be a Riemann integrable function from $[-\pi,\pi]$ to $\mathbb{R}$. Then the numbers
\begin{align*}
a_0 &= \frac{1}{2\pi}\int_{-\pi}^{\pi}f(x)dx,\\
a_n &= \frac{1}{\pi}\int_{-\pi}^{\pi}f(x)\cos(nx)dx,\\
b_n &= \frac{1}{\pi}\int_{-\pi}^{\pi}f(x)\sin(nx)dx.
\end{align*}
are called the Fourier coefficients of the function $f.$

The above can be repeated for a Lebesgue-integrable function $f$ if we use the Lebesgue integral in place of the Riemann integral.  This is the usual setting for modern Fourier analysis.

The trigonometric series 
$$ a_0 + \sum_{n=1}^{\infty}(a_n\cos(nx)+b_n\sin(nx))$$ is called the trigonometric series of the function $f$, or Fourier series of the function $f.$
%%%%%
%%%%%
\end{document}
