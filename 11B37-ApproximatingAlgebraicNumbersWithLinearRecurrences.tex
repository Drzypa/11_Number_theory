\documentclass[12pt]{article}
\usepackage{pmmeta}
\pmcanonicalname{ApproximatingAlgebraicNumbersWithLinearRecurrences}
\pmcreated{2013-03-22 16:53:03}
\pmmodified{2013-03-22 16:53:03}
\pmowner{rspuzio}{6075}
\pmmodifier{rspuzio}{6075}
\pmtitle{approximating algebraic numbers with linear recurrences}
\pmrecord{16}{39138}
\pmprivacy{1}
\pmauthor{rspuzio}{6075}
\pmtype{Definition}
\pmcomment{trigger rebuild}
\pmclassification{msc}{11B37}

% this is the default PlanetMath preamble.  as your knowledge
% of TeX increases, you will probably want to edit this, but
% it should be fine as is for beginners.

% almost certainly you want these
\usepackage{amssymb}
\usepackage{amsmath}
\usepackage{amsfonts}

% used for TeXing text within eps files
%\usepackage{psfrag}
% need this for including graphics (\includegraphics)
%\usepackage{graphicx}
% for neatly defining theorems and propositions
%\usepackage{amsthm}
% making logically defined graphics
%%%\usepackage{xypic}

% there are many more packages, add them here as you need them

% define commands here

\begin{document}
Linear recurrences can be used to obtain rational approximations for real
algebraic numbers.  Suppose that $\rho$ is the root of a polynomial
$p(x) = \sum_{k=0}^n c_k x^k$ with rational coefficients $c_k$ and 
further assume that the roots of $p$ are distinct and that all the
other roots of $p$ are strictly smaller in absolute value than $\rho$.

Consider the recursion
\[
 \sum_{k=0}^n c_k a_{m+k} = 0.
\]
As discussed in the parent entry, the solution of this recurrence
is
\[
 a_m = \sum_{k=1}^n b_k r_k^m,
\]
where $r_1, \ldots, r_n$ are the roots of $p$ --- let us agree that
$r_1 = \rho$ --- and the $b_k$'s are determined by the initial conditions
of the recurrence.  If $b_1 \neq 0$, we have
\[
 {a_{m+1} \over a_m} = 
 {b_1 \rho^{m+1} + \sum_{k=2}^n b_k r_k^{m+1} \over 
  b_1 \rho^m + \sum_{k=1}^n b_k r_k^m} =
 \rho {1 + \sum_{k=2}^n \left( {b_k \over b_1} \right) 
                  \left( {r_k \over \rho} \right)^{m+1} \over 
       1 + \sum_{k=2}^n \left( {b_k \over b_1} \right) 
                        \left( {r_k \over \rho} \right)^m}.
\]
Because $|\frac{r_k}{\rho}| < 1$ when $k > 1$, we have $\lim_{m \to \infty}
(r_k / \rho)^m = 0$, and hence
\[
 \lim_{m \to \infty} {a_{m+1} \over a_m} = \rho.
\]

To illustrate this method, we will compute the square root of two.  Now,
we cannot use the equation $x^2 - 2 = 0$ because it has two roots,
$+\sqrt{2}$ and $-\sqrt{2}$, which are equal in absolute value.  So what
we shall do instead is to use the equation $(x - 1)^2 = 2$, or $x^2 - 2 x
- 3 = 0$, whose roots are $1 - \sqrt{2}$ and $1 + \sqrt{2}$.  Since
$|1 + \sqrt{2}| > |1 - \sqrt{2}|$, we can use our method to approximate
the larger of these roots, namely $1 + \sqrt{2}$; to approximate $\sqrt{2}$,
we subtract $1$ from our answer.  The recursion we should use is
\[
 a_{m+2} - 2 a_{m+1} - a_m = 0
\]
or, moving terms around,
\[
 a_{m+2} = 2 a_{m+1} + a_m.
\]
If we choose $b_1 = - b_2 = 1/(2 \sqrt{2})$, then we have
\begin{align*}
 a_0 &= b_1 \left( 1 + \sqrt{2} \right )^0 +
       b_2 \left( 1 - \sqrt{2} \right )^0 = b_1 + b_2 = 0 \\
 a_1 &= b_1 \left( 1 + \sqrt{2} \right )^1 +
        b_2 \left( 1 - \sqrt{2} \right )^1 =
        b_1 + b_2 + (b_1 - b_2) \sqrt{2} = 1.
\end{align*}
Starting with these values, we obtain the following sequence:
\begin{align*}
 a_0 &= 0 \\
 a_1 &= 1 \\
 a_2 &= 2 \\
 a_3 &= 5 \\
 a_4 &= 12 \\
 a_5 &= 29 \\
 a_6 &= 70 \\
 a_7 &= 169 \\
 a_8 &= 408 \\
 a_9 &= 985 \\
 a_{10} &= 2738 \\
 a_{11} &= 5741 \\
 a_{12} &= 13860
\end{align*}
Therefore, as our approximations to $\sqrt{2}$, we have
\[
 1, \frac{3}{2}, \frac{7}{5}, \frac{17}{12}, \frac{41}{29}, 
 \frac{99}{70}, \frac{239}{169}, \frac{577}{408},
 \frac{1753}{985}, \frac{3003}{2738}, \frac{8119}{5741}.
\]

By making suitable transformations, one can compute all the roots
of a polynomial using this technique.  A way to do this is to start
with a rational number $h$ which is closer to the desired root than 
to the other roots, then make a change of variable $x = 1 / (y + h)$.

As an example, we shall examine the roots of $x^3 + 9 x + 1$.  
Approximating the polynomial by leaving off the constant term, we
guess that the roots are close to $0$, $+3i$, and $-3i$.  Since the
two complex roots are conjugates, it suffices to find one of them.

To look for the root near $0$, we make the change of variable $x = 
1 / y$ to obtain $y^3 + 9 y^2 + 1 = 0$, which gives the recursion
$a_{k+3} + 9 a_{k+2} + a_k = 0$.  Picking some initial values, this
recursion gives us the sequence
\[
0, 0, 1, -9, 81, -738, 6723, -61236, 557766, -5090401, \ldots,
\]
whence we obtain the approximations
\[
-\frac{1}{9}, -\frac{1}{9}, -\frac{81}{738}, -\frac{738}{6723},
-\frac{6723}{61236}, -\frac{61236}{557766}, 
-\frac{557766}{5090401}, \ldots.
\]

To look for the root near $3i$, we make the change of variable
$x = 1 / (y + 3 i)$ to obtain $y^3 + (9 + 9 i) y^2 + (-27 + 54 i) y
+ 82 - 27 i = 0$, which gives the recursion
\[
 a_{k+3} + (9 + 9 i) a_{k+2} + (-27 + 54 i) a_{k+1} + 
           (82 - 27 i) a_k = 0.
\]
Picking some initial values, this
recursion gives us the sequence
\[
 0,\, 0,\, 1,\, - 9 - 9 i,\, 27 + 108 i,\, -82 - 945 i,\, 
 - 225 + 11196 i,\, 44415 - 127953 i, \ldots,
\]
%%%%%
%%%%%
\end{document}
