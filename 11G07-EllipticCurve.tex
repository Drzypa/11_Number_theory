\documentclass[12pt]{article}
\usepackage{pmmeta}
\pmcanonicalname{EllipticCurve}
\pmcreated{2013-03-22 12:03:02}
\pmmodified{2013-03-22 12:03:02}
\pmowner{djao}{24}
\pmmodifier{djao}{24}
\pmtitle{elliptic curve}
\pmrecord{32}{31097}
\pmprivacy{1}
\pmauthor{djao}{24}
\pmtype{Definition}
\pmcomment{trigger rebuild}
\pmclassification{msc}{11G07}
\pmclassification{msc}{11G05}
\pmclassification{msc}{14H52}
%\pmkeywords{curve}
%\pmkeywords{variety}
%\pmkeywords{abelian variety}
\pmrelated{Isogeny}
\pmrelated{ComplexMultiplication}
\pmrelated{RankOfAnEllipticCurve}
\pmrelated{HeightFunction}
\pmrelated{LSeriesOfAnEllipticCurve}
\pmrelated{BirchAndSwinnertonDyerConjecture}
\pmrelated{JInvariant}
\pmrelated{MordellWeilTheorem}
\pmrelated{ConductorOfAnEllipticCurve}
\pmdefines{$\mathfrak{p}$-function}

\usepackage{amssymb}
\usepackage{amsmath}
\usepackage{amsthm}
\usepackage{amsfonts}
\usepackage{graphicx}
%%%\usepackage{xypic}
\usepackage{epsfig}
\newcommand{\p}{{\mathfrak{p}}}
\newcommand{\C}{\mathbb{C}}
\newcommand{\R}{\mathbb{R}}

\newtheorem{theorem}{Theorem}
\newtheorem{proposition}[theorem]{Proposition}
\newtheorem{lemma}[theorem]{Lemma}
\newtheorem{corollary}[theorem]{Corollary}

\theoremstyle{definition}
\newtheorem{definition}[theorem]{Definition}
\begin{document}
\section{Basics}
An {\em elliptic curve} over a field $K$ is a projective nonsingular algebraic curve $E$ over $K$ of genus 1 together with a point $O$ of $E$ defined over $K$. The word ``genus'' is taken here in the algebraic geometry sense, and has no relation with the topological notion of genus (defined as $1 - \chi/2$, where $\chi$ is the Euler characteristic) except when the field of definition $K$ is the complex numbers $\mathbb{C}$.

Using the Riemann-Roch theorem for curves, one can show that every elliptic curve $E$ is the zero set of a {\em Weierstrass equation} of the form
$$
E: y^2 + a_1 xy + a_3 y = x^3 + a_2 x^2 + a_4 x + a_6,
$$
for some $a_i \in K$, where the polynomial on the right hand side has no double roots. When $K$ has characteristic other than 2 or 3, one can further simpify this Weierstrass equation into the form
$$
E: y^2 = x^3 - 27 c_4 x - 54 c_6.
$$
The extremely strange numbering of the coefficients is an artifact of the process by which the above equations are derived. Also, note that these equation are for affine curves; to translate them to projective curves, one has to homogenize the equations (replace $x$ with $X/Z$, and $y$ with $Y/Z$).

\section{Examples}

We present here some pictures of elliptic curves over the field $\mathbb{R}$ of real numbers. These pictures are in some sense {\bf not} representative of most of the elliptic curves that people work with, since many of the interesting cases tend to be of elliptic curves over algebraically closed fields. However, curves over the complex numbers (or, even worse, over algebraically closed fields in characteristic $p$) are very difficult to graph in three dimensions, let alone two.

Figure~\ref{01-1} is a graph of the elliptic curve $y^2 = x^3 - x$.
\begin{figure}
\epsfig{file=graph1.eps}
\caption{Graph of $y^2 = x(x-1)(x+1)$}
\label{01-1}
\end{figure}

Figure~\ref{0-11} shows the graph of $y^2 = x^3 - x + 1$:
\begin{figure}
\epsfig{file=graph2.eps}
\caption{Graph of $y^2 = x^3 - x + 1$}
\label{0-11}
\end{figure}

Finally, Figures~\ref{001} and~\ref{000} are examples of algebraic curves that are {\bf not} elliptic curves. Both of these curves have singularities at the origin.
\begin{figure}
\epsfig{file=graph3.eps}
\caption{Graph of $y^2 = x^2 (x+1)$. Has two tangents at the origin.}
\label{001}
\end{figure}

\begin{figure}
\epsfig{file=graph4.eps}
\caption{Graph of $y^2 = x^3$. Has a cusp at the origin.}
\label{000}
\end{figure}

\section{The Group Law}

The points on an elliptic curve have a natural group structure, which makes the elliptic curve into an abelian variety. There are many equivalent ways to define this group structure; two of the most common are:
\begin{itemize}
\item Every Weyl divisor on $E$ is linearly equivalent to a unique divisor of the form $[P] - [O]$ for some $P \in E$, where $O \in E$ is the base point. The divisor class group of $E$ then yields a group structure on the points of $E$, by way of this correspondence.
\item Let $O \in E$ denote the base point. Then one can show that every line joining two points on $E$ intersects a unique third point of $E$ (after properly accounting for tangent lines as a multiple intersection). For any two points $P,Q \in E$, define their sum as:
\begin{enumerate}
\item Form the line between $P$ and $Q$; let $R$ be the third point on $E$ that intersects this line;
\item Form the line between $O$ and $R$; define $P+Q$ to be the third point on $E$ that intersects this line.
\end{enumerate}
This addition operation yields a group operation on the points of $E$ having the base point $O$ for identity.
\end{itemize}

\section{Elliptic Curves over $\mathbb{C}$}

Over the complex numbers, the general correspondence between algebraic and analytic theory specializes in the elliptic curves case to yield some very useful insights into the structure of elliptic curves over $\mathbb{C}$. The starting point for this investigation is the Weierstrass $\p$--function, which we define here.

\begin{definition}
A {\em lattice} in $\C$ is a subgroup $L$ of the additive group $\mathbb{C}$ which is generated by two elements $\omega_1, \omega_2 \in \C$ that are linearly independent over $\R$.
\end{definition}

\begin{definition}
For any lattice $L$ in $\C$, the {\em Weierstrass $\p_L$--function} of $L$ is the function $\p_L: \C \longrightarrow \C$ given by
$$
\p_L(z) := \frac{1}{z^2} + \sum_{\omega \in L \setminus \{0\}} \left( \frac{1}{(z-\omega)^2} - \frac{1}{\omega^2}\right).
$$
\end{definition}

When the lattice $L$ is clear from context, it is customary to suppress it from the notation and simply write $\p$ for the Weierstrass $\p$--function.

{\bf Properties of the Weierstrass $\p$--function:}

\begin{itemize}
\item $\p(z)$ is a meromorphic function with double poles at points in $L$.
\item $\p(z)$ is constant on each coset of $\C/L$.
\item $\p(z)$ satisfies the differential equation
$$
\p'(z)^2 = 4 \p(z)^3 - g_2 \p(z) - g_3
$$
where the constants $g_2$ and $g_3$ are given by
\begin{eqnarray*}
g_2 & := & 60 \sum_{\omega \in L \setminus \{0\}} \frac{1}{\omega^4} \\
g_3 & := & 140 \sum_{\omega \in L \setminus \{0\}} \frac{1}{\omega^6}
\end{eqnarray*}
\end{itemize}

The last property above implies that, for any $z \in \C/L$, the point $(\p(z),\p'(z))$ lies on the elliptic curve $E: y^2 = 4x^3 - g_2 x - g_3$. Let $\phi: \C/L \longrightarrow E$ be the map given by
$$
\phi(z) := 
\begin{cases}
(\p(z),\p'(z)) & z \notin L \\
\infty & z \in L
\end{cases}
$$
(where $\infty$ denotes the point at infinity on $E$). Then $\phi$ is actually a bijection (!), and moreover the map $\phi: \C/L \longrightarrow E$ is an isomorphism of Riemann surfaces as well as a group isomorphism (with the addition operation on $\C/L$ inherited from $\C$, and the elliptic curve group operation on $E$).

We can go even further: it turns out that {\bf every} elliptic curve $E$ over $\C$ can be obtained in this way from some lattice $L$. More precisely, the following is true:

\begin{theorem}
\begin{enumerate}
\item For every elliptic curve $E: y^2 = 4x^3 - b x - c$ over $\C$, there is a unique lattice $L \subset \C$ whose constants $g_2$ and $g_3$ satisfy $b = g_2$ and $c = g_3$.
\item Two elliptic curves $E$ and $E'$ over $\C$ are isomorphic if and only if their corresponding lattices $L$ and $L'$ satisfy the equation $L' = \alpha L$ for some scalar $\alpha \in \C$.
\end{enumerate}
\end{theorem}

\begin{thebibliography}{9}
\bibitem{husemoller} Dale Husemoller, {\em Elliptic Curves}. Springer--Verlag, New York, 1997.
\bibitem{milne} James Milne, {\em Elliptic Curves}, online course notes. \PMlinkexternal{http://www.jmilne.org/math/CourseNotes/math679.html}{http://www.jmilne.org/math/CourseNotes/math679.html}
\bibitem{silverman} Joseph H. Silverman, {\em The Arithmetic of Elliptic Curves}. Springer--Verlag, New York, 1986.
\end{thebibliography}
%%%%%
%%%%%
%%%%%
\end{document}
