\documentclass[12pt]{article}
\usepackage{pmmeta}
\pmcanonicalname{ConditionForPowerBasis}
\pmcreated{2013-03-22 17:49:56}
\pmmodified{2013-03-22 17:49:56}
\pmowner{pahio}{2872}
\pmmodifier{pahio}{2872}
\pmtitle{condition for power basis}
\pmrecord{9}{40302}
\pmprivacy{1}
\pmauthor{pahio}{2872}
\pmtype{Theorem}
\pmcomment{trigger rebuild}
\pmclassification{msc}{11R04}
\pmrelated{IntegralBasis}
\pmrelated{PowerBasis}
\pmrelated{CanonicalBasis}
\pmrelated{PropertiesOfDiscriminantInAlgebraicNumberField}

% this is the default PlanetMath preamble.  as your knowledge
% of TeX increases, you will probably want to edit this, but
% it should be fine as is for beginners.

% almost certainly you want these
\usepackage{amssymb}
\usepackage{amsmath}
\usepackage{amsfonts}

% used for TeXing text within eps files
%\usepackage{psfrag}
% need this for including graphics (\includegraphics)
%\usepackage{graphicx}
% for neatly defining theorems and propositions
 \usepackage{amsthm}
% making logically defined graphics
%%%\usepackage{xypic}

% there are many more packages, add them here as you need them

% define commands here

\theoremstyle{definition}
\newtheorem*{thmplain}{Theorem}

\begin{document}
\textbf{Lemma.}\, If $K$ is an algebraic number field of \PMlinkname{degree}{Degree} $n$ and the elements $\alpha_1,\,\alpha_2,\,\ldots,\,\alpha_n$ of $K$ can be expressed as linear combinations
\begin{align*}
\begin{cases}
\alpha_1 = c_{11}\beta_1+c_{12}\beta_2+\ldots+c_{1n}\beta_n\\
\alpha_2 = c_{21}\beta_1+c_{22}\beta_2+\ldots+c_{2n}\beta_n\\
\qquad\cdots\\
\alpha_n = c_{n1}\beta_1+c_{n2}\beta_2+\ldots+c_{nn}\beta_n
\end{cases}
\end{align*}
of the elements $\beta_1,\,\beta_2,\,\ldots,\,\beta_n$ of $K$ with rational coefficients $c_{ij}$, then the discriminants of $\alpha_i$ and $\beta_j$ are \PMlinkescapetext{related} by the equation
$$\Delta(\alpha_1,\,\alpha_2,\,\ldots,\,\alpha_n) = \det(c_{ij})^2\cdot\Delta(\beta_1,\,\beta_2,\,\ldots,\,\beta_n).\\$$

\textbf{Theorem.}\, Let $\vartheta$ be an algebraic integer of \PMlinkname{degree}{DegreeOfAnAlgebraicNumber} $n$.\, The set\, $\{1,\,\vartheta,\,\ldots,\,\vartheta^{n-1}\}$\, is an integral basis of $\mathbb{Q}(\vartheta)$ if the discriminant \,$d(\vartheta) := \Delta(1,\,\vartheta,\,\ldots,\,\vartheta^{n-1})$\, is square-free.

{\em Proof.}\, The adjusted canonical basis

$\displaystyle\omega_1 = 1,$\\
$\displaystyle\omega_2 = \frac{a_{21}\!+\!\vartheta}{d_2},$\\
$\displaystyle\omega_3 = \frac{a_{31}\!+\!a_{32}\vartheta\!+\!\vartheta^2}{d_3},$\\
$\vdots\,\qquad\vdots\,\qquad\vdots$\\
$\displaystyle\omega_n = \frac{a_{n1}\!+\!a_{n2}\vartheta\!+\ldots+\!a_{n,n-1}\vartheta^{n-2}\!+\!\vartheta^{n-1}}{d_n}$\\

\noindent of $\mathbb{Q}(\vartheta)$ is an integral basis, where $d_2,\,d_3,\,\ldots,\,d_n$ are integers.\, Its discriminant is the fundamental number $d$ of the field.\, By the lemma, we obtain

\begin{equation*}
d = \Delta(\omega_1,\,\omega_2,\,\ldots,\,\omega_n) = \left|
\begin{array}{cccc}
1 & 0 &\ldots & 0 \\
\frac{a_{21}}{d_2} & \frac{1}{d_2} & \ddots & 0\\
\vdots & \vdots & \ddots & 0 \\
\frac{a_{n1}}{d_n} & \frac{a_{n2}}{d_n}& \ldots & \frac{1}{d_n}
\end{array}\right|^2 \Delta(1,\,\vartheta,\,\ldots,\,\vartheta^{n-1})
= \frac{d(\vartheta)}{(d_2d_3\cdots d_n)^2}.
\end{equation*}
Thus\, $(d_2d_3\cdots d_n)^2d = d(\vartheta)$,\, and since $d(\vartheta)$ is assumed to be square-free, we have 
$(d_2d_3\cdots d_n)^2 = 1$,\, and accordingly\, $d(\vartheta)$ equals the \PMlinkname{discriminant of the field}{MinimalityOfIntegralBasis}.\, This implies (see minimality of integral basis) that the numbers $1,\,\vartheta,\,\ldots,\,\vartheta^{n-1}$ form an integral basis of the field $\mathbb{Q}(\vartheta)$.
%%%%%
%%%%%
\end{document}
