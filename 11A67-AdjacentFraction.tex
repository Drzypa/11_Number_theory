\documentclass[12pt]{article}
\usepackage{pmmeta}
\pmcanonicalname{AdjacentFraction}
\pmcreated{2013-03-22 12:48:23}
\pmmodified{2013-03-22 12:48:23}
\pmowner{XJamRastafire}{349}
\pmmodifier{XJamRastafire}{349}
\pmtitle{adjacent fraction}
\pmrecord{17}{33124}
\pmprivacy{1}
\pmauthor{XJamRastafire}{349}
\pmtype{Definition}
\pmcomment{trigger rebuild}
\pmclassification{msc}{11A67}
%\pmkeywords{difference}
%\pmkeywords{fraction}
%\pmkeywords{successive term}
\pmrelated{FareySequence}
\pmrelated{UnitFraction}
\pmrelated{ContinuedFraction}
\pmrelated{NumeratorAndDenominatorIncreasedBySameAmount}

\endmetadata

% this is the default PlanetMath preamble.  as your knowledge
% of TeX increases, you will probably want to edit this, but
% it should be fine as is for beginners.

% almost certainly you want these
\usepackage{amssymb}
\usepackage{amsmath}
\usepackage{amsfonts}

% used for TeXing text within eps files
%\usepackage{psfrag}
% need this for including graphics (\includegraphics)
%\usepackage{graphicx}
% for neatly defining theorems and propositions
%\usepackage{amsthm}
% making logically defined graphics
%%%\usepackage{xypic}

% there are many more packages, add them here as you need them

% define commands here
\begin{document}
Two fractions $\frac{a}{b}$ and $\frac{c}{d}$, $\frac{a}{b} > \frac{c}{d}$ of the positive integers $a, b, c, d$ are {\it adjacent} if their difference is some unit fraction $\frac{1}{n}$, $n>0$ that is, if we can write:
\begin{equation*}\frac{a}{b} - \frac{c}{d} = \frac{1}{n}.
\end{equation*}

For example the two proper fractions and unit fractions $\frac{1}{11}$ and $\frac{1}{12}$ are adjacent since: 
\begin{equation*}
\frac{1}{11} - \frac{1}{12} = \frac{1}{132} \; .
\end{equation*}

$\frac{1}{17}$ and $\frac{1}{19}$ are not since: 
$$\frac{1}{17} - \frac{1}{19} = \frac{2}{323} \; . $$

It is not necessary of course that fractions are both proper fractions:
$$\frac{20}{19} - \frac{19}{19} = \frac{1}{19} \; . $$
or unit fractions:
$$\frac{3}{4} - \frac{2}{3} = \frac{1}{12} \; . $$

All successive terms of some Farey sequence $F_{n}$ of a degree $n$ are always adjacent fractions. In the first Farey sequence $F_{1}$ of a degree 1 there are only two adjacent fractions, namely $\frac{1}{1}$ and $\frac{0}{1}$.

Adjacent unit fractions can be parts of many Egyptian fractions:

$$\frac{1}{70} + \frac{1}{71} = \frac{141}{4970} \; . $$
%%%%%
%%%%%
\end{document}
