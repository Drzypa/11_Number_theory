\documentclass[12pt]{article}
\usepackage{pmmeta}
\pmcanonicalname{PureCubicField}
\pmcreated{2013-03-22 16:02:19}
\pmmodified{2013-03-22 16:02:19}
\pmowner{Wkbj79}{1863}
\pmmodifier{Wkbj79}{1863}
\pmtitle{pure cubic field}
\pmrecord{14}{38088}
\pmprivacy{1}
\pmauthor{Wkbj79}{1863}
\pmtype{Definition}
\pmcomment{trigger rebuild}
\pmclassification{msc}{11R16}

\endmetadata

% this is the default PlanetMath preamble.  as your knowledge
% of TeX increases, you will probably want to edit this, but
% it should be fine as is for beginners.

% almost certainly you want these
\usepackage{amssymb}
\usepackage{amsmath}
\usepackage{amsfonts}

% used for TeXing text within eps files
%\usepackage{psfrag}
% need this for including graphics (\includegraphics)
%\usepackage{graphicx}
% for neatly defining theorems and propositions
%\usepackage{amsthm}
% making logically defined graphics
%%%\usepackage{xypic}

% there are many more packages, add them here as you need them

% define commands here

\begin{document}
A {\sl pure cubic field\/} is an extension of $\mathbb{Q}$ of the form $\mathbb{Q}(\sqrt[3]{n})$ for some $n \in \mathbb{Z}$ such that $\sqrt[3]{n} \notin \mathbb{Q}$.  If $n<0$, then $\sqrt[3]{n}=\sqrt[3]{-|n|}=-\sqrt[3]{|n|}$, causing $\mathbb{Q}(\sqrt[3]{n})=\mathbb{Q}(\sqrt[3]{|n|})$.  Thus, without loss of generality, it may be assumed that $n>1$.

Note that no pure cubic field is \PMlinkname{Galois}{GaloisExtension} over $\mathbb{Q}$.  For if $n \in \mathbb{Z}$ is cubefree with $|n| \neq 1$, then $x^3-n$ is its minimal polynomial over $\mathbb{Q}$.  This polynomial factors as $(x-\sqrt[3]{n})(x^2+x\sqrt[3]{n}+\sqrt[3]{n^2})$ over $K=\mathbb{Q}(\sqrt[3]{|n|})$.  The \PMlinkname{discriminant}{PolynomialDiscriminant} of $x^2+x\sqrt[3]{n}+\sqrt[3]{n^2}$ is $\left( \sqrt[3]{n} \right)^2-4(1)\left( \sqrt[3]{n^2} \right)=\sqrt[3]{n^2}-4\sqrt[3]{n^2}=-3\sqrt[3]{n^2}$.  Since the \PMlinkescapetext{discriminant} of $x^2+x\sqrt[3]{n}+\sqrt[3]{n^2}$ is negative, it does not factor in $\mathbb{R}$.  Note that $K \subseteq \mathbb{R}$.  Thus, $x^3-n$ has a \PMlinkname{root}{Root} in $K$ but does not split completely in $K$.

Note also that pure cubic fields are real cubic fields with exactly one real embedding.  Thus, a possible method of determining all of the units of pure cubic fields is outlined in the entry regarding units of real cubic fields with exactly one real embedding.
%%%%%
%%%%%
\end{document}
