\documentclass[12pt]{article}
\usepackage{pmmeta}
\pmcanonicalname{AnyNonzeroIntegerIsQuadraticResidue}
\pmcreated{2013-03-22 18:01:03}
\pmmodified{2013-03-22 18:01:03}
\pmowner{pahio}{2872}
\pmmodifier{pahio}{2872}
\pmtitle{any nonzero integer is quadratic residue}
\pmrecord{6}{40533}
\pmprivacy{1}
\pmauthor{pahio}{2872}
\pmtype{Theorem}
\pmcomment{trigger rebuild}
\pmclassification{msc}{11A15}
\pmrelated{FundamentalTheoremOfArithmetic}

% this is the default PlanetMath preamble.  as your knowledge
% of TeX increases, you will probably want to edit this, but
% it should be fine as is for beginners.

% almost certainly you want these
\usepackage{amssymb}
\usepackage{amsmath}
\usepackage{amsfonts}

% used for TeXing text within eps files
%\usepackage{psfrag}
% need this for including graphics (\includegraphics)
%\usepackage{graphicx}
% for neatly defining theorems and propositions
 \usepackage{amsthm}
% making logically defined graphics
%%%\usepackage{xypic}

% there are many more packages, add them here as you need them

% define commands here

\theoremstyle{definition}
\newtheorem*{thmplain}{Theorem}

\begin{document}
\textbf{Theorem.}\, For every nonzero integer $a$ there exists an odd prime number $p$ such that $a$ is a quadratic residue modulo $p$.


{\em Proof.}\, $1^\circ.$\; $a = 2$.\; We see that\, $3^2 \equiv 2 \pmod7$\, and $7 \nmid 2$,\, whence 2 is a quadratic residue modulo $7$.\\
$2^\circ.$\; $2 \mid a$\, but\, $a \neq 2$.\; The number\, $1^2-a = 1-a$\, (which is odd and $\neq \pm1$) has an odd prime factor $p$ which does not divide $a$.\, Thus $a$ is a quadratic residue modulo $p$.\\
$3^\circ.$\; $a = 3$.\; We state that\, $4^2-3 = 13 \equiv 0 \pmod{13}$\, and\, $13 \nmid 3$.\, Therefore 3 is a quadratic residue modulo 13.\\
$4^\circ.$\; $a = 5$.\; We see that\, $4^2-5 = 11 \equiv 0 \pmod{11}$\, and\, $11 \nmid 5$, i.e. 5 is a quadratic residue modulo 11.\\
$5^\circ.$\; $2 \nmid a$\, but\, $a \neq 3$,\, $a \neq 5$.\; Now the number\, $2^2-a = 4-a$\, (which is odd and $\neq \pm1$) has an odd prime factor $p$.\, Moreover, $p \nmid a$\, since\, $p \nmid 4$.\, Accordingly, $a$ is a quadratic residue modulo $p$.

%%%%%
%%%%%
\end{document}
