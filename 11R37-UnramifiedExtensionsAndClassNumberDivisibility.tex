\documentclass[12pt]{article}
\usepackage{pmmeta}
\pmcanonicalname{UnramifiedExtensionsAndClassNumberDivisibility}
\pmcreated{2013-03-22 15:02:59}
\pmmodified{2013-03-22 15:02:59}
\pmowner{alozano}{2414}
\pmmodifier{alozano}{2414}
\pmtitle{unramified extensions and class number divisibility}
\pmrecord{5}{36765}
\pmprivacy{1}
\pmauthor{alozano}{2414}
\pmtype{Corollary}
\pmcomment{trigger rebuild}
\pmclassification{msc}{11R37}
\pmclassification{msc}{11R32}
\pmclassification{msc}{11R29}
%\pmkeywords{class number divisibility}
\pmrelated{IdealClass}
\pmrelated{PExtension}
\pmrelated{Ramify}
\pmrelated{ClassNumbersAndDiscriminantsTopicsOnClassGroups}

\endmetadata

% this is the default PlanetMath preamble.  as your knowledge
% of TeX increases, you will probably want to edit this, but
% it should be fine as is for beginners.

% almost certainly you want these
\usepackage{amssymb}
\usepackage{amsmath}
\usepackage{amsthm}
\usepackage{amsfonts}

% used for TeXing text within eps files
%\usepackage{psfrag}
% need this for including graphics (\includegraphics)
%\usepackage{graphicx}
% for neatly defining theorems and propositions
%\usepackage{amsthm}
% making logically defined graphics
%%%\usepackage{xypic}

% there are many more packages, add them here as you need them

% define commands here

\newtheorem{thm}{Theorem}
\newtheorem{defn}{Definition}
\newtheorem{prop}{Proposition}
\newtheorem{lemma}{Lemma}
\newtheorem{cor}{Corollary}

% Some sets
\newcommand{\Nats}{\mathbb{N}}
\newcommand{\Ints}{\mathbb{Z}}
\newcommand{\Reals}{\mathbb{R}}
\newcommand{\Complex}{\mathbb{C}}
\newcommand{\Rats}{\mathbb{Q}}
\begin{document}
The following is a corollary of the existence of the Hilbert class field.

\begin{cor}
Let $K$ be a number field, $h_K$ is its class number and let $p$ be a prime. Then $K$ has an everywhere unramified Galois extension of degree $p$ if and only if $h_K$ is divisible by $p$.
\end{cor}

\begin{proof}
Let $K$ be a number field and let $H$ be the Hilbert class field of $K$. Then:
$$|\operatorname{Gal}(H/K)|=[H:K]=h_K.$$
Let $p$ be a prime number. Suppose that there exists a Galois extension $F/K$, such that $[F:K]=p$ and $F/K$ is everywhere unramified. Notice that any Galois extension of prime degree is abelian (because any group of prime degree $p$ is abelian, isomorphic to $\Ints/p\Ints$). Since $H$ is the maximal abelian unramified extension of $K$ the following inclusions occur:
$$K \subsetneq F\subseteq H$$
Moreover, 
$$h_K=[H:K]=[H:F]\cdot[F:K]=[H:F]\cdot p.$$
Therefore $p$ divides $h_K$.\\

Next we prove the remaining direction. Suppose that $p$ divides $h_K=|\operatorname{Gal}(H/K)|$. Since $G=\operatorname{Gal}(H/K)$ is an abelian group (isomorphic to the class group of $K$) there exists a normal subgroup $J$ of $G$ such that $|G/J|=p$. Let $F=H^J$ be the fixed field by the subgroup $J$, which is, by the main theorem of Galois theory, a Galois extension of $K$. This field satisfies $[F:K]=p$ and, since $F$ is included in $H$, the extension $F/K$ is abelian and everywhere unramified, as claimed.    
\end{proof}
%%%%%
%%%%%
\end{document}
