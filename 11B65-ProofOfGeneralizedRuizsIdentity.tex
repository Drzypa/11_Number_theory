\documentclass[12pt]{article}
\usepackage{pmmeta}
\pmcanonicalname{ProofOfGeneralizedRuizsIdentity}
\pmcreated{2013-03-22 14:32:02}
\pmmodified{2013-03-22 14:32:02}
\pmowner{GeraW}{6138}
\pmmodifier{GeraW}{6138}
\pmtitle{proof of generalized Ruiz's identity}
\pmrecord{8}{36078}
\pmprivacy{1}
\pmauthor{GeraW}{6138}
\pmtype{Proof}
\pmcomment{trigger rebuild}
\pmclassification{msc}{11B65}
\pmclassification{msc}{05A10}

% this is the default PlanetMath preamble.  as your knowledge
% of TeX increases, you will probably want to edit this, but
% it should be fine as is for beginners.

% almost certainly you want these
\usepackage{amssymb}
\usepackage{amsmath}
\usepackage{amsfonts}

% used for TeXing text within eps files
%\usepackage{psfrag}
% need this for including graphics (\includegraphics)
%\usepackage{graphicx}
% for neatly defining theorems and propositions
\usepackage{amsthm}
% making logically defined graphics
%%%\usepackage{xypic}

% there are many more packages, add them here as you need them

% define commands here
\newtheorem*{thm}{Theorem}
\begin{document}
\begin{thm}
Consider the polynomials $c_{i,j}(x)=(x+i)^{j}-(x+i-1)^{j}$. Then, for every positive natural number $n$,
\[ \det (c_{i,j})_{i,j=1}^n = \prod_{k=1}^{n} k! \]
\end{thm}
\begin{proof}
Consider the matrices $M,C$ defined by $M_{i,j}=(-1)^j{i-1 \choose j-1}$ and $C_{i,j}=(x+i)^j-(x+i-1)^j$.
\[\begin{aligned}
(MC)_{i,j}&=\sum_{k=1}^n (-1)^k {i-1 \choose k-1}\bigl((x+k)^j-(x+k-1)^j\bigr) \\
          &=\sum_{k=0}^n (-1)^k \left({i-1 \choose k-1}+{i-1 \choose k}\right)(x+k)^j \\
          &=\sum_{k=0}^n (-1)^k {i \choose k}(x+k)^j \\
          &=(-1)^j\sum_{k=0}^i (-1)^{k} {i \choose k}(-x-k)^j
\end{aligned}\]
Therefore, by Ruiz's identity, $(MC)_{i,i}=(-1)^i i!$ for every $i \in \{1,...,d\}$ and $(MC)_{i,j}=0$ for every $i,j \in \{1,...,n\}$ such that $i>j$. This
means that $MC$ is an upper triangular matrix whose main diagonal is $-1!,2!,-3!,...,(-1)^n n!$. Since the determinant of such a matrix is
the product of the elements in the main diagonal, we get that $\det MC=(-1)^{n} \prod_{k=1}^{n}k!$. It is easy to see that $M$ itself is lower
triangular with determinant $(-1)^n$. Therefore $\det C=\prod_{k=1}^{n}k!$.
\end{proof}
%%%%%
%%%%%
\end{document}
