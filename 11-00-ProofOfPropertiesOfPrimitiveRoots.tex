\documentclass[12pt]{article}
\usepackage{pmmeta}
\pmcanonicalname{ProofOfPropertiesOfPrimitiveRoots}
\pmcreated{2013-03-22 18:43:48}
\pmmodified{2013-03-22 18:43:48}
\pmowner{rm50}{10146}
\pmmodifier{rm50}{10146}
\pmtitle{proof of properties of primitive roots}
\pmrecord{4}{41498}
\pmprivacy{1}
\pmauthor{rm50}{10146}
\pmtype{Proof}
\pmcomment{trigger rebuild}
\pmclassification{msc}{11-00}

% this is the default PlanetMath preamble.  as your knowledge
% of TeX increases, you will probably want to edit this, but
% it should be fine as is for beginners.

% almost certainly you want these
\usepackage{amssymb}
\usepackage{amsmath}
\usepackage{amsfonts}

% used for TeXing text within eps files
%\usepackage{psfrag}
% need this for including graphics (\includegraphics)
%\usepackage{graphicx}
% for neatly defining theorems and propositions
\usepackage{amsthm}
% making logically defined graphics
%%%\usepackage{xypic}

% there are many more packages, add them here as you need them

% define commands here
\newcommand{\Ints}{\mathbb{Z}}
\newcommand{\UI}[1]{(\Ints/{#1}\Ints)^{\times}}
%
%% \theoremstyle{plain} %% This is the default
\newtheorem{thm}{Theorem}
\newtheorem{cor}[thm]{Corollary}
\newtheorem{lem}[thm]{Lemma}
\newtheorem{prop}[thm]{Proposition}
\begin{document}
The material in the main article is conveniently recast in terms of the groups $\UI{m}$, the multiplicative group of units in $\Ints/m\Ints$. Note that the order of this group is exactly $\phi(m)$ where $\phi$ is the Euler phi function. Then saying that an integer $g$ is a primitive root of $m$ is equivalent to saying that the residue class of $g \pmod m$ generates $\UI{m}$.

\begin{proof}(of Theorem):
\newline
The proof of the theorem is an immediate consequence of the structure of $\UI{m}$ as an abelian group; $\UI{m}$ is cyclic precisely for $m=2, 4, p^k$, or $2p^k$.
\end{proof}

\begin{proof}(of Proposition):

\begin{enumerate}
\item Restated, this says that if the residue class of $g\pmod m$ generates $\UI{m}$, then the set $\{1,g,g^2,\ldots,g^{\phi(m)}\}$ is a complete set of representatives for $\UI{m}$; this is obvious.

\item Restated, this says that $g\pmod m$ generates $\UI{m}$ if and only if $g$ has exact order $m$, which is also obvious.

\item If $g\pmod m$ generates $\UI{m}$, then $g$ has exact order $\phi(m)$ and thus $g^s=g^t\pmod m$ if and only if $g^{s-t}=1$ if and only if $\phi(m)\mid s-t$.

\item Suppose $g$ generates $\UI{m}$. Then $(g^k)^r=1$ if and only if $g^{kr}=1$ if and only if $\phi(m)\mid kr$. Clearly we can choose $r<\phi(m)$ if and only if $\gcd(k,\phi(m))>1$.

\item This follows immediately from (4).\qedhere
\end{enumerate}
\end{proof}

%%%%%
%%%%%
\end{document}
