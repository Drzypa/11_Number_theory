\documentclass[12pt]{article}
\usepackage{pmmeta}
\pmcanonicalname{TwogeneratorProperty}
\pmcreated{2015-05-05 15:25:37}
\pmmodified{2015-05-05 15:25:37}
\pmowner{pahio}{2872}
\pmmodifier{pahio}{2872}
\pmtitle{two-generator property}
\pmrecord{38}{35645}
\pmprivacy{1}
\pmauthor{pahio}{2872}
\pmtype{Theorem}
\pmcomment{trigger rebuild}
\pmclassification{msc}{11R04}
\pmsynonym{Ideal of Dedekind domain}{TwogeneratorProperty}
\pmrelated{SumOfIdeals}
\pmrelated{FamousAndInfamousOpenQuestionsInMathematics}
\pmrelated{AnyDivisorIsGcdOfTwoPrincipalDivisors}

% this is the default PlanetMath preamble.  as your knowledge
% of TeX increases, you will probably want to edit this, but
% it should be fine as is for beginners.

% almost certainly you want these
\usepackage{amssymb}
\usepackage{amsmath}
\usepackage{amsfonts}

% used for TeXing text within eps files
%\usepackage{psfrag}
% need this for including graphics (\includegraphics)
%\usepackage{graphicx}
% for neatly defining theorems and propositions
 \usepackage{amsthm}
% making logically defined graphics
%%%\usepackage{xypic}

% there are many more packages, add them here as you need them

% define commands here
\theoremstyle{definition}
\newtheorem*{thmplain}{Theorem}
\begin{document}
\textbf{Theorem.}\, Every ideal of a Dedekind domain can be generated by two of its elements.

 
{\em Proof}.\, Let $\mathfrak{a}$ be an arbitrary ideal of a Dedekind domain $R$.\, Let $\mathfrak{b}$ be such an ideal of $R$ that $\mathfrak{ab}$ is a principal ideal $(\beta)$.\, The lemma to which this entry is attached gives also an element $\gamma$ and an ideal $\mathfrak{c}$ of $R$ such that\, $\mathfrak{ac} = (\gamma)$\, and\, $\mathfrak{b+c} = R$.\, Then we have
      $$\mathfrak{a} = \gcd(\mathfrak{ab},\,\mathfrak{ac}) = 
        \gcd((\beta),\,(\gamma)) = (\beta,\,\gamma)$$
because\, $\gcd(\mathfrak{b},\,\mathfrak{c}) = \mathfrak{b+c} = R = (1)$. $\Box$

The Dedekind domains are trivially Pr\"ufer domains, but the {\em two-generator property} can not be generalized to the invertible ideals of all Pr\"ufer domains (and Pr\"ufer rings):\, Sch\"ulting has constructed an invertible ideal of a Pr\"ufer domain that can not be generated by less than three generators.\, The example of Sch\"ulting is the fractional ideal\, $(1,\,X,\,Y)$\, of the Pr\"ufer domain\, $\bigcap_j B_j$\, where the $B_j$'s run all valuation rings of the rational function field\, $\mathbb{R}(X,\,Y)$\, which have the residue fields formally real.

\begin{thebibliography}{9}
\bibitem{EM}{\sc Eben Matlis:} ``{The two-generator problem for ideals}''. \,-- {\em The Michigan Mathematical Journal} \textbf{17}\, $\mbox{N}\sp\circ$ 3 (1970).
\bibitem{HWS}{\sc Heinz-Werner Sch\"ulting:} ``{\"Uber die Erzeugendenanzahl invertierbarer Ideale in Pr\"uferringen}''. \,-- {\em Communications in Algebra} \textbf{7}\, $\mbox{N}\sp\circ$ 13 (1979). [Zentralblatt 432.13010]
\end{thebibliography}
%%%%%
%%%%%
\end{document}
