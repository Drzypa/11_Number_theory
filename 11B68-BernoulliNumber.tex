\documentclass[12pt]{article}
\usepackage{pmmeta}
\pmcanonicalname{BernoulliNumber}
\pmcreated{2013-03-22 11:45:58}
\pmmodified{2013-03-22 11:45:58}
\pmowner{alozano}{2414}
\pmmodifier{alozano}{2414}
\pmtitle{Bernoulli number}
\pmrecord{14}{30219}
\pmprivacy{1}
\pmauthor{alozano}{2414}
\pmtype{Definition}
\pmcomment{trigger rebuild}
\pmclassification{msc}{11B68}
\pmclassification{msc}{49J24}
\pmclassification{msc}{49J22}
\pmclassification{msc}{49J20}
\pmclassification{msc}{49J15}
%\pmkeywords{number theory}
\pmrelated{GeneralizedBernoulliNumber}
\pmrelated{BernoulliPolynomials}
\pmrelated{SumOfKthPowersOfTheFirstNPositiveIntegers}
\pmrelated{EulerMaclaurinSummationFormula}
\pmrelated{ValuesOfTheRiemannZetaFunctionInTermsOfBernoulliNumbers}
\pmrelated{TaylorSeriesViaDivision}
\pmrelated{BernoulliPolynomialsAndNumbers}
\pmrelated{EulerNumbers2}

\usepackage{amssymb}
\usepackage{amsmath}
\usepackage{amsfonts}
%\usepackage{graphicx}
%%%%%\usepackage{xypic}
\begin{document}
Let $B_r$ be the $r$th Bernoulli polynomial. Then the $r$th {\bf Bernoulli number} is
\[ 
B_r := B_r(0). 
\]

This means, in particular, that the Bernoulli numbers are given by an exponential generating function in the following way:
\[
\sum_{r=0}^{\infty} B_r \frac{y^r}{r!} = \frac{y}{e^y-1} 
\]
and, in fact, the Bernoulli numbers are usually defined as the coefficients that appear in such expansion.

Observe that this generating function can be rewritten:
\[
\frac{y}{e^y-1} = \frac{y}{2}\frac{e^y+1}{e^y-1} - \frac{y}{2} = (y/2)(\operatorname{tanh}(y/2) -1).
\]
 Since $\operatorname{tanh}$ is an odd function, one can see that $B_{2r+1}=0$ for $r \geq 1$. Numerically, $B_0 = 1, B_1 = -\frac{1}{2}, B_2 = \frac{1}{6}, B_4 = -\frac{1}{30}, \cdots$

These combinatorial numbers occur in a number of contexts; the most elementary is perhaps that they occur in the formulas for the \PMlinkname{sum of the $r$th powers of the first $n$ positive integers}{SumOfKthPowersOfTheFirstNPositiveIntegers}.  They also occur in the Maclaurin expansion for the tangent function and in the Euler-Maclaurin summation formula.
%%%%%
%%%%%
%%%%%
%%%%%
\end{document}
