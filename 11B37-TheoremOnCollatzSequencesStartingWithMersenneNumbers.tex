\documentclass[12pt]{article}
\usepackage{pmmeta}
\pmcanonicalname{TheoremOnCollatzSequencesStartingWithMersenneNumbers}
\pmcreated{2013-03-22 17:34:32}
\pmmodified{2013-03-22 17:34:32}
\pmowner{PrimeFan}{13766}
\pmmodifier{PrimeFan}{13766}
\pmtitle{theorem on Collatz sequences starting with Mersenne numbers}
\pmrecord{12}{39987}
\pmprivacy{1}
\pmauthor{PrimeFan}{13766}
\pmtype{Theorem}
\pmcomment{trigger rebuild}
\pmclassification{msc}{11B37}

% this is the default PlanetMath preamble.  as your knowledge
% of TeX increases, you will probably want to edit this, but
% it should be fine as is for beginners.

% almost certainly you want these
\usepackage{amssymb}
\usepackage{amsmath}
\usepackage{amsfonts}

% used for TeXing text within eps files
%\usepackage{psfrag}
% need this for including graphics (\includegraphics)
%\usepackage{graphicx}
% for neatly defining theorems and propositions
\usepackage{amsthm}
% making logically defined graphics
%%%\usepackage{xypic}

% there are many more packages, add them here as you need them

% define commands here

\begin{document}
\PMlinkescapeword{even}
\PMlinkescapeword{odd number}
\PMlinkescapeword{even number}
\PMlinkescapeword{alternating}
\PMlinkescapeword{right}

{\bf Theorem.} Given a Mersenne number $m = 2^n - 1$ (with $n$ a nonnegative integer), the Collatz sequence starting with $m$ reaches $3^n - 1$ in precisely $2n$ steps. Also, the parity of such a sequence consistenly alternates parity until $3^n - 1$ is reached. For example, given $2^2 - 1 = 3$ gives the Collatz sequence 3, 10, 5, 16, 8, 4, 2, 1, in which $3^2 - 1 = 8$ is reached at the fourth step. Also, the least significant bits of this particular sequence are 1, 0, 1, 0, 0, 0, 0, 1.

As you might already know, a Collatz sequence results from the repeated application of the Collatz function $C(n) = 3n + 1$ for odd $n$ and $C(n) = \frac{n}{2}$ for even $n$. If I may, I'd like to introduce the iterated Collatz function notation as a recurrence relation thus: $C_0(n) = n$ and $C_i(n) = C(C_{i - 1}(n))$ for all $i > 0$. In our example, $C_0(3) = 3$, $C_1(3) = 10$, $C_2(3) = 5$, etc. (We could choose to have $C_1(n) = n$ instead with only slight changes to the theorem and its proof).

\begin{proof}
Obviously $m = C_0(2^n - 1) = 2^n - 1$ is an odd number. Therefore $C_1(m) = 2^n 3 - 2$, an even number, and then $C_2(m) = 2^{n - 1} 3 - 1$, $C_3(m) = 2^{n - 1} 9 - 2$, $C_4(m) = 2^{n - 2} 9 - 1$, $C_5(m) = 2^{n - 2} 27 - 2$, etc. We can now generalize that if $i$ is odd, then $C_i(m) = 2^{n - \frac{i - 1}{2}} 3^{\frac{i + 1}{2}} - 2$ and $C_i(m) = 2^{n - \frac{i}{2}} 3^{\frac{i}{2}}$ if $i$ is even. By plugging in $i = 2n$, we get $C_i(m) = 2^{n - \frac{2n}{2}} 3^{\frac{2n}{2}} = 2^{n - n} 3^n - 1 = 2^0 3^n - 1 = 3^n - 1$, proving the theorem.
\end{proof}

Of course the generalized formulas do not work when $i > 2n$, nor does any of this give any insight into when a Collatz sequence starting with a Mersenne number reaches a power of 2. Likewise, the pattern of consistently alternating parity usually breaks down on or right after the $2n$th step.
%%%%%
%%%%%
\end{document}
