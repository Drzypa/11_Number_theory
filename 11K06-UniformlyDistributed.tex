\documentclass[12pt]{article}
\usepackage{pmmeta}
\pmcanonicalname{UniformlyDistributed}
\pmcreated{2013-03-22 14:17:29}
\pmmodified{2013-03-22 14:17:29}
\pmowner{bbukh}{348}
\pmmodifier{bbukh}{348}
\pmtitle{uniformly distributed}
\pmrecord{6}{35746}
\pmprivacy{1}
\pmauthor{bbukh}{348}
\pmtype{Definition}
\pmcomment{trigger rebuild}
\pmclassification{msc}{11K06}
\pmclassification{msc}{11K38}
\pmsynonym{equidistributed}{UniformlyDistributed}
\pmrelated{WeylsCriterion}

\usepackage{amssymb}
\usepackage{amsmath}
\usepackage{amsfonts}

% used for TeXing text within eps files
%\usepackage{psfrag}
% need this for including graphics (\includegraphics)
%\usepackage{graphicx}
% for neatly defining theorems and propositions
%\usepackage{amsthm}
% making logically defined graphics
%%%\usepackage{xypic}

\makeatletter
\@ifundefined{bibname}{}{\renewcommand{\bibname}{References}}
\makeatother
\begin{document}
Let $\{u_n\}$ be a sequence of real numbers. For
$0\leq\alpha<\beta\leq 1$ put
\begin{equation*}
Z(N,\alpha,\beta)=\operatorname{card}\{n\in[1..N] : \alpha \leq
(u_n \bmod 1)< \beta \}.
\end{equation*}
The sequence $\{u_n\}$ is \emph{uniformly distributed modulo $1$}
if
\begin{equation*}\label{eq:modcond}
\lim_{N\to\infty} \frac{1}{N} Z(N,\alpha,\beta)=\beta-\alpha
\end{equation*}
for all $0\leq\alpha<\beta\leq 1$. In other words a sequence is
uniformly distributed modulo $1$ if each subinterval of $[0,1]$
gets its ``fair share'' of fractional parts of $\{u_n\}$.

More generally, a sequence $\{u_n\}$ of points in a finite measure
space $(X,\mathcal{A},\mu)$ is uniformly distributed with respect
to a family of sets $\mathcal{F}\subset\mathcal{A}$ if
\begin{equation*}
\lim_{N\to\infty} \frac{\operatorname{card}\{n\in[1..N] :u_n\in
S\}}{N}=\frac{\mu(S)}{\mu(X)}\qquad\text{for every
}S\in\mathcal{F}.
\end{equation*}

\begin{thebibliography}{1}

\bibitem{cite:chen_irreg_dist}
William Chen.
\newblock Lectures on irregularities of point distribution.
\newblock Available at \PMlinkexternal{http://www.maths.mq.edu.au/~wchen/ln.html}{http://www.maths.mq.edu.au/~wchen/ln.html}, 2000.

\bibitem{cite:montgomery_tenlect}
Hugh~L. Montgomery.
\newblock {\em Ten Lectures on the Interface Between Analytic Number Theory and
  Harmonic Analysis}, volume~84 of {\em Regional Conference Series in
  Mathematics}.
\newblock AMS, 1994.
\newblock \PMlinkexternal{Zbl 0814.11001}{http://www.emis.de/cgi-bin/zmen/ZMATH/en/quick.html?type=html&an=0814.11001}.

\end{thebibliography}
%%%%%
%%%%%
\end{document}
