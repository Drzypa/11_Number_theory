\documentclass[12pt]{article}
\usepackage{pmmeta}
\pmcanonicalname{ElementaryProofOfOrders}
\pmcreated{2013-03-22 15:56:52}
\pmmodified{2013-03-22 15:56:52}
\pmowner{Algeboy}{12884}
\pmmodifier{Algeboy}{12884}
\pmtitle{elementary proof of orders}
\pmrecord{8}{37959}
\pmprivacy{1}
\pmauthor{Algeboy}{12884}
\pmtype{Proof}
\pmcomment{trigger rebuild}
\pmclassification{msc}{11E57}
\pmclassification{msc}{05E15}

\usepackage{latexsym}
\usepackage{amssymb}
\usepackage{amsmath}
\usepackage{amsfonts}
\usepackage{amsthm}

%%\usepackage{xypic}

%-----------------------------------------------------

%       Standard theoremlike environments.

%       Stolen directly from AMSLaTeX sample

%-----------------------------------------------------

%% \theoremstyle{plain} %% This is the default

\newtheorem{thm}{Theorem}

\newtheorem{coro}[thm]{Corollary}

\newtheorem{lem}[thm]{Lemma}

\newtheorem{lemma}[thm]{Lemma}

\newtheorem{prop}[thm]{Proposition}

\newtheorem{conjecture}[thm]{Conjecture}

\newtheorem{conj}[thm]{Conjecture}

\newtheorem{defn}[thm]{Definition}

\newtheorem{remark}[thm]{Remark}

\newtheorem{ex}[thm]{Example}



%\countstyle[equation]{thm}



%--------------------------------------------------

%       Item references.

%--------------------------------------------------


\newcommand{\exref}[1]{Example-\ref{#1}}

\newcommand{\thmref}[1]{Theorem-\ref{#1}}

\newcommand{\defref}[1]{Definition-\ref{#1}}

\newcommand{\eqnref}[1]{(\ref{#1})}

\newcommand{\secref}[1]{Section-\ref{#1}}

\newcommand{\lemref}[1]{Lemma-\ref{#1}}

\newcommand{\propref}[1]{Prop\-o\-si\-tion-\ref{#1}}

\newcommand{\corref}[1]{Cor\-ol\-lary-\ref{#1}}

\newcommand{\figref}[1]{Fig\-ure-\ref{#1}}

\newcommand{\conjref}[1]{Conjecture-\ref{#1}}


% Normal subgroup or equal.

\providecommand{\normaleq}{\unlhd}

% Normal subgroup.

\providecommand{\normal}{\lhd}

\providecommand{\rnormal}{\rhd}
% Divides, does not divide.

\providecommand{\divides}{\mid}

\providecommand{\ndivides}{\nmid}


\providecommand{\union}{\cup}

\providecommand{\bigunion}{\bigcup}

\providecommand{\intersect}{\cap}

\providecommand{\bigintersect}{\bigcap}










\begin{document}
When possible, our proofs avoid matrices so that the proofs retain some value to infinite dimensional settings.  When we use $k$ we mean any field, and $GF(q)$ indicates the special case of a finite field of order $q$.  $V$ is always our vector space.

\begin{remark}
There are many alternative methods for computing orders of classical groups,
for instance \PMlinkname{observing special subgroups}{TheoryFromOrdersOfClassicalGroups2} or from Lie theory and
the study of Chevalley groups.  The method explored here is intended to be
elementary linear algebra.
\end{remark}

The basic starting point in computing orders of classical groups is an application of elementary linear algebra rephrased in group theory terms.

\begin{prop}
$GL(V)$ acts regularly on the set of ordered bases of vector space $V$
over a field $k$.
\end{prop}
\begin{proof}
Given any two bases $B=\{b_i:i\in I\}$ and $C=\{c_i:i\in I\}$ of 
a vector space $V$, define the map $f:V\rightarrow V$ by
\begin{equation}\label{eq:1}
\left(\sum_{i\in I} l_i b_i\right)f=\sum_{i\in I}l_i c_i.
\end{equation}
[Note the above sum has only finitely many non-zero $l_i\in k$.]
It follows $f$ is an invertible linear transformation so $f\in GL(V)$.
Furthermore, any linear transformation $g\in GL(V)$ with $(b_i)g=c_i$
must satisfy (\ref{eq:1}) to be linear so indeed $g=f$.  Therefore
$GL(V)$ acts regularly on ordered bases of $V$.
\end{proof}

In the world of group theory, a regular action is a typical substitute for knowing the order of a group.  In particular, any two groups, even infinite, have the same order if they have a regular action on the same set.  However, we are presently after specific order of finite groups so we return to the case of $V$ a finite dimension vector space over a finite field $k=GF(q)$.  We do however attempt to establish the orders through bijections with other sets and groups so that the results apply in more general contexts as well.


\begin{thm}\label{thm:lin}
\[
\begin{array}{cc}
|SL(d,q)| = q^{\binom{d}{2}} \prod_{i=2}^d (q^i-1), 
  & |PSL(d,q)|=\frac{|SL(d,q)|}{(d,q-1)},\\
|GL(d,q)|  =  q^{\binom{d}{2}} \prod_{i=1}^d (q^i-1), 
  & |PGL(d,q)|=|SL(d,q)|,\\
|\Gamma L(d,q)|  =  (q-1) q^{\binom{d}{2}} \prod_{i=1}^d (q^i-1),
  & |P\Gamma L(d,q)|=\frac{|\Gamma L(d,q)|}{q-1}.
\end{array}
\]
\end{thm}
\begin{proof}
When $V=GF(q)^d$, the number of ordered bases can be counted.  A basis
is a set $B=\{b_1,\dots,b_d\}$ of linearly independent vectors.  So 
$b_1$ may be chosen freely from $V-\{0\}$, providing $q^d-1$ possible choices.
Next $b_2$ must be chosen independent form $b_1$ so $b_2$ can be freely 
chosen from $V-\langle b_1\rangle$ leaving $q^d-q$ choices.  In a similar fashion $b_3$ has $q^d-q^2$ possiblities and so continuing by induction we
find the total number of ordered bases to be:
\[(q^d-1)(q^d-q)\cdots (q^d-q^{d-1}).\]
Now we treat $q$ as the variable of a polynomial and factor this number into:
\[q^{\binom{d}{2}} \prod_{i=1}^d (q^i-1).\]
As $GL(V)$ acts regularly on ordered bases of $V$, this is the order of $GL(V)$.

For the order of $SL(V)$ recall the $SL(V)$ is the kernel of the determinant
homomorphism $\det:GL(V)\rightarrow k^\times$.  Furthermore, $\det$ is
surjective as a diagonal matrix can be used to exhibit any determinant we seek.  We conclude 
\[[GL(d,q):SL(d,q)]=|GF(q)^\times|=q-1\]
so that $|SL(d,q)|=|GL(d,q)|/(q-1)$.

In a similar process, $PGL(V)=GL(V)/Z(GL(V))$ so if we derive the order of the
center of $GL(V)$ we derive the order of $PGL(V)$.  The central transforms
are scalar (they must preserve every eigenspace of every linear transform) so 
$Z(GL(V))$ is isomorphic to $k^\times$.  Thus the order of $PGL(V)$ is the same as the order of $SL(V)$.

For $\Gamma L(V)$ and $P\Gamma L(V)$ simply notice $\Gamma L(V)=GL(V)\rtimes k^\times$ so when $k=GF(q)$ we get an additional $q-1$ term.

Finally, we consider $PSL(V)=SL(V)/(SL(V)\intersect Z(GL(V)))$.  The order of $SL(V)\intersect Z(GL(V))$ must be computed.  So we require scalar transforms with determinant 1.  As the such, if $r$ is the eigen value of the scalar transform we need $r^d=1$ in $GF(q)$.  From finite field theory we know $GF(q)^\times \cong \mathbb{Z}_{q-1}$.  As this group is cyclic we know that every element $r\in GF(q)^\times$ satisfying $r^d=1$ also satisfies $r^{q-1}=1$ and $r$ lies in the unique subgroup of order $(d,q-1)$ of $GF(q)^\times$.
Thus $|SL(V)\intersect Z(GL(V))|=(d,q-1)$.
\end{proof}

%%%%%
%%%%%
\end{document}
