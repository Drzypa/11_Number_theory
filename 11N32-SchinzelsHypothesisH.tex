\documentclass[12pt]{article}
\usepackage{pmmeta}
\pmcanonicalname{SchinzelsHypothesisH}
\pmcreated{2013-03-22 15:11:43}
\pmmodified{2013-03-22 15:11:43}
\pmowner{jtolliver}{9126}
\pmmodifier{jtolliver}{9126}
\pmtitle{Schinzel's Hypothesis H}
\pmrecord{5}{36953}
\pmprivacy{1}
\pmauthor{jtolliver}{9126}
\pmtype{Conjecture}
\pmcomment{trigger rebuild}
\pmclassification{msc}{11N32}

% this is the default PlanetMath preamble.  as your knowledge
% of TeX increases, you will probably want to edit this, but
% it should be fine as is for beginners.

% almost certainly you want these
\usepackage{amssymb}
\usepackage{amsmath}
\usepackage{amsfonts}

% used for TeXing text within eps files
%\usepackage{psfrag}
% need this for including graphics (\includegraphics)
%\usepackage{graphicx}
% for neatly defining theorems and propositions
%\usepackage{amsthm}
% making logically defined graphics
%%%\usepackage{xypic}

% there are many more packages, add them here as you need them

% define commands here
\begin{document}
Let a set of irreducible polynomials $P_1,P_2,P_3,...,P_k$ with integer coefficients have the property that for any prime $p$ there exists some $n$ such that $P_1(n)P_2(n)...P_k(n) \not\equiv 0 (mod\; p)$.  Schinzel's Hypothesis H \PMlinkescapetext{states} that there are infinitely many values of $n$ for which $P_1(n),P_2(n),...,$ and $P_k(n)$ are all prime numbers.  

The 1st condition is necessary since if $P_i$ is reducible then $P_i(n)$ cannot be prime except in the finite number of cases where all but one of its factors are equal to 1 or -1.  The second condition is necessary as otherwise there will always be at least 1 of the $P_i(n)$ divisible by $p$; and thus not all of the $P_i(n)$ are prime except in the finite number of cases where one of the $P_i(n)$ is equal to $p$. 

It includes several other conjectures, such as the twin prime conjecture.
%%%%%
%%%%%
\end{document}
