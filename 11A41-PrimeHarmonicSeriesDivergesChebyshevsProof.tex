\documentclass[12pt]{article}
\usepackage{pmmeta}
\pmcanonicalname{PrimeHarmonicSeriesDivergesChebyshevsProof}
\pmcreated{2013-03-22 16:23:48}
\pmmodified{2013-03-22 16:23:48}
\pmowner{rm50}{10146}
\pmmodifier{rm50}{10146}
\pmtitle{prime harmonic series diverges - Chebyshev's proof}
\pmrecord{9}{38544}
\pmprivacy{1}
\pmauthor{rm50}{10146}
\pmtype{Theorem}
\pmcomment{trigger rebuild}
\pmclassification{msc}{11A41}

% this is the default PlanetMath preamble.  as your knowledge
% of TeX increases, you will probably want to edit this, but
% it should be fine as is for beginners.

% almost certainly you want these
\usepackage{amssymb}
\usepackage{amsmath}
\usepackage{amsfonts}

% used for TeXing text within eps files
%\usepackage{psfrag}
% need this for including graphics (\includegraphics)
%\usepackage{graphicx}
% for neatly defining theorems and propositions
%\usepackage{amsthm}
% making logically defined graphics
%%%\usepackage{xypic}

% there are many more packages, add them here as you need them

% define commands here

\begin{document}
\textbf{Theorem}. $\sum_{p\text{ prime}}\frac{1}{p}$ diverges.

\textbf{Proof.} (Chebyshev, 1880)
\newline
Consider the product
\[\prod_{p\leq n} \left(1-\frac{1}{p}\right)^{-1}\]
Since $\left(1-\frac{1}{p}\right)^{-1}=1+\frac{1}{p}+\frac{1}{p^2}+\cdots$, we have
\[\prod_{p\leq n} \left(1-\frac{1}{p}\right)^{-1}=\left(1+\frac{1}{2}+\frac{1}{2^2}+\cdots\right)\left(1+\frac{1}{3}+\frac{1}{3^2}+\cdots\right)\left(1+\frac{1}{5}+\frac{1}{5^2}+\cdots\right)\cdots\]
So for each $m\leq n$, if we expand the above product, $\frac{1}{m}$ will be a term. Thus
\[\prod_{p\leq n}\left(1-\frac{1}{p}\right)^{-1}\geq\sum_{x=1}^n \frac{1}{x}\]
Taking logarithms, we have
\[\sum_{p\leq n}-\ln\left(1-\frac{1}{p}\right)\geq\ln\sum_{x=1}^n \frac{1}{x}\]
But $\ln(1-u)=-u-\frac{u^2}{2}-\frac{u^3}{3}-\cdots$, so
\[-\ln\left(1-\frac{1}{p}\right)=\frac{1}{p}+\frac{1}{2p^2}+\frac{1}{3p^3}+\cdots\leq \frac{1}{p}+\frac{1}{p^2}+\frac{1}{p^3}+\cdots\leq \frac{2}{p}\]
Hence
\[\sum_{p\leq n} \frac{2}{p}\geq \sum_{p\leq n}\left(-\ln\left(1-\frac{1}{p}\right)\right)\geq \ln\sum_{x=1}^{n}\frac{1}{x}\]
and thus
\[\sum_{p\leq n}\frac{1}{p}\geq \frac{1}{2}\ln\sum_{x=1}^{n}\frac{1}{x}\]
But the latter series diverges, and the result follows.

%%%%%
%%%%%
\end{document}
