\documentclass[12pt]{article}
\usepackage{pmmeta}
\pmcanonicalname{EuclidsProofOfTheInfinitudeOfPrimes}
\pmcreated{2013-03-22 12:44:07}
\pmmodified{2013-03-22 12:44:07}
\pmowner{mathwizard}{128}
\pmmodifier{mathwizard}{128}
\pmtitle{Euclid's proof of the infinitude of primes}
\pmrecord{9}{33036}
\pmprivacy{1}
\pmauthor{mathwizard}{128}
\pmtype{Proof}
\pmcomment{trigger rebuild}
\pmclassification{msc}{11A41}

\endmetadata

% this is the default PlanetMath preamble.  as your knowledge
% of TeX increases, you will probably want to edit this, but
% it should be fine as is for beginners.

% almost certainly you want these
\usepackage{amssymb}
\usepackage{amsmath}
\usepackage{amsfonts}

% used for TeXing text within eps files
%\usepackage{psfrag}
% need this for including graphics (\includegraphics)
%\usepackage{graphicx}
% for neatly defining theorems and propositions
%\usepackage{amsthm}
% making logically defined graphics
%%%\usepackage{xypic}

% there are many more packages, add them here as you need them

% define commands here
\begin{document}
If there were only a finite amount of primes then there would be some largest prime $p$. However $p!+1$ is not divisible by any number $1<n\leq p$, since $p!$ is, so $p!+1$ cannot be factored by the primes we already know, but every integer greater than one is divisible by at least one prime, so there must be some prime greater than $p$ by which $p!+1$ is divisible.

Actually Euclid did not use $p!$ for his proof but stated that if there were a finite list $p_1,\ldots,p_n$ of primes, then the number $p_1\cdots p_n+1$ is not divisible by any of these primes and thus either prime and not in the list or divisible by a prime not in the list.
%%%%%
%%%%%
\end{document}
