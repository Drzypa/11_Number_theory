\documentclass[12pt]{article}
\usepackage{pmmeta}
\pmcanonicalname{IntegralBasisOfQuadraticField}
\pmcreated{2014-02-27 10:24:31}
\pmmodified{2014-02-27 10:24:31}
\pmowner{pahio}{2872}
\pmmodifier{pahio}{2872}
\pmtitle{integral basis of quadratic field}
\pmrecord{11}{40490}
\pmprivacy{1}
\pmauthor{pahio}{2872}
\pmtype{Derivation}
\pmcomment{trigger rebuild}
\pmclassification{msc}{11R04}
\pmsynonym{canonical basis of quadratic field}{IntegralBasisOfQuadraticField}
\pmsynonym{quadratic integers}{IntegralBasisOfQuadraticField}
\pmrelated{PropertiesOfQuadraticEquation}
\pmrelated{Gcd}
\pmrelated{ExamplesOfRingOfIntegersOfANumberField}
\pmrelated{SomethingRelatedToFundamentalUnits}
\pmrelated{CanonicalBasis}

\endmetadata

% this is the default PlanetMath preamble.  as your knowledge
% of TeX increases, you will probably want to edit this, but
% it should be fine as is for beginners.

% almost certainly you want these
\usepackage{amssymb}
\usepackage{amsmath}
\usepackage{amsfonts}

% used for TeXing text within eps files
%\usepackage{psfrag}
% need this for including graphics (\includegraphics)
%\usepackage{graphicx}
% for neatly defining theorems and propositions
 \usepackage{amsthm}
% making logically defined graphics
%%%\usepackage{xypic}

% there are many more packages, add them here as you need them

% define commands here

\theoremstyle{definition}
\newtheorem*{thmplain}{Theorem}

\begin{document}
\PMlinkescapeword{order}
Let $m$ be a squarefree integer $\neq 1$.  All numbers of the 
quadratic field $\mathbb{Q}(\sqrt{m})$ may be written in the 
form
\begin{align}
\alpha \;=\; \frac{j+k\sqrt{m}}{l},
\end{align}
where $j,\,k,\,l$ are integers with\, $\gcd(j,\,k,\,l) = 1$\, 
and\, $l > 0$.\, Then $\alpha$ (and its algebraic conjugate\, 
$\alpha' = \frac{j-k\sqrt{m}}{l}$) satisfies the equation
\begin{align}
x^2+px+q \;=\; 0,
\end{align}
where
\begin{align}
p \;=\; -\frac{2j}{l}, \qquad q \;=\; \frac{j^2-k^2m}{l^2}.
\end{align}
We will find out when the number (1) is an algebraic integer, i.e. when the coefficients $p$ and $q$ are rational integers.

Naturally, $p$ and $q$ are integers always when\, $l = 1$.\, We 
suppose now that\, $l > 1$.\, The latter of the equations (3) 
says that $q$ can be integer only when
$$(\gcd(j,\,l))^2 = \gcd(j^2,\,l^2) \mid k^2m$$
(see divisibility in rings).\, Because\, $\gcd(j,\,k,\,l) = 1$,\, 
we have by Euclid's lemma that\, $\gcd(j,\,l) \mid m$.\, Since 
$m$ is squarefree, we infer that
\begin{align}
\gcd(j,\,l) = 1.
\end{align}
In order that also $p$ were an integer, the former of the 
equations (3) implies that\, $l = 2$.

So, by the latter of the equations (3),\; $4 \mid j^2\!-\!k^2m$, 
i.e.
\begin{align}
k^2m \equiv j^2 \pmod{4}.
\end{align}
Since by (4),\, $\gcd(j,\,2) = 1$,\, the integer $j$ has to be odd.\, In order that (5) would be valid, also $k$ must be odd.\, Therefore,\, $j^2 \equiv 1 \pmod{4}$\, and\, $k^2 \equiv 1 \pmod{4}$,\, and thus (5) changes to 
\begin{align}
m \equiv 1 \pmod{4}.
\end{align}

If we conversely assume (6) and that $j,\,k$ are odd and\, $l = 2$, then (5) is true, $p,\,q$ are integers and accordingly (1) is an algebraic integer.

We have now obtained the following result:
\begin{itemize}
\item When\, $m \not\equiv 1 \pmod{4}$,\, the integers of the field $\mathbb{Q}(\sqrt{m})$ are
$$a+b\sqrt{m}$$
where $a,\,b$ are arbitrary rational integers;
\item when\, $m \equiv 1 \pmod{4}$,\, in \PMlinkescapetext{addition} to the numbers $a+b\sqrt{m}$, also the numbers
$$\frac{j+k\sqrt{m}}{2},$$
with $j,\,k$ arbitrary odd integers, are integers of the field.
\end{itemize}

Then, it may be easily inferred the 

\textbf{Theorem.}\, If we denote
%\begin{align*}
\[ 
\omega := 
\begin{cases}
& \frac{1+\sqrt{m}}{2} \quad \mbox{when  } m \equiv 1\pmod{4},\\
& \sqrt{m} \quad \mbox{   when  } m \not\equiv 1\pmod{4},
\end{cases}
\]
%\end{align*}

then any integer of the quadratic field $\mathbb{Q}(\sqrt{m})$ 
may be expressed in the form
$$a\!+\!b\omega,$$
where $a$ and $b$ are uniquely determined rational integers.\, 
Conversely, every number of this form is an integer of the field.\, One says that 1 and $\omega$ form an integral basis of the field.

\begin{thebibliography}{9}
\bibitem{K.V.} {\sc K. V\"ais\"al\"a}: {\em Lukuteorian ja korkeamman algebran alkeet}.\, Tiedekirjasto No. 17.\quad  Kustannusosakeyhti\"o Otava, Helsinki (1950).
\end{thebibliography}\\


%%%%%
%%%%%
\end{document}
