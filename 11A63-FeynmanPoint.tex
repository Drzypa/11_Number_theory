\documentclass[12pt]{article}
\usepackage{pmmeta}
\pmcanonicalname{FeynmanPoint}
\pmcreated{2013-03-22 16:26:01}
\pmmodified{2013-03-22 16:26:01}
\pmowner{PrimeFan}{13766}
\pmmodifier{PrimeFan}{13766}
\pmtitle{Feynman point}
\pmrecord{5}{38586}
\pmprivacy{1}
\pmauthor{PrimeFan}{13766}
\pmtype{Example}
\pmcomment{trigger rebuild}
\pmclassification{msc}{11A63}

% this is the default PlanetMath preamble.  as your knowledge
% of TeX increases, you will probably want to edit this, but
% it should be fine as is for beginners.

% almost certainly you want these
\usepackage{amssymb}
\usepackage{amsmath}
\usepackage{amsfonts}

% used for TeXing text within eps files
%\usepackage{psfrag}
% need this for including graphics (\includegraphics)
%\usepackage{graphicx}
% for neatly defining theorems and propositions
%\usepackage{amsthm}
% making logically defined graphics
%%%\usepackage{xypic}

% there are many more packages, add them here as you need them

% define commands here

\begin{document}
In the base 10 representation of $\pi$, at positions 762 through 767 (after the decimal point) there are six instances of the digit 9, called the {\em Feynman point}. Physicist Richard Feynman joked that he wanted to memorize $\pi$ up to that point: 3.14159265358979323846264338 ... 77130996051870721134999999... Such a recitation would erroneously suggest that $\pi$ is a rational number, the quotient of two integers that happen to be coprime to the base.
%%%%%
%%%%%
\end{document}
