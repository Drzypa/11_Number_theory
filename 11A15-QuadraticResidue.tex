\documentclass[12pt]{article}
\usepackage{pmmeta}
\pmcanonicalname{QuadraticResidue}
\pmcreated{2013-03-22 11:55:19}
\pmmodified{2013-03-22 11:55:19}
\pmowner{mathcam}{2727}
\pmmodifier{mathcam}{2727}
\pmtitle{quadratic residue}
\pmrecord{9}{30622}
\pmprivacy{1}
\pmauthor{mathcam}{2727}
\pmtype{Definition}
\pmcomment{trigger rebuild}
\pmclassification{msc}{11A15}
\pmrelated{LegendreSymbol}
\pmrelated{EulersCriterion}
\pmdefines{quadratic non-residue}
\pmdefines{quadratic nonresidue}

\usepackage{amssymb}
\usepackage{amsmath}
\usepackage{amsfonts}
\usepackage{graphicx}
%%%%\usepackage{xypic}
\begin{document}
Let $a,n$ be relatively prime integers. If there exists an integer $x$ that satisfies $$x^2 \equiv a \pmod{n}$$ then $a$ is said to be a \emph{quadratic residue} of $n$.  Otherwise, $a$ is called a \emph{quadratic nonresidue} of $n$.
%%%%%
%%%%%
%%%%%
%%%%%
\end{document}
