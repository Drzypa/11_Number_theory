\documentclass[12pt]{article}
\usepackage{pmmeta}
\pmcanonicalname{EulerProduct}
\pmcreated{2013-03-22 14:10:58}
\pmmodified{2013-03-22 14:10:58}
\pmowner{bbukh}{348}
\pmmodifier{bbukh}{348}
\pmtitle{Euler product}
\pmrecord{8}{35609}
\pmprivacy{1}
\pmauthor{bbukh}{348}
\pmtype{Definition}
\pmcomment{trigger rebuild}
\pmclassification{msc}{11A05}
\pmclassification{msc}{11A51}
%\pmkeywords{infinitude of primes}
\pmrelated{MultiplicativeFunction}
\pmrelated{RiemannZetaFunction}

\usepackage{amssymb}
\usepackage{amsmath}
\usepackage{amsfonts}
\usepackage{amsthm}

\newcommand*{\abs}[1]{\left\lvert #1\right\rvert}

\makeatletter
\@ifundefined{bibname}{}{\renewcommand{\bibname}{References}}
\makeatother
\begin{document}
If $f$ is a multiplicative function, then
\begin{equation}\label{eq:euler_prod}
\sum_{n=1}^\infty f(n)=
\prod_{p\text{ is prime}}(1+f(p)+f(p^2)+\dotsb)
\end{equation}
provided the sum on the left converges absolutely. The product
on the right is called the \emph{Euler product} for the sum on the
left.

\begin{proof}[Proof of \eqref{eq:euler_prod}]
Expand partial products on right of \eqref{eq:euler_prod} to
obtain by fundamental theorem of arithmetic
\begin{align*}
\prod_{p<y}(1+f(p)+f(p^2)+\dotsb)&=
\sum_{k_1}f(p_1^{k_1})\sum_{k_2}f(p_2^{k_2}) \dotsb\sum_{k_t}f(p_t^{k_t})\\
&=\sum_{k_1,k_2,\dotsc,k_t} f(p_1^{k_1})f(p_2^{k_2})\dotsb f(p_t^{k_t})\\
&=\sum_{k_1,k_2,\dotsc,k_t} f(p_1^{k_1}p_2^{k_2}\dotsb p_t^{k_t})\\
&=\sum_{P_+(n)<y} f(n)
\end{align*}
where $p_1,p_2,\dotsc,p_t$ are all the primes between $1$ and $y$, and $P_+(n)$ denotes the largest prime factor of $n$. Since
every natural number less than $y$ has no factors exceeding $y$ we have
that
\begin{equation*} \abs{\sum_{n=1}^\infty f(n)-\sum_{P_+(n)<y}
f(n)}\leq\sum_{n=y}^\infty \abs{f(n)}
\end{equation*}
which tends to zero as $y\to \infty$.
\end{proof}

\section*{Examples}
\begin{itemize}
\item If the function $f$ is defined on prime powers by
$f(p^k)=1/p^k$ for all $p<x$ and $f(p^k)=0$ for all $p\geq x$,
then \PMlinkescapetext{Euler's product formula} allows one to estimate $\prod_{p<x}
(1+1/(p-1))$
\begin{equation*}
\prod_{p<x} \left(1+\frac{1}{p-1}\right) =\prod_{p<x}
\left(1+\frac{1}{p}+\frac{1}{p^2}+\dotsb\right)=\sum_{P_+(n)<x}
\frac{1}{n}>\sum_{n<x}\frac{1}{n}>\ln x.
\end{equation*}
One of the consequences of this formula is that there are
infinitely many primes.

\item The Riemann zeta function is defined by the means of the
series
\begin{equation*}
\zeta(s)=\sum_{n=1}^\infty n^{-s}\qquad\text{for $\Re s>1$}.
\end{equation*}
Since the series converges absolutely, the Euler product for the zeta function is
\begin{equation*}
\zeta(s)=\prod_{p}\frac{1}{1-p^{-s}}\qquad\text{for $\Re s>1$}.
\end{equation*}
If we set $s=2$, then on the one hand $\zeta(s)=\sum_n 1/n^2$ is
$\pi^2/6$ (proof is
\PMlinkname{here}{ValueOfTheRiemannZetaFunctionAtS2}), an
irrational number, and on the other hand $\zeta(2)$ is a product of rational functions of primes. This yields yet another proof of infinitude of
primes.
\end{itemize}
%%%%%
%%%%%
\end{document}
