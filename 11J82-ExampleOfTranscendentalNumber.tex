\documentclass[12pt]{article}
\usepackage{pmmeta}
\pmcanonicalname{ExampleOfTranscendentalNumber}
\pmcreated{2013-03-22 15:02:45}
\pmmodified{2013-03-22 15:02:45}
\pmowner{alozano}{2414}
\pmmodifier{alozano}{2414}
\pmtitle{example of transcendental number}
\pmrecord{7}{36760}
\pmprivacy{1}
\pmauthor{alozano}{2414}
\pmtype{Example}
\pmcomment{trigger rebuild}
\pmclassification{msc}{11J82}
\pmclassification{msc}{11J81}
%\pmkeywords{transcendental}
%\pmkeywords{algebraic}
\pmrelated{LiouvillesTheorem}
\pmrelated{RothsTheorem}

% this is the default PlanetMath preamble.  as your knowledge
% of TeX increases, you will probably want to edit this, but
% it should be fine as is for beginners.

% almost certainly you want these
\usepackage{amssymb}
\usepackage{amsmath}
\usepackage{amsthm}
\usepackage{amsfonts}

% used for TeXing text within eps files
%\usepackage{psfrag}
% need this for including graphics (\includegraphics)
%\usepackage{graphicx}
% for neatly defining theorems and propositions
%\usepackage{amsthm}
% making logically defined graphics
%%%\usepackage{xypic}

% there are many more packages, add them here as you need them

% define commands here

\newtheorem{thm}{Theorem}
\newtheorem{defn}{Definition}
\newtheorem{prop}{Proposition}
\newtheorem{lemma}{Lemma}
\newtheorem{cor}{Corollary}

% Some sets
\newcommand{\Nats}{\mathbb{N}}
\newcommand{\Ints}{\mathbb{Z}}
\newcommand{\Reals}{\mathbb{R}}
\newcommand{\Complex}{\mathbb{C}}
\newcommand{\Rats}{\mathbb{Q}}
\begin{document}
The following is a classical application of Liouville's approximation theorem. For completeness, we state Liouville's result here:

\begin{thm}
For any algebraic number $\alpha$ with degree $m>1$, there exists a constant $c=c(\alpha)>0$ such that:
$$|\alpha-\frac{p}{q}|> \frac{c}{q^m}$$
for all rationals $p/q$ (with $q>0$).
\end{thm}

Next we use the theorem to construct a transcendental number.

\begin{cor}
The real number
$$\psi= \sum_{n=1}^\infty \frac{1}{10^{n!}}=0.1100010\ldots$$
is transcendental.
\end{cor}
\begin{proof}
Clearly, the number $\psi$ is well defined, i.e. the series converges. Indeed, 
$$\frac{1}{10^{n!}}<\frac{1}{10^n}$$
and $\sum_{n=1}^\infty 10^{-n}=1/9$. Thus, by the comparison test, the series converges and $0<\psi<1/9$.\\

Suppose, for a contradiction, that $\psi$ is algebraic of degree $m$. We will construct infinitely many rationals $p/q$ such that 
$$|\psi - \frac{p}{q}|< \frac{c}{q^m}$$
where $c=c(\psi)$ is the constant given by the theorem above. Let $k\in \Nats$ be such that $1/2^k < c$. Then, in fact, we will show that there are infinitely many rationals $p/q$ with $q\geq 2$ such that 
$$|\psi -\frac{p}{q}|<\frac{1}{q^{m+k}}<\frac{1}{2^k}\cdot \frac{1}{q^m}<\frac{c}{q^m}$$
For all $j>k+m$ we define a rational number $p_j/q_j$ by:
$$p_j=10^{j!}\sum_{n=1}^j 10^{-n!},\quad q_j=10^{j!}$$
then $p_j$ and $q_j$ are relatively prime integers and we have:
\begin{eqnarray*}
|\psi - \frac{p_j}{q_j}| & = & \sum_{n=j+1}^\infty \frac{1}{10^{n!}}\\
& < & \frac{1}{10^{(j+1)!}}(1+\frac{1}{10}+\frac{1}{10^2}+\ldots)\\
& = & 10/9\cdot \frac{1}{q_j^{(j+1)}} \\
& < & \frac{1}{q_j^j}\\
& < & \frac{1}{q_j^{(k+m)}}
\end{eqnarray*}
where in the last inequality we have used the fact that $j>k+m$. Therefore, all rationals $\{ p_j/q_j \}_{j=k+m+1}^\infty$ satisfy the desired inequality, which leads to the contradiction with the theorem above. Thus $\psi$ cannot be algebraic and it must be transcendental.
\end{proof}

Many other similar transcendental numbers can be constructed in this fashion.
%%%%%
%%%%%
\end{document}
