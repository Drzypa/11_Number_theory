\documentclass[12pt]{article}
\usepackage{pmmeta}
\pmcanonicalname{NumericalVerificationOfTheGoldbachConjecture}
\pmcreated{2014-10-25 22:17:31}
\pmmodified{2014-10-25 22:17:31}
\pmowner{Paulo Fernandesky}{1000738}
\pmmodifier{unlord}{1}
\pmtitle{Numerical verification of the Goldbach conjecture}
\pmrecord{5}{88163}
\pmprivacy{1}
\pmauthor{Paulo Fernandesky}{1}
\pmtype{Conjecture}
\pmclassification{msc}{11N05}
\pmclassification{msc}{11A41}
\pmclassification{msc}{11A25}
\pmclassification{msc}{11Y11}
\pmclassification{msc}{11P32}
\pmclassification{msc}{05A10}
\pmclassification{msc}{11N56}
\pmclassification{msc}{11D99}
\pmclassification{msc}{11P99}
\pmclassification{msc}{11N32}
\pmclassification{msc}{05A17}

\endmetadata

% this is the default PlanetMath preamble.  as your knowledge
% of TeX increases, you will probably want to edit this, but
% it should be fine as is for beginners.

% almost certainly you want these
\usepackage{amssymb}
\usepackage{amsmath}
\usepackage{amsfonts}

% need this for including graphics (\includegraphics)
\usepackage{graphicx}
% for neatly defining theorems and propositions
\usepackage{amsthm}

% making logically defined graphics
%\usepackage{xypic}
% used for TeXing text within eps files
%\usepackage{psfrag}

% there are many more packages, add them here as you need them

% define commands here
\usepackage[latin1]{inputenc}
\usepackage{graphicx}
\usepackage{amsmath}
\usepackage{amsfonts}
\usepackage{amssymb}
\usepackage{psfrag}
\usepackage{xypic}
\usepackage{pstricks}
\usepackage{xypic} 
\usepackage{bbm}
\newcommand{\Z}{\mathbbmss{Z}}
\newcommand{\C}{\mathbbmss{C}}
\newcommand{\R}{\mathbbmss{R}}
\newcommand{\Q}{\mathbbmss{Q}}

\begin{document}
\begin{quote}
\item ABSTRACT:
\ \
The Strong Goldbach conjecture, GC, dates back to $1742 $. It states that every even integer greater than four can be written as the sum of two prime numbers. Since then, no one has been able to prove the conjecture. The conjecture has been verified to be true for all even integers up to $4.10^{18} $. In this article, we prove that the conjecture is true for all integers, with at least three different ways.
In short, this treaty has as objective show the proof of GC, and presents a new resolution to the conjecture. Knowing that, these infinities establish other groups of infinities, in a  logical way the conviction for the method and idea of proving it, we stand and separate these groups to prove, not only a sequence, but the whole embodiment of arithmetic properties called here as groups, as well as its infinity conjectured for centuries.
\\

\item Keywords:
Goldbach's Conjecture; Crystallographic group; Cobordism group; Algebraic number theory; Multiprime Theorem's;  Productoria Table.
\\

\item AMS Subject Classification:
11N05; 11A41; 11A25; 11Y11; 11P32; 05A10; 11N56; 11D99; 11P99; 11N32; 05A17.
\end{quote}



% \section{Introduction}
% The primes numbers have fascinated mathematicians around the world since many centuries. We agree that these are mathematical properties of all kinds. We know, for example, the prime factor: $\cite{SJ} $ a factorial prime is a number that is one less or one more than a factorial and is also a prime number. The first few factorial primes are:
% \newline $2, 3, 5, 7, 23, 719, 5039, 39916801, 479001599, 87178291199\dots $ It is conjectured that only for $\textit{n}=3$ are both $\textit{n}!+1$ and $\textit{n}!+1$ both primes.
% Factorial primes have a rôle in an argument that 1 is not a prime number. If n is apositive integer and \textit{p} is a prime number, $\textit{n}! $+p is never a prime for p$<n $, because obviously it will be a multiple of \textit{p}, just as n! is. But $\textit{n}! $+1, even though it certainly is a multiple of $1$, can be a prime, specifically, a factorial prime. (The same is also true if we subtract instead of add).
% On the other hand, we know your partitions: A prime partition is a partition of a given positive integer \textit{n} consisting only of prime numbers. For example, a prime partition of $42$ is $29 + 5 + 5 + 3$.
% If we accept partitions of length 1 as valid partitions, then it is obvious that only prime numbers have prime partitions of length $1$. Not accepting $1$ as a prime number makes the problem of prime partitions more interesting, otherwise there would always be for a given \textit{n}, if nothing else, a prime partition consisting of \textit{n}$1$s. Almost as bad, however, is a partion of \textit{n} into $[n2]$ 2s and $3s$.
% Both GC and Levy's conjecture can be restated in terms of prime partitions thus: for any even integer \textit{n}$>2$ there is always a prime partition of length $2$, and for any odd integer \textit{n}$>5$ there is always a prime partition of length $3$ with at most $2$ distinct elements.
% Assuming GC is true, the most efficient prime partition of an even integer is of length $2$, while Vinogradov's theorem has proven the most efficient prime partition of a sufficiently large composite odd integer is of length $3$.
% Still, we proceed as: Given two consecutive odd primes, the $i$th prime $p_i$ and the next one, $p_{i + 1}$, an {\em interprime} $n$ is the arithmetic mean of the two: $$n = \frac{p_i + p_{i + 1}}{2}$$ Thus, $n - p_i = p_{i + 1} - n$, so alternatively $$n = p_i + \frac{p_{i + 1} - p_i}{2} = p_{i + 1} - \frac{p_{i + 1} - p_i}{2}.$$ For example, given the 269th and 270th primes, 1723 and 1733, the interprime is 1728, and indeed $1728 - 1723 = 1733 - 1728 = 5$. Interprimes themselves are of course always composite, though not always even. An interprime between a twin prime will always be even, while an interprime between the second (ending in 3 in base 10) and third (ending in 7 in base 10) member of a prime quadruplet will always be odd and be divisible by 5. 
% The first few interprimes are $4, 6, 9, 12, 15, 18, 21, 26, 30, 34, 39, 42, 45, 50, 56, 60, 64, 69, 72, 76, etc\dots$
% Or still, To give an example of a probable prime by a pattern: this pattern suggests that $2^{170141183460469231731687303715884105727} - 1 $ might be a Mersenne prime. But since this is larger than the largest known Mersenne prime $2^{30402457} - 1$ (as of 2005), a Lucas-Lehmer test might take longer than the average human lifetime.
% Also found that: The ternary GC is the as sertion that every odd number $\textit{n}> 5 $ is the sum of three primes. The binary, or strong, GC states that every even number $> 2 $ is the sum of two primes; this would imply ternary Goldbach trivially. The binary GC has been numerically verified to 4 $\dot 1018 $ by Oliveira e Silva, Siegfried Herzog and Silvio Pardi $\cite{OS} $. In the First version of $\cite{HH} $, the First author used the same argument to show that the verification of the Riemann Hypothesis by the second author $\cite{DP} $ (to $T > 3.061$\dot $1010) $, Wedeniwski $\cite{SW} $ (to $T > 2.419 $\dot $ 1011 $) or Gourdon $\cite{XG} $ (to $T > 2.4599 $$\dot 1012$ ) imply ternary Goldbach holds to $1.23163 $$\dot 1027 $, $6.15697 $$\dot $1028 or 5.90698$\dot $ $1029 $ respectively.
% \newline
% But we wish to address in this paper is the same Goldbach Conjecture. GC.
% 
% 
% %%%%%
% Goldbach's conjecture states that: \cite{EG}every even number greater than 4 can be written as a sum of two primes, namely:
% \[ 2\textit{n}=p+q, \]
% where n$>4 $ and \textit{p, q} are prime numbers. 
% \\
% \subsection{Proth Primes}
%  Proving a number of general form is prime can be computationally expensive. For example, testing a number of size $\textit{ N} $  using Elliptic Curve Primality Proving (ECPP) has time complexity (heuristically) $\cite{ARK} $$\mathcal{O} $($\log^{4\in} N). $ Fortunately, there exist primes of special forms that are much easier to test. We use Proth primes.
% 
% 
% \newenvironment{dem}[1][Definition 2.1]{\textbf{#1:}\ } {\rule{1ex}{1ex}} \begin{dem} A Proth number is of the form $\textit{k}\cdot 2^n+1$ with $ k,n \in \mathbb{Z}_{>0} $ and $k < 2^n. $. \end{dem}
% \\
% 
% \begin{dem}[Definition 2.2, and Proth's Theorem] A Proth prime is a Proth number that is also prime.
% \textit{A Proth number N} = $ k \cdot 2^n+1 $  is prime if for some integer a with
% \begin{flushleft}
% (a) $(\dfrac{a}{N})=-1 $
% \end{flushleft}
% \\
% we have
% \begin{flushleft}
% (b) $a_$\dfrac{N-1}{2} \equiv$ -1 \mod \textit{N} $
% \end{flushleft}
% \end{dem}
% \\
% (Conversely, if \textit{N} is prime, then (b) holds for every a satisfying (a).)
% This suggests the following algorithm for testing a Proth number for primality.
% \newline
% (1) Let, $n,k \in \mathbb{Z}_{>0} $ with $k <2^n$
% \newline
% (2) Set $k \cdot $ 2^n + 1 $\to N $
% \newline
% (3) Set $2 $\to $ p $
% \newline
% (4) $\mbox{While} $ p <= if ($\dfrac{p}{N} $) $=-1 $, then break, else set $p $ to the first prime after $p$.
% \newline
% (5) If $p > B $(we ran out of small primes) return false.
% \newline
% (6) Return true if $p \dfrac{N-1}{2} $$\equiv -1 $ mod $N $ and false if $p \dfrac{N-1}{2} $ $\not\equiv $ mod $N $
% \newline
% The modular exponentiation at step $6 $ requires $\mathcal{O}(logN) $ multiplications. Computing the Jacobi symbol at step $4 $ has (since a is bounded) time complexity $\mathcal{O}(log(N)) $; we do it $\mathcal{O}(1) $ times. The algorithm may return a false negative if none of the primes $\leq B $  are quadratic non-residues but $N $ happens to be prime. However, the algorithm never returns a false positive.
% 
% \\
% 
% 
% \subsection{Definitions of the Goldbach's Conjecture}
% We have present a practical resolution for the GC, expressing that,
% \begin{center}
% \textit{ For every even number there are two primes both share a common prime $k $, and the value of $k$ being negative in one of the primes, and in the other positive. The addition of the number $3 $ in the two prime numbers we obtain any odd number}.
% \end{center}
% Recalling the famous statement concerning to GC had its first concept stated in a letter of Christian Goldbach sent to the Swiss mathematician Leonard Euler on July 7th, 1742 $\cite{MPM}$.
% \textit{"You can express even numbers greater than 2, as the sum of two primes?"} 
% 
% It is noticed that the concept of Goldbach motivates Euler to believe that the concept would be part of a broader conjecture about the numbers. Perhaps for this reason, the conjecture should be called refined GC, giving credit in a fairer way. Since Euler, in response to Goldbach'  letter, enunciated: $\dots$\textit{Dass ein jeder numerus par eine summa Duorum primorum sey, halte ich für ein ganz gewisses Theorema, ungeachtet ich kann nicht desselbe demonstriren.} (Every even number is the sum of two primes. I see this theorem  with absolute certainty, although I cannot prove it).
% 
% Later, due to the interest of some mathematicals, the conjecture has been considered an affirmation due to combinations. Eg, George Singer, who in 1894 made all possible decompositions for the sum of two prime numbers minor than $1,000 $. List extended to 2000 by Aubrym, and until 5000 by R. Haussnerm in 1897.
% In 1937, the Soviet mathematician Vinogradov proved using methods of trigonometry, that any sufficiently large odd number is the sum of three primes. Currently, with the help of the computer, it has been possible to verify the veracity of the hypothesis for numbers in the order of 10 to the 14.
% 
% 
% 
% \subsection{Aplications}
% 
% Given these facts, we conceptualize that there are infinite primes with differentiated value differences, which together form countless even numbers, even completing the gaps of primes and other sets, when compared to odd numbers or not. This is exemplified in the following statement:
% 
% For every even number, there are two primes, and both primes share a common k, and the value of k being negative in one of the primes, and in the other positive. The addition of the number 3 in the these two primes we obtain any odd number.
% 
% At first, we observed that even a small number like 19 combined with its predecessors already fills an almost perfect sequence of pairs, since its first combination with 3 up to the combination with itself, beyond the endless combination with infinite primes that are greater than 19.
% 
% In short, other primes can leave gaps missing in the first endless sets, as is the case of 37 and 31. However, if all p is added to its equal, its predecessors and successors, proves the truth that infinite groups, formed by these primes, for each prime: with their smaller equal and bigger, form sequences and countless groups, ie, infinite sequences. An example is shown in chart 1.
% We know that when the sequence of even numbers is achieved by simply adding only one 3 to have any odd number greater than 7, thus proving the Goldbach' weak conjecture $\cite{CC} $.
% \newline
% %
% \\
% \begin{tabular}{||r|r|r|r|r|r|r|r|r|r|r||}
% \hline
%  m$\to $ &  1  & 2 &  2 &  1 &  3  &  3 &  1 & 3 &  1 &  2$\dots $ \\  
% \hline
%   k=   &  8 $  & 18  &  30  &  36  &  52  &  68 & 84 & 100 & 120 & 138\dots $ \\  
% \hline
%     5  &&&&&&&&&&    \\
%     4  &&&&&&&&&&     \\
%     3  &&&&& 29 & 37 && 53 &&     \\
%     2  &&  11 &  17 &&&&&&& 71   \\
%   1  &  5 &&& 19 &&& 43 &&  61 &        \\
%  $\large {m}$ & (4)& (9) &(15) &(18)&(26)&(34)&(42)&(50)& (60)&(69)$\dots $ \\
%  $-1$  &  3 &&& 17 &&& 41 &&  59  &      \\
%  $-2$  &&   7 & 13 &&&&&&&  67  \\
%  $-3$  &&&&& 23  & 31 && 47 &&     \\
%  $-4$  &&&&&&&&&&     \\
%  $-5$  &&&&&&&&&&     \\
% \hline
% \end{tabular}
% \\
% \footnotesize{Table 1 -    Difference in $k $ aplication about $\dfrac{p+q}{2}$.}
% \newline
% % % % % % %%
% \\
% \newline
% \newenvironment{dem}[1][Lemma 1.1]{\textbf{#1:}\ } {\rule{1ex}{1ex}} \begin{dem} A productoria table is presented to illustrate these properties. This table $\cite{FP} $ , as we have seen, does not present all pairs of prime numbers that can be done to get the same pair, because it is infinite as well as integers: well, after the even number 668, we also have: \[  390, 462, 618, 630, 702, 712, 740, 752, 772, 872, 892, 1012, 1088, 1120, 1132, 1148. \] where $ k = 3 $ in this sequence. The chart also shows a practical way of viewing the values of $k $in each even number. In the second, the third and the fourth column, appear the even numbers which are found from the value of $k $. Dividing these results by two and subtracting the value of $k $, we find the first prime number, the closest to $\dfrac{n}{2} $.
% 
% 
% For example, the even number 44 was not shown because its axiom is farther than $ = 3 $. Ie, its first axiom is formed by $ k = 9 $, followed by: $ k = 15 $ and $ k = 19 $ what makes us find its first primes. The subtraction of $ 44 $ from this first prime number, generates another prime number, therefore, the sum of the same value of k above $\dfrac{n}{2} $: where, $p-k+(p+k)=m $. \end{dem}
% \\
% 
% 
% \newline
% \begin{tabular}{||c|c|c|c||}
% \hline (p+q)=m & p+q[k=±1] & p+q[k=±2] & p+q[k=±3]$\dots $$\to $ \\ 
% \hline 4=2+2  & 8 & 10 & 16 \\ 
% 6=3+3  & 12 & 18 & 20 \\ 
% 10=5+5  & 24 & 30 & 28 \\ 
% 14=7+7  & 36 & 42 & 32 \\ 
% 22=11+11  & 60 & 78 & 40 \\ 
% 26=13+13  & 84 & 88 & 52 \\ 
% 34=17+17  & 120 & 90 & 68 \\ 
% 38=19+19  & 144 & 138 & 80 \\ 
% 46=23+23  & 204 & 162 & 100 \\ 
% 58=29+29  & 216 & 198 & 112 \\ 
% 62=31+31  & 276 & 210 & 122 \\ 
% 74=37+37  & 300 & 222 & 128 \\ 
% 82=41+41  & 360 & 258 & 152 \\ 
% 86=43+43  & 384 & 390 & 172 \\ 
% 94=47+47  &  396 & 462 & 268 \\ 
% 106=53+53  & 456 & 618 &  308\\ 
% 118=59+59  & 540 & 630 & 320 \\ 
% 122=61+61  & 564 & 702 &  340\\ 
% 134=67+67 & 624 & 798 &  472\\ 
% 142=71+71 & 696 & 918 &  508\\ 
% 146=73+73 & 864 & 930 & 520 \\ 
% 158=79+79 & 924 & 978 &  532\\ 
% $\vdots $ & $\vdots $ & $\vdots $ & $\vdots $ \\ 
% \hline
% \end{tabular}
% \\
% 
% \footnotesize{Table 2 -    The productoria of numbers m-pairs of the Goldbach Conjecture.}
% \newline
% \newline
% \begin{dem}[Theorem 2.1] This fact implies the following simplified theorem: $x=\dfrac{m}{2}$\pm $k $\to $ x-p=q $
% \\
% \mbox{Whe its complete form is:}
% \newline
% \[ p=\dfrac{m}{2}+k + (q=\dfrac{m}{2}-k)$ if k$=0,1,2,3... \]
% \newline
% \\
% % % % % % %
% \mbox{Next,}
% \newline
% (1) Let, $\dfrac{m}{2} \pm k \to p$
% \newline
% (2) Set, $k \pm 1,2,3,4,\dots $
% \newline
% (3) Set, $p=\dfrac{m}{2} $
% \newline
% (4) if $k= 1,2,3,4,\dots \to \infty $.
% \newline
% \mbox{Where its theorem, and the proof of the conjecture is:}
% \\
% This theorem, where $m $ is an even number, is applied in the following manner to the number $32 $:
% 32 $\equiv ${p+q= $\dfrac{32}{2}+3$ + ($\dfrac{32}{2}-3} $ ) if $k=3 $
% \newline
% % % % % % % %
% \displaystyle \sum_{p=\dfrac{m}{2}+k}^{q=\dfrac{m}{2}-k} \left(+$\mbox{\textit{i}})=\prod\limits_{k=0 , i=3}^{n} \left={\dfrac {m_n}{2}$\pm $k, \dfrac {m_n}{2}$\pm $(k+1), \dfrac {m_n}{2}$\pm $(k+2),$\dots$+ $\dfrac {m_n}{2}$$\pm $(k+n).}$
% \end{dem}
% % % % % % % % % % % % % % % % % % %
% \\
% % % % % % % % % % % % % % % % % % % % %
% $$
% \newline
% \begin{dem}[Lemma 1.2] Thus, $k = 0 $, and $\textit{i} $ can be equal to any prime number, which in this case is exemplified by $i = 3$, because as said before, any even number, as well as its cadence sequence added to the prime number $3$, generates the complete sequence of odd numbers. Thus, if we apply the sum of the prime number $3 $ to the results got in the set ${m, m +2, m +3\dots m + n}$, we get the complete sequence of odd numbers, also reaching the proof of the weak conjecture of Goldbach}.
% \end{dem}
% \newline
% \subsection{First conclusion }
% A perfect resolution is given by the equation: $(x \pm k) = m$ where $m = p + q$. Then $p+(x+x)=m $.
% In this case we take for example any set $x$. We know that the two values of $k $ are only applicable when $k $ and $x $ have the same value. Ie when $k $ passes or exceeds the value of $x $, you can only view a value of $k $: The $+ k $ value.
% From that point on all natural numbers appear in a progressive and perfect manner. Then, if $x \to k $ can have the same value of any integer from the moment where $k \equiv x$, the set of $ x + x $ always find a M-pair throughout the whole infinite set of integers, because x represents an infinite of sets formed by the natural numbers. Where m is the sum of $(x + x) $, being a $k $ value applied to finding a prime. Therefore, $k$ is also an infinite number and steadily rhythmic from $x \equiv k$.
% \newline (1) Let, $(-x),+x \to k $
% \newline (2) Set, $\mathbb{N} \to p+(x+x)$
% \newline (3) Set, $m \equiv \mathbb{N} $
% \newline (4) Set, $(-x),+x \to k $, if $k \equiv x $
% \newline (5) Next, $+x \to k $
% \newline (6) While, $(\dfrac{m}{2}+x),-x \to \mathbb{N}. $
% 
% \newline
% \section{The Conjecture Refined}
% It would be interesting also designate an absolute form for the compositions of the odd integers and even numbers formed by many prime numbers, which we chose the following statement:
% %
% \begin{center}
% For every integer, twice higher than x, there are two primes, and both primes share a $k$ related to  $x$, being the value of $k$ negative in some primes, and in the others positive. Adding one $3$ to the value of  these $x$ primes we  obtain  any even number from its sequence^{*1}.
% \end{center}
% \\
% %
% This truth is somehow only possible for the veracity of GC, which we are based and otherwise express: that any integer, 2 times greater than the quantity of prime terms, can be written through the sum of the same quantity of prime terms.
% Actually, we don?t need to add the prime number $3 $ to find the even and prime numbers, when we apply a sequence of the prime $2 $ with or without the $3 $ in front, which makes us reach any integer. However, the refinement of the conjecture states that it is not possible only to find all the even integers only from formed sets from the summing of two primes, but it can be found also by the summing of three primes, four, five and successively. This is perfectly understandable, when we add a number $2 $ in front of the perfect sequence of pairs of Goldbach? conjecture. Also adding two numbers $2 $, to find the sequence of m-pairs formed by four primes.
% \newline
% It is noted from the outset that the symmetry found in the values of k, related to the Goldbach? conjecture, which has just been presented, is no longer found due to the inactivity which is among the quantity of the prime terms (x) and k values, when related to Goldbach?s conjecture.
% \newline
% If $+p ={x = odd}$ whenever $x $ is = even, $p $ will be $= 0$. Thus, we find the results; table 2.
% \newline
% 
% \begin{tabular}{||c|c|c|c||}
% \hline (x±1)=m & (3±k)=m & (4±k)=m & (5±k)=m$\dots $$\to $ \\ 
% \hline 3+1=4	& 3+1=4 & 4+1=5 & 5+1=6\\
% 3-1=2	& 3-1=2 & 4-1=3 & 5-1=4\\
% 4+1=5	& 3+2=5 & 4+2=6 & 5+2=7\\
% 4-1=3	& 3-2=1 & 4-2=2 & 5-2=3\\
% 5+1=6	& 3+3=6 & 4+3=7 & 5+3=8\\
% 5-1=4	& 3-3=0 & 4-3=1 & 5-3=2\\
% 6+1=7	& 3+4=7 & 4+4=8 & 5+4=9\\
% 6-1=5	& 3+5=8 & 4-4=0 & 5-4=1\\
% 7+1=8	& 3+6=9 & 4+5=9 & 5+5=10\\
% 7-1=6	& 3+7=10 & 4+6=10 & 5-5=0\\
% 8+1=9	& 3+8=11 & 4+7=11 & 5+6=11\\
% 8-1=7	& 3+9=12 & 4+8=12 & 5+7=12\\
% $\vdots $ & $\vdots $ & $\vdots $ & $\vdots $ \\ 
% \hline 
% \end{tabular}
% \\
% % % % % % % % % % % %
% \begin{center}
% \footnotesize{ Table 3. Table of the First Conclusion.}
% \end{center}
% \\
% \\
% \subsection{Aplications to Goldbach's Conjecture}
% The theorem is abbreviated as follows:
% 
% $
% \[ $m=\dfrac{m}{x}-k_{j+p}  \]
% Being $ j=\dfrac{x}{2}+p $ and $ x=odd$.
% And, being $k=0$ and $x=p+p+p+$\dots $n$.
% This happens due the fact that, there are primes which are composed of the sum of other primes. This is the main statement of this refining:
% There are infinite primes composed by the sum of infinite primes. Which presents the whole set of integers.
% In short, we apply these truths through the product,
% \newline
% \prod\limits_{ k=0, i=3}^{n} \left= x=p+p+p\dots+n.$\equiv $\prod\limits_{j=\dfrac{x}{2}+p  (x = odd.)}=\left $\dfrac {m_n}{x}-k_{j+p},\dfrac {m_n}{x}-(k+1)_{j+p}, \dfrac {m_n}{x}-(k+2)_{j+p}$\dots $, \dfrac {m_n}{x}-(k+n)_{j+p}$\rigth\ $+$\textit{i} $
% $
% \newline
% 
% \begin{flushleft}
% \newline (1) Let, $p+q \to m$ if $2^n$
% \newline (2) Set, $(x\pm k)-p \to m$
% \newline (3) Set, $+p{x=odd} $ and $x= pair; p \to=0$
% \newline (4) Set, $x=p+p+p+\dots n.$
% \newline (5) While $x= the number of terms.$
% \newline (6) Being $J+p = $the fragmentation of $x $ and the value of $ p = 1 $ when $x $ is an odd number.
% \newline
% \end{flushleft}
% \\
% 
% \begin{tabular}{||c|c|c|c||}
% \hline 
% im=(ppp)i=2 & im=(ppp)i=3 & im=(ppp)i=4 & im=(ppp)i=4$\dots $$\to $\\
% \hline 8=2,2,2,2 & 11=2,3,3,3 & 17=2,5,5,5 & 23=2,7,7,7\\
% 9=3,2,2,2 & 12=3,3,3,3 & 18=3,5,5,5 & 24=3,7,7,7\\
% 11=5,2,2,2 & 14=5,3,3,3 & 20=5,5,5,5 & 26=5,7,7,7\\
% 13=7,2,2,2 & 16=7,3,3,3 & 22=7,5,5,5 & 28=7,7,7,7\\
% 17=11,2,2,2 & 20=11,3,3,3 & 26=11,5,5,5 & 32=11,7,7,7\\
% 43=37,2,2,2 & 46=37,3,3,3 & 52=37,5,5,5 & 58=37,7,7,7\\
% 47=41,2,2,2 & 50=41,3,3,3 & 56=41,5,5,5 & 62=41,7,7,7\\
% 49=43,2,2,2 & 52=43,3,3,3 & 58=43,5,5,5 & 64=43,7,7,7\\
% 53=47,2,2,2 & 56=47,3,3,3 & 62=47,5,5,5 & 68=47,7,7,7\\
% 57=53,2,2,2 & 62=53,3,3,3 & 68=53,5,5,5 & 74=53,7,7,7\\
% $\vdots $ & $\vdots $ & $\vdots $ & $\vdots $\\
% \hline
% \end{tabular}
% % % % % % % % % % % %
% \begin{center}
% \footnotesize{ Table 4. Tabela procdutoria of the GC.}
% \end{center}
% \\
% \\
% 
% \subsubsection{Second conclusion }
% In conclusion, the resolution about refined GC is found using the equation: $(x ± k)-p = m $ where $m = p + q + [p + p + p ... n]$. Then $x+x=m $.
% Then we take for example a set of any x, which is added y terms related to p.
% That is, $p+p+p$\dots $n. $
% As said before, we know that the two values of $k $ are only applicable to the point that x has the value of $k $. Ie when k passes or exceeds the value of $x $, you can visualize only one value of $k $: The value + k, which is added to y terms of $p $.
% Thus,  $x,y±k=m $ establishing the set $m = x + x + y $. without the value of k aplicable only to $x + x $.
% Thus, all integers m-pairs will be found in a progressive manner and cadence. Therefore, $x + x + y $ is set, finds infinite sets containing an infinite sequence that reveals the entire m-pairs not revealed by its antecedent.
% \newline
% \\
% \begin{center}
% \begin{tabular}{||c|c|c|c||}
% \hline 
% im=(ppp)i=3 & im=(qpp)i=3 & im=(qqp)i=3 & im=(qppp)i=3$\dots $$\to $\\
% \hline 8=2,2,2,2 & 11=2,3,3,3 & 17=2,5,5,5 & 23=2,7,7,7\\
% 11=2,3,3,3 & 14=2,3,3,3,3 & 13=2,2,3,3,3 & 17=2,3,3,3,3,3\\
% 12=3,3,3,3 & 15=3,3,3,3,3 & 15=3,3,3,3,3 & 18=3,3,3,3,3,3\\
% 14=5,3,3,3 & 17=5,3,3,3,3 & 19=5,5,3,3,3 & 20=5,3,3,3,3,3\\
% 16=7,3,3,3 & 19=7,3,3,3,3 & 23=7,7,3,3,3 & 22=7,3,3,3,3,3\\
% 20=11,3,3,3 & 23=11,3,3,3,3 & 31=11,11,3,3,3 & 26=11,3,3,3,3,3\\
% 46=37,3,3,3 & 49=37,3,3,3,3 & 83=37,37,3,3,3 & 52=37,3,3,3,3,3\\
% 50=41,3,3,3 & 53=41,3,3,3,3 & 91=41,41,3,3,3 & 56=41,3,3,3,3,3\\
% 52=43,3,3,3 & 55=43,3,3,3,3 & 95=43,43,3,3,3 & 58=43,3,3,3,3,3\\
% 56=47,3,3,3 & 59=47,3,3,3,3 & 103=47,47,3,3,3 & 62=47,3,3,3,3,3\\
% 62=53,3,3,3 & 65=53,3,3,3,3 & 106=53,53,3,3,3 & 68=53,3,3,3,3,3\\
% $\vdots $ & $\vdots $ & $\vdots $ & $\vdots $\\
% \hline
% \end{tabular}
% \end{center}
% \\
% \newline
% \footnotesize{ Table 5. Table productoria $i=3 $ of the GC. Begin im=odd. } $
% \newline
% $
% \\
% 
% \subsection{Multiprime Theorem's}
% 
% In analysis to GC and Refined GC, we can still state that:
% 
% This theorem has as objective not only explore another possibility in the sequences of primes, but also be a basis of proof of GC. Knowing that its proof also can be used to the same application of the refined GC, referring to the number of terms. As an example, a number 2, added to the set x formed by three terms, we obtain a set with the same properties and four terms.
% The first set $x $ has the value of $(4=2+2) $. This means that the sequence of primes is added to four to present the first odd.  $ 3+(4=2+2)= 7$ 
% The second set $x $ has the value of $(6=3+3) $. This means that the first sequence has added an additional $+2 $. Revealing other odd numbers, related to the second sequence of odd numbers. $ 3+(6=3+3)=9 $ Each set $x $ can be added to infinitely primes. Then, the endless variations, are reflections of the infinite sets that put up the infinite groups. We can define this effect through the phenomenon of the number-rain. We imagine a endless road, where the rain drops, in this case the sets $x $, can reach in length. Each space between the drops is like the difference, or the gaps found in the own infinite sequence of primes. Even existing these differences between the drops there is not a single dry place in the endless road which we imagined. This happens due the fact that the drops are also endless. Of course we wouldn?t evaluate this scale about the road, but it would be more sensible a two-dimensional comparison of the principle, in which case the road surface, and the infinite, the opposite direction to the road surface.
% \\
% 
% 
% \begin{dem}[Lemma 3.1] Every odd number greater than 6 can be composed of any prime number, and two primes of equal value.
% \end{dem}
% 
% \begin{dem}[Theorem 3.1] 
% \newline
% \prod\limits_{ k=1}^{n} \left= p_n+p_{n+1}+p_{n+2}+p{n+3}+$\dots p_{n+n}.$
% \newline $\bigcup p_n+(q+q)_n$
% \newline
% \prod\limits_{ k=1}^{n} \left= (q+q)_n+(q+q)_{n+1}+(q+q)_{n+2}+$\dots $+(q+q)_{n+n}.
% \end{dem}
% \newline
% % % % % % % % %
% \\
% % % % % % % % %
% \begin{tabular}{||c|c|c|c|c||}
% \hline im=(qpp)j=3 & im=(pqq)i=2 & im=(pqq)i=3 & im=(pqq)i=5 & im=(pqq)i=7$\to $\\
% \hline 7=3,2,2 & 7=3,2,2 & 9=3,3,3 & 13=3,5,5 & 17=3,7,7\\
% 9=3,3,3 & 9=5,2,2 & 11=5,3,3 & 15=5,5,5 & 19=5,7,7\\
% 13=3,5,5 & 11=7,2,2 & 13=7,3,3 & 17=7,5,5 & 21=7,7,7\\
% 17=3,7,7 & 15=11,2,2 & 17=11,3,3 & 21=11,5,5 & 25=11,7,7\\
% 25=3,11,11 & 17=13,2,2 & 19=13,3,3 & 23=13,5,5 & 27=13,7,7\\
% 29=3,13,13 & 21=17,2,2 & 23=17,3,3 & 27=17,5,5 & 31=17,7,7\\
% 37=3,17,17 & 23=19,2,2 & 25=19,3,3 & 29=19,5,5 & 33=19,7,7\\
% 41=3,19,19 & 27=23,2,2 & 29=23,3,3 & 33=23,5,5 & 37=23,7,7\\
% 49=3,23,23 & 33=29,2,2 & 35=29,3,3 & 39=29,5,5 & 43=29,7,7\\
% 61=3,29,29 & 35=31,2,2 & 37=31,3,3 & 41=31,5,5 & 45=31,7,7\\
% 65=3,31,31 & 41=37,2,2 & 43=37,3,3 & 47=37,5,5 & 51=37,7,7\\
% 77=3,37,37 & 45=41,2,2 & 47=41,3,3 & 51=41,5,5 & 55=41,7,7\\
% 85=3,41,41 & 47=43,2,2 & 49=43,3,3 & 53=43,5,5 & 57=43,7,7\\
% 89=3,43,43 & 51=47,2,2 & 53=47,3,3 & 57=47,5,5 & 61=47,7,7\\
% 97=3,47,47 & 57=53,2,2 & 59=53,3,3 & 63=53,5,5 & 67=53,7,7\\
% 106=3,53,53 & 63=59,2,2 & 65=59,3,3 & 69=69,5,5 & 83=69,7,7\\
% $\vdots $ & $\vdots $ & $\vdots $ & $\vdots & $\vdots\\
% \hline 
% \end{tabular}
% $
% \newline
% \\
% \mbox{A number of combinations is of the quadratic number.}
% \newline
% \dfrac{Ct}{Cn}=\dfrac{Ct_n}{Cn_n}+\dfrac{Ct_{n+1}}{Cn_{n+1}}+\dfrac{Ct_{n+2}}{Cn_{n+2}}+\dfrac{Ct_{n+3}}{Cn_{n+3}}+$\dots $\to $$\infty $.
% \newline
% $$
% \begin{tabular}{||c|c&c&c&c&c&c|c|c|c|c|c||} 
% \hline $\vdots $ &&&&&&&&&&&  \\ 
%  100 &&&&&&&&&&& $\star $ \\ 
% 97 &&&&&&&&&&&  \\ 
% 95 &&&&&&&&&&&  \\ 
% 93 &&&&&&&&&&&  \\ 
% 91 &&&&&&&&&&&  \\ 
% 89 &&&&&&&&&&&  \\ 
% 85 &&&&&&&&&&&  \\ 
% 83 &&&&&&&&&&&  \\ 
% 81 &&&&&&&&&& $\star $ &  \\ 
% 79 &&&&&&&&&&&  \\ 
% 77 &&&&&&&&&&&  \\ 
% 75 &&&&&&&&&&&  \\ 
% 73 &&&&&&&&&&&  \\ 
% 71 &&&&&&&&&&&  \\ 
% 69 &&&&&&&&&&&  \\ 
% 67 &&&&&&&&&&&  \\ 
% 64 &&&&&&&&& $\star $ &&  \\ 
% 63 &&&&&&&&&&&  \\ 
% 61 &&&&&&&&&&&  \\ 
% 57 &&&&&&&&&&&  \\ 
% 55 &&&&&&&&&&&  \\ 
% 53 &&&&&&&&&&&  \\ 
% 51 &&&&&&&&&&&  \\ 
% 49 &&&&&&&& $\star $ &&&  \\ 
% 47 &&&&&&&&&&&  \\ 
% 45 &&&&&&&&&&&  \\ 
% 43 &&&&&&&&&&&  \\ 
% 41 &&&&&&&&&&&  \\ 
% 39 &&&&&&&&&&&  \\ 
% 36 &&&&&&& $\star $ &&&&  \\ 
% 33 &&&&&&&&&&&  \\ 
% 31 &&&&&&&&&&&  \\ 
% 29 &&&&&&&&&&&  \\ 
% 27 &&&&&&&&&&&  \\ 
% 25 &&&&& $\star $ &&&&&&  \\ 
% 23 &&&&&&&&&&&  \\ 
% 21 &&&&&&&&&&&  \\ 
% 19 &&&&&&&&&&&  \\ 
% 17 &&&&&&&&&&&  \\ 
% 16 &&&& $\star $ &&&&&&& $\dots $ \\ 
% 13 &&&&&&&&&& $ o $ &  \\ 
% 11 &&&&&&&&& $^o $ &&  \\ 
% 9 &&& $\star $ &&&&& $ o $ &&&  \\ 
% 7 &&&&&&& $_o $ &&&&  \\ 
% 5 &&&&&&&&&&&  \\ 
% 3 && $\star $ &&& $ o $ & $^o $ &&&&&  \\ 
% 1 & $\star $ $_o $ & $_o $ & $_o $ & $ o $ &&&&&&&  \\ 
% \hline p^{th} $\to $ & p_n & p_{n+1} & p_{n+2} & p_{n+3} & p_{n+4} & p_{n+5} & p_{n+6} & p_{n+7} & p_{n+8} & p_{n+9} & p_{n+10} $\dots $ \\ 
% \hline 
% \end{tabular} 
% ($Ct=\star $ the quadratic number) and  
% Cn=$o$
% 
% \begin{thebibliography}{99}
% 
% \bibitem{SJ}
% \textrm{Sylvester, J. J.},
% \textit{On the partition of an even number into two primes, Proc. London Math. Soc., s1-4(1). 4-6. 1871.}
% 
% \bibitem{OS}
% \textrm{Tom´as Oliveira e Silva, Siegreid Herzog, and Silvio Pardi},
% \textit{Empirical Veri?cation of the Even Goldbach Conjecture, and Computation of Prime Gaps, up to 4• 1018, Accepted for Publication in Math. Comp. (2013).}
% 
% 
% \bibitem{OR}
% \textrm{Olivier Ramar´e and Yannick Saouter},
% \textit{Short E?ective Intervals Containing Primes, J. Number Theory 98 (2003), no. 1, 10?33.}
% 
% \bibitem{HH}
% \textrm{H.A. Helfgott},
% \textit{Minor arcs for Goldbach?s problem, arXiv preprint arXiv:1205.5252 (2012).}
% 
% \bibitem{DP}
% \textrm{David J. Platt},
% \textit{Computing Degree 1 L-functions Rigorously, Ph.D. thesis, University of Bris- tol, 2011.}
% 
% \bibitem{SW}
% \textrm{S. Wedeniwski},
% \textit{ZetaGrid?Computations connected with the Veri?cation of the Riemann Hy- pothesis, Foundations of Computational Mathematics Conference, Minnesota, USA, 2002.}
% 
% \bibitem{XG}
% \textrm{X. Gourdon},
% \textit{The 1013 First Zeros of the Riemann Zeta Function, and Zeros Computation at Very Large Height, http://numbers.computation.free.fr/Constants/Miscellaneous/ zetazeros1e13-1e24.pdf.}
% 
% \bibitem{EG}
% \textrm{Goldbach to Euler},
% \textit{Goldbach, C., Letter to L. Euler, June 7, 1742.}
% 
% \bibitem{MPM}
% \textrm{Markakis, E., Provatidis, C. and Markakis, N},
% \textit{?An exploration on Goldbach's conjecture,? International Journal of Pure and Applied Mathematics, 84(1), 2013.}
% 
% \bibitem{CC}
% \textrm{CALDWELL, C.},
% \textit{The Prime Glossary: Goldbach's Conjecture. Universidade Tennessee.
% STOPPE, J. Exercises on binary quadratic, Cambridge University Press, 2003.}
% 
% \bibitem{ARK}
% \textrm{Marie Curie, P; Hart, B.},
% \textit{Mathematics Subject $Classification^{2010} $. Primary 11P32 Secondary 11A41; 11Y11. We would like to thank the staff of the Direction des Syst`emes d’Information at Universit´e Paris VI/VII (Pierre et Marie Curie) and Bill Hart at Warwick University for allowing us to use their computer systems for this project. We would also like to thank Andrew Booker for suggesting we exploit Proth primes. Travel was funded in part by the Leverhulme Prize.}
% 
% \newlinw
% For Chapter:
% 
% \bibitem{FP}
% \textrm{FERNANDESKY, P.},
% \textit{Os Teoremas: *¹. (Ed. $ Matemática/Ciências Exatas). Escrytos of the distribution. Lisboa, 2013. p.33-46.}
% 
% 
% 
% \newline
% 
% ^\star $\scriptsize{Mensa Brasil.\emph{ MB:1046}. Jundiaí, São Paulo, Brazil.}\emph{ email}:paulofernandesky@live.com
% 
% 
% \end{thebibliography}

\end{document}
