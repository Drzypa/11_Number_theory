\documentclass[12pt]{article}
\usepackage{pmmeta}
\pmcanonicalname{LeastCommonMultiple}
\pmcreated{2015-05-06 19:07:25}
\pmmodified{2015-05-06 19:07:25}
\pmowner{pahio}{2872}
\pmmodifier{pahio}{2872}
\pmtitle{least common multiple}
\pmrecord{32}{35723}
\pmprivacy{1}
\pmauthor{pahio}{2872}
\pmtype{Definition}
\pmcomment{trigger rebuild}
\pmclassification{msc}{11-00}
\pmsynonym{least common dividend}{LeastCommonMultiple}
\pmsynonym{lcm}{LeastCommonMultiple}
\pmrelated{Divisibility}
\pmrelated{PruferRing}
\pmrelated{SumOfIdeals}
\pmrelated{IdealOfElementsWithFiniteOrder}

% this is the default PlanetMath preamble.  as your knowledge
% of TeX increases, you will probably want to edit this, but
% it should be fine as is for beginners.

% almost certainly you want these
\usepackage{amssymb}
\usepackage{amsmath}
\usepackage{amsfonts}

% used for TeXing text within eps files
%\usepackage{psfrag}
% need this for including graphics (\includegraphics)
%\usepackage{graphicx}
% for neatly defining theorems and propositions
%\usepackage{amsthm}
% making logically defined graphics
%%%\usepackage{xypic}

% there are many more packages, add them here as you need them

\DeclareMathOperator{\lcm}{lcm}
\begin{document}
If $a$ and $b$ are two positive integers, then their {\it least 
common multiple}, denoted by 
$$\mathrm{lcm}\!(a,\,b),$$
is the positive 
integer $f$ satisfying the conditions
\begin{itemize}
  \item $a\mid f$ and $b\mid f$,
  \item if $a\mid c$ and $b\mid c$, then $f\mid c$.
\end{itemize}

\textbf{Note:} \, The definition can be generalized for several 
numbers. \,The positive {lcm of positive integers is 
uniquely determined. (Its negative satisfies the same two 
conditions.)

\subsection*{Properties}

\begin{enumerate} 
  \item If\, $a = \prod_{i=1}^{m}p_i^{\alpha_i}$\, and\, 
  $b = \prod_{i=1}^{m}p_i^{\beta_i}$\, are the prime factor 
  \PMlinkescapetext{presentations} of the positive integers $a$ and $b$ ($\alpha_{i} \geqq 0$, \,$\beta_{i} \geqq 0$ \,$\forall i$), then 
        $$\mathrm{lcm}\!(a,\,b)= 
        \prod_{i=1}^{m}p_i^{\max\{\alpha_i,\,\beta_i\}}.$$ 
This can be generalized for lcm of several numbers.
  \item  Because the greatest common divisor has the expression\, 
  $\gcd(a,\,b) = \prod_{i=1}^{m}p_i^{\min\{\alpha_i,\,\beta_i\}}$, we see that  
  $$\gcd(a,\,b)\cdot \mathrm{lcm}\!(a,\,b) = ab.$$
This formula is sensible only for two integers; it can not be 
generalized for several numbers, i.e., for example,
       $$\gcd(a,\,b,\,c)\cdot \mathrm{lcm}(a,\,b,\,c) \neq abc.$$
  \item The preceding formula may be presented in 
  \PMlinkescapetext{terms} of ideals of $\mathbb{Z}$; we may 
  replace the integers with the corresponding principal ideals.\, 
  The formula acquires the form
      $$((a)+(b))((a)\cap(b)) = (a)(b).$$
  \item The recent formula is valid also for other than principal ideals and even in so general systems as the Pr\"ufer rings; in fact, it could be taken as defining property of these rings: \, Let $R$ be a commutative ring with non-zero unity. \,$R$ is a Pr\"ufer ring iff {\em Jensen's formula}
 $$(\mathfrak{a}+\mathfrak{b})(\mathfrak{a}\cap\mathfrak{b}) = \mathfrak{ab}$$
is true for all ideals $\mathfrak{a}$ and $\mathfrak{b}$ of $R$, with at least one of them having \PMlinkname{non-zero-divisors}{ZeroDivisor}.
\end{enumerate}

\begin{thebibliography}{9}
\bibitem{Larsen & McCarthy} M. Larsen and P. McCarthy: {\em Multiplicative theory of ideals}. Academic Press. New York (1971).
\end{thebibliography}
%%%%%
%%%%%
\end{document}
