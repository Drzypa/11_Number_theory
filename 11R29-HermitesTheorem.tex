\documentclass[12pt]{article}
\usepackage{pmmeta}
\pmcanonicalname{HermitesTheorem}
\pmcreated{2013-03-22 15:05:35}
\pmmodified{2013-03-22 15:05:35}
\pmowner{alozano}{2414}
\pmmodifier{alozano}{2414}
\pmtitle{Hermite's theorem}
\pmrecord{5}{36820}
\pmprivacy{1}
\pmauthor{alozano}{2414}
\pmtype{Corollary}
\pmcomment{trigger rebuild}
\pmclassification{msc}{11R29}
\pmclassification{msc}{11H06}
\pmdefines{unramified outside a set of primes}

\endmetadata

% this is the default PlanetMath preamble.  as your knowledge
% of TeX increases, you will probably want to edit this, but
% it should be fine as is for beginners.

% almost certainly you want these
\usepackage{amssymb}
\usepackage{amsmath}
\usepackage{amsthm}
\usepackage{amsfonts}

% used for TeXing text within eps files
%\usepackage{psfrag}
% need this for including graphics (\includegraphics)
%\usepackage{graphicx}
% for neatly defining theorems and propositions
%\usepackage{amsthm}
% making logically defined graphics
%%%\usepackage{xypic}

% there are many more packages, add them here as you need them

% define commands here

\newtheorem{thm}{Theorem}
\newtheorem*{defn}{Definition}
\newtheorem{prop}{Proposition}
\newtheorem{lemma}{Lemma}
\newtheorem*{cor}{Corollary}

% Some sets
\newcommand{\Nats}{\mathbb{N}}
\newcommand{\Ints}{\mathbb{Z}}
\newcommand{\Reals}{\mathbb{R}}
\newcommand{\Complex}{\mathbb{C}}
\newcommand{\Rats}{\mathbb{Q}}
\begin{document}
The following is a corollary of Minkowski's theorem on ideal classes, which is a corollary of Minkowski's theorem on lattices.

\begin{defn}
Let $S=\{p_1,\ldots,p_r\}$ be a set of rational primes $p_i \in \Ints$. We say that a number field $K$ is {\bf unramified outside $S$} if any prime not in $S$ is unramified in $K$. In other words, if $p$ is ramified in $K$, then $p\in S$. In other words, the only primes that divide the discriminant of $K$ are elements of $S$.
\end{defn}

\begin{cor}[Hermite's Theorem]
Let $S=\{p_1,\ldots,p_r\}$ be a set of rational primes $p_i \in \Ints$ and let $N\in \Nats$ be arbitrary. There is only a finite number of fields $K$ which are unramified outside $S$ and bounded degree $[K:\Rats]\leq N$.  
\end{cor}
%%%%%
%%%%%
\end{document}
