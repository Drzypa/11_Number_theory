\documentclass[12pt]{article}
\usepackage{pmmeta}
\pmcanonicalname{DivisibilityByPrimeNumber}
\pmcreated{2013-03-22 14:48:18}
\pmmodified{2013-03-22 14:48:18}
\pmowner{pahio}{2872}
\pmmodifier{pahio}{2872}
\pmtitle{divisibility by prime number}
\pmrecord{18}{36460}
\pmprivacy{1}
\pmauthor{pahio}{2872}
\pmtype{Theorem}
\pmcomment{trigger rebuild}
\pmclassification{msc}{11A05}
\pmsynonym{divisibility by prime}{DivisibilityByPrimeNumber}
\pmrelated{PrimeElement}
\pmrelated{DivisibilityInRings}
\pmrelated{EulerPhiAtAProduct}
\pmrelated{RepresentantsOfQuadraticResidues}

% this is the default PlanetMath preamble.  as your knowledge
% of TeX increases, you will probably want to edit this, but
% it should be fine as is for beginners.

% almost certainly you want these
\usepackage{amssymb}
\usepackage{amsmath}
\usepackage{amsfonts}

% used for TeXing text within eps files
%\usepackage{psfrag}
% need this for including graphics (\includegraphics)
%\usepackage{graphicx}
% for neatly defining theorems and propositions
 \usepackage{amsthm}
% making logically defined graphics
%%%\usepackage{xypic}

% there are many more packages, add them here as you need them

% define commands here
\theoremstyle{definition}
\newtheorem*{thmplain}{Theorem}
\begin{document}
\begin{thmplain}
 \,Let $a$ and $b$ be integers and $p$ any prime number. \,Then we have:
\begin{align}
       p \mid ab \quad \Leftrightarrow \quad p \mid a\; \lor\; p \mid b
\end{align}
\end{thmplain}
{\em Proof.}  Suppose that\, $p \mid ab$.\,  Then either\; $p \mid a$\; or\; $p \nmid a$.\, In the latter case we have\, $\gcd(a,\,p) = 1$,\, and therefore the corollary of B\'ezout's lemma gives the result\, $p \mid b$.\,  Conversely, if\; $p \mid a$\; or\; $p \mid b$,\, then for example\, $a = mp$\, for some integer $m$; this implies that\, $ab = mb\cdot p$,\, i.e.\, $p \mid ab$.\\

\textbf{Remark 1.}  The theorem means, that if a product is divisible by a prime number, then at least one of the factor is divisibe by the prime number.  Also conversely.\\

\textbf{Remark 2.}  The condition (1) is expressed in \PMlinkescapetext{terms} of principal ideals as
\begin{align}
      (ab)\subseteq (p) \quad \Leftrightarrow \quad 
          (a)\subseteq (p)\, \lor\, (b)\subseteq (p).
\end{align}
Here, $(p)$ is a prime ideal of $\mathbb{Z}$.
%%%%%
%%%%%
\end{document}
