\documentclass[12pt]{article}
\usepackage{pmmeta}
\pmcanonicalname{CompletelyMultiplicativeFunctionsWhoseConvolutionInversesAreCompletelyMultiplicative}
\pmcreated{2013-03-22 16:55:12}
\pmmodified{2013-03-22 16:55:12}
\pmowner{Wkbj79}{1863}
\pmmodifier{Wkbj79}{1863}
\pmtitle{completely multiplicative functions whose convolution inverses are completely multiplicative}
\pmrecord{4}{39182}
\pmprivacy{1}
\pmauthor{Wkbj79}{1863}
\pmtype{Corollary}
\pmcomment{trigger rebuild}
\pmclassification{msc}{11A25}

\usepackage{amssymb}
\usepackage{amsmath}
\usepackage{amsfonts}

\usepackage{psfrag}
\usepackage{graphicx}
\usepackage{amsthm}
%%\usepackage{xypic}

\newtheorem{cor*}{Corollary}

\begin{document}
\begin{cor*}
The only completely multiplicative function whose convolution inverse is also completely multiplicative is $\varepsilon$, the convolution identity function.
\end{cor*}

\begin{proof}
Let $f$ be a completely multiplicative function whose convolution inverse is completely multiplicative.  By \PMlinkname{this entry}{FormulaForTheConvolutionInverseOfACompletelyMultiplicativeFunction}, $f\mu$ is the convolution inverse of $f$, where $\mu$ denotes the M\"{o}bius function.  Thus, $f\mu$ is completely multiplicative.

Let $p$ be any prime.  Then

\begin{center}
$\begin{array}{rl}
(f(p))^2 & =(f(p))^2(-1)^2 \\
\\
& =(f(p))^2(\mu(p))^2 \\
\\
& =(f(p)\mu(p))^2 \\
\\
& =f(p^2)\mu(p^2) \\
\\
& =f(p^2) \cdot 0 \\
\\
& =0. \end{array}$
\end{center}

Thus, $f(p)=0$ for every prime $p$.  Since $f$ is completely multiplicative,

$$f(n)=\begin{cases}
1 & \text{if } n=1 \\
0 & \text{if } n\neq 1. \end{cases}$$

Hence, $f=\varepsilon$.
\end{proof}
%%%%%
%%%%%
\end{document}
