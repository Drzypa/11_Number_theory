\documentclass[12pt]{article}
\usepackage{pmmeta}
\pmcanonicalname{EisensteinPrime}
\pmcreated{2013-03-22 16:10:10}
\pmmodified{2013-03-22 16:10:10}
\pmowner{PrimeFan}{13766}
\pmmodifier{PrimeFan}{13766}
\pmtitle{Eisenstein prime}
\pmrecord{6}{38253}
\pmprivacy{1}
\pmauthor{PrimeFan}{13766}
\pmtype{Definition}
\pmcomment{trigger rebuild}
\pmclassification{msc}{11R04}
\pmrelated{EisensteinIntegers}

% this is the default PlanetMath preamble.  as your knowledge
% of TeX increases, you will probably want to edit this, but
% it should be fine as is for beginners.

% almost certainly you want these
\usepackage{amssymb}
\usepackage{amsmath}
\usepackage{amsfonts}

% used for TeXing text within eps files
%\usepackage{psfrag}
% need this for including graphics (\includegraphics)
%\usepackage{graphicx}
% for neatly defining theorems and propositions
%\usepackage{amsthm}
% making logically defined graphics
%%%\usepackage{xypic}

% there are many more packages, add them here as you need them

% define commands here

\begin{document}
Given the complex cubic root of unity $\omega = e^{{2i\pi}\over{3}}$, an Eisenstein integer $a\omega + b$ (where $a$ and $b$ are natural integers) is said to be an \emph{Eisenstein prime} if its only divisors are 1, $\omega$, $1 + \omega$ and itself.

Eisenstein primes of the form $0\omega + b$ are ordinary natural primes $p \equiv 2 \mod 3$. Therefore no Mersenne prime is also an Eisenstein prime.
%%%%%
%%%%%
\end{document}
