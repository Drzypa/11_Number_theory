\documentclass[12pt]{article}
\usepackage{pmmeta}
\pmcanonicalname{DivisionAlgorithmForIntegers}
\pmcreated{2013-03-22 11:59:43}
\pmmodified{2013-03-22 11:59:43}
\pmowner{vampyr}{22}
\pmmodifier{vampyr}{22}
\pmtitle{division algorithm for integers}
\pmrecord{7}{30919}
\pmprivacy{1}
\pmauthor{vampyr}{22}
\pmtype{Theorem}
\pmcomment{trigger rebuild}
\pmclassification{msc}{11A51}
\pmsynonym{division algorithm}{DivisionAlgorithmForIntegers}
\pmrelated{ExistenceAndUniquenessOfTheGcdOfTwoIntegers}

\endmetadata

\usepackage{amssymb}
\usepackage{amsmath}
\usepackage{amsfonts}
\usepackage{graphicx}
%%%\usepackage{xypic}
\begin{document}
Given any two integers $a,b$ where $b > 0$, there exists a unique pair of integers $q,r$ such that $a = qb + r$ and $0 \leq r < b$.  $q$ is called the \emph{quotient} of $a$ and $b$, and $r$ is the \emph{remainder}.

The division algorithm is not an algorithm at all but rather a theorem.  Its name probably derives from the fact that it was first proved by showing that an algorithm to calculate the quotient of two integers yields this result.

There are similar forms of the division algorithm that apply to other rings (for example, polynomials).
%%%%%
%%%%%
%%%%%
\end{document}
