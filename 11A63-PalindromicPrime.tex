\documentclass[12pt]{article}
\usepackage{pmmeta}
\pmcanonicalname{PalindromicPrime}
\pmcreated{2013-03-22 15:55:30}
\pmmodified{2013-03-22 15:55:30}
\pmowner{PrimeFan}{13766}
\pmmodifier{PrimeFan}{13766}
\pmtitle{palindromic prime}
\pmrecord{5}{37932}
\pmprivacy{1}
\pmauthor{PrimeFan}{13766}
\pmtype{Definition}
\pmcomment{trigger rebuild}
\pmclassification{msc}{11A63}

% this is the default PlanetMath preamble.  as your knowledge
% of TeX increases, you will probably want to edit this, but
% it should be fine as is for beginners.

% almost certainly you want these
\usepackage{amssymb}
\usepackage{amsmath}
\usepackage{amsfonts}

% used for TeXing text within eps files
%\usepackage{psfrag}
% need this for including graphics (\includegraphics)
%\usepackage{graphicx}
% for neatly defining theorems and propositions
%\usepackage{amsthm}
% making logically defined graphics
%%%\usepackage{xypic}

% there are many more packages, add them here as you need them

% define commands here

\begin{document}
A prime number $p$ that in a given base $b$ is also a palindromic number. Although there are infinitely many palindromic numbers in any given base, it is not known if the same is true of palindromic primes.

If $b + 1$ is prime, then it is the only palindromic prime in base $b$ to have an even number of digits; all other palindromes with an even number of digits will be multiples of $b + 1$.
%%%%%
%%%%%
\end{document}
