\documentclass[12pt]{article}
\usepackage{pmmeta}
\pmcanonicalname{LucassTheorem}
\pmcreated{2013-03-22 13:17:31}
\pmmodified{2013-03-22 13:17:31}
\pmowner{mathcam}{2727}
\pmmodifier{mathcam}{2727}
\pmtitle{Lucas's theorem}
\pmrecord{6}{33779}
\pmprivacy{1}
\pmauthor{mathcam}{2727}
\pmtype{Theorem}
\pmcomment{trigger rebuild}
\pmclassification{msc}{11B65}
%\pmkeywords{binomial coefficient}
%\pmkeywords{congruence}

% this is the default PlanetMath preamble.  as your knowledge
% of TeX increases, you will probably want to edit this, but
% it should be fine as is for beginners.

% almost certainly you want these
\usepackage{amssymb}
\usepackage{amsmath}
\usepackage{amsfonts}

% used for TeXing text within eps files
%\usepackage{psfrag}
% need this for including graphics (\includegraphics)
%\usepackage{graphicx}
% for neatly defining theorems and propositions
%\usepackage{amsthm}
% making logically defined graphics
%%%\usepackage{xypic} 

% there are many more packages, add them here as you need them

% define commands here
\begin{document}
Let $m,n \in \mathbb{N}-\{0\}$ be two natural numbers . If $p$ is a prime number and :
$$ m = a_k p^k + a_{k-1} p^{k-1} + \cdots + a_1 p + a_0 , n = b_k p^k + b_{k-1} p^{k-1} + \cdots + b_1 p + b_0$$ are the base-p expansions of $m$ and $n$ , then the following congruence is true :
$$ {m \choose n} \equiv {a_0 \choose b_0} {a_1 \choose b_1} \cdots {a_k \choose b_k} (\verb|mod p|)$$
Note : the binomial coefficient is defined in the usual way , namely :
$$ {x \choose y} = \frac{x!}{y!(x-y)!} $$ if $x \geq y$ and $0$ otherwise (of course , x and y are natural numbers).
%%%%%
%%%%%
\end{document}
