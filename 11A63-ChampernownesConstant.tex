\documentclass[12pt]{article}
\usepackage{pmmeta}
\pmcanonicalname{ChampernownesConstant}
\pmcreated{2013-03-22 17:04:09}
\pmmodified{2013-03-22 17:04:09}
\pmowner{PrimeFan}{13766}
\pmmodifier{PrimeFan}{13766}
\pmtitle{Champernowne's constant}
\pmrecord{4}{39361}
\pmprivacy{1}
\pmauthor{PrimeFan}{13766}
\pmtype{Definition}
\pmcomment{trigger rebuild}
\pmclassification{msc}{11A63}
\pmsynonym{Champernowne constant}{ChampernownesConstant}

\endmetadata

% this is the default PlanetMath preamble.  as your knowledge
% of TeX increases, you will probably want to edit this, but
% it should be fine as is for beginners.

% almost certainly you want these
\usepackage{amssymb}
\usepackage{amsmath}
\usepackage{amsfonts}

% used for TeXing text within eps files
%\usepackage{psfrag}
% need this for including graphics (\includegraphics)
%\usepackage{graphicx}
% for neatly defining theorems and propositions
%\usepackage{amsthm}
% making logically defined graphics
%%%\usepackage{xypic}

% there are many more packages, add them here as you need them

% define commands here

\begin{document}
For a given base $b$, {\em Champernowne's constant} $C_b$ is the result of concatenating the base $b$ digits of the positive integers in order after 0 and a decimal point, that is, $$\sum_{i = 1}^\infty \frac{i}{b^{\sum_{j = 1}^i k}}$$ (where $k$ is the number of digits of $j$ in base $b$).

Kurt Mahler proved that $C_{10}$ (approximately 0.123456789101112131415161718192021...) is a transcendental number. Champernowne had earlier proved that $C_{10}$ is a normal number.
%%%%%
%%%%%
\end{document}
