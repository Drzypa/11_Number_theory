\documentclass[12pt]{article}
\usepackage{pmmeta}
\pmcanonicalname{ContraharmonicMeansAndPythagoreanHypotenuses}
\pmcreated{2013-11-03 21:13:57}
\pmmodified{2013-11-03 21:13:57}
\pmowner{pahio}{2872}
\pmmodifier{pahio}{2872}
\pmtitle{contraharmonic means and Pythagorean hypotenuses}
\pmrecord{24}{41259}
\pmprivacy{1}
\pmauthor{pahio}{2872}
\pmtype{Theorem}
\pmcomment{trigger rebuild}
\pmclassification{msc}{11D09}
\pmclassification{msc}{11D45}
\pmclassification{msc}{11Z05}
\pmclassification{msc}{11A05}
\pmsynonym{contraharmonic integers}{ContraharmonicMeansAndPythagoreanHypotenuses}
\pmsynonym{Pythagorean hypotenuses are contraharmonic means}{ContraharmonicMeansAndPythagoreanHypotenuses}
\pmrelated{FirstPrimitivePythagoreanTriplets}
\pmrelated{ProofOfPythagoreanTriplet2}
\pmrelated{SquareOfSum}
\pmrelated{PythagoreanTriple}
\pmrelated{DerivationOfPythagoreanTriples}
\pmrelated{LinearFormulasForPythagoreanTriples}

\endmetadata

% this is the default PlanetMath preamble.  as your knowledge
% of TeX increases, you will probably want to edit this, but
% it should be fine as is for beginners.

% almost certainly you want these
\usepackage{amssymb}
\usepackage{amsmath}
\usepackage{amsfonts}

% used for TeXing text within eps files
%\usepackage{psfrag}
% need this for including graphics (\includegraphics)
%\usepackage{graphicx}
% for neatly defining theorems and propositions
 \usepackage{amsthm}
% making logically defined graphics
%%%\usepackage{xypic}

% there are many more packages, add them here as you need them

% define commands here

\theoremstyle{definition}
\newtheorem*{thmplain}{Theorem}

\begin{document}
One can see that all values of $c$ in the table of the \PMlinkname{parent entry}{IntegerContraharmonicMeans} are hypotenuses in a right triangle with integer \PMlinkname{sides}{Triangle}.\, E.g., 41 is the contraharmonic mean of 5 and 45;\; $9^2\!+\!40^2 \;=\; 41^2$.\\

\textbf{Theorem.}\; Any integer contraharmonic mean of two different positive integers is the hypotenuse of a Pythagorean triple.\, Conversely, any hypotenuse of a Pythagorean triple is contraharmonic mean of two different positive integers.

{\em Proof.}\, $1^\circ.$\, Let the integer $c$ be the contraharmonic mean 
                 $$c \;=\; \frac{u^2\!+\!v^2}{u\!+\!v}$$
of the positive integers $u$ and $v$ with\, $u > v$.\, Then\; $u\!+\!v \,\mid\, u^2\!+\!v^2 \,=\, (u\!+\!v)^2-2uv$,\,
whence 
$$u\!+\!v \,\mid\, 2uv,$$
and we have the positive integers
$$a \;=:\; u\!-\!v \;=\; \frac{u^2\!-\!v^2}{u\!+\!v}, \quad b \;=:\; \frac{2uv}{u\!+\!v}$$
satisfying
$$a^2\!+\!b^2 \;=\; \frac{(u^2\!-\!v^2)^2\!+\!(2uv)^2}{(u\!+\!v)^2} 
= \frac{u^4\!-\!2u^2v^2+v^4\!+\!4u^2v^2}{(u\!+\!v)^2}  
= \frac{u^4\!+\!2u^2v^2\!+\!v^4}{(u\!+\!v)^2} = \frac{(u^2\!+\!v^2)^2}{(u\!+\!v)^2} \;=\; c^2.\\$$

$2^\circ.$\, Suppose that $c$ is the hypotenuse of the Pythagorean triple \,$(a,\,b,\,c)$,\, whence\, 
$c^2 = a^2\!+\!b^2$.\, Let us consider the rational numbers
\begin{align}
u =: \frac{c\!+\!b\!+\!a}{2}, \quad v =: \frac{c\!+\!b\!-\!a}{2}.
\end{align}
If the triple is \PMlinkname{primitive}{PythagoreanTriple}, then two of the integers $a,\,b,\,c$ are odd and one of them is even; if not, then similarly or all of $a,\,b,\,c$ are even.\, Therefore, $c\!+\!b\!\pm\!a$ are always even and accordingly $u$ and $v$ positive integers.\, We see also that\, $u\!+\!v = c\!+\!b$.\, Now we obtain
\begin{align*}
u^2\!+\!v^2\; & =\;  \frac{c^2\!+\!b^2\!+\!a^2\!+\!2ab\!+\!2bc\!+\!2ca\!+\!c^2\!+\!b^2\!+\!a^2\!-\!2ab\!+\!2bc\!-\!2ca}{4}\\
              & =\; \frac{2c^2\!+\!2(a^2\!+\!b^2)\!+\!4bc}{4} = \frac{4c^2\!+\!4bc}{4} = c(c\!+\!b)\\
              & =\; c(u\!+\!v).
\end{align*}
Thus, $c$ is the contraharmonic mean $\displaystyle\frac{u^2\!+\!v^2}{u\!+\!v}$ of the different integers $u$ and $v$. (N.B.:\, When the values of $a$ and $b$ in (1) are changed, another value of $v$ is obtained.\, Cf. the Proposition 4 in the 
\PMlinkname{parent entry}{IntegerContraharmonicMeans}.)

\begin{thebibliography}{8}
\bibitem{K}{\sc J. Pahikkala}: ``On contraharmonic mean and Pythagorean triples''.\, -- \emph{Elemente der Mathematik} \textbf{65}:2 (2010).
\end{thebibliography}



%%%%%
%%%%%
\end{document}
