\documentclass[12pt]{article}
\usepackage{pmmeta}
\pmcanonicalname{NewtonGirardFormulaForSymmetricPolynomials}
\pmcreated{2013-03-22 15:32:40}
\pmmodified{2013-03-22 15:32:40}
\pmowner{kschalm}{9486}
\pmmodifier{kschalm}{9486}
\pmtitle{Newton-Girard formula for symmetric polynomials}
\pmrecord{5}{37441}
\pmprivacy{1}
\pmauthor{kschalm}{9486}
\pmtype{Theorem}
\pmcomment{trigger rebuild}
\pmclassification{msc}{11C08}
\pmrelated{WaringsFormula}
\pmrelated{ElementarySymmetricPolynomialInTermsOfPowerSums}

% this is the default PlanetMath preamble.  as your knowledge
% of TeX increases, you will probably want to edit this, but
% it should be fine as is for beginners.

% almost certainly you want these
\usepackage{amssymb}
\usepackage{amsmath}
\usepackage{amsfonts}

% used for TeXing text within eps files
%\usepackage{psfrag}
% need this for including graphics (\includegraphics)
%\usepackage{graphicx}
% for neatly defining theorems and propositions
%\usepackage{amsthm}
% making logically defined graphics
%%%\usepackage{xypic}

% there are many more packages, add them here as you need them

% define commands here
\begin{document}
Let $E_k$ be the elementary symmetric polynomials in $n$ variables and $S_k$ be defined by 
\[ S_k(x_1, ... , x_n) = \sum_{i=1}^n{x_i^k}. \]

Then the $S_k$ and $E_k$ are related as follows:
\begin{align*}
S_1 &= E_1 \\
S_2 &= S_1 E_1 - 2E_2 \\
S_3 &= S_2 E_1 - S_1 E_2 + 3E_3\\
 &\vdots \\
S_k &= -\left(\sum_{j=1}^{k-1} {(-1)^j S_{k-j} E_j} \right) - (-1)^k k E_k 
\end{align*}

%or even more succinctly as

%\[ \sum_{j=0}^{k} {(-1)^j S_{k-j} E_j} = 0, \text{     }  1 \leq n \leq k \]

By applying these formulas recursively, $S_k$ can be expressed solely in terms of the $E_k$, which is often desirable. For example, since $S_1 = E_1$, $S_2 = E_1^2 - 2E_2$, and then $S_3 = (E_1^2-2E_2)E_1 - E_1 E_2 + 3E_3 = E_1^3 - 3E_1 E_2 + 3E_3$, and so on.

Note that $E_0 = 1$ and  $E_k=0$ for $k > n$.
%%%%%
%%%%%
\end{document}
