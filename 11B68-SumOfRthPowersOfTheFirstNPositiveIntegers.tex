\documentclass[12pt]{article}
\usepackage{pmmeta}
\pmcanonicalname{SumOfRthPowersOfTheFirstNPositiveIntegers}
\pmcreated{2013-03-22 14:14:38}
\pmmodified{2013-03-22 14:14:38}
\pmowner{mathcam}{2727}
\pmmodifier{mathcam}{2727}
\pmtitle{sum of $r$th powers of the first $n$ positive integers}
\pmrecord{14}{35689}
\pmprivacy{1}
\pmauthor{mathcam}{2727}
\pmtype{Theorem}
\pmcomment{trigger rebuild}
\pmclassification{msc}{11B68}
\pmclassification{msc}{05A15}
\pmrelated{BernoulliNumber}
\pmrelated{FormalPowerSeries}
\pmrelated{ArithmeticProgression}
\pmrelated{SumOfPowers}

% this is the default PlanetMath preamble.  as your knowledge
% of TeX increases, you will probably want to edit this, but
% it should be fine as is for beginners.

% almost certainly you want these
\usepackage{amssymb}
\usepackage{amsmath}
\usepackage{amsfonts}

% used for TeXing text within eps files
%\usepackage{psfrag}
% need this for including graphics (\includegraphics)
%\usepackage{graphicx}
% for neatly defining theorems and propositions
\usepackage{amsthm}
% making logically defined graphics
%%%\usepackage{xypic}

% there are many more packages, add them here as you need them

% define commands here

\newtheorem{theorem}{Theorem}
\newtheorem{defn}{Definition}
\newtheorem{prop}{Proposition}
\newtheorem{lemma}{Lemma}
\newtheorem{cor}{Corollary}
\begin{document}
\PMlinkescapeword{terms}
A very classic problem is this: what is the sum of the numbers from $1$ to $100$?  The answer is $5050$, and there are a number of ways to see it.  Before stating them, we generalize the problem somewhat: What is $\sum_{k=1}^n k$?

\begin{theorem}
\[
\sum_{k=1}^n k = \frac{n(n+1)}{2}
\]
\end{theorem}
\begin{proof}
Write the sum twice:
\[
\begin{array}{cccccccccc}
  & 1   & + & 2     & + & 3     & + & \cdots & + & n \\
+ & n   & + & (n-1) & + & (n-2) & + & \cdots & + & 1 \\
= & n+1 & + & n+1   & + & n+1   & + & \cdots & + & n+1 \\
\end{array}
\]
which is exactly $n(n+1)$, so 
\[
\sum_{k=1}^n k = \frac{n(n+1)}{2}.\qedhere
\]
\end{proof}

Having seen this, it is natural to generalize this question further, and ask: What is $\sum_{k=1}^n k^r$, for any integer $r>0$?

This question, being so general, does not have such a tidy answer, but it can be solved.

\begin{theorem}
Let $B_m$ denote the $m$th Bernoulli number, and let $r$ be a positive integer.  Then
\[
\sum_{k=0}^n k^r = \sum_{l=1}^{r+1} \binom{r+1}{l}\frac{B_{r+1-l}}{r+1}(n+1)^l.
\]
\end{theorem}
Observe that this formula is a polynomial in $n$ of \PMlinkname{degree}{OrderAndDegreeOfPolynomial} $r+1$; for a given $r$, we can look up all the Bernoulli numbers we need and write down the polynomial explicitly. For example:
\[
1^2 + 2^2 + 3^2 + \cdots + n^2 = \frac{n(n+1)(2n+1)}{6},
\]
\[
1^3 + 2^3 + 3^3 + \cdots + n^3 = \frac{(n(n+1))^2}{4},
\]
and
\[
1^4 + 2^4 + 3^4 + \cdots + n^4 = \frac{n(2n+1)(n+1)(3n^2+3n-1)}{30}.
\]

We will prove the result by a generating function argument.
\begin{proof}
For no obvious reason, let
\[
f_n(x) := \frac{x(e^{(n+1)x} - 1)}{(e^x-1)}.
\]

On the one hand, we have
\[
f_n(x) = x\frac{1-(e^x)^{n+1}}{1-e^x} = x\sum_{k=0}^{n} e^{kx}, 
\]
and so the coefficient of $x^{r+1}$ is $\sum_{k=0}^{n} k^r/r!$, more or less the sum we want.

On the other hand, using the definition of the Bernoulli numbers, 
\begin{align*}
f_n(x) & = (e^{(n+1)x}-1) \sum_{j=0}^{\infty} B_j \frac{x^j}{j!}\\
       & = \sum_{l=1}^{\infty} \sum_{j=0}^{\infty} B_j (n+1)^l \frac{x^{l+j}}{l!j!}
\end{align*}
so the coefficient of $x^{r+1}$ is 
\[
\sum_{l=1}^{r+1} \frac{(n+1)^l}{l!}\frac{B_{r+1-l}}{(r+1-l)!}.
\]

Combining these two results, we get
\[
\sum_{k=0}^n k^r = \sum_{l=1}^{r+1} \binom{r+1}{l}\frac{B_{r+1-l}}{r+1}(n+1)^l.\qedhere
\]
\end{proof}

Observe that we need not use the Bernoulli numbers directly; any way we can extract the Taylor coefficients of $f_n$ will give us the same results.  In practical terms, we can hand $f_n$ to a computer algebra system and it will print out the needed formula.

This proof is given as Exercise 2.23 in \PMlinkexternal{generatingfunctionology}{http://www.math.upenn.edu/~wilf/DownldGF.html}.
%%%%%
%%%%%
\end{document}
