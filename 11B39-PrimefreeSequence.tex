\documentclass[12pt]{article}
\usepackage{pmmeta}
\pmcanonicalname{PrimefreeSequence}
\pmcreated{2013-03-22 15:54:49}
\pmmodified{2013-03-22 15:54:49}
\pmowner{CompositeFan}{12809}
\pmmodifier{CompositeFan}{12809}
\pmtitle{primefree sequence}
\pmrecord{7}{37917}
\pmprivacy{1}
\pmauthor{CompositeFan}{12809}
\pmtype{Definition}
\pmcomment{trigger rebuild}
\pmclassification{msc}{11B39}

% this is the default PlanetMath preamble.  as your knowledge
% of TeX increases, you will probably want to edit this, but
% it should be fine as is for beginners.

% almost certainly you want these
\usepackage{amssymb}
\usepackage{amsmath}
\usepackage{amsfonts}

% used for TeXing text within eps files
%\usepackage{psfrag}
% need this for including graphics (\includegraphics)
%\usepackage{graphicx}
% for neatly defining theorems and propositions
%\usepackage{amsthm}
% making logically defined graphics
%%%\usepackage{xypic}

% there are many more packages, add them here as you need them

% define commands here

\begin{document}
\PMlinkescapeword{term}
\PMlinkescapeword{potential}
\PMlinkescapeword{order}

Consider the sequence defined by $a_1 = 20615674205555510$, $a_2 = 3794765361567513$ and $a_n = a_{n - 1} + a_{n - 2}$ for all $n > 2$. As it has been verified not to contain any primes, it is called a {\em primefree sequence}. The initial terms must be coprime, or else the lack of primes is a trivial consequence of the initial terms sharing a divisor other than 1.

Any Fibonacci-like sequence will naturally exhibit some patterns in the factorizations of its terms in relation to their indices. The initial terms are chosen so that these patterns cover any possible value of $n$. So, for our example sequence, discovered by Wilf in 1990, $2|a_{3x + 1}$, $3|a_{4x + 2}$, $5|a_{5x + 1}$, $7|a_{8x}$, etc. for a finite number of potential prime factors (and $x \ge 0$ in each case).

Order is very important: switching the initial terms can cause primes to arise in the sequence. Switching the initial terms in our example causes $a_{138}$ and a few others afterwards to be prime.

The example sequence is listed in A083216 of the OEIS.

\begin{thebibliography}{3}
\bibitem{ph}P. Hoffman. {\it The Man Who Loved Only Numbers: The Story of Paul Erdos and the Search for Mathematical Truth}. New York: Hyperion, 1998.
\bibitem{hn}H. Nicol. A Fibonacci-like sequence of composite numbers. {\it Electronic J. of Combinatorics} 6, 1999.
\bibitem{hw}H. S. Wilf. Letters to the Editor. {\it Math. Mag.} 63, 284, 1990.
\end{thebibliography}
%%%%%
%%%%%
\end{document}
