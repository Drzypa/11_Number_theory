\documentclass[12pt]{article}
\usepackage{pmmeta}
\pmcanonicalname{InvertibleIdealsInSemilocalRings}
\pmcreated{2013-03-22 18:36:17}
\pmmodified{2013-03-22 18:36:17}
\pmowner{gel}{22282}
\pmmodifier{gel}{22282}
\pmtitle{invertible ideals in semi-local rings}
\pmrecord{6}{41336}
\pmprivacy{1}
\pmauthor{gel}{22282}
\pmtype{Theorem}
\pmcomment{trigger rebuild}
\pmclassification{msc}{11R04}
\pmclassification{msc}{13F05}
%\pmkeywords{semi-local ring}
%\pmkeywords{invertible ideal}
\pmrelated{PruferDomain}

% this is the default PlanetMath preamble.  as your knowledge
% of TeX increases, you will probably want to edit this, but
% it should be fine as is for beginners.

% almost certainly you want these
\usepackage{amssymb}
\usepackage{amsmath}
\usepackage{amsfonts}

% used for TeXing text within eps files
%\usepackage{psfrag}
% need this for including graphics (\includegraphics)
%\usepackage{graphicx}
% for neatly defining theorems and propositions
\usepackage{amsthm}
% making logically defined graphics
%%%\usepackage{xypic}

% there are many more packages, add them here as you need them

% define commands here
\newtheorem*{theorem*}{Theorem}
\newtheorem*{lemma*}{Lemma}
\newtheorem*{corollary*}{Corollary}
\newtheorem{theorem}{Theorem}
\newtheorem{lemma}{Lemma}
\newtheorem{corollary}{Corollary}


\begin{document}
\PMlinkescapeword{expand}
\begin{theorem*}
Let $R$ be a commutative ring in which there are only finitely many maximal ideals. Then, a fractional ideal over $R$ is invertible if and only if it is principal and generated by a regular element.
\end{theorem*}

In particular, a  \PMlinkname{semi-local}{SemiLocalRing} Dedekind domain is a principal ideal domain and every finitely generated ideal in a semi-local Pr\"ufer domain is principal.

\begin{proof}
First, if $a$ is regular then $(a)$ is invertible, with inverse $(a^{-1})$, so only the converse needs to be shown.

Suppose that $\mathfrak{a}$ is invertible, and $\mathfrak{a}\mathfrak{b}=R$.
Then let the maximal ideals of $R$ be $\mathfrak{m}_1,\ldots,\mathfrak{m}_n$. As $\mathfrak{a}\mathfrak{b}\not\subseteq\mathfrak{m}_k$, there exist $a_k\in\mathfrak{a},b_k\in\mathfrak{b}$ such that $a_kb_k\in R\setminus\mathfrak{m}_k$.

By maximality, $\mathfrak{m}_j\not\subseteq\mathfrak{m}_k$ whenever $j\not=k$, so we may choose $\lambda_{jk}\in\mathfrak{m}_j\setminus\mathfrak{m}_k$.
Setting $\lambda_k=\prod_{j\not=k}\lambda_{jk}$ gives $\lambda_k\in\mathfrak{m}_j$ for all $j\not=k$ and, as $\mathfrak{m}_k$ is \PMlinkname{prime}{PrimeIdeal}, $\lambda_k\not\in\mathfrak{m}_k$.
Then, writing
\begin{equation*}
a = \lambda_1a_1+\cdots+\lambda_na_n\in\mathfrak{a},\ b=\lambda_1b_1+\cdots+\lambda_n b_n\in\mathfrak{b}
\end{equation*}
we can expand the product to get
\begin{equation}\label{eq:1}
ab = \sum_{i,j}\lambda_i\lambda_ja_ib_j.
\end{equation}
However, $a_ib_j\in\mathfrak{a}\mathfrak{b}\subseteq R$ so $\lambda_i\lambda_ja_ib_j$ is in $\mathfrak{m}_k$ whenever either $i$ or $j$ is not equal to $k$. On the other hand, $\lambda_k\lambda_ka_kb_k\not\in\mathfrak{m}_k$ and, consequently, there is exactly one term on the right hand side of (\ref{eq:1}) which is not in $\mathfrak{m}_k$, so $ab\not\in\mathfrak{m}_k$.

We have shown that $ab$ is not in any maximal ideal of $R$, and must therefore be a unit. So a is regular and,
\begin{equation*}
(a)\subseteq \mathfrak{a}=ab\mathfrak{a}\subseteq a\mathfrak{b}\mathfrak{a}=aR=(a)
\end{equation*}
as required.
\end{proof}

%%%%%
%%%%%
\end{document}
