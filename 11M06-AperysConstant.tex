\documentclass[12pt]{article}
\usepackage{pmmeta}
\pmcanonicalname{AperysConstant}
\pmcreated{2013-03-22 13:27:19}
\pmmodified{2013-03-22 13:27:19}
\pmowner{bbukh}{348}
\pmmodifier{bbukh}{348}
\pmtitle{Ap\'ery's constant}
\pmrecord{8}{34021}
\pmprivacy{1}
\pmauthor{bbukh}{348}
\pmtype{Definition}
\pmcomment{trigger rebuild}
\pmclassification{msc}{11M06}
\pmclassification{msc}{11J81}

\endmetadata

\usepackage{amssymb}
\usepackage{amsmath}
\usepackage{amsfonts}

% used for TeXing text within eps files
%\usepackage{psfrag}
% need this for including graphics (\includegraphics)
%\usepackage{graphicx}
% for neatly defining theorems and propositions
%\usepackage{amsthm}
% making logically defined graphics
%%%\usepackage{xypic}

\makeatletter
\@ifundefined{bibname}{}{\renewcommand{\bibname}{References}}
\makeatother
\begin{document}
The number 
\begin{align*}
\zeta(3) &= \sum_{n=1}^\infty\frac{1}{n^3} \\
         &= 1.202056903159594285399738161511449990764986292\ldots
\end{align*}
has been called Ap\'ery's constant since 1979, when Roger Ap\'ery published
a remarkable proof that it is irrational \cite{cite:apery_irr}. 

\begin{thebibliography}{1}

\bibitem{cite:apery_irr}
Roger Ap{\'e}ry.
\newblock Irrationalit{\'e} de $\zeta(2)$ et $\zeta(3)$.
\newblock {\em Ast{\'e}risque}, 61:11--13, 1979.

\bibitem{cite:poorten_aperyconst}
Alfred van~der Poorten.
\newblock A proof that {Euler} missed. {Ap{\'e}ry's} proof of the irrationality
  of $\zeta(3)$. An informal report.
\newblock {\em Math. Intell.}, 1:195--203, 1979.

\end{thebibliography}

%@ARTICLE{cite:apery_irr,
% author   = {Roger Ap{\'e}ry},
% title    = {Irrationalit{\'e} de $\zeta(2)$ et $\zeta(3)$},
% journal  = {Ast{\'e}risque},
% volume   = 61,
% year     = 1979,
% pages    = {11--13}
%}
%
%@ARTICLE{cite:poorten_aperyconst,
% author   = {van der Poorten, Alfred},
% journal  = {Math. Intell.},
% title    = {A proof that {Euler} missed. {Ap{\'e}ry's} proof of the %irrationality of $\zeta(3)$. {An} informal report.},
% volume   = 1,
% year     = 1979,
% pages    = {195--203}
%}
%%%%%
%%%%%
\end{document}
