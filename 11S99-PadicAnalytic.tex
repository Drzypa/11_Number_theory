\documentclass[12pt]{article}
\usepackage{pmmeta}
\pmcanonicalname{PadicAnalytic}
\pmcreated{2013-03-22 15:13:53}
\pmmodified{2013-03-22 15:13:53}
\pmowner{alozano}{2414}
\pmmodifier{alozano}{2414}
\pmtitle{p-adic analytic}
\pmrecord{4}{37001}
\pmprivacy{1}
\pmauthor{alozano}{2414}
\pmtype{Definition}
\pmcomment{trigger rebuild}
\pmclassification{msc}{11S99}
\pmclassification{msc}{12J12}
\pmclassification{msc}{11S80}
\pmsynonym{$p$-adic analytic}{PadicAnalytic}
\pmrelated{Analytic}
\pmrelated{PAdicExponentialAndPAdicLogarithm}
\pmdefines{$p$-adic analysis}
\pmdefines{p-adic analysis}

\endmetadata

% this is the default PlanetMath preamble.  as your knowledge
% of TeX increases, you will probably want to edit this, but
% it should be fine as is for beginners.

% almost certainly you want these
\usepackage{amssymb}
\usepackage{amsmath}
\usepackage{amsthm}
\usepackage{amsfonts}

% used for TeXing text within eps files
%\usepackage{psfrag}
% need this for including graphics (\includegraphics)
%\usepackage{graphicx}
% for neatly defining theorems and propositions
%\usepackage{amsthm}
% making logically defined graphics
%%%\usepackage{xypic}

% there are many more packages, add them here as you need them

% define commands here

\newtheorem{thm}{Theorem}
\newtheorem*{defn}{Definition}
\newtheorem{prop}{Proposition}
\newtheorem{lemma}{Lemma}
\newtheorem{cor}{Corollary}

\theoremstyle{definition}
\newtheorem{exa}{Example}

% Some sets
\newcommand{\Nats}{\mathbb{N}}
\newcommand{\Ints}{\mathbb{Z}}
\newcommand{\Reals}{\mathbb{R}}
\newcommand{\Complex}{\mathbb{C}}
\newcommand{\Rats}{\mathbb{Q}}
\newcommand{\Gal}{\operatorname{Gal}}
\newcommand{\Cl}{\operatorname{Cl}}
\begin{document}
\begin{defn}
Let $\Complex_p$ be the field of \PMlinkname{complex $p$-adic numbers}{ComplexPAdicNumbers}. Let $U$ be a domain in $\Complex_p$. A function $f: U \longrightarrow \Complex_p$  is {\em $p$-adic analytic}  if $f$ has a Taylor series (with coefficients in $\Complex_p$) about each point $z \in U$ that converges to the function $f$ in an open neighborhood of $z$.
\end{defn}

For example, the \PMlinkname{$p$-adic exponential function}{PAdicExponentialAndPAdicLogarithm} is analytic on its domain of definition:
$$U=\{ z\in \Complex_p : |z|_p<\frac{1}{p^{1/(p-1)}}\}.$$

The study of $p$-adic analytic functions is usually called {\em $p$-adic analysis} and it is very similar to complex analysis in many respects, although there are important differences coming from the distinct topologies of $\Complex$ and $\Complex_p$.
%%%%%
%%%%%
\end{document}
