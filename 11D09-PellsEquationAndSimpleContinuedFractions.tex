\documentclass[12pt]{article}
\usepackage{pmmeta}
\pmcanonicalname{PellsEquationAndSimpleContinuedFractions}
\pmcreated{2013-03-22 13:21:04}
\pmmodified{2013-03-22 13:21:04}
\pmowner{Thomas Heye}{1234}
\pmmodifier{Thomas Heye}{1234}
\pmtitle{Pell's equation and simple continued fractions}
\pmrecord{9}{33870}
\pmprivacy{1}
\pmauthor{Thomas Heye}{1234}
\pmtype{Theorem}
\pmcomment{trigger rebuild}
\pmclassification{msc}{11D09}

\endmetadata

% this is the default PlanetMath preamble.  as your knowledge
% of TeX increases, you will probably want to edit this, but
% it should be fine as is for beginners.

% almost certainly you want these
\usepackage{amssymb}
\usepackage{amsmath}
\usepackage{amsfonts}

% used for TeXing text within eps files
%\usepackage{psfrag}
% need this for including graphics (\includegraphics)
%\usepackage{graphicx}
% for neatly defining theorems and propositions
\usepackage{amsthm}
% making logically defined graphics
%%%\usepackage{xypic}

% there are many more packages, add them here as you need them

% define commands here
\newtheorem{thm}{Theorem}
\begin{document}
\begin{thm}
Let $d$ be a positive integer which is not a perfect square, and let $(x,y)$ be
a solution of $x^2 -dy^2 =1$. Then $\frac{x}{y}$ is a convergent in the simple
continued fraction expansion of $\sqrt{d}$.
\end{thm}
\begin{proof}
Suppose we have a non-trivial solution $x,y$ of Pell's equation, i.e. $y \neq
0$. Let $x,y$ both be positive integers. From
\[\left(\frac{x}{y}\right)^2 =d +\frac{1}{y^2}\]
we see that $\left(\frac{x}{y}\right)^2 > d$, hence $\frac{x}{y} >
\sqrt{d}$. So we get
\begin{eqnarray*}
\left\vert \frac{x}{y} -\sqrt{d}\right\vert =\frac{1}{y^2\left(\frac{x}{y}
+\sqrt{d}\right)} & < \frac{1}{y^2\left(2\sqrt{d}\right)} \\
& < \frac{1}{2y^2}.
\end{eqnarray*}
This implies that $\frac{x}{y}$ is a convergent of the continued fraction of
$\sqrt{d}$.
\end{proof}
%%%%%
%%%%%
\end{document}
