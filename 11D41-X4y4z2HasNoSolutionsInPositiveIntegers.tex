\documentclass[12pt]{article}
\usepackage{pmmeta}
\pmcanonicalname{X4y4z2HasNoSolutionsInPositiveIntegers}
\pmcreated{2013-03-22 17:05:04}
\pmmodified{2013-03-22 17:05:04}
\pmowner{rm50}{10146}
\pmmodifier{rm50}{10146}
\pmtitle{$x^4-y^4=z^2$ has no solutions in positive integers}
\pmrecord{7}{39377}
\pmprivacy{1}
\pmauthor{rm50}{10146}
\pmtype{Theorem}
\pmcomment{trigger rebuild}
\pmclassification{msc}{11D41}
\pmclassification{msc}{14H52}
\pmclassification{msc}{11F80}
\pmrelated{ExampleOfFermatsLastTheorem}
\pmrelated{IncircleRadiusDeterminedByPythagoreanTriple}

\endmetadata

% this is the default PlanetMath preamble.  as your knowledge
% of TeX increases, you will probably want to edit this, but
% it should be fine as is for beginners.

% almost certainly you want these
\usepackage{amssymb}
\usepackage{amsmath}
\usepackage{amsfonts}

% used for TeXing text within eps files
%\usepackage{psfrag}
% need this for including graphics (\includegraphics)
%\usepackage{graphicx}
% for neatly defining theorems and propositions
\usepackage{amsthm}
% making logically defined graphics
%%%\usepackage{xypic}

% there are many more packages, add them here as you need them

% define commands here
\newtheorem{thm}{Theorem}
\newtheorem{cor}{Corollary}
\begin{document}
We know (see example of Fermat's Last Theorem) that the sum of two fourth powers can never be a square unless all are zero. This article shows that the difference of two fourth powers can never be a square unless at least one of the numbers is zero. Fermat proved this fact as part of his proof that the area of a right triangle with integral sides is never a square; see the corollary below. The proof of the main theorem is a great example of the method of infinite descent.

\begin{thm}
\[x^4-y^4=z^2\]
has no solutions in positive integers.
\end{thm}
\begin{proof}
Suppose the equation has a solution in positive integers, and choose a solution that minimizes $x^2+y^2$. Note that $x,y$, and $z$ are pairwise coprime, since otherwise we could divide out by their common divisor to get a smaller solution. Thus
\[z^2 + (y^2)^2 = (x^2)^2\]
so that $z, y^2, x^2$ form a pythagorean triple. There are thus positive integers $p,q$ of opposite parity (and coprime since $x, y$, and $z$ are) such that $x^2 = p^2+q^2$ and either $y^2 = p^2-q^2$ or $y^2 = 2pq$.

Factoring the original equation, we get
\[(x^2-y^2)(x^2+y^2)=z^2\]
If $y^2 = p^2-q^2$, then $(xy)^2 = p^4-q^4$, and clearly $p^2+q^2 = x^2 < x^2+y^2$, so we have found a solution smaller than the assumed minimal solution.

Assume therefore that $y^2 = 2pq$. Now, $x^2 = p^2+q^2$; we may assume by relabeling if necessary that $q$ is even and $p$ odd. Then $p, q, x$ are pairwise coprime and form a pythagorean triple; thus there are $P>Q>0$ of opposite parity and coprime such that 
\[
  q = 2PQ,\quad p=P^2-Q^2,\quad x=P^2+Q^2
\]
Then
\[
  PQ(P^2-Q^2) = \frac{1}{2}pq = \frac{y^2}{4}
\]
is a square; it follows that $P, Q$, and $P^2-Q^2$ are all (nonzero) squares since they are pairwise coprime. Write
\[
  P = R^2,\quad Q=S^2,\quad P^2-Q^2=T^2
\]
for positive integers $R,S,T$. Then $T^2=R^4-S^4$, and
\[
  R^2+S^2 = P+Q < (P+Q)(PQ)(P-Q) = \frac{1}{2}pq = \frac{y^2}{4}\leq y^2 < x^2+y^2
\]
We have thus found a smaller solution in positive integers, contradicting the hypothesis.
\end{proof}

\begin{cor}
No right triangle with integral sides has area that is an integral square.
\end{cor}
\begin{proof}
Suppose $x,y,z$ is a right triangle with $z$ the hypotenuse, and let $d=\gcd(x,y,z)$. Either $x/d$ or $y/d$ is even; by relabeling if necessary, assume $x/d$ is even. Then we can choose relatively prime integers $p,q$ with $p>q$ and of opposite parity such that
\begin{gather*}
x=(2pq)d\\
y=(p^2-q^2)d\\
z=(p^2+q^2)d
\end{gather*}

If the triangle's area is to be a square, then
\[\frac{1}{2}xy=pq(p^2-q^2)d^2\]
must be a square, and thus $pq(p^2-q^2)$ must be a square. Since $p$ and $q$ are coprime, it follows that $p$, $q$, and $p^2-q^2$ are all squares, and thus that $p^2-q^2$ is the difference of two fourth powers. But then
\[\frac{\frac{1}{2}xy}{pqd^2}=p^2-q^2\]
must also be a square. Since both $p$ and $q$ are squares, this is impossible by the theorem.
\end{proof}
%%%%%
%%%%%
\end{document}
