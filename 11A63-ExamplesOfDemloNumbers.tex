\documentclass[12pt]{article}
\usepackage{pmmeta}
\pmcanonicalname{ExamplesOfDemloNumbers}
\pmcreated{2013-03-22 16:22:09}
\pmmodified{2013-03-22 16:22:09}
\pmowner{PrimeFan}{13766}
\pmmodifier{PrimeFan}{13766}
\pmtitle{examples of Demlo numbers}
\pmrecord{4}{38507}
\pmprivacy{1}
\pmauthor{PrimeFan}{13766}
\pmtype{Example}
\pmcomment{trigger rebuild}
\pmclassification{msc}{11A63}

% this is the default PlanetMath preamble.  as your knowledge
% of TeX increases, you will probably want to edit this, but
% it should be fine as is for beginners.

% almost certainly you want these
\usepackage{amssymb}
\usepackage{amsmath}
\usepackage{amsfonts}

% used for TeXing text within eps files
%\usepackage{psfrag}
% need this for including graphics (\includegraphics)
%\usepackage{graphicx}
% for neatly defining theorems and propositions
%\usepackage{amsthm}
% making logically defined graphics
%%%\usepackage{xypic}

% there are many more packages, add them here as you need them

% define commands here

\begin{document}
In base 10, the first 20 Demlo numbers are: 1, 121, 12321, 1234321, 123454321, 12345654321, 1234567654321, 123456787654321, 12345678987654321, 1234567900987654321, 123456790120987654321, 12345679012320987654321, 1234567901234320987654321, 123456790123454320987654321, 12345679012345654320987654321, 1234567901234567654320987654321, 123456790123456787654320987654321, 12345679012345678987654320987654321, 1234567901234567900987654320987654321, 123456790123456790120987654320987654321 (these are listed in sequence A002477 of Sloane's OEIS).

In hexadecimal, the first few Demlo numbers are: 1, 289, 74529, 19088161, 4886709025, 1250999747361, 320255971115809, 81985529178309409, 20988295478809805601, 5373003642721911784225, 1375488932539155041567521, 352125166730061220638180129, 90144042682896272963324429089, 23076874926821455486290258903841, 5907679981266292758213173560296225, 1512366075204170948562138307930440481, etc. Written in hexadecimal, the fifteenth of these is 123456789ABCDEFEDCBA987654321. The next one is 
123456789ABCDF00FEDCBA987654321. Although it is a pandigital number, it is not palindromic because the "left" side is "missing" the E that appears on the right side. As is also the case with base 10, the first non-palindromic Demlo number has two "significant" zeroes in the middle.
%%%%%
%%%%%
\end{document}
