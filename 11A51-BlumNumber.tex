\documentclass[12pt]{article}
\usepackage{pmmeta}
\pmcanonicalname{BlumNumber}
\pmcreated{2013-03-22 17:53:17}
\pmmodified{2013-03-22 17:53:17}
\pmowner{PrimeFan}{13766}
\pmmodifier{PrimeFan}{13766}
\pmtitle{Blum number}
\pmrecord{4}{40371}
\pmprivacy{1}
\pmauthor{PrimeFan}{13766}
\pmtype{Definition}
\pmcomment{trigger rebuild}
\pmclassification{msc}{11A51}
\pmsynonym{Blum integer}{BlumNumber}

\endmetadata

% this is the default PlanetMath preamble.  as your knowledge
% of TeX increases, you will probably want to edit this, but
% it should be fine as is for beginners.

% almost certainly you want these
\usepackage{amssymb}
\usepackage{amsmath}
\usepackage{amsfonts}

% used for TeXing text within eps files
%\usepackage{psfrag}
% need this for including graphics (\includegraphics)
%\usepackage{graphicx}
% for neatly defining theorems and propositions
%\usepackage{amsthm}
% making logically defined graphics
%%%\usepackage{xypic}

% there are many more packages, add them here as you need them

% define commands here

\begin{document}
Given a semiprime $n = pq$, if both $p$ and $q$ are Gaussian primes with no imaginary part, then $n$ is called a {\em Blum number}. The first few Blum numbers are 21, 33, 57, 69, 77, 93, 129, 133, 141, 161, 177, 201, 209, 213, 217, 237, 249, 253, 301, 309, 321, 329, 341, 381, 393, 413, 417, 437, 453, 469, 473, 489, 497, etc., listed in A016105 of Sloane's OEIS.

A semiprime that is a Blum number is also a semiprime among the Gaussian integers and its prime factors also have no imaginary parts. The other real semiprimes are not semiprimes among the Gaussian integers. For example, 177 can only be factored as $3 \times 59$ whether Gaussian integers are allowed or not. 159, on the other hand can be factored as either $3 \times 53$ or $3(-i)(2 + 7i)(7 + 2i)$.

Large Blum numbers had applications in cryptography prior to advances in integer factorization by means of quadratic sieves.
%%%%%
%%%%%
\end{document}
