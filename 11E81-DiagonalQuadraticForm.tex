\documentclass[12pt]{article}
\usepackage{pmmeta}
\pmcanonicalname{DiagonalQuadraticForm}
\pmcreated{2013-03-22 15:42:05}
\pmmodified{2013-03-22 15:42:05}
\pmowner{CWoo}{3771}
\pmmodifier{CWoo}{3771}
\pmtitle{diagonal quadratic form}
\pmrecord{12}{37646}
\pmprivacy{1}
\pmauthor{CWoo}{3771}
\pmtype{Definition}
\pmcomment{trigger rebuild}
\pmclassification{msc}{11E81}
\pmclassification{msc}{15A63}
\pmclassification{msc}{11H55}
\pmsynonym{canonical quadratic form}{DiagonalQuadraticForm}
\pmrelated{DiagonalizationOfQuadraticForm}

\endmetadata

\usepackage{amssymb,amscd}
\usepackage{amsmath}
\usepackage{amsfonts}

% used for TeXing text within eps files
%\usepackage{psfrag}
% need this for including graphics (\includegraphics)
%\usepackage{graphicx}
% for neatly defining theorems and propositions
%\usepackage{amsthm}
% making logically defined graphics
%%%\usepackage{xypic}

% define commands here
\begin{document}
Let $Q(\boldsymbol{x})\in k[x_1,\ldots,x_n]$ be a quadratic form over a field $k$ ($\operatorname{char}(k)\neq 2$), where $\boldsymbol{x}$ is the column vector $(x_1,\ldots,x_n)^T$.  We write $Q$ as 

$$Q(\boldsymbol{x})=\boldsymbol{x}^TM(Q)\boldsymbol{x},$$

where $M(Q)$ is the associated $n\times n$ symmetric matrix over $k$.  We say that $Q$ is a \emph{diagonal quadratic form} if $M(Q)$ is a diagonal matrix.

Let's see what a diagonal quadratic form looks like.  If $M=M(Q)$ is diagonal whose diagonal entry in cell $(i,i)$ is $r_i$, then 

\begin{center}$
Q(\boldsymbol{x})=\boldsymbol{x}^T
\begin{pmatrix}r_1 & \cdots & 0 \\
\vdots & \ddots & \vdots \\
0 & \cdots & r_n\end{pmatrix}
\begin{pmatrix}x_1 \\
\vdots \\ x_n\end{pmatrix}
=\begin{pmatrix}x_1 & \cdots & x_n\end{pmatrix}
\begin{pmatrix}r_1x_1 \\ \vdots \\ r_nx_n\end{pmatrix}
=r_1x_1^2+\cdots+r_nx_n^2.
$\end{center}

So the coefficients of $x_ix_j$ for $i\neq j$ are all $0$ in a diagonal quadratic form.  A diagonal quadratic form is completely determined by the diagonal entries of $M(Q)$.

\textbf{Remark.} Every quadratic form is \PMlinkname{equivalent}{EquivalentQuadraticForms} to a diagonal quadratic form.  On the other hand, a quadratic form may be \PMlinkescapetext{equivalent} to more than one diagonal quadratic form.
%%%%%
%%%%%
\end{document}
