\documentclass[12pt]{article}
\usepackage{pmmeta}
\pmcanonicalname{AlgebraicNumbersAreCountable}
\pmcreated{2013-03-22 15:13:47}
\pmmodified{2013-03-22 15:13:47}
\pmowner{pahio}{2872}
\pmmodifier{pahio}{2872}
\pmtitle{algebraic numbers are countable}
\pmrecord{14}{36999}
\pmprivacy{1}
\pmauthor{pahio}{2872}
\pmtype{Theorem}
\pmcomment{trigger rebuild}
\pmclassification{msc}{11R04}
\pmclassification{msc}{03E10}
\pmsynonym{algebraic numbers may be set in a sequence}{AlgebraicNumbersAreCountable}
\pmrelated{HeightOfAnAlgebraicNumber2}
\pmrelated{ProofOfTheExistenceOfTranscendentalNumbers}
\pmrelated{A_nAreCountableSoIsA_1XXA_nIfA_1}
\pmrelated{ExamplesOfCountableSets}
\pmrelated{FieldOfAlgebraicNumbers}

% this is the default PlanetMath preamble.  as your knowledge
% of TeX increases, you will probably want to edit this, but
% it should be fine as is for beginners.

% almost certainly you want these
\usepackage{amssymb}
\usepackage{amsmath}
\usepackage{amsfonts}

% used for TeXing text within eps files
%\usepackage{psfrag}
% need this for including graphics (\includegraphics)
%\usepackage{graphicx}
% for neatly defining theorems and propositions
 \usepackage{amsthm}
% making logically defined graphics
%%%\usepackage{xypic}

% there are many more packages, add them here as you need them

% define commands here

\theoremstyle{definition}
\newtheorem*{thmplain}{Theorem}
\begin{document}
\begin{thmplain}
The set of (a) all algebraic numbers, (b) the real algebraic numbers is countable.
\end{thmplain}

{\em Proof.}\, Let's consider the algebraic equations
\begin{align}
                 P(x) \;=\; 0
\end{align}
where 
     $$P(x) \;:=\; a_0x^n\!+\!a_1x^{n-1}\!+\!\ldots\!+\!a_{n-1}x\!+\!a_n$$
is an \PMlinkname{irreducible}{IrreduciblePolynomial2} and primitive polynomial with integer coefficients $a_j$ and\, $a_0 > 0$.\, Each algebraic number \PMlinkescapetext{satisfies} exactly one such equation (see the minimal polynomial).\, For every integer\, $N = 2,\,3,\,4,\,\ldots$\, there exists a finite number of equations (1) such that
     $$n\!+\!a_0\!+\!|a_1|\!+\ldots+\!|a_n| \;=\; N$$
(e.g. if\, $N = 3$,\, then one has the equations\, $x\!-\!1 = 0$\, and\, 
$x\!+\!1 = 0$) and thus only a finite set of algebraic numbers as the \PMlinkescapetext{roots} of these equations.\, These algebraic numbers may be ordered to a \PMlinkname{finite sequence}{OrderedTuplet} $S_N$ using a \PMlinkescapetext{fixed ordering} system, for example by the magnitude of the real part and the imaginary part.\, When one forms the concatenated sequence
        $$S_2,\,S_3,\,S_4,\,\ldots$$
it comprises all algebraic numbers in a countable setting, which defines a bijection from the set onto $\mathbb{Z}_+$.

\begin{thebibliography}{9}
\bibitem{EK}{\sc E. Kamke:} {\em Mengenlehre}.\, Sammlung G\"oschen: Band 999/999a.\, -- Walter de Gruyter \& Co., Berlin (1962).
\end{thebibliography}
%%%%%
%%%%%
\end{document}
