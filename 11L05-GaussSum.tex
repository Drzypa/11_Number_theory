\documentclass[12pt]{article}
\usepackage{pmmeta}
\pmcanonicalname{GaussSum}
\pmcreated{2013-03-22 12:48:28}
\pmmodified{2013-03-22 12:48:28}
\pmowner{djao}{24}
\pmmodifier{djao}{24}
\pmtitle{Gauss sum}
\pmrecord{7}{33126}
\pmprivacy{1}
\pmauthor{djao}{24}
\pmtype{Definition}
\pmcomment{trigger rebuild}
\pmclassification{msc}{11L05}
\pmrelated{KloostermanSum}

\endmetadata

% this is the default PlanetMath preamble.  as your knowledge
% of TeX increases, you will probably want to edit this, but
% it should be fine as is for beginners.

% almost certainly you want these
\usepackage{amssymb}
\usepackage{amsmath}
\usepackage{amsfonts}

% used for TeXing text within eps files
%\usepackage{psfrag}
% need this for including graphics (\includegraphics)
%\usepackage{graphicx}
% for neatly defining theorems and propositions
%\usepackage{amsthm}
% making logically defined graphics
%%%\usepackage{xypic} 

% there are many more packages, add them here as you need them

% define commands here
\newcommand{\Z}{\mathbb{Z}}
\newcommand{\C}{\mathbb{C}}
\begin{document}
Let $p$ be a prime. Let $\chi$ be any multiplicative group character on $\Z/p\Z$ (that is, any group homomorphism of multiplicative groups $(\Z/p\Z)^\times \to \C^\times$). For any $a \in \Z/p\Z$, the complex number
$$
g_a(\chi) := \sum_{t \in \Z/p\Z} \chi(t) e^{2 \pi i a t/p}
$$
is called a {\em Gauss sum} on $\Z/p\Z$ associated to $\chi$.

In general, the equation $g_a(\chi) = \chi(a^{-1}) g_1(\chi)$ (for nontrivial $a$ and $\chi$) reduces the computation of general Gauss sums to that of $g_1(\chi)$. The absolute value of $g_1(\chi)$ is always $\sqrt{p}$ as long as $\chi$ is nontrivial, and if $\chi$ is a quadratic character (that is, $\chi(t)$ is the Legendre symbol $\left(\frac{t}{p}\right)$), then the value of the Gauss sum is known to be
$$
g_1(\chi) =
\begin{cases}
\sqrt{p}, & p \equiv 1 \pmod{4}, \\
i \sqrt{p}, & p \equiv 3 \pmod{4}.
\end{cases}
$$
\begin{thebibliography}{9}
\bibitem{ir} Kenneth Ireland \& Michael Rosen, {\em A Classical Introduction to Modern Number Theory}, Second Edition, Springer--Verlag, 1990.
\end{thebibliography}
%%%%%
%%%%%
\end{document}
