\documentclass[12pt]{article}
\usepackage{pmmeta}
\pmcanonicalname{LeastSignificantDigit}
\pmcreated{2013-03-22 16:21:06}
\pmmodified{2013-03-22 16:21:06}
\pmowner{PrimeFan}{13766}
\pmmodifier{PrimeFan}{13766}
\pmtitle{least significant digit}
\pmrecord{5}{38486}
\pmprivacy{1}
\pmauthor{PrimeFan}{13766}
\pmtype{Definition}
\pmcomment{trigger rebuild}
\pmclassification{msc}{11A63}
\pmdefines{least significant bit}

\endmetadata

% this is the default PlanetMath preamble.  as your knowledge
% of TeX increases, you will probably want to edit this, but
% it should be fine as is for beginners.

% almost certainly you want these
\usepackage{amssymb}
\usepackage{amsmath}
\usepackage{amsfonts}

% used for TeXing text within eps files
%\usepackage{psfrag}
% need this for including graphics (\includegraphics)
%\usepackage{graphicx}
% for neatly defining theorems and propositions
%\usepackage{amsthm}
% making logically defined graphics
%%%\usepackage{xypic}

% there are many more packages, add them here as you need them

% define commands here

\begin{document}
The {\em least significant digit} of a number $n$ written in a given positional base $b$ is the digit in the least significant place value, and has to be in the range $-1 < d_1 < b$. In the case of an integer, the least significant digit is the 1's place value, usually written to the right of the $b$'s place value. In the case of a transcendental number, there is no actual least significant digit, but for computational purposes the rational approximation would have a least significant digit.

In an array of digits $k$ long meant for mathematical manipulation, it might be convenient to index the least significant digit with index 1 or 0, and the more significant digits with larger integers. (This enables the calculation of the value of a given digit as $d_ib^i$  rather than $d_ib^{k - i}$.) For an array of digits meant for text string manipulation, however, the least significant digit might be placed at position $k$ (for example, by Mathematica's IntegerDigits function).

In binary, the least significant digit is often called the {\em least significant bit}.
%%%%%
%%%%%
\end{document}
