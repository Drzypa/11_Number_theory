\documentclass[12pt]{article}
\usepackage{pmmeta}
\pmcanonicalname{DirichletsUnitTheorem}
\pmcreated{2013-03-22 13:22:42}
\pmmodified{2013-03-22 13:22:42}
\pmowner{yark}{2760}
\pmmodifier{yark}{2760}
\pmtitle{Dirichlet's unit theorem}
\pmrecord{10}{33911}
\pmprivacy{1}
\pmauthor{yark}{2760}
\pmtype{Theorem}
\pmcomment{trigger rebuild}
\pmclassification{msc}{11R04}
\pmclassification{msc}{11R27}
\pmrelated{Regulator}

\usepackage{amssymb}
\usepackage{amsmath}
\usepackage{amsfonts}

\begin{document}
\PMlinkescapephrase{occur in}

Let $K$ be a number field, and let $\mathcal{O}_K$ be its ring of integers.
Then
\[
  \mathcal{O}_K^*\cong \mu(K)\times\mathbb{Z}^{r+s-1}.
\]
Here $\mathcal{O}_K^*$ is the group of units of $\mathcal{O}_K$,
$\mu(K)$ is the finite cyclic group of the roots of unity in $\mathcal{O}_K^*$,
$r$ is the number of real embeddings $K\rightarrow \mathbb{R}$,
and $2s$ is the number of non-real complex embeddings $K\rightarrow \mathbb{C}$ (which occur in complex conjugate pairs, so $s$ is an integer).
%%%%%
%%%%%
\end{document}
