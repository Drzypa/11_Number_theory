\documentclass[12pt]{article}
\usepackage{pmmeta}
\pmcanonicalname{GaloisGroupsOfFiniteAbelianExtensionsOfmathbbQ}
\pmcreated{2013-03-22 16:18:40}
\pmmodified{2013-03-22 16:18:40}
\pmowner{Wkbj79}{1863}
\pmmodifier{Wkbj79}{1863}
\pmtitle{Galois groups of finite abelian extensions of $\mathbb{Q}$}
\pmrecord{11}{38434}
\pmprivacy{1}
\pmauthor{Wkbj79}{1863}
\pmtype{Theorem}
\pmcomment{trigger rebuild}
\pmclassification{msc}{11R32}
\pmclassification{msc}{11N13}
\pmclassification{msc}{11R20}
\pmclassification{msc}{12F10}
\pmrelated{AbelianNumberField}

\usepackage{amssymb}
\usepackage{amsmath}
\usepackage{amsfonts}

\usepackage{psfrag}
\usepackage{graphicx}
\usepackage{amsthm}
%%\usepackage{xypic}

\newtheorem*{thm*}{Theorem}

\begin{document}
\PMlinkescapeword{order}

\begin{thm*}
Let $G$ be a finite abelian group with $|G|>1$.  Then there exist infinitely many number fields $K$ with $K/\mathbb{Q}$ Galois and $\operatorname{Gal}(K/\mathbb{Q}) \cong G$.
\end{thm*}

\begin{proof}
This will first be proven for $G$ cyclic.

Let $|G|=n$.  By Dirichlet's theorem on primes in arithmetic progressions, there exists a prime $p$ with $p \equiv 1 \operatorname{mod} n$.  Let $\zeta_p$ denote a \PMlinkescapetext{primitive} $p^{\text{th}}$ root of unity.  Let $L=\mathbb{Q}(\zeta_p)$.  Then $L/\mathbb{Q}$ is Galois with $\operatorname{Gal}(L/\mathbb{Q})$ cyclic of \PMlinkname{order}{OrderGroup} $p-1$.  Since $n$ divides $p-1$, there exists a subgroup $H$ of $\operatorname{Gal}(L/\mathbb{Q})$ such that $\displaystyle |H|=\frac{p-1}{n}$.  Since $\operatorname{Gal}(L/\mathbb{Q})$ is cyclic, it is abelian, and $H$ is a normal subgroup of $\operatorname{Gal}(L/\mathbb{Q})$.  Let $K=L^H$, the subfield of $L$ \PMlinkname{fixed}{FixedField} by $H$.  Then $K/\mathbb{Q}$ is Galois with $\operatorname{Gal}(K/\mathbb{Q})$ cyclic of order $n$.  Thus, $\operatorname{Gal}(K/\mathbb{Q}) \cong G$.

Let $p$ and $q$ be distinct primes with $p \equiv 1 \operatorname{mod} n$ and $q \equiv 1 \operatorname{mod} n$.  Then there exist subfields $K_1$ and $K_2$ of $\mathbb{Q}(\zeta_p)$ and $\mathbb{Q}(\zeta_q)$, respectively, such that $\operatorname{Gal}(K_1/\mathbb{Q}) \cong G$ and $\operatorname{Gal}(K_2/\mathbb{Q}) \cong G$.  Note that $K_1 \cap K_2=\mathbb{Q}$ since $\mathbb{Q} \subseteq K_1 \cap K_2 \subseteq \mathbb{Q}(\zeta_p) \cap \mathbb{Q}(\zeta_q)=\mathbb{Q}$.  Thus, $K_1 \neq K_2$.  Therefore, for every prime $p$ with $p \equiv 1 \operatorname{mod} n$, there exists a distinct number field $K$ such that $K/\mathbb{Q}$ is Galois and $\operatorname{Gal}(K/\mathbb{Q}) \cong G$.  The theorem in the cyclic case follows from using the full \PMlinkescapetext{force} of Dirichlet's theorem on primes in arithmetic progressions:  There exist infinitely many primes $p$ with $p \equiv 1 \operatorname{mod} n$.

The general case follows immediately from the above \PMlinkescapetext{argument}, the \PMlinkname{fundamental theorem of finite abelian groups}{FundamentalTheoremOfFinitelyGeneratedAbelianGroups}, and a theorem regarding the Galois group of the compositum of two Galois extensions.
\end{proof}
%%%%%
%%%%%
\end{document}
