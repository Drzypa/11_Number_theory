\documentclass[12pt]{article}
\usepackage{pmmeta}
\pmcanonicalname{ProofOfSquareRootOfSquareRootBinomial}
\pmcreated{2013-03-22 17:42:45}
\pmmodified{2013-03-22 17:42:45}
\pmowner{rspuzio}{6075}
\pmmodifier{rspuzio}{6075}
\pmtitle{proof of square root of square root binomial}
\pmrecord{5}{40157}
\pmprivacy{1}
\pmauthor{rspuzio}{6075}
\pmtype{Proof}
\pmcomment{trigger rebuild}
\pmclassification{msc}{11A25}

% this is the default PlanetMath preamble.  as your knowledge
% of TeX increases, you will probably want to edit this, but
% it should be fine as is for beginners.

% almost certainly you want these
\usepackage{amssymb}
\usepackage{amsmath}
\usepackage{amsfonts}

% used for TeXing text within eps files
%\usepackage{psfrag}
% need this for including graphics (\includegraphics)
%\usepackage{graphicx}
% for neatly defining theorems and propositions
%\usepackage{amsthm}
% making logically defined graphics
%%%\usepackage{xypic}

% there are many more packages, add them here as you need them

% define commands here

\begin{document}
We square the expression on the right-hand-side and expand
using the binomial formula:
\begin{align*}
 \left(
  \sqrt{\frac{a+\sqrt{a^2-b}}{2}} \pm
  \sqrt{\frac{a-\sqrt{a^2-b}}{2}}
 \right)^2 &=
 \left( \sqrt{\frac{a+\sqrt{a^2-b}}{2}} \right)^2 \\ &+
 \left( \sqrt{\frac{a-\sqrt{a^2-b}}{2}} \right)^2 \pm 
 2 \sqrt{\frac{a+\sqrt{a^2-b}}{2}}
   \sqrt{\frac{a-\sqrt{a^2-b}}{2}}
\end{align*} 
Since the squaring operation undoes the square roots, we
obtain the following:
\[
 \left( \sqrt{\frac{a+\sqrt{a^2-b}}{2}} \right)^2 +
 \left( \sqrt{\frac{a-\sqrt{a^2-b}}{2}} \right)^2 =
 \frac{a+\sqrt{a^2-b}}{2} +
 \frac{a-\sqrt{a^2-b}}{2} = a
\] 
Since the product of square roots equals the square root
of the product, we have the following:
\begin{align*}
 \sqrt{\frac{a+\sqrt{a^2-b}}{2}}
 \sqrt{\frac{a-\sqrt{a^2-b}}{2}} &=
 \sqrt{\frac{a+\sqrt{a^2-b}}{2} \cdot
       \frac{a-\sqrt{a^2-b}}{2}} \\ &=
 \sqrt{\frac{a^2 - (\sqrt{a^2-b})^2}{4}} \\ &=
 \sqrt{\frac{a^2 - (a^2-b)}{4}} \\ &=
 \sqrt{\frac{b}{4}} = 
  \frac{\sqrt{b}}{2}
\end{align*}
Combining what we have calculated above, we obtain
\[
\left(
  \sqrt{\frac{a+\sqrt{a^2-b}}{2}} \pm
  \sqrt{\frac{a-\sqrt{a^2-b}}{2}}
 \right)^2 =
 a \pm \sqrt{b} .
\]
Because the square of the asserted value of the square root equals the radicand ($a\pm\sqrt{b}$) of the square root, and the asserted value of the square root is clearly non-negative, we have justified the validity of the formulas
\[
 \sqrt{a \pm \sqrt{b}} =
 \sqrt{\frac{a+\sqrt{a^2-b}}{2}} \pm
 \sqrt{\frac{a-\sqrt{a^2-b}}{2}}.
\]
%%%%%
%%%%%
\end{document}
