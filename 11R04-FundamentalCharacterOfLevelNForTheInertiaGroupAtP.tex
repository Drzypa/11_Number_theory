\documentclass[12pt]{article}
\usepackage{pmmeta}
\pmcanonicalname{FundamentalCharacterOfLevelNForTheInertiaGroupAtP}
\pmcreated{2013-03-22 15:36:26}
\pmmodified{2013-03-22 15:36:26}
\pmowner{alozano}{2414}
\pmmodifier{alozano}{2414}
\pmtitle{fundamental character of level $n$ for the inertia group at $p$}
\pmrecord{4}{37525}
\pmprivacy{1}
\pmauthor{alozano}{2414}
\pmtype{Definition}
\pmcomment{trigger rebuild}
\pmclassification{msc}{11R04}
\pmclassification{msc}{11R32}
\pmclassification{msc}{11R34}

% this is the default PlanetMath preamble.  as your knowledge
% of TeX increases, you will probably want to edit this, but
% it should be fine as is for beginners.

% almost certainly you want these
\usepackage{amssymb}
\usepackage{amsmath}
\usepackage{amsthm}
\usepackage{amsfonts}

% used for TeXing text within eps files
%\usepackage{psfrag}
% need this for including graphics (\includegraphics)
%\usepackage{graphicx}
% for neatly defining theorems and propositions
%\usepackage{amsthm}
% making logically defined graphics
%%%\usepackage{xypic}

% there are many more packages, add them here as you need them

% define commands here

\newtheorem{thm}{Theorem}
\newtheorem{defn}{Definition}
\newtheorem{prop}{Proposition}
\newtheorem{lemma}{Lemma}
\newtheorem{cor}{Corollary}

\theoremstyle{definition}
\newtheorem{exa}{Example}

% Some sets
\newcommand{\Nats}{\mathbb{N}}
\newcommand{\Ints}{\mathbb{Z}}
\newcommand{\Reals}{\mathbb{R}}
\newcommand{\Complex}{\mathbb{C}}
\newcommand{\Rats}{\mathbb{Q}}
\newcommand{\Gal}{\operatorname{Gal}}
\newcommand{\Cl}{\operatorname{Cl}}
\newcommand{\F}{\mathbb{F}}
\begin{document}
Let $p>2$ be a prime, fix algebraic closures $\overline{\Rats}$ and $\overline{\Rats_p}$, and fix an embedding of $\overline{\Rats}\hookrightarrow \overline{\Rats_p}$. This embedding corresponds with an inclusion of the absolute Galois groups:
$$\Gal(\overline{\Rats_p}/\Rats_p)\hookrightarrow \Gal(\overline{\Rats}/\Rats), \quad \sigma \mapsto \sigma|_{\overline{\Rats}}.$$
Let $I_p$ be the inertia subgroup of $\Gal(\overline{\Rats_p}/\Rats_p)$ which we regard as a subgroup of $\Gal(\overline{\Rats}/\Rats)$ via the injection above (for more information on the inertia subgroup at $p$, $I_p$, see the entry on Galois representations). Let $\F_{p^n}$ be the finite field of $p^n$ elements. The purpose of this entry is to define $\F_{p^n}$-valued characters $\Psi_n$, for every $n\geq 1$:
$$\Psi_n : I_p \longrightarrow \F_{p^n}^\times \cong \Ints/(p^n-1)\Ints$$
which we will refer to as the fundamental character of level $n$ of $I_p$.

\begin{defn}
Let $\chi_p:\Gal(\overline{\Rats}/\Rats)\to \Ints_p^\times$ be the $p$-adic cyclotomic character and let $\overline{\chi_p}$ be the reduction of $\chi_p$ modulo $p$. The fundamental character of level $1$ is $\Psi_1=\overline{\chi_p}|_{I_p}$, i.e. $\Psi_1$ is the restriction of the $p$-adic cyclotomic character $\chi_p$ to $I_p$, composed with reduction modulo $p$.
\end{defn}

Next, we define the fundamental characters in more generality. Let $K_n/\Rats_p$ be the unique unramified field extension of degree $n$ (it is unique by local field theory). The residue field of $K_n$ is the field $k_n=\F_{p^n}$ (because $k$ must be an extension of degree $n$ of $\F_p$).

\begin{lemma}
The field $K_n$ contains all $(p^n-1)$th roots of unity.
\end{lemma}
\begin{proof}
Clearly, the polynomial $x^{p^n-1}-1=0$ has $p^n-1$ distinct roots in $k_n=\F_{p^n}$. Using Hensel's lemma, one can check that each root in $k_n$ lifts to an element of $K_n$.
\end{proof}

Let $K_n'=K_n((-p)^{\frac{1}{p^n-1}})$. By the lemma, the $(p^n-1)$th roots of unity are contained in $K_n$. Therefore, the extension $K_n'/K_n$ is Galois. Moreover, by Kummer theory one has:
$$\Gal(K_n'/K_n)=k_n^\times=\F_{p^n}^\times.$$
Notice that the fact that $K_n/\Rats_p$ is unramified implies that the inertia group $I_p$ injects into $\Gal(\overline{\Rats_p}/K_n)\hookrightarrow \Gal(\overline{\Rats_p}/\Rats_p)$. Therefore there is a map:
\begin{eqnarray} 
\label{psi} I_p \hookrightarrow \Gal(\overline{\Rats_p}/K_n)\to \Gal(K_n'/K_n) \to \F_{p^n}^\times
\end{eqnarray}
where the second map is simply given by restriction to $K_n'$.

\begin{defn}
The fundamental character of level $n\geq 1$ is the map $\Psi_n : I_p \to \F_{p^n}^\times$ given by Eq. (\ref{psi}).
\end{defn}

Note from the author: I would like to thank Eknath Ghate for explaining this to me.
%%%%%
%%%%%
\end{document}
