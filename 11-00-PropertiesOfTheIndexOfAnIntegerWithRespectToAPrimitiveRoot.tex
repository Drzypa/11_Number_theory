\documentclass[12pt]{article}
\usepackage{pmmeta}
\pmcanonicalname{PropertiesOfTheIndexOfAnIntegerWithRespectToAPrimitiveRoot}
\pmcreated{2013-03-22 16:20:52}
\pmmodified{2013-03-22 16:20:52}
\pmowner{alozano}{2414}
\pmmodifier{alozano}{2414}
\pmtitle{properties of the index of an integer with respect to a primitive root}
\pmrecord{4}{38481}
\pmprivacy{1}
\pmauthor{alozano}{2414}
\pmtype{Theorem}
\pmcomment{trigger rebuild}
\pmclassification{msc}{11-00}

% this is the default PlanetMath preamble.  as your knowledge
% of TeX increases, you will probably want to edit this, but
% it should be fine as is for beginners.

% almost certainly you want these
\usepackage{amssymb}
\usepackage{amsmath}
\usepackage{amsthm}
\usepackage{amsfonts}

% used for TeXing text within eps files
%\usepackage{psfrag}
% need this for including graphics (\includegraphics)
%\usepackage{graphicx}
% for neatly defining theorems and propositions
%\usepackage{amsthm}
% making logically defined graphics
%%%\usepackage{xypic}

% there are many more packages, add them here as you need them

% define commands here

\newtheorem{thm}{Theorem}
\newtheorem*{defn}{Definition}
\newtheorem*{prop}{Proposition}
\newtheorem{lemma}{Lemma}
\newtheorem{cor}{Corollary}

\theoremstyle{definition}
\newtheorem{exa}{Example}

% Some sets
\newcommand{\Nats}{\mathbb{N}}
\newcommand{\Ints}{\mathbb{Z}}
\newcommand{\Reals}{\mathbb{R}}
\newcommand{\Complex}{\mathbb{C}}
\newcommand{\Rats}{\mathbb{Q}}
\newcommand{\Gal}{\operatorname{Gal}}
\newcommand{\Cl}{\operatorname{Cl}}
\newcommand{\ind}{\operatorname{ind}}
\begin{document}
\begin{defn}
Let $m>1$ be an integer such that the integer $g$ is a primitive root for $m$. Suppose $a$ is another integer relatively prime to $g$. The index of $a$ (to base $g$) is the smallest positive integer $n$ such that $g^n\equiv a \mod m$, and it is denoted by $\operatorname{ind} a$ or $\operatorname{ind}_g a$.
\end{defn}

\begin{prop}
Suppose $g$ is a primitive root of $m$.
\begin{enumerate}
\item $\ind 1 \equiv 0 \mod \phi(m)$; $\ind g \equiv 1 \mod \phi(m)$, where $\phi$ is the Euler phi function.

\item $a\equiv b \mod m$ if and only if $\ind a \equiv \ind b \mod \phi(m)$.

\item $\ind (ab) \equiv \ind a + \ind b \mod \phi(m)$.

\item $\ind a^k \equiv k\ind a \mod \phi(m)$ for any $k\geq 0$.
\end{enumerate}
\end{prop}
%%%%%
%%%%%
\end{document}
