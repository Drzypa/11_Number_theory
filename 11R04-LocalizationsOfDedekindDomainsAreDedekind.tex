\documentclass[12pt]{article}
\usepackage{pmmeta}
\pmcanonicalname{LocalizationsOfDedekindDomainsAreDedekind}
\pmcreated{2013-03-22 18:35:13}
\pmmodified{2013-03-22 18:35:13}
\pmowner{gel}{22282}
\pmmodifier{gel}{22282}
\pmtitle{localizations of Dedekind domains are Dedekind}
\pmrecord{4}{41313}
\pmprivacy{1}
\pmauthor{gel}{22282}
\pmtype{Theorem}
\pmcomment{trigger rebuild}
\pmclassification{msc}{11R04}
\pmclassification{msc}{13F05}
\pmclassification{msc}{13H10}
%\pmkeywords{localization}
%\pmkeywords{Dedekind domain}
%\pmkeywords{integral domain}
\pmrelated{Localization}

% this is the default PlanetMath preamble.  as your knowledge
% of TeX increases, you will probably want to edit this, but
% it should be fine as is for beginners.

% almost certainly you want these
\usepackage{amssymb}
\usepackage{amsmath}
\usepackage{amsfonts}

% used for TeXing text within eps files
%\usepackage{psfrag}
% need this for including graphics (\includegraphics)
%\usepackage{graphicx}
% for neatly defining theorems and propositions
\usepackage{amsthm}
% making logically defined graphics
%%%\usepackage{xypic}

% there are many more packages, add them here as you need them

% define commands here
\newtheorem*{theorem*}{Theorem}
\newtheorem*{lemma*}{Lemma}
\newtheorem*{corollary*}{Corollary}
\newtheorem{theorem}{Theorem}
\newtheorem{lemma}{Lemma}
\newtheorem{corollary}{Corollary}


\begin{document}
If $R$ is an integral domain with field of fractions $k$ and $S\subseteq R\setminus\{0\}$ is a multiplicative set, then the localization at $S$ is given by
\begin{equation*}
S^{-1}R=\left\{s^{-1}x:x\in R,s\in S\right\}
\end{equation*}
(up to isomorphism). This is a subring of $k$, and the following theorem states that localizations of Dedekind domains are again Dedekind domains.

\begin{theorem*}
Let $R$ be a Dedekind domain and $S\subseteq R\setminus\{0\}$ be a multiplicative set. Then $S^{-1}R$ is a Dedekind domain.
\end{theorem*}

A special case of this is the localization at a prime ideal $\mathfrak{p}$, which is defined as $R_\mathfrak{p}\equiv (R\setminus\mathfrak{p})^{-1}R$, and is therefore a Dedekind domain. In fact, if $\mathfrak{p}$ is nonzero then it can be shown that $R_\mathfrak{p}$ is a discrete valuation ring.

%%%%%
%%%%%
\end{document}
