\documentclass[12pt]{article}
\usepackage{pmmeta}
\pmcanonicalname{ProofThatAZeckendorfRepresentationRepresentsAUniquePositiveInteger}
\pmcreated{2013-03-22 16:36:46}
\pmmodified{2013-03-22 16:36:46}
\pmowner{PrimeFan}{13766}
\pmmodifier{PrimeFan}{13766}
\pmtitle{proof that a Zeckendorf representation represents a unique positive integer}
\pmrecord{8}{38810}
\pmprivacy{1}
\pmauthor{PrimeFan}{13766}
\pmtype{Proof}
\pmcomment{trigger rebuild}
\pmclassification{msc}{11B39}
\pmclassification{msc}{11A63}

% this is the default PlanetMath preamble.  as your knowledge
% of TeX increases, you will probably want to edit this, but
% it should be fine as is for beginners.

% almost certainly you want these
\usepackage{amssymb}
\usepackage{amsmath}
\usepackage{amsfonts}

% used for TeXing text within eps files
%\usepackage{psfrag}
% need this for including graphics (\includegraphics)
%\usepackage{graphicx}
% for neatly defining theorems and propositions
\usepackage{amsthm}
% making logically defined graphics
%%%\usepackage{xypic}

% there are many more packages, add them here as you need them

% define commands here

\begin{document}
{\bf Theorem.} For any positive integer $n$, the Zeckendorf representation $Z$ (with $k$ elements all 0s or 1s) of $n$ is unique.

For our proof, we accept it as axiomatic that $F_0 = F_1 = 1$ but $F_0$ is not used in the Zeckendorf representation of any number, and we also accept as axiomatic that all $F_i$ are distinct as long as $i > 0$.

\begin{proof}
Assume that there are two integers $a$ and $b$ such that $0 < a < b$, yet they both have the same Zeckendorf representation $Z$ with $k$ elements all 0s or 1s. We compute $$c = \sum_{i = 1}^k Z_iF_i,$$ where $F_x$ is the $x$th Fibonacci number. We can be assured that there is only one possible value for $c$ since all $F_i$ are distinct for $i > 0$ and each $F_i$ was added only once or not at all, since each $Z_i$ is limited by definition to 0 or 1. Now $c$ holds the value of the Zeckendorf representation $Z$. If $c = a$, it follows that $c < b$, but that would mean that $Z$ is not the Zeckendorf representation of $b$ after all, hence this results in a contradiction of our initial assumption. And if on the other hand $c = b$ and $c > a$, then this leads to a similar contradiction as to what is the Zeckendorf representation of $a$ really is.
\end{proof}
%%%%%
%%%%%
\end{document}
