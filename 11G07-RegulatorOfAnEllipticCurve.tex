\documentclass[12pt]{article}
\usepackage{pmmeta}
\pmcanonicalname{RegulatorOfAnEllipticCurve}
\pmcreated{2013-03-22 16:23:24}
\pmmodified{2013-03-22 16:23:24}
\pmowner{alozano}{2414}
\pmmodifier{alozano}{2414}
\pmtitle{regulator of an elliptic curve}
\pmrecord{8}{38535}
\pmprivacy{1}
\pmauthor{alozano}{2414}
\pmtype{Definition}
\pmcomment{trigger rebuild}
\pmclassification{msc}{11G07}
\pmclassification{msc}{11G05}
\pmclassification{msc}{14H52}
\pmrelated{CanonicalHeightOnAnEllipticCurve}
\pmrelated{BirchAndSwinnertonDyerConjecture}
\pmrelated{Regulator}
\pmdefines{elliptic regulator}
\pmdefines{height matrix}

\endmetadata

% this is the default PlanetMath preamble.  as your knowledge
% of TeX increases, you will probably want to edit this, but
% it should be fine as is for beginners.

% almost certainly you want these
\usepackage{amssymb}
\usepackage{amsmath}
\usepackage{amsthm}
\usepackage{amsfonts}

% used for TeXing text within eps files
%\usepackage{psfrag}
% need this for including graphics (\includegraphics)
%\usepackage{graphicx}
% for neatly defining theorems and propositions
%\usepackage{amsthm}
% making logically defined graphics
%%%\usepackage{xypic}

% there are many more packages, add them here as you need them

% define commands here

\newtheorem{thm}{Theorem}
\newtheorem{defn}{Definition}
\newtheorem{prop}{Proposition}
\newtheorem{lemma}{Lemma}
\newtheorem{cor}{Corollary}

\theoremstyle{definition}
\newtheorem{exa}{Example}

% Some sets
\newcommand{\Nats}{\mathbb{N}}
\newcommand{\Ints}{\mathbb{Z}}
\newcommand{\Reals}{\mathbb{R}}
\newcommand{\Complex}{\mathbb{C}}
\newcommand{\Rats}{\mathbb{Q}}
\newcommand{\Gal}{\operatorname{Gal}}
\newcommand{\Cl}{\operatorname{Cl}}
\begin{document}
Let $E/\Rats$ be an elliptic curve, let $E(\Rats)$ be the group of rational points on the curve and let $\langle \cdot, \cdot \rangle$ be the N\'eron-Tate pairing:
$$\langle P,Q \rangle=\hat{h}(P+Q)-\hat{h}(P)-\hat{h}(Q)$$
where $\hat{h}$ is the canonical height on the elliptic curve $E/\Rats$.

\begin{defn}
Let $E/\Rats$ be an elliptic curve and let $\{P_1,\ldots,P_r\}$ be a set of generators of the free part of $E(\Rats)$, i.e. the points $P_i$ generate $E(\Rats)$ modulo the torsion subgroup $E_{\operatorname{tors}}(\Rats)$. The {\bf height matrix} of $E/\Rats$ is the $r\times r$ matrix $H$ whose $ij$th component is $\langle P_i, P_j \rangle$, i.e.
$$H = (\langle P_i, P_j \rangle).$$
If $r=0$ then we define $H=1$.
\end{defn}

\begin{defn}
The {\bf \PMlinkescapetext{regulator}} of $E/\Rats$ (or the elliptic regulator), denoted by $\operatorname{Reg}(E/\Rats)$ or $R_{E/\Rats}$ is defined by
$$\operatorname{Reg}(E/\Rats)=\det(H)$$
where $H$ is the height matrix.
\end{defn}

Notice the similarities with the regulator of a number field. The regulator of an elliptic curve is the volume of a fundamental domain for $E(\Rats)$ modulo torsion, with respect to the quadratic form defined by the N\'eron-Tate pairing.
%%%%%
%%%%%
\end{document}
