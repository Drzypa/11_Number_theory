\documentclass[12pt]{article}
\usepackage{pmmeta}
\pmcanonicalname{KnodelNumber}
\pmcreated{2013-03-22 16:06:54}
\pmmodified{2013-03-22 16:06:54}
\pmowner{PrimeFan}{13766}
\pmmodifier{PrimeFan}{13766}
\pmtitle{Kn\"odel number}
\pmrecord{6}{38181}
\pmprivacy{1}
\pmauthor{PrimeFan}{13766}
\pmtype{Definition}
\pmcomment{trigger rebuild}
\pmclassification{msc}{11A51}
\pmsynonym{Knodel number}{KnodelNumber}

\endmetadata

% this is the default PlanetMath preamble.  as your knowledge
% of TeX increases, you will probably want to edit this, but
% it should be fine as is for beginners.

% almost certainly you want these
\usepackage{amssymb}
\usepackage{amsmath}
\usepackage{amsfonts}

% used for TeXing text within eps files
%\usepackage{psfrag}
% need this for including graphics (\includegraphics)
%\usepackage{graphicx}
% for neatly defining theorems and propositions
%\usepackage{amsthm}
% making logically defined graphics
%%%\usepackage{xypic}

% there are many more packages, add them here as you need them

% define commands here

\begin{document}
The {\em Kn\"odel numbers} $K_n$ for a given positive integer $n$ are the set of composite integers $m > n$ such that any $b < m$ coprime to $m$ satisfies $b^{m - n} \equiv 1 \mod m$. The Carmichael numbers are $K_1$. There are infinitely many Knodel number $K_n$ for a given $n$, something which was first proven only for $n > 2$. Erd\H{o}s speculated that this was also true for $n = 1$ but two decades passed before this was conclusively proved by Alford, Granville and Pomerance.

\begin{thebibliography}{5}
\bibitem{wa} W. R. Alford, A. Granville, and C. Pomerance. ``There are Infinitely Many Carmichael Numbers'' {\it Annals of Mathematics} {\bf 139} (1994): 703 - 722 
\bibitem{pr} P. Ribenboim, {\it The Little Book of Bigger Primes}, (2004), New York: Springer-Verlag, p. 102.
\end{thebibliography}
%%%%%
%%%%%
\end{document}
