\documentclass[12pt]{article}
\usepackage{pmmeta}
\pmcanonicalname{AllAlgebraicNumbersInASequence}
\pmcreated{2013-03-22 15:13:58}
\pmmodified{2013-03-22 15:13:58}
\pmowner{pahio}{2872}
\pmmodifier{pahio}{2872}
\pmtitle{all algebraic numbers in a sequence}
\pmrecord{11}{37003}
\pmprivacy{1}
\pmauthor{pahio}{2872}
\pmtype{Result}
\pmcomment{trigger rebuild}
\pmclassification{msc}{11R04}
\pmclassification{msc}{03E10}
\pmsynonym{counting the algebraic numbers}{AllAlgebraicNumbersInASequence}
\pmrelated{FieldOfAlgebraicNumbers}

% this is the default PlanetMath preamble.  as your knowledge
% of TeX increases, you will probably want to edit this, but
% it should be fine as is for beginners.

% almost certainly you want these
\usepackage{amssymb}
\usepackage{amsmath}
\usepackage{amsfonts}

% used for TeXing text within eps files
%\usepackage{psfrag}
% need this for including graphics (\includegraphics)
%\usepackage{graphicx}
% for neatly defining theorems and propositions
 \usepackage{amsthm}
% making logically defined graphics
%%%\usepackage{xypic}

% there are many more packages, add them here as you need them

% define commands here

\theoremstyle{definition}
\newtheorem*{thmplain}{Theorem}
\begin{document}
The beginning of the sequence of all algebraic numbers ordered as explained in the \PMlinkname{parent}{AlgebraicNumbersAreCountable} entry is as follows:

$0;\,\,-1,\,1;\,\,-2,\,-\frac{1}{2},\,-i,\,i,\,\frac{1}{2},\,2;\,\,-3,\,
\frac{-1-\sqrt{5}}{2},\,-\sqrt{2},\,
-\frac{1}{\sqrt{2}},\,\frac{1-\sqrt{5}}{2},\,\frac{-1-i\sqrt{3}}{2},
\,\frac{-1+i\sqrt{3}}{2},\,-\frac{1}{3},\,$

$-i\sqrt{2},\,-\frac{i}{\sqrt{2}},\,\frac{i}{\sqrt{2}},\,
i\sqrt{2},\,\frac{1}{3},\,\frac{1-i\sqrt{3}}{2},\,
\frac{1+i\sqrt{3}}{2},\,\frac{-1+\sqrt{5}}{2},\,\frac{1}{\sqrt{2}},\,\sqrt{2},\,
\frac{1+\sqrt{5}}{2},\,3;\,\ldots$

The first number corresponds to the algebraic equation \,$x = 0$,\, the two following numbers to the equations \,$x\pm 1 =0$,\, the six following to the equations \,$x\pm 2 = 0$,\, $2x\pm 1 = 0$,\, $x^2+1 = 0$,\, the twenty following to the equations \,$x\pm 3 = 0$,\, $3x\pm 1 = 0$,\, $x^2\pm x \pm 1 = 0$,\, $x^2\pm 2 = 0$,\, $2x^2\pm 1 = 0$.

In practice, one cannot continue the sequence very far since the higher degree equations -- quintic and so on -- are non-solvable by \PMlinkname{radicals}{NthRoot}; instead we can list the equations satisfied by the numbers as far we want and tell how many \PMlinkname{roots}{Root} they have. \,In principle, the number sequence does exist!
%%%%%
%%%%%
\end{document}
