\documentclass[12pt]{article}
\usepackage{pmmeta}
\pmcanonicalname{AntiharmonicNumber}
\pmcreated{2013-11-28 10:15:29}
\pmmodified{2013-11-28 10:15:29}
\pmowner{pahio}{2872}
\pmmodifier{pahio}{2872}
\pmtitle{antiharmonic number}
\pmrecord{10}{87981}
\pmprivacy{1}
\pmauthor{pahio}{2872}
\pmtype{Definition}
\pmclassification{msc}{11A05}
\pmclassification{msc}{11A25}

% this is the default PlanetMath preamble.  as your knowledge
% of TeX increases, you will probably want to edit this, but
% it should be fine as is for beginners.

% almost certainly you want these
\usepackage{amssymb}
\usepackage{amsmath}
\usepackage{amsfonts}

% need this for including graphics (\includegraphics)
\usepackage{graphicx}
% for neatly defining theorems and propositions
\usepackage{amsthm}

% making logically defined graphics
%\usepackage{xypic}
% used for TeXing text within eps files
%\usepackage{psfrag}

% there are many more packages, add them here as you need them

% define commands here

\begin{document}
The antiharmonic, a.k.a. contraharmonic mean of some set of 
positive numbers is defined as the sum of their squares 
divided by their sum.\, There exist positive integers $n$ 
whose sum $\sigma_1(n)$ of all their positive divisors divides 
the sum $\sigma_2(n)$ of the squares of those divisors.\, For 
example, 4 is such an integer:
$$1+2+4 \,=\, 7 \,\mid\, 21 \,=\, 1^2+2^2+4^2$$
Such integers are called 
{\it antiharmonic numbers} (or {\it contraharmonic numbers}), 
since the contraharmonic mean of their positive divisors is an 
integer.
 
The antiharmonic numbers form the
\PMlinkexternal{OEIS}{HTTP://oeis.org/} integer sequence
\PMlinkexternal{A020487}{http://oeis.org/search?q=A020487&language=english&go=Search}:
$$1,\,4,\,9,\,16,\,20,\,25,\,36,\,49,\,50,\,64,\,81,\,100,\,117,\,121,\,144,\,169,\,180,\,\ldots$$\\


Using the expressions of \PMlinkname{divisor function}{DivisorFunction} 
$\sigma_z(n)$, the condition for an 
integer $n$ to be an antiharmonic number, is that the quotient
$$\sigma_2(n):\sigma_1(n) 
\;=\; \sum_{0 < d \mid n}\!d^2:\!\sum_{0 < d \mid n}\!d
\;=\; \prod_{i=1}^k\frac{p_i^{2(m_i+1)}-1}{p_i^2-1}
     :\prod_{i=1}^k\frac{p_i^{m_i+1}-1}{p_i-1}$$
is an integer; here the $p_i$'s are the distinct prime divisors 
of $n$ and $m_i$'s their multiplicities.\, The last form is 
simplified to
\begin{align}
\prod_{i=1}^k\frac{p_i^{m_i+1}+1}{p_i+1}.
\end{align}
The OEIS sequence A020487 contains all nonzero perfect squares, 
since in the case of such numbers the antiharmonic mean (1) of 
the divisors has the form
$$\prod_{i=1}^k\frac{p_i^{2m_i+1}+1}{p_i+1} \;=\;
\prod_{i=1}^k\left(p_i^{2m_i}-p_i^{2m_i-1}-\!+\ldots-p_i+1\right)$$
(cf. irreducibility of binomials with unity coefficients).\\




\textbf{Note.}\, It would in a manner be legitimated to define 
a positive integer to be an antiharmonic number (or an 
antiharmonic integer) if it is the antiharmonic mean of two 
distinct positive integers; see integer contraharmonic mean 
and \PMlinkname{contraharmonic Diophantine equation}
{ContraharmonicDiophantineEquation}.

\end{document}
