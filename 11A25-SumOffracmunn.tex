\documentclass[12pt]{article}
\usepackage{pmmeta}
\pmcanonicalname{SumOffracmunn}
\pmcreated{2013-03-22 14:25:46}
\pmmodified{2013-03-22 14:25:46}
\pmowner{mathcam}{2727}
\pmmodifier{mathcam}{2727}
\pmtitle{sum of $\frac{\mu(n)}{n}$}
\pmrecord{13}{35938}
\pmprivacy{1}
\pmauthor{mathcam}{2727}
\pmtype{Result}
\pmcomment{trigger rebuild}
\pmclassification{msc}{11A25}
\pmrelated{MoebiusFunction}

\endmetadata

% this is the default PlanetMath preamble.  as your knowledge
% of TeX increases, you will probably want to edit this, but
% it should be fine as is for beginners.

% almost certainly you want these
\usepackage{amssymb}
\usepackage{amsmath}
\usepackage{amsfonts}

% used for TeXing text within eps files
%\usepackage{psfrag}
% need this for including graphics (\includegraphics)
%\usepackage{graphicx}
% for neatly defining theorems and propositions
%\usepackage{amsthm}
% making logically defined graphics
%%%\usepackage{xypic}

% there are many more packages, add them here as you need them

% define commands here
\begin{document}
The following result holds:

\[ \sum_{n=1}^{\infty} \frac{\mu(n)}{n} = 0 \]

where $\mu(n)$ is the \PMlinkname{M\"obius function}{MoebiusFunction}.

{\bf Proof:}\\
Let $\sum_{n=1}^{\infty} \frac{\mu(n)}{n} = \alpha$.  Assume $\alpha \neq 0$.

For $\operatorname{Re}(s) > 1$ we have the Euler product expansion

\[ \frac{1}{\zeta(s)} = \sum_{n=1}^\infty \frac{\mu(n)}{n^s} \]

where $\zeta(s)$ is the Riemann zeta function.  

We recall the following properties of the Riemann zeta function (which can be found in the PlanetMath entry \PMlinkname{Riemann Zeta Function}{RiemannZetaFunction}).
\begin{itemize}
\item $\zeta(s)$ is analytic except at the point $s=1$ where it has a simple pole with residue $1$.
\item $\zeta(s)$ has no zeroes in the region $\operatorname{Re}(s) \geq 1$.  
\item The function $(s-1) \zeta(s)$ is analytic and nonzero for $\operatorname{Re}(s) \geq 1$.
\item Therefore, the function $\frac{1}{\zeta(s)}$ is analytic for $\operatorname{Re}(s) \geq 1$.
\end{itemize}

Further, as a corollary of the proof of the prime number theorem, we also know that this sum, $\sum_{n=1}^\infty \frac{\mu(n)}{n^s}$ converges to $\frac{1}{\zeta(s)}$ for $\operatorname{Re}(s) \geq 1$; in particular, it converges \emph{at} $s=1$).

But then
 \[ \zeta(1) = \frac{1}{\sum_{n=1}^\infty \frac{\mu(n)}{n}} = \frac{1}{\alpha} \]

So $\zeta(1)=\frac{1}{\alpha}$, but this is a contradiction since $\zeta$ has a simple pole at $s=1$.  Therefore $\alpha = 0$.
%%%%%
%%%%%
\end{document}
