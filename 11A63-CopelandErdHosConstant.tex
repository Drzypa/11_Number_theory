\documentclass[12pt]{article}
\usepackage{pmmeta}
\pmcanonicalname{CopelandErdHosConstant}
\pmcreated{2013-03-22 15:55:06}
\pmmodified{2013-03-22 15:55:06}
\pmowner{Mravinci}{12996}
\pmmodifier{Mravinci}{12996}
\pmtitle{Copeland-Erd\H{o}s constant}
\pmrecord{10}{37923}
\pmprivacy{1}
\pmauthor{Mravinci}{12996}
\pmtype{Definition}
\pmcomment{trigger rebuild}
\pmclassification{msc}{11A63}
\pmsynonym{Copeland-Erdos constant}{CopelandErdHosConstant}
\pmsynonym{Copeland-Erd\"os constant}{CopelandErdHosConstant}
\pmsynonym{Copeland Erdos constant}{CopelandErdHosConstant}
\pmsynonym{Copeland Erd\"os constant}{CopelandErdHosConstant}

% this is the default PlanetMath preamble.  as your knowledge
% of TeX increases, you will probably want to edit this, but
% it should be fine as is for beginners.

% almost certainly you want these
\usepackage{amssymb}
\usepackage{amsmath}
\usepackage{amsfonts}

% used for TeXing text within eps files
%\usepackage{psfrag}
% need this for including graphics (\includegraphics)
%\usepackage{graphicx}
% for neatly defining theorems and propositions
%\usepackage{amsthm}
% making logically defined graphics
%%%\usepackage{xypic}

% there are many more packages, add them here as you need them

% define commands here

\begin{document}
The \emph{Copeland-Erd\H{o}s constant} is the concatenation of ``0.'' with the base 10 representations of the prime numbers in order. Its value is approximately 0.235711131719232931374143... listed in A033308 of Sloane's OEIS.

The larger Smarandache-wellin numbers approximate the value of this constant multiplied by the appropriate power of 10.

Its continued fraction is

$0 + \frac{1}{4 + \frac{1}{4 + \frac{1}{8+\,\cdots}}}$

and is listed in A030168 of the OEIS.

In base 10, this is a normal number, a fact proven by A.H. Copeland and Paul Erd\H{o}s in 1946 (hence the name of the constant).
%%%%%
%%%%%
\end{document}
