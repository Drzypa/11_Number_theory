\documentclass[12pt]{article}
\usepackage{pmmeta}
\pmcanonicalname{ConvolutionMethod}
\pmcreated{2013-03-22 15:58:24}
\pmmodified{2013-03-22 15:58:24}
\pmowner{Wkbj79}{1863}
\pmmodifier{Wkbj79}{1863}
\pmtitle{convolution method}
\pmrecord{13}{37989}
\pmprivacy{1}
\pmauthor{Wkbj79}{1863}
\pmtype{Definition}
\pmcomment{trigger rebuild}
\pmclassification{msc}{11N37}
\pmrelated{DirichletHyperbolaMethod}
\pmrelated{MoebiusFunction}
\pmrelated{DisplaystyleYOmeganOleftFracxlogXy12YRightFor1LeY2}
\pmrelated{DisplaystyleXlog2xOleftsum_nLeX2OmeganRight}

\endmetadata

% this is the default PlanetMath preamble.  as your knowledge
% of TeX increases, you will probably want to edit this, but
% it should be fine as is for beginners.

% almost certainly you want these
\usepackage{amssymb}
\usepackage{amsmath}
\usepackage{amsfonts}

% used for TeXing text within eps files
%\usepackage{psfrag}
% need this for including graphics (\includegraphics)
%\usepackage{graphicx}
% for neatly defining theorems and propositions
%\usepackage{amsthm}
% making logically defined graphics
%%%\usepackage{xypic}

% there are many more packages, add them here as you need them

% define commands here

\begin{document}
Let $f$, $g$, and $h$ be multiplicative functions such that $f=g*h$, where $*$ denotes the \PMlinkname{convolution}{MultiplicativeFunction} of $g$ and $h$.  The {\sl convolution method\/} is a way to \PMlinkescapetext{calculate} $\displaystyle \sum_{n \le x} f(n)$ by using the fact that $f=g*h$:

\begin{center}
$\begin{array}{ll}
\displaystyle \sum_{n \le x} f(n) & \displaystyle = \sum_{n \le x} \sum_{d|n} g(d) h \left( \frac{n}{d} \right) \\
& \displaystyle = \sum_{d \le x} \sum_{m \le \frac{x}{d}} g(d) h(m) \\
& \displaystyle = \sum_{d \le x} g(d) \sum_{m \le \frac{x}{d}} h(m)
\end{array}$
\end{center}

This method for calculating $\displaystyle \sum_{n \le x} f(n)$ is advantageous when the sums in \PMlinkescapetext{terms} of $g$ and $h$ are easier to handle.

As an example, the sum $\displaystyle \sum_{n \le x} \mu^2(n)$ will be calculated using the convolution method.

Since $\mu^2=\mu^2*\varepsilon=\mu^2*(\mu*1)=(\mu^2*\mu)*1$, the functions $\mu^2*\mu$ and $1$ can be used.

To use the convolution method, a nice way to \PMlinkescapetext{calculate} $(\mu^2*\mu)(n)$ needs to be found.  Note that $\mu^2*\mu$ is \PMlinkname{multiplicative}{MultiplicativeFunction}, so it only needs to be evaluated at prime powers.

Let $k \in \mathbb{N}$.  Then

$$ (\mu^2*\mu)(p^k) = \sum_{j=0}^k \mu^2(p^j) \mu(p^{k-j}) = \mu(p^k)+\mu(p^{k-1}) = \begin{cases}
0 & \text{if } k \neq 2 \\
-1 & \text{if } k=2. \end{cases}$$

Since $\mu^2*\mu$ is \PMlinkname{multiplicative}{MultiplicativeFunction}, then $(\mu^2*\mu)(n)= \begin{cases}
0 & \text{if } n \text{ is not a square} \\
\mu(\sqrt{n}) & \text{if } n \text{ is a square.} \end{cases}$

The convolution method yields:

\begin{center}
$\begin{array}{ll}
\displaystyle \sum_{n \le x} \mu^2(n) & \displaystyle = \sum_{d \le x} (\mu^2*\mu)(d) \sum_{m \le \frac{x}{d}} 1 \\
& \\
& \displaystyle = \sum_{d \le x} (\mu^2*\mu)(d) \left( \frac{x}{d} + O(1) \right) \\
& \\
& \displaystyle = \sum_{a \le \sqrt{x}} \mu(a) \left( \frac{x}{a^2} + O(1) \right) \\
& \\
& \displaystyle = x \sum_{a \le \sqrt{x}} \frac{\mu(a)}{a^2} + O \left( \sum_{a \le \sqrt{x}} |\mu(a)| \right) \\
& \\
& \displaystyle = x \left( \sum_{a \in \mathbb{N}} \frac{\mu(a)}{a^2} - \sum_{a>\sqrt{x}} \frac{\mu(a)}{a^2} \right) + O \left( \sum_{a \le \sqrt{x}} 1 \right) \\
& \\
& \displaystyle = x \sum_{a \in \mathbb{N}} \frac{\mu(a)}{a^2} + O \left( x \sum_{a>\sqrt{x}} \left| \frac{\mu(a)}{a^2} \right| \right) + O(\sqrt{x}) \\
& \\
& \displaystyle = \frac{6}{\pi^2}x + O \left( x \sum_{a>\sqrt{x}} \frac{1}{a^2} \right) + O(\sqrt{x}) \\
& \\
& \displaystyle = \frac{6}{\pi^2}x + O \left( x \left( \frac{1}{\sqrt{x}} \right) \right) + O(\sqrt{x}) \\
& \\
& \displaystyle = \frac{6}{\pi^2}x + O(\sqrt{x}) \end{array}$
\end{center}
%%%%%
%%%%%
\end{document}
