\documentclass[12pt]{article}
\usepackage{pmmeta}
\pmcanonicalname{ProofOfValuesOfTheRiemannZetaFunctionInTermsOfBernoulliNumbers}
\pmcreated{2013-03-22 17:46:37}
\pmmodified{2013-03-22 17:46:37}
\pmowner{rm50}{10146}
\pmmodifier{rm50}{10146}
\pmtitle{proof of values of the Riemann zeta function in terms of Bernoulli numbers}
\pmrecord{5}{40235}
\pmprivacy{1}
\pmauthor{rm50}{10146}
\pmtype{Proof}
\pmcomment{trigger rebuild}
\pmclassification{msc}{11M99}

\endmetadata

% this is the default PlanetMath preamble.  as your knowledge
% of TeX increases, you will probably want to edit this, but
% it should be fine as is for beginners.

% almost certainly you want these
\usepackage{amssymb}
\usepackage{amsmath}
\usepackage{amsfonts}

% used for TeXing text within eps files
%\usepackage{psfrag}
% need this for including graphics (\includegraphics)
%\usepackage{graphicx}
% for neatly defining theorems and propositions
\usepackage{amsthm}
% making logically defined graphics
%%%\usepackage{xypic}

% there are many more packages, add them here as you need them

% define commands here
\newtheorem{thm}{Theorem}
\begin{document}
This article proves part of the theorem given in the \PMlinkescapetext{parent} article.
\begin{thm} For any positive integer $n$
\[\zeta(2n)=\frac{(2\pi)^{2n}\lvert B_{2n}\rvert}{2(2n)!}\]
where $B_{2n}$ is the $2n^{\mathrm{th}}$ Bernoulli number.
\end{thm}

\textbf{Proof.}
The method is as follows. Using Fourier series together with induction on $n$, we derive a formula for the Bernoulli periodic function $B_{2n}(x)$ involving an infinite sum. On setting $x$ to $0$, this sum reduces to a constant times the appropriate zeta function, and the result follows.

We first compute the Fourier series for $B_2(x)$. $B_2(x)$ is periodic with period $1$, so
\[c_n=\int_0^1B_2(x)e^{-2\pi i n x}dx = \int_0^1x^2e^{-2\pi i n x}dx - \int_0^1 xe^{-2\pi i n x}dx + \frac{1}{6}\int_0^1 e^{-2\pi i n x}dx\]

We have
\begin{gather*}
\int_0^1 e^{-2\pi i n x}dx=0\\
\int_0^1 xe^{-2\pi i n x}dx = \frac{-1}{2\pi n}xe^{-2\pi i n x}\big\lvert_0^1 + \frac{1}{2\pi i n}\int_0^1e^{-2\pi i n x}dx = \frac{i}{2\pi n}\\
\int_0^1 x^2e^{-2\pi i n x} dx = \frac{-1}{2\pi i n}x^2e^{-2\pi i n x}\big\lvert_0^1 + \frac{2}{2\pi i n}\int_0^1 xe^{-2\pi i n x}dx = \frac{1}{2\pi^2n^2}+\frac{i}{2\pi n}
\end{gather*}
so that
\[c_n = \frac{1}{2\pi^2n^2}\]
But then $b_n=c_n-c_{-n}=0$ for all $n$, $a_0=0$, and for $n>0$, $\displaystyle a_n=c_n+c_{-n}=\frac{1}{\pi^2n^2}$ (where $a_n$ are the coefficients of $\cos$ and $b_n$ the coefficients of $\sin$ in the Fourier series). Thus
\[B_2(x)=\sum_{k=1}^{\infty}\frac{1}{\pi^2k^2}\cos(2\pi k x)=\frac{1}{\pi^2}\sum_{k=1}^{\infty}\frac{1}{k^2}\cos(2\pi k x)\]

Using this case as an inductive hypothesis, assume that for some $n\geq 2$
\[B_{2(n-1)}(x)=\frac{(-1)^n2\cdot(2(n-1))!}{(2\pi)^{2(n-1)}}\sum_{k=1}^{\infty}\frac{1}{k^{2(n-1)}}\cos(2\pi k x)\]
Then on $(0,1)$
\[B_{2n}''(x)=(2n)(2n-1)B_{2(n-1)}(x) = \frac{(-1)^n2\cdot(2n)!}{(2\pi)^{2(n-1)}}\sum_{k=1}^{\infty}\frac{1}{k^{2(n-1)}}\cos(2\pi k x)\]
and thus
\[B_{2n}(x) = \frac{(-1)^n2\cdot(2n)!}{(2\pi)^{2(n-1)}}\int\int\sum_{k=1}^{\infty}\frac{1}{k^{2(n-1)}}\cos(2\pi k x)dx dx\]
Since $n\geq 2$, the sum converges absolutely, so we can move the sum outside the integrals, and we get
\begin{align*}
B_{2n}(x) &= \frac{(-1)^n2\cdot(2n)!}{(2\pi)^{2(n-1)}}\sum_{k=1}^{\infty}\frac{1}{k^{2(n-1)}}\int\int\cos(2\pi k x)dx dx\\
&=\frac{(-1)^n2\cdot(2n)!}{(2\pi)^{2(n-1)}}\sum_{k=1}^{\infty}\frac{1}{k^{2(n-1)}}\frac{-1}{4\pi^2k^2}\cos(2\pi k x)\\
&=\frac{(-1)^{n+1}2\cdot(2n)!}{(2\pi)^{2n}}\sum_{k=1}^{\infty}\frac{1}{k^{2n}}\cos(2\pi k x)
\end{align*}
Thus we have established this formula for all $n\geq 1$. Setting $x=0$, then, we get
\[B_{2n} = \frac{(-1)^{n+1}2\cdot(2n)!}{(2\pi)^{2n}}\sum_{k=1}^{\infty}\frac{1}{k^{2n}}=\frac{(-1)^{n+1}2\cdot(2n)!}{(2\pi)^{2n}}\zeta(2n)\]
or, trivially rewriting,
\[\zeta(2n) = \frac{(-1)^{n+1}(2\pi)^{2n}B_{2n}}{2(2n)!}\]
But clearly $\zeta(2n)>0$ for $n\geq 1$, so it must be that the $B_{2n}$ alternate in sign, and thus
\[\zeta(2n) = \frac{(2\pi)^{2n}\lvert B_{2n}\rvert}{2(2n)!}\]

Note that as a \PMlinkescapetext{side} effect of this proof, we see that the even-index Bernoulli numbers alternate in sign!
%%%%%
%%%%%
\end{document}
