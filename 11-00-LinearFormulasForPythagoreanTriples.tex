\documentclass[12pt]{article}
\usepackage{pmmeta}
\pmcanonicalname{LinearFormulasForPythagoreanTriples}
\pmcreated{2014-12-22 21:59:51}
\pmmodified{2014-12-22 21:59:51}
\pmowner{pahio}{2872}
\pmmodifier{pahio}{2872}
\pmtitle{linear formulas for Pythagorean triples}
\pmrecord{13}{42140}
\pmprivacy{1}
\pmauthor{pahio}{2872}
\pmtype{Result}
\pmcomment{trigger rebuild}
\pmclassification{msc}{11-00}
\pmrelated{DerivationOfPythagoreanTriples}
\pmrelated{ContraharmonicMeansAndPythagoreanHypotenuses}
\pmrelated{DeterminingIntegerContraharmonicMeans}

\endmetadata

% this is the default PlanetMath preamble.  as your knowledge
% of TeX increases, you will probably want to edit this, but
% it should be fine as is for beginners.

% almost certainly you want these
\usepackage{amssymb}
\usepackage{amsmath}
\usepackage{amsfonts}

% used for TeXing text within eps files
%\usepackage{psfrag}
% need this for including graphics (\includegraphics)
%\usepackage{graphicx}
% for neatly defining theorems and propositions
 \usepackage{amsthm}
% making logically defined graphics
%%%\usepackage{xypic}
\usepackage{pstricks}
\usepackage{pst-plot}

% there are many more packages, add them here as you need them

% define commands here

\theoremstyle{definition}
\newtheorem*{thmplain}{Theorem}

\begin{document}
It is easy to see that the equation
\begin{align}
a^2\!+\!b^2 \;=\; c^2
\end{align}
of the \PMlinkname{Pythagorean theorem}{PythagorasTheorem} is 
\PMlinkname{equivalent}{Equivalent3} with
\begin{align}
(a\!+\!b\!-\!c)^2 \;=\; 2(c\!-\!a)(c\!-\!b).
\end{align}



\begin{center}
\begin{pspicture}(-1,-1)(9,9.5)
\pspolygon(0,0)(8,0)(8,8)(0,8)
\psline(0,6.5)(6.5,6.5)(6.5,0)
\psline(3.5,8)(3.5,3.5)(8,3.5)
\rput(3.4,-0.2){$a$}
\rput(-0.2,3.4){$a$}
\rput(-0.5,7.2){$c\!-\!a$}
\rput(5.7,8.3){$b$}
\rput(8.3,5.5){$b$}
\rput(7.2,-0.2){$c\!-\!a$}
\rput(8.5,1.7){$c\!-\!b$}
\rput(5,3.15){$a\!+\!b\!-\!c$}
\rput(2.6,4.85){$a\!+\!b\!-\!c$}
\rput(1.67,8.3){$c\!-\!b$}
\end{pspicture}
\end{center}

When\, $(a,\,b,\,c)$\, is a Pythagorean triple, i.e. $a$, $b$, $c$ 
are positive integers, $a\!+\!b\!-\!c$ must be an even positive 
integer which we denote by $2r$.\, We get from (2) the equation
$$(c\!-\!a)(c\!-\!b) \;=\; 2r^2,$$
whose \PMlinkname{factors}{Product} on the left hand side we denote 
by $t$ and $s$.\, Thus we have the linear equation system
\begin{align*}
\begin{cases}
a\!+\!b\!-\!c \;=\; 2r, \\
c\!-\!a \;=\; t, \\
c\!-\!b \;=\; s. \\
\end{cases}
\end{align*}
Its solution is
\begin{align}
\begin{cases}
a \;=\; 2r\!+\!s, \\
b \;=\; 2r\!+\!t, \\
c \;=\; 2r\!+\!s\!+\!t.
\end{cases}
\end{align}
Here, $r$ is an arbitrary positive integer, $s$ and $t$ are two 
positive integers whose product is $2r^2$.\, It's clear that then 
(3) produces all Pythagorean triples.

\begin{thebibliography}{8}
\bibitem{ES}{\sc Egon Scheffold}: ``Ein Bild der pythagoreischen 
Zahlentripel''.\, -- \emph{Elemente der Mathematik} \textbf{50} 
(1995).
\end{thebibliography}

%%%%%
%%%%%
\end{document}
