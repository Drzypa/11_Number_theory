\documentclass[12pt]{article}
\usepackage{pmmeta}
\pmcanonicalname{GoldbachsConjecture}
\pmcreated{2013-03-22 12:13:43}
\pmmodified{2013-03-22 12:13:43}
\pmowner{drini}{3}
\pmmodifier{drini}{3}
\pmtitle{Goldbach's conjecture}
\pmrecord{11}{31602}
\pmprivacy{1}
\pmauthor{drini}{3}
\pmtype{Conjecture}
\pmcomment{trigger rebuild}
\pmclassification{msc}{11P32}
\pmclassification{msc}{11-00}
\pmrelated{Prime}


\begin{document}
The conjecture states that every even integer $n>2$ is expressible as the sum of two primes.

In 1966 Chen proved that every sufficiently large even number can be expressed as the sum of a prime and a number with at most two prime divisors.

Vinogradov proved that every sufficiently large \emph{odd} number is a sum of three primes. In 1997 it was shown by J.-M. Deshouillers, G. Effinger, H. Te Riele, and D. Zinoviev that, assuming a generalized Riemann hypothesis, every odd number $n>5$ can be represented as sum of three primes.

The conjecture was first proposed in a 1742 letter from Christian Goldbach to Euler and still remains unproved.
%%%%%
%%%%%
%%%%%
\end{document}
