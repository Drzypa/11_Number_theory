\documentclass[12pt]{article}
\usepackage{pmmeta}
\pmcanonicalname{ProofOfLucasLehmerPrimalityTest}
\pmcreated{2013-03-22 18:42:16}
\pmmodified{2013-03-22 18:42:16}
\pmowner{rm50}{10146}
\pmmodifier{rm50}{10146}
\pmtitle{proof of Lucas-Lehmer primality test}
\pmrecord{5}{41468}
\pmprivacy{1}
\pmauthor{rm50}{10146}
\pmtype{Proof}
\pmcomment{trigger rebuild}
\pmclassification{msc}{11A51}

\endmetadata

% this is the default PlanetMath preamble.  as your knowledge
% of TeX increases, you will probably want to edit this, but
% it should be fine as is for beginners.

% almost certainly you want these
\usepackage{amssymb}
\usepackage{amsmath}
\usepackage{amsfonts}

% used for TeXing text within eps files
%\usepackage{psfrag}
% need this for including graphics (\includegraphics)
%\usepackage{graphicx}
% for neatly defining theorems and propositions
\usepackage{amsthm}
% making logically defined graphics
%%%\usepackage{xypic}

% there are many more packages, add them here as you need them

% define commands here
\newcommand{\Leg}[2]{\left(\frac{#1}{#2}\right)}
\newtheorem{thm}{Theorem}
\newtheorem{cor}[thm]{Corollary}
\newtheorem{lem}[thm]{Lemma}

\begin{document}
The objective of this article is to prove the Lucas-Lehmer primality test:
\newline
Let $p>2$ be a prime, and let $M_p=2^p-1$ be the corresponding Mersenne number. Then $M_p$ is prime if and only if $M_p$ divides $s_{p-1}$ (equivalently, if and only if $s_{p-1}\equiv 0\pod {M_p}$) where the numbers $(s_n)_{n\geq1}$ are given by the following recurrence relation: 
\begin{align*}
  s_1 &= 4 \\
  s_{n+1} &= {s_n}^2-2,\quad n\geq 1
\end{align*}

We show that the validity of the primality test is equivalent to the following theorem, which is then proved directly:
\begin{thm} \label{thm:one}(Lucas) $M_p$ is prime if and only if $\alpha^{(M_p+1)/2}\equiv -1\pod{M_p}$, where $\alpha = 2+\sqrt{3}$.
\end{thm}

To see that the two are in fact equivalent, let $\beta=2-\sqrt{3}$. Then $\alpha+\beta=4,\ \alpha\beta=1$. Thus
\begin{align*}
  s_1 &= \alpha+\beta\\
  s_2 &= (\alpha+\beta)^2 - 2 = \alpha^2 + \beta^2 + 2\alpha\beta-2 = \alpha^2+\beta^2\\
  s_3 &= \alpha^4 + \beta^4\\
  \cdots &\\
  s_{p-1} &= \alpha^{2^{p-2}}+\beta^{2^{p-2}}
\end{align*}
Note that $2^{p-2} = \frac{M_p+1}{4}$. Then
\begin{align*}
  s_{p-1}\equiv 0\pod{M_p} & \iff \alpha^{(M_p+1)/4}+\beta^{(M_p+1)/4} \equiv 0\pod {M_p}\\
    &\iff \alpha^{(M_p+1)/2}+(\alpha\beta)^{(M_p+1)/4} \equiv 0\pod {M_p}\\
    &\iff \alpha^{(M_p+1)/2}\equiv -1\pod {M_p}
\end{align*}

It thus remains to prove Theorem \ref{thm:one}. We start with two simple lemmas:
\begin{lem} \label{lem:two}If $p>3$ is prime, then $\alpha^{p-1}\equiv 1\pod p$ or $\alpha^{p+1}\equiv 1\pod p$.
\end{lem}
\begin{proof}
\[
  \alpha^p \equiv 2^p + 3^{(p-1)/2}\sqrt{3} \equiv \begin{cases}\alpha\pod p & \text{if }\Leg{3}{p}=1\\
  \beta \pod p & \text{if }\Leg{3}{p}=-1\end{cases}
\]
where $\Leg{\cdot}{\cdot}$ is the Legendre symbol. Thus
\begin{gather*}
    \Leg{3}{p} = 1 \ \Rightarrow\ \alpha^{p-1} = \alpha^p\alpha^{-1}
      = \alpha^p\beta \equiv \alpha\beta = 1\pod p\\
  \Leg{3}{p} = -1 \ \Rightarrow\ \alpha^{p+1} = \alpha^p\alpha \equiv \beta\alpha = 1\pod p
\end{gather*}
\end{proof}

\begin{lem} \label{lem:three}Let $p$ be a prime with $p\equiv 7\pod 8$ and $p\equiv 7\pod {12}$. Then $\alpha^{(p+1)/2}\equiv -1\pod p$.
\end{lem}
\begin{proof}
$(1+\sqrt{3})^2 = 4+2\sqrt{3} = 2\alpha$, so that
\[
  (1+\sqrt{3})^{p+1} = 2^{(p+1)/2}\alpha^{(p+1)/2}
\]
But $p\equiv 7\pod 8$, so that $\Leg{2}{p}=1$. Thus $2^{(p+1)/2} \equiv 2\cdot 2^{(p-1)/2} \equiv 2 \pod p$ and therefore
\[
  (1+\sqrt{3})^{p+1} \equiv 2\alpha^{(p+1)/2}\pod p
\]
Also,
\[
  (1+\sqrt{3})^{p+1} = (1+\sqrt{3})(1+\sqrt{3})^p \equiv (1+\sqrt{3})(1+3^{(p-1)/2}\sqrt{3})\pod p
\]
But $p\equiv 7\pod {12}$, so $3^{(p-1)/2}\equiv -1\pod p$ and thus
\[
  (1+\sqrt{3})^{p+1} \equiv (1+\sqrt{3})(1-\sqrt{3}) = -2\pod p
\]
Putting together the two expressions for $(1+\sqrt{3})^{p+1}$, we get $\alpha^{(p+1)/2}\equiv -1\pod p$.
\end{proof}

We are now in a position to prove Theorem \ref{thm:one}:
\begin{proof}
$(\Rightarrow):$ If $M_p$ is prime where $p>3$ is prime, then note that $M_p\equiv 7\pod 8, 7\pod 12$ so that $M_p$ satisfies the conditions of Lemma \ref{lem:three}. The result follows.

$(\Leftarrow):$ If $\alpha^{(M_p+1)/2} \equiv -1\pod{M_p}$, choose $q\mid M_p$ for $q$ a prime. Since $M_p\equiv 7\pod 12$, we have $q>3$. Since $\alpha^{(M_p+1)/2}\equiv -1\pod{M_p}$ also $\alpha^{(M_p+1)/2}\equiv -1\pod q$ and thus $\alpha^{M_p+1}\equiv 1\pod q$. But $M_p+1=2^p$, so
\[
  \alpha^{2^p} \equiv 1\pod q
\]
Thus the order of $\alpha\pod q$ divides $2^n$. It can't divide $2^{n-1}$ since $\alpha^{(M_p+1)/2} \equiv -1\pod q$, so its order is precisely $2^n = M_p+1$. However, $\alpha^{q+1}\equiv 1\pod q$ or $\alpha^{q-1}\equiv 1\pod q$ by Lemma \ref{lem:two} and thus $q\geq M_p$. But $q\mid M_p$, so $q=M_p$ and $M_p$ is in fact prime.
\end{proof}
%%%%%
%%%%%
\end{document}
