\documentclass[12pt]{article}
\usepackage{pmmeta}
\pmcanonicalname{ChineseHypothesis}
\pmcreated{2013-03-22 18:11:08}
\pmmodified{2013-03-22 18:11:08}
\pmowner{FourDozens}{21006}
\pmmodifier{FourDozens}{21006}
\pmtitle{Chinese hypothesis}
\pmrecord{6}{40757}
\pmprivacy{1}
\pmauthor{FourDozens}{21006}
\pmtype{Definition}
\pmcomment{trigger rebuild}
\pmclassification{msc}{11A51}

% this is the default PlanetMath preamble.  as your knowledge
% of TeX increases, you will probably want to edit this, but
% it should be fine as is for beginners.

% almost certainly you want these
\usepackage{amssymb}
\usepackage{amsmath}
\usepackage{amsfonts}

% used for TeXing text within eps files
%\usepackage{psfrag}
% need this for including graphics (\includegraphics)
%\usepackage{graphicx}
% for neatly defining theorems and propositions
%\usepackage{amsthm}
% making logically defined graphics
%%%\usepackage{xypic}

% there are many more packages, add them here as you need them

% define commands here

\begin{document}
The Chinese hypothesis states that a number $n$ is prime if and only if $2^n -2$ is a multiple of $n$. By Fermat's little theorem we have that $2^p \equiv 2 \mod p$, so that means $n$ does divide $2^n- 2$ if $n$ is prime. However, if $n$ is composite Fermat's little theorem does not rule out that $n$ could divide $2^n-2$. The Chinese hypothesis checks out for the small powers of two. The first counterexample is $n=341$, but since $2^{341}$ has more than a hundred digits, it wasn't easy to check it back in the 18th century when this test was first proposed. Though back then they attributed it to ancient Chinese mathematicians, hence the name.
%%%%%
%%%%%
\end{document}
