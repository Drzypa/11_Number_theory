\documentclass[12pt]{article}
\usepackage{pmmeta}
\pmcanonicalname{CoefficientsOfBernoulliPolynomials}
\pmcreated{2013-03-22 17:46:08}
\pmmodified{2013-03-22 17:46:08}
\pmowner{rm50}{10146}
\pmmodifier{rm50}{10146}
\pmtitle{coefficients of Bernoulli polynomials}
\pmrecord{4}{40224}
\pmprivacy{1}
\pmauthor{rm50}{10146}
\pmtype{Derivation}
\pmcomment{trigger rebuild}
\pmclassification{msc}{11B68}
\pmrelated{BernoulliPolynomialsAndNumbers}

% this is the default PlanetMath preamble.  as your knowledge
% of TeX increases, you will probably want to edit this, but
% it should be fine as is for beginners.

% almost certainly you want these
\usepackage{amssymb}
\usepackage{amsmath}
\usepackage{amsfonts}

% used for TeXing text within eps files
%\usepackage{psfrag}
% need this for including graphics (\includegraphics)
%\usepackage{graphicx}
% for neatly defining theorems and propositions
%\usepackage{amsthm}
% making logically defined graphics
%%%\usepackage{xypic}

% there are many more packages, add them here as you need them

% define commands here

\begin{document}
The coefficient of $x^k$ in $b_r(x)$ for $k=1,2,\ldots,r$ is $\binom{r}{k}B_{r-k}$.

The proof is by induction on $r$. For $r=1$, note that $b_1(x)=x-\frac{1}{2}$, so that $[x]b_1(x)=1=\binom{1}{1}B_0$.

Writing $[x^k]f(x)$ for the coefficient of $x^k$ in a polynomial $f(x)$, note that for $k=1,2,\ldots,r$,
\[[x^k]b_r(x)=\frac{1}{k}[x^{k-1}]b_r'(x)=\frac{r}{k}[x^{k-1}]b_{r-1}(x)\]
since $b_r'(x)=rb_{r-1}(x)$. By induction,
\[\frac{r}{k}[x^{k-1}]b_{r-1}(x)=\frac{r}{k}\binom{r-1}{k-1}B_{r-k}=\binom{r}{k}B_{r-k}\]

Thus the Bernoulli polynomials can be written
\[b_r(x) = \sum_{k=1}^r \binom{r}{k}B_{r-k}x^k + B_r\]
%%%%%
%%%%%
\end{document}
