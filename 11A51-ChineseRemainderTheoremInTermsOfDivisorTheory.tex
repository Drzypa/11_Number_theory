\documentclass[12pt]{article}
\usepackage{pmmeta}
\pmcanonicalname{ChineseRemainderTheoremInTermsOfDivisorTheory}
\pmcreated{2013-03-22 18:01:58}
\pmmodified{2013-03-22 18:01:58}
\pmowner{pahio}{2872}
\pmmodifier{pahio}{2872}
\pmtitle{Chinese remainder theorem in terms of divisor theory}
\pmrecord{6}{40552}
\pmprivacy{1}
\pmauthor{pahio}{2872}
\pmtype{Theorem}
\pmcomment{trigger rebuild}
\pmclassification{msc}{11A51}
\pmclassification{msc}{13A05}
\pmrelated{ChineseRemainderTheorem}
\pmrelated{ChineseRemainderTheorem2}
\pmrelated{CongruenceInAlgebraicNumberField}
\pmrelated{WeakApproximationTheorem}

\endmetadata

% this is the default PlanetMath preamble.  as your knowledge
% of TeX increases, you will probably want to edit this, but
% it should be fine as is for beginners.

% almost certainly you want these
\usepackage{amssymb}
\usepackage{amsmath}
\usepackage{amsfonts}

% used for TeXing text within eps files
%\usepackage{psfrag}
% need this for including graphics (\includegraphics)
%\usepackage{graphicx}
% for neatly defining theorems and propositions
 \usepackage{amsthm}
 \usepackage[T2A]{fontenc}
 \usepackage[russian, english]{babel}

% making logically defined graphics
%%%\usepackage{xypic}

% there are many more packages, add them here as you need them

% define commands here

\theoremstyle{definition}
\newtheorem*{thmplain}{Theorem}
\begin{document}
In a ring with a divisor theory, a congruence \,$\alpha \equiv \beta \pmod{\mathfrak{a}}$\, with respect to a divisor \PMlinkname{module}{Congruences} $\mathfrak{a}$ \PMlinkescapetext{means} that\, $\mathfrak{a} \mid \alpha\!-\!\beta$.\\


\textbf{Theorem.}\, Let $\mathcal{O}$ be an integral domain having the divisor theory \,$\mathcal{O}^* \to \mathfrak{D}$.\, For arbitrary pairwise coprime divisors $\mathfrak{a}_1,\,\ldots,\,\mathfrak{a}_s$\, in $\mathfrak{D}$ and for arbitrary elements \,$\alpha_1,\,\ldots,\,\alpha_s$\, of the domain $\mathcal{O}$ there exists an element $\xi$ in $\mathcal{O}$ such that
\begin{align*}
\begin{cases}
\xi\, \equiv\, \alpha_1 \pmod{\mathfrak{a}_1}\\
\cdots \qquad \cdots \qquad \cdots\\
\xi\, \equiv\, \alpha_s \pmod{\mathfrak{a}_s}
\end{cases}
\end{align*}


{\em Proof.}\, Let
$$\mathfrak{b}_i \,:=\, \prod_{j \neq i}\mathfrak{a}_j \quad (i = 1,\,\ldots,\,s).$$
Apparently, the divisors\, $\mathfrak{b}_1,\,\ldots,\,\mathfrak{b}_s$\, are mutually coprime, whence there are in the ring $\mathcal{O}$ the elements\, $\beta_1,\,\ldots,\,\beta_s$\, divisible by\, the divisors\, $\mathfrak{b}_1,\,\ldots,\,\mathfrak{b}_s$,\, respectively, such that 
\begin{align}
\beta_1+\ldots+\beta_s = 1.
\end{align}
For every\, $i \neq j$,\, the divisor $\mathfrak{a}_i$ divides $\mathfrak{b}_j$ and therefore also the element $\beta_j$.\, Then the equation (1) implies that\, $\beta_i \equiv 1 \pmod{\mathfrak{a}_i}$ and thus the element
$$\xi \,:=\, \alpha_1\beta_1+\ldots+\alpha_s\beta_s$$
satisfies
$$\xi \,\equiv\, \alpha_i\beta_i \,\equiv\, \alpha_i\! \pmod{\mathfrak{a}_i}$$
for each\, $i = 1,\,\ldots,\,s$.\, Q.E.D.
 


\begin{thebibliography}{9}
\bibitem{MMP} \CYRM. \CYRM. \CYRP\cyro\cyrs\cyrt\cyrn\cyri\cyrk\cyro\cyrv: 
{\em \CYRV\cyrv\cyre\cyrd\cyre\cyrn\cyri\cyre\, \cyrv\, \cyrt\cyre\cyro\cyrr\cyri\cyryu\, \cyra\cyrl\cyrg\cyre\cyrb\cyrr\cyra\cyri\cyrch\cyre\cyrs\cyrk\cyri\cyrh \,
\cyrch\cyri\cyrs\cyre\cyrl}. \,\CYRI\cyrz\cyrd\cyra\cyrt\cyre\cyrl\cyrsftsn\cyrs\cyrt\cyrv\cyro \,
``\CYRN\cyra\cyru\cyrk\cyra''. \CYRM\cyro\cyrs\cyrk\cyrv\cyra \,(1982).
\end{thebibliography}

%%%%%
%%%%%
\end{document}
