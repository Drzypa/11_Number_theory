\documentclass[12pt]{article}
\usepackage{pmmeta}
\pmcanonicalname{GeneralizedAndricaConjecture}
\pmcreated{2013-03-22 17:17:34}
\pmmodified{2013-03-22 17:17:34}
\pmowner{dankomed}{17058}
\pmmodifier{dankomed}{17058}
\pmtitle{generalized Andrica conjecture}
\pmrecord{32}{39636}
\pmprivacy{1}
\pmauthor{dankomed}{17058}
\pmtype{Conjecture}
\pmcomment{trigger rebuild}
\pmclassification{msc}{11A41}
\pmrelated{FlorentinSmarandache}
\pmrelated{SmarandacheFunction}

\endmetadata

% this is the default PlanetMath preamble.  as your knowledge
% of TeX increases, you will probably want to edit this, but
% it should be fine as is for beginners.

% almost certainly you want these
\usepackage{amssymb}
\usepackage{amsmath}
\usepackage{amsfonts}

% used for TeXing text within eps files
%\usepackage{psfrag}
% need this for including graphics (\includegraphics)
\usepackage{graphicx}
% for neatly defining theorems and propositions
%\usepackage{amsthm}
% making logically defined graphics
%%%\usepackage{xypic}

% there are many more packages, add them here as you need them

% define commands here

\begin{document}
The \emph{Andrica function} $A_{n}\equiv\sqrt{p_{n+1}}-\sqrt{p_{n}}$, where $p_n$ is the n$^{\text{th}}$ prime number
can be plotted with mathematical software and for large $n$ it seems that $1\gg A_{n}$, however
the \emph{Andrica conjecture} $1>A_{n}$ has not been yet proven and
remains an open problem.

Similarly one can consider the \emph{generalized Andrica function}
$A_{G}(x,n)\equiv p_{n+1}^{x}-p_{n}^{x}$ and plot it for $x\in\mathbb{R}$.

It is clear that $ A_{G}(0,n)=0$.

For $x<0$, $A_{G}(x,n)$ is negative, and if $x\rightarrow-\infty$ then $A_{G}(x,n)\rightarrow-\infty$.

For $x>0$, $A_{G}(x,n)$ is positive, and if $x\rightarrow+\infty$ then $A_{G}(x,n)\rightarrow+\infty$.

Therefore if one considers the \emph{generalized Andrica equation}
$A_{G}(x,n)=1$ and solves for $x$ then solutions for each $n$ will occur
for $x>0$. What is more it is easily provable that the biggest solution of generalized Andrica equation $x_{\max}=1$
occurs for $n=1$, and for $n>1$ it is always the case that each solution of generalized Andrica equation $x_n<1$
because the minimal difference between two consequtive primes is at
best 2 for \emph{twin primes}. However the value of the smallest solution of generalized Andrica equation 
$x_{\min}$ at the present time remains unknown and its existence is unproven.

The existence of minimal solution $x_{\min}$ of the generalized Andrica equation is still unproven. However according to the \emph{generalized Andrica conjecture} proposed by Florentin Smarandache the value of $x_{\min}$, also known as the \emph{Smarandache constant}, is  $x_{\min}\approx0.5671481302\ldots$ and occurs for $n=30$. If stated as an inequality the generalized Andrica conjecture states:

$p _ {n+1} ^ x - p_ n ^ x < 1$ for $x < 0.567148 \ldots$

Numerical plots for the first $2\times10^{11}$ primes show that the solutions $x_n$ of $A_{G}(x,n)=1$ tend to be confined in the interval $(0.9,1)$ and according to generalized Andrica conjecture one hopes that this behavior remains true as $n\rightarrow\infty$.

The following plots of $A_G(x,n)$ were created with \emph{Wolfram's Mathematica 5.2}, the function plot range was cut off at $A_G(x,n)=1$, so the edge of the plateau is visualizing the exact solutions $x_n$ of the equation $A_G(x,n)=1$.

\includegraphics{200} \includegraphics[scale=.86]{200b}

Plots for the first 200 primes. This plot most clearly visualizes the putative minimal solution $x_{\min}$ known also as the Smarandache constant, which seems to occur for $n=30$.

\includegraphics{10-3} \includegraphics[scale=.86]{10-3b}

Plots for the first 1000 primes.

\includegraphics{2x10-3} \includegraphics[scale=.86]{2x10-3b}

Plots for the first $2 \times 10^3$ primes.

\includegraphics{2x10-4} \includegraphics[scale=.86]{2x10-4b}

Plots for the first $2 \times 10^4$ primes.

\includegraphics{2x10-5} \includegraphics[scale=.86]{2x10-5b}

Plots for the first $2 \times 10^5$ primes.

\includegraphics{2x10-6} \includegraphics{2x10-6b}

Plots for the first $2 \times 10^6$ primes.

\includegraphics{2x10-9} \includegraphics{2x10-9b}

Plots for the first $2 \times 10^9$ primes.

\includegraphics{2x10-11} \includegraphics{2x10-11b}

Plots for the first $2 \times 10^{11}$ primes.


%%%%%
%%%%%
\end{document}
