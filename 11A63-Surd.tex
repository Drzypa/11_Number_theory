\documentclass[12pt]{article}
\usepackage{pmmeta}
\pmcanonicalname{Surd}
\pmcreated{2013-03-22 14:17:44}
\pmmodified{2013-03-22 14:17:44}
\pmowner{mathwizard}{128}
\pmmodifier{mathwizard}{128}
\pmtitle{surd}
\pmrecord{10}{35751}
\pmprivacy{1}
\pmauthor{mathwizard}{128}
\pmtype{Definition}
\pmcomment{trigger rebuild}
\pmclassification{msc}{11A63}

% this is the default PlanetMath preamble.  as your knowledge
% of TeX increases, you will probably want to edit this, but
% it should be fine as is for beginners.

% almost certainly you want these
\usepackage{amssymb}
\usepackage{amsmath}
\usepackage{amsfonts}

% used for TeXing text within eps files
%\usepackage{psfrag}
% need this for including graphics (\includegraphics)
%\usepackage{graphicx}
% for neatly defining theorems and propositions
%\usepackage{amsthm}
% making logically defined graphics
%%%\usepackage{xypic}

% there are many more packages, add them here as you need them

% define commands here
\begin{document}
In mathematics, numbers often have to be written using roots or fractional exponents. For example, the number $\sqrt{2}$ cannot be written without using a root or writing $2^{1/2}$. A \emph{surd} is an irrational number which can be written as the sum of rational powers of rational numbers.
Another example would be $3\sqrt{2}-\sqrt[3]{5}$.
\subsection*{References}
\PMlinkexternal{What is a surd?}{http://www.mathcentre.ac.uk/students.php/all_subjects/arithmetic/surds/resources/34}
%%%%%
%%%%%
\end{document}
