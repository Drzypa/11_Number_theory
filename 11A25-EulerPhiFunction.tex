\documentclass[12pt]{article}
\usepackage{pmmeta}
\pmcanonicalname{EulerPhiFunction}
\pmcreated{2013-03-22 11:45:11}
\pmmodified{2013-03-22 11:45:11}
\pmowner{Wkbj79}{1863}
\pmmodifier{Wkbj79}{1863}
\pmtitle{Euler phi function}
\pmrecord{19}{30196}
\pmprivacy{1}
\pmauthor{Wkbj79}{1863}
\pmtype{Definition}
\pmcomment{trigger rebuild}
\pmclassification{msc}{11A25}
\pmclassification{msc}{11-00}
\pmclassification{msc}{14F35}
\pmclassification{msc}{14H30}
\pmclassification{msc}{20F34}
\pmsynonym{Euler totient function}{EulerPhiFunction}
\pmsynonym{Euler $\varphi$ function}{EulerPhiFunction}
\pmsynonym{Euler $\phi$ function}{EulerPhiFunction}
%\pmkeywords{number theory}
\pmrelated{ProofThatEulerPhiIsAMultiplicativeFunction}
\pmrelated{ValuesOfNForWhichVarphintaun}
\pmrelated{PrimeResidueClass}
\pmrelated{EulerPhiAtAProduct}
\pmrelated{SummatoryFunctionOfArithmeticFunction}

\usepackage{amssymb}
\usepackage{amsmath}
\usepackage{amsfonts}
\usepackage{graphicx}
%%%%\usepackage{xypic}
\begin{document}
For any positive integer $n$, $\varphi(n)$ is the number of positive integers less than or equal to $n$ which are coprime to $n$.  The function $\varphi$ is known as the \emph{\PMlinkescapetext{Euler} $\varphi$ function}.  This function may also be denoted by $\phi$.

Among the most useful \PMlinkescapetext{properties} of $\varphi$ are the facts that it is multiplicative (meaning if $\gcd(a,b)=1$, then $\varphi(ab)=\varphi(a)\varphi(b)$) and that $\varphi(p^k)=p^{k-1}(p-1)$ for any prime $p$ and any positive integer $k$. These two facts combined give a numeric computation of $\varphi$ for all positive integers:
\begin{align}
\label{varphi}
\varphi(n) & =n \prod_{p|n} \left( 1-\frac{1}{p} \right).
\end{align}
For example, 
\begin{align*}
\varphi (2000) & = 2000 \prod_{p|2000} \left(1-\frac{1}{p} \right) \\
& = 2000 \left( 1 - \frac{1}{2} \right) \left( 1 - \frac{1}{5} \right) \\
& = 2000 \left( \frac{1}{2} \right) \left( \frac{4}{5} \right) \\
& = \frac{8000}{10} \\
& = 800.
\end{align*}

From equation (\ref{varphi}), it is clear that $\varphi(n)\le n$ for any positive integer $n$ with equality holding exactly when $n=1$.  This is because
\[
\prod_{p|n} \left( 1-\frac{1}{p} \right) \le 1,
\]
with equality holding exactly when $n=1$.

Another important fact about the \PMlinkescapetext{Euler} $\varphi$ function is that
\[
\sum_{d|n} \varphi(d)=n,
\]
where the sum extends over all positive divisors of $n$.  Also, by definition, $\varphi(n)$ is the number of units in the ring $\mathbb{Z}/n\mathbb{Z}$ of integers modulo $n$.

The sequence $\{\varphi(n)\}$ appears in the OEIS as sequence \PMlinkexternal{A000010}{http://www.research.att.com/~njas/sequences/A000010}.
%%%%%
%%%%%
%%%%%
%%%%%
\end{document}
