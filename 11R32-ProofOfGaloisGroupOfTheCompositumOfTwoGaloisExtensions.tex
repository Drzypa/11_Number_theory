\documentclass[12pt]{article}
\usepackage{pmmeta}
\pmcanonicalname{ProofOfGaloisGroupOfTheCompositumOfTwoGaloisExtensions}
\pmcreated{2013-03-22 18:42:01}
\pmmodified{2013-03-22 18:42:01}
\pmowner{rm50}{10146}
\pmmodifier{rm50}{10146}
\pmtitle{proof of Galois group of the compositum of two Galois extensions}
\pmrecord{6}{41462}
\pmprivacy{1}
\pmauthor{rm50}{10146}
\pmtype{Proof}
\pmcomment{trigger rebuild}
\pmclassification{msc}{11R32}
\pmclassification{msc}{12F99}

% this is the default PlanetMath preamble.  as your knowledge
% of TeX increases, you will probably want to edit this, but
% it should be fine as is for beginners.

% almost certainly you want these
\usepackage{amssymb}
\usepackage{amsmath}
\usepackage{amsfonts}

% used for TeXing text within eps files
%\usepackage{psfrag}
% need this for including graphics (\includegraphics)
%\usepackage{graphicx}
% for neatly defining theorems and propositions
\usepackage{amsthm}
% making logically defined graphics
%%\usepackage{xypic}

% there are many more packages, add them here as you need them

% define commands here
\DeclareMathOperator{\Gal}{Gal}
\newcommand{\Order}[1]{\left\lvert #1 \right\rvert}
%
%% \theoremstyle{plain} %% This is the default
\newtheorem{thm}{Theorem}
\newtheorem{cor}[thm]{Corollary}
\newtheorem{lem}[thm]{Lemma}
\newtheorem{prop}[thm]{Proposition}
\newtheorem{ax}{Axiom}

\begin{document}
\begin{proof} Consider the diagram
\[\xymatrix @R1pc@C1pc{
 & \ar@{-}[ld]\ar@{-}[rd]EF \\
\ar@{-}[rd]E & & \ar@{-}[ld]F\\
 & \ar@{-}[d]E\cap F \\
 & K
}
\]
(1): Let $p(x)\in K[x]$ with a root $\alpha\in E\cap F$. Then since $E$ (resp. $F$) is Galois over $K$, all the roots of $p$ lie in $E$ (resp. $F$) and thus in $E\cap F$. The result follows.

(2): We first show that $EF$ is Galois over $K$. Choose separable polynomials $p(x),q(x)\in K[x]$ so that $E$ (resp. $F$) is a splitting field for $p$ (resp. $q$). Then $EF$ is a splitting field for the squarefree part of $pq$, which is separable since it is squarefree and since $p(x),q(x)$ are separable.

Now, define
\[\theta: \Gal(EF/K)\to \Gal(E/K)\times \Gal(F/K): \sigma\mapsto (\sigma|_E,\sigma|_F)\]
This map is a group homomorphism; its kernel is precisely those elements that leave both $E$ and $F$ fixed. Any such element must thus leave $EF$ fixed, so that $\theta$ is injective. The image obviously lies in
\[H=\{ (\sigma, \tau) : \sigma|_{E\cap F}=\tau|_{E\cap F} \}\]
by construction: $(\sigma|_E)|_{E\cap F} = \sigma|_{E\cap F} = (\sigma|_F)|_{E\cap F}$. We will show that $H$ is precisely the image of $\theta$ by showing that the order of $H$ is the same as the index of the field extension $[EF:K]$.

For each $\sigma\in \Gal(E/K)$, there are precisely $\Order{\Gal(F/E\cap F)}$ elements of $\Gal(F/K)$ whose restrictions to $E\cap F$ are $\sigma|_{E\cap F}$. Thus directly from the definition of $H$,
\[
  \Order{H} = \Order{\Gal(E/K)}\cdot\Order{\Gal(F/E\cap F)} 
            = \Order{\Gal(E/K)}\cdot\frac{\Order{\Gal(F/K)}}{\Order{\Gal((E\cap F)/K)}}
\]
By the \PMlinkname{corollary to the theorem regarding the compositum of a Galois extension and another extension}{CorollaryToTheCompositumOfAGaloisExtensionAndAnotherExtensionIsGalois}, we have
\[[EF:K] =[EF:F][F:K] = [E:E\cap F][F:K] = \frac{[E:K][F:K]}{[E\cap F:K]}\]
so that
\[
  \Order{H} = [EF:K]
\]
\end{proof} 

\begin{thebibliography}{10}
\bibitem{bib:df}
Dummit,~D.,~Foote,~R.M., \emph{Abstract Algebra, Third Edition}, Wiley, 2004.
\end{thebibliography}

%%%%%
%%%%%
\end{document}
