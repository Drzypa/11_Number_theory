\documentclass[12pt]{article}
\usepackage{pmmeta}
\pmcanonicalname{EulerPolynomial}
\pmcreated{2013-03-22 19:07:07}
\pmmodified{2013-03-22 19:07:07}
\pmowner{pahio}{2872}
\pmmodifier{pahio}{2872}
\pmtitle{Euler polynomial}
\pmrecord{7}{42013}
\pmprivacy{1}
\pmauthor{pahio}{2872}
\pmtype{Definition}
\pmcomment{trigger rebuild}
\pmclassification{msc}{11B68}
\pmrelated{BernoulliPolynomial}
\pmrelated{BernoulliPolynomialsAndNumbers}

\endmetadata

% this is the default PlanetMath preamble.  as your knowledge
% of TeX increases, you will probably want to edit this, but
% it should be fine as is for beginners.

% almost certainly you want these
\usepackage{amssymb}
\usepackage{amsmath}
\usepackage{amsfonts}

% used for TeXing text within eps files
%\usepackage{psfrag}
% need this for including graphics (\includegraphics)
%\usepackage{graphicx}
% for neatly defining theorems and propositions
 \usepackage{amsthm}
% making logically defined graphics
%%%\usepackage{xypic}

% there are many more packages, add them here as you need them

% define commands here

\theoremstyle{definition}
\newtheorem*{thmplain}{Theorem}

\begin{document}
The \emph{Euler polynomials}\, $E_0(x),\,E_1(x),\,E_2(x),\,\ldots$\, are certain polynomials of the indeterminate $x$ with rational coefficients (whose denominators may only be powers $1,\,2,\,4,\,8,\,\ldots$\, of 2).\, The Euler polynomials may be defined by means of the generating function such that
$$\frac{2e^{xt}}{e^t\!+\!1} \;=\; \sum_{n=0}^\infty E_n(x)\frac{t^n}{n!},$$
i.e. one can get them by dividing the Taylor series $2+2xt+x^2t^2+\frac{1}{3}x^3t^3+\ldots$\, by the Taylor series $2+t+\frac{1}{2}t^2+\frac{1}{6}t^3+\ldots$.\, There are also explicit formulae for the polynomials, e.g.
$$E_n(x) \;=\; \sum_{k=0}^n {n \choose k}\frac{E_k}{2^k}\left(x\!-\!\frac{1}{2}\right)^{n-k}$$
via the Euler numbers $E_k$.\, Conversely, the Euler numbers are expressed with the Euler polynomials through
$$E_k \;=\; 2^kE_k\!\!\left(\!\frac{1}{2}\!\right).$$\\
The first seven Euler polynomials are
\begin{align*}
    & E_0(x) \;=\; 1\\
    & E_1(x) \;=\; x\!-\!\frac{1}{2}\\
    & E_2(x) \;=\; x^2\!-\!x\\
    & E_3(x) \;=\; x^3\!-\!\frac{3}{2}x^2\!+\!\frac{1}{4}\\
    & E_4(x) \;=\; x^4\!-\!2x^3\!+\!x\\
    & E_5(x) \;=\; x^5\!-\!\frac{5}{2}x^4\!+\!\frac{5}{2}x^2\!-\!\frac{1}{2}\\
    & E_6(x) \;=\; x^6\!-\!3x^5\!+\!5x^3\!-\!3x
\end{align*}

The Euler polynomials have the beautiful addition formula
$$E_n(x\!+\!y) \;=\; \sum_{k=0}^n{n \choose k}E_k(x)y^k$$
and the derivative
$$E_n'(x) \;=\; nE_{n-1}(x) \qquad (\textrm{for  } n = 1,\,2,\,\ldots).$$
The Euler polynomials form an example of Appell sequences.


%%%%%
%%%%%
\end{document}
