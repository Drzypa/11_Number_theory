\documentclass[12pt]{article}
\usepackage{pmmeta}
\pmcanonicalname{BrocardsConjecture}
\pmcreated{2013-03-22 16:40:53}
\pmmodified{2013-03-22 16:40:53}
\pmowner{PrimeFan}{13766}
\pmmodifier{PrimeFan}{13766}
\pmtitle{Brocard's conjecture}
\pmrecord{8}{38890}
\pmprivacy{1}
\pmauthor{PrimeFan}{13766}
\pmtype{Conjecture}
\pmcomment{trigger rebuild}
\pmclassification{msc}{11A41}
\pmsynonym{Brocard conjecture}{BrocardsConjecture}
\pmrelated{LegendresConjecture}

\endmetadata

% this is the default PlanetMath preamble.  as your knowledge
% of TeX increases, you will probably want to edit this, but
% it should be fine as is for beginners.

% almost certainly you want these
\usepackage{amssymb}
\usepackage{amsmath}
\usepackage{amsfonts}

% used for TeXing text within eps files
%\usepackage{psfrag}
% need this for including graphics (\includegraphics)
%\usepackage{graphicx}
% for neatly defining theorems and propositions
%\usepackage{amsthm}
% making logically defined graphics
%%%\usepackage{xypic}

% there are many more packages, add them here as you need them

% define commands here

\begin{document}
(Henri Brocard) With the exception of 4 and 9, there are always at least four prime numbers between the square of a prime and the square of the next prime. To put it algebraically, given the $n$th prime $p_n$ (with $n > 1$), the inequality $(\pi({p_{n + 1}}^2) - \pi({p_n}^2)) > 3$ is always true, where $\pi(x)$ is the prime counting function.

For example, between $2^2$ and $3^2$ there are only two primes: 5 and 7. But between $3^2$ and $5^2$ there are five primes: a prime quadruplet (11, 13, 17, 19) and 23.

This conjecture remains unproven as of 2007. Thanks to computers, brute force searches have shown that the conjecture holds true as high as $n = 10^4$. % I look forward to updates.
%%%%%
%%%%%
\end{document}
