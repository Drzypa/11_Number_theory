\documentclass[12pt]{article}
\usepackage{pmmeta}
\pmcanonicalname{2omeganletaunle2Omegan}
\pmcreated{2013-03-22 16:07:28}
\pmmodified{2013-03-22 16:07:28}
\pmowner{Wkbj79}{1863}
\pmmodifier{Wkbj79}{1863}
\pmtitle{$2^{\omega(n)} \le \tau(n) \le 2^{\Omega(n)}$}
\pmrecord{14}{38194}
\pmprivacy{1}
\pmauthor{Wkbj79}{1863}
\pmtype{Theorem}
\pmcomment{trigger rebuild}
\pmclassification{msc}{11A25}
\pmrelated{NumberOfDistinctPrimeFactorsFunction}
\pmrelated{TauFunction}
\pmrelated{NumberOfNondistinctPrimeFactorsFunction}
\pmrelated{DisplaystyleYOmeganOleftFracxlogXy12YRightFor1LeY2}

\endmetadata

\usepackage{amssymb}
\usepackage{amsmath}
\usepackage{amsfonts}

\usepackage{psfrag}
\usepackage{graphicx}
\usepackage{amsthm}
%%\usepackage{xypic}

\newtheorem*{thm*}{Theorem}
\newtheorem*{cor*}{Corollary}
\begin{document}
Throughout this entry, $\omega$, $\tau$, and $\Omega$ denote the number of distinct prime factors function, the divisor function, and the \PMlinkname{number of (nondistinct) prime factors function}{NumberOfNondistinctPrimeFactorsFunction}, respectively.

\begin{thm*}
For any positive integer $n$, $2^{\omega(n)} \le \tau(n) \le 2^{\Omega(n)}$.
\end{thm*}

\begin{proof}
Note that $2^{\omega(n)}$, $\tau(n)$, and $2^{\Omega(n)}$ are multiplicative.  Also note that, for any positive integer $n$, the numbers $2^{\omega(n)}$, $\tau(n)$, and $2^{\Omega(n)}$ are positive integers.  Therefore, it will suffice to prove the inequality for prime powers.

Let $p$ be a prime and $k$ be a positive integer.  Thus:

\begin{center}
$\begin{array}{rl}
\displaystyle 2^{\omega(p^k)} & =2 \\
\\
\tau(p^k) & =k+1 \\
\\
\displaystyle 2^{\Omega(p^k)} & = 2^k \end{array}$
\end{center}

Hence, $2^{\omega(p^k)} \le \tau(p^k) \le 2^{\Omega(p^k)}$.  It follows that $2^{\omega(n)} \le \tau(n) \le 2^{\Omega(n)}$.
\end{proof}

This theorem has an obvious corollary.

\begin{cor*}
For any squarefree positive integer $n$, $2^{\omega(n)}=\tau(n)=2^{\Omega(n)}$.
\end{cor*}
%%%%%
%%%%%
\end{document}
