\documentclass[12pt]{article}
\usepackage{pmmeta}
\pmcanonicalname{RiemannXiFunction}
\pmcreated{2013-03-22 13:24:06}
\pmmodified{2013-03-22 13:24:06}
\pmowner{PrimeFan}{13766}
\pmmodifier{PrimeFan}{13766}
\pmtitle{Riemann $\Xi$ function}
\pmrecord{11}{33943}
\pmprivacy{1}
\pmauthor{PrimeFan}{13766}
\pmtype{Definition}
\pmcomment{trigger rebuild}
\pmclassification{msc}{11M06}

\endmetadata

% this is the default PlanetMath preamble.  as your knowledge
% of TeX increases, you will probably want to edit this, but
% it should be fine as is for beginners.

% almost certainly you want these
\usepackage{amssymb}
\usepackage{amsmath}
\usepackage{amsfonts}

% used for TeXing text within eps files
%\usepackage{psfrag}
% need this for including graphics (\includegraphics)
%\usepackage{graphicx}
% for neatly defining theorems and propositions
%\usepackage{amsthm}
% making logically defined graphics
%%%\usepackage{xypic}

% there are many more packages, add them here as you need them

% define commands here
\begin{document}
The \emph{Riemann Xi function} $$\Xi(s) = \pi^{-\frac{1}{2}s} \Gamma(\frac{1}{2}s) \zeta(s),$$ (where $\Gamma(s)$ is Euler's Gamma function and $\zeta(s)$ is the Riemann zeta function), is the key to the functional equation for the Riemann zeta function.

Riemann himself used the notation of a lower case xi ($\xi$). The famous Riemann hypothesis is equivalent to the assertion that all the zeros of $\xi$ are real, in fact Riemann himself presented his original hypothesis in terms of that function.

Riemann's lower case xi is defined as $$\xi(s) = \frac{1}{2} s(s-1) \Xi(s).$$
%%%%%
%%%%%
\end{document}
