\documentclass[12pt]{article}
\usepackage{pmmeta}
\pmcanonicalname{MetricallyConvexSpace}
\pmcreated{2013-12-03 10:01:22}
\pmmodified{2013-12-03 10:01:22}
\pmowner{Entropy}{1000820}
\pmmodifier{Entropy}{1000820}
\pmtitle{Metrically Convex Space}
\pmrecord{2}{87990}
\pmprivacy{1}
\pmauthor{Entropy}{1000820}
\pmtype{Definition}

\endmetadata

% this is the default PlanetMath preamble.  as your knowledge
% of TeX increases, you will probably want to edit this, but
% it should be fine as is for beginners.

% almost certainly you want these
\usepackage{amssymb}
\usepackage{amsmath}
\usepackage{amsfonts}

% need this for including graphics (\includegraphics)
\usepackage{graphicx}
% for neatly defining theorems and propositions
\usepackage{amsthm}

% making logically defined graphics
%\usepackage{xypic}
% used for TeXing text within eps files
%\usepackage{psfrag}

% there are many more packages, add them here as you need them

% define commands here

\begin{document}
% this is the default PlanetMath preamble.  as your knowledge
% of TeX increases, you will probably want to edit this, but
% it should be fine as is for beginners.

% almost certainly you want these
\usepackage{amssymb}
\usepackage{amsmath}
\usepackage{amsfonts}

% need this for including graphics (\includegraphics)
\usepackage{graphicx}
% for neatly defining theorems and propositions
\usepackage{amsthm}

% making logically defined graphics
%\usepackage{xypic}
% used for TeXing text within eps files
%\usepackage{psfrag}

% there are many more packages, add them here as you need them

% define commands here
\begin{document}
A metric space $(M, d)$ is said to be metrically convex if for any
two points $x,y \in M$ with $x\not=y$ there exists $z\in M$: $x\not=z\not=y$, such that
$d(x,z)+d(z,y) = d(x,y)$.
\end{document
\end{document}
