\documentclass[12pt]{article}
\usepackage{pmmeta}
\pmcanonicalname{DegreeOfAlgebraicNumber}
\pmcreated{2013-03-22 19:08:51}
\pmmodified{2013-03-22 19:08:51}
\pmowner{pahio}{2872}
\pmmodifier{pahio}{2872}
\pmtitle{degree of algebraic number}
\pmrecord{12}{42050}
\pmprivacy{1}
\pmauthor{pahio}{2872}
\pmtype{Theorem}
\pmcomment{trigger rebuild}
\pmclassification{msc}{11R04}
\pmclassification{msc}{11C08}
\pmclassification{msc}{12F05}
\pmclassification{msc}{12E05}
%\pmkeywords{algebraic conjugates}
%\pmkeywords{characteristic equation}
%\pmkeywords{degree of algebraic number field}
\pmrelated{KConjugates}
\pmrelated{CharacteristicPolynomialOfAlgebraicNumber}

% this is the default PlanetMath preamble.  as your knowledge
% of TeX increases, you will probably want to edit this, but
% it should be fine as is for beginners.

% almost certainly you want these
\usepackage{amssymb}
\usepackage{amsmath}
\usepackage{amsfonts}

% used for TeXing text within eps files
%\usepackage{psfrag}
% need this for including graphics (\includegraphics)
%\usepackage{graphicx}
% for neatly defining theorems and propositions
 \usepackage{amsthm}
% making logically defined graphics
%%%\usepackage{xypic}

% there are many more packages, add them here as you need them

% define commands here

\theoremstyle{definition}
\newtheorem*{thmplain}{Theorem}

\begin{document}
\PMlinkescapeword{irreducible} \PMlinkescapeword{degree}

\textbf{Theorem.}\, The \PMlinkname{degree}{DegreeOfAnAlgebraicNumber} of any algebraic number $\alpha$ in the number field $\mathbb{Q}(\vartheta)$ divides the degree of $\vartheta$.\, The zeroes of the characteristic polynomial $g(x)$ of $\alpha$ consist of the algebraic conjugates of $\alpha$, each of which having equal multiplicity as zero of $g(x)$.
\\

\emph{Proof.}\, Let the minimal polynomial of \,$\alpha$\, be
$$a(x) \;:=\; x^k+a_1x^{k-1}+\ldots+a_k$$
and all zeroes of this be\, $\alpha_1 = \alpha,\,\alpha_2,\,\ldots,\,\alpha_k$.\, Denote the canonical polynomial of 
$\alpha$ with respect to the \PMlinkname{primitive element}{SimpleFieldExtension} $\vartheta$ by $r(x)$; then 
$$a(r(\vartheta)) \;=\; a(\alpha) \;=\; 0.$$
If\, $a(r(x)) \;:=\; \varphi(x)$,\, then the equation
$$\varphi(x) \;=\; 0$$
has rational coefficients and is satisfied by $\vartheta$.\, Since the minimal polynomial $f(x)$ of $\vartheta$ is \PMlinkname{irreducible}{IrreduciblePolynomial}, it must divide $\varphi(x)$ and all algebraic conjugates\, 
$\vartheta_1 = \vartheta,\,\vartheta_2,\,\ldots,\,\vartheta_n$ of $\vartheta$\, make $\varphi(x)$ zero.\, Hence we have
$$a(\alpha^{(i)}) \;=\; a(r(\vartheta_i)) \;=\; 0 \quad \mbox{for}\quad i \,=\, 1,\,2,\,\ldots,\,n$$
where the numbers $\alpha^{(i)}$ are the 
\PMlinkname{$\mathbb{Q}(\vartheta)$-conjugates}{CharacteristicPolynomialOfAlgebraicNumber} of $\alpha$.\, Thus these 
$\mathbb{Q}(\vartheta)$-conjugates are roots of the irreducible equation \,$a(x) = 0$, whence $a(x)$ must divide the characteristic polynomial $g(x)$.\, Let the \PMlinkname{power}{GeneralAssociativity} 
$[a(x)]^m$ exactly divide $g(x)$, when
$$g(x) \;=\; [a(x)]^mb(x), \quad a(x) \nmid b(x).$$
Antithesis:\; $\mbox{deg}(b(x)) \,\geqq\, 1$ \;\; and\;\; $b(\beta) \,=\, 0$.\\
This implies that\, $g(\beta) = 0$,\, i.e. $\beta$ is one of the numbers $\alpha^{(i)}$.\, Therefore, $\beta$ were a zero of $a(x)$ and thus $a(x) \mid b(x)$, which is impossible.\, Consequently,the antithesis is wrong, i.e. $b(x)$ is a constant, which must be 1 because $g(x)$ and $a(x)$ are monic polynomials.\, So,\, $g(x) = [a(x)]^m$.\, Since
$$a(x) \;=\; (x\!-\!\alpha_1)(x\!-\!\alpha_2)\cdots(x\!-\!\alpha_k),$$
it follows that
$$g(x) \;=\; (x\!-\!\alpha_1)^m(x\!-\!\alpha_2)^m\cdots(x\!-\!\alpha_k)^m.$$
Hence\, $km \,=\, n$\, and $k$ divides $n$, as asserted.\, Moreover, each $\alpha_j$ is a zero of order $m$ of $g(x)$, i.e. appears among the roots $\alpha^{(1)},\,\alpha^{(2)},\,\ldots,\,\alpha^{(n)}$ of the equation\, $g(x) = 0$\, $m$ times.




%%%%%
%%%%%
\end{document}
