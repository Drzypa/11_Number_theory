\documentclass[12pt]{article}
\usepackage{pmmeta}
\pmcanonicalname{CyclotomicPolynomial}
\pmcreated{2013-03-22 12:36:00}
\pmmodified{2013-03-22 12:36:00}
\pmowner{yark}{2760}
\pmmodifier{yark}{2760}
\pmtitle{cyclotomic polynomial}
\pmrecord{14}{32852}
\pmprivacy{1}
\pmauthor{yark}{2760}
\pmtype{Definition}
\pmcomment{trigger rebuild}
\pmclassification{msc}{11R60}
\pmclassification{msc}{11R18}
\pmclassification{msc}{11C08}
\pmrelated{AllOnePolynomial}
\pmrelated{FactoringAllOnePolynomialsUsingTheGroupingMethod}
\pmrelated{CyclotomicField}
\pmrelated{RootOfUnity}

\endmetadata

\usepackage{amsmath}
\usepackage{amsfonts}

\def\Q{\mathbb{Q}}
\def\Z{\mathbb{Z}}
\begin{document}
\PMlinkescapeword{primitive}
\PMlinkescapeword{ranges}
\PMlinkescapephrase{root of unity}

\section*{Definition}

For any positive integer $n$,
the $n$-th {\em cyclotomic polynomial} $\Phi_n(x)$ is defined as
\[
  \Phi_n(x)=\prod_\zeta(x-\zeta),
\]
where $\zeta$ ranges over the
\PMlinkname{primitive $n$-th roots of unity}{RootOfUnity}.

\section*{Examples}

The first few cyclotomic polynomials are as follows:
\begin{align*}
  \Phi_1(x)&=x-1 \\
  \Phi_2(x)&=x+1 \\
  \Phi_3(x)&=x^2+x+1 \\
  \Phi_4(x)&=x^2+1 \\
  \Phi_5(x)&=x^4+x^3+x^2+x+1 \\
  \Phi_6(x)&=x^2-x+1 \\
  \Phi_7(x)&=x^6+x^5+x^4+x^3+x^2+x+1 \\
  \Phi_8(x)&=x^4+1 \\
  \Phi_9(x)&=x^6+x^3+1 \\
  \Phi_{10}(x)&=x^4-x^3+x^2-x+1 \\
  \Phi_{11}(x)&=x^{10}+x^9+x^8+x^7+x^6+x^5+x^4+x^3+x^2+x+1 \\
  \Phi_{12}(x)&=x^4-x^2+1 \\
\end{align*}

The preceding examples may give the impression that the coefficients
are always $-1$, $0$ or $1$, but this is not true in general.
For example,
\begin{align*}
\Phi_{105}(x)=\,&x^{48}+x^{47}+x^{46}-x^{43}-x^{42}-2x^{41}-x^{40}-x^{39}+x^{36}+x^{35}+x^{34} \\
           &\phantom{x^{48}}+x^{33}+x^{32}+x^{31}-x^{28}-x^{26}-x^{24}-x^{22}-x^{20}+x^{17}+x^{16}+x^{15} \\
           &\phantom{x^{48}}+x^{14}+x^{13}+x^{12}-x^9-x^8-2x^7-x^6-x^5+x^2+x+1 \\
\end{align*}

\section*{Properties}

For every positive integer $n$,
$\Phi_n(x)$ is an irreducible polynomial of degree $\phi(n)$ in $\Q[x]$,
and is the minimal polynomial of each primitive $n$-th root of unity.
Here $\phi(n)$ is Euler's phi function.

%%%%%
%%%%%
\end{document}
