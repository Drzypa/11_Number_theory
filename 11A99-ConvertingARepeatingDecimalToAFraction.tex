\documentclass[12pt]{article}
\usepackage{pmmeta}
\pmcanonicalname{ConvertingARepeatingDecimalToAFraction}
\pmcreated{2013-03-22 16:55:22}
\pmmodified{2013-03-22 16:55:22}
\pmowner{Wkbj79}{1863}
\pmmodifier{Wkbj79}{1863}
\pmtitle{converting a repeating decimal to a fraction}
\pmrecord{10}{39185}
\pmprivacy{1}
\pmauthor{Wkbj79}{1863}
\pmtype{Algorithm}
\pmcomment{trigger rebuild}
\pmclassification{msc}{11A99}
\pmclassification{msc}{11-00}

\endmetadata

\usepackage{amssymb}
\usepackage{amsmath}
\usepackage{amsfonts}

\usepackage{psfrag}
\usepackage{graphicx}
\usepackage{amsthm}
%%\usepackage{xypic}

\begin{document}
\PMlinkescapeword{sides}
\PMlinkescapeword{divide}

The following algorithm can be used to convert a repeating decimal to a fraction:

\begin{enumerate}

\item Set the repeating decimal equal to $x$.

\item Multiply both sides of the equation by $10^n$, where $n$ is the number of digits that appear under the bar.

\item If applicable, rewrite the second equation so that its repeating part \PMlinkescapetext{lines} up with the repeating part in the original equation.

\item Subtract the original equation from the most recently obtained equation.  (The repeating part should cancel at this step.)

\item If applicable, multiply both sides by a large enough power of $10$ so that the equation is of the form $ax=b$, where $a$ and $b$ are integers.

\item Divide both sides of the equation by the coefficient of $x$.

\item Reduce the fraction to lowest terms.

\end{enumerate}

Below, this algorithm is demonstrated for $0.58\overline{3}$ with the steps indicated on the far \PMlinkescapetext{right}.

\begin{equation}
x=0.58\overline{3}
\end{equation}

\begin{equation}
10x=5.8\overline{3}
\end{equation}

\begin{equation}
10x=5.83\overline{3}
\end{equation}

\begin{equation}
9x=5.25
\end{equation}

\begin{equation}
900x=525
\end{equation}

\begin{equation}
x=\frac{525}{900}
\end{equation}

\begin{equation}
x=\frac{7}{12}
\end{equation}

An important application of this algorithm is that it supplies a proof for the fact that $0.\overline{9}=1$:

\begin{align*}
x & =0.\overline{9} \\
10x & =9.\overline{9} \\
9x & =9 \\
x & =1
\end{align*}
%%%%%
%%%%%
\end{document}
