\documentclass[12pt]{article}
\usepackage{pmmeta}
\pmcanonicalname{ValuesOf4nAnddisplaystyleprodi1pinPiFor0N26}
\pmcreated{2013-03-22 17:04:51}
\pmmodified{2013-03-22 17:04:51}
\pmowner{PrimeFan}{13766}
\pmmodifier{PrimeFan}{13766}
\pmtitle{values of $4^n$ and $\displaystyle \prod_{i = 1}^{\pi(n)} p_i$ for $0 < n < 26$}
\pmrecord{5}{39374}
\pmprivacy{1}
\pmauthor{PrimeFan}{13766}
\pmtype{Example}
\pmcomment{trigger rebuild}
\pmclassification{msc}{11N05}
\pmclassification{msc}{11A25}
\pmclassification{msc}{11A41}

\endmetadata

% this is the default PlanetMath preamble.  as your knowledge
% of TeX increases, you will probably want to edit this, but
% it should be fine as is for beginners.

% almost certainly you want these
\usepackage{amssymb}
\usepackage{amsmath}
\usepackage{amsfonts}

% used for TeXing text within eps files
%\usepackage{psfrag}
% need this for including graphics (\includegraphics)
%\usepackage{graphicx}
% for neatly defining theorems and propositions
%\usepackage{amsthm}
% making logically defined graphics
%%%\usepackage{xypic}

% there are many more packages, add them here as you need them

% define commands here

\begin{document}
The following table is in \PMlinkescapetext{support} of the statement that $$4^x > \prod_{i = 1}^{\pi(x)} p_i$$ is always true for any nonnegative $x$.

\begin{tabular}{|r|r|r|r|}
$n$ & $4^n$ & $\displaystyle \prod_{i = 1}^{\pi(n)} p_i$ & $\displaystyle 4^n - \prod_{i = 1}^{\pi(n)} p_i$ \\
1 & 4 & 1 & 3 \\ 
2 & 16 & 2 & 14 \\ 
3 & 64 & 6 & 58 \\ 
4 & 256 & 6 & 250 \\ 
5 & 1024 & 30 & 994 \\ 
6 & 4096 & 30 & 4066 \\ 
7 & 16384 & 210 & 16174 \\ 
8 & 65536 & 210 & 65326 \\ 
9 & 262144 & 210 & 261934 \\ 
10 & 1048576 & 210 & 1048366 \\ 
11 & 4194304 & 2310 & 4191994 \\ 
12 & 16777216 & 2310 & 16774906 \\ 
13 & 67108864 & 30030 & 67078834 \\ 
14 & 268435456 & 30030 & 268405426 \\ 
15 & 1073741824 & 30030 & 1073711794 \\ 
16 & 4294967296 & 30030 & 4294937266 \\ 
17 & 17179869184 & 510510 & 17179358674 \\ 
18 & 68719476736 & 510510 & 68718966226 \\ 
19 & 274877906944 & 9699690 & 274868207254 \\ 
20 & 1099511627776 & 9699690 & 1099501928086 \\ 
21 & 4398046511104 & 9699690 & 4398036811414 \\ 
22 & 17592186044416 & 9699690 & 17592176344726 \\ 
23 & 70368744177664 & 223092870 & 70368521084794 \\ 
24 & 281474976710656 & 223092870 & 281474753617786 \\ 
25 & 1125899906842624 & 223092870 & 1125899683749754 \\
\end{tabular}
%%%%%
%%%%%
\end{document}
