\documentclass[12pt]{article}
\usepackage{pmmeta}
\pmcanonicalname{EventuallyCoincide}
\pmcreated{2013-03-22 15:09:30}
\pmmodified{2013-03-22 15:09:30}
\pmowner{mathcam}{2727}
\pmmodifier{mathcam}{2727}
\pmtitle{eventually coincide}
\pmrecord{5}{36907}
\pmprivacy{1}
\pmauthor{mathcam}{2727}
\pmtype{Definition}
\pmcomment{trigger rebuild}
\pmclassification{msc}{11B13}
\pmsynonym{eventually coinciding}{EventuallyCoincide}

\endmetadata

% this is the default PlanetMath preamble.  as your knowledge
% of TeX increases, you will probably want to edit this, but
% it should be fine as is for beginners.

% almost certainly you want these
\usepackage{amssymb}
\usepackage{amsmath}
\usepackage{amsfonts}
\usepackage{amsthm}

% used for TeXing text within eps files
%\usepackage{psfrag}
% need this for including graphics (\includegraphics)
%\usepackage{graphicx}
% for neatly defining theorems and propositions
%\usepackage{amsthm}
% making logically defined graphics
%%%\usepackage{xypic}

% there are many more packages, add them here as you need them

% define commands here

\newcommand{\mc}{\mathcal}
\newcommand{\mb}{\mathbb}
\newcommand{\mf}{\mathfrak}
\newcommand{\ol}{\overline}
\newcommand{\ra}{\rightarrow}
\newcommand{\la}{\leftarrow}
\newcommand{\La}{\Leftarrow}
\newcommand{\Ra}{\Rightarrow}
\newcommand{\nor}{\vartriangleleft}
\newcommand{\Gal}{\text{Gal}}
\newcommand{\GL}{\text{GL}}
\newcommand{\Z}{\mb{Z}}
\newcommand{\R}{\mb{R}}
\newcommand{\Q}{\mb{Q}}
\newcommand{\C}{\mb{C}}
\newcommand{\<}{\langle}
\renewcommand{\>}{\rangle}
\begin{document}
Let $A$ and $B$ be two nonempty sets of integers.  We say that $A$ and $B$ \emph{eventually coincide} if there is an integer $C$ such that $n\in A$ if and only if $n\in B$ for all $n\geq C$.  In this case, we write $A\sim B$, noting that the relation of eventually coinciding is clearly an equivalence relation.  While a seemingly trivial notation, this turns out to be the ``right'' notion of \PMlinkescapetext{equivalence} of sets when dealing with asymptotic properties such as \PMlinkescapetext{density}.


\begin{thebibliography}{9}
\bibitem{Nathanson}
Nathanson, Melvyn B., \emph{Elementary Methods in Number Theory}, Graduate Texts in Mathematics, Volume 195.  Springer-Verlag, 2000.
\end{thebibliography}
%%%%%
%%%%%
\end{document}
