\documentclass[12pt]{article}
\usepackage{pmmeta}
\pmcanonicalname{LandsbergSchaarRelation}
\pmcreated{2013-03-22 13:23:20}
\pmmodified{2013-03-22 13:23:20}
\pmowner{mathcam}{2727}
\pmmodifier{mathcam}{2727}
\pmtitle{Landsberg-Schaar relation}
\pmrecord{8}{33926}
\pmprivacy{1}
\pmauthor{mathcam}{2727}
\pmtype{Theorem}
\pmcomment{trigger rebuild}
\pmclassification{msc}{11L05}
\pmsynonym{Schaar's identity}{LandsbergSchaarRelation}
\pmrelated{RiemannThetaFunction}

\usepackage{amssymb}
\usepackage{amsmath}
\usepackage{amsfonts}
\begin{document}
\PMlinkescapeword{series} \PMlinkescapeword{integrals}
\PMlinkescapeword{one way} \PMlinkescapeword{functions}
\PMlinkescapeword{arithmetic} \PMlinkescapeword{theory}
\PMlinkescapeword{identity} \PMlinkescapeword{formula}
\PMlinkescapeword{states}
The Landsberg-Schaar relation states that for any positive integers $p$
and $q$: \begin{equation}
\frac{1}{\sqrt{p}}\sum_{n=0}^{p-1}\exp\left(\frac{2\pi in^2q}{p}\right)=
\frac{e^{\pi i/4}}{\sqrt{2q}}\sum_{n=0}^{2q-1}\exp\left(-\frac{\pi in^2p}{2q}\right)
\end{equation}
Although both sides of (1) are mere finite sums,
no one has yet found a proof which uses no infinite
limiting process. One way to prove it is to put
$\tau=2iq/p+\epsilon$, where $\epsilon>0$, in
this identity due to Jacobi:
\begin{equation}
\sum_{n=-\infty}^{+\infty}e^{-\pi n^2\tau}=\frac{1}{\sqrt{\tau}}
\sum_{n=-\infty}^{+\infty}e^{-\pi n^2/\tau}
\end{equation}
\noindent
and let $\epsilon\to 0$. The details can be found \PMlinkname{here}{ProofOfJacobisIdentityForVarthetaFunctions}. The identity (2) is a basic one in the theory of
theta functions. It is sometimes called the functional equation for the Riemann theta function. See e.g. [2 VII.6.2].

If we just let $q=1$ in the Landsberg-Schaar identity, it reduces to a formula
for the quadratic Gauss sum mod $p$; notice that $p$ need not be prime.

\textbf{References:}

\noindent
[1] H. Dym and H.P. McKean. \emph{Fourier Series and Integrals}. Academic Press, 1972.

\noindent
[2] J.-P. Serre. \emph{A Course in Arithmetic}. Springer, 1970.
%%%%%
%%%%%
\end{document}
