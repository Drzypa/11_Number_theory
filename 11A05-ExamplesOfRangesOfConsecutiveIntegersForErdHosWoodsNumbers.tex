\documentclass[12pt]{article}
\usepackage{pmmeta}
\pmcanonicalname{ExamplesOfRangesOfConsecutiveIntegersForErdHosWoodsNumbers}
\pmcreated{2013-03-22 17:38:16}
\pmmodified{2013-03-22 17:38:16}
\pmowner{PrimeFan}{13766}
\pmmodifier{PrimeFan}{13766}
\pmtitle{examples of ranges of consecutive integers for Erd\H{o}s-Woods numbers}
\pmrecord{5}{40059}
\pmprivacy{1}
\pmauthor{PrimeFan}{13766}
\pmtype{Example}
\pmcomment{trigger rebuild}
\pmclassification{msc}{11A05}

\endmetadata

% this is the default PlanetMath preamble.  as your knowledge
% of TeX increases, you will probably want to edit this, but
% it should be fine as is for beginners.

% almost certainly you want these
\usepackage{amssymb}
\usepackage{amsmath}
\usepackage{amsfonts}

% used for TeXing text within eps files
%\usepackage{psfrag}
% need this for including graphics (\includegraphics)
%\usepackage{graphicx}
% for neatly defining theorems and propositions
%\usepackage{amsthm}
% making logically defined graphics
%%%\usepackage{xypic}

% there are many more packages, add them here as you need them

% define commands here

\begin{document}
The most famous example of 16 as an \PMlinkname{Erd\H{o}s-Woods number}{ErdHosWoodsNumber} is the range of 16 consecutive integers starting with 2184.

Another $n$ for $k = 16$ is 2044224, which we obtained by multiplying 2184 by 936. The factorization is $2044224 = 2^6 \times 3^3 \times 7 \times 13^2$, while $2044224 + 16 = 2044240 = 2^4 \times 5 \times 11 \times 23 \times 101$. The table of factorizations

\begin{tabular}{|r|l|}
2044225 & $5^2 \times 81769$ \\
2044226 & $2 \times 1022113$ \\
2044227 & $3 \times 681409$ \\
2044228 & $2^2 \times 511057$ \\
2044229 & $11 \times 19 \times 9781$ \\
2044230 & $2 \times 3 \times 5 \times 68141$ \\
2044231 & $7^2 \times 41719$ \\
2044232 & $2^3 \times 59 \times 61 \times 71$ \\
2044233 & $3^2 \times 17 \times 31 \times 431$ \\
2044234 & $2 \times 1009 \times 1013$ \\
2044235 & $5 \times 107 \times 3821$ \\
2044236 & $2^2 \times 3 \times 170353$ \\
2044237 & $13 \times 67 \times 2347$ \\
2044238 & $2 \times 7 \times 151 \times 967$ \\
2044239 & $3 \times 29 \times 23497$ \\
\end{tabular}

shows that each of the numbers in this range shares at least one factor with one if not both of the numbers capping the range.

Next we have a slightly longer example, this one for $k = 34$. The smallest matching $n$ is 47563752566, a squarefree number with a factorization of $2 \times 11 \times 17 \times 23 \times 41 \times 157 \times 859$. The number capping the end of the range is the decidedly non-squarefree 47563752600, with a factorization of $2^3 \times 3^2 \times 5^2 \times 7 \times 13 \times 17 \times 19 \times 29 \times 31$. While the size of these numbers forbids verification on your typical pocket calculator, these numbers are well within the reach of a Javascript implementation of trial division. Here we could be tempted to omit the even numbers, as they obviously share 2 as a prime factor with the range start and the range end, as well as multiples of 3 or 5 as they thus share factors with the range end. But, on the hope that it turns out to be at least a little bit instructive, the factorizations of all the numbers in our chosen range is given.

\begin{tabular}{|r|l|}
47563752567 & $3 \times 3719 \times 4263131$ \\
47563752568 & $2^3 \times 71 \times 199 \times 420799$ \\
47563752569 & $31 \times 163 \times 9412973$ \\
47563752570 & $2 \times 3 \times 5 \times 1585458419$ \\
47563752571 & $29 \times 12941 \times 126739$ \\
47563752572 & $2^2 \times 7^2 \times 242672207$ \\
47563752573 & $3^2 \times 4657 \times 1134821$ \\
47563752574 & $2 \times 13 \times 823 \times 991 \times 2243$ \\
47563752575 & $5^2 \times 31769 \times 59887$ \\
47563752576 & $2^7 \times 3 \times 349 \times 354911$ \\
47563752577 & $11 \times 397 \times 593 \times 18367$ \\
47563752578 & $2 \times 173 \times 137467493$ \\
47563752579 & $3 \times 7 \times 3257 \times 695407$ \\
47563752580 & $2^2 \times 5 \times 83 \times 617 \times 46439$ \\
47563752581 & $19 \times 2503355399$ \\
47563752582 & $2 \times 3^4 \times 53 \times 59 \times 93893$ \\
47563752583 & $17 \times 43 \times 5171 \times 12583$ \\
47563752584 & $2^3 \times 149 \times 39902477$ \\
47563752585 & $3 \times 5 \times 67 \times 47327117$ \\
47563752586 & $2 \times 7 \times 3397410899$ \\
47563752587 & $13^2 \times 281442323$ \\
47563752588 & $2^2 \times 3 \times 11 \times 4513 \times 79843$ \\
47563752589 & $23 \times 61 \times 151 \times 224513$ \\
47563752590 & $2 \times 5 \times 4756375259$ \\
47563752591 & $3^2 \times 5284861399$ \\
47563752592 & $2^4 \times 47 \times 63249671$ \\
47563752593 & $7 \times 6794821799$ \\
47563752594 & $2 \times 3 \times 7927292099$ \\
47563752595 & $5 \times 32503 \times 292673$ \\
47563752596 & $2^2 \times 11890938149$ \\
47563752597 & $3 \times 15854584199$ \\
47563752598 & $2 \times 23781876299$ \\
47563752599 & $11 \times 37 \times 127 \times 373 \times 2467$ \\
\end{tabular} 
%%%%%
%%%%%
\end{document}
