\documentclass[12pt]{article}
\usepackage{pmmeta}
\pmcanonicalname{EveryPositiveIntegerGreaterThan30HasAtLeastOneCompositeTotative}
\pmcreated{2013-03-22 16:58:19}
\pmmodified{2013-03-22 16:58:19}
\pmowner{mps}{409}
\pmmodifier{mps}{409}
\pmtitle{every positive integer greater than 30 has at least one composite totative}
\pmrecord{7}{39247}
\pmprivacy{1}
\pmauthor{mps}{409}
\pmtype{Result}
\pmcomment{trigger rebuild}
\pmclassification{msc}{11A25}
\pmrelated{SmallIntegersThatAreOrMightBeTheLargestOfTheirKind}

% this is the default PlanetMath preamble.  as your knowledge
% of TeX increases, you will probably want to edit this, but
% it should be fine as is for beginners.

% almost certainly you want these
\usepackage{amssymb}
\usepackage{amsmath}
\usepackage{amsfonts}

% used for TeXing text within eps files
%\usepackage{psfrag}
% need this for including graphics (\includegraphics)
%\usepackage{graphicx}
% for neatly defining theorems and propositions
\usepackage{amsthm}
% making logically defined graphics
%%%\usepackage{xypic}

% there are many more packages, add them here as you need them

% define commands here
\newtheorem*{proposition*}{Proposition}
\begin{document}
\begin{proposition*}
Every positive integer greater than 30 has at least one composite
totative.
\end{proposition*}

\begin{proof}
Suppose we are given a positive integer $n$ which is greater than 30.
Let $p$ be the smallest prime number which does not divide $n$.  Hence
$\gcd(n,p^2) = 1$.  If $n\le 50$, then $p < 7$, so $p^2 \le 25 < n$.
But if $n>50$ and $p\le 7$, then $p^2 < 50 < n$.  In either case we
get that $p^2$ is a composite totative of $n$.

So now suppose $p > 7$.  Then $p = p_k$ for some $k>4$.  To complete
the proof, it is enough to show that $p^2$ is strictly smaller than
the primorial $(k-1)\# = p_1 p_2\cdots p_{k-1}$, which by assumption
divides $n$.  For then we would have $\gcd(n, p^2) = 1$ and $p^2 < n$,
showing that $p^2$ is a composite totative of $n$.

We now prove by induction that for any $k>4$, the inequality $p_k^2 <
(k-1)\#$ holds.  For the base case $k=5$ we need to verify that
\[
p_5^2 = 121 < 210 = 4\#.
\]
Now suppose $p_k^2 < (k-1)\#$ for some $k>4$.  By Bertrand's
postulate, $p_{k+1} < 2p_k$, so applying the induction assumption, we
get that
\[
p_{k+1}^2 < 4p_k^2 < 4(k-1)\#.
\]
But $4 < k < p_k$, so $p_{k+1}^2 < k\#$ as desired.
\end{proof}

\PMlinkescapeword{argument}
\PMlinkescapeword{base}
\PMlinkescapeword{complete}
%%%%%
%%%%%
\end{document}
