\documentclass[12pt]{article}
\usepackage{pmmeta}
\pmcanonicalname{HurwitzEquation}
\pmcreated{2013-03-22 17:26:13}
\pmmodified{2013-03-22 17:26:13}
\pmowner{PrimeFan}{13766}
\pmmodifier{PrimeFan}{13766}
\pmtitle{Hurwitz equation}
\pmrecord{4}{39816}
\pmprivacy{1}
\pmauthor{PrimeFan}{13766}
\pmtype{Definition}
\pmcomment{trigger rebuild}
\pmclassification{msc}{11J06}
\pmclassification{msc}{11D72}

\endmetadata

% this is the default PlanetMath preamble.  as your knowledge
% of TeX increases, you will probably want to edit this, but
% it should be fine as is for beginners.

% almost certainly you want these
\usepackage{amssymb}
\usepackage{amsmath}
\usepackage{amsfonts}

% used for TeXing text within eps files
%\usepackage{psfrag}
% need this for including graphics (\includegraphics)
%\usepackage{graphicx}
% for neatly defining theorems and propositions
%\usepackage{amsthm}
% making logically defined graphics
%%%\usepackage{xypic}

% there are many more packages, add them here as you need them

% define commands here

\begin{document}
Given a length $k$ and a multiplier $m$, a {\em Hurwitz equation} for a set of integers $x_1, x_2, \ldots , x_k$ states that

$$\sum_{i = 1}^k {x_i}^2 = m\prod_{i = 1}^k x_i.$$

There is no solution to the equation if $m > k$. Solutions can sometimes be found for $m \leq k$ by starting from a $k$-long set of 1s. The Markov equation for the Markov numbers has $k = m = 3$.

\begin{thebibliography}{2}
\bibitem{rg} R. K. Guy, {\it Unsolved Problems in Number Theory} New York: Springer-Verlag 2004: D12
\end{thebibliography}

%%%%%
%%%%%
\end{document}
