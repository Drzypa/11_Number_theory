\documentclass[12pt]{article}
\usepackage{pmmeta}
\pmcanonicalname{BaselProblem}
\pmcreated{2013-03-22 18:05:22}
\pmmodified{2013-03-22 18:05:22}
\pmowner{PrimeFan}{13766}
\pmmodifier{PrimeFan}{13766}
\pmtitle{Basel problem}
\pmrecord{5}{40627}
\pmprivacy{1}
\pmauthor{PrimeFan}{13766}
\pmtype{Definition}
\pmcomment{trigger rebuild}
\pmclassification{msc}{11A25}

% this is the default PlanetMath preamble.  as your knowledge
% of TeX increases, you will probably want to edit this, but
% it should be fine as is for beginners.

% almost certainly you want these
\usepackage{amssymb}
\usepackage{amsmath}
\usepackage{amsfonts}

% used for TeXing text within eps files
%\usepackage{psfrag}
% need this for including graphics (\includegraphics)
%\usepackage{graphicx}
% for neatly defining theorems and propositions
%\usepackage{amsthm}
% making logically defined graphics
%%%\usepackage{xypic}

% there are many more packages, add them here as you need them

% define commands here

\begin{document}
The {\em Basel problem}, first posed by Pietro Mengoli in 1644, asks for a finite formula for the infinite sum $$\sum_{i = 1}^{\infty} \frac{1}{i^2}.$$ Though Mengoli verified the Wallis formulae for $\pi$, it did not occur to him that $\pi$ was also involved in the solution of this problem. Jakob Bernoulli also tried in vain to solve this problem. Even an approximate decimal value eluded contemporary mathematicians: an answer accurate to just five decimal places requires iterating up to at least $i = 112000$, which without the aid of a computer was wholly impractical in Mengoli's day. The problem was finally solved in 1741, when, after almost a decade of work, Leonhard Euler conclusively proved that $$\frac{\pi^2}{6} = \zeta(2) = \sum_{i = 1}^{\infty}\frac{1}{i^2}.$$ The value, 1.6449340668482264365... could then be computed to almost as many decimal places as were known of $\pi$. See \PMlinkname{value of the Riemann zeta function at $s = 2$}{ValueOfTheRiemannZetaFunctionAtS2}

\begin{thebibliography}{1}
\bibitem{es} Ed Sandifer, ``Euler's Solution of the Basel Problem - The Longer Story''. Danbury, Connecticut: Western Connecticut State University (2003)
\end{thebibliography}
%%%%%
%%%%%
\end{document}
