\documentclass[12pt]{article}
\usepackage{pmmeta}
\pmcanonicalname{ThereAreAnInfiniteNumberOfPrimesequivpm1pmod4}
\pmcreated{2013-03-22 16:56:59}
\pmmodified{2013-03-22 16:56:59}
\pmowner{rm50}{10146}
\pmmodifier{rm50}{10146}
\pmtitle{there are an infinite number of primes $\equiv \pm 1\pmod 4$}
\pmrecord{4}{39218}
\pmprivacy{1}
\pmauthor{rm50}{10146}
\pmtype{Theorem}
\pmcomment{trigger rebuild}
\pmclassification{msc}{11N13}

\endmetadata

% this is the default PlanetMath preamble.  as your knowledge
% of TeX increases, you will probably want to edit this, but
% it should be fine as is for beginners.

% almost certainly you want these
\usepackage{amssymb}
\usepackage{amsmath}
\usepackage{amsfonts}

% used for TeXing text within eps files
%\usepackage{psfrag}
% need this for including graphics (\includegraphics)
%\usepackage{graphicx}
% for neatly defining theorems and propositions
\usepackage{amsthm}
% making logically defined graphics
%%%\usepackage{xypic}

% there are many more packages, add them here as you need them

% define commands here
\newcommand{\Nats}{\mathbb{N}}
\newcommand{\Ints}{\mathbb{Z}}
\newcommand{\Reals}{\mathbb{R}}
\newcommand{\Complex}{\mathbb{C}}
\newcommand{\Rats}{\mathbb{Q}}
\newcommand{\Gal}{\operatorname{Gal}}
\newcommand{\Cl}{\operatorname{Cl}}
\newcommand{\Alg}{\mathcal{O}}
\newcommand{\ol}{\overline}
\newcommand{\Leg}[2]{\left(\frac{#1}{#2}\right)}
%
%% \theoremstyle{plain} %% This is the default
\newtheorem{thm}{Theorem}
\newtheorem{cor}[thm]{Corollary}
\newtheorem{lem}[thm]{Lemma}
\newtheorem{prop}[thm]{Proposition}
\newtheorem{ax}{Axiom}

\theoremstyle{definition}
\newtheorem{defn}{Definition}
\begin{document}
\begin{thm} \label{thm:one}There are an infinite number of primes congruent to $3 \pmod 4$.
\end{thm}
\begin{proof}
Choose any prime $p\equiv 3\pmod 4$; we find a prime of that form that exceeds $p$.
\[N=(2^2\cdot 3\cdot 5\cdot 7\cdots p) - 1\]
Clearly $N\equiv 3\pmod 4$, and thus must have at least one prime factor that is also $\equiv 3\pmod 4$. But $N$ is not divisible by any prime less than or equal to $p$, so must be divisible by some prime congruent to $3\pmod 4$ that exceeds $p$.
\end{proof}

\begin{thm} \label{thm:two}There are an infinite number of primes congruent to $1\pmod 4$.
\end{thm}
\begin{proof}
Given $N>1$, we find a prime $p>N$ with $p\equiv 1\pmod 4$. Let $p$ be the smallest (odd) prime factor of $(N!)^2+1$; note that $p>N$. Now
\[(N!)^2\equiv -1\pmod p\]
and therefore
\[(N!)^{p-1}\equiv (-1)^{(p-1)/2}\pmod p.\]
By Fermat's little theorem, $(N!)^{p-1}\equiv 1\pmod p$, so we have
\[(-1)^{(p-1)/2}\equiv 1\pmod p\]
The left-hand side cannot be -1, since then $0\equiv 2\pmod p$. Thus $(-1)^{(p-1)/2}=1$ and it follows that $p\equiv 1\pmod 4$.
\end{proof}

Note that the variant of Euclid's proof of the infinitude of primes used in the proof of Theorem \ref{thm:one} does not work for Theorem \ref{thm:two}, since we cannot conclude that an integer $\equiv 1\pmod 4$ has a factor of the same kind.

\begin{thebibliography}{10}
\bibitem{bib:apostol}
Apostol,~T \emph{Introduction to Analytic Number Theory}, Springer 1976.
\end{thebibliography}
%%%%%
%%%%%
\end{document}
