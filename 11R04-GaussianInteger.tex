\documentclass[12pt]{article}
\usepackage{pmmeta}
\pmcanonicalname{GaussianInteger}
\pmcreated{2013-03-22 11:45:32}
\pmmodified{2013-03-22 11:45:32}
\pmowner{Daume}{40}
\pmmodifier{Daume}{40}
\pmtitle{Gaussian integer}
\pmrecord{11}{30207}
\pmprivacy{1}
\pmauthor{Daume}{40}
\pmtype{Definition}
\pmcomment{trigger rebuild}
\pmclassification{msc}{11R04}
\pmclassification{msc}{55-00}
\pmclassification{msc}{55U05}
\pmclassification{msc}{32M10}
\pmclassification{msc}{32C11}
\pmclassification{msc}{14-02}
\pmclassification{msc}{18-00}
%\pmkeywords{number theory}
\pmrelated{EisensteinIntegers}

\usepackage{amssymb}
\usepackage{amsmath}
\usepackage{amsfonts}
\begin{document}
\newcommand{\Z}{\mathbb{Z}}
\newcommand{\G}{\mathbb{Z}[i]}
\PMlinkescapeword{principal}
A complex number of the form $a+bi$, where $a,b\in\mathbb{Z}$, is called
a Gaussian integer.

It is easy to see that the set $S$ of all Gaussian integers is a subring
of $\mathbb{C}$; specifically, $S$ is the smallest subring containing
$\{1,i\}$, whence $S=\G$.

$\G$ is a Euclidean ring, hence a principal ring, hence a
unique factorization domain.

There are four units (i.e. invertible elements)
in the ring $\G$, namely $\pm 1$ and $\pm i$.
Up to multiplication by units, the primes in $\G$ are
\begin{itemize}
\item ordinary prime numbers $\equiv 3\mod 4$
\item elements of the form $a\pm bi$ where $a^2+b^2$ is an ordinary
prime $\equiv 1\mod 4$ (see Thue's lemma)
\item the element $1+i$.
\end{itemize}

Using the ring of Gaussian integers, it is not hard to show, for example,
that the Diophantine equation $x^2+1=y^3$ has no solutions $(x,y)\in\Z\times\Z$
except $(0,1)$.
%%%%%
%%%%%
%%%%%
%%%%%
\end{document}
