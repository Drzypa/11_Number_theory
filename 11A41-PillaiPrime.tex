\documentclass[12pt]{article}
\usepackage{pmmeta}
\pmcanonicalname{PillaiPrime}
\pmcreated{2013-03-22 16:33:14}
\pmmodified{2013-03-22 16:33:14}
\pmowner{PrimeFan}{13766}
\pmmodifier{PrimeFan}{13766}
\pmtitle{Pillai prime}
\pmrecord{5}{38739}
\pmprivacy{1}
\pmauthor{PrimeFan}{13766}
\pmtype{Definition}
\pmcomment{trigger rebuild}
\pmclassification{msc}{11A41}

\endmetadata

% this is the default PlanetMath preamble.  as your knowledge
% of TeX increases, you will probably want to edit this, but
% it should be fine as is for beginners.

% almost certainly you want these
\usepackage{amssymb}
\usepackage{amsmath}
\usepackage{amsfonts}

% used for TeXing text within eps files
%\usepackage{psfrag}
% need this for including graphics (\includegraphics)
%\usepackage{graphicx}
% for neatly defining theorems and propositions
%\usepackage{amsthm}
% making logically defined graphics
%%%\usepackage{xypic}

% there are many more packages, add them here as you need them

% define commands here

\begin{document}
If for a given prime $p$ we can find an integer $n > 0$ such that $n! \equiv -1 \mod p$ but $p \not\equiv 1 \mod n$ then $p$ is a called a {\em Pillai prime}. These are listed in A063980 of Sloane's OEIS. Sarinya Intaraprasert proved that there are infinitely many Pillai primes. The first few are 23, 29, 59, 61, 67, 71, 79, 83, 109, 137, 139, 149, 193, \ldots

\begin{thebibliography}{2}
\bibitem{rg} R. K. Guy, {\it Unsolved Problems in Number Theory} New York: Springer-Verlag 2004: A2
\end{thebibliography}
%%%%%
%%%%%
\end{document}
