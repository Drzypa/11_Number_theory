\documentclass[12pt]{article}
\usepackage{pmmeta}
\pmcanonicalname{TableOfValuesOfTheMobiusFunctionAndTheMertensFunction}
\pmcreated{2013-03-22 18:06:27}
\pmmodified{2013-03-22 18:06:27}
\pmowner{PrimeFan}{13766}
\pmmodifier{PrimeFan}{13766}
\pmtitle{table of values of the M\"obius function and the Mertens function}
\pmrecord{5}{40652}
\pmprivacy{1}
\pmauthor{PrimeFan}{13766}
\pmtype{Data Structure}
\pmcomment{trigger rebuild}
\pmclassification{msc}{11A25}

% this is the default PlanetMath preamble.  as your knowledge
% of TeX increases, you will probably want to edit this, but
% it should be fine as is for beginners.

% almost certainly you want these
\usepackage{amssymb}
\usepackage{amsmath}
\usepackage{amsfonts}

% used for TeXing text within eps files
%\usepackage{psfrag}
% need this for including graphics (\includegraphics)
%\usepackage{graphicx}
% for neatly defining theorems and propositions
%\usepackage{amsthm}
% making logically defined graphics
%%%\usepackage{xypic}

% there are many more packages, add them here as you need them

% define commands here

\begin{document}
The following table lists the values of the M\"obius function $\mu(n)$ and the Mertens function $M(n)$ for $0 < n < 101$. The M\"obius function is defined as $\mu(n) = (-1)^{\omega(n)}$ (where $\omega(n)$ is the number of distinct prime factors function) for squarefree numbers, and $\mu(n) = 0$ for any integer with a repeated prime factor. The Mertens function is the matching summatory function for the M\"obius function, $$M(n) = \sum_{i = 1}^n \mu(i).$$

\begin{tabular}{|r|r|r|r|r|r|r|r|r|r|r|r|}
$n$ & $\mu(n)$ & $M(n)$ & $n$ & $\mu(n)$ & $M(n)$ & $n$ & $\mu(n)$ & $M(n)$ & $n$ & $\mu(n)$ & $M(n)$ \\
1 & 1 & 1 & 26 & 1 & $-1$ & 51 & 1 & $-2$ & 76 & 0 & $-3$ \\
2 & $-1$ & 0 & 27 & 0 & $-1$ & 52 & 0 & $-2$ & 77 & 1 & $-2$ \\
3 & $-1$ & $-1$ & 28 & 0 & $-1$ & 53 & $-1$ & $-3$ & 78 & $-1$ & $-3$ \\
4 & 0 & $-1$ & 29 & $-1$ & $-2$ & 54 & 0 & $-3$ & 79 & $-1$ & $-4$ \\
5 & $-1$ & $-2$ & 30 & $-1$ & $-3$ & 55 & 1 & $-2$ & 80 & 0 & $-4$ \\
6 & 1 & $-1$ & 31 & $-1$ & $-4$ & 56 & 0 & $-2$ & 81 & 0 & $-4$ \\
7 & $-1$ & $-2$ & 32 & 0 & $-4$ & 57 & 1 & $-1$ & 82 & 1 & $-3$ \\
8 & 0 & $-2$ & 33 & 1 & $-3$ & 58 & 1 & 0 & 83 & $-1$ & $-4$ \\
9 & 0 & $-2$ & 34 & 1 & $-2$ & 59 & $-1$ & $-1$ & 84 & 0 & $-4$ \\
10 & 1 & $-1$ & 35 & 1 & $-1$ & 60 & 0 & $-1$ & 85 & 1 & $-3$ \\
11 & $-1$ & $-2$ & 36 & 0 & $-1$ & 61 & $-1$ & $-2$ & 86 & 1 & $-2$ \\
12 & 0 & $-2$ & 37 & $-1$ & $-2$ & 62 & 1 & $-1$ & 87 & 1 & $-1$ \\
13 & $-1$ & $-3$ & 38 & 1 & $-1$ & 63 & 0 & $-1$ & 88 & 0 & $-1$ \\
14 & 1 & $-2$ & 39 & 1 & 0 & 64 & 0 & $-1$ & 89 & $-1$ & $-2$ \\
15 & 1 & $-1$ & 40 & 0 & 0 & 65 & 1 & 0 & 90 & 0 & $-2$ \\
16 & 0 & $-1$ & 41 & $-1$ & $-1$ & 66 & $-1$ & $-1$ & 91 & 1 & $-1$ \\
17 & $-1$ & $-2$ & 42 & $-1$ & $-2$ & 67 & $-1$ & $-2$ & 92 & 0 & $-1$ \\
18 & 0 & $-2$ & 43 & $-1$ & $-3$ & 68 & 0 & $-2$ & 93 & 1 & 0 \\
19 & $-1$ & $-3$ & 44 & 0 & $-3$ & 69 & 1 & $-1$ & 94 & 1 & 1 \\
20 & 0 & $-3$ & 45 & 0 & $-3$ & 70 & $-1$ & $-2$ & 95 & 1 & 2 \\
21 & 1 & $-2$ & 46 & 1 & $-2$ & 71 & $-1$ & $-3$ & 96 & 0 & 2 \\
22 & 1 & $-1$ & 47 & $-1$ & $-3$ & 72 & 0 & $-3$ & 97 & $-1$ & 1 \\
23 & $-1$ & $-2$ & 48 & 0 & $-3$ & 73 & $-1$ & $-4$ & 98 & 0 & 1 \\
24 & 0 & $-2$ & 49 & 0 & $-3$ & 74 & 1 & $-3$ & 99 & 0 & 1 \\
25 & 0 & $-2$ & 50 & 0 & $-3$ & 75 & 0 & $-3$ & 100 & 0 & 1 \\
\end{tabular}

%%%%%
%%%%%
\end{document}
