\documentclass[12pt]{article}
\usepackage{pmmeta}
\pmcanonicalname{ExtensionsWithoutUnramifiedSubextensionsAndClassNumberDivisibility}
\pmcreated{2013-03-22 15:07:35}
\pmmodified{2013-03-22 15:07:35}
\pmowner{alozano}{2414}
\pmmodifier{alozano}{2414}
\pmtitle{extensions without unramified subextensions and class number divisibility}
\pmrecord{4}{36866}
\pmprivacy{1}
\pmauthor{alozano}{2414}
\pmtype{Theorem}
\pmcomment{trigger rebuild}
\pmclassification{msc}{11R29}
\pmclassification{msc}{11R32}
\pmclassification{msc}{11R37}
%\pmkeywords{divisibility}
%\pmkeywords{class number}
%\pmkeywords{tower of number fields}
\pmrelated{PushDownTheoremOnClassNumbers}
\pmrelated{ClassNumberDivisibilityInExtensions}
\pmrelated{IdealClass}
\pmrelated{ExistenceOfHilbertClassField}
\pmrelated{CompositumOfAGaloisExtensionAndAnotherExtensionIsGalois}
\pmrelated{DecompositionGroup}

\endmetadata

% this is the default PlanetMath preamble.  as your knowledge
% of TeX increases, you will probably want to edit this, but
% it should be fine as is for beginners.

% almost certainly you want these
\usepackage{amssymb}
\usepackage{amsmath}
\usepackage{amsthm}
\usepackage{amsfonts}

% used for TeXing text within eps files
%\usepackage{psfrag}
% need this for including graphics (\includegraphics)
%\usepackage{graphicx}
% for neatly defining theorems and propositions
%\usepackage{amsthm}
% making logically defined graphics
%%%\usepackage{xypic}

% there are many more packages, add them here as you need them

% define commands here

\newtheorem{thm}{Theorem}
\newtheorem{defn}{Definition}
\newtheorem{prop}{Proposition}
\newtheorem{lemma}{Lemma}
\newtheorem{cor}{Corollary}

% Some sets
\newcommand{\Nats}{\mathbb{N}}
\newcommand{\Ints}{\mathbb{Z}}
\newcommand{\Reals}{\mathbb{R}}
\newcommand{\Complex}{\mathbb{C}}
\newcommand{\Rats}{\mathbb{Q}}
\newcommand{\Gal}{\operatorname{Gal}}
\newcommand{\Cl}{\operatorname{Cl}}
\begin{document}
\begin{thm}
Let $F/K$ be an extension of number fields such that for any intermediate Galois extension $L/K$, with $K\subsetneq L \subsetneq F$, there is at least one finite place or infinite place which ramifies in the extension $L/K$. Then, $h_K$, the class number of $K$, divides the class number of $F$, $h_F$.
\end{thm}

First, we deduce some immediate corollaries.

\begin{cor}
Let $F/K$ be an extension of number fields which is totally ramified at some prime (or at an archimedean place). Then $h_K$ divides $h_F$. 
\end{cor}
\begin{proof}
The proof is clear since there cannot be unramified subextensions. The theorem applies.
\end{proof}

\begin{cor}
Let $F/K$ be a Galois extension of number fields such that $\Gal(F/K)$ is a non-abelian simple group. Then $h_K$ divides $h_F$.
\end{cor}
\begin{proof}
In this case, there cannot be subextensions with abelian Galois group and the theorem applies.
\end{proof}

\begin{proof}[Proof of the Theorem]
Let $H$ be the Hilbert class field of $K$. By definition, $H$ is the maximal unramified abelian extension of $K$, $\Gal(H/K)$ is isomorphic to $\Cl(K)$, the ideal class group of $K$ and $[H:K]=h_K$. Since there are no nontrivial unramified abelian subextensions of $F/K$, we have $F\cap H=K$ and so $[FH:F]=[H:K]=h_K$. One can show that the extension $FH/F$ is unramified and abelian (in fact $\Gal(FH/F)\cong \Gal(H/K)$). Therefore $FH$ is contained in $L$, the Hilbert class field of $F$. Hence:
$$h_F=[L:F]=[L:FH]\cdot[FH:F]=[L:FH]\cdot [H:K]=[L:FH]\cdot h_K$$
and so, $h_K$ divides $h_F$.
\end{proof}
%%%%%
%%%%%
\end{document}
