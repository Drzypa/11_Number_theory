\documentclass[12pt]{article}
\usepackage{pmmeta}
\pmcanonicalname{EuclidsCoefficients}
\pmcreated{2013-03-22 15:15:20}
\pmmodified{2013-03-22 15:15:20}
\pmowner{pahio}{2872}
\pmmodifier{pahio}{2872}
\pmtitle{Euclid's coefficients}
\pmrecord{16}{37038}
\pmprivacy{1}
\pmauthor{pahio}{2872}
\pmtype{Algorithm}
\pmcomment{trigger rebuild}
\pmclassification{msc}{11A05}
\pmclassification{msc}{03-04}
%\pmkeywords{integers}
%\pmkeywords{gcd}
\pmrelated{BezoutsLemma}

\endmetadata

% this is the default PlanetMath preamble.  as your knowledge
% of TeX increases, you will probably want to edit this, but
% it should be fine as is for beginners.

% almost certainly you want these
\usepackage{amssymb}
\usepackage{amsmath}
\usepackage{amsfonts}

% used for TeXing text within eps files
%\usepackage{psfrag}
% need this for including graphics (\includegraphics)
%\usepackage{graphicx}
% for neatly defining theorems and propositions
 \usepackage{amsthm}
% making logically defined graphics
%%%\usepackage{xypic}

% there are many more packages, add them here as you need them

% define commands here

\theoremstyle{definition}
\newtheorem*{thmplain}{Theorem}

\begin{document}
The following program is based on Euclid's algorithm and it determines {\em Euclid's coefficients} $t$ and $d$ of the integers $x$ and $y$.

INPUT $x$, $y$ \quad  (positive integers)\\
$m := x$, \quad  $n := y$\\
$b := 0$, \quad  $d := 1$\\
$p := 1$, \quad  $t := 0$\\
WHILE \, $m \neq 0$ \, DO\\
BEGIN\\
$q := n$ DIV $m$ \quad (integer division)\\
$h := m$, \quad  $m := n-qm$, \quad  $n := h$\\
$h := b$, \quad  $b := d-qb$, \quad  $d := h$\\
$h := p$, \quad  $p := t-qp$, \quad  $t := h$\\
END\\
WRITE  \,\,--- The gcd of the numbers $x$ and $y$ is\,\, $n = tx+dy$.

\textbf{Remark.} \,The values of $t$ and $d$ produced by the program have the absolute values as close each other as possible (Proof = ?). \,They differ very often only by 1 when $x$ and $y$ are two successive prime numbers, e.g.
     $$1 \,=\, 30\cdot523-29\cdot541.$$

%%%%%
%%%%%
\end{document}
