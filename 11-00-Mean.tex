\documentclass[12pt]{article}
\usepackage{pmmeta}
\pmcanonicalname{Mean}
\pmcreated{2013-03-22 12:43:43}
\pmmodified{2013-03-22 12:43:43}
\pmowner{matte}{1858}
\pmmodifier{matte}{1858}
\pmtitle{mean}
\pmrecord{16}{33028}
\pmprivacy{1}
\pmauthor{matte}{1858}
\pmtype{Definition}
\pmcomment{trigger rebuild}
\pmclassification{msc}{11-00}
\pmclassification{msc}{62-07}
\pmrelated{ArithmeticMean}
\pmrelated{GeometricMean}
\pmrelated{ContraharmonicProportion}
\pmrelated{OrderOfSixMeans}
\pmrelated{AverageValueOfFunction}

\endmetadata

% this is the default PlanetMath preamble.  as your knowledge
% of TeX increases, you will probably want to edit this, but
% it should be fine as is for beginners.

% almost certainly you want these
\usepackage{amssymb}
\usepackage{amsmath}
\usepackage{amsfonts}

% used for TeXing text within eps files
%\usepackage{psfrag}
% need this for including graphics (\includegraphics)
%\usepackage{graphicx}
% for neatly defining theorems and propositions
%\usepackage{amsthm}
% making logically defined graphics
%%%\usepackage{xypic} 

% there are many more packages, add them here as you need them

% define commands here
\begin{document}
Loosely speaking, a mean is a way to describe a collection of numbers such
that the mean in some sense describe the ``average'' entry of these numbers. 
The most familiar mean is the arithmetic mean, and unless otherwise noted, by mean,
we always mean the arithmetic mean. 

\subsubsection*{Example} 
The mean of the numbers $\{1,\,2,\,\ldots,\,n\}$ is $\frac{n+1}{2}$. 

Mathematically, we define a mean as follows:

\subsubsection*{Definition}
A \emph{mean} is a function $f$ whose domain is the collection of
all finite multisets of $\mathbb{R}$ and whose codomain is $\mathbb{R}$, 
such that 
\begin{itemize}
\item $f$ is a homogeneous function of degree 1.\, That is, if $\{x_1, \ldots, x_n\}$ is a multiset, then 
$$ 
  f(\{ \lambda x_1, \ldots, \lambda x_n\}) = \lambda f(\{x_1, \ldots, x_n\}),\quad \lambda\ge 0.
$$
\item For any set $S = \{x_1,\ldots,x_n\}$ of real numbers,
$$ \min\{x_1,\ldots,x_n\} \leq f(S) \leq \max\{x_1,\ldots,x_n\}.$$
\end{itemize}

Pythagoras identified three types of means: the \PMlinkname{arithmetic mean}{ArithmeticMean}, the geometric
mean, and the harmonic mean. However, in the sense of the above definition, 
there is a wealth of ther means too. For instance, the minimum function and maximum
functions can be seen as ``trivial'' means. Other well-known means include: 

\begin{itemize}
\item median, 
\item mode, 
\item generalized mean
\item power mean
\item Lehmer mean
\item arithmetic-geometric mean, 
\item arithmetic-harmonic mean, 
\item harmonic-geometric mean, 
\item root-mean-square (sometimes called the quadratic mean), 
\item identric mean, 
\item contraharmonic mean,
\item Heronian mean, 
\item Cesaro mean,
\item \PMlinkname{maximum function, minimum function}{MinimalAndMaximalNumber}
\end{itemize}
%%%%%
%%%%%
\end{document}
