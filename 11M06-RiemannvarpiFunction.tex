\documentclass[12pt]{article}
\usepackage{pmmeta}
\pmcanonicalname{RiemannvarpiFunction}
\pmcreated{2013-03-22 13:24:12}
\pmmodified{2013-03-22 13:24:12}
\pmowner{rspuzio}{6075}
\pmmodifier{rspuzio}{6075}
\pmtitle{Riemann $\varpi$  function}
\pmrecord{12}{33945}
\pmprivacy{1}
\pmauthor{rspuzio}{6075}
\pmtype{Definition}
\pmcomment{trigger rebuild}
\pmclassification{msc}{11M06}

\endmetadata

% this is the default PlanetMath preamble.  as your knowledge
% of TeX increases, you will probably want to edit this, but
% it should be fine as is for beginners.

% almost certainly you want these
\usepackage{amssymb}
\usepackage{amsmath}
\usepackage{amsfonts}

% used for TeXing text within eps files
%\usepackage{psfrag}
% need this for including graphics (\includegraphics)
%\usepackage{graphicx}
% for neatly defining theorems and propositions
%\usepackage{amsthm}
% making logically defined graphics
%%%\usepackage{xypic}

% there are many more packages, add them here as you need them

% define commands here
\begin{document}
The Riemann $\varpi$ function is used in the proof of the analytic continuation for the Riemann Xi function to the whole complex plane. It is defined as:

\[
\varpi(x) = \sum_{n=1}^{\infty} e^{-n^2 \pi x}
\]

This function is a special case of a \PMlinkname{Jacobi $\vartheta$ function}{JacobiVarthetaFunctions}:
\[
\varpi (x) = \vartheta_3 ( 0 | i x)
\]

As such the $\varpi$ function satisfies a functional equation, which a special case of \PMlinkname{Jacobi's Identity for the $\vartheta$ function}{JacobisIdentityForVarthetaFunctions}.
%%%%%
%%%%%
\end{document}
