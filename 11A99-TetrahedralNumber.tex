\documentclass[12pt]{article}
\usepackage{pmmeta}
\pmcanonicalname{TetrahedralNumber}
\pmcreated{2013-03-22 15:56:34}
\pmmodified{2013-03-22 15:56:34}
\pmowner{PrimeFan}{13766}
\pmmodifier{PrimeFan}{13766}
\pmtitle{tetrahedral number}
\pmrecord{7}{37952}
\pmprivacy{1}
\pmauthor{PrimeFan}{13766}
\pmtype{Definition}
\pmcomment{trigger rebuild}
\pmclassification{msc}{11A99}
\pmsynonym{triangular pyramidal number}{TetrahedralNumber}

\endmetadata

% this is the default PlanetMath preamble.  as your knowledge
% of TeX increases, you will probably want to edit this, but
% it should be fine as is for beginners.

% almost certainly you want these
\usepackage{amssymb}
\usepackage{amsmath}
\usepackage{amsfonts}

% used for TeXing text within eps files
%\usepackage{psfrag}
% need this for including graphics (\includegraphics)
%\usepackage{graphicx}
% for neatly defining theorems and propositions
%\usepackage{amsthm}
% making logically defined graphics
%%%\usepackage{xypic}

% there are many more packages, add them here as you need them

% define commands here

\begin{document}
An integer of the form $${{(n^2 + n)(n + 2)} \over 6},$$ where $n$ is a nonnegative integer. Sometimes referred to as $T_n$, {\em tetrahedral numbers} are listed in A000292 of Sloane's OEIS. $2|T_n$ except when $n \equiv 1 \mod 4$.

With $t_n$ the $n$th triangular number, the $n$th tetrahedral number can be calculated with this formula: $$T_n = \sum_{i = 1}^n t_i.$$ Another way to calculate tetrahedral numbers is with the binomial coefficient $$T_n={n+2\choose3}.$$ This means that tetrahedral numbers can be looked up in Pascal's triangle.

Tetrahedral numbers have practical applications in sphere packing.
%%%%%
%%%%%
\end{document}
