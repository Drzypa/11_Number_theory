\documentclass[12pt]{article}
\usepackage{pmmeta}
\pmcanonicalname{AlmostPerfectNumber}
\pmcreated{2013-03-22 17:41:51}
\pmmodified{2013-03-22 17:41:51}
\pmowner{PrimeFan}{13766}
\pmmodifier{PrimeFan}{13766}
\pmtitle{almost perfect number}
\pmrecord{4}{40139}
\pmprivacy{1}
\pmauthor{PrimeFan}{13766}
\pmtype{Definition}
\pmcomment{trigger rebuild}
\pmclassification{msc}{11A05}
\pmsynonym{least deficient number}{AlmostPerfectNumber}
\pmrelated{QuasiperfectNumber}

% this is the default PlanetMath preamble.  as your knowledge
% of TeX increases, you will probably want to edit this, but
% it should be fine as is for beginners.

% almost certainly you want these
\usepackage{amssymb}
\usepackage{amsmath}
\usepackage{amsfonts}

% used for TeXing text within eps files
%\usepackage{psfrag}
% need this for including graphics (\includegraphics)
%\usepackage{graphicx}
% for neatly defining theorems and propositions
%\usepackage{amsthm}
% making logically defined graphics
%%%\usepackage{xypic}

% there are many more packages, add them here as you need them

% define commands here

\begin{document}
An {\em almost perfect number} or {\em least deficient number} is a number $n$ whose proper divisors add up to just one less than itself. That is, $\sigma(n) - n = n - 1$, with $\sigma(n)$ being the sum of divisors function. Currently, the only known almost perfect numbers are the integer powers of 2 (e.g., 1, 2, 4, 8, 16, 32, 64, 128, etc.) No one has been able to prove that there are almost perfect numbers of a different form.
%%%%%
%%%%%
\end{document}
