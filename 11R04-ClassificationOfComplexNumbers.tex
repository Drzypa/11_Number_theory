\documentclass[12pt]{article}
\usepackage{pmmeta}
\pmcanonicalname{ClassificationOfComplexNumbers}
\pmcreated{2013-03-22 16:56:49}
\pmmodified{2013-03-22 16:56:49}
\pmowner{pahio}{2872}
\pmmodifier{pahio}{2872}
\pmtitle{classification of complex numbers}
\pmrecord{11}{39215}
\pmprivacy{1}
\pmauthor{pahio}{2872}
\pmtype{Topic}
\pmcomment{trigger rebuild}
\pmclassification{msc}{11R04}
\pmrelated{NegativeNumber}
\pmrelated{Number}

% this is the default PlanetMath preamble.  as your knowledge
% of TeX increases, you will probably want to edit this, but
% it should be fine as is for beginners.

% almost certainly you want these
\usepackage{amssymb}
\usepackage{amsmath}
\usepackage{amsfonts}

% used for TeXing text within eps files
%\usepackage{psfrag}
% need this for including graphics (\includegraphics)
%\usepackage{graphicx}
% for neatly defining theorems and propositions
 \usepackage{amsthm}
% making logically defined graphics
%%%\usepackage{xypic}

% there are many more packages, add them here as you need them

% define commands here

\theoremstyle{definition}
\newtheorem*{thmplain}{Theorem}

\begin{document}
\PMlinkescapeword{contain} \PMlinkescapeword{imaginary numbers}

The set $\mathbb{C}$ of all complex numbers and many of its subsets may be 
partitioned (classified) into two subsets by certain criterion of the numbers.\\

\textbf{A.}\, F i r s t \, c l a s s i f i c a t i o n :\\

\textbf{Complex numbers} contain
\begin{enumerate}
\item algebraic numbers
\item transcendental numbers
\end{enumerate}

\textbf{Algebraic numbers} contain
\begin{enumerate}
\item algebraic integers (\PMlinkescapetext{entire} algebraic numbers)
\item algebraic fractions (fractional algebraic numbers)
\end{enumerate}

\textbf{Algebraic integers} contain
\begin{enumerate}
\item rational integers
\item non-rational integers
\end{enumerate}

\textbf{Algebraic fractions} contain
\begin{enumerate}
\item rational fractions
\item non-rational fractions
\end{enumerate}

\textbf{Transcendental numbers} contain
\begin{enumerate}
\item real transcendental numbers
\item imaginary transcendental numbers
\end{enumerate}
$$ $$\\
\textbf{B.}\; S e c o n d \, c l a s s i f i c a t i o n :\\

\textbf{Complex numbers} contain
\begin{enumerate}
\item \PMlinkname{real numbers}{RealNumber} (the set $\mathbb{R}$)
\item imaginary numbers (i.e. non-real complex numbers)
\end{enumerate}

\textbf{Real numbers} contain
\begin{enumerate}
\item rational numbers (the set $\mathbb{Q}$)
\item irrational numbers
\end{enumerate}

\textbf{Rational numbers} contain
\begin{enumerate}
\item \PMlinkname{integers}{Integer} (the set $\mathbb{Z}$)
\item fractional numbers
\end{enumerate}

\textbf{Imaginary numbers} contain
\begin{enumerate}
\item pure imaginary numbers (with real part 0)
\item other imaginary numbers (with real part $\neq 0$)
\end{enumerate}

$$ $$ 
One can also combine the criterions of \textbf{A} and \textbf{B}; thus e.g. 
the irrational numbers consist of the algebraic irrational numbers and 
the \PMlinkescapetext{transcendental} irrational numbers.

In \PMlinkescapetext{addition}, any of the sets $\mathbb{R}$, 
$\mathbb{Q}$ and $\mathbb{Z}$ may be partitioneded into positive numbers, negative numbers and \PMlinkname{0}{Null}.

Number-theoretically, the set $\mathbb{Z}$ consists of four \PMlinkescapetext{types} of integers:\\
$1^\mathrm{o}$\; the number 0,\\
$2^\mathrm{o}$\; the units of $\mathbb{Z}$ (only $+1$ and $-1$),\\
$3^\mathrm{o}$\; the prime numbers\, ($\pm2,\,\pm3,\,\pm5,\,\pm7,\,\pm11,\,\ldots$),\\
$4^\mathrm{o}$\; the composite numbers\, ($\pm4,\,\pm6,\,\pm8,\,\pm9,\,\pm10,\,\ldots$)



%%%%%
%%%%%
\end{document}
