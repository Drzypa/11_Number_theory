\documentclass[12pt]{article}
\usepackage{pmmeta}
\pmcanonicalname{NumberOfPrimeIdealsInANumberField}
\pmcreated{2013-03-22 19:12:51}
\pmmodified{2013-03-22 19:12:51}
\pmowner{pahio}{2872}
\pmmodifier{pahio}{2872}
\pmtitle{number of prime ideals in a number field}
\pmrecord{4}{42133}
\pmprivacy{1}
\pmauthor{pahio}{2872}
\pmtype{Theorem}
\pmcomment{trigger rebuild}
\pmclassification{msc}{11R04}
%\pmkeywords{infinitude of prime ideals}

% this is the default PlanetMath preamble.  as your knowledge
% of TeX increases, you will probably want to edit this, but
% it should be fine as is for beginners.

% almost certainly you want these
\usepackage{amssymb}
\usepackage{amsmath}
\usepackage{amsfonts}

% used for TeXing text within eps files
%\usepackage{psfrag}
% need this for including graphics (\includegraphics)
%\usepackage{graphicx}
% for neatly defining theorems and propositions
 \usepackage{amsthm}
% making logically defined graphics
%%%\usepackage{xypic}

% there are many more packages, add them here as you need them

% define commands here

\theoremstyle{definition}
\newtheorem*{thmplain}{Theorem}

\begin{document}
\textbf{Theorem.}\, The ring of integers of an algebraic number field contains infinitely many prime ideals.\\

\emph{Proof.}\, Let $\mathcal{O}$ be the ring of integers of a number field.\, If $p$ is a rational prime number, then the principal ideal $(p)$ of $\mathcal{O}$ does not coincide with\, $(1) = \mathcal{O}$\, and thus $(p)$ has a set of prime ideals of $\mathcal{O}$ as factors.\, Two different (positive) rational primes $p$ and $q$ satisfy
$$\gcd((p),\,(q)) \;=\; (p,\,q) \;=\; (1),$$
since there exist integers $x$ and $y$ such that\, $xp\!+\!yq = 1$\, and consequently\, $1 \in (p,\,q)$.\, Therefore, the principal ideals $(p)$ and $(q)$ of $\mathcal{O}$ have no common prime ideal factors.\, Because there are \PMlinkid{infinitely many rational prime numbers}{3036}, also the corresponding principal ideals have infinitely many different prime ideal factors.
%%%%%
%%%%%
\end{document}
