\documentclass[12pt]{article}
\usepackage{pmmeta}
\pmcanonicalname{MultidimensionalArithmeticProgression}
\pmcreated{2013-03-22 13:39:02}
\pmmodified{2013-03-22 13:39:02}
\pmowner{bbukh}{348}
\pmmodifier{bbukh}{348}
\pmtitle{multidimensional arithmetic progression}
\pmrecord{7}{34303}
\pmprivacy{1}
\pmauthor{bbukh}{348}
\pmtype{Definition}
\pmcomment{trigger rebuild}
\pmclassification{msc}{11B25}
\pmsynonym{generalized arithmetic progression}{MultidimensionalArithmeticProgression}
\pmrelated{ArithmeticProgression}

\endmetadata

\usepackage{amssymb}
\usepackage{amsmath}
\usepackage{amsfonts}

\makeatletter
\@ifundefined{bibname}{}{\renewcommand{\bibname}{References}}
\makeatother
\begin{document}
An $n$-dimensional arithmetic progresssion is a set of the form
\begin{align*}
Q&=Q(a; q_1, \dotsc, q_n; l_1, \dotsc, l_n)\\
 &=\{\,a+x_1 q_1+\dotsb+x_n q_n \mid 0\leq x_i<l_i\text{ for } i=1,\dotsc,n \,\}.
\end{align*}
The length of the progression is defined as $l_1\dotsb l_n$. The progression is \emph{proper} if $|Q|=l_1\dotsb l_n$.

\begin{thebibliography}{1}

\bibitem{cite:nathanson_inverseprob}
Melvyn~B. Nathanson.
\newblock {\em Additive Number Theory: Inverse Problems and Geometry of
  Sumsets}, volume 165 of {\em GTM}.
\newblock Springer, 1996.
\newblock \PMlinkexternal{Zbl 0859.11003}{http://www.emis.de/cgi-bin/zmen/ZMATH/en/quick.html?type=html&an=0859.11003}.

\end{thebibliography}
%%%%%
%%%%%
\end{document}
