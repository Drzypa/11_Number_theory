\documentclass[12pt]{article}
\usepackage{pmmeta}
\pmcanonicalname{ParasiticNumber}
\pmcreated{2013-03-22 16:22:17}
\pmmodified{2013-03-22 16:22:17}
\pmowner{CompositeFan}{12809}
\pmmodifier{CompositeFan}{12809}
\pmtitle{parasitic number}
\pmrecord{4}{38510}
\pmprivacy{1}
\pmauthor{CompositeFan}{12809}
\pmtype{Definition}
\pmcomment{trigger rebuild}
\pmclassification{msc}{11A63}
\pmsynonym{left-transposable integer}{ParasiticNumber}

% this is the default PlanetMath preamble.  as your knowledge
% of TeX increases, you will probably want to edit this, but
% it should be fine as is for beginners.

% almost certainly you want these
\usepackage{amssymb}
\usepackage{amsmath}
\usepackage{amsfonts}

% used for TeXing text within eps files
%\usepackage{psfrag}
% need this for including graphics (\includegraphics)
%\usepackage{graphicx}
% for neatly defining theorems and propositions
%\usepackage{amsthm}
% making logically defined graphics
%%%\usepackage{xypic}

% there are many more packages, add them here as you need them

% define commands here

\begin{document}
Given a base $b$ integer $n$ with $k$ digits $d_1, \ldots, d_k$ (with $d_1$ being the least significant digit), and an integer $1 < m < b$, if it is the case that $$m(\sum_{i = 1}^k d_ib^{i - 1}) = d_1b^{k - 1} + (\sum_{i = 2}^k d_ib^{i - 2}),$$ then $n$ is called an $m$-{\em parasitic number} or $m$-{\em left-transposable number}.

If one takes an $m$-parasitic number, concatenates copies of its digits in order whatever number of times one wants, the resulting new number will also be $m$-parasitic.
%%%%%
%%%%%
\end{document}
