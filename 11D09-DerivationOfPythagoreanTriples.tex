\documentclass[12pt]{article}
\usepackage{pmmeta}
\pmcanonicalname{DerivationOfPythagoreanTriples}
\pmcreated{2013-03-22 18:34:40}
\pmmodified{2013-03-22 18:34:40}
\pmowner{pahio}{2872}
\pmmodifier{pahio}{2872}
\pmtitle{derivation of Pythagorean triples}
\pmrecord{8}{41302}
\pmprivacy{1}
\pmauthor{pahio}{2872}
\pmtype{Derivation}
\pmcomment{trigger rebuild}
\pmclassification{msc}{11D09}
\pmclassification{msc}{11A05}
\pmrelated{LinearFormulasForPythagoreanTriples}
\pmrelated{ContraharmonicMeansAndPythagoreanHypotenuses}

% this is the default PlanetMath preamble.  as your knowledge
% of TeX increases, you will probably want to edit this, but
% it should be fine as is for beginners.

% almost certainly you want these
\usepackage{amssymb}
\usepackage{amsmath}
\usepackage{amsfonts}

% used for TeXing text within eps files
%\usepackage{psfrag}
% need this for including graphics (\includegraphics)
%\usepackage{graphicx}
% for neatly defining theorems and propositions
 \usepackage{amsthm}
% making logically defined graphics
%%%\usepackage{xypic}

% there are many more packages, add them here as you need them

% define commands here

\theoremstyle{definition}
\newtheorem*{thmplain}{Theorem}

\begin{document}
For finding all positive solutions of the Diophantine equation
\begin{align}
x^2\!+\!y^2 \;=\; z^2
\end{align}
we first can determine such triples\, $x,\,y,\,z$\, which are coprime.\, When these are then multiplied by all positive integers, one obtains all positive solutions.\\

Let\, $(x,\,y,\,z)$\, be a solution of the mentioned kind.\, Then the numbers are pairwise coprime, since by (1), a common divisor of two of them is also a common divisor of the third.\, Especially, $x$ and $y$ cannot both be even.\, Neither can they both be odd, since because the square of any odd number is\, $\equiv 1 \pmod{4}$, the equation (1) would imply an impossible congruence\, $2 \equiv z^2 \pmod{4}$.\, Accordingly, one of the numbers, e.g. $x$, is even and the other, $y$, odd.

Write (1) to the form
\begin{align}
x^2 \;=\; (z\!+\!y)(z\!-\!y).
\end{align}
Now, both \PMlinkname{factors}{Product} on the right hand side are even, whence one may denote
\begin{align}
z\!+\!y \;=:\; 2u, \quad z\!-\!y \;=:\; 2v
\end{align}
giving
\begin{align}
z \;=\; u\!+\!v, \quad y \;=\; u\!-\!v,
\end{align}
and thus (2) reads
\begin{align}
x^2 \;=\; 4uv.
\end{align}
Because $z$ and $y$ are coprime and\, $z > y > 0$,\, one can infer from (4) and (3) that also $u$ and $v$ must be coprime and\, $u > v > 0$.\, Therefore, it follows from (5) that
$$u \;=\; m^2, \quad v \;=\; n^2$$
where $m$ and $n$ are coprime and\, $m > n > 0$.\, Thus, (5) and (4) yield
\begin{align}
x \;=\; 2mn, \quad y \;=\; m^2\!-\!n^2, \quad z \;=\; m^2\!+\!n^2.
\end{align}
Here, one of $m$ and $n$ is odd and the other even, since $y$ is odd.

By substituting the expressions (6) to the equation (1), one sees that it is satisfied by arbitrary values of $m$ and 
$n$.\, If $m$ and $n$ have all the properties stated above, then $x,\,y,\,z$ are positive integers and, as one may deduce from two first of the equations (6), the numbers $x$ and $y$ and thus all three numbers are coprime.\\

Thus one has proved the

\textbf{Theorem.}\, All coprime positive solutions\, $x,\,y,\,z$,\, and only them, are gotten when one substitutes for $m$ and $n$ to the formulae (6) all possible coprime value pairs, from which always one is odd and the other even and\, $m > n$.

\begin{thebibliography}{9}
\bibitem{K.V.} {\sc K. V\"ais\"al\"a}: {\em Lukuteorian ja korkeamman algebran alkeet}.\, Tiedekirjasto No. 17.\quad  Kustannusosakeyhti\"o Otava, Helsinki (1950).
\end{thebibliography}




%%%%%
%%%%%
\end{document}
