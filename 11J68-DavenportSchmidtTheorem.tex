\documentclass[12pt]{article}
\usepackage{pmmeta}
\pmcanonicalname{DavenportSchmidtTheorem}
\pmcreated{2013-03-22 13:32:58}
\pmmodified{2013-03-22 13:32:58}
\pmowner{Daume}{40}
\pmmodifier{Daume}{40}
\pmtitle{Davenport-Schmidt theorem}
\pmrecord{9}{34151}
\pmprivacy{1}
\pmauthor{Daume}{40}
\pmtype{Theorem}
\pmcomment{trigger rebuild}
\pmclassification{msc}{11J68}

% this is the default PlanetMath preamble.  as your knowledge
% of TeX increases, you will probably want to edit this, but
% it should be fine as is for beginners.

% almost certainly you want these
\usepackage{amssymb}
\usepackage{amsmath}
\usepackage{amsfonts}

% used for TeXing text within eps files
%\usepackage{psfrag}
% need this for including graphics (\includegraphics)
%\usepackage{graphicx}
% for neatly defining theorems and propositions
%\usepackage{amsthm}
% making logically defined graphics
%%%\usepackage{xypic} 

% there are many more packages, add them here as you need them

% define commands here
\begin{document}
For any real $\xi$ which is not rational or quadratic irrational, there are infinitely many rational or real quadratic irrational $\alpha$ which satisfy
\begin{displaymath}
\mid \xi - \alpha \mid < C\cdot H(\alpha)^{-3},
\end{displaymath}
where
\begin{displaymath}
C = \left\{
\begin{array}{ll}
C_0, & \textrm{if} \mid\xi\mid < 1, \\
C_0\cdot \xi^2, & \textrm{if} \mid\xi\mid >1.
\end{array}\right.
\end{displaymath}
$C_0$ is any fixed number greater than $\frac{160}{9}$ and $H(\alpha )$ is the \PMlinkescapetext{height} of $\alpha$.\cite{DS}\\
The \emph{\PMlinkescapetext{height} of the rational or quadratic irrational number} $\alpha$ is 
$$H(\alpha)=\operatorname{max}(|x|,|y|,|z|)$$
where $x$,$y$,$z$ are from the unique equation
$$x\alpha^2+y\alpha+z=0$$
such that $x$,$y$,$z$ are not all zero relatively prime integral coefficients.\cite{DS}
\begin{thebibliography}{1}
\bibitem[DS]{DS} Davenport, H. Schmidt, M. Wolfgang:  Approximation to real numbers by quadratic irrationals. Acta Arithmetica XIII, 1967.
\end{thebibliography}
%%%%%
%%%%%
\end{document}
