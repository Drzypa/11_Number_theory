\documentclass[12pt]{article}
\usepackage{pmmeta}
\pmcanonicalname{FreimanIsomorphism}
\pmcreated{2013-03-22 13:40:39}
\pmmodified{2013-03-22 13:40:39}
\pmowner{bbukh}{348}
\pmmodifier{bbukh}{348}
\pmtitle{Freiman isomorphism}
\pmrecord{8}{34344}
\pmprivacy{1}
\pmauthor{bbukh}{348}
\pmtype{Definition}
\pmcomment{trigger rebuild}
\pmclassification{msc}{11B75}
\pmclassification{msc}{20K30}
\pmrelated{Isomorphism2}

\usepackage{amssymb}
\usepackage{amsmath}
\usepackage{amsfonts}

% used for TeXing text within eps files
%\usepackage{psfrag}
% need this for including graphics (\includegraphics)
%\usepackage{graphicx}
% for neatly defining theorems and propositions
%\usepackage{amsthm}
% making logically defined graphics
%%%\usepackage{xypic}

\makeatletter
\@ifundefined{bibname}{}{\renewcommand{\bibname}{References}}
\makeatother
\begin{document}
Let $A$ and $B$ be subsets of abelian groups $G_A$ and $G_B$
respectively. A Freiman isomorphism of order $s$ is a bijective mapping 
$f\colon A\to B$ such that
\begin{equation*}
a_1+a_2+\dotsb+a_s=a'_1+a'_2+\dotsb+a'_s
\end{equation*}
holds if and only if
\begin{equation*}
f(a_1)+f(a_2)+\dotsb+f(a_s)=f(a'_1)+f(a'_2)+\dotsb+f(a'_s).
\end{equation*}

The Freiman isomorphism is a restriction of the conventional notion of
a group isomorphism to a limited number of group operations. In
particular, a Freiman isomorphism of order $s$ is also a Freiman
isomorphism of order $s-1$, and the mapping is a Freiman isomorphism of
every order precisely when it is the conventional isomorphism.

Freiman isomorphisms were introduced by Freiman in his monograph
\cite{cite:freiman_setaddition} to build a general theory of
\PMlinkname{set addition}{Sumset} that is independent of the
underlying group.

The number of equivalence classes of $n$-element sets of integers under Freiman isomorphisms of order~$2$ is~$n^{2n(1+o(1))}$
\cite{cite:lev_konyagin_freimannonism}.


\begin{thebibliography}{1}

\bibitem{cite:freiman_setaddition}
Gregory Freiman.
\newblock {\em Foundations of Structural Theory of Set Addition}, volume~37 of
  {\em Translations of Mathematical Monographs}.
\newblock AMS, 1973.
\newblock \PMlinkexternal{Zbl
  0271.10044}{http://www.emis.de/cgi-bin/zmen/ZMATH/en/quick.html?type=html&an=0271.10044}.

\bibitem{cite:lev_konyagin_freimannonism}
Sergei~V. Konyagin and Vsevolod~F. Lev.
\newblock Combinatorics and linear algebra of {F}reiman's isomorphism.
\newblock {\em Mathematika}, 47:39--51, 2000.
\newblock Available at \PMlinkexternal{http://math.haifa.ac.il/~seva/}{http://math.haifa.ac.il/~seva/pub_list.html}.

\bibitem{cite:nathanson_inverseprob}
Melvyn~B. Nathanson.
\newblock {\em Additive Number Theory: Inverse Problems and Geometry of
  Sumsets}, volume 165 of {\em GTM}.
\newblock Springer, 1996.
\newblock \PMlinkexternal{Zbl 0859.11003}{http://www.emis.de/cgi-bin/zmen/ZMATH/en/quick.html?type=html&an=0859.11003}.

\end{thebibliography}

%@BOOK{cite:freiman_setaddition,
% author    = {Freiman, Gregory},
% title     = {Foundations of Structural Theory of Set Addition},
% year      = {1973},
% series    = {Translations of Mathematical Monographs},
% volume    = 37,
% publisher = {AMS},
% note      = {\PMlinkexternal{Zbl %0271.10044}{http://www.emis.de/cgi-bin/zmen/ZMATH/en/quick.html?type=html&an=0271.10044}}
%}
%
%@BOOK{cite:nathanson_inverseprob,
% author    = {Melvyn B. Nathanson},
% title     = {Additive Number Theory: Inverse Problems and Geometry of Sumsets},
% series    = {GTM},
% volume    = 165,
% year      = 1996,
% publisher = {Springer},
% note      = {\PMlinkexternal{Zbl %0859.11003}{http://www.emis.de/cgi-bin/zmen/ZMATH/en/quick.html?type=html&an=0859.11003}}
%}
%
%%%%%
%%%%%
\end{document}
