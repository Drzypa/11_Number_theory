\documentclass[12pt]{article}
\usepackage{pmmeta}
\pmcanonicalname{WeirdNumber}
\pmcreated{2013-03-22 16:18:54}
\pmmodified{2013-03-22 16:18:54}
\pmowner{CompositeFan}{12809}
\pmmodifier{CompositeFan}{12809}
\pmtitle{weird number}
\pmrecord{6}{38440}
\pmprivacy{1}
\pmauthor{CompositeFan}{12809}
\pmtype{Definition}
\pmcomment{trigger rebuild}
\pmclassification{msc}{11D85}

% this is the default PlanetMath preamble.  as your knowledge
% of TeX increases, you will probably want to edit this, but
% it should be fine as is for beginners.

% almost certainly you want these
\usepackage{amssymb}
\usepackage{amsmath}
\usepackage{amsfonts}

% used for TeXing text within eps files
%\usepackage{psfrag}
% need this for including graphics (\includegraphics)
%\usepackage{graphicx}
% for neatly defining theorems and propositions
%\usepackage{amsthm}
% making logically defined graphics
%%%\usepackage{xypic}

% there are many more packages, add them here as you need them

% define commands here

\begin{document}
Given an abundant number $n$, if no subset of its divisors $d_1, \ldots , d_{\tau(n) - 1}$ (where $\tau$ is the divisor function) can be selected that adds up to $n$ (that is, $n$ is {\em not} a semiperfect number), then $n$ is said to be a {\em weird number}.

The first few weird numbers are 70, 836, 4030, 5830, 7192, 7912, 9272, (listed in A006037 of Sloane's OEIS). All the known weird numbers are even. In 1972, Benkoski wondered if there are any odd weird numbers; to this day $10^{17}$ is an accepted lower bound. Despite this question of parity, it has been proven that there are infinitely many weird numbers and that they have positive Schnirelmann density.

Even so, weird numbers are rarer than semiperfect numbers; twice a weird number is usually a semiperfect number, which makes all subsequent multiples of a weird number also semiperfect.

\subsection{Trivia}

The band Boards of Canada records for the label Music70. Track 9 of their album {\it Geogaddi} is titled "The Smallest Weird Number."

\begin{thebibliography}{9}
\bibitem{cite:SB1972}
S. Benkoski, "Are All Weird Numbers Even?", {\it Amer. Math. Monthly} {\bf 79} (1972), 774.

\bibitem
WWikipedia, \PMlinkexternal{Weird number}{http://en.wikipedia.org/wiki/Weird_number}
\end{thebibliography}
%%%%%
%%%%%
\end{document}
