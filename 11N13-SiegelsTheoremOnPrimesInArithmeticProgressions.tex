\documentclass[12pt]{article}
\usepackage{pmmeta}
\pmcanonicalname{SiegelsTheoremOnPrimesInArithmeticProgressions}
\pmcreated{2013-03-22 17:58:29}
\pmmodified{2013-03-22 17:58:29}
\pmowner{rm50}{10146}
\pmmodifier{rm50}{10146}
\pmtitle{Siegel's theorem on primes in arithmetic progressions}
\pmrecord{4}{40483}
\pmprivacy{1}
\pmauthor{rm50}{10146}
\pmtype{Theorem}
\pmcomment{trigger rebuild}
\pmclassification{msc}{11N13}
\pmclassification{msc}{11A41}
\pmrelated{PrimeNumberTheorem}

% this is the default PlanetMath preamble.  as your knowledge
% of TeX increases, you will probably want to edit this, but
% it should be fine as is for beginners.

% almost certainly you want these
\usepackage{amssymb}
\usepackage{amsmath}
\usepackage{amsfonts}

% used for TeXing text within eps files
%\usepackage{psfrag}
% need this for including graphics (\includegraphics)
%\usepackage{graphicx}
% for neatly defining theorems and propositions
%\usepackage{amsthm}
% making logically defined graphics
%%%\usepackage{xypic}

% there are many more packages, add them here as you need them

% define commands here
\newtheorem{thm}{Theorem}
\newtheorem{defn}{Definition}
\DeclareMathOperator{\bigoh}{O}
\begin{document}
\PMlinkescapeword{states}
\begin{defn} For $x>0$ real, $q,a$ positive integers, we define
\[\pi(x;q,a)=\sum_{\substack{p\leq x\\p\equiv a\pmod q\\p \text{ prime}}} 1\]
i.e. the number of primes not exceeding $x$ that are congruent to $a$ modulo $q$.
\end{defn}

Then the following holds:
\begin{thm} (Siegel) For all $A>0$, there is some constant $c=c(A)>0$ such that
\[\pi(x;q,a) = \frac{Li(x)}{\varphi(q)} + \bigoh\left(x\exp(-c\sqrt{\log x})\right)\]
for every $1\leq q\leq (\log x)^A$ with $\gcd(q,a)=1$.
\end{thm}

Note that it follows from this theorem that the distribution of primes among invertible residue classes $\mod q$ does not depend on the residue class - that is, primes are evenly distributed into such classes.

A form of Dirichlet's theorem on primes in arithmetic progressions states that
\[\frac{\pi(x;q,a)}{\pi(x)} \sim \frac{1}{\varphi(q)}\]
This follows easily from \PMlinkescapetext{Siegel's theorem} on noting that $Li(x) = \frac{x}{\log x}+\cdots$.
%%%%%
%%%%%
\end{document}
