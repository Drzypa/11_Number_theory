\documentclass[12pt]{article}
\usepackage{pmmeta}
\pmcanonicalname{NumberOfDistinctPrimeFactorsFunction}
\pmcreated{2013-03-22 16:06:52}
\pmmodified{2013-03-22 16:06:52}
\pmowner{CompositeFan}{12809}
\pmmodifier{CompositeFan}{12809}
\pmtitle{number of distinct prime factors function}
\pmrecord{9}{38180}
\pmprivacy{1}
\pmauthor{CompositeFan}{12809}
\pmtype{Definition}
\pmcomment{trigger rebuild}
\pmclassification{msc}{11A25}
\pmrelated{NumberOfNondistinctPrimeFactorsFunction}
\pmrelated{2omeganLeTaunLe2Omegan}

\endmetadata

% this is the default PlanetMath preamble.  as your knowledge
% of TeX increases, you will probably want to edit this, but
% it should be fine as is for beginners.

% almost certainly you want these
\usepackage{amssymb}
\usepackage{amsmath}
\usepackage{amsfonts}

% used for TeXing text within eps files
%\usepackage{psfrag}
% need this for including graphics (\includegraphics)
%\usepackage{graphicx}
% for neatly defining theorems and propositions
%\usepackage{amsthm}
% making logically defined graphics
%%%\usepackage{xypic}

% there are many more packages, add them here as you need them

% define commands here

\begin{document}
The {\em number of distinct prime factors function} $\omega(n)$ counts how many distinct prime factors $n$ has. Expressing $n$ as $$n = \prod_{i = 1}^k {p_i}^{a_i},$$ where the $p_i$ are distinct primes, there being $k$ of them, and  the $a_i$ are positive integers (not necessarily distinct), then $\omega(n) = k$.

Obviously for a prime $p$ it follows that $\omega(p) = 1$. When $n$ is a squarefree number, then $\Omega(n) = \omega(n)$, where $\Omega(n)$ is the \PMlinkname{number of (nondistinct) prime factors function}{NumberOfNondistinctPrimeFactorsFunction}. Otherwise, $\Omega(n) > \omega(n)$.

$\omega(n)$ is an additive function, and it can be used to define a multiplicative function like the M\"obius function $\mu(n) = (-1)^{\omega(n)}$ (as long as $n$ is squarefree).
%%%%%
%%%%%
\end{document}
