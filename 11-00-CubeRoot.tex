\documentclass[12pt]{article}
\usepackage{pmmeta}
\pmcanonicalname{CubeRoot}
\pmcreated{2013-03-22 11:57:22}
\pmmodified{2013-03-22 11:57:22}
\pmowner{Daume}{40}
\pmmodifier{Daume}{40}
\pmtitle{cube root}
\pmrecord{12}{30748}
\pmprivacy{1}
\pmauthor{Daume}{40}
\pmtype{Definition}
\pmcomment{trigger rebuild}
\pmclassification{msc}{11-00}
%\pmkeywords{cube root}
%\pmkeywords{root}
%\pmkeywords{arithmetic operator}
%\pmkeywords{operator}
\pmrelated{NthRoot}
\pmrelated{SquareRoot}
\pmrelated{RationalNumber}
\pmrelated{IrrationalNumber}
\pmrelated{RealNumber}
\pmrelated{ComplexNumber}
\pmrelated{CubeOfANumber}

\endmetadata

\usepackage{amssymb}
\usepackage{amsmath}
\usepackage{amsfonts}
\usepackage{graphicx}
%%%\usepackage{xypic}
\begin{document}
\PMlinkescapeword{properties}

The cube root of a real number $x$, written as $\sqrt[3]{x}$, is the real number $y$ such that $y^3 = x$.
Equivalently, $\sqrt[3]{x}^3 = x$.  Or, $\sqrt[3]{x}\sqrt[3]{x}\sqrt[3]{x} = x$. The cube root notation is actually an alternative to exponentiation.  That is, $\sqrt[3]{x} = x^\frac{1}{3}$.  

\textbf{Properties:}
\begin{itemize}
\item The cube root operation of an exponentiation has the following property:  $\sqrt[3]{x^n} = \sqrt[3]{x}^n$.
\item The cube root operation is distributive for multiplication and division, but not for addition and subtraction. That is, $\sqrt[3]{xy} = \sqrt[3]{x} \sqrt[3]{y}$, and $\sqrt[3]{\frac{x}{y}} = \frac{\sqrt[3]{x}}{\sqrt[3]{y}}$.

\item  However, in general, the cube root operation is not distributive for addition and substraction.  That is, $\sqrt[3]{x + y} \not= \sqrt[3]{x} + \sqrt[3]{y}$ and $\sqrt[3]{x - y} \not= \sqrt[3]{x} - \sqrt[3]{y}$.

\item The cube root is a special case of the general nth root.

\item The cube root is a continuous mapping from $\mathbb{R} \to \mathbb{R}$.

\item The cube root function from $\mathbb{R} \to \mathbb{R}$ defined as $f(x)=\sqrt[3]{x}$ is an odd function.

\end{itemize}

\textbf{Examples:}
\begin{enumerate}
\item $\sqrt[3]{-8} = -2$ because $(-2)^3 = (-2) \times (-2) \times (-2) = -8$.
\item $\sqrt[3]{x^3 + 3x^2 + 3x + 1} = x + 1$ because
$(x + 1)^3 = (x + 1)(x + 1)(x + 1) = (x^2 + 2x + 1)(x + 1) = x^3 + 3x^2 + 3x + 1$.
\item $\sqrt[3]{x^{3}y^{3}} = xy$ because
$(xy)^3 = xy \times xy \times xy = x^{3}y^{3}$.
\item $\sqrt[3]{\frac{8}{125}} = \frac{2}{5}$ because $(\frac{2}{5})^3 = \frac{2^3}{5^3} = \frac{8}{125}$.
\end{enumerate}
%%%%%
%%%%%
%%%%%
\end{document}
