\documentclass[12pt]{article}
\usepackage{pmmeta}
\pmcanonicalname{UnitFraction}
\pmcreated{2013-03-22 12:48:25}
\pmmodified{2013-03-22 12:48:25}
\pmowner{XJamRastafire}{349}
\pmmodifier{XJamRastafire}{349}
\pmtitle{unit fraction}
\pmrecord{10}{33125}
\pmprivacy{1}
\pmauthor{XJamRastafire}{349}
\pmtype{Definition}
\pmcomment{trigger rebuild}
\pmclassification{msc}{11A67}
\pmrelated{AdjacentFraction}
\pmrelated{AnyRationalNumberIsASumOfUnitFractions}
\pmrelated{AnyRationalNumberWithOddDenominatorIsASumOfUnitFractionsWithOddDenominators}
\pmrelated{UnitFraction}
\pmrelated{SierpinskiErdosEgyptianFractionConjecture}
\pmdefines{Egyptian fraction}

\endmetadata

% this is the default PlanetMath preamble.  as your knowledge
% of TeX increases, you will probably want to edit this, but
% it should be fine as is for beginners.

% almost certainly you want these
\usepackage{amssymb}
\usepackage{amsmath}
\usepackage{amsfonts}

% used for TeXing text within eps files
%\usepackage{psfrag}
% need this for including graphics (\includegraphics)
%\usepackage{graphicx}
% for neatly defining theorems and propositions
%\usepackage{amsthm}
% making logically defined graphics
%%%\usepackage{xypic}

% there are many more packages, add them here as you need them

% define commands here
\begin{document}
An {\it unit fraction} $\frac{n}{d}$ is a fraction whose numerator $n = 1$. 
If its integer denominator $d > 1$, then a fraction is also a proper fraction. So there is only one unit fraction which is improper, namely 1.

Such fractions are known from Egyptian mathematics where we can find a lot of special representations of the numbers as a sum of an unit fractions, which are now called {\it Egyptian fractions}. From the Rhind papyrus as an example:

$$\frac{2}{71} = \frac{1}{40} + \frac{1}{568} + \frac{1}{710} \; . $$

Many unit fractions are in the pairs of the adjacent fractions. An unit fractions are some successive or non-successive terms of any Farey sequence $F_{n}$ of a degree $n$. For example the fractions $\frac{1}{2}$ and $\frac{1}{4}$ are adjacent, but they are not the successive terms in the Farey sequence $F_{5}$. The fractions $\frac{1}{3}$ and $\frac{1}{4}$ are also adjacent and they are successive terms in the $F_{5}$.
%%%%%
%%%%%
\end{document}
