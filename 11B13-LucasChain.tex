\documentclass[12pt]{article}
\usepackage{pmmeta}
\pmcanonicalname{LucasChain}
\pmcreated{2013-03-22 18:29:22}
\pmmodified{2013-03-22 18:29:22}
\pmowner{PrimeFan}{13766}
\pmmodifier{PrimeFan}{13766}
\pmtitle{Lucas chain}
\pmrecord{4}{41167}
\pmprivacy{1}
\pmauthor{PrimeFan}{13766}
\pmtype{Definition}
\pmcomment{trigger rebuild}
\pmclassification{msc}{11B13}

% this is the default PlanetMath preamble.  as your knowledge
% of TeX increases, you will probably want to edit this, but
% it should be fine as is for beginners.

% almost certainly you want these
\usepackage{amssymb}
\usepackage{amsmath}
\usepackage{amsfonts}

% used for TeXing text within eps files
%\usepackage{psfrag}
% need this for including graphics (\includegraphics)
%\usepackage{graphicx}
% for neatly defining theorems and propositions
%\usepackage{amsthm}
% making logically defined graphics
%%%\usepackage{xypic}

% there are many more packages, add them here as you need them

% define commands here

\begin{document}
A {\em Lucas chain} $a$ is an addition chain with the additional requirement that not only each term be the sum of two previous (not necessarily distinct) terms, but also that the difference of those two terms also be a term in the sequence. That is, each $a_i = a_m + a_n$ and also $|a_n - a_m| = a_j$, with $j$ being some nonnegative integer.

For example, the Fibonacci sequence (1, 1, 2, 3, 5, 8, 13, 21, 34, 55, etc.) is a Lucas chain because not only is each term the sum of the previous two terms, each term is the difference of the next two terms. On the other hand, 1, 2, 4, 8, 9, 13, 21, 30, etc., is an addition chain but not a Lucas chain, since 8 + 13 = 21, but $13 - 8 = 5$, which is not a member of the chain.
%%%%%
%%%%%
\end{document}
