\documentclass[12pt]{article}
\usepackage{pmmeta}
\pmcanonicalname{DemloNumber}
\pmcreated{2013-03-22 16:22:01}
\pmmodified{2013-03-22 16:22:01}
\pmowner{PrimeFan}{13766}
\pmmodifier{PrimeFan}{13766}
\pmtitle{Demlo number}
\pmrecord{5}{38504}
\pmprivacy{1}
\pmauthor{PrimeFan}{13766}
\pmtype{Definition}
\pmcomment{trigger rebuild}
\pmclassification{msc}{11A63}
\pmsynonym{wonderful Demlo number}{DemloNumber}

% this is the default PlanetMath preamble.  as your knowledge
% of TeX increases, you will probably want to edit this, but
% it should be fine as is for beginners.

% almost certainly you want these
\usepackage{amssymb}
\usepackage{amsmath}
\usepackage{amsfonts}

% used for TeXing text within eps files
%\usepackage{psfrag}
% need this for including graphics (\includegraphics)
%\usepackage{graphicx}
% for neatly defining theorems and propositions
%\usepackage{amsthm}
% making logically defined graphics
%%%\usepackage{xypic}

% there are many more packages, add them here as you need them

% define commands here

\begin{document}
Given base $b$, a number of the form $({{b^n - 1} \over {b - 1}})^2$ for $n > 0$ (that is, the square of a repunit) is a {\em Demlo number}, sometimes called a {\em wonderful Demlo number}.

The Demlo numbers for $n < b$ are also palindromic numbers, and specifically of the form $$nb^{n - 1} + \sum_{i = 1}^{n - 1} (ib^{2n - i} + ib^{i - 1}),$$ that is, the most significant digits are the first $n$ digits of base $b$ in order and the least significant digits are the first $n$ digits of base $b$ backwards.
%%%%%
%%%%%
\end{document}
