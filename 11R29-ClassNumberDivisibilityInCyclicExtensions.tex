\documentclass[12pt]{article}
\usepackage{pmmeta}
\pmcanonicalname{ClassNumberDivisibilityInCyclicExtensions}
\pmcreated{2013-03-22 15:07:41}
\pmmodified{2013-03-22 15:07:41}
\pmowner{alozano}{2414}
\pmmodifier{alozano}{2414}
\pmtitle{class number divisibility in cyclic extensions}
\pmrecord{4}{36868}
\pmprivacy{1}
\pmauthor{alozano}{2414}
\pmtype{Theorem}
\pmcomment{trigger rebuild}
\pmclassification{msc}{11R29}
\pmclassification{msc}{11R32}
\pmclassification{msc}{11R37}
\pmrelated{IdealClass}
\pmrelated{ClassNumbersAndDiscriminantsTopicsOnClassGroups}

\endmetadata

% this is the default PlanetMath preamble.  as your knowledge
% of TeX increases, you will probably want to edit this, but
% it should be fine as is for beginners.

% almost certainly you want these
\usepackage{amssymb}
\usepackage{amsmath}
\usepackage{amsthm}
\usepackage{amsfonts}

% used for TeXing text within eps files
%\usepackage{psfrag}
% need this for including graphics (\includegraphics)
%\usepackage{graphicx}
% for neatly defining theorems and propositions
%\usepackage{amsthm}
% making logically defined graphics
%%%\usepackage{xypic}

% there are many more packages, add them here as you need them

% define commands here

\newtheorem{thm}{Theorem}
\newtheorem{defn}{Definition}
\newtheorem{prop}{Proposition}
\newtheorem{lemma}{Lemma}
\newtheorem{cor}{Corollary}

% Some sets
\newcommand{\Nats}{\mathbb{N}}
\newcommand{\Ints}{\mathbb{Z}}
\newcommand{\Reals}{\mathbb{R}}
\newcommand{\Complex}{\mathbb{C}}
\newcommand{\Rats}{\mathbb{Q}}
\newcommand{\Gal}{\operatorname{Gal}}
\newcommand{\Cl}{\operatorname{Cl}}
\begin{document}
In this entry, the class number of a number field $L$ is denoted by $h_L$.

\begin{thm}
Let $F/K$ be a cyclic Galois extension of degree $n$. Let $p$ be a prime such that $n$ is not divisible by $p$, and assume that $p$ does not divide $h_E$, the class number of any intermediate field $K\subseteq E \subsetneq F$. If $p$ divides $h_F$ then $p^f$ also divides $h_F$, where $f$ is the multiplicative order of $p$ modulo $n$.
\end{thm}

Recall that the multiplicative order of $p$ modulo $n$ is a number $f$ such that $p^f\equiv 1 \mod n$ and $p^m$ is not congruent to $1$ modulo $n$ for any $1\leq m <f$.

\begin{cor}
Let $F/K$ be a Galois extension such that $[F:K]=q$ is a prime distinct from the prime $p$. Assume that $p$ does not divide $h_K$. If $p$ divides $h_F$ then $p^f$ divides $h_F$, where $f$ is the multiplicative order of $p$ modulo $q$.
\end{cor}

Note that a Galois extension $F/K$ of prime degree has no non-trivial subextensions.
%%%%%
%%%%%
\end{document}
