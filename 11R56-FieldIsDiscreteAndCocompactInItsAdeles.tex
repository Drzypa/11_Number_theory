\documentclass[12pt]{article}
\usepackage{pmmeta}
\pmcanonicalname{FieldIsDiscreteAndCocompactInItsAdeles}
\pmcreated{2013-03-22 18:00:05}
\pmmodified{2013-03-22 18:00:05}
\pmowner{rm50}{10146}
\pmmodifier{rm50}{10146}
\pmtitle{field is discrete and cocompact in its ad\`{e}les}
\pmrecord{4}{40515}
\pmprivacy{1}
\pmauthor{rm50}{10146}
\pmtype{Theorem}
\pmcomment{trigger rebuild}
\pmclassification{msc}{11R56}

\endmetadata

% this is the default PlanetMath preamble.  as your knowledge
% of TeX increases, you will probably want to edit this, but
% it should be fine as is for beginners.

% almost certainly you want these
\usepackage{amssymb}
\usepackage{amsmath}
\usepackage{amsfonts}

% used for TeXing text within eps files
%\usepackage{psfrag}
% need this for including graphics (\includegraphics)
%\usepackage{graphicx}
% for neatly defining theorems and propositions
\usepackage{amsthm}
% making logically defined graphics
%%%\usepackage{xypic}

% there are many more packages, add them here as you need them

% define commands here
\newcommand{\Ints}{\mathbb{Z}}
\newcommand{\Reals}{\mathbb{R}}
\newcommand{\Complex}{\mathbb{C}}
\newcommand{\Ade}{\mathbb{A}}
\newcommand{\Rats}{\mathbb{Q}}
\newcommand{\Alg}{\mathcal{O}}
\renewcommand{\o}{\mathfrak{o}}
\newcommand{\smp}{\mathfrak{p}}
\newcommand{\Abs}[1]{\left\lvert #1\right\rvert}
\newcommand{\Rprod}{\sideset{}{'}\prod}

\newtheorem{thm}{Theorem}
\newtheorem{prop}[thm]{Proposition}

\begin{document}
\PMlinkescapeword{even}
For brevity, we write $P_f$ for the set of finite places of $K$, and $P_{\infty}$ for the set of infinite places. We also write $\Rprod$ for a restricted direct product. Then 
\[\Ade_K = \Rprod_{v\in P_f} K_v \times \prod_{v\in P_{\infty}} K_v\]

\begin{thm} $K$ is discrete as a subgroup of $\Ade_K$.
\end{thm}
\textbf{Proof. } Since $\Ade_K$ is a topological ring, it suffices to show that there is a neighborhood in $\Ade_K$ meeting $K$ in only $0$.

Let
\[U=\prod_{v\in P_f} \o_v \times \prod_{v\in P_{\infty}} B\left(0,\frac{1}{2}\right)\]
Since $\o_v$ is open in $K_v$ for $v$ finite, this is an open set. (Note that $\o_v = \Alg_{K_v}$, the ring of algebraic integers of $K_v$).

Now consider an element $x\in U\cap K\subset \Ade_K$. If $x=(x_v)$, then for $v$ finite, $x_v\in \o_v$, and for $v$ infinite, $x_v\in B\left(0,\frac{1}{2}\right)$. Assume $x\neq 0$. Then
\[\Abs{x_v}_v \leq\begin{cases} 1 & v \text{ finite}\\ \frac{1}{2}<1 & v \text{ infinite}\end{cases}\]
but then
\[\prod \Abs{x}_v = \prod \Abs{x_v}_v < 1\]
in contradiction to the product formula. Thus $x=0$ and we are done.


The above theorem is very sensitive to the fact that all places are included in $\Ade_K$. For example, it is clear that the image of $\Ade_K$ in $\prod_{v\in P_{\infty}} K_v$ is \emph{dense}, since $K_v$ is characterized by an embedding $K\hookrightarrow K_v\cong\Reals,\Complex,\Rats_p$. Then by an argument familiar from Minkowski's theorem, $\Alg_K$ is a full-rank lattice in the image of $K$. But $K$ is the $\Rats$-span of that lattice, so is dense in $K_v$.

Furthermore, the same is true for the finite places:
\begin{prop} The image of $K$ in $\Rprod_{v\in P_f} \o_v$ is dense.
\end{prop}
\textbf{Proof. }
Suppose $x=(x_v)_v\in \Rprod_{v\in P_f} \o_v$. We show that $x$ can be approximated as closely as desired by an element of $K$ by showing that for any ideal $I\subset \Alg_K$, there is $y\in K$ such that $y-x_v\in I\o_v$ for each $v\in P_f$.

First multiply through by some $z$ so that everything is in $\Alg_K$: choose $z\neq 0$ such that $zx_v\in\o_v$ for all $v\in P_f$. This is possible since all but finitely many $x_v$ are already in $\o_v$. Thus $y-x_v\in I\o_v$ is equivalent to $zy-zx_v\in(zI)\o_v\subset I\o_v$. So assume wlog that $x_v\in\o_v$ for all $v$; in the end simply divide by $z$ to recover the general case. But then the existence of $y$ is guaranteed by the Chinese Remainder Theorem, since if $I=\prod \smp_i^{e_i}$, then $I\o_v = \smp_i^{e_i}$ for some $i$.


It is true, though somewhat harder to prove, that $K$ is in fact dense in $\Ade_K$ if even one place is missing from the product!

\begin{thm} $\Ade_K/K$ is compact.
\end{thm}
\textbf{Proof. } The set
\[U = \prod_{v\in P_f}\o_v \times \prod_{v\in P_{\infty}} K_v\]
is open in $\Ade_K$.

Claim first that $K+U=\Ade_K$. Choose $(x_v)\in \Ade_K$. There is a finite set $S$ of finite places $v$ such that $x_v\notin \o_v$ for $v\in S$. Using an argument identical to the approximation argument above, choose $y\in K$ such that $y-x_v \in \o_v, v\in S$ and $y\in\o_v, v\notin S$. Then $(x_v-y)_v$ is in $\o_v$ for $v\in S$, is in $\o_v$ for $v\notin S$ but finite, and is in $K_v$ for $v$ infinite. Thus $(x_v-y)_v\in U$ and we are done.

Claim next that $K\cap U = \Alg_K$. $\supset$ is obvious. To see $\subset$, note that an element of $K\cap U$ is an element of $K$ that is integral at every finite place, so it is integral and is in $\Alg_K$.

Thus we get a natural map $U\hookrightarrow \Ade_K \twoheadrightarrow \Ade_K/K$. This map is surjective since $K+U=\Ade_K$, and its kernel is $K\cap U$. So it suffices to show that $U/(K\cap U)=U/\Alg_K$ is compact. There is obviously an exact sequence induced by the projection $U\to\prod_{v\in P_{\infty}} K_v$,
\[\prod_{v\in P_f} \o_v\to U/\Alg_K\to \prod_{v\in P_{\infty}} K_v/\Alg_K\to 0\]
The left-hand side is compact since each $\o_v$ is, and the right-hand side is 
\[\Reals^{r_1+2r_2}/\Alg_K\]
which know is compact since $\Alg_K$ forms a full-rank lattice in $\Reals^n$. Thus $U/\Alg_K$ is also compact and we are done.


So we have shown that $\Ade_K$ is a locally compact ring, and that $K\subset \Ade_K$ is discrete and cocompact. This is analogous to two other situations with which we are familiar:
\begin{gather*}
\Reals\text{ is locally compact, }\Ints\subset\Reals\text{ is discrete and cocompact}\\
\prod_{v\in P_{\infty}}K_v\text{ is locally compact, }\Alg_K\subset\prod_{v\in P_{\infty}}K_v\text{ is discrete and cocompact}
\end{gather*}
This is a useful concept because in such a situation one can do Fourier analysis. For example, if $f:\Reals\to\Reals$ is a $C^{\infty}$ function with exponential decay (or at least integrable on all of $\Reals$), then we can define its Fourier transform $\hat{f}$, and the Poisson summation formula
\[\sum_{n\in\Ints} f(n) = \sum_{n\in\Ints} \hat{f}(n)\]
relates the two. The same theory thus exists for appropriately defined functions $f:\Ade_K\to \Reals$, and the Poisson formula again holds with the sum over $K$ rather than over $\Ints$. This can be used to show that the $L$-functions have analytic continuations, just as the real Poisson formula is used to show this for $\zeta$.

%%%%%
%%%%%
\end{document}
