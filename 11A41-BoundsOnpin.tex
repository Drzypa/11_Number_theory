\documentclass[12pt]{article}
\usepackage{pmmeta}
\pmcanonicalname{BoundsOnpin}
\pmcreated{2013-03-22 16:23:51}
\pmmodified{2013-03-22 16:23:51}
\pmowner{rm50}{10146}
\pmmodifier{rm50}{10146}
\pmtitle{bounds on $\pi(n)$}
\pmrecord{4}{38545}
\pmprivacy{1}
\pmauthor{rm50}{10146}
\pmtype{Theorem}
\pmcomment{trigger rebuild}
\pmclassification{msc}{11A41}

% this is the default PlanetMath preamble.  as your knowledge
% of TeX increases, you will probably want to edit this, but
% it should be fine as is for beginners.

% almost certainly you want these
\usepackage{amssymb}
\usepackage{amsmath}
\usepackage{amsfonts}

% used for TeXing text within eps files
%\usepackage{psfrag}
% need this for including graphics (\includegraphics)
%\usepackage{graphicx}
% for neatly defining theorems and propositions
%\usepackage{amsthm}
% making logically defined graphics
%%%\usepackage{xypic}

% there are many more packages, add them here as you need them

% define commands here
\newtheorem{thm}{Theorem}
\newtheorem{defn}{Definition}
\newtheorem{lem}{Lemma}
\begin{document}
This article shows that
\[\ln 2\left(\frac{n}{\ln n}\right)-1\leq \pi(n)\leq 8\ln 2\left(\frac{n}{\ln n}\right)\]
and thus that the function
\[\pi(n)/\frac{n}{\ln n}\]
is bounded above and below.

\begin{defn} If $n$ is an integer, write $n=\prod p_i^{r_i}$ where the $p_i$ are distinct primes. Then
\[ord_p(n)=\begin{cases}
r_i&\text{if } p=p_i\\
0&\text{otherwise}
\end{cases}
\]
\end{defn}

\begin{lem}  
\[ord_p(n!)=\sum_{i=1}^{\infty}\lfloor \frac{n}{p^i}\rfloor\]
\end{lem}
\textbf{Proof.}
Note that the number of multiples of $p\leq n$ is simply $\lfloor \frac{n}{p}\rfloor$. But that does not result in $ord_p(n!)$ since some of those $\lfloor \frac{n}{p}\rfloor$ numbers may have additional factors of $p$. So divide each by $p$; then clearly 
\[ord_p(n!)=\lfloor \frac{n}{p}\rfloor+ord_p(\lfloor \frac{n}{p}\rfloor!)\]
The result follows inductively. Note that the sum is actually finite.


\begin{lem} If $\binom{m+n}{n}=\prod q_i, q_i=p_i^{r_i}$, all $p_i$ distinct, then $\forall i, q_i\leq m+n$.
\end{lem}
\textbf{Proof.}
Consider $q_1=p^c$. Choose $a$ such that $p^a\leq m+n<p^{a+1}$; it suffices to show $c\leq a$.
\[c=ord_p(\binom{m+n}{n})=ord_p((m+n)!)-ord_p(m!)-ord_p(n!)=\sum_1^{\infty} (\lfloor\frac{m+n}{p^i}\rfloor-\lfloor\frac{m}{p^i}\rfloor-\lfloor\frac{n}{p^i}\rfloor)\]
Now, in general, if $\lfloor x\rfloor=r, \lfloor y\rfloor=s$, then $\lfloor x+y\rfloor=r+s$ or $r+s+1$. So the sum above has at most $a$ terms (since $p^{a+1}>m+n$), each of which is either $0$ or $1$, and thus $c\leq a$.

\begin{thm} \emph{(Chebyshev)} If $n\geq 2$
\[\pi(n)\geq \ln 2\left(\frac{n}{\ln n}\right)-1\]
\end{thm}
\textbf{Proof.}
Choose $r$ such that $0<r<n$, and write $\binom{n}{r}=\prod p_i\!^{\lambda_i}$. Each $p_i^{\lambda_i}\leq n$ by the preceding lemma, and there are at most $\pi(n)$ terms on the right-hand side. Thus
\[n^{\pi(n)}\geq\binom{n}{r}\ \forall r\]
Summing from $r=1$ to $r=n-1$, we get
\[(n-1)n^{\pi(n)}\geq\sum_{r=1}^{n-1}\binom{n}{r}=2^n-2\]
But $n^{\pi(n)}\geq 2$ since $n\geq 2$, so
\[n^{\pi(n)+1}\geq 2^n\]
so $(\pi(n)+1)\ln n\geq n\ln 2$ and thus
\[\pi(n)\geq \ln 2\left(\frac{n}{\ln n}\right)-1
\]


\begin{thm} \emph{(Chebyshev)}
\[\pi(n)\leq 8\ln 2\left(\frac{n}{\ln n}\right)\]
\end{thm}
\textbf{Proof.}
Note that if $p$ is prime, $n\leq p\leq 2n$, then $p$ divides $\binom{2n}{n}$ since $p$ occurs in the numerator but not in the denominator. Thus $\prod_{n\leq p\leq 2n} p$ divides $\binom{2n}{n}$ as well, and thus
\[\prod_{n\leq p\leq 2n} p\leq\binom{2n}{n}\leq 2^{2n}\]
and therefore
\[n^{\pi(2n)-\pi(n)}\leq 2^{2n}\]
Taking logs, we get
\[\pi(2n)-\pi(n)\leq\frac{2n\ln 2}{\ln n}=2\ln 2\frac{n}{\ln n}\]
Let $f(n)=\pi(n)\ln n$. Then
\begin{eqnarray*}
&f(2n)-f(n)&=\pi(2n)\ln 2n-\pi(n)\ln n\\
&&=(\pi(2n)-\pi(n))\ln n + \pi(2n)(\ln 2n-\ln n)\\
&&=(\pi(2n)-\pi(n))\ln n+\pi(2n)\ln 2\\
&&\leq 2n\ln 2+2n\ln 2\\
&&=4n\ln 2
\end{eqnarray*}
Then
\begin{eqnarray*}
&f(2^t)&=(f(2^t)-f(2^{t-1}))+(f(2^{t-1})-f(2^{t-2}))+\ldots+(f(2)-f(1))\\
&&\leq 4 \ln 2(2^{t-1}+2^{t-2}+\ldots+2^0)\\
&&\leq 4\ln 2\cdot 2^t
\end{eqnarray*}
Note that $f$ is monotonically increasing, so if we choose $t$ with $2^{t-1}\leq n<2^t$, then
\[f(n)\leq f(2^t)\leq 4\ln 2\cdot 2^t\leq 4\ln 2\cdot 2n=8n\ln 2
\]

%%%%%
%%%%%
\end{document}
