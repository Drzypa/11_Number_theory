\documentclass[12pt]{article}
\usepackage{pmmeta}
\pmcanonicalname{ModularForm}
\pmcreated{2013-03-22 14:07:37}
\pmmodified{2013-03-22 14:07:37}
\pmowner{olivierfouquetx}{2421}
\pmmodifier{olivierfouquetx}{2421}
\pmtitle{modular form}
\pmrecord{31}{35534}
\pmprivacy{1}
\pmauthor{olivierfouquetx}{2421}
\pmtype{Definition}
\pmcomment{trigger rebuild}
\pmclassification{msc}{11F11}
\pmrelated{TaniyamaShimuraConecture}
\pmrelated{HeckeAlgebra}
\pmrelated{AlgebraicNumberTheory}
\pmrelated{RamanujanTauFunction}
\pmdefines{cusp form}

% this is the default PlanetMath preamble.  as your knowledge
% of TeX increases, you will probably want to edit this, but
% it should be fine as is for beginners.

% almost certainly you want these
\usepackage{amssymb}
\usepackage{amsmath}
\usepackage{amsfonts}

% used for TeXing text within eps files
%\usepackage{psfrag}
% need this for including graphics (\includegraphics)
%\usepackage{graphicx}
% for neatly defining theorems and propositions
%\usepackage{amsthm}
% making logically defined graphics
%%%\usepackage{xypic}

% there are many more packages, add them here as you need them

\newcommand{\sldeuxz}{\textrm{SL}_{2}(\mathbb{Z})}
\newcommand{\sldeuxr}{\textrm{SL}_{2}(\mathbb{R})}
\begin{document}
Let $\sldeuxr$ be the group of real $2\times 2$ matrices with determinant $1$ (see entry on special linear groups).  The group $\sldeuxr$ acts on $H$, the upper half plane, through \emph{fractional linear transformations}.  That is, if
\[
\gamma = \begin{pmatrix}a & b \\ c & d\end{pmatrix},
\]
and $\tau\in H$, then we let
\begin{equation}
\gamma \tau=\frac{a\tau+b}{c\tau+d}.
\end{equation}

For any natural number $N \geq 1$, define the {\em congruence subgroup} $\Gamma_0(N)$ of level $N$ to be the following subgroup of the group $\sldeuxz$ of integer coefficient matrices of determinant $1$:
$$
\Gamma_0(N) := \left\{ \left.
\begin{pmatrix}
a & b \\
c & d
\end{pmatrix}
\in \sldeuxz\ \right|\ c \equiv 0 \pmod{N} \right\}.
$$

Fix an integer $k$.  For $\gamma\in\sldeuxz$ and a function $f$ defined on $H$, we define $$f_{\mid\gamma}(\tau)=\frac{f(\gamma \tau)}{(c\tau+d)^k}.$$
For a finite index subgroup $\Gamma$ of $\sldeuxz$ containing a congruence subgroup, a function $f$ defined on $H$ is said to be a weight $k$ \emph{modular form} if:
\begin{enumerate}
\item $f=f_{\mid \gamma}$ for $\gamma \in \Gamma$.
\item $f$ is holomorphic on $H$.
\item $f$ is holomorphic at the cusps.
\end{enumerate}

This last condition requires some explanation.  First observe that the element 
\[
\mu =
\begin{pmatrix}
1 & m \\
0 & 1
\end{pmatrix}
\in \Gamma_0(N),
\]
and $\mu z = z + m$, while if $f$ satisfies all the other conditions above, $f_{\mid \mu} = f$.  In other words, $f$ is periodic with period $1$.  Thus, convergence permitting, $f$ admits a Fourier expansion.  Therefore, we say that $f$ is holomorphic at the cusps if, for all $\gamma \in \Gamma$, $f_{\mid \gamma}$ admits a a Fourier expansion
\begin{equation}
f_{\mid \gamma}(\tau)=\sum_{n=0}^{\infty}a_{n}q^{n},
\end{equation}
where $q=e^{2i\pi \tau}$.

If all the $a_n$ are zero for $n\le 0$, then a modular form $f$ is said to be a \emph{cusp form}.  The set of modular forms for $\Gamma$ (respectively cusp forms for $\Gamma$) is often denoted by $M_{k}(\Gamma)$ (respectively $S_{k}(\Gamma)$).  Both $M_{k}(\Gamma)$ and $S_{k}(\Gamma)$ are finite dimensional vector spaces.

The space of modular forms for $\sldeuxz$ (respectively cusp forms) is non-trivial for any $k$ even and greater than 4 (respectively greater than $12$ and not $14$).  Examples of modular forms for $\sldeuxz$ are:
\begin{enumerate}
\item The Eisenstein series $E_{m}$, where $m$ is even and greater than $4$, is a modular form of weight $m$.  Here $B_{m}$ denotes the $m$-th Bernoulli number and, as usual, $q=e^{2i\pi \tau}$:
\begin{equation}
E_{m}(\tau)=1-\frac{2m}{B_{m}}\underset{n=1}{\overset{\infty}{\sum}}\sigma_{m-
1}(n)q^n.
\end{equation}
For instance,
\begin{equation}
E_{4}(\tau)=1+240\underset{n=1}{\overset{\infty}{\sum}}\sigma_{3}(n)q^n
\end{equation}
and
\begin{equation}
E_{6}(\tau)=1-504\underset{n=1}{\overset{\infty}{\sum}}\sigma_{5}(n)q^n.
\end{equation}

\item The Weierstrass $\Delta$ function, also called the modular discriminant, is a modular form of weight $12$:
\begin{equation}
\Delta(\tau)=q\underset{n=1}{\overset{\infty}{\prod}}(1-q^n)^{24}.
\end{equation}
\end{enumerate}

Every modular form is expressible as
\begin{equation}
f(\tau)=\underset{n=0}{\overset{\lfloor{k/12}\rfloor}{\sum}}{a_n}{E_{k-12n}(\tau)}{(\Delta(\tau))^n},
\end{equation}
where the $a_n$ are arbitrary constants, $E_0(\tau)=1$ and $E_2(\tau)=0$.  Cusp forms are the forms with $a_0=0$.
%%%%%
%%%%%
\end{document}
