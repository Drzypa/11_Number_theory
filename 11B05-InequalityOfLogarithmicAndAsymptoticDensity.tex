\documentclass[12pt]{article}
\usepackage{pmmeta}
\pmcanonicalname{InequalityOfLogarithmicAndAsymptoticDensity}
\pmcreated{2014-03-24 9:16:11}
\pmmodified{2014-03-24 9:16:11}
\pmowner{kompik}{10588}
\pmmodifier{kompik}{10588}
\pmtitle{inequality of logarithmic and asymptotic density}
\pmrecord{8}{37424}
\pmprivacy{1}
\pmauthor{kompik}{10588}
\pmtype{Theorem}
\pmcomment{trigger rebuild}
\pmclassification{msc}{11B05}
\pmrelated{AsymptoticDensity}
\pmrelated{LogarithmicDensity}

% this is the default PlanetMath preamble. as your knowledge
% of TeX increases, you will probably want to edit this, but
% it should be fine as is for beginners.

% almost certainly you want these
\usepackage{amssymb}
\usepackage{amsmath}
\usepackage{amsfonts}
\usepackage{amsthm}

% used for TeXing text within eps files
%\usepackage{psfrag}
% need this for including graphics (\includegraphics)
%\usepackage{graphicx}
% for neatly defining theorems and propositions
%
% making logically defined graphics
%%%\usepackage{xypic}

% there are many more packages, add them here as you need them

% define commands here

\newcommand{\sR}[0]{\mathbb{R}}
\newcommand{\sC}[0]{\mathbb{C}}
\newcommand{\sN}[0]{\mathbb{N}}
\newcommand{\sZ}[0]{\mathbb{Z}}
\newcommand{\N}[0]{\mathbb{N}}

\usepackage{bbm}
\newcommand{\Z}{\mathbbmss{Z}}
\newcommand{\C}{\mathbbmss{C}}
\newcommand{\R}{\mathbbmss{R}}
\newcommand{\Q}{\mathbbmss{Q}}



\newcommand*{\norm}[1]{\lVert #1 \rVert}
\newcommand*{\abs}[1]{| #1 |}

\newcommand{\Map}[3]{#1:#2\to#3}
\newcommand{\Emb}[3]{#1:#2\hookrightarrow#3}
\newcommand{\Mor}[3]{#2\overset{#1}\to#3}

\newcommand{\Cat}[1]{\mathcal{#1}}
\newcommand{\Kat}[1]{\mathbf{#1}}
\newcommand{\Func}[3]{\Map{#1}{\Cat{#2}}{\Cat{#3}}}
\newcommand{\Funk}[3]{\Map{#1}{\Kat{#2}}{\Kat{#3}}}

\newcommand{\vp}{\varphi}
\newcommand{\ve}{\varepsilon}

\newcommand{\Invimg}[2]{\inv{#1}(#2)}
\newcommand{\Img}[2]{#1[#2]}
\newcommand{\ol}[1]{\overline{#1}}
\newcommand{\ul}[1]{\underline{#1}}
\newcommand{\inv}[1]{#1^{-1}}
\newcommand{\limti}[1]{\lim\limits_{#1\to\infty}}

%fonts
\newcommand{\mc}{\mathcal}

%shortcuts
\newcommand{\Ob}{\mathrm{Ob}}
\newcommand{\Hom}{\mathrm{hom}}
\newcommand{\homs}[2]{\mathrm{hom(}{#1},{#2}\mathrm )}
\newcommand{\Eq}{\mathrm{Eq}}
\newcommand{\Coeq}{\mathrm{Coeq}}

%theorems
\newtheorem{THM}{Theorem}
\newtheorem{DEF}{Definition}
\newtheorem{PROP}{Proposition}
\newtheorem{LM}{Lemma}
\newtheorem{COR}{Corollary}
\newtheorem{EXA}{Example}

%categories
\newcommand{\Top}{\Kat{Top}}
\newcommand{\Haus}{\Kat{Haus}}
\newcommand{\Set}{\Kat{Set}}

%diagrams
\newcommand{\UnimorCD}[6]{
\xymatrix{ {#1} \ar[r]^{#2} \ar[rd]_{#4}& {#3} \ar@{-->}[d]^{#5} \\
& {#6} } }

\newcommand{\RovnostrCD}[6]{
\xymatrix@C=10pt@R=17pt{
& {#1} \ar[ld]_{#2} \ar[rd]^{#3} \\
{#4} \ar[rr]_{#5} && {#6} } }

\newcommand{\RovnostrCDii}[6]{
\xymatrix@C=10pt@R=17pt{
{#1} \ar[rr]^{#2} \ar[rd]_{#4}&& {#3} \ar[ld]^{#5} \\
& {#6} } }

\newcommand{\RovnostrCDiiop}[6]{
\xymatrix@C=10pt@R=17pt{
{#1}  && {#3} \ar[ll]_{#2}  \\
& {#6} \ar[lu]^{#4} \ar[ru]_{#5} } }

\newcommand{\StvorecCD}[8]{
\xymatrix{
{#1} \ar[r]^{#2} \ar[d]_{#4} & {#3} \ar[d]^{#5} \\
{#6} \ar[r]_{#7} & {#8}
}
}

\newcommand{\TriangCD}[6]{
\xymatrix{ {#1} \ar[r]^{#2} \ar[rd]_{#4}&
{#3} \ar[d]^{#5} \\
& {#6} } }
\begin{document}
For any $A\subseteq\N$ we denote
$A(n):=\abs{A\cap\{1,2,\ldots,n\}}$ and $S(n):=\sum\limits_{k=1}^n
\frac 1k$.

Recall that the values
$$\ul d(A)=\liminf_{n\to\infty} \frac{A(n)}n \qquad \ol d(A) = \limsup_{n\to\infty} \frac{A(n)}n$$
are called lower and upper asymptotic density of $A$.

The values
$$\ul \delta(A)=\liminf_{n\to\infty} \frac{\sum\limits_{k\in A; k\leq n} \frac 1k}{S(n)} \qquad \ol \delta(A) = \limsup_{n\to\infty}
\frac{\sum\limits_{k\in A; k\leq n} \frac 1k}{S(n)}$$ are called
lower and upper logarithmic density of $A$.

We have $S(n)\sim \ln n$ (we use the Landau notation). This
follows from the fact that $\limti n S(n)-\ln n=\gamma$ is Euler's
constant. Therefore we can use $\ln n$ instead of $S(n)$ in the
definition of logarithmic density as well.

The sum in the definition of logarithmic density can be rewritten
using Iverson's convention as $\sum_{k=1}^n \frac 1k [k\in A]$.
(This means that we only add elements fulfilling the condition
$k\in A$. This notation is introduced in \cite[p.24]{knuth}.)

\begin{THM}
For any subset $A\subseteq\N$ 
$$\ul d(A) \leq \ul \delta (A) \leq \ol \delta(A) \leq \ol d(A)$$
holds.
\end{THM}

\begin{proof}
We first observe that
\begin{gather*}
\frac 1k [k\in A]=\frac{A(k)-A(k-1)}k,\\
D(n):=\sum_{k=1}^n \frac 1k [k\in A] = \frac{A(n)}n+
\sum_{k=1}^{n-1} \frac{A(k)}{k(k+1)}
\end{gather*}
There exists an $n_0\in\N$ such that for each $n\geq n_0$ it holds
$\ul d(A)-\ve \leq \frac{A(n)}n \leq \ol d(A) + \ve$.

We denote $C:=1+S(n_0)$. For $n\geq n_0$ we get
\begin{gather*}
D(n) \leq C + \sum_{k=n_0}^{n-1} \frac{A(k)}k\cdot \frac 1{k+1}
\leq C + (\ol d(A)+\ve) \sum_{k=n_0}^{n-1} \frac 1{k+1} \sim (\ol
d(A)+\ve) \ln n,\\
\ol\delta(A)=\limsup_{n\to\infty} \frac{D(n)}{\ln n} \leq \ol
d(A)+\ve.
\end{gather*}
This inequality holds for any $\ve>0$, thus $\ol\delta(A)\leq \ol
d(A)$.

For the proof of the inequality for lower densities we put
$C':=\sum_{k=1}^{n_0-1} \frac{A(k)}{k(k+1)}-(\ul d(A)-\ve)S(n_0)$.
We get
\begin{multline*}
D(n)\geq C' + (\ul d(A)-\ve)S(n_0) + (\ul d(A)-\ve) \sum_{k=n_0}^n
\frac1{k+1} =\\ C'+(\ul d(A)-\ve)S(n) \sim (\ul d(A)-\ve)\ln n
\end{multline*}
and this implies $\ul\delta(A)\geq\ul d(A)$.
\end{proof}

For the proof using Abel's partial summation see \cite{steuding}
or \cite{tenenbaum}.

\begin{COR}
If a set has asymptotic density, then it has logarithmic density,
too.
\end{COR}

A well-known example of a set having logarithmic density but not
having asymptotic density is the set of all numbers with the first
digit equal to 1.

It can be moreover proved, that for any real numbers $0\leq\ul
\alpha\leq\ul \beta\leq \ol \beta \leq \ol \alpha \leq 1$ there
exists a set $A\subseteq\N$ such that $\ul d(A)=\ul\alpha$,
$\ul\delta(A)=\ul\beta$, $\ol\delta(A)=\ol\beta$ and $\ol
d(A)=\ol\alpha$ (see \cite{misik}).

\begin{thebibliography}{1}

\bibitem{knuth}
R.~L. Graham, D.~E. Knuth, and O.~Patashnik.
\newblock {\em {Concrete mathematics. A foundation for computer science.}}
\newblock {Addison-Wesley}, 1989.

\bibitem{misik}
L.~Mi\v{s}{\'\i}k.
\newblock {Sets of positive integers with prescribed values of densities.}
\newblock {\em Mathematica Slovaca}, 52(3):289--296, 2002.

\bibitem{ostmann}
H.~H. Ostmann.
\newblock {\em Additive {Z}ahlentheorie {I}}.
\newblock Springer-Verlag, Berlin-G\"ottingen-Heidelberg, 1956.

\bibitem{steuding}
J.~Steuding.
\newblock \PMlinkexternal{Probabilistic number theory.}{http://www.math.uni-frankfurt.de/\~{}steuding/steuding/prob.pdf}

\bibitem{tenenbaum}
G.~Tenenbaum.
\newblock {\em {Introduction to analytic and probabilistic number theory}}.
\newblock {Cambridge Univ. Press}, Cambridge, 1995.

\end{thebibliography}
%%%%%
%%%%%
\end{document}
