\documentclass[12pt]{article}
\usepackage{pmmeta}
\pmcanonicalname{SchnirelmannDensity}
\pmcreated{2013-03-22 13:19:36}
\pmmodified{2013-03-22 13:19:36}
\pmowner{bbukh}{348}
\pmmodifier{bbukh}{348}
\pmtitle{Schnirelmann density}
\pmrecord{9}{33838}
\pmprivacy{1}
\pmauthor{bbukh}{348}
\pmtype{Definition}
\pmcomment{trigger rebuild}
\pmclassification{msc}{11B13}
\pmclassification{msc}{11B05}
\pmsynonym{Shnirel'man density}{SchnirelmannDensity}
\pmsynonym{Shnirelman density}{SchnirelmannDensity}
\pmrelated{Basis2}
\pmrelated{EssentialComponent}
\pmrelated{MannsTheorem}

\endmetadata

% this is the default PlanetMath preamble.  as your knowledge
% of TeX increases, you will probably want to edit this, but
% it should be fine as is for beginners.

% almost certainly you want these
\usepackage{amssymb}
\usepackage{amsmath}
\usepackage{amsfonts}

% used for TeXing text within eps files
%\usepackage{psfrag}
% need this for including graphics (\includegraphics)
%\usepackage{graphicx}
% for neatly defining theorems and propositions
%\usepackage{amsthm}
% making logically defined graphics
%%%\usepackage{xypic}

% there are many more packages, add them here as you need them

% define commands here
\begin{document}
Let $A$ be a subset of $\mathbb{Z}$, and let $A(n)$ be number of elements of $A$ in $[1,n]$. \emph{\PMlinkescapetext{Schnirelmann density}} of $A$ is
\begin{equation*}
\sigma A = \inf_n \frac{A(n)}{n}.
\end{equation*}

\PMlinkescapetext{Schnirelmann density} has the following properties:
\begin{enumerate}
\item $A(n)\geq n \sigma A$ for all $n$.
\item $\sigma A=1$ if and only if $\mathbb{N}\subseteq A$
\item if $1$ does not belong to $A$, then $\sigma A=0$.
\end{enumerate}

Schnirelmann proved that if $0 \in A \cap B$ then
\begin{equation*}
\sigma(A+B)\geq \sigma A + \sigma B - \sigma A \cdot \sigma B
\end{equation*}
and also if $\sigma A + \sigma B \geq 1$, then $\sigma (A+B)=1$. From these he deduced that if $\sigma A>0$ then $A$ is an additive basis.
%%%%%
%%%%%
\end{document}
