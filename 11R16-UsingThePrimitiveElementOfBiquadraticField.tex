\documentclass[12pt]{article}
\usepackage{pmmeta}
\pmcanonicalname{UsingThePrimitiveElementOfBiquadraticField}
\pmcreated{2013-03-22 17:54:25}
\pmmodified{2013-03-22 17:54:25}
\pmowner{pahio}{2872}
\pmmodifier{pahio}{2872}
\pmtitle{using the primitive element of biquadratic field}
\pmrecord{12}{40399}
\pmprivacy{1}
\pmauthor{pahio}{2872}
\pmtype{Application}
\pmcomment{trigger rebuild}
\pmclassification{msc}{11R16}
\pmsynonym{expressing two square roots with their sum}{UsingThePrimitiveElementOfBiquadraticField}
\pmsynonym{irrational sum of square roots}{UsingThePrimitiveElementOfBiquadraticField}
\pmrelated{BinomialTheorem}

\endmetadata

% this is the default PlanetMath preamble.  as your knowledge
% of TeX increases, you will probably want to edit this, but
% it should be fine as is for beginners.

% almost certainly you want these
\usepackage{amssymb}
\usepackage{amsmath}
\usepackage{amsfonts}

% used for TeXing text within eps files
%\usepackage{psfrag}
% need this for including graphics (\includegraphics)
%\usepackage{graphicx}
% for neatly defining theorems and propositions
 \usepackage{amsthm}
% making logically defined graphics
%%%\usepackage{xypic}

% there are many more packages, add them here as you need them

% define commands here

\theoremstyle{definition}
\newtheorem*{thmplain}{Theorem}

\begin{document}
Let $m$ and $n$ be two distinct squarefree integers $\neq 1$.\, We want to express their square roots as polynomials of
\begin{align}
\alpha \;:=\; \sqrt{m}\!+\!\sqrt{n}
\end{align}
with rational coefficients.\\

If $\alpha$ is \PMlinkname{cubed}{CubeOfANumber}, the result \PMlinkescapetext{contains} no terms with $\sqrt{mn}$:
\begin{align*}
\alpha^3 &\;=\; (\sqrt{m})^3+3(\sqrt{m})^2\sqrt{n}+3\sqrt{m}(\sqrt{n})^2+(\sqrt{n})^3\\ 
 &\;=\; m\sqrt{m}+3m\sqrt{n}+3n\sqrt{m}+n\sqrt{n}\\ 
 &\;=\; (m+3n)\sqrt{m}+(3m+n)\sqrt{n}
\end{align*}
Thus, if we subtract from this the product $(3m\!+\!n)\alpha$, the $\sqrt{n}$ term vanishes:
$$\alpha^3\!-\!(3m\!+\!n)\alpha \;=\; (-2m\!+\!2n)\sqrt{m}$$
Dividing this equation by $-2m\!+\!2n$ ($\neq 0$) yields
\begin{align}
\sqrt{m} \;=\; \frac{\alpha^3\!-\!(3m\!+\!n)\alpha}{2(-m\!+\!n)}.
\end{align}
Similarly, we have
\begin{align}
\sqrt{n} \;=\;\frac{\alpha^3\!-\!(m\!+\!3n)\alpha}{2(m\!-\!n)}.
\end{align}
The \PMlinkescapetext{representations} (2) and (3) may be interpreted as such polynomials as intended.

Multiplying the equations (2) and (3) we obtain a corresponding \PMlinkescapetext{representation} for the square root of $mn$ which also lies in the quartic field \,$\mathbb{Q}(\sqrt{m},\,\sqrt{n}) = \mathbb{Q}(\sqrt{m}\!+\!\sqrt{n})$:
\begin{align*}
\sqrt{mn} \;=\; \frac{\alpha^6\!-\!4(m\!+\!n)\alpha^4\!+\!(3m^2\!+\!10mn\!+\!3n^2)\alpha^2}{4(-m^2\!+\!2mn\!-\!n^2)}\\
\end{align*}


For example, in the special case \,$m := 2,\; n := 3$\, we have
$$\sqrt{2} \;=\; \frac{\alpha^3\!-\!9\alpha}{2}, \quad  \sqrt{3} \;=\; -\frac{\alpha^3\!-\!11\alpha}{2}, \quad 
 \sqrt{6} \;=\; \frac{-\alpha^6\!+\!20\alpha^4\!-\!99\alpha^2}{4}.\\$$

\textbf{Remark.}\, The sum (1) of two square roots of positive squarefree integers is always irrational, since in the contrary case, the equation (3) would say that $\sqrt{n}$ would be rational; this has been proven impossible \PMlinkname{here}{SquareRootOf2IsIrrationalProof}.






%%%%%
%%%%%
\end{document}
