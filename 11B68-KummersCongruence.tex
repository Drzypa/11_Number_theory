\documentclass[12pt]{article}
\usepackage{pmmeta}
\pmcanonicalname{KummersCongruence}
\pmcreated{2013-03-22 15:12:01}
\pmmodified{2013-03-22 15:12:01}
\pmowner{alozano}{2414}
\pmmodifier{alozano}{2414}
\pmtitle{Kummer's congruence}
\pmrecord{5}{36958}
\pmprivacy{1}
\pmauthor{alozano}{2414}
\pmtype{Theorem}
\pmcomment{trigger rebuild}
\pmclassification{msc}{11B68}
\pmsynonym{Kummer congruence}{KummersCongruence}
%\pmkeywords{Kummer}
%\pmkeywords{Bernoulli number}
\pmrelated{CongruenceOfClausenAndVonStaudt}
\pmrelated{IntegralElement}
\pmrelated{OddBernoulliNumbersAreZero}

\endmetadata

% this is the default PlanetMath preamble.  as your knowledge
% of TeX increases, you will probably want to edit this, but
% it should be fine as is for beginners.

% almost certainly you want these
\usepackage{amssymb}
\usepackage{amsmath}
\usepackage{amsthm}
\usepackage{amsfonts}

% used for TeXing text within eps files
%\usepackage{psfrag}
% need this for including graphics (\includegraphics)
%\usepackage{graphicx}
% for neatly defining theorems and propositions
%\usepackage{amsthm}
% making logically defined graphics
%%%\usepackage{xypic}

% there are many more packages, add them here as you need them

% define commands here

\newtheorem*{thm}{Theorem}
\newtheorem{defn}{Definition}
\newtheorem{prop}{Proposition}
\newtheorem{lemma}{Lemma}
\newtheorem{cor}{Corollary}

\theoremstyle{definition}
\newtheorem{exa}{Example}

% Some sets
\newcommand{\Nats}{\mathbb{N}}
\newcommand{\Ints}{\mathbb{Z}}
\newcommand{\Reals}{\mathbb{R}}
\newcommand{\Complex}{\mathbb{C}}
\newcommand{\Rats}{\mathbb{Q}}
\newcommand{\Gal}{\operatorname{Gal}}
\newcommand{\Cl}{\operatorname{Cl}}
\begin{document}
Let $B_k$ denote the $k$th Bernoulli number:
$$B_0=1,\quad B_1=-\frac{1}{2},\quad B_2=\frac{1}{6},\quad B_3=0,\quad B_4=-\frac{1}{30},\ldots,\ B_{10}=\frac{5}{66},\ldots $$
In fact, $B_k=0$ for all odd $k\geq 3$, so we will only consider $B_k$ for even $k$. The following congruence is due to Ernst Eduard Kummer:

\begin{thm}[Kummer's congruence]
Let $p$ be a prime. Suppose that $k\geq 2$ is an even integer which is not divisible by $(p-1)$. Then the quotient $B_k/k$ is $p$-integral, that is, as a fraction in lower terms, $p$ does not divide its denominator. Furthermore, if $h$ is another even integer with $(p-1)\nmid k$ and $k\equiv h \mod (p-1)$ then
$$\frac{B_k}{k}\equiv \frac{B_h}{h} \mod p.$$ 

\end{thm}

The interested reader should see also the congruence of Clausen and von Staudt for a similar result. As an example of Kummer's congruence, let $p=7$ and $k=4$. Then:
$$\frac{B_4}{4}=\frac{-\frac{1}{30}}{4}=-\frac{1}{120}\equiv 6 \mod 7$$
If we pick $h=10$ (so that $10\equiv 4 \mod (p-1)$) then:
$$\frac{B_{10}}{10}=\frac{\frac{5}{66}}{10}=\frac{1}{132}\equiv 6 \mod 7$$
which is what the theorem predicted.
%%%%%
%%%%%
\end{document}
