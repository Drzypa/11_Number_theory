\documentclass[12pt]{article}
\usepackage{pmmeta}
\pmcanonicalname{SuperPouletNumber}
\pmcreated{2013-03-22 18:14:12}
\pmmodified{2013-03-22 18:14:12}
\pmowner{PrimeFan}{13766}
\pmmodifier{PrimeFan}{13766}
\pmtitle{super-Poulet number}
\pmrecord{4}{40825}
\pmprivacy{1}
\pmauthor{PrimeFan}{13766}
\pmtype{Definition}
\pmcomment{trigger rebuild}
\pmclassification{msc}{11A51}

% this is the default PlanetMath preamble.  as your knowledge
% of TeX increases, you will probably want to edit this, but
% it should be fine as is for beginners.

% almost certainly you want these
\usepackage{amssymb}
\usepackage{amsmath}
\usepackage{amsfonts}

% used for TeXing text within eps files
%\usepackage{psfrag}
% need this for including graphics (\includegraphics)
%\usepackage{graphicx}
% for neatly defining theorems and propositions
%\usepackage{amsthm}
% making logically defined graphics
%%%\usepackage{xypic}

% there are many more packages, add them here as you need them

% define commands here

\begin{document}
A {\em super-Poulet number} $n$ is a Poulet number which besides satisfying the congruence $2^n \equiv 2 \mod n$, each of its divisors $d_i$ (for $1 < i \leq \tau(n)$) also satisfies the congruence $2^{d_i} \equiv 2 \mod d_i$.

Two examples: 341 is a super-Poulet number, with its divisors being 1, 11, 31 and 341 itself. We verify that $2^{11} = 2048 = 11 \times 186 + 2$ and $2^{31} = 2147483648 = 31 \times 69273666 + 2$. 341 itself has already been checked when confirmed as a Poulet number. Now, 561 is a Poulet number but not a super-Poulet number since one of its divisors, 33, does not satisfy the congruence: $\frac{2^{33} - 2}{33} \approx 260301048.18181818 \ldots$.

The first few super-Poulet numbers are 341, 1387, 2047, 2701, 3277, 4033, 4369, 4681, 5461, 7957, 8321, which are listed in A050217 of Sloane's OEIS.
%%%%%
%%%%%
\end{document}
