\documentclass[12pt]{article}
\usepackage{pmmeta}
\pmcanonicalname{DigitSum}
\pmcreated{2013-03-22 16:22:53}
\pmmodified{2013-03-22 16:22:53}
\pmowner{PrimeFan}{13766}
\pmmodifier{PrimeFan}{13766}
\pmtitle{digit sum}
\pmrecord{5}{38525}
\pmprivacy{1}
\pmauthor{PrimeFan}{13766}
\pmtype{Definition}
\pmcomment{trigger rebuild}
\pmclassification{msc}{11A63}

% this is the default PlanetMath preamble.  as your knowledge
% of TeX increases, you will probably want to edit this, but
% it should be fine as is for beginners.

% almost certainly you want these
\usepackage{amssymb}
\usepackage{amsmath}
\usepackage{amsfonts}

% used for TeXing text within eps files
%\usepackage{psfrag}
% need this for including graphics (\includegraphics)
%\usepackage{graphicx}
% for neatly defining theorems and propositions
%\usepackage{amsthm}
% making logically defined graphics
%%%\usepackage{xypic}

% there are many more packages, add them here as you need them

% define commands here

\begin{document}
Given an integer $m$ consisting of $k$ digits $d_1, \dots, d_k$ in base $b$, let $$j = \sum_{i = 1}^{k} d_i,$$ then $j$ is the \emph{digit sum} of $m$. Iterating this operation on the digits of $j$ until $j < b$ gives the digital root or repeated digit sum of $m$. The digit sum and digital root of a number are the same only if the additive persistence of the digital root is 1.
%%%%%
%%%%%
\end{document}
