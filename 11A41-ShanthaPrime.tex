\documentclass[12pt]{article}
\usepackage{pmmeta}
\pmcanonicalname{ShanthaPrime}
\pmcreated{2013-03-22 17:49:54}
\pmmodified{2013-03-22 17:49:54}
\pmowner{PrimeFan}{13766}
\pmmodifier{PrimeFan}{13766}
\pmtitle{Shantha  prime}
\pmrecord{26}{40301}
\pmprivacy{1}
\pmauthor{PrimeFan}{13766}
\pmtype{Definition}
\pmcomment{trigger rebuild}
\pmclassification{msc}{11A41}

% this is the default PlanetMath preamble.  as your knowledge
% of TeX increases, you will probably want to edit this, but
% it should be fine as is for beginners.

% almost certainly you want these
\usepackage{amssymb}
\usepackage{amsmath}
\usepackage{amsfonts}

% used for TeXing text within eps files
%\usepackage{psfrag}
% need this for including graphics (\includegraphics)
%\usepackage{graphicx}
% for neatly defining theorems and propositions
%\usepackage{amsthm}
% making logically defined graphics
%%%\usepackage{xypic}

% there are many more packages, add them here as you need them

% define commands here

\begin{document}
A {\em Shantha prime} $p$ is a prime number of the form $p = 3^q - 2$ with $q$ being a \PMlinkname{Mangammal prime}{MangammalPrime}. The smallest Shantha prime is $7 = 3^2 - 2$. The next is $3^{541} - 2$ and has 259 digits. Shantha primes are very rare among the smaller numbers. The above formulation generates mostly composite numbers.

\begin{thebibliography}{1}
\bibitem{ad} A. K. Devaraj, "Euler's Generalization of Fermat's Theorem-A Further Generalization", in {\it Proceedings of Hawaii International Conference on Statistics, Mathematics \& Related Fields}, 2004.
\end{thebibliography}


%%%%%
%%%%%
\end{document}
