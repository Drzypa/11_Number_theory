\documentclass[12pt]{article}
\usepackage{pmmeta}
\pmcanonicalname{UsingPrimitiveRootsAndIndexToSolveCongruences}
\pmcreated{2013-03-22 16:20:58}
\pmmodified{2013-03-22 16:20:58}
\pmowner{alozano}{2414}
\pmmodifier{alozano}{2414}
\pmtitle{using primitive roots and index to solve congruences}
\pmrecord{4}{38483}
\pmprivacy{1}
\pmauthor{alozano}{2414}
\pmtype{Example}
\pmcomment{trigger rebuild}
\pmclassification{msc}{11-00}
\pmrelated{PrimitiveRoot}
\pmrelated{Congruence}
\pmrelated{PolynomialCongruence}

\endmetadata

% this is the default PlanetMath preamble.  as your knowledge
% of TeX increases, you will probably want to edit this, but
% it should be fine as is for beginners.

% almost certainly you want these
\usepackage{amssymb}
\usepackage{amsmath}
\usepackage{amsthm}
\usepackage{amsfonts}

% used for TeXing text within eps files
%\usepackage{psfrag}
% need this for including graphics (\includegraphics)
%\usepackage{graphicx}
% for neatly defining theorems and propositions
%\usepackage{amsthm}
% making logically defined graphics
%%%\usepackage{xypic}

% there are many more packages, add them here as you need them

% define commands here

\newtheorem{thm}{Theorem}
\newtheorem{defn}{Definition}
\newtheorem{prop}{Proposition}
\newtheorem{lemma}{Lemma}
\newtheorem{cor}{Corollary}

\theoremstyle{definition}
\newtheorem{exa}{Example}

% Some sets
\newcommand{\Nats}{\mathbb{N}}
\newcommand{\Ints}{\mathbb{Z}}
\newcommand{\Reals}{\mathbb{R}}
\newcommand{\Complex}{\mathbb{C}}
\newcommand{\Rats}{\mathbb{Q}}
\newcommand{\Gal}{\operatorname{Gal}}
\newcommand{\Cl}{\operatorname{Cl}}
\newcommand{\ind}{\operatorname{ind}}
\begin{document}
The aim of the following example is to illustrate how to use primitive roots and the index of an integer to solve seemingly complicated congruences.

For this example, let $p=13$ and let us attempt to solve $$7x^{10}\equiv 5 \mod 13.$$ Since every prime has a primitive root, we can easily find one. In particular, the number $2$ is a primitive root for $p=13$. Indeed, the powers of $2$ are the following modulo $13$:
\begin{center}
\begin{tabular}{|c|c|c|c|c|c|c|c|c|c|c|c|}
  \hline
  % after \\: \hline or \cline{col1-col2} \cline{col3-col4} ...
  $x$ & $x^2$ & $x^3$ & $x^4$ & $x^5$ & $x^6$ & $x^7$ & $x^8$ & $x^9$ 
& $x^{10}$ & $x^{11}$ & $x^{12}$ \\
  \hline
  $2$ & $4$ & $8$ & $3$ & $6$ & $12$ & $11$ & $9$ & $5$ & $10$ & $7$ & $1$ \\
  \hline
\end{tabular}
\end{center}
The table above allows us to build a table of indices with base $2$ (for the definition of index and its properties which will be used below, see \PMlinkid{this entry}{PropertiesOfTheIndexOfAnIntegerWithRespectToAPrimitiveRoot}):

\begin{center}
\begin{tabular}{|c|c|c|c|c|c|c|c|c|c|c|c|c|}
  \hline
  % after \\: \hline or \cline{col1-col2} \cline{col3-col4} ...
  $x$ & $1$ & $2$ & $3$ & $4$ & $5$ & $6$ & $7$ & $8$ & $9$ 
& $10$ & $11$ & $12$ \\
  \hline
  $\ind x$ & $12$ & $1$ & $4$ & $2$ & $9$ & $5$ & $11$ & $3$ & $8$ & $10$ & $7$ & $6$ \\
  \hline
\end{tabular}
\end{center}

Now we can use the properties of index to solve the equation $7x^{10}\equiv 5 \mod 13$. By taking indices on both sides we obtain
$$\ind (7x^{10})\equiv \ind 7 +10\ind x \equiv \ind 5 \mod 12$$
and so $\ind x \equiv \ind 5 - \ind 7 \equiv 9-11\equiv 10 \mod 12$. The equivalence $10\ind x \equiv 10 \mod 12$ implies $5\ind x\equiv 5 \mod 6$ and hence $\ind x \equiv 1 \mod 6$. Lifting this solution to modulo $12$ we obtain $\ind x \equiv 1$ or $7$ modulo $12$, and by the table of indices, $x\equiv 2$ or $11$ modulo $13$ are the unique solutions of the congruence.
%%%%%
%%%%%
\end{document}
