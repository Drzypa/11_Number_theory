\documentclass[12pt]{article}
\usepackage{pmmeta}
\pmcanonicalname{MillsConstant}
\pmcreated{2013-03-22 16:42:18}
\pmmodified{2013-03-22 16:42:18}
\pmowner{PrimeFan}{13766}
\pmmodifier{PrimeFan}{13766}
\pmtitle{Mills' constant}
\pmrecord{4}{38920}
\pmprivacy{1}
\pmauthor{PrimeFan}{13766}
\pmtype{Definition}
\pmcomment{trigger rebuild}
\pmclassification{msc}{11A41}
\pmsynonym{Mills constant}{MillsConstant}
\pmsynonym{Mills's constant}{MillsConstant}

\endmetadata

% this is the default PlanetMath preamble.  as your knowledge
% of TeX increases, you will probably want to edit this, but
% it should be fine as is for beginners.

% almost certainly you want these
\usepackage{amssymb}
\usepackage{amsmath}
\usepackage{amsfonts}

% used for TeXing text within eps files
%\usepackage{psfrag}
% need this for including graphics (\includegraphics)
%\usepackage{graphicx}
% for neatly defining theorems and propositions
%\usepackage{amsthm}
% making logically defined graphics
%%%\usepackage{xypic}

% there are many more packages, add them here as you need them

% define commands here

\begin{document}
Find the smallest real positive number $M$ such that $\lfloor M^{3^n} \rfloor$ is a prime number for any integer $n > 0$. This is {\em Mills' constant}. Assuming that the Riemann hypothesis is true, the constant's value would be approximately 1.3063778838630806904686144926026 (see A051021 in Sloane's OEIS). According to Caldwell and Cheng, Mills' original paper ``contained no numerics,'' it only proved the existence of such a number; moreover it referred to $\lfloor M^{c^n} \rfloor$, and it is those who hunt for the specific value of $M$ who often choose $c = 3$. Armed with these assignments and assumptions (including certain probable primes), Caldwell and Cheng computed almost seven thousand base 10 digits of Mills' constant. The first few primes generated by Mills' constant would be 2, 11, 1361, 2521008887, etc. (listed in A051254).

\begin{thebibliography}{2}
\bibitem{cc} C. K. Caldwell \& Y. Cheng, ``Determining Mills' Constant and a Note on
Honaker's Problem'' {\it J. Integer Sequences} {\bf 8} (2005): 05.4.1
\bibitem{wm} W. H. Mills, ``A prime-representing function'', {\it Bull. Amer. Math. Soc.} {\bf 53} (1947): 604
\end{thebibliography}
%%%%%
%%%%%
\end{document}
