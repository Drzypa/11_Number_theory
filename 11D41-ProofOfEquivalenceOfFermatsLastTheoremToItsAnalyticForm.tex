\documentclass[12pt]{article}
\usepackage{pmmeta}
\pmcanonicalname{ProofOfEquivalenceOfFermatsLastTheoremToItsAnalyticForm}
\pmcreated{2013-03-22 16:19:04}
\pmmodified{2013-03-22 16:19:04}
\pmowner{whm22}{2009}
\pmmodifier{whm22}{2009}
\pmtitle{proof of equivalence of Fermat's Last Theorem to its analytic form}
\pmrecord{7}{38444}
\pmprivacy{1}
\pmauthor{whm22}{2009}
\pmtype{Proof}
\pmcomment{trigger rebuild}
\pmclassification{msc}{11D41}

\endmetadata

% this is the default PlanetMath preamble.  as your knowledge
% of TeX increases, you will probably want to edit this, but
% it should be fine as is for beginners.

% almost certainly you want these
\usepackage{amssymb}
\usepackage{amsmath}
\usepackage{amsfonts}

% used for TeXing text within eps files
%\usepackage{psfrag}
% need this for including graphics (\includegraphics)
%\usepackage{graphicx}
% for neatly defining theorems and propositions
%\usepackage{amsthm}
% making logically defined graphics
%%%\usepackage{xypic}

% there are many more packages, add them here as you need them

% define commands here

\begin{document}
Consider the Taylor expansion of the cosine function.  We have 

${\rm lim}_{s \to \infty} (A_s)=2-{\rm cos}\, x - {\rm cos}\, y$

and

${\rm lim}_{s \to \infty} (B_s)=1- {\rm cos}\, z$.

For $r>x,y$ the sequence $a_r$ is decreasing as the denominator grows faster than the numerator. 
Hence for $s>x,y$ the sequence $A_s$ is increasing as $A_{s+4}=A_s+a_{s+2}-a_{s+4}$ and
$a_{s+2}>a_{s+4}$.   So if $A_N>0$ for some $N>x,y$, we have $2-{\rm cos}\, x - {\rm cos}\, y>0$. 
Conversely if no such $N$ exists then $A_s \leq 0$ for $s>x,y$, so its limit, $2-{\rm cos}\, x - {\rm cos}\, y$, is also  less than or equal to $0$.  However as this
expression cannot be negative we would have $2-{\rm cos}\, x - {\rm cos}\, y =0$.


Similarly for $r>z$ the sequence $b_r$ is decreasing, and for $s>z$ the sequence $B_s$ is
increasing.  So if $B_M>0$ for some $M>z$ we have $1- {\rm cos}\, z>0$. 
Conversely if no such $M$ exists then $1- {\rm cos}\, z \leq 0$.  However as this
expression cannot be negative we would have $1- {\rm cos}\, z = 0$.

Note that $2-{\rm cos}\, x - {\rm cos}\, y =0$ precisely when $x,y \in 2\pi \mathbb{Z}$.  Also 
$1- {\rm cos}\, z = 0$ precisely when $z \in 2\pi \mathbb{Z}$.

So the \PMlinkescapetext{analytic} form of the theorem may be read:

If for positive reals $x,y,z$ we have $x^n+y^n=z^n$ for some odd integer $n>2$, then either $x$ or
$y$ not in $2\pi \mathbb{Z}$ or $z$ not in $2\pi \mathbb{Z}$.

Clearly this only fails if for positive integers $a,b,c$ and some odd $n>2$, we have 

$(2\pi a)^n+(2\pi b)^n = (2\pi c)^n$.

Dividing through by $(2\pi)^n$ we see that $a^n + b^n =c^n$.

Conversely suppose we have non-zero integers satisfying $a^n + b^n =c^n$ for some $n>2$.  If 
$n=4k$ we have $(a^k)^4+(b^k)^4=(c^k)^4$, contradicting 
example of Fermat's last theorem.  Hence if $n$ is even we may replace $a,b,c$ with
$a^2,b^2,c^2$ and $n$ with $n/2$, which will be odd and greater than 1 (and hence greater than 2 as it is odd).  So without loss of generality we may 
assume $n$ odd. 

Finally replace $a,b,c$ with their absolute values and if 
\PMlinkescapetext{necessary} reorder to obtain a positive 
integer solution.  This would be a counterexample to the 
\PMlinkescapetext{analytic} form of the theorem as stated above.



%%%%%
%%%%%
\end{document}
