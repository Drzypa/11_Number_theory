\documentclass[12pt]{article}
\usepackage{pmmeta}
\pmcanonicalname{Integer}
\pmcreated{2013-03-22 11:50:39}
\pmmodified{2013-03-22 11:50:39}
\pmowner{CWoo}{3771}
\pmmodifier{CWoo}{3771}
\pmtitle{integer}
\pmrecord{13}{30403}
\pmprivacy{1}
\pmauthor{CWoo}{3771}
\pmtype{Definition}
\pmcomment{trigger rebuild}
\pmclassification{msc}{11-00}
\pmclassification{msc}{03-00}
\pmsynonym{rational integer}{Integer}
\pmsynonym{$\mathbb{Z}$}{Integer}
\pmrelated{Irrational}

\usepackage{amssymb}
\usepackage{amsmath}
\usepackage{amsfonts}
\usepackage{graphicx}
%%%%\usepackage{xypic}
\begin{document}
The set of integers, denoted by the symbol $\mathbb{Z}$, is the set $\{\dots -3, -2, -1, 0, 1, 2, 3, \dots\}$ consisting of the natural numbers and their negatives.

Mathematically, $\mathbb{Z}$ is defined to be the set of equivalence classes of pairs of natural numbers $\mathbb{N} \times \mathbb{N}$ under the equivalence relation $(a,b) \sim (c,d)$ if $a+d = b+c$.

Addition and multiplication of integers are defined as follows:
\begin{itemize}
\item $(a,b)+(c,d) := (a+c,b+d)$
\item $(a,b)\cdot(c,d) := (ac+bd,ad+bc)$
\end{itemize}
Typically, the class of $(a,b)$ is denoted by symbol $n$ if $b \leq a$ (resp. $-n$ if $a \leq b$), where $n$ is the unique natural number such that $a=b+n$ (resp. $a+n=b$). Under this notation, we recover the familiar representation of the integers as $\{\dots, -3, -2, -1, 0, 1, 2, 3, \dots\}$. Here are some examples:
\begin{itemize}
\item $0 = $ equivalence class of $(0,0) = $ equivalence class of $(1,1) = \dots$
\item $1 = $ equivalence class of $(1,0) = $ equivalence class of $(2,1) = \dots$
\item $-1 = $ equivalence class of $(0,1) = $ equivalence class of $(1,2) = \dots$
\end{itemize}
The set of integers $\mathbb{Z}$ under the addition and multiplication operations defined above form an integral domain. The integers admit the following ordering relation making $\mathbb{Z}$ into an ordered ring: $(a,b) \leq (c,d)$ in $\mathbb{Z}$ if $a+d \leq b+c$ in $\mathbb{N}$.

The ring of integers is also a Euclidean domain, with valuation given by the absolute value function.
%%%%%
%%%%%
%%%%%
%%%%%
\end{document}
