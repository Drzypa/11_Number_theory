\documentclass[12pt]{article}
\usepackage{pmmeta}
\pmcanonicalname{SmarandacheWellinNumber}
\pmcreated{2013-03-22 15:55:01}
\pmmodified{2013-03-22 15:55:01}
\pmowner{CompositeFan}{12809}
\pmmodifier{CompositeFan}{12809}
\pmtitle{Smarandache-Wellin number}
\pmrecord{9}{37921}
\pmprivacy{1}
\pmauthor{CompositeFan}{12809}
\pmtype{Definition}
\pmcomment{trigger rebuild}
\pmclassification{msc}{11A63}
\pmrelated{FlorentinSmarandache}

% this is the default PlanetMath preamble.  as your knowledge
% of TeX increases, you will probably want to edit this, but
% it should be fine as is for beginners.

% almost certainly you want these
\usepackage{amssymb}
\usepackage{amsmath}
\usepackage{amsfonts}

% used for TeXing text within eps files
%\usepackage{psfrag}
% need this for including graphics (\includegraphics)
%\usepackage{graphicx}
% for neatly defining theorems and propositions
%\usepackage{amsthm}
% making logically defined graphics
%%%\usepackage{xypic}

% there are many more packages, add them here as you need them

% define commands here

\begin{document}
Given a base $b$, concatenate the base $b$ representations of the first $n$ primes into a single integer, placing the first prime as the most significant digit(s) and the $n$th prime as the least significant digit(s). This is the {\em Smarandache-Wellin number} $S_n$.

For example, in base 10, $S_8$ is 235711131719, the concatenation of the strings ``2'', ``3'', ``5'', ``7'', ``11'', ``13'', ``17'' and ``19'' reinterpreted as a single integer.

Placing a decimal point immediately preceding a base 10 Smarandache-Wellin number turns it into an approximation of the Copeland-Erdos constant.

References

R. Crandall and C. Pomerance, Prime Numbers: A Computational Perspective, Springer, NY, 2001: 72

H. Ibstedt, A Few Smarandache Sequences, Smarandache Notions Journal, Vol. 8, No. 1-2-3, 1997: 170 - 183
%%%%%
%%%%%
\end{document}
