\documentclass[12pt]{article}
\usepackage{pmmeta}
\pmcanonicalname{DerivationOfGaussSumUpToASign}
\pmcreated{2013-03-22 13:39:45}
\pmmodified{2013-03-22 13:39:45}
\pmowner{bbukh}{348}
\pmmodifier{bbukh}{348}
\pmtitle{derivation of Gauss sum up to a sign}
\pmrecord{8}{34319}
\pmprivacy{1}
\pmauthor{bbukh}{348}
\pmtype{Derivation}
\pmcomment{trigger rebuild}
\pmclassification{msc}{11L05}

\usepackage{amssymb}
\usepackage{amsmath}
\usepackage{amsfonts}

\newcommand{\Z}{\mathbb{Z}}
\newcommand{\C}{\mathbb{C}}
\newcommand{\leg}[1]{\left(\frac{#1}{p}\right)}

\makeatletter
\@ifundefined{bibname}{}{\renewcommand{\bibname}{References}}
\makeatother
\begin{document}
The Gauss sum can be easily evaluated up to a sign by squaring the original series 
\begin{align*}
g_1^2(\chi) &= \sum_{s \in \Z/p\Z} \leg{s} e^{2 \pi i s/p} \sum_{t \in \Z/p\Z} \leg{t} e^{2 \pi i t/p}\\
            &= \sum_{s,t \in \Z/p\Z} \leg{st} e^{2 \pi i(s+t)/p}\\
\intertext{and summing over a new variable $n=s^{-1} t\pmod p$}
            &= \sum_{s,n \in \Z/p\Z} \leg{n} e^{2 \pi i(s+n s)}\\
            &= \sum_{n \in \Z/p\Z} \leg{n} \sum_{s \in \Z/p\Z} e^{2 \pi i s(n+1)}\\
            &= \sum_{n \in \Z/p\Z} \leg{n} (q[n\equiv -1 \pmod p]-1)\\
            &= p\leg{-1}-\sum_{n \in \Z/p\Z} \leg{n}\\
            &= p\leg{-1}=\begin{cases}\hphantom{+{}}p,&{\rm if\ }p\equiv 1 \pmod 4 ,\\-p,&{\rm if\ }p\equiv 3 \pmod 4.\end{cases}
\end{align*}
% The horrible {\rm if\ } construct is needed to avoid \text or \mbox which 
% makes LaTeX2HTML to choke

\begin{thebibliography}{1}

\bibitem{cite:davenport_multnumtheory}
Harold Davenport.
\newblock {\em Multiplicative Number Theory}.
\newblock Markham Pub.\ Co., 1967.
\newblock \PMlinkexternal{Zbl 0159.06303}{http://www.emis.de/cgi-bin/zmen/ZMATH/en/quick.html?type=html&an=0159.06303}.

\end{thebibliography}

%@BOOK{cite:davenport_multnumtheory,
% author    = {Harold Davenport},
% title     = "Multiplicative Number Theory",
% editor    = {William J. Leveque},
% publisher = {Markham Pub.\ Co.},
% year      = 1967,
% notes     = {\PMlinkexternal{Zbl 0159.06303}{http://www.emis.de/cgi-bin/zmen/ZMATH/en/quick.html?type=html&an=0159.06303}}
%}
%%%%%
%%%%%
\end{document}
