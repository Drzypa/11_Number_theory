\documentclass[12pt]{article}
\usepackage{pmmeta}
\pmcanonicalname{BiquadraticExtension}
\pmcreated{2013-03-22 15:56:21}
\pmmodified{2013-03-22 15:56:21}
\pmowner{Wkbj79}{1863}
\pmmodifier{Wkbj79}{1863}
\pmtitle{biquadratic extension}
\pmrecord{7}{37948}
\pmprivacy{1}
\pmauthor{Wkbj79}{1863}
\pmtype{Definition}
\pmcomment{trigger rebuild}
\pmclassification{msc}{11R16}
\pmrelated{BiquadraticField}
\pmrelated{BiquadraticEquation2}

\endmetadata

% this is the default PlanetMath preamble.  as your knowledge
% of TeX increases, you will probably want to edit this, but
% it should be fine as is for beginners.

% almost certainly you want these
\usepackage{amssymb}
\usepackage{amsmath}
\usepackage{amsfonts}

% used for TeXing text within eps files
%\usepackage{psfrag}
% need this for including graphics (\includegraphics)
%\usepackage{graphicx}
% for neatly defining theorems and propositions
%\usepackage{amsthm}
% making logically defined graphics
%%%\usepackage{xypic}

% there are many more packages, add them here as you need them

% define commands here

\begin{document}
A {\sl biquadratic extension\/} of a field $F$ is a Galois extension $K$ of $F$ such that $\operatorname{Gal} (K/F)$ is isomorphic to the Klein 4-group.  It receives its name from the fact that any such $K$ is the compositum of two distinct quadratic extensions of $F$.  The name can be somewhat misleading, however, since biquadratic extensions of $F$ have exactly three distinct subfields that are quadratic extensions of $F$.  This is easily seen to be true by the fact that the Klein 4-group has exactly three distinct subgroups of \PMlinkname{order}{OrderGroup} 2.

Note that, if $\alpha, \beta \in F$, then $F(\sqrt{\alpha}, \sqrt{\beta})$ is a biquadratic extension of $F$ if and only if none of $\alpha$, $\beta$, and $\alpha \beta$ are squares in $F$.
%%%%%
%%%%%
\end{document}
