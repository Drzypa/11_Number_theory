\documentclass[12pt]{article}
\usepackage{pmmeta}
\pmcanonicalname{Riemannthetafunction}
\pmcreated{2013-03-22 13:23:58}
\pmmodified{2013-03-22 13:23:58}
\pmowner{PrimeFan}{13766}
\pmmodifier{PrimeFan}{13766}
\pmtitle{Riemann $\theta$-function}
\pmrecord{13}{33940}
\pmprivacy{1}
\pmauthor{PrimeFan}{13766}
\pmtype{Definition}
\pmcomment{trigger rebuild}
\pmclassification{msc}{11M06}
\pmsynonym{Riemann theta-function}{Riemannthetafunction}
\pmsynonym{Riemann theta function}{Riemannthetafunction}
\pmrelated{LandsbergSchaarRelation}


\begin{document}
The \emph{Riemann theta function} is a number-theoretic function
which is only really used in
the derivation of the functional equation for the Riemann xi function.
 
The Riemann theta function is defined as:
\[
  \theta(x) = 2\omega (x) + 1,
\]
where $\omega$ is the Riemann omega function.

The \PMlinkname{domain}{Function} of the Riemann theta function is $x > 0$.

To give an explicit form for the theta function, note that
\begin{eqnarray*}
\omega(x) &=& \sum_{n=1}^{\infty} e^{-n^2 \pi x}\\
          &=& \sum_{n=-1}^{-\infty} e^{-(-n)^2 \pi x}\\
          &=& \sum_{n=-1}^{-\infty} e^{-n^2 \pi x}
\end{eqnarray*}
and so
\begin{eqnarray*}
2\omega(x) + 1 &=& \sum_{n=-1}^{-\infty} e^{-n^2 \pi x} + \omega(x) + 1\\
               &=& \sum_{n=-1}^{-\infty} e^{-n^2 \pi x} + \sum_{n=1}^{\infty} e^{-n^2 \pi x} + e^{-0^2 \pi x}\\
               &=& \sum_{n=-\infty}^{\infty} e^{-n^2 \pi x}.
\end{eqnarray*}
Thus we have
\[
\theta(x) = \sum_{n=-\infty}^{\infty} e^{-n^2 \pi x}.
\]

Riemann showed that the theta function satisfied a functional equation,
which was the key step
in the proof of the analytic continuation for the Riemann xi function.
This has direct consequences for the Riemann zeta function.

%%%%%
%%%%%
\end{document}
