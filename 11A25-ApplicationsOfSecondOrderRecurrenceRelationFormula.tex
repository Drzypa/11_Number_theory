\documentclass[12pt]{article}
\usepackage{pmmeta}
\pmcanonicalname{ApplicationsOfSecondOrderRecurrenceRelationFormula}
\pmcreated{2013-03-22 17:51:46}
\pmmodified{2013-03-22 17:51:46}
\pmowner{Wkbj79}{1863}
\pmmodifier{Wkbj79}{1863}
\pmtitle{applications of second order recurrence relation formula}
\pmrecord{8}{40340}
\pmprivacy{1}
\pmauthor{Wkbj79}{1863}
\pmtype{Application}
\pmcomment{trigger rebuild}
\pmclassification{msc}{11A25}
\pmclassification{msc}{11F11}
\pmclassification{msc}{11B39}
\pmclassification{msc}{11B37}
\pmclassification{msc}{03D20}
\pmrelated{FibonacciSequence}
\pmrelated{RamanujanTauFunction}

\endmetadata

\usepackage{amssymb}
\usepackage{amsmath}
\usepackage{amsfonts}
\usepackage{pstricks}
\usepackage{psfrag}
\usepackage{graphicx}
\usepackage{amsthm}
%%\usepackage{xypic}

\begin{document}
\PMlinkescapeword{formula}
\PMlinkescapeword{satisfies}
\PMlinkescapeword{theorem}

We give two applications of the formula for sequences satisfying second order recurrence relations:

\begin{enumerate}
\item Recall that the Fibonacci sequence satisfies the recurrence relation
\[
f_{n+1}=f_n + f_{n-1}.
\]
Thus, $f_0=1$, $A=1$, and $B=1$.  Therefore, the theorem yields the following formula for the Fibonacci sequence:
\[
f_n=\sum_{k=0}^{\lfloor\frac{n}{2}\rfloor} \binom{n-k}{k}
\]
\item \PMlinkname{Fix}{Fix2} a prime $p$ and define a sequence $s$ by $s_n=\tau(p^n)$, where $\tau$ denotes the Ramanujan tau function.  Recall that $\tau$ satisfies
\[
\tau(p^{n+1})=\tau(p)\tau(p^n)-p^{11}\tau(p^{n-1}).
\]
Thus, $s_0=1$, $A=\tau(p)$, and $B=-p^{11}$.  Therefore, the theorem yields
\[
\tau(p^n)=\sum_{k=0}^{\lfloor\frac{n}{2}\rfloor} \binom{n-k}{k} (-p^{11})^k (\tau(p))^{n-2k}.
\]
This formula is valid for all primes $p$ and all nonnegative integers $n$.
\end{enumerate}
%%%%%
%%%%%
\end{document}
