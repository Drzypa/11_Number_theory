\documentclass[12pt]{article}
\usepackage{pmmeta}
\pmcanonicalname{HermiteNumbers}
\pmcreated{2013-03-22 19:08:32}
\pmmodified{2013-03-22 19:08:32}
\pmowner{pahio}{2872}
\pmmodifier{pahio}{2872}
\pmtitle{Hermite numbers}
\pmrecord{8}{42042}
\pmprivacy{1}
\pmauthor{pahio}{2872}
\pmtype{Definition}
\pmcomment{trigger rebuild}
\pmclassification{msc}{11B68}
\pmrelated{EulerNumbers2}
\pmrelated{AppellSequence}

% this is the default PlanetMath preamble.  as your knowledge
% of TeX increases, you will probably want to edit this, but
% it should be fine as is for beginners.

% almost certainly you want these
\usepackage{amssymb}
\usepackage{amsmath}
\usepackage{amsfonts}

% used for TeXing text within eps files
%\usepackage{psfrag}
% need this for including graphics (\includegraphics)
%\usepackage{graphicx}
% for neatly defining theorems and propositions
 \usepackage{amsthm}
% making logically defined graphics
%%%\usepackage{xypic}

% there are many more packages, add them here as you need them

% define commands here

\theoremstyle{definition}
\newtheorem*{thmplain}{Theorem}

\begin{document}
The \emph{Hermite numbers} $H_n$\, may be defined by the generating function
\begin{align}
e^{-t^2} \;:=\; \sum_{n=0}^\infty H_n\frac{t^n}{n!}
\end{align}
which is same as the generating function of Hermite polynomials at the value 0 of the argument $z$.\, After expanding the left hand side of (1) to Taylor series $1-\frac{t^2}{1!}+\frac{t^4}{2!}-\frac{t^6}{3!}+-\ldots$, one can write
\begin{align}
1-2\!\cdot\!\frac{t^2}{2!}+12\!\cdot\!\frac{t^4}{4!}-120\!\cdot\!\frac{t^6}{6!}+-\ldots \;\;=\; \sum_{n=0}^\infty H_n\frac{t^n}{n!}.
\end{align}
Thus one sees that
$$H_0 = 1,\quad H_1 = 0,\quad H_2 = -2,\quad H_3 = 0,\quad H_4 = 12,\quad H_5 = 0,\quad H_6 = -120,\quad\ldots$$
Evidently,
\begin{align}
H_n \;=\; 
\begin{cases}
\frac{(-1)^{\frac{n}{2}}n!}{(\frac{n}{2})!} \textrm{  when  } 2 \mid n\\
0\; \qquad \textrm{ when  } 2 \nmid n 
\end{cases}
\end{align}
The Hermite numbers form the sequence (\PMlinkexternal{Sloane A067994}{http://www.research.att.com/~njas/sequences/index.html?q=A067994&language=english&go=Search})
$$1,\,0,\,-2,\,0,\,12,\,0,\,-120,\,0,\,1680,\,0,\,-30240,\,0,\,665280,\,0,\,-17297280,\,0,\,\ldots$$
which obeys the recurrence relation
\begin{align}
H_n \;=\; 2(1\!-\!n)H_{n-2}.
\end{align}

According to (1), the Hermite numbers satisfy\, $H_n \,=\, H_n(0)$\, where $H_n(x)$ is the Hermite polynomial of degree $n$.\, The \PMlinkescapetext{connection} of Hermite numbers and Hermite polynomials may be expressed also by using symbolic powers
$$H^\nu \;=:\; H_\nu$$
as follows:
\begin{align}
(2x+H)^n \;=\; H_n(x).
\end{align}
This means e.g. that
$$(2x+H)^2 \;=\; (2x)^2+2\cdot2xH^1+H^2 \;=\; 4x^2+4xH_1+H_2 \;=\; 4x^2-2 \;=\; H_2(x).$$

%%%%%
%%%%%
\end{document}
