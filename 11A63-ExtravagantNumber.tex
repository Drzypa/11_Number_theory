\documentclass[12pt]{article}
\usepackage{pmmeta}
\pmcanonicalname{ExtravagantNumber}
\pmcreated{2013-03-22 16:41:20}
\pmmodified{2013-03-22 16:41:20}
\pmowner{PrimeFan}{13766}
\pmmodifier{PrimeFan}{13766}
\pmtitle{extravagant number}
\pmrecord{4}{38900}
\pmprivacy{1}
\pmauthor{PrimeFan}{13766}
\pmtype{Definition}
\pmcomment{trigger rebuild}
\pmclassification{msc}{11A63}
\pmsynonym{wasteful number}{ExtravagantNumber}
\pmrelated{FrugalNumber}
\pmrelated{EquidigitalNumber}

\endmetadata

% this is the default PlanetMath preamble.  as your knowledge
% of TeX increases, you will probably want to edit this, but
% it should be fine as is for beginners.

% almost certainly you want these
\usepackage{amssymb}
\usepackage{amsmath}
\usepackage{amsfonts}

% used for TeXing text within eps files
%\usepackage{psfrag}
% need this for including graphics (\includegraphics)
%\usepackage{graphicx}
% for neatly defining theorems and propositions
%\usepackage{amsthm}
% making logically defined graphics
%%%\usepackage{xypic}

% there are many more packages, add them here as you need them

% define commands here

\begin{document}
An {\em extravagant number} or {\em wasteful number} $n$ is an integer with a base $b$ representation of $k$ digits for which the prime factorization uses more than $k$ digits (with repeated prime factors grouped with exponents and the digits of those exponents counted whenever greater than 1). The first few extravagant numbers in base 10 are 4, 6, 8, 9, 12, 18, 20, 22, 24, 26, 28, 30, 33, 34, 36, 38, 39, 40, 42, 44, 45, 46, 48, etc. (listed in A046760 of Sloane's OEIS).

\begin{thebibliography}{1}
\bibitem{dd} D. Darling, ``Economical number'' in {\it The Universal Book of Mathematics: From Abracadabra To Zeno's paradoxes}. Hoboken, New Jersey: Wiley (2004)
\end{thebibliography}
%%%%%
%%%%%
\end{document}
