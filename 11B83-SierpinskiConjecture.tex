\documentclass[12pt]{article}
\usepackage{pmmeta}
\pmcanonicalname{SierpinskiConjecture}
\pmcreated{2013-03-22 13:34:16}
\pmmodified{2013-03-22 13:34:16}
\pmowner{yark}{2760}
\pmmodifier{yark}{2760}
\pmtitle{Sierpi\'nski conjecture}
\pmrecord{12}{34184}
\pmprivacy{1}
\pmauthor{yark}{2760}
\pmtype{Conjecture}
\pmcomment{trigger rebuild}
\pmclassification{msc}{11B83}
\pmsynonym{Sierpinski conjecture}{SierpinskiConjecture}

\endmetadata

% this is the default PlanetMath preamble.  as your knowledge
% of TeX increases, you will probably want to edit this, but
% it should be fine as is for beginners.

% almost certainly you want these
\usepackage{amssymb}
\usepackage{amsmath}
\usepackage{amsfonts}

% used for TeXing text within eps files
%\usepackage{psfrag}
% need this for including graphics (\includegraphics)
%\usepackage{graphicx}
% for neatly defining theorems and propositions
%\usepackage{amsthm}
% making logically defined graphics
%%%\usepackage{xypic}

% there are many more packages, add them here as you need them

% define commands here
\begin{document}
\PMlinkescapeword{conjecture}
\PMlinkescapeword{property}
\PMlinkescapeword{sufficient}
\PMlinkescapeword{theorem}

In 1960 Wac{\l}aw Sierpi\'nski (1882-1969) proved the following interesting result:

\textbf{Theorem:}
There exist infinitely many odd integers $k$ such that 
$k2^n + 1$ is composite for every integer $n \geq 1$. 

A multiplier $k$ with this property
is called a \PMlinkname{Sierpi\'nski number}{SierpinskiNumbers}.
The Sierpi\'nski problem consists in
determining the smallest Sierpi\'nski number.
In 1962, John Selfridge discovered the Sierpi\'nski number $k = 78557$,
which is now believed to be in fact the smallest such number. 

\textbf{Conjecture:}
The integer $k = 78557$ is the smallest Sierpi\'nski number.

To prove the conjecture, it would be sufficient to exhibit
a prime $k2^n+1$ for each $k < 78557$.
%%%%%
%%%%%
\end{document}
