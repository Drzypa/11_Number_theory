\documentclass[12pt]{article}
\usepackage{pmmeta}
\pmcanonicalname{TheoryOfRationalAndIrrationalNumbers}
\pmcreated{2013-03-22 15:14:10}
\pmmodified{2013-03-22 15:14:10}
\pmowner{alozano}{2414}
\pmmodifier{alozano}{2414}
\pmtitle{theory of rational and irrational numbers}
\pmrecord{15}{37007}
\pmprivacy{1}
\pmauthor{alozano}{2414}
\pmtype{Topic}
\pmcomment{trigger rebuild}
\pmclassification{msc}{11R04}
%\pmkeywords{rational}
%\pmkeywords{irrational}
%\pmkeywords{algebraic}
%\pmkeywords{transcendental}
\pmrelated{TheoryOfAlgebraicNumbers}
\pmrelated{AlgebraicNumberTheory}

\endmetadata

% this is the default PlanetMath preamble.  as your knowledge
% of TeX increases, you will probably want to edit this, but
% it should be fine as is for beginners.

% almost certainly you want these
\usepackage{amssymb}
\usepackage{amsmath}
\usepackage{amsthm}
\usepackage{amsfonts}

% used for TeXing text within eps files
%\usepackage{psfrag}
% need this for including graphics (\includegraphics)
%\usepackage{graphicx}
% for neatly defining theorems and propositions
%\usepackage{amsthm}
% making logically defined graphics
%%%\usepackage{xypic}

% there are many more packages, add them here as you need them

% define commands here

\newtheorem{thm}{Theorem}
\newtheorem{defn}{Definition}
\newtheorem{prop}{Proposition}
\newtheorem{lemma}{Lemma}
\newtheorem{cor}{Corollary}

\theoremstyle{definition}
\newtheorem{exa}{Example}

% Some sets
\newcommand{\Nats}{\mathbb{N}}
\newcommand{\Ints}{\mathbb{Z}}
\newcommand{\Reals}{\mathbb{R}}
\newcommand{\Complex}{\mathbb{C}}
\newcommand{\Rats}{\mathbb{Q}}
\newcommand{\Gal}{\operatorname{Gal}}
\newcommand{\Cl}{\operatorname{Cl}}
\begin{document}
The following entry is some sort of index of articles in PlanetMath about the basic theory of rational and irrational numbers, and it should be studied together with its complement: the \PMlinkname{theory of algebraic and transcendental numbers}{TheoryOfAlgebraicNumbers}. The reader should follow the links in each bullet-point to learn more about each topic. For a somewhat deeper approach to the subject, the reader should read about Algebraic Number Theory. In this entry we will concentrate on the properties of the complex numbers and the extension $\Complex/\Rats$.

There is also a topic entry on rational numbers.

\section{Basic Definitions}

\begin{enumerate}
\item A number is said to be rational if it can be expressed as a quotient of integers (with non-zero denominator). The set of all rational numbers forms a field, denoted by $\Rats$. 

\item Such rational numbers, which are not integers, may be expressed as sum of partial fractions (the denominators being powers of distinct prime numbers). 

\item The real numbers are the set of all possible decimal expansions (where we don't allow any expansion to end in all $9$'s). For the formal definition please see the entry \PMlinkname{real number}{RealNumber}. The real numbers form a field, usually denoted by $\Reals$.

\item A real number is said to be \PMlinkname{irrational}{IrrationalNumber} if it is not rational, i.e. it cannot be expressed as a quotient of integers. \,The decimal expansion is non-periodic for any irrational, but periodic for any rational number.

\item For example, $\sqrt{2}$ is irrational.

\item Commensurable numbers have a rational ratio.\, See also \PMlinkid{sine at irrational multiples of full angle.}{12150}

\end{enumerate}


\section{Small Results}

\begin{enumerate}

\item The field $\Rats$ is, up to an isomorphism, subfield (prime subfield) in  every field where no sum of unities can be 0. \,One may also say that $\Rats$ is the least {\em field of numbers}.

\item \PMlinkname{$\sqrt{2}$ is irrational}{SquareRootOf2IsIrrationalProof}. Similarly $\sqrt{d}$ is irrational as long as $d\in \Nats$ is not a perfect square.

\item The sum of \PMlinkname{two square roots of positive squarefree integers}{UsingThePrimitiveElementOfBiquadraticField} is irrational.

\item Rational and irrational: \,the sum, difference, \PMlinkname{product}{Ring} and quotient of two non-zero real numbers, from which one is rational and the other irrational, is irrational.

\item There exists real functions, which are continuous at any irrational but discontinuous at any rational number (e.g. the Dirichlet's function).

\item Every irrational (and also rational) number is a limit of sequence of rational numbers (see \PMlinkname{real numbers}{RealNumber}). \,An example: \,the sequence \,$(1+\frac{1}{1})^1,\, (1+\frac{1}{2})^2,\, (1+\frac{1}{3})^3,\, ...$\, converges to the number $e$.

\item The number e is irrational (this is not as difficult to prove as it is to show that e is transcendental). In fact, if $r\in \Rats \setminus \{0\}$ then \PMlinkname{$e^r$ is also irrational}{ErIsIrrationalForRinmathbbQsetminus0}. There is an easier way to show that e is not a quadratic irrational.

\item Every real transcendental number (such as $e$) is irrational, but not all irrational numbers are transcendental --- some (such as $\sqrt{2}$) are algebraic.

\item \PMlinkname{``Most'' logarithms}{RationalBriggsianLogarithmsOfIntegers} of positive integers are irrational (and transcendental).

\item If $a^n$ is irrational then $a$ is irrational (see \PMlinkname{here}{IfAnIsIrrationalThenAIsIrrational}).

\item A surprising fact: an irrational to an irrational power can be rational.

\item \PMlinkname{$\pi$ and $\pi^2$ are irrational}{PiAndPi2AreIrrational}.

\end{enumerate}

\section{BIG Results}

The irrational numbers are, in general, ``easily'' understood. The BIG theorems appear in the theory of transcendental numbers. Still, there are some open problems: \,is Euler's constant irrational? is $\pi+e$ rational?
%%%%%
%%%%%
\end{document}
