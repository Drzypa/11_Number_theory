\documentclass[12pt]{article}
\usepackage{pmmeta}
\pmcanonicalname{Regulator}
\pmcreated{2013-03-22 13:54:34}
\pmmodified{2013-03-22 13:54:34}
\pmowner{alozano}{2414}
\pmmodifier{alozano}{2414}
\pmtitle{regulator}
\pmrecord{8}{34663}
\pmprivacy{1}
\pmauthor{alozano}{2414}
\pmtype{Definition}
\pmcomment{trigger rebuild}
\pmclassification{msc}{11R27}
%\pmkeywords{regulator}
%\pmkeywords{unit}
\pmrelated{NumberField}
\pmrelated{DirichletsUnitTheorem}
\pmrelated{ClassNumberFormula}
\pmrelated{RegulatorOfAnEllipticCurve}
\pmdefines{regulator of a number field}

% this is the default PlanetMath preamble.  as your knowledge
% of TeX increases, you will probably want to edit this, but
% it should be fine as is for beginners.

% almost certainly you want these
\usepackage{amssymb}
\usepackage{amsmath}
\usepackage{amsthm}
\usepackage{amsfonts}

% used for TeXing text within eps files
%\usepackage{psfrag}
% need this for including graphics (\includegraphics)
%\usepackage{graphicx}
% for neatly defining theorems and propositions
%\usepackage{amsthm}
% making logically defined graphics
%%%\usepackage{xypic}

% there are many more packages, add them here as you need them

% define commands here

\newtheorem{thm}{Theorem}
\newtheorem*{defn}{Definition}
\newtheorem{prop}{Proposition}
\newtheorem{lemma}{Lemma}
\newtheorem{cor}{Corollary}

% Some sets
\newcommand{\Nats}{\mathbb{N}}
\newcommand{\Ints}{\mathbb{Z}}
\newcommand{\Reals}{\mathbb{R}}
\newcommand{\Complex}{\mathbb{C}}
\newcommand{\Rats}{\mathbb{Q}}
\begin{document}
Let $K$ be a number field with $[K:\Rats]=n=r_1+2r_2$. Here $r_1$
denotes the number of real embeddings:
$$\sigma_i\colon K \hookrightarrow \Reals,\quad 1\leq i\leq r_1$$
while $r_2$ is half of the number of complex embeddings:
$$\tau_j\colon K \hookrightarrow \Complex,\quad 1\leq j\leq r_2$$
Note that $\{\tau_j, \bar{\tau}_j\mid 1\leq j\leq r_2\}$ are
\emph{all} the complex embeddings of $K$. Let $r=r_1+r_2$ and for
$1\leq i\leq r$ define the ``norm'' in $K$ corresponding to each
embedding:
$$ \parallel \cdot \parallel _i\colon K^{\times} \to \Reals^+$$
$$ \parallel \alpha \parallel_i = \mid\sigma_i(\alpha)\mid, \quad
1\leq i \leq r_1$$
$$ \parallel \alpha \parallel_{r_1+j} = \mid\tau_j(\alpha)\mid^2, \quad
1\leq j \leq r_2$$ Let $\mathcal{O}_K$ be the ring of integers of
$K$. By Dirichlet's unit theorem, we know that the rank of the
unit group $\mathcal{O}_K^{\times}$ is exactly $r-1=r_1+r_2-1$.
Let
$$\{ \epsilon_1,\epsilon_2,\ldots,\epsilon_{r-1}\}$$
be a fundamental system of generators of $\mathcal{O}_K^{\times}$
modulo roots of unity (this is, modulo the torsion subgroup). Let
$A$ be the $r\times (r-1)$ matrix $$A=\left(
\begin{array}{cccc}
  \log \parallel \epsilon_1 \parallel_1 & \log \parallel \epsilon_2 \parallel_1 & \ldots & \log \parallel \epsilon_{r-1} \parallel_1 \\
  \log \parallel \epsilon_1 \parallel_2 & \log \parallel \epsilon_2 \parallel_2 & \ldots & \log \parallel \epsilon_{r-1} \parallel_2 \\
  \vdots & \vdots & \ddots & \vdots \\
  \log \parallel \epsilon_1 \parallel_r & \log \parallel \epsilon_2 \parallel_r & \ldots & \log \parallel \epsilon_{r-1} \parallel_r \\
\end{array}
\right)$$ and let $A_i$ be the $(r-1)\times(r-1)$ matrix obtained
by deleting the $i$-th row from $A$, $1\leq i\leq r$. It can be
checked that the determinant of $A_i$, $\det{A_i}$, is independent
up to sign of the choice of fundamental system of generators of
$\mathcal{O}_K^{\times}$ and is also independent of the choice of
$i$.
\begin{defn}
The \emph{regulator of} $K$ is defined to be
$$\operatorname{Reg}_K=\mid\det{A_1}\mid$$
\end{defn}

The regulator is one of the main ingredients in the analytic class number formula for number fields.

\begin{thebibliography}{9}
\bibitem{marcus} Daniel A. Marcus, {\em Number Fields},
Springer, New York.
\bibitem{lang} Serge Lang, {\em Algebraic Number Theory}. Springer-Verlag, New York.
\end{thebibliography}
%%%%%
%%%%%
\end{document}
