\documentclass[12pt]{article}
\usepackage{pmmeta}
\pmcanonicalname{IdealClassGroupIsFinite}
\pmcreated{2013-03-22 17:57:23}
\pmmodified{2013-03-22 17:57:23}
\pmowner{rm50}{10146}
\pmmodifier{rm50}{10146}
\pmtitle{ideal class group is finite}
\pmrecord{7}{40456}
\pmprivacy{1}
\pmauthor{rm50}{10146}
\pmtype{Theorem}
\pmcomment{trigger rebuild}
\pmclassification{msc}{11R29}
\pmrelated{IdealNorm}

% this is the default PlanetMath preamble.  as your knowledge
% of TeX increases, you will probably want to edit this, but
% it should be fine as is for beginners.

% almost certainly you want these
\usepackage{amssymb}
\usepackage{amsmath}
\usepackage{amsfonts}

% used for TeXing text within eps files
%\usepackage{psfrag}
% need this for including graphics (\includegraphics)
%\usepackage{graphicx}
% for neatly defining theorems and propositions
\usepackage{amsthm}
% making logically defined graphics
%%%\usepackage{xypic}

% there are many more packages, add them here as you need them

% define commands here
\newcommand{\Ints}{\mathbb{Z}}
\newcommand{\Rats}{\mathbb{Q}}
\newcommand{\bigp}{\mathfrak{P}}
\newcommand{\Alg}{\mathcal{O}}
\DeclareMathOperator{\N}{N}
\DeclareMathOperator{\Cl}{Cl}
\newtheorem{thm}{Theorem}
\newtheorem{lem}{Lemma}
\begin{document}
We give two proofs of the finiteness of the class group, one using the bound provided by \PMlinkname{Minkowski's theorem}{MinkowskisConstant} and a second, more elementary, proof that does not provide the same computational benefits as Minkowski's bound does. Both proofs rely on the following lemma:
\begin{lem} If $K$ is an algebraic extension of $\Rats$ and $0<d\in\Ints$, then there are only a finite number of ideals of norm $d$.
\end{lem}
\textbf{Proof. }
\newline
The norm of a prime ideal $\bigp$ of $K$ lying over a rational prime $p$ is $p^f$, where $f$ is the residue field degree $[\Alg_K/\bigp\Alg_K:\Ints/p\Ints]$, and there are at most $[K:\Rats]$ prime ideals lying over any given rational prime. There are thus only a finite number of possibilities for ideals with norm $d$ - simply factor $d$ into a product of prime powers and note that each prime power must correspond to one of a finite number of possibilities.


The finiteness of the class group now follows trivially from Minkowski's theorem:
\begin{thm} If $K$ is an algebraic extension of $\Rats$, then the class group of $K$, denoted $\Cl(K)$, is finite.
\end{thm}
\textbf{Proof. }
\newline
Minkowski's theorem guarantees that each ideal class contains a representative integral ideal whose norm is bounded by a constant depending only on the field, and the lemma shows that there are only a finite number of integral ideals with norm less than that constant.

Minkowski's theorem gives enough information about the size of the class group to be computationally useful in some cases (see the topic on using Minkowski's constant to find a class number). It does, however, require quite a bit of machinery. To see in a more elementary way that $\Cl(K)$ is finite, one can proceed as follows:

\textbf{Proof. } (alternate proof of theorem)
\newline
By the lemma, it suffices to show that there is \emph{some} constant $C$, depending only on $K$, such that every class in $\Cl(K)$ has a representative $I\subset\Alg_K$ with $\N(I)\leq C$.
For $u\in K$, denote by $T_u$ the linear map left multiplication by $u$, and let $e_1,\ldots,e_n$ be a basis of $\Alg_K$ as a $\Ints$-module (where $[K:\Rats]=n$). Then if $u=\sum b_i(u)e_i$, it follows that
\[\N(u) = \det(T_u)=\det\left(\sum b_i(u)T_{e_i}\right)\]
is a polynomial of total degree at most $n$ in the $b_i(u)$, whose coefficients are functions of the $T_{e_i}$ and thus depend only on $K$ (and not on $u$). Let $C$ be the sum of the magnitude of those coefficients.

Let $c$ be a class in $\Cl(K)$ and let $J\subset \Alg_K$ be a representative of the class $c^{-1}$. Consider $S=\{\sum_{i=1}^n r_ie_i\ \mid\  1\leq r_i\leq \lfloor \N(J)^{1/n}+1\rfloor\}$. The cardinality of $S$ is strictly greater than $\N(J)$, while $\lvert\Alg_K/J\rvert=\N(J)$. So by the pigenhole principle, two distinct elements of $S$ are in the same $J$-coset of $\Alg_K$. Taking their difference, we get an element $0\neq x=\sum b_i(x)e_i\in J$, where not all the $b_i(x)$ are zero, and $\lvert b_i(x)\rvert \leq \N(J)^{1/n}$. 

Now, since $x\in J$, using unique factorization of ideals in the Dedekind ring $\Alg_K$, we can construct an integral ideal $I$ such that $IJ = (x)$, so that $I$ is in the class $c\in\Cl(K)$. Finally,
\[\N(I)\N(J) = \N(x)\leq C(\max(b_1(x),\ldots,b_n(x)))^n \leq C\N(J)\]
so that $\N(I)\leq C$.
%%%%%
%%%%%
\end{document}
