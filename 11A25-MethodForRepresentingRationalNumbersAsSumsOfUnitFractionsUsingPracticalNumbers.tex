\documentclass[12pt]{article}
\usepackage{pmmeta}
\pmcanonicalname{MethodForRepresentingRationalNumbersAsSumsOfUnitFractionsUsingPracticalNumbers}
\pmcreated{2013-03-22 18:07:00}
\pmmodified{2013-03-22 18:07:00}
\pmowner{PrimeFan}{13766}
\pmmodifier{PrimeFan}{13766}
\pmtitle{method for representing rational numbers as sums of unit fractions using practical numbers}
\pmrecord{4}{40663}
\pmprivacy{1}
\pmauthor{PrimeFan}{13766}
\pmtype{Algorithm}
\pmcomment{trigger rebuild}
\pmclassification{msc}{11A25}

% this is the default PlanetMath preamble.  as your knowledge
% of TeX increases, you will probably want to edit this, but
% it should be fine as is for beginners.

% almost certainly you want these
\usepackage{amssymb}
\usepackage{amsmath}
\usepackage{amsfonts}

% used for TeXing text within eps files
%\usepackage{psfrag}
% need this for including graphics (\includegraphics)
%\usepackage{graphicx}
% for neatly defining theorems and propositions
%\usepackage{amsthm}
% making logically defined graphics
%%%\usepackage{xypic}

% there are many more packages, add them here as you need them

% define commands here

\begin{document}
Fibonacci's application for practical numbers $n$ was an algorithm to represent proper fractions $\frac{m}{n}$ (with $m > 1$) as sums of unit fractions $\displaystyle \sum \frac{d_i}{n}$, with the $d_i$ being divisors of the practical number $n$. (By the way, there are infinitely many practical numbers which are also Fibonacci numbers). The method is:

\begin{enumerate}
\item Reduce the fraction to lowest terms. If the numerator is then 1, we're done.
\item Rewrite $m$ as a sum of divisors of $n$.
\item Make those divisors of $n$ that add up to $m$ into the numerators of fractions with $n$ as denominator.
\item Reduce those fractions to lowest terms, thus obtaining the representation $\displaystyle \frac{m}{n} = \sum \frac{d_i}{n}$.
\end{enumerate}

To illustrate the algorithm, let's rewrite $\frac{37}{42}$ as a sum of unit fractions. Since 42 is practical, success is guaranteed. 

At the first step we can't reduce this fraction because 37 is a prime number. So we go on to the second step, and represent 37 as 2 + 14 + 21. This gives us the fractions $$\frac{2}{42} + \frac{14}{42} + \frac{21}{42},$$ which we then reduce to lowest terms: $$\frac{1}{21} + \frac{1}{3} + \frac{1}{2},$$ giving us the desired unit fractions.

\begin{thebibliography}{2}
\bibitem{mh}  M. R. Heyworth, ``More on panarithmic numbers'' {\it New Zealand Math. Mag.} {\bf 17} (1980): 28 - 34
\bibitem{gm} Giuseppe Melfi, ``A survey on practical numbers'' {\it Rend. Sem. Mat. Univ. Pol. Torino} {\bf 53} (1995): 347 - 359
\end{thebibliography}
%%%%%
%%%%%
\end{document}
