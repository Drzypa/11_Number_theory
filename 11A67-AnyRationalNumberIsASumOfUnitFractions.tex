\documentclass[12pt]{article}
\usepackage{pmmeta}
\pmcanonicalname{AnyRationalNumberIsASumOfUnitFractions}
\pmcreated{2013-03-22 12:48:31}
\pmmodified{2013-03-22 12:48:31}
\pmowner{Mathprof}{13753}
\pmmodifier{Mathprof}{13753}
\pmtitle{any rational number is a sum of unit fractions}
\pmrecord{7}{33127}
\pmprivacy{1}
\pmauthor{Mathprof}{13753}
\pmtype{Derivation}
\pmcomment{trigger rebuild}
\pmclassification{msc}{11A67}
\pmsynonym{Egyptian algorithm}{AnyRationalNumberIsASumOfUnitFractions}
%\pmkeywords{greedy algorithm}

% this is the default PlanetMath preamble.  as your knowledge
% of TeX increases, you will probably want to edit this, but
% it should be fine as is for beginners.

% almost certainly you want these
\usepackage{amssymb}
\usepackage{amsmath}
\usepackage{amsfonts}

% used for TeXing text within eps files
%\usepackage{psfrag}
% need this for including graphics (\includegraphics)
%\usepackage{graphicx}
% for neatly defining theorems and propositions
\usepackage{amsthm}
% making logically defined graphics
%%%\usepackage{xypic}

% there are many more packages, add them here as you need them

% define commands here

\newcommand{\Prob}[2]{\mathbb{P}_{#1}\left\{#2\right\}}
\newcommand{\Expect}{\mathbb{E}}
\newcommand{\norm}[1]{\left\|#1\right\|}

% Some sets
\newcommand{\Nats}{\mathbb{N}}
\newcommand{\Ints}{\mathbb{Z}}
\newcommand{\Rats}{\mathbb{Q}}
\newcommand{\Reals}{\mathbb{R}}
\newcommand{\Complex}{\mathbb{C}}


%%%%%% END OF SAVED PREAMBLE %%%%%%
\begin{document}
\subsubsection*{Representation}
Any rational number $\frac{a}{b}\in\Rats$ between 0 and 1 can be represented as a sum of different unit fractions.  This result was known to the Egyptians, whose way for representing rational numbers was as a sum of different unit fractions.

The following greedy algorithm can represent any $0\le \frac{a}{b} <1$ as such a sum:
\begin{enumerate}
\item Let
$$
n = \left \lceil \frac{b}{a} \right \rceil
$$
be the smallest natural number for which $\frac{1}{n} \le \frac{a}{b}$.  If $a=0$, terminate.
\item Output $\frac{1}{n}$ as the next term of the sum.
\item Continue from step 1, setting
$$\frac{a'}{b'} = \frac{a}{b}-\frac{1}{n}.$$
\end{enumerate}

\subsubsection*{Proof of correctness}
\begin{proof}
The algorithm can never output the same unit fraction twice.  Indeed, any $n$ selected in step 1 is at least 2, so $\frac{1}{n-1} < \frac{2}{n}$ -- so the same $n$ cannot be selected twice by the algorithm, as then $n-1$ could have been selected instead of $n$.

It remains to prove that the algorithm terminates.  We do this by induction on $a$.
\begin{description}
\item[For $a=0$:] The algorithm terminates immediately.
\item[For $a>0$:]
The $n$ selected in step 1 satisfies
$$b \le an < b+a.$$
So
$$
\frac{a}{b}-\frac{1}{n} = \frac{an-b}{bn},
$$
and $0 \le an-b < a$ -- by the induction hypothesis, the algorithm terminates for $\frac{a}{b}-\frac{1}{n}$.
\end{description}
\end{proof}

\subsubsection*{Problems}
\begin{enumerate}
\item
The greedy algorithm always works, but it tends to produce unnecessarily large denominators.  For instance, $$\frac{47}{60}=\frac{1}{3}+\frac{1}{4}+\frac{1}{5},$$
but the greedy algorithm selects $\frac{1}{2}$, leading to the representation
$$
\frac{47}{60}=\frac{1}{2}+\frac{1}{4}+\frac{1}{30}.
$$
\item
The representation is \emph{never} unique.  For instance, for any $n$ we have the representations
$$
\frac{1}{n} = \frac{1}{n+1} + \frac{1}{n\cdot(n+1)}
$$
So given any one representation of $\frac{a}{b}$ as a sum of different unit fractions we can take the largest denominator appearing $n$ and replace it with two (larger) denominators.  Continuing the process indefinitely, we see infinitely many such representations, always.
\end{enumerate}
%%%%%
%%%%%
\end{document}
