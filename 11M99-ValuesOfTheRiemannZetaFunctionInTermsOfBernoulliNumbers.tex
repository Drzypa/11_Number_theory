\documentclass[12pt]{article}
\usepackage{pmmeta}
\pmcanonicalname{ValuesOfTheRiemannZetaFunctionInTermsOfBernoulliNumbers}
\pmcreated{2013-03-22 15:12:07}
\pmmodified{2013-03-22 15:12:07}
\pmowner{Mathprof}{13753}
\pmmodifier{Mathprof}{13753}
\pmtitle{values of the Riemann zeta function in terms of Bernoulli numbers}
\pmrecord{7}{36960}
\pmprivacy{1}
\pmauthor{Mathprof}{13753}
\pmtype{Theorem}
\pmcomment{trigger rebuild}
\pmclassification{msc}{11M99}
\pmrelated{BernoulliNumber}
\pmrelated{ValueOfTheRiemannZetaFunctionAtS2}

\endmetadata

% this is the default PlanetMath preamble.  as your knowledge
% of TeX increases, you will probably want to edit this, but
% it should be fine as is for beginners.

% almost certainly you want these
\usepackage{amssymb}
\usepackage{amsmath}
\usepackage{amsthm}
\usepackage{amsfonts}

% used for TeXing text within eps files
%\usepackage{psfrag}
% need this for including graphics (\includegraphics)
%\usepackage{graphicx}
% for neatly defining theorems and propositions
%\usepackage{amsthm}
% making logically defined graphics
%%%\usepackage{xypic}

% there are many more packages, add them here as you need them

% define commands here

\newtheorem*{thm}{Theorem}
\newtheorem{defn}{Definition}
\newtheorem{prop}{Proposition}
\newtheorem{lemma}{Lemma}
\newtheorem{cor}{Corollary}

\theoremstyle{definition}
\newtheorem{exa}{Example}
\newtheorem*{rem}{Remark}

% Some sets
\newcommand{\Nats}{\mathbb{N}}
\newcommand{\Ints}{\mathbb{Z}}
\newcommand{\Reals}{\mathbb{R}}
\newcommand{\Complex}{\mathbb{C}}
\newcommand{\Rats}{\mathbb{Q}}
\newcommand{\Gal}{\operatorname{Gal}}
\newcommand{\Cl}{\operatorname{Cl}}
\begin{document}
\begin{thm}
Let $k$ be an even integer and let $B_k$ be the $k$th Bernoulli number. Let $\zeta(s)$ be the Riemann zeta function. Then:
$$\zeta(k)=\frac{2^{k-1}|B_k|\pi^k}{k!}$$
Moreover, by using the \PMlinkname{functional equation}{RiemannZetaFunction}
, one calculates for all $n\geq 1$:
$$\zeta(1-n)=\frac{(-1)^{n+1}B_n}{n}$$
which shows that $\zeta(1-n)=0$ for $n\geq 3$ odd. For $k\geq 2$ even, one has:
$$\zeta(1-k)=-\frac{B_k}{k}.$$
\end{thm}
\begin{rem}
The zeroes of the zeta function shown above, $\zeta(1-n)=0$ for $n\geq 3$ odd, are usually called the {\it trivial} zeroes of the Riemann zeta function, while the {\it non-trivial} zeroes are those in the critical strip. 
\end{rem}
%%%%%
%%%%%
\end{document}
