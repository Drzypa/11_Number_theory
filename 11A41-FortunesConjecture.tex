\documentclass[12pt]{article}
\usepackage{pmmeta}
\pmcanonicalname{FortunesConjecture}
\pmcreated{2013-03-22 17:31:17}
\pmmodified{2013-03-22 17:31:17}
\pmowner{PrimeFan}{13766}
\pmmodifier{PrimeFan}{13766}
\pmtitle{Fortune's conjecture}
\pmrecord{4}{39913}
\pmprivacy{1}
\pmauthor{PrimeFan}{13766}
\pmtype{Conjecture}
\pmcomment{trigger rebuild}
\pmclassification{msc}{11A41}

\endmetadata

% this is the default PlanetMath preamble.  as your knowledge
% of TeX increases, you will probably want to edit this, but
% it should be fine as is for beginners.

% almost certainly you want these
\usepackage{amssymb}
\usepackage{amsmath}
\usepackage{amsfonts}

% used for TeXing text within eps files
%\usepackage{psfrag}
% need this for including graphics (\includegraphics)
%\usepackage{graphicx}
% for neatly defining theorems and propositions
%\usepackage{amsthm}
% making logically defined graphics
%%%\usepackage{xypic}

% there are many more packages, add them here as you need them

% define commands here

\begin{document}
(Reo F. Fortune) For any integer $n > 0$, the difference between the primorial $$n\# = \prod_{i = 1}^{\pi(n)} p_i$$ (where $\pi(x)$ is the prime counting function and $p_i$ is the $i$th prime number) and the nearest prime number above (excluding the possible primorial prime $n\# + 1$) is always a prime number. That is, any Fortunate number is a Fortunate prime.

It is obvious that since $n\#$ is divisible by each prime $p < p_{\pi(n)}$, then each $n\# + p$ will also be divisible by that same $p$ and thus not prime. If there is a prime $q > n\# + 1$ such that there is a composite number $m = q - n\#$, then $m$ would have to have at least two prime factors both of which would have to be divisible by primes greater than $p_{\pi(n)}$.

Despite verification for the first thousand primorials, this conjecture remains unproven as of 2007. Disproof could require finding a composite Fortunate number. Such a number would have to be odd, and indeed not divisible by the first thousand primes. Chris Caldwell, writing for the Prime Pages, argues that by the prime number theorem, finding a composite Fortunate number is tantamount to searching for a prime gap at least $(\log n\#)^2$ long immediately following a primorial, something he considers unlikely.

\begin{thebibliography}{1}
\bibitem{sg} S. W. Golomb, ``The evidence for Fortune's conjecture,'' {\it Math. Mag.} {\bf 54} (1981): 209 - 210. MR 82i:10053 
\end{thebibliography}
%%%%%
%%%%%
\end{document}
