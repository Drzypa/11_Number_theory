\documentclass[12pt]{article}
\usepackage{pmmeta}
\pmcanonicalname{DiscriminantOfAlgebraicNumber}
\pmcreated{2013-03-22 17:49:59}
\pmmodified{2013-03-22 17:49:59}
\pmowner{pahio}{2872}
\pmmodifier{pahio}{2872}
\pmtitle{discriminant of algebraic number}
\pmrecord{10}{40303}
\pmprivacy{1}
\pmauthor{pahio}{2872}
\pmtype{Theorem}
\pmcomment{trigger rebuild}
\pmclassification{msc}{11R29}
\pmrelated{Discriminant}
\pmrelated{DerivativeOfPolynomial}
\pmdefines{discriminant of number}

% this is the default PlanetMath preamble.  as your knowledge
% of TeX increases, you will probably want to edit this, but
% it should be fine as is for beginners.

% almost certainly you want these
\usepackage{amssymb}
\usepackage{amsmath}
\usepackage{amsfonts}

% used for TeXing text within eps files
%\usepackage{psfrag}
% need this for including graphics (\includegraphics)
%\usepackage{graphicx}
% for neatly defining theorems and propositions
 \usepackage{amsthm}
% making logically defined graphics
%%%\usepackage{xypic}

% there are many more packages, add them here as you need them

% define commands here

\theoremstyle{definition}
\newtheorem*{thmplain}{Theorem}

\begin{document}
\textbf{Theorem.}\, If $\vartheta$ is an algebraic number of degree $n$ with minimal polynomial $f(x)$, then the {\em \PMlinkescapetext{discriminant} of the number} $\vartheta$, i.e. the discriminant $\Delta(1,\,\vartheta,\,\ldots,\,\vartheta^{n-1})$,\, is
$$d(\vartheta) = (-1)^\frac{n(n-1)}{2}\mbox{N}(f'(\vartheta)),$$
where N means the absolute norm.\\

{\em Proof.}  Let the algebraic conjugates of the number $\vartheta$, i.e. all complex zeroes of $f(x)$,\, be\,  $\vartheta_1 = \vartheta,\,\vartheta_2,\,\ldots,\,\vartheta_n$.\, If\, $f(x) = x^n+a_1x^{n-1}+\ldots+a_n$,\, we have 
$$f'(\vartheta) = n\vartheta^{n-1}+(n-1)a_1\vartheta^{n-2}+\ldots+2a_{n-2}\vartheta+a_{n-1} \in \mathbb{Q}(\vartheta).$$
The \PMlinkname{norm}{AbsoluteNorm} of $f'(\vartheta)$ in $\mathbb{Q}(\vartheta)/\mathbb{Q}$ is the product of all 
\PMlinkid{$\mathbb{Q}(\vartheta)$-conjugates}{12046} $[f'(\vartheta)]^{(i)}$ of $f'(\vartheta)$, which is
$$\mbox{N}(f'(\vartheta)) = 
[f'(\vartheta)]^{(1)}[f'(\vartheta)]^{(2)}\cdots[f'(\vartheta)]^{(n)} = f'(\vartheta_1)f'(\vartheta_2)\cdots f'(\vartheta_n).$$
On the other side, the polynonomial $f(x)$ in its linear factors is 
$$f(x) = (x-\vartheta_1)(x-\vartheta_2)\cdots(x-\vartheta_n),$$
whence its derivative may be written
$$f'(x) = \sum_{\nu=1}^n(x-\vartheta_1)\cdots(x-\vartheta_{\nu-1})\,(x-\vartheta_{\nu+1})\cdots(x-\vartheta_n).$$
Substituting\, $x = \vartheta_\nu$\, gives simply
$$f'(\vartheta_\nu) = \prod_{j\neq\nu}(\vartheta_\nu-\vartheta_j)\quad \mbox{for\;\;} \nu = 1,\,\ldots,\,n.$$
Multiplying these equations we obtain 
$$\mbox{N}(f'(\vartheta)) = \prod_{\nu=1}^nf'(\vartheta_\nu) = \prod_{i\neq j}(\vartheta_i-\vartheta_j).$$
The discriminant of $\vartheta$ is same as the discriminant of the equation \,$f(x) = 0$.\, Therefore
$$d(\vartheta) = \left[\prod_{i<j}(\vartheta_i-\vartheta_j)\right]^2,$$
where the number of the factors in the brackets is\, $(n-1)+(n-2)+\ldots+1 = \frac{(n-1)n}{2}$.\, Thus we obtain the asserted result
$$d(\vartheta) = 
\left[\prod_{i<j}(\vartheta_i-\vartheta_j)\right]\cdot(-1)^\frac{n(n-1)}{2}\left[\prod_{j<i}(\vartheta_i-\vartheta_j)\right] 
= (-1)^\frac{n(n-1)}{2}\prod_{i\neq j}(\vartheta_i-\vartheta_j) = (-1)^\frac{n(n-1)}{2}\mbox{N}(f'(\vartheta)).$$

%%%%%
%%%%%
\end{document}
