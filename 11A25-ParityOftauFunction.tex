\documentclass[12pt]{article}
\usepackage{pmmeta}
\pmcanonicalname{ParityOftauFunction}
\pmcreated{2013-03-22 18:55:43}
\pmmodified{2013-03-22 18:55:43}
\pmowner{pahio}{2872}
\pmmodifier{pahio}{2872}
\pmtitle{parity of $\tau$ function}
\pmrecord{6}{41781}
\pmprivacy{1}
\pmauthor{pahio}{2872}
\pmtype{Feature}
\pmcomment{trigger rebuild}
\pmclassification{msc}{11A25}
%\pmkeywords{square of an integer}
\pmrelated{TauFunction}

\endmetadata

% this is the default PlanetMath preamble.  as your knowledge
% of TeX increases, you will probably want to edit this, but
% it should be fine as is for beginners.

% almost certainly you want these
\usepackage{amssymb}
\usepackage{amsmath}
\usepackage{amsfonts}

% used for TeXing text within eps files
%\usepackage{psfrag}
% need this for including graphics (\includegraphics)
%\usepackage{graphicx}
% for neatly defining theorems and propositions
 \usepackage{amsthm}
% making logically defined graphics
%%%\usepackage{xypic}

% there are many more packages, add them here as you need them

% define commands here

\theoremstyle{definition}
\newtheorem*{thmplain}{Theorem}

\begin{document}
If the prime factor decomposition of a positive integer $n$ is
\begin{align}
n \;=\; p_1^{\alpha_1}p_2^{\alpha_2}\cdots p_r^{\alpha_r},
\end{align}
then all positive divisors of $n$ are of the form
$$
p_1^{\nu_1}p_2^{\nu_2}\cdots p_r^{\nu_r} \quad\mbox{where}\quad 0 \le \nu_i \le \alpha_i \quad(i = 1,\,2,\,\ldots,\,r).
$$
Thus the total number of the divisors is
\begin{align}
\tau(n) \;=\; (\alpha_1\!+\!1)(\alpha_2\!+\!1)\cdots(\alpha_r\!+\!1).
\end{align}


From this we see that in \PMlinkescapetext{order} to $\tau(n)$ be an odd number, every sum $\alpha_i\!+\!1$ shall be odd, i.e. every exponent $\alpha_i$ in (1) must be even.\, It means that $n$ has an even number of each of its prime divisors $p_i$; so $n$ is a square of an integer, a perfect square.

Consequently, the number of all positive divisors of an integer is always even, except if the integer is a perfect square.\\

\textbf{Examples}.\, 15 has four positive divisors 1, 3, 5, 15 and the square number 16 five divisors\\ 1, 2, 4, 8, 16.



%%%%%
%%%%%
\end{document}
