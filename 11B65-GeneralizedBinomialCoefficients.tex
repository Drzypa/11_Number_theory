\documentclass[12pt]{article}
\usepackage{pmmeta}
\pmcanonicalname{GeneralizedBinomialCoefficients}
\pmcreated{2013-03-22 14:41:53}
\pmmodified{2013-03-22 14:41:53}
\pmowner{pahio}{2872}
\pmmodifier{pahio}{2872}
\pmtitle{generalized binomial coefficients}
\pmrecord{26}{36309}
\pmprivacy{1}
\pmauthor{pahio}{2872}
\pmtype{Definition}
\pmcomment{trigger rebuild}
\pmclassification{msc}{11B65}
\pmclassification{msc}{05A10}
\pmrelated{BinomialFormula}
\pmrelated{GeneralPower}
\pmrelated{BinomialFormulaForNegativeIntegerPowers}
\pmdefines{Pascal's formula}
\pmdefines{Vandermonde's formula}

\endmetadata

% this is the default PlanetMath preamble.  as your knowledge
% of TeX increases, you will probably want to edit this, but
% it should be fine as is for beginners.

% almost certainly you want these
\usepackage{amssymb}
\usepackage{amsmath}
\usepackage{amsfonts}

% used for TeXing text within eps files
%\usepackage{psfrag}
% need this for including graphics (\includegraphics)
%\usepackage{graphicx}
% for neatly defining theorems and propositions
%\usepackage{amsthm}
% making logically defined graphics
%%%\usepackage{xypic}

% there are many more packages, add them here as you need them

% define commands here
\begin{document}
The binomial coefficients
\begin{align}
    {n\choose r} = \frac{n!}{(n\!-\!r)!r!},
\end{align}
where $n$ is a non-negative integer and\, $r \in \{0,\,1,\,2,\,\ldots,\,n\}$,\,
can be generalized for all integer and non-integer values of $n$ by using the \PMlinkname{reduced}{Division} form
\begin{align}
    {n\choose r} = \frac{n(n\!-\!1)(n\!-\!2)\ldots(n\!-\!r\!+\!1)}{r!}; 
\end{align}
here $r$ may be any non-negative integer.\, Then \PMlinkname{Newton's binomial series}{BinomialFormula} gets the \PMlinkescapetext{simple} form 
\begin{align}
    (1\!+\!z)^{\alpha} = \sum_{r = 0}^{\infty}{\alpha\choose r}z^r 
 = 1\!+\!{\alpha\choose1}z\!+\!{\alpha\choose 2}z^2\!+\cdots
\end{align}
It is not hard to show that the radius of convergence of this series is 1.\, This series expansion is valid for every complex number $\alpha$ when\, $|z| < 1$,\, and it presents such a \PMlinkname{branch}{GeneralPower} of the \PMlinkname{power}{GeneralPower} $(1\!+\!z)^{\alpha}$ which gets the value 1 in the point\, $z = 0$.

In the case that $\alpha$ is a non-negative integer and $r$ is great enough, one factor in the numerator of 
\begin{align}
{\alpha\choose r} = 
  \frac{\alpha(\alpha\!-\!1)(\alpha\!-\!2)\ldots(\alpha\!-\!r\!+\!1)}{r!} 
\end{align}
vanishes, and hence the corresponding binomial coefficient ${\alpha\choose r}$ equals to zero; accordingly also all following binomial coefficients with a greater $r$ are equal to zero.\, It means that the series is left to being a finite sum, which gives the binomial theorem.

For all complex values of $\alpha$, $\beta$ and non-negative integer values of $r$, $s$, the {\em Pascal's formula}
\begin{align}
  {\alpha\choose r}\!+\!{\alpha\choose r\!+\!1} = {{\alpha\!+\!1}\choose{r\!+\!1}}
\end{align}
and {\em Vandermonde's convolution}
\begin{align}
 \sum_{r = 0}^s{\alpha\choose r}\!{\beta\choose{s\!-\!r}} = {{\alpha\!+\!\beta}\choose s}
\end{align}
hold (the latter is proved by expanding the power $(1\!+\!z)^{\alpha+\beta}$ to series).\, Cf. Pascal's rule and Vandermonde identity.
%%%%%
%%%%%
\end{document}
