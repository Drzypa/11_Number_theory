\documentclass[12pt]{article}
\usepackage{pmmeta}
\pmcanonicalname{Primality}
\pmcreated{2013-03-22 17:45:33}
\pmmodified{2013-03-22 17:45:33}
\pmowner{PrimeFan}{13766}
\pmmodifier{PrimeFan}{13766}
\pmtitle{primality}
\pmrecord{6}{40212}
\pmprivacy{1}
\pmauthor{PrimeFan}{13766}
\pmtype{Definition}
\pmcomment{trigger rebuild}
\pmclassification{msc}{11A41}

% this is the default PlanetMath preamble.  as your knowledge
% of TeX increases, you will probably want to edit this, but
% it should be fine as is for beginners.

% almost certainly you want these
\usepackage{amssymb}
\usepackage{amsmath}
\usepackage{amsfonts}

% used for TeXing text within eps files
%\usepackage{psfrag}
% need this for including graphics (\includegraphics)
%\usepackage{graphicx}
% for neatly defining theorems and propositions
%\usepackage{amsthm}
% making logically defined graphics
%%%\usepackage{xypic}

% there are many more packages, add them here as you need them

% define commands here

\begin{document}
A general dictionary would define {\em primality} as ``the quality or condition of being a prime number.'' In mathematics, it might be more useful to define primality as a Boolean-valued function that returns True if the input number is prime and False otherwise. Two examples: the primality of 47 is True; the primality of 42 is False.

It is not necessary to perform integer factorization to know the primality of a given integer, as there are various congruences and other relations which prime numbers satisfy but non-primes don't; these can serve as primality tests. The primality of certain large numbers, such as the thirtieth Fermat number, has been determined even though all we know of its least prime factor is that it is less than the square root of the composite Fermat number. Before the primality of a large number is ascertained, it might be considered a probable prime. 1 is the only integer to be declared non-prime without a previously unknown factor being discovered.
%%%%%
%%%%%
\end{document}
