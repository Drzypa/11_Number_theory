\documentclass[12pt]{article}
\usepackage{pmmeta}
\pmcanonicalname{RegularPrime}
\pmcreated{2013-03-22 12:44:20}
\pmmodified{2013-03-22 12:44:20}
\pmowner{djao}{24}
\pmmodifier{djao}{24}
\pmtitle{regular prime}
\pmrecord{6}{33040}
\pmprivacy{1}
\pmauthor{djao}{24}
\pmtype{Definition}
\pmcomment{trigger rebuild}
\pmclassification{msc}{11R18}
\pmclassification{msc}{11R29}
\pmdefines{irregular prime}

\endmetadata

% this is the default PlanetMath preamble.  as your knowledge
% of TeX increases, you will probably want to edit this, but
% it should be fine as is for beginners.

% almost certainly you want these
\usepackage{amssymb}
\usepackage{amsmath}
\usepackage{amsfonts}

% used for TeXing text within eps files
%\usepackage{psfrag}
% need this for including graphics (\includegraphics)
%\usepackage{graphicx}
% for neatly defining theorems and propositions
%\usepackage{amsthm}
% making logically defined graphics
%%%\usepackage{xypic} 

% there are many more packages, add them here as you need them

% define commands here
\begin{document}
A prime $p$ is {\em regular} if the class number of the cyclotomic field $\mathbb{Q}(\zeta_p)$ is not divisible by $p$ (where $\zeta_p := e^{2 \pi i/p}$ denotes a primitive $p^\mathrm{th}$ root of unity). An {\em irregular prime} is a prime that is not regular.

Regular primes rose to prominence as a result of Ernst Kummer's work in the 1850's on Fermat's Last Theorem. Kummer was able to prove Fermat's Last Theorem in the case where the exponent is a regular prime, a result that prior to Wiles's recent work was the only demonstration of Fermat's Last Theorem for a large class of exponents. In the course of this work Kummer also established the following numerical criterion for determining whether a prime is regular:

\begin{itemize}
\item $p$ is regular if and only if none of the numerators of the Bernoulli numbers $B_0$, $B_2$, $B_4, \ldots, B_{p-3}$ is a multiple of $p$.
\end{itemize}

Based on this criterion it is possible to give a heuristic argument that the regular primes have density $e^{-1/2}$ in the set of all primes~\cite{ir}. Despite this, there is no known proof that the set of regular primes is infinite, although it is known that there are infinitely many irregular primes.

\begin{thebibliography}{9}
\bibitem{ir} Kenneth Ireland \& Michael Rosen, \emph{A
Classical Introduction to Modern Number Theory,} Springer-Verlag, New
York, Second Edition, 1990.
\end{thebibliography}
%%%%%
%%%%%
\end{document}
