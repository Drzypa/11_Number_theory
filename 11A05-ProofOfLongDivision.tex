\documentclass[12pt]{article}
\usepackage{pmmeta}
\pmcanonicalname{ProofOfLongDivision}
\pmcreated{2013-03-22 15:36:00}
\pmmodified{2013-03-22 15:36:00}
\pmowner{Thomas Heye}{1234}
\pmmodifier{Thomas Heye}{1234}
\pmtitle{proof of long division}
\pmrecord{6}{37515}
\pmprivacy{1}
\pmauthor{Thomas Heye}{1234}
\pmtype{Proof}
\pmcomment{trigger rebuild}
\pmclassification{msc}{11A05}
\pmclassification{msc}{12E99}
\pmclassification{msc}{00A05}

% this is the default PlanetMath preamble.  as your knowledge
% of TeX increases, you will probably want to edit this, but
% it should be fine as is for beginners.

% almost certainly you want these
\usepackage{amssymb}
\usepackage{amsmath}
\usepackage{amsfonts}

% used for TeXing text within eps files
%\usepackage{psfrag}
% need this for including graphics (\includegraphics)
%\usepackage{graphicx}
% for neatly defining theorems and propositions
\usepackage{amsthm}
% making logically defined graphics
%%%\usepackage{xypic}

% there are many more packages, add them here as you need them

% define commands here
\begin{document}
\begin{proof}[Proof of theorem 1]
Let $a,b$ be integers, $b \ne 0$. Set
\[q=\begin{cases}
\left\lfloor \frac{a}{b}\right\rfloor &\text{if $b>0$} \\
-\left\lfloor \frac{a}{\lvert b\rvert}\right\rfloor &\text{otherwise},
\end{cases}
\]
and $r=a-q\cdot b$. Since $0\le x -\lfloor x\rfloor < 1$ for any real $x$, we get for positive $b$
\[0\le \frac{a}{b} -q =\frac{r}{b} < 1\],
and for $b<0$
\[0 \le \frac{a}{\lvert b \rvert} -\left\lfloor\frac{a}{\lvert b\rvert}\right\rfloor =\frac{a}{\lvert b\rvert} +q=\frac{r}{\lvert b\rvert} <1,\]
and the statement follows immediately.
\end{proof}

\begin{proof}[Proof of theorem 2]
Let $R$ be a commutative ring with 1, and take $b(x)$ from $R[x]$, where the leading coefficient of $b(x)$ is a unit in $R$. Without loss of generality we may assume the leading coefficient of $b(x)$ is 1.

If $n$ is the degree of $b(x)$, then set
\[q(x)=
\begin{cases}
0&\text{if $\deg(a(x)) <n$}\\
a_n&\text{if $\deg(a(x))=n$},
\end{cases}\]
where $a_n$ is the leading coefficient of $a(x)$. Then $r(x)=a(x) -q(x)\cdot b(x)$ is either 0 or $\deg(r(x))<\deg(b(x))$, as desired.

Now let $m \ge \deg(b(x))$. Then the degree of the polynomial
\[\check{a}(x)=a(x) -a_{m+1}b(x)\cdot x^{m+1-n}\]
is at most $m$. So by assumption we can write $a(x)$ as
\[a(x)=b(x)\cdot(\check{q}(x)+a_{m+1}x^{m+1-n}) +\check{r}(x)\]
where $\check{r}(x)$ is either 0, or its degree is $<b(x)$.
\end{proof}
%%%%%
%%%%%
\end{document}
