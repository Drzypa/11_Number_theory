\documentclass[12pt]{article}
\usepackage{pmmeta}
\pmcanonicalname{IntegerHarmonicMeans}
\pmcreated{2013-11-06 17:18:49}
\pmmodified{2013-11-06 17:18:49}
\pmowner{pahio}{2872}
\pmmodifier{pahio}{2872}
\pmtitle{integer harmonic means}
\pmrecord{20}{42182}
\pmprivacy{1}
\pmauthor{pahio}{2872}
\pmtype{Topic}
\pmcomment{trigger rebuild}
\pmclassification{msc}{11Z05}
\pmclassification{msc}{11D45}
\pmclassification{msc}{11D09}
\pmclassification{msc}{11A05}
\pmrelated{HarmonicMean}
\pmrelated{HarmonicMeanInTrapezoid}

\endmetadata

% this is the default PlanetMath preamble.  as your knowledge
% of TeX increases, you will probably want to edit this, but
% it should be fine as is for beginners.

% almost certainly you want these
\usepackage{amssymb}
\usepackage{amsmath}
\usepackage{amsfonts}

% used for TeXing text within eps files
%\usepackage{psfrag}
% need this for including graphics (\includegraphics)
%\usepackage{graphicx}
% for neatly defining theorems and propositions
 \usepackage{amsthm}
% making logically defined graphics
%%%\usepackage{xypic}

% there are many more packages, add them here as you need them

% define commands here

\theoremstyle{definition}
\newtheorem*{thmplain}{Theorem}

\begin{document}
Let $u$ and $v$ be positive integers.\, As is seen in the \PMlinkname{parent entry}{IntegerContraharmonicMeans}, there exist nontrivial cases 
($u \neq v$) where their contraharmonic mean
\begin{align}
c \;:=\; \frac{u^2\!+\!v^2}{u\!+\!v} \;=\; u\!+\!v-\frac{2uv}{u\!+\!v}
\end{align}
is an integer.\,  Because the subtrahend of the last \PMlinkescapetext{form} is the harmonic mean of $u$ and $v$, the equation means that the contraharmonic mean $c$ and the harmonic mean 
\begin{align}
h \;:=\; \frac{2uv}{u\!+\!v}
\end{align}
of $u$ and $v$ are simultaneously integers.

The integer contraharmonic mean of two distinct positive 
integers ranges exactly the set of hypotenuses of Pythagorean 
triples (see contraharmonic integers), but the integer harmonic 
mean of two distinct positive integers the wider set\, 
$\{3,\,4,\,5,\,6,\,\ldots\}$.\, As a matter of fact, one 
cathetus of a right triangle is the harmonic mean of the same 
positive integers $u$ and $v$ the contraharmonic mean of which 
is the hypotenuse of the triangle (see 
\PMlinkname{Pythagorean triangle}{PythagoreanTriangle}).\\

The following table allows to compare the values of $u$, $v$, 
$c$, $h$ when\, $1\,<\,u\,<\,v$.

\begin{center}
\begin{tabular}{||c||c|c|c|c|c|c|c|c|c|c|c|c|c|c|c|c|c|c|c||}
\hline
$u$ & $2$ & $3$ & $3$ & $4$ & $4$ & $5$ & $5$ & $6$
& $6$ & $6$ & $6$ & $7$ & $7$ & $8$ & $8$ & $8$ & $9$ & $9$ & $...$\\
\hline
$v$ & $6$ & $6$ & $15$ & $12$ & $28$ & $20$ & $45$ & $12$ & $18$
& $30$ & $66$ & $42$ & $91$ & $24$ & $56$ & $120$ & $18$ & $45$ & $...$\\
\hline
$c$ & $5$ & $5$ & $13$ & $10$ & $25$ & $17$ & $41$ & $10$ & $15$
& $26$ & $61$ & $37$ & $85$ & $20$ & $50$& $113$ & $15$ & $39$ & $...$\\
\hline
$h$ & $3$ & $4$ & $5$ & $6$ & $7$ & $8$ & $9$ & $8$ & $9$
& $10$ & $11$ & $12$ & $13$ & $12$ & $14$& $15$ & $12$ & $15$ & $...$\\
\hline
\end{tabular}
\end{center}

Some of the propositions concerning the integer contraharmonic means directly imply corresponding propositions of the integer harmonic means:\\

\textbf{Proposition 1.}\, For any value of $u > 2$, there are at least two \textbf{greater} values  
\begin{align}
v_1 \;:=\; (u\!-\!1)u, \quad v_2 \;:=\; (2u\!-\!1)u
\end{align}
of $v$ such that $h$ in (2) is an integer.\\


\textbf{Proposition 2.}\, For all\, $u > 1$, a necessary condition for $h \in \mathbb{Z}$\, is that
                       $$\gcd(u,\,v) \;>\; 1.$$


\textbf{Proposition 3.}\, If $u$ is an odd prime number, then the values (3) are the only possibilities for\, $v > u$\, enabling integer harmonic means with $u$.\\



\textbf{Proposition 5.}\, When the harmonic mean of two different positive integers $u$ and $v$ is an integer, their sum is never squarefree.\\


\textbf{Proposition 6.}\, For each integer $u > 0$ there are only a finite number of solutions\, $(u,\,v,\,h)$\, of the Diophantine equation (2).\\

Proposition 6 follows also from the inequality
$$\frac{1}{h} \;=\; \frac{1}{2}\!\left(\frac{1}{u}+\frac{1}{v}\right) \;>\; \frac{1}{2u}$$
which yields the estimation
\begin{align}
0 \;<\; h \;<\; 2u
\end{align}
(cf. the above table).\, This is of course true for any harmonic means $h$ of positive numbers $u$ and $v$.\, The difference of $2u$ and $h$ is $\frac{2u^2}{u+v}$.

The estimation (4) implies that the number of solutions is less than $2u$.\, From the proof of the corresponding proposition in the \PMlinkid{parent entry}{11241} one can see that the number in fact does not exceed $u\!-\!1$.


%%%%%
%%%%%
\end{document}
