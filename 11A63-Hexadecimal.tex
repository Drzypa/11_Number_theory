\documentclass[12pt]{article}
\usepackage{pmmeta}
\pmcanonicalname{Hexadecimal}
\pmcreated{2013-03-22 16:20:27}
\pmmodified{2013-03-22 16:20:27}
\pmowner{PrimeFan}{13766}
\pmmodifier{PrimeFan}{13766}
\pmtitle{hexadecimal}
\pmrecord{6}{38472}
\pmprivacy{1}
\pmauthor{PrimeFan}{13766}
\pmtype{Definition}
\pmcomment{trigger rebuild}
\pmclassification{msc}{11A63}
\pmsynonym{hexadecadic}{Hexadecimal}

% this is the default PlanetMath preamble.  as your knowledge
% of TeX increases, you will probably want to edit this, but
% it should be fine as is for beginners.

% almost certainly you want these
\usepackage{amssymb}
\usepackage{amsmath}
\usepackage{amsfonts}

% used for TeXing text within eps files
%\usepackage{psfrag}
% need this for including graphics (\includegraphics)
%\usepackage{graphicx}
% for neatly defining theorems and propositions
%\usepackage{amsthm}
% making logically defined graphics
%%%\usepackage{xypic}

% there are many more packages, add them here as you need them

% define commands here

\begin{document}
The {\em hexadecimal system} is a positional number system with base 16, using the digits 0, 1, 2, 3, 4, 5, 6, 7, 8, 9, A, B, C, D, E and F.\, It offers a compact way of expressing binary numbers.

In hexadecimal, all Mersenne numbers greater than 127 end with the digit F repeated several times, while all Fermat numbers greater than 17 are written with several significant zeroes book-ended by two 1's.

The \PMlinkescapetext{term} hexadecimal is a mixed formation of a Greek begin and a Latin end.\, There is also a less used synonym {\em hexadecadic} of purely Greek \PMlinkescapetext{derivation}.

Some divisibility tests in hexadecimal are:

$n$ is divisible by 2 if its least significant digit is 0, 2, 4, 6, 8, A, C or E.

$n$ is divisible by 4 if its least significant digit is 0, 4, 8 or C.

$n$ is divisible by 8 if its least significant digit is 0, or 8.

$n$ is divisible by 15 if it has digital root F.

$n$ is of course divisible by 16 if it ends in a 0.

$n$ is divisible by 17 if the difference of the odd placed digits and the even place digits of $n$ is a multiple of 17.
%%%%%
%%%%%
\end{document}
