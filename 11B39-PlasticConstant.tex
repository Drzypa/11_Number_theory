\documentclass[12pt]{article}
\usepackage{pmmeta}
\pmcanonicalname{PlasticConstant}
\pmcreated{2013-03-22 16:10:39}
\pmmodified{2013-03-22 16:10:39}
\pmowner{PrimeFan}{13766}
\pmmodifier{PrimeFan}{13766}
\pmtitle{plastic constant}
\pmrecord{8}{38264}
\pmprivacy{1}
\pmauthor{PrimeFan}{13766}
\pmtype{Definition}
\pmcomment{trigger rebuild}
\pmclassification{msc}{11B39}
\pmsynonym{plastic number}{PlasticConstant}
\pmsynonym{silver number}{PlasticConstant}
\pmsynonym{silver constant}{PlasticConstant}

\endmetadata

% this is the default PlanetMath preamble.  as your knowledge
% of TeX increases, you will probably want to edit this, but
% it should be fine as is for beginners.

% almost certainly you want these
\usepackage{amssymb}
\usepackage{amsmath}
\usepackage{amsfonts}

% used for TeXing text within eps files
%\usepackage{psfrag}
% need this for including graphics (\includegraphics)
%\usepackage{graphicx}
% for neatly defining theorems and propositions
%\usepackage{amsthm}
% making logically defined graphics
%%%\usepackage{xypic}

% there are many more packages, add them here as you need them

% define commands here

\begin{document}
Given the equation $P^3 = P + 1$, solve for $P$. The only solution in real numbers is $P = \sqrt[3]{\frac{1}{2}+\frac{1}{6}\sqrt{\frac{23}{3}}}+\sqrt[3]{\frac{1}{2}-\frac{1}{6}\sqrt{\frac{23}{3}}} = \frac{\sqrt[3]{12(9+\sqrt{69})}+\sqrt[3]{12(9-\sqrt{69})}}{6} \approx 1.3247179572447$, and $P$ is the \emph{plastic constant}, also known as the \emph{silver number}.

Another way to calculate the plastic constant is ${{P(n)} \over {P(n - 1)}}$, where $P(n)$ is the $n^{th}$ term of either the Padovan sequence or the Perrin sequence. For about $n > 20$ the approximation is adequate for all practical purposes.

%%%%%
%%%%%
\end{document}
