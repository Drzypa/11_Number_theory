\documentclass[12pt]{article}
\usepackage{pmmeta}
\pmcanonicalname{ExamplesOfPolydivisibleNumbers}
\pmcreated{2013-03-22 16:22:41}
\pmmodified{2013-03-22 16:22:41}
\pmowner{PrimeFan}{13766}
\pmmodifier{PrimeFan}{13766}
\pmtitle{examples of polydivisible numbers}
\pmrecord{4}{38521}
\pmprivacy{1}
\pmauthor{PrimeFan}{13766}
\pmtype{Example}
\pmcomment{trigger rebuild}
\pmclassification{msc}{11A63}

\endmetadata

% this is the default PlanetMath preamble.  as your knowledge
% of TeX increases, you will probably want to edit this, but
% it should be fine as is for beginners.

% almost certainly you want these
\usepackage{amssymb}
\usepackage{amsmath}
\usepackage{amsfonts}

% used for TeXing text within eps files
%\usepackage{psfrag}
% need this for including graphics (\includegraphics)
%\usepackage{graphicx}
% for neatly defining theorems and propositions
%\usepackage{amsthm}
% making logically defined graphics
%%%\usepackage{xypic}

% there are many more packages, add them here as you need them

% define commands here

\begin{document}
Obviously, if one knows a polydivisible number with $k$ digits then one automatically also knows $k - 1$ other polydivisible numbers.

A consequence of a number being polydivisible is that it's also divisible by the number of digits it has. Taking sequence A098952 from Sloane's OEIS and striking out: first, odd numbers less than 10; second, in the range $99 < i < 1000$, all numbers where $d_2 \not\vert 2$; etc., we obtain the sequence of base 10 polydivisible numbers: 1, 2, 4, 6, 8, 10, 12, 14, 16, 18, 20, 22, 24, 26, 28, 30, 32, 34, 36, 38, 40, 42, 44, 46, 48, 50, 52, 54, 56, 58, 60, 62, 64, 66, 68, 70, 72, 74, 76, 78, 80, 82, 84, 86, 88, 90, 92, 94, 96, 98, 102, 105, 108, 120, 123, 126, 129, 132, 135, 138, 141, etc.


%%%%%
%%%%%
\end{document}
