\documentclass[12pt]{article}
\usepackage{pmmeta}
\pmcanonicalname{LegendreSymbol}
\pmcreated{2013-03-22 11:44:46}
\pmmodified{2013-03-22 11:44:46}
\pmowner{alozano}{2414}
\pmmodifier{alozano}{2414}
\pmtitle{Legendre symbol}
\pmrecord{15}{30183}
\pmprivacy{1}
\pmauthor{alozano}{2414}
\pmtype{Definition}
\pmcomment{trigger rebuild}
\pmclassification{msc}{11-00}
\pmclassification{msc}{97U20}
%\pmkeywords{Legendre}
%\pmkeywords{Character}
%\pmkeywords{Jacobi}
\pmrelated{JacobiSymbol}
\pmrelated{EulersCriterion}
\pmrelated{QuadraticResidue}
\pmrelated{KroneckerSymbol}
\pmrelated{QuadraticReciprocityRule}
\pmrelated{QuadraticCongruence}

\endmetadata

\usepackage{amssymb}
\usepackage{amsmath}
\usepackage{amsfonts}
\usepackage{graphicx}
\begin{document}
\textbf{Legendre Symbol.}\\
Let $p$ be an odd prime. The \emph{Legendre symbol} $\left(\frac{a}{p}\right)$ or $(a|p)$ is defined as:
\[
\left(\frac{a}{p}\right) =
\begin{cases}
1 &\text{if }a \text{ is a quadratic residue }\pmod{p}\\
-1 &\text{if }a \text{ is a quadratic nonresidue }\pmod{p}\\
0 & \text{if } p \text{ divides }a
\end{cases}
\]

The Legendre symbol can be computed by means of Euler's criterion or Gauss' lemma.

Generalizations of this symbol are the Jacobi Symbol and the Kronecker symbol.
%%%%%
%%%%%
%%%%%
%%%%%
%%%%%
\end{document}
