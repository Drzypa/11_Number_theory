\documentclass[12pt]{article}
\usepackage{pmmeta}
\pmcanonicalname{ExamplesOfSelfNumbersInAFewSelectedBases}
\pmcreated{2013-03-22 15:56:18}
\pmmodified{2013-03-22 15:56:18}
\pmowner{PrimeFan}{13766}
\pmmodifier{PrimeFan}{13766}
\pmtitle{examples of self numbers in a few selected bases}
\pmrecord{5}{37947}
\pmprivacy{1}
\pmauthor{PrimeFan}{13766}
\pmtype{Example}
\pmcomment{trigger rebuild}
\pmclassification{msc}{11A63}

\endmetadata

% this is the default PlanetMath preamble.  as your knowledge
% of TeX increases, you will probably want to edit this, but
% it should be fine as is for beginners.

% almost certainly you want these
\usepackage{amssymb}
\usepackage{amsmath}
\usepackage{amsfonts}

% used for TeXing text within eps files
%\usepackage{psfrag}
% need this for including graphics (\includegraphics)
%\usepackage{graphicx}
% for neatly defining theorems and propositions
%\usepackage{amsthm}
% making logically defined graphics
%%%\usepackage{xypic}

% there are many more packages, add them here as you need them

% define commands here

\begin{document}
In base 10, the first few  self numbers are 1, 3, 5, 7, 9, 20, 31, 42, 53, 64, 75, 86, 97 (listed in Sloane's A003052). The recurrence relation $S_i = 8 \cdot 10^{i - 1} + S_{i - 1} + 8$ with $S_1 = 9$ gives the list 9, 97, 905, 8913...

In binary, the first few self numbers are 1, 4, 6, 13, 15, 18, 21, 23, 30, 32, 37, 39, 46, 48, 51, 54 (listed in A010061 of Sloane's OEIS). The recurrence relation $S_i = 2^k + S_{i - 1} + 1$ with $S_1 = 1$ and $k$ being the number of bits in the number $S_{i - 1}$ has, gives the list 1, 4, 13, 30...

In hexadecimal, the first few self numbers are 1, 3, 5, 7, 9, 11, 13, 15, 32, 49, 66, 83, 100, 117, 134 (not currently listed in the OEIS). The recurrence relation $S_i = 14 \cdot 16^{i - 1} + S_{i - 1} + 14$ gives the list 1, 239, 3837, 61195...

In factorial base, the first few self numbers are 1, 4, 11, 18, 36, 43, 61, 68, 86, 93, 111, 118, 125, 132. I don't know of a recurrence relation that will generate factorial base self numbers.

Lastly, in Roman numerals, the first few self numbers are 1, 3, 5, 7, 8, 9, 11, 13, 15.
%%%%%
%%%%%
\end{document}
