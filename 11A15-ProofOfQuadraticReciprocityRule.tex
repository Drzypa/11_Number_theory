\documentclass[12pt]{article}
\usepackage{pmmeta}
\pmcanonicalname{ProofOfQuadraticReciprocityRule}
\pmcreated{2013-03-22 13:16:15}
\pmmodified{2013-03-22 13:16:15}
\pmowner{mathcam}{2727}
\pmmodifier{mathcam}{2727}
\pmtitle{proof of quadratic reciprocity rule}
\pmrecord{12}{33752}
\pmprivacy{1}
\pmauthor{mathcam}{2727}
\pmtype{Proof}
\pmcomment{trigger rebuild}
\pmclassification{msc}{11A15}
%\pmkeywords{reciprocity}

\usepackage{amssymb}
\usepackage{amsmath}
\usepackage{amsfonts}
\begin{document}
%11A15
The quadratic reciprocity law is:

\textbf{Theorem:} (Gauss) Let $p$ and $q$ be distinct odd primes,
and write $p=2a+1$ and $q=2b+1$. Then
$\left(\frac{p}{q}\right)\left(\frac{q}{p}\right)=(-1)^{ab}$.

($\left(\frac{v}{w}\right)$ is the Legendre symbol.)

\textbf{Proof:}
Let $R$ be the subset $[-a,a] \times [-b,b]$
of ${\mathbb Z} \times {\mathbb Z}$. Let $S$
be the interval $$[-(pq-1)/2, (pq-1)/2]$$ of ${\mathbb Z}$.
By the Chinese remainder theorem, there exists a unique
bijection $f: S\to R$ such that, for any $s\in S$, if we
write $f(s)=(x,y)$, then
$ x\equiv s \pmod p $ and
$ y\equiv s \pmod q$.
Let $P$ be the subset of $R$ consisting of the values of $f$ on
$[1, (pq-1)/2 ]$. 
$P$ contains, say, $u$ elements of the form
$(x,0)$ such that $x<0$, and $v$ elements of the form
$(0,y)$ with $y<0$. Intending to apply Gauss's lemma,
we seek some kind of comparison between $u$ and $v$.

We define three subsets of $P$ by
\begin{eqnarray*}
R_{0} & = & \{(x,y) \in P | x > 0, y > 0 \} \\
R_{1} & = & \{(x,y) \in P | x < 0, y \ge 0 \} \\
R_{2} & = & \{(x,y) \in P | x \ge 0, y < 0 \}
\end{eqnarray*}
and we let $N_{i}$ be the cardinal of $R_{i}$ for each $i$.

$P$ has $ab+b$ elements in the region $y>0$, namely $f(m)$ for all
$m$ of the form $k+lq$ with
$1 \le k \le b$ and $0 \le l \le a$.
Thus
$$N_{0}+N_{1} = ab + b - (b-v) + u$$
i.e.
\begin{eqnarray}
N_{0}+N_{1} & = & ab+u+v.
\end{eqnarray}
Swapping $p$ and $q$, we have likewise
\begin{eqnarray}
N_{0}+N_{2} & = & ab+u+v.
\end{eqnarray}

Furthermore, for any $s \in S$, if $f(s)=(x,y)$ then $f(-s)=(-x,-y)$.
It follows that for any $(x,y) \in R$
other than $(0,0)$, either $(x,y)$ or $(-x,-y)$ is
in $P$, but not both.
Therefore
\begin{eqnarray}
N_{1}+N_{2} & = & ab+u+v.
\end{eqnarray}
Adding (1), (2), and (3) gives us
$$0 \equiv ab + u + v \pmod 2$$ so $$(-1)^{ab}=(-1)^{u}(-1)^{v}$$
which, in view of Gauss's lemma, is the desired conclusion.

For a bibliography of the more than 200 known proofs of
the QRL, see
\PMlinkexternal{Lemmermeyer}{http://www.rzuser.uni-heidelberg.de/~hb3/fchrono.html}
.
%%%%%
%%%%%
\end{document}
