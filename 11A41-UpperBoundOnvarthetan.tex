\documentclass[12pt]{article}
\usepackage{pmmeta}
\pmcanonicalname{UpperBoundOnvarthetan}
\pmcreated{2013-03-22 16:09:47}
\pmmodified{2013-03-22 16:09:47}
\pmowner{mps}{409}
\pmmodifier{mps}{409}
\pmtitle{upper bound on $\vartheta(n)$}
\pmrecord{5}{38246}
\pmprivacy{1}
\pmauthor{mps}{409}
\pmtype{Theorem}
\pmcomment{trigger rebuild}
\pmclassification{msc}{11A41}

\endmetadata

% this is the default PlanetMath preamble.  as your knowledge
% of TeX increases, you will probably want to edit this, but
% it should be fine as is for beginners.

% almost certainly you want these
\usepackage{amssymb}
\usepackage{amsmath}
\usepackage{amsfonts}

% used for TeXing text within eps files
%\usepackage{psfrag}
% need this for including graphics (\includegraphics)
%\usepackage{graphicx}
% for neatly defining theorems and propositions
\usepackage{amsthm}
% making logically defined graphics
%%%\usepackage{xypic}

% there are many more packages, add them here as you need them

% define commands here
\newtheorem*{theorem*}{Theorem}
\begin{document}
\PMlinkescapeword{term}

\begin{theorem*}
Let $\vartheta(n)$ be the Chebyshev function
\[
\vartheta(n)=\sum_{\substack{p\le n\\p{\:\mathrm{prime}}}} \log p.
\]
Then $\vartheta(n)\le n\log 4$ for all $n\ge 1$.
\end{theorem*}

\begin{proof}
By induction.

The cases for $n=1$ and $n=2$ follow by inspection.

For even $n>2$, the case follows immediately from the case for $n-1$ since $n$ is not prime.

So let $n=2m+1$ with $m>0$ and consider $(1+1)^{2m+1}$ and its \PMlinkname{binomial expansion}{BinomialTheorem}. Since $\displaystyle {{2m+1}\choose m}={{2m+1}\choose{m+1}}$ and each term occurs exactly once, it follows that $\displaystyle {{2m+1}\choose m}\leq 4^m$. Each prime $p$ with $m+1<p\leq 2m+1$ divides ${2m+1}\choose m$, implying that their product also divides $\displaystyle {{2m+1}\choose m}$.  Hence
\[
\vartheta(2m+1)-\vartheta(m+1)\le\log{{2m+1}\choose m}\le m\log 4.
\]
By the induction hypothesis, $\vartheta(m+1)\leq(m+1)\log4$ and so $\vartheta(2m+1)\leq(2m+1)\log4$.
\end{proof}

\begin{thebibliography}{9}
\bibitem{HW}
G.H. Hardy, E.M. Wright, \emph{An Introduction to the Theory of Numbers}, Oxford University Press, 1938.
\end{thebibliography}
%%%%%
%%%%%
\end{document}
