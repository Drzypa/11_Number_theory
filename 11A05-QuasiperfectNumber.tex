\documentclass[12pt]{article}
\usepackage{pmmeta}
\pmcanonicalname{QuasiperfectNumber}
\pmcreated{2013-03-22 16:05:51}
\pmmodified{2013-03-22 16:05:51}
\pmowner{CompositeFan}{12809}
\pmmodifier{CompositeFan}{12809}
\pmtitle{quasiperfect number}
\pmrecord{6}{38160}
\pmprivacy{1}
\pmauthor{CompositeFan}{12809}
\pmtype{Definition}
\pmcomment{trigger rebuild}
\pmclassification{msc}{11A05}
\pmrelated{AlmostPerfectNumber}

% this is the default PlanetMath preamble.  as your knowledge
% of TeX increases, you will probably want to edit this, but
% it should be fine as is for beginners.

% almost certainly you want these
\usepackage{amssymb}
\usepackage{amsmath}
\usepackage{amsfonts}

% used for TeXing text within eps files
%\usepackage{psfrag}
% need this for including graphics (\includegraphics)
%\usepackage{graphicx}
% for neatly defining theorems and propositions
%\usepackage{amsthm}
% making logically defined graphics
%%%\usepackage{xypic}

% there are many more packages, add them here as you need them

% define commands here

\begin{document}
If there exists an abundant number $n$ with divisors $d_1, \ldots, d_k$, such that $$\sum_{i = 1}^k d_i = 2n + 1,$$ that number would be called a {\em quasiperfect number}. Such a number would be $n > 10^{35}$ and have $\omega(n) > 6$ (where $\omega$ is the number of distinct prime factors function).

A quasiperfect number would thus overshoot the mark for being a perfect number by a margin of just 1. (The powers of 2 are short of perfection by a margin of 1).
%%%%%
%%%%%
\end{document}
