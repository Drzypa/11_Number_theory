\documentclass[12pt]{article}
\usepackage{pmmeta}
\pmcanonicalname{ResidueSystems}
\pmcreated{2013-03-22 16:57:08}
\pmmodified{2013-03-22 16:57:08}
\pmowner{pahio}{2872}
\pmmodifier{pahio}{2872}
\pmtitle{residue systems}
\pmrecord{12}{39221}
\pmprivacy{1}
\pmauthor{pahio}{2872}
\pmtype{Definition}
\pmcomment{trigger rebuild}
\pmclassification{msc}{11-00}
\pmrelated{PrimeResidueClass}
\pmrelated{Totative}
\pmrelated{MathbbZ_n}
\pmdefines{complete residue system}
\pmdefines{reduced residue system}
\pmdefines{least nonnegative remainder}
\pmdefines{absolutely least remainder}

% this is the default PlanetMath preamble.  as your knowledge
% of TeX increases, you will probably want to edit this, but
% it should be fine as is for beginners.

% almost certainly you want these
\usepackage{amssymb}
\usepackage{amsmath}
\usepackage{amsfonts}

% used for TeXing text within eps files
%\usepackage{psfrag}
% need this for including graphics (\includegraphics)
%\usepackage{graphicx}
% for neatly defining theorems and propositions
 \usepackage{amsthm}
% making logically defined graphics
%%%\usepackage{xypic}

% there are many more packages, add them here as you need them

% define commands here

\theoremstyle{definition}
\newtheorem*{thmplain}{Theorem}

\begin{document}
\textbf{Definition.}\, Let $m$ be a positive integer.\, A {\em complete residue system modulo $m$} is a set\, $\{a_0,\,a_1,\,\ldots,\,a_{m-1}\}$\, of integers containing one and only one representant from every residue class modulo $m$.\, Thus the numbers $a_\nu$ may be ordered such that for all $\nu = 0,\,1,\,\ldots,\,m\!-\!1$,\, they satisfy \, $a_\nu \equiv \nu \pmod{m}$.\\

The {\em least nonnegative remainders} modulo $m$, i.e. the numbers
$$0,\,1,\,\ldots,\,m\!-\!1$$
form a complete residue system modulo $m$ (see long division).\, If $m$ is odd, then also the {\em absolutely least remainders}
$$-\frac{m\!-\!1}{2},\,-\frac{m\!-\!3}{2},\,\ldots,\,
-2,\,-1,\,0,\,1,\,2,\,\ldots,\,\frac{m\!-\!3}{2},\,\frac{m\!-\!1}{2}$$
offer a complete residue system modulo $m$.\\

\textbf{Theorem 1.}\, If\, $\gcd(a,\,m) = 1$\, and $b$ is an arbitrary integer, then the numbers
$$0a\!+\!b,\,1a\!+\!b,\,2a\!+\!b,\,\ldots,\,(m\!-\!1)a\!+\!b$$
form a complete residue system modulo $m$.\\


One may speak also of a {\em reduced residue system modulo $m$}, which contains only one representant from each prime residue class modulo $m$.\, Such a system consists of $\varphi(m)$ integers, where $\varphi$ means the Euler's totient function.\\

\textbf{Remark.}\, A set of integers forms a reduced residue system modulo $m$ if and only if\\
$1^\circ$ it contains $\varphi(m)$ numbers,\\
$2^\circ$ its numbers are coprime with $m$,\\
$3^\circ$ its numbers are incongruent modulo $m$.\\


\textbf{Theorem 2.}\, If\, $\gcd(a,\,m) = 1$\, and\, $\{k_1,\,k_2,\,\ldots,\,k_{\varphi(m)}\}$\, is a reduced residue system modulo $m$, then also the numbers
$$k_1a,\,k_2a,\,\ldots,\,k_{\varphi(a)}a$$
form a reduced residue system modulo $m$.\\

\begin{thebibliography}{9}
\bibitem{EK}{\sc K. V\"ais\"al\"a:} {\em Lukuteorian ja korkeamman algebran alkeet}.\, Otava, Helsinki (1950).
\end{thebibliography}


%%%%%
%%%%%
\end{document}
