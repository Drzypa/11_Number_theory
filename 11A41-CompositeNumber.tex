\documentclass[12pt]{article}
\usepackage{pmmeta}
\pmcanonicalname{CompositeNumber}
\pmcreated{2013-03-22 12:39:37}
\pmmodified{2013-03-22 12:39:37}
\pmowner{mathcam}{2727}
\pmmodifier{mathcam}{2727}
\pmtitle{composite number}
\pmrecord{9}{32929}
\pmprivacy{1}
\pmauthor{mathcam}{2727}
\pmtype{Definition}
\pmcomment{trigger rebuild}
\pmclassification{msc}{11A41}
\pmsynonym{composite}{CompositeNumber}

\usepackage{amssymb}
\usepackage{amsmath}
\usepackage{amsfonts}

%\usepackage{psfrag}
%\usepackage{graphicx}
%%%\usepackage{xypic}
\begin{document}
A \emph{composite number} is a positive integer which is not prime and not equal to 1.  That is, $n$ is composite if $n = ab$, with $a$ and $b$ natural numbers both not equal to 1.

\paragraph{Examples.}

\begin{itemize}

\item 1 is not composite (and also not prime), by definition.

\item 2 is not composite, as it is prime.

\item 15 is composite, since $15 = 3\cdot 5$.

\item 93555 is composite, since $93555 = 3^5\cdot 5 \cdot 7 \cdot 11$.

\item 52223 is not composite, since it is prime.

\end{itemize}

More generally, we can define compositeness any time there is an ambient notion of an irreducible element. In an integral domain, for example, an element is said to be \emph{composite} if it neither zero, a unit, nor irreducible.
%%%%%
%%%%%
\end{document}
