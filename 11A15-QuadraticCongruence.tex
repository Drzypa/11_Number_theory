\documentclass[12pt]{article}
\usepackage{pmmeta}
\pmcanonicalname{QuadraticCongruence}
\pmcreated{2013-03-22 17:45:30}
\pmmodified{2013-03-22 17:45:30}
\pmowner{pahio}{2872}
\pmmodifier{pahio}{2872}
\pmtitle{quadratic congruence}
\pmrecord{11}{40211}
\pmprivacy{1}
\pmauthor{pahio}{2872}
\pmtype{Theorem}
\pmcomment{trigger rebuild}
\pmclassification{msc}{11A15}
\pmclassification{msc}{11A07}
\pmrelated{LinearCongruence}
\pmrelated{LegendreSymbol}
\pmrelated{QuadraticFormula}
\pmrelated{ConditionalCongruences}
\pmdefines{quadratic congruence}

% this is the default PlanetMath preamble.  as your knowledge
% of TeX increases, you will probably want to edit this, but
% it should be fine as is for beginners.

% almost certainly you want these
\usepackage{amssymb}
\usepackage{amsmath}
\usepackage{amsfonts}

% used for TeXing text within eps files
%\usepackage{psfrag}
% need this for including graphics (\includegraphics)
%\usepackage{graphicx}
% for neatly defining theorems and propositions
 \usepackage{amsthm}
% making logically defined graphics
%%%\usepackage{xypic}

% there are many more packages, add them here as you need them

% define commands here

\theoremstyle{definition}
\newtheorem*{thmplain}{Theorem}

\begin{document}
\PMlinkescapeword{roots} \PMlinkescapeword{root}
Let $a,\,b,\,c$ be known integers and $p$ an odd prime number not dividing $a$.\, The number of non-congruent roots of the quadratic congruence
\begin{align}
               ax^2\!+\!bx\!+\!c \;\equiv\; 0 \pmod{p}
\end{align}
is
\begin{itemize}
\item two, if\, $b^2\!-\!4ac$\, is a quadratic residue modulo $p$;
\item one, if\, $b^2\!-\!4ac \equiv 0 \pmod{p}$;
\item zero, if\, $b^2\!-\!4ac$\, is a quadratic nonresidue modulo $p$.\\
\end{itemize}



{\em Proof.}\, Since\, $\gcd(p,\,4a) = 1$,\, multiplying (1) by $4a$ gives an \PMlinkname{equivalent}{Equivalent3} congruence
$$4a^2x^2\!+\!4abx\!+\!4ac \;\equiv\; 0 \pmod{p}$$
which may furthermore be written as
$$(2ax\!+\!b)^2 \;\equiv\; b^2\!-\!4ac \pmod{p}.$$
Accordingly, one can obtain the the solution of the given congruence from the solution of the pair of congruences
\begin{align*}
\begin{cases}
               y^2 \;\equiv\; b^2\!-\!4ac \pmod{p} \qquad\qquad (2)\\
               2ax\!+\!b \;\equiv\; y \pmod{p}.\; \qquad\qquad (3) \\
\end{cases}
\end{align*}
Case 1:\, $b^2\!-\!4ac$ is a quadratic residue$\pmod{p}$.\, Then (2) has a root \,$y = y_0 \neq 0$,\, and therefore also the second root\, $y = -y_0$.\, The roots\, $y = \pm y_0$ are incongruent, because otherwise one had\; 
$p \mid 2y_0$\; and thus\; $p \mid y_0 \mid y_0^2 \equiv b^2\!-\!4ac$\; which is not possible in this case.\\
Case 2:\, $b^2\!-\!4ac \equiv 0 \pmod{p}$.\, Now (2) implies that\, $y \equiv 0 \pmod{p}$,\, whence the corresponding root $x_0$ of the linear congruence (3) does not allow other incongruent roots for (1).\\
Case 3:\, $b^2\!-\!4ac$ is a quadratic nonresidue$\pmod{p}$.\, The congruence (2) cannot have solutions; the same concerns thus also (1).\\

\textbf{Example.}\, Solve the congruence
$$4x^2+6x-3 \;\equiv\; 0 \pmod{43}.$$
We have\, $b^2\!-\!4ac = 36+4\cdot4\cdot3 = 84 \equiv -2 \pmod{43}$\, and the Legendre symbol
$$\left(\frac{-2}{43}\right) \;=\; \left(\frac{-1}{43}\right)\left(\frac{2}{43}\right) \;=\; -1\cdot(-1) \;=\; 1$$
(see values of the Legendre symbol) says that $-2$ is a quadratic residue modulo 43.\, The congruence corresponding (2) is
$$y^2 \;\equiv\; -2 \pmod{43},$$
which is satisfied by\, $y \equiv \pm16 \pmod{43}$ as one finds after a little experimenting.\, Then we have the two linear congruences\, $2\cdot4x+6 \equiv \pm16 \pmod{43}$,\, i.e.
$$4x \;\equiv\; \pm8-3 \pmod{43}$$
corresponding (3).\, The first of them,\, $4x\equiv 5 \pmod{43}$,\, is satisfied by\, $x = 12$\, and the second,\, $4x\equiv -11 \pmod{43}$,\, by\, $x = 8$.\, Thus the solution of the given congruence is
$$x \;\equiv\; 8 \pmod{43} \quad \mbox{or} \quad x \;\equiv\; 12 \pmod{43}.$$





%%%%%
%%%%%
\end{document}
