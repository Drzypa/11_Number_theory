\documentclass[12pt]{article}
\usepackage{pmmeta}
\pmcanonicalname{ArithmeticFunction}
\pmcreated{2013-03-22 13:50:49}
\pmmodified{2013-03-22 13:50:49}
\pmowner{mathcam}{2727}
\pmmodifier{mathcam}{2727}
\pmtitle{arithmetic function}
\pmrecord{10}{34584}
\pmprivacy{1}
\pmauthor{mathcam}{2727}
\pmtype{Definition}
\pmcomment{trigger rebuild}
\pmclassification{msc}{11A25}
\pmrelated{ConvolutionInversesForArithmeticFunctions}
\pmrelated{PropertyOfCompletelyMultiplicativeFunctions}
\pmrelated{DivisorSumOfAnArithmeticFunction}
\pmdefines{Dirichlet convolution}

% this is the default PlanetMath preamble.  as your knowledge
% of TeX increases, you will probably want to edit this, but
% it should be fine as is for beginners.

% almost certainly you want these
\usepackage{amssymb}
\usepackage{amsmath}
\usepackage{amsfonts}
\usepackage{amsthm}

% used for TeXing text within eps files
%\usepackage{psfrag}
% need this for including graphics (\includegraphics)
%\usepackage{graphicx}
% for neatly defining theorems and propositions
%\usepackage{amsthm}
% making logically defined graphics
%%%\usepackage{xypic}

% there are many more packages, add them here as you need them

% define commands here

\newcommand{\mc}{\mathcal}
\newcommand{\mb}{\mathbb}
\newcommand{\mf}{\mathfrak}
\newcommand{\ol}{\overline}
\newcommand{\ra}{\rightarrow}
\newcommand{\la}{\leftarrow}
\newcommand{\La}{\Leftarrow}
\newcommand{\Ra}{\Rightarrow}
\newcommand{\nor}{\vartriangleleft}
\newcommand{\Gal}{\text{Gal}}
\newcommand{\GL}{\text{GL}}
\newcommand{\Z}{\mb{Z}}
\newcommand{\R}{\mb{R}}
\newcommand{\Q}{\mb{Q}}
\newcommand{\<}{\langle}
\renewcommand{\>}{\rangle}
\begin{document}
An \emph{arithmetic function} is a function $f:\Z^+\ra\mathbb{C}$ from the positive integers to the complex numbers.

Any algebraic function over $\Z^+$, as well as transcendental functions such as $\sin(n\pi)$ and $e^{n\pi i}$ with $n\in \Z^+$ are arithmetic functions.

There are two noteworthy operations on the set of arithmetic functions:

If $f$ and $g$ are two arithmetic functions, the \emph{sum} of $f$ and $g$, denoted $f+g$, is given by 
\begin{align*}
(f+g)(n)=f(n)+g(n),
\end{align*}
and the \emph{Dirichlet convolution} of $f$ and $g$, denoted by $f*g$, is given by
\begin{align*}
(f*g)(n)=\sum_{d|n}f(d)g\left(\frac{n}{d}\right).
\end{align*}

The set of arithmetic functions, equipped with these two binary operations, forms a commutative ring with unity.  The 0 of the ring is the function $f$ such that $f(n)=0$ for any positive integer $n$.  The 1 of the ring is the function $f$ with $f(1)=1$ and  $f(n)=0$ for any $n>1$, and the units of the ring are those arithmetic function $f$ such that $f(1)\neq 0$.

Note that giving a sequence $\{a_n\}$ of complex numbers is equivalent to giving an arithmetic function by associating $a_n$ with $f(n)$.
%%%%%
%%%%%
\end{document}
