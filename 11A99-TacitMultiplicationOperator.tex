\documentclass[12pt]{article}
\usepackage{pmmeta}
\pmcanonicalname{TacitMultiplicationOperator}
\pmcreated{2013-03-22 16:29:36}
\pmmodified{2013-03-22 16:29:36}
\pmowner{CompositeFan}{12809}
\pmmodifier{CompositeFan}{12809}
\pmtitle{tacit multiplication operator}
\pmrecord{5}{38667}
\pmprivacy{1}
\pmauthor{CompositeFan}{12809}
\pmtype{Definition}
\pmcomment{trigger rebuild}
\pmclassification{msc}{11A99}

\endmetadata

% this is the default PlanetMath preamble.  as your knowledge
% of TeX increases, you will probably want to edit this, but
% it should be fine as is for beginners.

% almost certainly you want these
\usepackage{amssymb}
\usepackage{amsmath}
\usepackage{amsfonts}

% used for TeXing text within eps files
%\usepackage{psfrag}
% need this for including graphics (\includegraphics)
%\usepackage{graphicx}
% for neatly defining theorems and propositions
%\usepackage{amsthm}
% making logically defined graphics
%%%\usepackage{xypic}

% there are many more packages, add them here as you need them

% define commands here

\begin{document}
A {\em tacit multiplication operator} is a multiplication operator not actually written in a formula but usually understood from the context, at least by humans. For example, the expression $2\pi r$ is understood to mean $2 \times \pi \times r$. Most advanced computer algebra systems (such as Mathematica) and even the Google calculator are capable of understanding the tacit multiplication operator provided that it does not clash in meaning with another possible command.
%%%%%
%%%%%
\end{document}
