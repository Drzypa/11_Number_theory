\documentclass[12pt]{article}
\usepackage{pmmeta}
\pmcanonicalname{NormAndTraceOfAlgebraicNumber}
\pmcreated{2013-03-22 15:19:08}
\pmmodified{2013-03-22 15:19:08}
\pmowner{pahio}{2872}
\pmmodifier{pahio}{2872}
\pmtitle{norm and trace of algebraic number}
\pmrecord{15}{37125}
\pmprivacy{1}
\pmauthor{pahio}{2872}
\pmtype{Theorem}
\pmcomment{trigger rebuild}
\pmclassification{msc}{11R04}
%\pmkeywords{norm}
%\pmkeywords{trace}
\pmrelated{TheoryOfAlgebraicNumbers}
\pmrelated{AlgebraicNumberTheory}
\pmrelated{IdealNorm}
\pmrelated{UnitsOfRealCubicFieldsWithExactlyOneRealEmbedding}
\pmrelated{IndependenceOfCharacteristicPolynomialOnPrimitiveElement}
\pmdefines{absolute norm}
\pmdefines{absolute trace}

% this is the default PlanetMath preamble.  as your knowledge
% of TeX increases, you will probably want to edit this, but
% it should be fine as is for beginners.

% almost certainly you want these
\usepackage{amssymb}
\usepackage{amsmath}
\usepackage{amsfonts}

% used for TeXing text within eps files
%\usepackage{psfrag}
% need this for including graphics (\includegraphics)
%\usepackage{graphicx}
% for neatly defining theorems and propositions
 \usepackage{amsthm}
% making logically defined graphics
%%%\usepackage{xypic}

% there are many more packages, add them here as you need them

% define commands here

\theoremstyle{definition}
\newtheorem{thmplain}{Theorem}
\begin{document}
\begin{thmplain}
\, Let $K$ be an algebraic number field and $\alpha$ an element of $K$.\, The norm  $\mbox{N}(\alpha)$ and the trace $\mbox{S}(\alpha)$ of $\alpha$ in the field extension $K/\mathbb{Q}$ both are rational numbers and especially rational integers in the case $\alpha$ is an algebraic integer.\, If $\beta$ is another element of $K$, then 
\begin{align}
\mbox{N}(\alpha\beta) \;=\; \mbox{N}(\alpha)\mbox{N}(\beta), \quad
  \mbox{S}(\alpha\!+\!\beta) \;=\; \mbox{S}(\alpha)\!+\!\mbox{S}(\beta),
\end{align}
i.e. the norm is multiplicative and the trace additive.\, If\,  
$[K\!:\!\mathbb{Q}] = n$\, and\, $a\in\mathbb{Q}$, then
$$\mbox{N}(a) = a^n, \quad \mbox{S}(a) = na.$$
\end{thmplain}

\textbf{Remarks}

1.\, The notions norm and trace were originally introduced in German \PMlinkescapetext{language} as ``die Norm'' and ``die Spur''.\, Therefore in German and many other literature the symbol of trace is S, Sp or sp.\, Nowadays the symbols T and Tr are common. 

2.\, The norm and trace of an algebraic number $\alpha$ in the field extension\, $\mathbb{Q}(\alpha)/\mathbb{Q}$,\, i.e. the product and sum of all algebraic conjugates of $\alpha$, are called the {\em absolute norm} and the {\em absolute trace} of $\alpha$.\, Formulae like (1) concerning the absolute norms and traces are not sensible.\, 

\begin{thmplain}
\, An algebraic integer $\varepsilon$ is a unit if and only if 
        $$\mbox{N}(\varepsilon) \;=\; \pm 1,$$
i.e. iff the absolute norm of $\varepsilon$ is a rational unit.\, Thus \PMlinkescapetext{the constant term} in the minimal polynomial of an algebraic unit is always \,$\pm 1$.
\end{thmplain}

\textbf{Example.}\, The minimal polynomial of the number $2\!+\!\sqrt{3}$, which is the fundamental unit of the quadratic field $\mathbb{Q}(\sqrt{3})$, is \,$x^2\!-\!4x\!+\!1$.
%%%%%
%%%%%
\end{document}
