\documentclass[12pt]{article}
\usepackage{pmmeta}
\pmcanonicalname{EvenNumber}
\pmcreated{2013-03-22 13:56:29}
\pmmodified{2013-03-22 13:56:29}
\pmowner{mathcam}{2727}
\pmmodifier{mathcam}{2727}
\pmtitle{even number}
\pmrecord{10}{34703}
\pmprivacy{1}
\pmauthor{mathcam}{2727}
\pmtype{Definition}
\pmcomment{trigger rebuild}
\pmclassification{msc}{11-00}
\pmclassification{msc}{03-00}
\pmrelated{NumberOdd}
\pmdefines{odd number}
\pmdefines{even integer}
\pmdefines{odd integer}
\pmdefines{even}
\pmdefines{odd}

% this is the default PlanetMath preamble.  as your knowledge
% of TeX increases, you will probably want to edit this, but
% it should be fine as is for beginners.

% almost certainly you want these
\usepackage{amssymb}
\usepackage{amsmath}
\usepackage{amsfonts}

% used for TeXing text within eps files
%\usepackage{psfrag}
% need this for including graphics (\includegraphics)
%\usepackage{graphicx}
% for neatly defining theorems and propositions
%\usepackage{amsthm}
% making logically defined graphics
%%%\usepackage{xypic}

% there are many more packages, add them here as you need them

% define commands here

\newcommand{\sR}[0]{\mathbb{R}}
\newcommand{\sC}[0]{\mathbb{C}}
\newcommand{\sN}[0]{\mathbb{N}}
\newcommand{\sZ}[0]{\mathbb{Z}}
\begin{document}
{\bf Definition} Suppose $k$ is an integer. 
If there exists an integer $r$ such that $k=2r+1$, then $k$ is an {\bf odd number}. 
If there exists an integer $r$ such that $k=2r$, then $k$ is an {\bf even number}. 

The concept of even and odd numbers are most easily understood in 
the binary base. Then the above definition simply \PMlinkescapetext{states} that even numbers end
with a $0$, and odd numbers end with a $1$.

\subsubsection{Properties}
\begin{enumerate}
\item Every integer is either even or \PMlinkescapetext{odd}. This can be proven
using induction, or using the fundamental theorem of arithmetic.
\item An integer $k$ is even (\PMlinkescapetext{odd}) if and only if $k^2$ is even (\PMlinkescapetext{odd}).
\end{enumerate}
%%%%%
%%%%%
\end{document}
