\documentclass[12pt]{article}
\usepackage{pmmeta}
\pmcanonicalname{Ultrafactorial}
\pmcreated{2013-03-22 19:20:49}
\pmmodified{2013-03-22 19:20:49}
\pmowner{Kausthub}{26471}
\pmmodifier{Kausthub}{26471}
\pmtitle{Ultrafactorial}
\pmrecord{4}{42296}
\pmprivacy{1}
\pmauthor{Kausthub}{26471}
\pmtype{Definition}
\pmcomment{trigger rebuild}
\pmclassification{msc}{11A41}

\endmetadata

% this is the default PlanetMath preamble.  as your knowledge
% of TeX increases, you will probably want to edit this, but
% it should be fine as is for beginners.

% almost certainly you want these
\usepackage{amssymb}
\usepackage{amsmath}
\usepackage{amsfonts}

% used for TeXing text within eps files
%\usepackage{psfrag}
% need this for including graphics (\includegraphics)
%\usepackage{graphicx}
% for neatly defining theorems and propositions
%\usepackage{amsthm}
% making logically defined graphics
%%%\usepackage{xypic}

% there are many more packages, add them here as you need them

% define commands here

\begin{document}
Factorials raised to the power of themselves are called ultrafactorials. The first few ultrafactorials are 1, 1, 4, 46656 ... (Sequence A046882 of OEIS). According to the website cnki.com, functions like ultrafactorial and duplicate exponents are used to rectify the systematic errors in increasing load forecasting by linear regression. Hence they are helpful in forecasting annual electricity. Similarly ultraprimorials can also be defined as primorials raised to the power of themselves. The first few ultraprimorials 1, 4, 46656, ... (Sequence A165812 of OEIS).
%%%%%
%%%%%
\end{document}
