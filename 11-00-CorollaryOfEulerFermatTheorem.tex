\documentclass[12pt]{article}
\usepackage{pmmeta}
\pmcanonicalname{CorollaryOfEulerFermatTheorem}
\pmcreated{2013-03-22 14:23:14}
\pmmodified{2013-03-22 14:23:14}
\pmowner{kamala}{5486}
\pmmodifier{kamala}{5486}
\pmtitle{corollary of Euler-Fermat theorem}
\pmrecord{9}{35882}
\pmprivacy{1}
\pmauthor{kamala}{5486}
\pmtype{Result}
\pmcomment{trigger rebuild}
\pmclassification{msc}{11-00}

% this is the default PlanetMath preamble.  as your knowledge
% of TeX increases, you will probably want to edit this, but
% it should be fine as is for beginners.

% almost certainly you want these
\usepackage{amssymb}
\usepackage{amsmath}
\usepackage{amsfonts}

% used for TeXing text within eps files
%\usepackage{psfrag}
% need this for including graphics (\includegraphics)
%\usepackage{graphicx}
% for neatly defining theorems and propositions
%\usepackage{amsthm}
% making logically defined graphics
%%%\usepackage{xypic}

% there are many more packages, add them here as you need them

% define commands here
\begin{document}
Corollary of Euler-Fermat theorem (F. Smarandache):
\newline Let $a, m \in \mathbb{N}$, $m \neq 0$, and $\phi$ be the Euler totient function. Then:
$$a^{\phi(m_s)+s} \equiv a^s \pmod{m}$$
where $s$ and $m_s$ depend on $a$ and $m$, also $s$ is one more than the number of steps in the algorithm, while $m_s$ is a divisor of $m$, and they are both obtained from the following integer algorithm:

Step (0):
\newline calculate the gcd of $a$ and $m$ and denote it by $d_0$;
\newline therefore $d_0=(a,m)$, and also denote $m_0=m/d_0$;
\newline if $d_0 \neq 1$ go to the next step, otherwise stop;

Step (1):
\newline calculate the gcd of $d_0$ and $m_0$ and denote it by $d_1$;
\newline therefore $d_1=(d_0,m_0)$, and also denote $m_1=m_0/d_1$;
\newline if $d_1 \neq 1$ go to the next step, otherwise stop;

$ \ldots \ldots \ldots \ldots \ldots \ldots \ldots $

Step (s-1):
\newline calculate the gcd of $d_{s-2}$ and $m_{s-2}$ and denote it by $d_{s-1}$;
\newline therefore $d_{s-1}=(d_{s-2},m_{s-2})$, and also denote $m_{s-1}=m_{s-2}/d_{s-1}$;
\newline if $d_{s-1} \neq 1$ go to the next step, otherwise stop;

Step (s):
\newline calculate the gcd of $d_{s-1}$ and $m_{s-1}$ and denote it by $d_{s}$;
\newline therefore $d_{s}=(d_{s-1},m_{s-1})$, and also denote $m_{s}=m_{s-1}/d_{s}$;
\newline eventually one arrives at a gcd $d_{s} = 1$, stop the algorithm.

The algorithm ends when the gcd=1.  Actually at each step the gcd decreases: from the maximum gcd=(a,m) at step (0) to the minimum gcd=1 at step (s).  The algorithm is finite because the first gcd of (a,m) is finite and at each step one gets a smaller gcd.

For the particular case when $(a,m)=1$ one has $s=0$ (hence the algorithm finishes at step (0)) and $m_s=m$, which is Euler-Fermat theorem.

\begin{thebibliography}{9}
\bibitem{smarandache} Florentin Smarandache, {\em A Generalization of Euler Theorem}, Bulet. Univ. Brasov, Series C, Vol. XXIII, 7-12, 1981;
\PMlinkexternal{online article in arXiv}{http://xxx.lanl.gov/pdf/math.GM/0610607}.
\bibitem{smarandache2} Florentin Smarandache, {\em Collected Papers}, Vol. I, 184-191 (in French), Tempus, Bucharest, 1996; 
\PMlinkexternal{online book}{http://www.gallup.unm.edu/~smarandache/CP1.pdf}.
\end{thebibliography}

%%%%%
%%%%%
\end{document}
