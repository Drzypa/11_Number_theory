\documentclass[12pt]{article}
\usepackage{pmmeta}
\pmcanonicalname{RestrictedDirectProduct}
\pmcreated{2013-03-22 12:35:38}
\pmmodified{2013-03-22 12:35:38}
\pmowner{djao}{24}
\pmmodifier{djao}{24}
\pmtitle{restricted direct product}
\pmrecord{5}{32845}
\pmprivacy{1}
\pmauthor{djao}{24}
\pmtype{Definition}
\pmcomment{trigger rebuild}
\pmclassification{msc}{11R56}
\pmclassification{msc}{22D05}

\endmetadata

% this is the default PlanetMath preamble.  as your knowledge
% of TeX increases, you will probably want to edit this, but
% it should be fine as is for beginners.

% almost certainly you want these
\usepackage{amssymb}
\usepackage{amsmath}
\usepackage{amsfonts}

% used for TeXing text within eps files
%\usepackage{psfrag}
% need this for including graphics (\includegraphics)
%\usepackage{graphicx}
% for neatly defining theorems and propositions
%\usepackage{amsthm}
% making logically defined graphics
%%%\usepackage{xypic} 

% there are many more packages, add them here as you need them

% define commands here
\begin{document}
Let $\{G_v\}_{v \in V}$ be a collection of locally compact topological groups. For all but finitely many $v \in V$, let $H_v \subset G_v$ be a compact open subgroup of $G_v$. The {\em restricted direct product} of the collection $\{G_v\}$ with respect to the collection $\{H_v\}$ is the subgroup
$$
G := \left\{ \left. (g_v)_{v \in V} \in \prod_{v \in V} G_v\ \right| \ g_v \in H_v \text{  for all but finitely many $v \in V$} \right\}
$$
of the direct product $\prod_{v \in V} G_v$.

We define a topology on $G$ as follows. For every finite subset $S \subset V$ that contains all the elements $v$ for which $H_v$ is undefined, form the topological group
$$
G_S := \prod_{v \in S} G_v \times \prod_{v \notin S} H_v
$$
consisting of the direct product of the $G_v$'s, for $v \in S$, and the $H_v$'s, for $v \notin S$. The topological group $G_S$ is a subset of $G$ for each such $S$, and we take for a topology on $G$ the weakest topology such that the $G_S$ are open subsets of $G$, with the subspace topology on each $G_S$ equal to the topology that $G_S$ already has in its own right.
%%%%%
%%%%%
\end{document}
