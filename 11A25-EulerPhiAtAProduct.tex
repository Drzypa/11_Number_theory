\documentclass[12pt]{article}
\usepackage{pmmeta}
\pmcanonicalname{EulerPhiAtAProduct}
\pmcreated{2014-02-18 14:02:24}
\pmmodified{2014-02-18 14:02:24}
\pmowner{pahio}{2872}
\pmmodifier{pahio}{2872}
\pmtitle{Euler phi at a product}
\pmrecord{6}{41779}
\pmprivacy{1}
\pmauthor{pahio}{2872}
\pmtype{Theorem}
\pmcomment{trigger rebuild}
\pmclassification{msc}{11A25}
\pmclassification{msc}{11-00}
\pmrelated{EulerPhifunction}
\pmrelated{DivisibilityByPrimeNumber}

% this is the default PlanetMath preamble.  as your knowledge
% of TeX increases, you will probably want to edit this, but
% it should be fine as is for beginners.

% almost certainly you want these
\usepackage{amssymb}
\usepackage{amsmath}
\usepackage{amsfonts}

% used for TeXing text within eps files
%\usepackage{psfrag}
% need this for including graphics (\includegraphics)
%\usepackage{graphicx}
% for neatly defining theorems and propositions
 \usepackage{amsthm}
% making logically defined graphics
%%%\usepackage{xypic}

% there are many more packages, add them here as you need them

% define commands here

\theoremstyle{definition}
\newtheorem*{thmplain}{Theorem}

\begin{document}
If the positive greatest common divisor of the integers $a$ and 
$b$ is $d$, then
$$\varphi(ab) \;=\; \frac{\varphi(a)\,\varphi(b)\,d}{\varphi(d)}.$$\\


\emph{Proof.}\, Using the positive prime factors $p$, the right hand side of the asserted equation is 
\begin{align*}
\frac{d\cdot a\prod_{p\mid a}\frac{p-1}{p}\cdot b\prod_{p\mid b}\frac{p-1}{p}}{d\prod_{p\mid a,\,p\mid b}\frac{p-1}{p}}
& \;=\; 
\frac{ab\prod_{p\mid a,\,p\nmid b}\frac{p-1}{p}\cdot\prod_{p\mid a,\,p\mid b}\frac{p-1}{p}\cdot\prod_{p\mid b,\,p\nmid a}\frac{p-1}{p}\cdot\prod_{p\mid b,\,p\mid a}\frac{p-1}{p}}{\prod_{p\mid a,\,p\mid b}\frac{p-1}{p}}\\
& \;=\; ab\prod_{p \mid a\;\lor\;p \mid b}\frac{p\!-\!1}{p} \;=\; ab\prod_{p \mid ab}\frac{p\!-\!1}{p}
\;=\; \varphi(ab),
\end{align*}
Q.E.D.
%%%%%
%%%%%
\end{document}
