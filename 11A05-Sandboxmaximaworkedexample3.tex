\documentclass[12pt]{article}
\usepackage{pmmeta}
\pmcanonicalname{Sandboxmaximaworkedexample3}
\pmcreated{2014-12-30 17:48:15}
\pmmodified{2014-12-30 17:48:15}
\pmowner{robert_dodier}{1000903}
\pmmodifier{robert_dodier}{1000903}
\pmtitle{sandbox-maxima-worked-example-3}
\pmrecord{1}{}
\pmprivacy{1}
\pmauthor{robert_dodier}{1000903}
\pmtype{Example}

\endmetadata

% this is the default PlanetMath preamble.  as your knowledge
% of TeX increases, you will probably want to edit this, but
% it should be fine as is for beginners.

% almost certainly you want these
\usepackage{amssymb}
\usepackage{amsmath}
\usepackage{amsfonts}

% need this for including graphics (\includegraphics)
\usepackage{graphicx}
% for neatly defining theorems and propositions
\usepackage{amsthm}

% making logically defined graphics
%\usepackage{xypic}
% used for TeXing text within eps files
%\usepackage{psfrag}

% there are many more packages, add them here as you need them

% define commands here

\begin{document}
Before we get started, load ezunits.
\begin{verbatim}
load (ezunits);
\end{verbatim}
$$
 \mbox{{}/usr/local/share/maxima/branch\_5\_32\_base\_2\_g0e99686/share/ezunits/ezunits.mac{}}$$
\section{Units in integrals and derivatives
}
\subsection{Integral
}
Units for integrals and derivatives are derived as the same units that a finite approximation would yield.
Turn off simplification for a moment so that we can have a look at the unsimplified form.
\begin{verbatim}
simp : false $
foo : 'integrate ((3 * t^2) ` m/s, t ` s, 0 ` s, t[1] ` s);
\end{verbatim}
$$\int_{0\;\mathrm{s}}^{t_{1}\;\mathrm{s}}{3\,t^2\;{{\mathrm{m}
 }\over{\mathrm{s}}}\;dt\;\mathrm{s}}$$
Now go ahead and simplify it.
\begin{verbatim}
simp : true $
foo : ''foo;
\end{verbatim}
$$3\,\left(\int_{0}^{t_{1}}{t^2\;dt}\right)\;\mathrm{m}$$
\begin{verbatim}
%, nouns;
\end{verbatim}
$$t_{1}^3\;\mathrm{m}$$
\subsection{Derivative
}
Again, have a look at the unsimplified form.
\begin{verbatim}
simp : false $
foo : 'diff (sqrt (t) ` inch, t ` year, 1);
\end{verbatim}
$${{d+^{1}}\over{d\,`^1\left(t , {\it year}\right)}}\,\left(\sqrt{t}
 \;\mathrm{inch}\right)$$
Go ahead and simplify it.
\begin{verbatim}
simp : true $
foo : ''foo;
\end{verbatim}
$${{d}\over{d\,t}}\,\sqrt{t}\;{{\mathrm{inch}}\over{\mathrm{year}}}$$
\begin{verbatim}
foo, nouns;
\end{verbatim}
$${{1}\over{2\,\sqrt{t}}}\;{{\mathrm{inch}}\over{\mathrm{year}}}$$
\section{Units in solving equations
}
`solve_dimensional' can solve equations in which unit-ful expressions contain unknowns.
\begin{verbatim}
foo : solve_dimensional (10^6 ` Btu = 1/2 * (10 ` lbm) * v^2, v);
\end{verbatim}
$$\left[ v=\left(-8\,5^{{{5}\over{2}}}\right)\;{{\sqrt{\mathrm{Btu}}
 }\over{\sqrt{\mathrm{lbm}}}} , v=8\,5^{{{5}\over{2}}}\;{{\sqrt{
 \mathrm{Btu}}}\over{\sqrt{\mathrm{lbm}}}} \right] $$
\begin{verbatim}
dimensions (foo);
\end{verbatim}
$$\left[ {\it dimensions}\left(v\right)={{{\it length}}\over{
 {\it time}}} , {\it dimensions}\left(v\right)={{{\it length}}\over{
 {\it time}}} \right] $$
\begin{verbatim}
foo1 : foo `` m/s;
\end{verbatim}
$$\left[ {\it ``}\left(v , {{m}\over{s}}\right)=\left(-{{128\,5^{{{13
 }\over{2}}}\,\sqrt{1055}}\over{\sqrt{45359237}}}\right)\;{{
 \mathrm{m}}\over{\mathrm{s}}} , {\it ``}\left(v , {{m}\over{s}}
 \right)={{128\,5^{{{13}\over{2}}}\,\sqrt{1055}}\over{\sqrt{45359237}
 }}\;{{\mathrm{m}}\over{\mathrm{s}}} \right] $$
\begin{verbatim}
float (foo1);
\end{verbatim}
$$\left[ {\it ``}\left(v , {{m}\over{s}}\right)=\left(-
 21567.92463845541\right)\;{{\mathrm{m}}\over{\mathrm{s}}} , {\it ``}
 \left(v , {{m}\over{s}}\right)=21567.92463845541\;{{\mathrm{m}
 }\over{\mathrm{s}}} \right] $$
\section{Selecting a named field from a unit-ful expression
}
`@' distributes over unit-ful expressions.
\begin{verbatim}
defstruct (my3d (x, y, z));
\end{verbatim}
$$\left[ {\it my3d}\left(x , y , z\right) \right] $$
\begin{verbatim}
foo : my3d (17, 29, 42) ` m;
\end{verbatim}
$${\it my3d}\left(17 , 29 , 42\right)\;\mathrm{m}$$
\begin{verbatim}
foo @ z;
\end{verbatim}
$$@\left({\it my3d}\left(17 , 29 , 42\right) , z\right)\;\mathrm{m}$$
Hmm, looks like it needs a little encouragement.
\begin{verbatim}
''%;
\end{verbatim}
$$42\;\mathrm{m}$$
\section{Real part and imaginary part of a unit-ful expression
}
Simplifications for unit-ful expressions apply only to noun forms of `realpart' and `imagpart'.
Oh well.
\begin{verbatim}
declare (a, complex) $
realpart (a ` m);
\end{verbatim}
$${\it realpart}\left(a\;\mathrm{m}\right)$$
\begin{verbatim}
''%;
\end{verbatim}
$${\it realpart}\left(a\right)\;\mathrm{m}$$
\begin{verbatim}
imagpart (a ` m);
\end{verbatim}
$${\it imagpart}\left(a\;\mathrm{m}\right)$$
\begin{verbatim}
''%;
\end{verbatim}
$${\it imagpart}\left(a\right)\;\mathrm{m}$$
\section{Splitting off units of varying magnitudes
}
This one is for converting to units which come in conventional magnitude order,
such as hours, minutes, and seconds, or miles, feet, and inches.
When the units are specified as a list,
the unit conversion operator will split off the first,
then the second from the remainder of the first,
the third from the remainder, etc.
\begin{verbatim}
10 ` km `` mile;
\end{verbatim}
$${{78125}\over{12573}}\;\mathrm{mile}$$
\begin{verbatim}
10 ` km `` [mile, yard, foot, inch];
\end{verbatim}
$$\left[ 6\;\mathrm{mile} , 376\;\mathrm{yard} , 0\;\mathrm{foot} , 
 {{608}\over{127}}\;\mathrm{inch} \right] $$
\begin{verbatim}
apply ("+", %) `` km;
\end{verbatim}
$$10\;\mathrm{km}$$

\end{document}
