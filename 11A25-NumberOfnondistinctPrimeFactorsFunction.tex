\documentclass[12pt]{article}
\usepackage{pmmeta}
\pmcanonicalname{NumberOfnondistinctPrimeFactorsFunction}
\pmcreated{2013-03-22 16:07:00}
\pmmodified{2013-03-22 16:07:00}
\pmowner{Wkbj79}{1863}
\pmmodifier{Wkbj79}{1863}
\pmtitle{number of (nondistinct) prime factors function}
\pmrecord{16}{38183}
\pmprivacy{1}
\pmauthor{Wkbj79}{1863}
\pmtype{Definition}
\pmcomment{trigger rebuild}
\pmclassification{msc}{11A25}
\pmrelated{NumberOfDistinctPrimeFactorsFunction}
\pmrelated{2omeganLeTaunLe2Omegan}

% this is the default PlanetMath preamble.  as your knowledge
% of TeX increases, you will probably want to edit this, but
% it should be fine as is for beginners.

% almost certainly you want these
\usepackage{amssymb}
\usepackage{amsmath}
\usepackage{amsfonts}

% used for TeXing text within eps files
%\usepackage{psfrag}
% need this for including graphics (\includegraphics)
%\usepackage{graphicx}
% for neatly defining theorems and propositions
%\usepackage{amsthm}
% making logically defined graphics
%%%\usepackage{xypic}

% there are many more packages, add them here as you need them

% define commands here

\begin{document}
The \PMlinkescapetext{{\em number of (nondistinct) prime factors function}} $\Omega(n)$ counts with repetition how many prime factors a natural number $n$ has.  If $\displaystyle n= \prod_{j= 1}^k {p_j}^{a_j}$ where the $k$ primes $p_j$ are distinct and the $a_j$ are natural numbers, then $\displaystyle \Omega(n)=\sum_{j=1}^k a_j$.

Note that, if $n$ is a squarefree number, then $\omega(n)=\Omega(n)$, where $\omega(n)$ is the number of distinct prime factors function.  Otherwise, $\omega(n)<\Omega(n)$.

Note also that $\Omega(n)$ is a completely additive function and thus can be exponentiated to define a completely multiplicative function.  For example, the Liouville function can be defined as $\lambda(n) = (-1)^{\Omega(n)}$.

The sequence $\{\Omega(n)\}$ appears in the OEIS as sequence \PMlinkexternal{A001222}{http://www.research.att.com/~njas/sequences/?q=A001222}.

The sequence $\{2^{\Omega(n)}\}$ appears in the \PMlinkname{OEIS}{OEIS} as sequence \PMlinkexternal{A061142}{http://www.research.att.com/~njas/sequences/?q=A061142}.
%%%%%
%%%%%
\end{document}
