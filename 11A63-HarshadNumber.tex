\documentclass[12pt]{article}
\usepackage{pmmeta}
\pmcanonicalname{HarshadNumber}
\pmcreated{2013-03-22 15:47:04}
\pmmodified{2013-03-22 15:47:04}
\pmowner{PrimeFan}{13766}
\pmmodifier{PrimeFan}{13766}
\pmtitle{Harshad number}
\pmrecord{5}{37742}
\pmprivacy{1}
\pmauthor{PrimeFan}{13766}
\pmtype{Definition}
\pmcomment{trigger rebuild}
\pmclassification{msc}{11A63}
\pmsynonym{Niven number}{HarshadNumber}
\pmdefines{Harshad number}

% this is the default PlanetMath preamble.  as your knowledge
% of TeX increases, you will probably want to edit this, but
% it should be fine as is for beginners.

% almost certainly you want these
\usepackage{amssymb}
\usepackage{amsmath}
\usepackage{amsfonts}

% used for TeXing text within eps files
%\usepackage{psfrag}
% need this for including graphics (\includegraphics)
%\usepackage{graphicx}
% for neatly defining theorems and propositions
%\usepackage{amsthm}
% making logically defined graphics
%%%\usepackage{xypic}

% there are many more packages, add them here as you need them

% define commands here
\begin{document}
Consider the integer 1729. Adding up its digits, $$1 + 7 + 2 + 9 = 19$$ and $${{1729} \over {19}} = 91.$$

When an integer is divisible by the sum of its digits, it's called a {\em Harshad number} or {\em Niven number}. That is, given m is the number of digits of n and d is an integer of n,

$${\sum_{i = 1}^m d_i}|n$$

All 1-digit numbers and the base number itself are Harshad numbers. 1, 2, 4 and 6 are always Harshad numbers regardless of the base.

It is possible for an integer to be divisible by its digital root and yet not be a Harshad number because it doesn't divide its first digit sum evenly (for example, 38 in base 10 has digital root 2 but is not divisible by 3 + 8 = 11). The reverse is also possible (for example, 195 is divisible by 1 + 9 + 5 = 15, but not by its digital root 4).
%%%%%
%%%%%
\end{document}
