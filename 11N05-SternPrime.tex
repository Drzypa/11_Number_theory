\documentclass[12pt]{article}
\usepackage{pmmeta}
\pmcanonicalname{SternPrime}
\pmcreated{2013-03-22 16:19:10}
\pmmodified{2013-03-22 16:19:10}
\pmowner{PrimeFan}{13766}
\pmmodifier{PrimeFan}{13766}
\pmtitle{Stern prime}
\pmrecord{4}{38446}
\pmprivacy{1}
\pmauthor{PrimeFan}{13766}
\pmtype{Definition}
\pmcomment{trigger rebuild}
\pmclassification{msc}{11N05}

\endmetadata

% this is the default PlanetMath preamble.  as your knowledge
% of TeX increases, you will probably want to edit this, but
% it should be fine as is for beginners.

% almost certainly you want these
\usepackage{amssymb}
\usepackage{amsmath}
\usepackage{amsfonts}

% used for TeXing text within eps files
%\usepackage{psfrag}
% need this for including graphics (\includegraphics)
%\usepackage{graphicx}
% for neatly defining theorems and propositions
%\usepackage{amsthm}
% making logically defined graphics
%%%\usepackage{xypic}

% there are many more packages, add them here as you need them

% define commands here

\begin{document}
If for a given prime number $q$ there is no smaller prime $p$ and nonzero integer $b$ such that $q  = 2b^2 + p$, then $q$ is a \emph{Stern prime}. These primes were first studied by Moritz Abraham Stern, in connection to a lesser known conjecture of Goldbach's. Like other mathematicians of the time, Stern considered 1 to be a prime number. Thus his list of Stern primes read thus: 2, 17, 137, 227, 977, 1187, 1493. A century later the list has been amended to include 3 (as in A042978 of Sloane's OEIS) but no terms larger than 1493 have been found. The larger of a twin prime is not a Stern prime.
%%%%%
%%%%%
\end{document}
