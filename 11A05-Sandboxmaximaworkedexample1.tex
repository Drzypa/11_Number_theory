\documentclass[12pt]{article}
\usepackage{pmmeta}
\pmcanonicalname{Sandboxmaximaworkedexample1}
\pmcreated{2014-12-31 17:36:26}
\pmmodified{2014-12-31 17:36:26}
\pmowner{robert_dodier}{1000903}
\pmmodifier{robert_dodier}{1000903}
\pmtitle{sandbox-maxima-worked-example-1}
\pmrecord{2}{88010}
\pmprivacy{1}
\pmauthor{robert_dodier}{1000903}
\pmtype{Example}

\endmetadata

% this is the default PlanetMath preamble.  as your knowledge
% of TeX increases, you will probably want to edit this, but
% it should be fine as is for beginners.

% almost certainly you want these
\usepackage{amssymb}
\usepackage{amsmath}
\usepackage{amsfonts}

% need this for including graphics (\includegraphics)
\usepackage{graphicx}
% for neatly defining theorems and propositions
\usepackage{amsthm}

% making logically defined graphics
%\usepackage{xypic}
% used for TeXing text within eps files
%\usepackage{psfrag}

% there are many more packages, add them here as you need them

% define commands here

\begin{document}
Here is some text which opens the discussion.
I'll find out stuff later.
Third line in 1st part.
More text here. I wonder if it makes a separate paragraph.
2nd line in 2nd part.
First formula. Something about random blurfage.
2nd line in 3rd part.
\begin{verbatim}
eqn : x^2 + 17*x + 29 = 0;
\end{verbatim}
$$x^2+17\,x+29=0$$
\begin{verbatim}
solve (eqn, x);
\end{verbatim}
$$\left[ x=-{{\sqrt{173}+17}\over{2}} , x={{\sqrt{173}-17}\over{2}}
  \right] $$
\begin{verbatim}
subst (first (%), eqn);
\end{verbatim}
$${{\left(\sqrt{173}+17\right)^2}\over{4}}-{{17\,\left(\sqrt{173}+17
 \right)}\over{2}}+29=0$$
\begin{verbatim}
expand (%);
\end{verbatim}
$$0=0$$
More text here. I wonder if percent sign stuff works.
\begin{verbatim}
abc : '(def + xyz);
\end{verbatim}
$${\it xyz}+{\it def}$$
$${\mathit xyz}+{\mathit def}$$
$$xyz + def$$
\begin{verbatim}
bcd : '(sin(x)+sin(x))$
\end{verbatim}
Dollar sign terminator on previous line. Hmm.
\begin{verbatim}
ev (bcd);
\end{verbatim}
$$2\,\sin x$$
Now bcd is simplified.
\begin{verbatim}
print(bcd);
\end{verbatim}
$$2\,\sin x$$
bcd was simplified by printing too.
All finished.
\includegraphics{foo1_0.png}
No, really!

\end{document}
