\documentclass[12pt]{article}
\usepackage{pmmeta}
\pmcanonicalname{LeylandNumber}
\pmcreated{2013-03-22 15:50:01}
\pmmodified{2013-03-22 15:50:01}
\pmowner{CompositeFan}{12809}
\pmmodifier{CompositeFan}{12809}
\pmtitle{Leyland number}
\pmrecord{5}{37806}
\pmprivacy{1}
\pmauthor{CompositeFan}{12809}
\pmtype{Definition}
\pmcomment{trigger rebuild}
\pmclassification{msc}{11A63}

\endmetadata

% this is the default PlanetMath preamble.  as your knowledge
% of TeX increases, you will probably want to edit this, but
% it should be fine as is for beginners.

% almost certainly you want these
\usepackage{amssymb}
\usepackage{amsmath}
\usepackage{amsfonts}

% used for TeXing text within eps files
%\usepackage{psfrag}
% need this for including graphics (\includegraphics)
%\usepackage{graphicx}
% for neatly defining theorems and propositions
%\usepackage{amsthm}
% making logically defined graphics
%%%\usepackage{xypic}

% there are many more packages, add them here as you need them

% define commands here
\begin{document}
An integer of the form $x^y + y^x$, with $1 < x \leq y$.

Leyland numbers with $x < y < 11$ are

8, 17,  32,   57,   100,     177,      320,       593,        1124
   54, 145,  368,   945,    2530,     7073,     20412,       60049
       512, 1649,  5392,   18785,    69632,    268705,     1058576
            6250, 23401,   94932,   423393,   2012174,     9865625
                  93312,  397585,  1941760,  10609137,    61466176
                         1647086,  7861953,  45136576,   292475249
                                  33554432, 177264449,  1173741824
                                            774840978,  4486784401
                                                       20000000000

The largest known prime Leyland number $2638^{4405} + 4405^{2638}$.

Reference

R. Crandall, C. Pomerance, Prime Numbers: A Computational Perspective, Springer, 2001
%%%%%
%%%%%
\end{document}
