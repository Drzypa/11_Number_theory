\documentclass[12pt]{article}
\usepackage{pmmeta}
\pmcanonicalname{SierpinskiNumber}
\pmcreated{2013-03-22 13:55:33}
\pmmodified{2013-03-22 13:55:33}
\pmowner{CWoo}{3771}
\pmmodifier{CWoo}{3771}
\pmtitle{Sierpinski number}
\pmrecord{7}{34683}
\pmprivacy{1}
\pmauthor{CWoo}{3771}
\pmtype{Definition}
\pmcomment{trigger rebuild}
\pmclassification{msc}{11B83}
\pmdefines{Riesel number}
\pmdefines{Sierpi\'nski number}

\endmetadata

% this is the default PlanetMath preamble.  as your knowledge
% of TeX increases, you will probably want to edit this, but
% it should be fine as is for beginners.

% almost certainly you want these
\usepackage{amssymb}
\usepackage{amsmath}
\usepackage{amsfonts}
\usepackage{amsthm}
\usepackage{url}

% used for TeXing text within eps files
%\usepackage{psfrag}
% need this for including graphics (\includegraphics)
%\usepackage{graphicx}
% for neatly defining theorems and propositions
%\usepackage{amsthm}
% making logically defined graphics
%%%\usepackage{xypic}

% there are many more packages, add them here as you need them

% define commands here

\newcommand{\mc}{\mathcal}
\newcommand{\mb}{\mathbb}
\newcommand{\mf}{\mathfrak}
\newcommand{\ol}{\overline}
\newcommand{\ra}{\rightarrow}
\newcommand{\la}{\leftarrow}
\newcommand{\La}{\Leftarrow}
\newcommand{\Ra}{\Rightarrow}
\newcommand{\nor}{\vartriangleleft}
\newcommand{\Gal}{\text{Gal}}
\newcommand{\GL}{\text{GL}}
\newcommand{\Z}{\mb{Z}}
\newcommand{\R}{\mb{R}}
\newcommand{\Q}{\mb{Q}}
\newcommand{\C}{\mb{C}}
\newcommand{\<}{\langle}
\renewcommand{\>}{\rangle}
\begin{document}
An integer $k$ is a \emph{Sierpinski number} if for every positive integer $n$, the number $k2^n+1$ is composite.

That such numbers exist is amazing, and even more surprising is that there are infinitely many of them (in fact, infinitely many odd ones).  The smallest \emph{known} Sierpinski number is 78557, but it is not known whether or not this is the smallest one.  The smallest number $m$ for which it 
is unknown whether or not $m$ is a Sierpinski number is 10223.

A process for generating Sierpinski numbers using covering sets of primes can be found at

\url{http://www.glasgowg43.freeserve.co.uk/siercvr.htm}

Visit

\url{http://www.seventeenorbust.com/}

for the distributed computing effort to show that 78557 is indeed the smallest Sierpinski number (or find a smaller one).

Similarly, a \emph{Riesel number} is a number $k$ such that for every positive integer $n$, the number $k2^n-1$ is composite.  The smallest known Riesel number is 509203, but again, it is not known for sure that this is the smallest.
%%%%%
%%%%%
\end{document}
