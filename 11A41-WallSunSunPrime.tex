\documentclass[12pt]{article}
\usepackage{pmmeta}
\pmcanonicalname{WallSunSunPrime}
\pmcreated{2013-03-22 18:04:18}
\pmmodified{2013-03-22 18:04:18}
\pmowner{PrimeFan}{13766}
\pmmodifier{PrimeFan}{13766}
\pmtitle{Wall-Sun-Sun prime}
\pmrecord{4}{40605}
\pmprivacy{1}
\pmauthor{PrimeFan}{13766}
\pmtype{Definition}
\pmcomment{trigger rebuild}
\pmclassification{msc}{11A41}
\pmsynonym{Fibonacci Wieferich prime}{WallSunSunPrime}

% this is the default PlanetMath preamble.  as your knowledge
% of TeX increases, you will probably want to edit this, but
% it should be fine as is for beginners.

% almost certainly you want these
\usepackage{amssymb}
\usepackage{amsmath}
\usepackage{amsfonts}

% used for TeXing text within eps files
%\usepackage{psfrag}
% need this for including graphics (\includegraphics)
%\usepackage{graphicx}
% for neatly defining theorems and propositions
%\usepackage{amsthm}
% making logically defined graphics
%%%\usepackage{xypic}

% there are many more packages, add them here as you need them

% define commands here

\begin{document}
A {\em Wall-Sun-Sun prime} is a prime number $p > 5$ such that $p^2 | F_{p - \left(\frac{p}{5}\right)}$, with $F_n$ being the $n$th Fibonacci number and $\left(\frac{p}{5}\right)$ being a Legendre symbol. The prime $p$ always divides $F_{p - \left(\frac{p}{5}\right)}$, but no case is known for the square of a prime $p^2$ also dividing that.

The search for these primes started in the 1990s as Donald Dines Wall, Zhi-Hong Sun and Zhi-Wei Sun searched for counterexamples to Fermat's last theorem. But Andrew Wiles's proof does not rule out the existence of these primes: if Fermat's last theorem was false and there existed a prime exponent $p$ such that $x^p + y^p = z^p$, the square of such a prime would also divide $F_{p - \left(\frac{p}{5}\right)}$, but with Fermat's last theorem being true, the existence of a Wall-Sun-Sun prime would not present a contradiction.

As of 2005, the lower bound was $3.2 \times 10^{12}$, given by McIntosh.

\begin{thebibliography}{1}
\bibitem{rccp} Richard Crandall \& Carl Pomerance, {\it Prime Numbers: A Computational Perspective}, 2nd Edition. New York: Springer (2005): 32
\end{thebibliography}

%%%%%
%%%%%
\end{document}
