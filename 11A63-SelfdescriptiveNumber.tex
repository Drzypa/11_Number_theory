\documentclass[12pt]{article}
\usepackage{pmmeta}
\pmcanonicalname{SelfdescriptiveNumber}
\pmcreated{2013-03-22 15:53:27}
\pmmodified{2013-03-22 15:53:27}
\pmowner{PrimeFan}{13766}
\pmmodifier{PrimeFan}{13766}
\pmtitle{self-descriptive number}
\pmrecord{9}{37892}
\pmprivacy{1}
\pmauthor{PrimeFan}{13766}
\pmtype{Definition}
\pmcomment{trigger rebuild}
\pmclassification{msc}{11A63}
\pmsynonym{self descriptive number}{SelfdescriptiveNumber}

\endmetadata

% this is the default PlanetMath preamble.  as your knowledge
% of TeX increases, you will probably want to edit this, but
% it should be fine as is for beginners.

% almost certainly you want these
\usepackage{amssymb}
\usepackage{amsmath}
\usepackage{amsfonts}

% used for TeXing text within eps files
%\usepackage{psfrag}
% need this for including graphics (\includegraphics)
%\usepackage{graphicx}
% for neatly defining theorems and propositions
%\usepackage{amsthm}
% making logically defined graphics
%%%\usepackage{xypic}

% there are many more packages, add them here as you need them

% define commands here

\begin{document}
A {\em self-descriptive number} $n$ in base $b$ is an integer such that each base $b$ digit $$d_x = \sum_{d_i = x} 1$$ where each $d_i$ is a digit of $n$, $i$ is a very simple, standard iterator operating in the range $-1 < i < b$, and $x$ is a position of a digit; thus $n$ ``describes'' itself.

For example, the integer 6210001000 written in base 10. It has six instances of the digit 0, two instances of the digit 1, a single instance of the digit 2, a single instance of the digit 6 and no instances of any other base 10 digits. 

Base 4 might be the only base with two self-descriptive numbers, $1210_4$ and $2020_4$. From base 7 onwards, every base $b$ has at least one self-descriptive number of the form $(b - 4)^{b - 1} + 2b^{b - 2} + b^{b - 3} + b^4$. It has been proven that 6210001000 is the only self-descriptive number in base 10, but it's not known if any higher bases have any self-descriptive numbers of any other form.

Sequence A108551 of the OEIS lists self-descriptive numbers from quartal to hexadecimal.
%%%%%
%%%%%
\end{document}
