\documentclass[12pt]{article}
\usepackage{pmmeta}
\pmcanonicalname{PellNumber}
\pmcreated{2013-03-22 15:46:44}
\pmmodified{2013-03-22 15:46:44}
\pmowner{CompositeFan}{12809}
\pmmodifier{CompositeFan}{12809}
\pmtitle{Pell number}
\pmrecord{6}{37736}
\pmprivacy{1}
\pmauthor{CompositeFan}{12809}
\pmtype{Definition}
\pmcomment{trigger rebuild}
\pmclassification{msc}{11B39}
\pmdefines{Pell number}

% this is the default PlanetMath preamble.  as your knowledge
% of TeX increases, you will probably want to edit this, but
% it should be fine as is for beginners.

% almost certainly you want these
\usepackage{amssymb}
\usepackage{amsmath}
\usepackage{amsfonts}

% used for TeXing text within eps files
%\usepackage{psfrag}
% need this for including graphics (\includegraphics)
%\usepackage{graphicx}
% for neatly defining theorems and propositions
%\usepackage{amsthm}
% making logically defined graphics
%%%\usepackage{xypic}

% there are many more packages, add them here as you need them

% define commands here
\begin{document}
A number in the sequence created from the recurrence relation $$P_n = 2P_{n - 1} + P_{n - 2},$$ with $$P_0 = 0$$ and $$P_1 = 1.$$ The first few Pell numbers are 0, 1, 2, 5, 12, 29, 70, 169, 408, 985, listed in A000129 of Sloane's OEIS.

A Pell number for any given index can also be calculated from earlier Pell numbers with $$P_{a + b} = P_aP_{b + 1} + P_{a - 1}P_b.$$

The formula $${ {-{(1 - \sqrt 2)}^n + {(1 + \sqrt 2)}^n} \over {2\sqrt 2}}$$ works too. From this particular formula it can be deduced that the sequence of Pell numbers can be used in a continued fraction of the square root of 2 as well as the silver ratio.

Yet another way to calculate Pell numbers is by squaring the terms of Pascal's triangle and adding up the antidiagonals. Arranging the Markov numbers in a binary graph tree and reading the numbers on 2's branch gives the Pell numbers with odd indices.

Only Pell numbers with prime indexes can also be prime. This fact is used in some tests for pseudoprimality.
%%%%%
%%%%%
\end{document}
