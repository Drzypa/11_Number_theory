\documentclass[12pt]{article}
\usepackage{pmmeta}
\pmcanonicalname{ArmstrongNumber}
\pmcreated{2013-03-22 16:04:06}
\pmmodified{2013-03-22 16:04:06}
\pmowner{CompositeFan}{12809}
\pmmodifier{CompositeFan}{12809}
\pmtitle{Armstrong number}
\pmrecord{4}{38123}
\pmprivacy{1}
\pmauthor{CompositeFan}{12809}
\pmtype{Definition}
\pmcomment{trigger rebuild}
\pmclassification{msc}{11A63}
\pmsynonym{narcissistic number}{ArmstrongNumber}
\pmsynonym{plus perfect number}{ArmstrongNumber}
\pmsynonym{perfect digital invariant}{ArmstrongNumber}

% this is the default PlanetMath preamble.  as your knowledge
% of TeX increases, you will probably want to edit this, but
% it should be fine as is for beginners.

% almost certainly you want these
\usepackage{amssymb}
\usepackage{amsmath}
\usepackage{amsfonts}

% used for TeXing text within eps files
%\usepackage{psfrag}
% need this for including graphics (\includegraphics)
%\usepackage{graphicx}
% for neatly defining theorems and propositions
%\usepackage{amsthm}
% making logically defined graphics
%%%\usepackage{xypic}

% there are many more packages, add them here as you need them

% define commands here

\begin{document}
Given a base $b$ integer $$n = \sum_{i = 1}^k d_ib^{i - 1}$$ where $d_1$ is the least significant digit and $d_k$ is the most significant, if it's also the case that for some power $m$ the equality $$n = \sum_{i = 1}^k {d_i}^m$$ also holds, then $n$ is an {\em Armstrong number} or {\em narcissistic number} or {\em plus perfect number} or {\em perfect digital invariant}.

In any given base $b$ there is a finite amount of Armstrong numbers, since the inequality $k(b - 1)^m > b^{k - 1}$ is false after a certain threshold.
%%%%%
%%%%%
\end{document}
