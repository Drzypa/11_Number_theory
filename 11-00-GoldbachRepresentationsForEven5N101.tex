\documentclass[12pt]{article}
\usepackage{pmmeta}
\pmcanonicalname{GoldbachRepresentationsForEven5N101}
\pmcreated{2013-03-22 17:25:47}
\pmmodified{2013-03-22 17:25:47}
\pmowner{PrimeFan}{13766}
\pmmodifier{PrimeFan}{13766}
\pmtitle{Goldbach representations for even $5 < n < 101$}
\pmrecord{7}{39805}
\pmprivacy{1}
\pmauthor{PrimeFan}{13766}
\pmtype{Example}
\pmcomment{trigger rebuild}
\pmclassification{msc}{11-00}
\pmclassification{msc}{11P32}

\endmetadata

% this is the default PlanetMath preamble.  as your knowledge
% of TeX increases, you will probably want to edit this, but
% it should be fine as is for beginners.

% almost certainly you want these
\usepackage{amssymb}
\usepackage{amsmath}
\usepackage{amsfonts}

% used for TeXing text within eps files
%\usepackage{psfrag}
% need this for including graphics (\includegraphics)
%\usepackage{graphicx}
% for neatly defining theorems and propositions
%\usepackage{amsthm}
% making logically defined graphics
%%%\usepackage{xypic}

% there are many more packages, add them here as you need them

% define commands here

\begin{document}
Christian Goldbach conjectured that all sufficiently large even integers can be represented as the sum of two odd primes. This has been checked for all even integers up to $10^{18}$. The table below gives Goldbach representations for the integers from 8 to 100.

\begin{tabular}{|r|l|}
6 & 3 + 3 \\
8 & 5 + 3 \\
10 & 7 + 3 = 5 + 5 \\
12 & 7 + 5 \\
14 & 11 + 3 = 7 + 7 \\
16 & 13 + 3 = 11 + 5 \\
18 & 13 + 5 = 11 + 7 \\
20 & 17 + 3 = 13 + 7 \\
22 & 19 + 3 = 17 + 5 = 11 + 11 \\
24 & 19 + 5 = 17 + 7 = 13 + 11 \\
26 & 23 + 3 = 19 + 7 = 13 + 13 \\
28 & 23 + 5 = 17 + 11 \\
30 & 23 + 7 = 19 + 11 = 17 + 13 \\
32 & 29 + 3 = 19 + 13 \\
34 & 31 + 3 = 29 + 5 = 23 + 11 = 17 + 17 \\
36 & 31 + 5 = 29 + 7 = 23 + 13 = 19 + 17 \\
38 & 31 + 7 = 19 + 19 \\
40 & 37 + 3 = 29 + 11 = 23 + 17 \\
42 & 37 + 5 = 31 + 11 = 29 + 13 = 23 + 19 \\
44 & 41 + 3 = 37 + 7 = 31 + 13 \\
46 & 43 + 3 = 41 + 5 = 29 + 17 = 23 + 23 \\
48 & 43 + 5 = 41 + 7 = 37 + 11 = 31 + 17 = 29 + 19 \\
50 & 47 + 3 = 43 + 7 = 37 + 13 = 31 + 19 \\
52 & 47 + 5 = 41 + 11 = 29 + 23 \\
54 & 47 + 7 = 43 + 11 = 41 + 13 = 37 + 17 = 31 + 23 \\
56 & 53 + 3 = 43 + 13 = 37 + 19 \\
58 & 53 + 5 = 47 + 11 = 41 + 17 = 29 + 29 \\
60 & 53 + 7 = 47 + 13 = 43 + 17 = 41 + 19 = 37 + 23 = 31 + 29 \\
62 & 59 + 3 = 43 + 19 = 31 + 31 \\
64 & 61 + 3 = 59 + 5 = 53 + 11 = 47 + 17 = 41 + 23 \\
66 & 61 + 5 = 59 + 7 = 53 + 13 = 47 + 19 = 43 + 23 = 37 + 29 \\
68 & 61 + 7 = 37 + 31 \\
70 & 67 + 3 = 59 + 11 = 53 + 17 = 47 + 23 = 41 + 29 \\
72 & 67 + 5 = 61 + 11 = 59 + 13 = 53 + 19 = 43 + 29 = 41 + 31 \\
74 & 71 + 3 = 67 + 7 = 61 + 13 = 43 + 31 = 37 + 37 \\
76 & 73 + 3 = 71 + 5 = 59 + 17 = 53 + 23 = 47 + 29 \\
78 & 73 + 5 = 71 + 7 = 67 + 11 = 61 + 17 = 59 + 19 = 47 + 31 = 41 + 37 \\
80 & 73 + 7 = 67 + 13 = 61 + 19 = 43 + 37 \\
82 & 79 + 3 = 71 + 11 = 59 + 23 = 53 + 29 = 41 + 41 \\
84 & 79 + 5 = 73 + 11 = 71 + 13 = 67 + 17 = 61 + 23 = 53 + 31 = 47 + 37 = 43 + 41 \\
86 & 83 + 3 = 79 + 7 = 73 + 13 = 67 + 19 = 43 + 43 \\
88 & 83 + 5 = 71 + 17 = 59 + 29 = 47 + 41 \\
90 & 83 + 7 = 79 + 11 = 73 + 17 = 71 + 19 = 67 + 23 = 61 + 29 = 59 + 31 = 53 + 37 = 47 + 43 \\
92 & 89 + 3 = 79 + 13 = 73 + 19 = 61 + 31 \\
94 & 89 + 5 = 83 + 11 = 71 + 23 = 53 + 41 = 47 + 47 \\
96 & 89 + 7 = 83 + 13 = 79 + 17 = 73 + 23 = 67 + 29 = 59 + 37 = 53 + 43 \\
98 & 79 + 19 = 67 + 31 = 61 + 37 \\
100 & 97 + 3 = 89 + 11 = 83 + 17 = 71 + 29 = 59 + 41 = 53 + 47 \\
\end{tabular}

%%%%%
%%%%%
\end{document}
