\documentclass[12pt]{article}
\usepackage{pmmeta}
\pmcanonicalname{ValueOfDirichletEtaFunctionAtS2}
\pmcreated{2013-03-22 18:22:09}
\pmmodified{2013-03-22 18:22:09}
\pmowner{pahio}{2872}
\pmmodifier{pahio}{2872}
\pmtitle{value of Dirichlet eta function at $s = 2$}
\pmrecord{8}{41010}
\pmprivacy{1}
\pmauthor{pahio}{2872}
\pmtype{Result}
\pmcomment{trigger rebuild}
\pmclassification{msc}{11M41}
\pmrelated{CosineAtMultiplesOfStraightAngle}
\pmrelated{ValueOfTheRiemannZetaFunctionAtS2}

\endmetadata

% this is the default PlanetMath preamble.  as your knowledge
% of TeX increases, you will probably want to edit this, but
% it should be fine as is for beginners.

% almost certainly you want these
\usepackage{amssymb}
\usepackage{amsmath}
\usepackage{amsfonts}

% used for TeXing text within eps files
%\usepackage{psfrag}
% need this for including graphics (\includegraphics)
%\usepackage{graphicx}
% for neatly defining theorems and propositions
 \usepackage{amsthm}
% making logically defined graphics
%%%\usepackage{xypic}

% there are many more packages, add them here as you need them

% define commands here

\theoremstyle{definition}
\newtheorem*{thmplain}{Theorem}

\begin{document}
The value 
$$\eta(2) = 1-\frac{1}{2^2}+\frac{1}{3^2}-\frac{1}{4^2}+\!-\ldots$$
of the Dirichlet eta function can be found by using the Fourier cosine series \PMlinkescapetext{expansion} of the function \,$x \mapsto x\!-\!x^2$
on the interval \,$[0,\,1]$:
\begin{align}
x\!-\!x^2 \;=\; \frac{1}{6}-\frac{1}{\pi^2}\sum_{n=1}^\infty\frac{\cos{2n\pi x}}{n^2} 
\quad \mbox{for}\;\; 0 \leqq x \leqq 1
\end{align}
Substituting\, $x := \frac{1}{2}$\, to the equation (1) yields
$$\frac{1}{4} \;=\; \frac{1}{6}-\frac{1}{\pi^2}\sum_{n=1}^\infty\frac{\cos{n\pi}}{n^2}
\;=\; \frac{1}{6}+\frac{1}{\pi^2}\sum_{n=1}^\infty\frac{(-1)^{n+1}}{n^2},$$
which we can solve to the form
\begin{align}
\eta(2) \;=\; \sum_{n=1}^\infty\frac{(-1)^{n+1}}{n^2} \;=\; \frac{\pi^2}{12}.
\end{align}
This result could be obtained very simply by using the functional equation connecting Dirichlet eta function to Riemann zeta function.

Combining the equation (2) with the result concerning the \PMlinkname{Riemann zeta function at 2}{ValueOfTheRiemannZetaFunctionAtS2} shows that
\begin{align}
1+\frac{1}{3^2}+\frac{1}{5^2}+\frac{1}{7^2}+\ldots \;=\; \frac{\pi^2}{8}.
\end{align}


%%%%%
%%%%%
\end{document}
