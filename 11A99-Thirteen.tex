\documentclass[12pt]{article}
\usepackage{pmmeta}
\pmcanonicalname{Thirteen}
\pmcreated{2013-03-22 17:24:58}
\pmmodified{2013-03-22 17:24:58}
\pmowner{PrimeFan}{13766}
\pmmodifier{PrimeFan}{13766}
\pmtitle{thirteen}
\pmrecord{6}{39789}
\pmprivacy{1}
\pmauthor{PrimeFan}{13766}
\pmtype{Feature}
\pmcomment{trigger rebuild}
\pmclassification{msc}{11A99}

% this is the default PlanetMath preamble.  as your knowledge
% of TeX increases, you will probably want to edit this, but
% it should be fine as is for beginners.

% almost certainly you want these
\usepackage{amssymb}
\usepackage{amsmath}
\usepackage{amsfonts}

% used for TeXing text within eps files
%\usepackage{psfrag}
% need this for including graphics (\includegraphics)
%\usepackage{graphicx}
% for neatly defining theorems and propositions
%\usepackage{amsthm}
% making logically defined graphics
%%%\usepackage{xypic}

% there are many more packages, add them here as you need them

% define commands here

\begin{document}
{\em Thirteen} is in many ways a rather ordinary number, yet some people are afraid of it (see triskaidekaphobia). For a mathematician there is nothing particularly unlucky about 13; in fact, 13 is a lucky prime (since it is not crossed off in the well-defined sieving process for lucky numbers).

13 is a Fibonacci number (being the sum of 5 and 8), and like all other odd-indexed Fibonacci numbers, it is also a Markov number. It appears in the following solutions to $13^2 + y^2 + z^2 = 39yz$: (1, 5, 13), (1, 13, 34), (5, 13, 194), (13, 34, 1325), (13, 194, 7561), (13, 1325, 51641), (13, 7561, 294685), (13, 51641, 2012674), (13, 294685, 11485154), (13, 2012674, 78442645), (13, 11485154, 447626321), etc. Starting with 1, the number 13 represents a new low for the Mertens function, but this is neither unlucky nor special (A051401 in Sloane's OEIS lists other lows of the Mertens function).

This is not to say that there isn't anything special about 13: for example, it is the only integer solution to $x^4 = n^2 + (n + 1)^2$ (besides the obvious $x = 1$ and $n = 0$).

%%%%%
%%%%%
\end{document}
