\documentclass[12pt]{article}
\usepackage{pmmeta}
\pmcanonicalname{Rootdiscriminant}
\pmcreated{2013-03-22 15:05:44}
\pmmodified{2013-03-22 15:05:44}
\pmowner{alozano}{2414}
\pmmodifier{alozano}{2414}
\pmtitle{root-discriminant}
\pmrecord{5}{36824}
\pmprivacy{1}
\pmauthor{alozano}{2414}
\pmtype{Definition}
\pmcomment{trigger rebuild}
\pmclassification{msc}{11R29}
\pmsynonym{root discriminant}{Rootdiscriminant}
%\pmkeywords{discriminant}
%\pmkeywords{root discriminant}
\pmrelated{ExistenceOfHilbertClassField}

% this is the default PlanetMath preamble.  as your knowledge
% of TeX increases, you will probably want to edit this, but
% it should be fine as is for beginners.

% almost certainly you want these
\usepackage{amssymb}
\usepackage{amsmath}
\usepackage{amsthm}
\usepackage{amsfonts}

% used for TeXing text within eps files
%\usepackage{psfrag}
% need this for including graphics (\includegraphics)
%\usepackage{graphicx}
% for neatly defining theorems and propositions
%\usepackage{amsthm}
% making logically defined graphics
%%%\usepackage{xypic}

% there are many more packages, add them here as you need them

% define commands here

\newtheorem{thm}{Theorem}
\newtheorem{defn}{Definition}
\newtheorem{prop}{Proposition}
\newtheorem{lemma}{Lemma}
\newtheorem{cor}{Corollary}

% Some sets
\newcommand{\Nats}{\mathbb{N}}
\newcommand{\Ints}{\mathbb{Z}}
\newcommand{\Reals}{\mathbb{R}}
\newcommand{\Complex}{\mathbb{C}}
\newcommand{\Rats}{\mathbb{Q}}
\newcommand{\rd}{\operatorname{rd}}
\begin{document}
\begin{defn}
Let $K$ be a number field, let $d_K$ be its discriminant and let $n=[K:\Rats]$ be the degree over $\Rats$. The quantity:
$$|\sqrt[n]{d_K}|$$
is called the {\bf root-discriminant} of $K$ and it is usually denoted by $\operatorname{rd}_K$.
\end{defn}

The following lemma is one of the motivations for the previous definition:

\begin{lemma}
Let $E/F$ be an extension of number fields which is unramified at all finite primes. Then $\rd_E=\rd_F$. In particular, the Hilbert class field of a number field has the same root-discriminant as the number field.
\end{lemma}

\begin{proof}
Notice that the relative discriminant ideal (or different) for $E/F$ is the ring of integers in $F$. Therefore we have:
$$|d_E|=|d_F|^{[E:F]}$$
The results follows by taking $[E:\Rats]$-th roots on both sides of the previous equation.
\end{proof}
%%%%%
%%%%%
\end{document}
