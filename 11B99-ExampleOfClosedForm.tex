\documentclass[12pt]{article}
\usepackage{pmmeta}
\pmcanonicalname{ExampleOfClosedForm}
\pmcreated{2013-03-22 15:03:07}
\pmmodified{2013-03-22 15:03:07}
\pmowner{drini}{3}
\pmmodifier{drini}{3}
\pmtitle{example of closed form}
\pmrecord{4}{36769}
\pmprivacy{1}
\pmauthor{drini}{3}
\pmtype{Example}
\pmcomment{trigger rebuild}
\pmclassification{msc}{11B99}

\endmetadata

\usepackage{graphicx}
%%%\usepackage{xypic} 
\usepackage{bbm}
\newcommand{\Z}{\mathbbmss{Z}}
\newcommand{\C}{\mathbbmss{C}}
\newcommand{\R}{\mathbbmss{R}}
\newcommand{\Q}{\mathbbmss{Q}}
\newcommand{\mathbb}[1]{\mathbbmss{#1}}
\newcommand{\figura}[1]{\begin{center}\includegraphics{#1}\end{center}}
\newcommand{\figuraex}[2]{\begin{center}\includegraphics[#2]{#1}\end{center}}
\newtheorem{dfn}{Definition}
\begin{document}
Consider the recurrence given by
$x_0 = 7$ and $x_n = 2 x_{n-1}$. Then it can be proved by induction that
$x_n = 2^n\cdot 7$.

The expression $x_n = 2^n \cdot 7$ is a closed form expression the  recurrence given, since it depends exclusively on $n$ , whereas the recurrence depends on $n$ and $x_{n-1}$ (the previous value).
\bigskip

Now consider Fibonacci's recurrence:
\[
f_n = f_{n-1}+ f_{n-2},\qquad f_0 = 1.
\]

It is not a closed formula, since if we wanted to compute $f_{10}$ we would need to know $f_9$, $f_8$ (and for knowing them, the previous terms too). However, such recurrence has a closed formula, known as Binet formula:
\[
f_n = \frac{\left(\frac{1+\sqrt5}{2}\right)^n -\left(\frac{1-\sqrt5}{2}\right)^n }{\sqrt5}.
\]
Binet formula is closed, since if we wanted to compute $f_n$ we only need to substitute $n$ on the previous formula, and no additional information is required.
%%%%%
%%%%%
\end{document}
