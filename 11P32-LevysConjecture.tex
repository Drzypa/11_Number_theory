\documentclass[12pt]{article}
\usepackage{pmmeta}
\pmcanonicalname{LevysConjecture}
\pmcreated{2013-03-22 17:26:32}
\pmmodified{2013-03-22 17:26:32}
\pmowner{PrimeFan}{13766}
\pmmodifier{PrimeFan}{13766}
\pmtitle{Levy's conjecture}
\pmrecord{6}{39822}
\pmprivacy{1}
\pmauthor{PrimeFan}{13766}
\pmtype{Conjecture}
\pmcomment{trigger rebuild}
\pmclassification{msc}{11P32}
\pmsynonym{Levy conjecture}{LevysConjecture}
\pmsynonym{Lemoine's conjecture}{LevysConjecture}

% this is the default PlanetMath preamble.  as your knowledge
% of TeX increases, you will probably want to edit this, but
% it should be fine as is for beginners.

% almost certainly you want these
\usepackage{amssymb}
\usepackage{amsmath}
\usepackage{amsfonts}

% used for TeXing text within eps files
%\usepackage{psfrag}
% need this for including graphics (\includegraphics)
%\usepackage{graphicx}
% for neatly defining theorems and propositions
%\usepackage{amsthm}
% making logically defined graphics
%%%\usepackage{xypic}

% there are many more packages, add them here as you need them

% define commands here

\begin{document}
Conjecture (\'Emile Lemoine). All odd integers greater than 5 can be represented as the sum of an odd prime and an even semiprime. In other words, $2n + 1 = p + 2q$ always has a solution in primes $p$ and $q$ (not necessarily distinct) for $n > 2$.

For example, $47 = 13 + 2 \times 17 = 37 + 2 \times 5 = 41 + 2 \times 3 = 43 + 2 \times 2$. A046927 in Sloane's OEIS counts how many different ways $2n + 1$ can be represented as $p + 2q$.

The conjecture was first stated by \'Emile Lemoine in 1894. In 1963, Hyman Levy published a paper mentioning this conjecture in relation to Goldbach's conjecture.

\begin{thebibliography}{3}
\bibitem{ld} L. E. Dickson, {\it History of the Theory of Numbers} Vol. I. Providence, Rhode Island: American Mathematical Society \& Chelsea Publications (1999): 424
\bibitem{rg} R. K. Guy, {\it Unsolved Problems in Number Theory} New York: Springer-Verlag 2004: C1
\bibitem{lh} L. Hodges, ``A lesser-known Goldbach conjecture'', {\it Math. Mag.}, {\bf 66} (1993): 45 - 47. 
\bibitem{el} \'E. Lemoine, ``title'' {\it L'intermediaire des mathematiques} {\bf 179} 3 (1896): 151
\bibitem{hl} H. Levy, ``On Goldbach's Conjecture'', {\it Math. Gaz.} {\bf 47} (1963): 274
\end{thebibliography}
%%%%%
%%%%%
\end{document}
