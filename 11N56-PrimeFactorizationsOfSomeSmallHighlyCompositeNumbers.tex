\documentclass[12pt]{article}
\usepackage{pmmeta}
\pmcanonicalname{PrimeFactorizationsOfSomeSmallHighlyCompositeNumbers}
\pmcreated{2013-03-22 18:51:29}
\pmmodified{2013-03-22 18:51:29}
\pmowner{PrimeFan}{13766}
\pmmodifier{PrimeFan}{13766}
\pmtitle{prime factorizations of some small highly composite numbers}
\pmrecord{5}{41670}
\pmprivacy{1}
\pmauthor{PrimeFan}{13766}
\pmtype{Example}
\pmcomment{trigger rebuild}
\pmclassification{msc}{11N56}

\endmetadata

% this is the default PlanetMath preamble.  as your knowledge
% of TeX increases, you will probably want to edit this, but
% it should be fine as is for beginners.

% almost certainly you want these
\usepackage{amssymb}
\usepackage{amsmath}
\usepackage{amsfonts}

% used for TeXing text within eps files
%\usepackage{psfrag}
% need this for including graphics (\includegraphics)
%\usepackage{graphicx}
% for neatly defining theorems and propositions
%\usepackage{amsthm}
% making logically defined graphics
%%%\usepackage{xypic}

% there are many more packages, add them here as you need them

% define commands here

\begin{document}
All highly composite numbers have prime factorizations containing consecutive prime numbers, starting with 2, while the exponents form a sequence in descending order (the number 1 as an exponent is tacit in the following table).

\begin{tabular}{|r|l|}
HCN & Factorization \\
4 & $2^2$ \\
6 & $2 \times 3$ \\
12 & $2^2 \times 3$ \\
24 & $2^3 \times 3$ \\
36 & $2^2 \times 3^2$ \\
48 & $2^4 \times 3$ \\
60 & $2^2 \times 3 \times 5$ \\
120 & $2^3 \times 3 \times 5$ \\
180 & $2^2 \times 3^2 \times 5$ \\
240 & $2^4 \times 3 \times 5$ \\
360 & $2^3 \times 3^2 \times 5$ \\
720 & $2^4 \times 3^2 \times 5$ \\
840 & $2^3 \times 3 \times 5 \times 7$ \\
1260 & $2^2 \times 3^2 \times 5 \times 7$ \\
1680 & $2^4 \times 3 \times 5 \times 7$ \\
2520 & $2^3 \times 3^2 \times 5 \times 7$ \\
5040 & $2^4 \times 3^2 \times 5 \times 7$ \\
7560 & $2^3 \times 3^3 \times 5 \times 7$ \\
10080 & $2^5 \times 3^2 \times 5 \times 7$ \\
15120 & $2^4 \times 3^3 \times 5 \times 7$ \\
20160 & $2^6 \times 3^2 \times 5 \times 7$ \\
25200 & $2^4 \times 3^2 \times 5^2 \times 7$ \\
27720 & $2^3 \times 3^2 \times 5 \times 7 \times 11$ \\
45360 & $2^4 \times 3^4 \times 5 \times 7$ \\
50400 & $2^5 \times 3^2 \times 5^2 \times 7$ \\
55440 & $2^4 \times 3^2 \times 5 \times 7 \times 11$ \\
83160 & $2^3 \times 3^3 \times 5 \times 7 \times 11$ \\
110880 & $2^5 \times 3^2 \times 5 \times 7 \times 11$ \\
166320 & $2^4 \times 3^3 \times 5 \times 7 \times 11$ \\
221760 & $2^6 \times 3^2 \times 5 \times 7 \times 11$ \\
277200 & $2^4 \times 3^2 \times 5^2 \times 7 \times 11$ \\
332640 & $2^5 \times 3^3 \times 5 \times 7 \times 11$ \\
498960 & $2^4 \times 3^4 \times 5 \times 7 \times 11$ \\
554400 & $2^5 \times 3^2 \times 5^2 \times 7 \times 11$ \\
665280 & $2^6 \times 3^3 \times 5 \times 7 \times 11$ \\
720720 & $2^4 \times 3^2 \times 5 \times 7 \times 11 \times 13$ \\
1081080 & $2^3 \times 3^3 \times 5 \times 7 \times 11 \times 13$ \\
1441440 & $2^5 \times 3^2 \times 5 \times 7 \times 11 \times 13$ \\
2162160 & $2^4 \times 3^3 \times 5 \times 7 \times 11 \times 13$ \\
\end{tabular}
%%%%%
%%%%%
\end{document}
