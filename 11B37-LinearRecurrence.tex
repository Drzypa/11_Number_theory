\documentclass[12pt]{article}
\usepackage{pmmeta}
\pmcanonicalname{LinearRecurrence}
\pmcreated{2013-03-22 16:52:29}
\pmmodified{2013-03-22 16:52:29}
\pmowner{Mathprof}{13753}
\pmmodifier{Mathprof}{13753}
\pmtitle{linear recurrence}
\pmrecord{8}{39125}
\pmprivacy{1}
\pmauthor{Mathprof}{13753}
\pmtype{Definition}
\pmcomment{trigger rebuild}
\pmclassification{msc}{11B37}

\endmetadata

% this is the default PlanetMath preamble.  as your knowledge
% of TeX increases, you will probably want to edit this, but
% it should be fine as is for beginners.

% almost certainly you want these
\usepackage{amssymb}
\usepackage{amsmath}
\usepackage{amsfonts}

% used for TeXing text within eps files
%\usepackage{psfrag}
% need this for including graphics (\includegraphics)
%\usepackage{graphicx}
% for neatly defining theorems and propositions
%\usepackage{amsthm}
% making logically defined graphics
%%%\usepackage{xypic}

% there are many more packages, add them here as you need them

% define commands here

\begin{document}
A sequence $x_0, x_1, \ldots,$ is said to satisfy a \emph{general linear recurrence} if there are constants $a_{n,i}$ and $f_n$ such that
$$
x_n = \sum_{i=1}^n a_{n,i} x_{n-i} + f_n
$$
The sequence $\{f_n\}$ is called the \emph{non-homogeneous part}
of the recurrence. 

If $f_n = 0$ for all $n$ then the recurrence is said to be \emph{homogeneous}.


In this case if $x$ and $x'$ are two solutions to the recurrence,
and $A$ and $B$ are constants, then $Ax +Bx'$ are also solutions.

If $a_{n,i} = a_i$ for all $i$ then the recurrence is called
a \emph{linear recurrence with constant coefficients}. Such a recurrence
with $a_{n,i} = 0$ for all $i>k$ and $a_k \ne 0$ is said to be of 
\emph{finite order} and the smallest such integer $k$ is called the
\emph{order} of the recurrence. If no such $k$ exists, the recurrence is
said to be of \emph{infinite order}.

For example, the Fibonacci sequence is a linear recurrence with constant coefficients and order 2.

Now suppose that $x=\{x_n\}$ satisfies a homogeneous linear recurrence of 
finite order $k$. The polynomial 
$F_x = u^k - a_1u^{k-1} - \ldots - a_k$ is called the \emph{companion polynomial}
of $x$. 
If we assume that the coefficients $a_i$ come from a field $K$ we can
write 
$$
F_x = (u-u_1)^{e_1} \cdots (u-u_t)^{e_t}
$$ 
where $u_1, \ldots , u_t$ are the distinct roots of $F_x$ in some
splitting field of $F_x$ containing $K$ and $e_1, \ldots , e_t$ are
positive integers. 
A theorem about such recurrence relations is that
$$
x_n = \sum_{i=1}^t g_i(n){u_i}^n
$$ 
for some polynomials $g_i$ in $K[X]$ where $deg(g_i) <e_i$.

Now suppose further that all $K=\mathbb{C}$ and 

let $U$ be the largest value of the $|u_i|$.

It follows that if $U>1$ then 
$$
|x_n| \le CU^n 
$$
for some constant $C$. 


(to be continued)

 
%%%%%
%%%%%
\end{document}
