\documentclass[12pt]{article}
\usepackage{pmmeta}
\pmcanonicalname{QuadraticCharacterOf2}
\pmcreated{2013-03-22 13:58:03}
\pmmodified{2013-03-22 13:58:03}
\pmowner{mathcam}{2727}
\pmmodifier{mathcam}{2727}
\pmtitle{quadratic character of 2}
\pmrecord{5}{34737}
\pmprivacy{1}
\pmauthor{mathcam}{2727}
\pmtype{Theorem}
\pmcomment{trigger rebuild}
\pmclassification{msc}{11A15}
\pmrelated{ValuesOfTheLegendreSymbol}

\usepackage{amssymb}
\usepackage{amsmath}
\usepackage{amsfonts}
\begin{document}
For any odd prime $p$, Gauss's lemma quickly yields
\begin{eqnarray}
\left( \frac{2}{p} \right) &=& 1 \text{ if }p\equiv\pm 1\pmod{8} \\
\left( \frac{2}{p} \right) &=& -1 \text{ if }p\equiv\pm 3\pmod{8}
\end{eqnarray}
But there is another way, which goes back to Euler, and is worth
seeing, inasmuch as it is the prototype of certain more general arguments
about character sums.

Let $\sigma$ be a primitive eighth root of unity in an algebraic closure
of $\mathbb{Z}/p\mathbb{Z}$, and write $\tau=\sigma+\sigma^{-1}$. 
We have $\sigma^4=-1$,
whence $\sigma^2+\sigma^{-2}=0$, whence
$$\tau^2=2\;.$$
By the binomial formula, we have
$$\tau^p=\sigma^p+\sigma^{-p}\;.$$
If $p\equiv\pm 1\pmod 8$, this implies $\tau^p=\tau$.
If $p\equiv\pm 3\pmod 8$, we get instead $\tau^p=\sigma^5+\sigma^{-5}=
-\sigma^{-1}-\sigma=-\tau$.
In both cases, we get $\tau^{p-1}=\left( \frac{2}{p} \right)$,
proving (1) and (2).

A variation of the argument, closer to Euler's, goes as follows.
Write
$$\sigma=\exp(2\pi i/8)$$
$$\tau=\sigma+\sigma^{-1}$$
Both are algebraic integers. Arguing much as above, we end up with
$$\tau^{p-1}\equiv\left( \frac{2}{p} \right)\pmod{p}$$
which is enough.
%%%%%
%%%%%
\end{document}
