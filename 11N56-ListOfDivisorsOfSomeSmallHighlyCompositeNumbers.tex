\documentclass[12pt]{article}
\usepackage{pmmeta}
\pmcanonicalname{ListOfDivisorsOfSomeSmallHighlyCompositeNumbers}
\pmcreated{2013-03-22 18:51:12}
\pmmodified{2013-03-22 18:51:12}
\pmowner{PrimeFan}{13766}
\pmmodifier{PrimeFan}{13766}
\pmtitle{list of divisors of some small highly composite numbers}
\pmrecord{4}{41663}
\pmprivacy{1}
\pmauthor{PrimeFan}{13766}
\pmtype{Example}
\pmcomment{trigger rebuild}
\pmclassification{msc}{11N56}

% this is the default PlanetMath preamble.  as your knowledge
% of TeX increases, you will probably want to edit this, but
% it should be fine as is for beginners.

% almost certainly you want these
\usepackage{amssymb}
\usepackage{amsmath}
\usepackage{amsfonts}

% used for TeXing text within eps files
%\usepackage{psfrag}
% need this for including graphics (\includegraphics)
%\usepackage{graphicx}
% for neatly defining theorems and propositions
%\usepackage{amsthm}
% making logically defined graphics
%%%\usepackage{xypic}

% there are many more packages, add them here as you need them

% define commands here

\begin{document}
The following table gives the divisors from 1 to $n$ of highly composite numbers $n$ from 12 to 840. The number 1 is in the leftmost column, while the rightmost column (ignoring blank cells) gives $n$.

\begin{tabular}{|r|r|r|r|r|r|r|r|r|r|r|r|r|r|r|r|r|r|r|r|r|r|r|r|r|r|r|r|r|r|r|r|}
1 & 2 & 3 & 4 & 6 & 12 &  &  &  &  &  &  &  &  &  &  &  &  &  &  &  &  &  &  &  &  &  &  &  &  &  &  \\
1 & 2 & 3 & 4 & 6 & 8 & 12 & 24 &  &  &  &  &  &  &  &  &  &  &  &  &  &  &  &  &  &  &  &  &  &  &  &  \\
1 & 2 & 3 & 4 & 6 & 9 & 12 & 18 & 36 &  &  &  &  &  &  &  &  &  &  &  &  &  &  &  &  &  &  &  &  &  &  &  \\
1 & 2 & 3 & 4 & 6 & 8 & 12 & 16 & 24 & 48 &  &  &  &  &  &  &  &  &  &  &  &  &  &  &  &  &  &  &  &  &  &  \\
1 & 2 & 3 & 4 & 5 & 6 & 10 & 12 & 15 & 20 & 30 & 60 &  &  &  &  &  &  &  &  &  &  &  &  &  &  &  &  &  &  &  &  \\
1 & 2 & 3 & 4 & 5 & 6 & 8 & 10 & 12 & 15 & 20 & 24 & 30 & 40 & 60 & 120 &  &  &  &  &  &  &  &  &  &  &  &  &  &  &  &  \\
1 & 2 & 3 & 4 & 5 & 6 & 9 & 10 & 12 & 15 & 18 & 20 & 30 & 36 & 45 & 60 & 90 & 180 &  &  &  &  &  &  &  &  &  &  &  &  &  &  \\
1 & 2 & 3 & 4 & 5 & 6 & 8 & 10 & 12 & 15 & 16 & 20 & 24 & 30 & 40 & 48 & 60 & 80 & 120 & 240 &  &  &  &  &  &  &  &  &  &  &  &  \\
1 & 2 & 3 & 4 & 5 & 6 & 8 & 9 & 10 & 12 & 15 & 18 & 20 & 24 & 30 & 36 & 40 & 45 & 60 & 72 & 90 & 120 & 180 & 360 &  &  &  &  &  &  &  &  \\
1 & 2 & 3 & 4 & 5 & 6 & 8 & 9 & 10 & 12 & 15 & 16 & 18 & 20 & 24 & 30 & 36 & 40 & 45 & 48 & 60 & 72 & 80 & 90 & 120 & 144 & 180 & 240 & 360 & 720 &  &  \\
1 & 2 & 3 & 4 & 5 & 6 & 7 & 8 & 10 & 12 & 14 & 15 & 20 & 21 & 24 & 28 & 30 & 35 & 40 & 42 & 56 & 60 & 70 & 84 & 105 & 120 & 140 & 168 & 210 & 280 & 420 & 840 \\
\end{tabular}
%%%%%
%%%%%
\end{document}
