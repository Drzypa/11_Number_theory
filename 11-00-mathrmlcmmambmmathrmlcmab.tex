\documentclass[12pt]{article}
\usepackage{pmmeta}
\pmcanonicalname{mathrmlcmmambmmathrmlcmab}
\pmcreated{2013-03-22 15:03:22}
\pmmodified{2013-03-22 15:03:22}
\pmowner{drini}{3}
\pmmodifier{drini}{3}
\pmtitle{$\mathrm{lcm}(ma,mb) =m \mathrm{lcm}(a,b)$}
\pmrecord{12}{36774}
\pmprivacy{1}
\pmauthor{drini}{3}
\pmtype{Theorem}
\pmcomment{trigger rebuild}
\pmclassification{msc}{11-00}

\endmetadata

\newcommand{\lcm}{\,\mathrm{lcm}}
\usepackage{amssymb}
\begin{document}
For simplicity, let us work only with positive integers.

We want to prove that if a,b,m are integers, then
\[
\lcm(ma,mb) = m\lcm(a,b).
\]

First notice that any common multiple of $ma$ and $mb$ is also a multiple of $m$, so any common multiple of $ma$ and $mb$ is of the form $mk$ with some integer $k$.

Now notice that if $t=\lcm(a,b)$ and $u<t$, it cannot happen that $a\mid u$ and $b\mid u$, since $t$ is the smallest number, So, when $a\nmid u$ then $ma\nmid mu$, and if $b\nmid u$ then $mb \nmid mu$. We conclude that $mu$ is \emph{not} a common multiple of $ma$ and $mb$ when $u<t$.

So far, we proved that $mt = m\lcm(a,b)$ is a common multiple of $ma$ and $mb$, and previous paragraph shows that there is no smaller common multiple, therefore $m\lcm(a,b)$  is the \emph{least} common multiple of $ma$ and $mb$, in other words:
\[
\lcm(ma,mb) = m\lcm(a,b).
\]
%%%%%
%%%%%
\end{document}
