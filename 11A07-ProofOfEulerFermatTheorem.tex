\documentclass[12pt]{article}
\usepackage{pmmeta}
\pmcanonicalname{ProofOfEulerFermatTheorem}
\pmcreated{2013-03-22 11:47:57}
\pmmodified{2013-03-22 11:47:57}
\pmowner{KimJ}{5}
\pmmodifier{KimJ}{5}
\pmtitle{proof of Euler-Fermat theorem}
\pmrecord{10}{30335}
\pmprivacy{1}
\pmauthor{KimJ}{5}
\pmtype{Proof}
\pmcomment{trigger rebuild}
\pmclassification{msc}{11A07}
\pmclassification{msc}{11A25}
%\pmkeywords{number theory}

\endmetadata

\usepackage{amssymb}
\usepackage{amsmath}
\usepackage{amsfonts}
\usepackage{graphicx}
%%%%\usepackage{xypic}
\begin{document}
Let $a_1, a_2, \ldots , a_{\phi (n)}$ be all positive integers less than $n$ which are coprime to $n$. Since $\text{gcd}(a,n)=1$, then the set $aa_1, aa_2,\ldots ,aa_{\phi (n)}$ are each congruent to one of the integers $a_1, a_2, \ldots ,a_{\phi (n)}$ in some order. Taking the product of these congruences, we get
\[ (aa_1)(aa_2) \cdots (aa_{\phi (n)}) \equiv a_1 a_2 \cdots a_{\phi (n)} \pmod{n} \]
hence
\[ a^{\phi (n)}(a_1 a_2 \cdots a_{\phi (n)}) \equiv a_1 a_2 \cdots a_{\phi (n)} \pmod{n}. \]

Since $\text{gcd}(a_1a_2\cdots a_{\phi (n)},n) = 1$, we can divide both sides by $a_1a_2\cdots a_{\phi (n)}$, and the desired result follows.
%%%%%
%%%%%
%%%%%
%%%%%
\end{document}
