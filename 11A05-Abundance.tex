\documentclass[12pt]{article}
\usepackage{pmmeta}
\pmcanonicalname{Abundance}
\pmcreated{2013-03-22 16:05:49}
\pmmodified{2013-03-22 16:05:49}
\pmowner{CompositeFan}{12809}
\pmmodifier{CompositeFan}{12809}
\pmtitle{abundance}
\pmrecord{9}{38159}
\pmprivacy{1}
\pmauthor{CompositeFan}{12809}
\pmtype{Definition}
\pmcomment{trigger rebuild}
\pmclassification{msc}{11A05}
\pmrelated{Deficiency}

\endmetadata

% this is the default PlanetMath preamble.  as your knowledge
% of TeX increases, you will probably want to edit this, but
% it should be fine as is for beginners.

% almost certainly you want these
\usepackage{amssymb}
\usepackage{amsmath}
\usepackage{amsfonts}

% used for TeXing text within eps files
%\usepackage{psfrag}
% need this for including graphics (\includegraphics)
%\usepackage{graphicx}
% for neatly defining theorems and propositions
%\usepackage{amsthm}
% making logically defined graphics
%%%\usepackage{xypic}

% there are many more packages, add them here as you need them

% define commands here

\begin{document}
Given an integer $n$ with divisors $d_1, \ldots , d_k$ (where the divisors are in ascending order and $d_1 = 1$, $d_k = n$) the difference $$\left( \sum_{i = 1}^k d_i \right) - 2n$$ is the {\em abundance} of $n$. Or if one prefers, $$\left( \sum_{i = 1}^{k - 1} d_i \right) - n.$$

For example, the divisors of 12 (which are 1, 2, 3, 4, 6 and 12) add up to 28, which is 4 more than 24 (twice 12). Therefore, 12 has an abundance of 4. For the sake of comparison, the divisors of 13 are 1 and 13, adding up to 14, which is 12 less than 26 (twice 13). Therefore, 13 has an abundance of $-12$. A033880 in Sloane's OEIS lists the abundance of the first sixty-three positive integers.

Thus numbers with positive abundance are abundant numbers. A number with an abundance of exactly 1 is a quasiperfect number, while a number with 0 abundance is a perfect number. A number with an abundance of $-1$ is an almost perfect number (this is true of all powers of 2); all numbers with negative abundance are deficient numbers.
%%%%%
%%%%%
\end{document}
