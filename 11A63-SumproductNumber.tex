\documentclass[12pt]{article}
\usepackage{pmmeta}
\pmcanonicalname{SumproductNumber}
\pmcreated{2013-03-22 15:46:50}
\pmmodified{2013-03-22 15:46:50}
\pmowner{Mravinci}{12996}
\pmmodifier{Mravinci}{12996}
\pmtitle{sum-product number}
\pmrecord{8}{37738}
\pmprivacy{1}
\pmauthor{Mravinci}{12996}
\pmtype{Definition}
\pmcomment{trigger rebuild}
\pmclassification{msc}{11A63}
\pmsynonym{sum product number}{SumproductNumber}

\endmetadata

% this is the default PlanetMath preamble.  as your knowledge
% of TeX increases, you will probably want to edit this, but
% it should be fine as is for beginners.

% almost certainly you want these
\usepackage{amssymb}
\usepackage{amsmath}
\usepackage{amsfonts}

% used for TeXing text within eps files
%\usepackage{psfrag}
% need this for including graphics (\includegraphics)
%\usepackage{graphicx}
% for neatly defining theorems and propositions
%\usepackage{amsthm}
% making logically defined graphics
%%%\usepackage{xypic}

% there are many more packages, add them here as you need them

% define commands here
\begin{document}
A \emph{sum-product number} is an integer $n$ that in a given base satisfies the equality

$$n = \sum_{i = 1}^m d_i \prod_{i = 1}^m d_i$$

where $d_i$ is a digit of $n$, and $m$ is the number of digits of $n$. This means a test of whether the sum of the digits of $n$ times the product of the digits of $n$ is equal to $n$.

For example, the number 128 in base 7 is a sum-product number since

$$242_7 = (2 + 4 + 2)(2 \cdot 4 \cdot 2)$$

All sum-product numbers are Harshad numbers, too.

0 and 1 are sum-product numbers in any positional base. The proof that the set of sum-product numbers in base 2 is finite is elementary enough not to inspire claims of authorship. The proof that the set of sum-product numbers in base 10 is finite (specifically, 0, 1, 135 and 144) is more involved but within the realm of basic algebra, and it points the way to a formulation of the proof that number of sum-product numbers in any base is finite.
%%%%%
%%%%%
\end{document}
