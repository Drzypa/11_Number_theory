\documentclass[12pt]{article}
\usepackage{pmmeta}
\pmcanonicalname{ProofOfFaulhabersFormula}
\pmcreated{2013-03-22 18:43:50}
\pmmodified{2013-03-22 18:43:50}
\pmowner{rm50}{10146}
\pmmodifier{rm50}{10146}
\pmtitle{proof of Faulhaber's formula}
\pmrecord{4}{41499}
\pmprivacy{1}
\pmauthor{rm50}{10146}
\pmtype{Theorem}
\pmcomment{trigger rebuild}
\pmclassification{msc}{11B68}

\endmetadata

% this is the default PlanetMath preamble.  as your knowledge
% of TeX increases, you will probably want to edit this, but
% it should be fine as is for beginners.

% almost certainly you want these
\usepackage{amssymb}
\usepackage{amsmath}
\usepackage{amsfonts}

% used for TeXing text within eps files
%\usepackage{psfrag}
% need this for including graphics (\includegraphics)
%\usepackage{graphicx}
% for neatly defining theorems and propositions
\usepackage{amsthm}
% making logically defined graphics
%%%\usepackage{xypic}

% there are many more packages, add them here as you need them

% define commands here
\newcommand{\Nats}{\mathbb{N}}
\newcommand{\Ints}{\mathbb{Z}}
\newtheorem{thm}{Theorem}[section]

\begin{document}
\begin{thm}If $k\in\Nats, 2\leq n\in\Ints$, then
\[\sum_{m=1}^{n-1}m^k = \frac{1}{k+1}\sum_{i=0}^k \binom{k+1}{i}B_in^{k+1-i}=\int_1^n b_k(x)dx\]
where the $B_i$ are the Bernoulli numbers and $b_i$ the Bernoulli polynomials.
\end{thm}

The exponential generating function for the Bernoulli numbers is
\[\sum_{n=0}^{\infty}B_n\frac{x^n}{n!}=\frac{x}{e^x-1}\]
We develop an equation involving sums of Bernoulli numbers on one side, and a simple generating involving powers of $e$ that gives us the appropriate sum of powers on the other side. Equating coefficients of powers of $x$ then gives the result.

To get a generating function where the coefficient of $x^n/n!$ is $\sum_{m=1}^{n-1}m^k$, we can use
\begin{align*}
  \sum_{m=0}^{n-1} e^{mx} 
     &= \sum_{m=0}^{n-1} \sum_{k=0}^{\infty} \frac{m^k x^k}{k!}\\
     &= \sum_{k=0}^{\infty}\left(\sum_{m=1}^{n-1} m^k\right)\frac{x^k}{k!}
\end{align*}
But this is also a geometric series, so
\begin{align*}
  \sum_{k=0}^{n-1}e^{kx} &= \frac{1-e^{nx}}{1-e^x}\\
    &= \frac{e^{nx}-1}{x}\cdot\frac{x}{e^x-1} \\
    &= \frac{e^{nx}-1}{x}\sum_{l=0}^{\infty}B_l\frac{x^l}{l!} \\
    &= \left(\sum_{k=0}^{\infty}\frac{n^{k+1}}{k+1}\cdot\frac{x^k}{k!}\right)
       \left(\sum_{l=0}^{\infty}B_l\frac{x^l}{l!}\right) \\
    &=\sum_{k=0}^{\infty}
            \left(\sum_{i=0}^j \frac{1}{k-i+1}\binom{k}{i}B_in^{k+1-i}\right)\frac{x^k}{k!}
\end{align*}
Equating coefficients of $x^k/k!$ we get
\begin{align*}\sum_{m=1}^{n-1}m^k&=\sum_{i=0}^k \frac{1}{k-i+1}\binom{k}{i}B_i n^{k+1-i}\\
&=\sum_{i=0}^k\frac{k!}{(k-i+1)i!(k-i)!}B_in^{k+1-i}=\frac{1}{k+1}\sum_{i=0}^k\binom{k+1}{i}B_in^{k+1-i}
\end{align*}
which proves the first equality.

If $f(x)$ is a polynomial, write $[x^r]f(x)$ for the coefficient of $x^r$ in $f(x)$. Then
\[[x^r]b_k(x)=\frac{1}{r}[x^{r-1}]b_k'(x)=\frac{k}{r}[x^{r-1}]b_{k-1}(x)\]
and thus if $r\leq k$, iterating, we get
\[[x^r]b_k(x)=\binom{k}{r}[x^0]b_{k-r}(x)=\binom{k}{r}B_{k-r}\]
Then using the fact that $b_k'=kb_{k-1}$, we have
\begin{align*}
\int_1^n b_k(x)&=\frac{1}{k+1}(b_{k+1}(n)-b_{k+1}(1)) = \frac{1}{k+1}\sum_{r=0}^{k+1}[x^r]b_{k+1}(x)(n^r-1)\\
&= \frac{1}{k+1}\sum_{r=0}^{k+1}\binom{k+1}{r}B_{k+1-r}(n^r-1) = \frac{1}{k+1}\sum_{r=1}^{k+1}\binom{k+1}{r}B_{k+1-r}n^r
\end{align*}
Now reverse the order of summation (i.e. replace $r$ by $k+1-r$) to get
\[
\int_1^n b_k(x)=\frac{1}{k+1}\sum_{r=0}^k\binom{k+1}{k+1-r}B_rn^{k+1-r}=\frac{1}{k+1}\sum_{r=0}^k\binom{k+1}{r}B_r n^{k+1-r}\]
%%%%%
%%%%%
\end{document}
