\documentclass[12pt]{article}
\usepackage{pmmeta}
\pmcanonicalname{FreshmansDream}
\pmcreated{2013-03-22 15:51:17}
\pmmodified{2013-03-22 15:51:17}
\pmowner{Algeboy}{12884}
\pmmodifier{Algeboy}{12884}
\pmtitle{freshman's dream}
\pmrecord{18}{37839}
\pmprivacy{1}
\pmauthor{Algeboy}{12884}
\pmtype{Theorem}
\pmcomment{trigger rebuild}
\pmclassification{msc}{11T23}
\pmclassification{msc}{11T30}
\pmsynonym{Frobenius Automorphism}{FreshmansDream}
%\pmkeywords{Field automorphism}
%\pmkeywords{finite characteristic}
%\pmkeywords{positive characteristic}
\pmrelated{PolynomialCongruence}

\endmetadata

\usepackage{latexsym}
\usepackage{amssymb}
\usepackage{amsmath}
\usepackage{amsfonts}
\usepackage{amsthm}

%%\usepackage{xypic}

%-----------------------------------------------------

%       Standard theoremlike environments.

%       Stolen directly from AMSLaTeX sample

%-----------------------------------------------------

%% \theoremstyle{plain} %% This is the default

\newtheorem{thm}{Theorem}

\newtheorem{coro}[thm]{Corollary}

\newtheorem{lem}[thm]{Lemma}

\newtheorem{lemma}[thm]{Lemma}

\newtheorem{prop}[thm]{Proposition}

\newtheorem{conjecture}[thm]{Conjecture}

\newtheorem{conj}[thm]{Conjecture}

\newtheorem{defn}[thm]{Definition}

\newtheorem{remark}[thm]{Remark}

\newtheorem{ex}[thm]{Example}



%\countstyle[equation]{thm}



%--------------------------------------------------

%       Item references.

%--------------------------------------------------


\newcommand{\exref}[1]{Example-\ref{#1}}

\newcommand{\thmref}[1]{Theorem-\ref{#1}}

\newcommand{\defref}[1]{Definition-\ref{#1}}

\newcommand{\eqnref}[1]{(\ref{#1})}

\newcommand{\secref}[1]{Section-\ref{#1}}

\newcommand{\lemref}[1]{Lemma-\ref{#1}}

\newcommand{\propref}[1]{Prop\-o\-si\-tion-\ref{#1}}

\newcommand{\corref}[1]{Cor\-ol\-lary-\ref{#1}}

\newcommand{\figref}[1]{Fig\-ure-\ref{#1}}

\newcommand{\conjref}[1]{Conjecture-\ref{#1}}


% Normal subgroup or equal.

\providecommand{\normaleq}{\unlhd}

% Normal subgroup.

\providecommand{\normal}{\lhd}

\providecommand{\rnormal}{\rhd}
% Divides, does not divide.

\providecommand{\divides}{\mid}

\providecommand{\ndivides}{\nmid}


\providecommand{\union}{\cup}

\providecommand{\bigunion}{\bigcup}

\providecommand{\intersect}{\cap}

\providecommand{\bigintersect}{\bigcap}
\begin{document}
\begin{thm}[Freshman's dream]
If $k$ if a field of \PMlinkname{characteristic}{Characteristic} $p>0$ (so $p$ is prime) 
then for all $x,y\in k$ we have
\[(x+y)^{p^i}=x^{p^i}+y^{p^i}.\]
Therefore $x\mapsto x^{p^i}$ is a field monomorphism (called a Frobenius monomorphism.)  
\end{thm}

When $k$ is finite then it is indeed an automorphism.
A field $k$ is called a perfect field when the map is surjective.

The theorem is so named because it is a common mistake for freshman math students 
to make over the real numbers.  However, as the characteristic of the real numbers 
is 0, this does not apply in any interesting way to that setting. 

It should also be noted that the result applies only to powers of the characteristic, 
and not all exponents.


\begin{proof}
The proof is an application of the binomial theorem.  We prove it for $p$ first.
\[(x+y)^{p}=\sum_{i=0}^p \binom{p}{i} x^{i} y^{p-i}.\]
Now observe
\[\binom{p}{i}=\frac{p!}{(p-i)!i!}=p\!\cdot\!\frac{(p-1)!}{(p-i)! i!}.\]
As $p$ is prime and $1\leq i\leq p-1$ it follows $i!$ and $(p-i)!$ do not divide $p$.  
As the field $k$ has characteristic $p$, $\frac{(p-1)!}{(p-i)!i!}$ is an integer $m$ where
\[\binom{p}{i}=pm\equiv 0 .\]
Thus $(x+y)^p=x^p+y^p$.

Now for $p^i$ simply use induction:
\[(x+y)^{p^i}=((x+y)^p)^{p^{i-1}}=(x^p+y^p)^{p^{i-1}}
=x^{p^i}+y^{p^i}.\]
\end{proof}
%%%%%
%%%%%
\end{document}
