\documentclass[12pt]{article}
\usepackage{pmmeta}
\pmcanonicalname{SumproductTheorem}
\pmcreated{2013-03-22 16:54:48}
\pmmodified{2013-03-22 16:54:48}
\pmowner{Algeboy}{12884}
\pmmodifier{Algeboy}{12884}
\pmtitle{sum-product theorem}
\pmrecord{8}{39174}
\pmprivacy{1}
\pmauthor{Algeboy}{12884}
\pmtype{Theorem}
\pmcomment{trigger rebuild}
\pmclassification{msc}{11T99}
\pmclassification{msc}{05B25}
\pmsynonym{Sum-product estimate}{SumproductTheorem}

\endmetadata

\usepackage{latexsym}
\usepackage{amssymb}
\usepackage{amsmath}
\usepackage{amsfonts}
\usepackage{amsthm}

%%\usepackage{xypic}

%-----------------------------------------------------

%       Standard theoremlike environments.

%       Stolen directly from AMSLaTeX sample

%-----------------------------------------------------

%% \theoremstyle{plain} %% This is the default

\newtheorem{thm}{Theorem}

\newtheorem{coro}[thm]{Corollary}

\newtheorem{lem}[thm]{Lemma}

\newtheorem{lemma}[thm]{Lemma}

\newtheorem{prop}[thm]{Proposition}

\newtheorem{conjecture}[thm]{Conjecture}

\newtheorem{conj}[thm]{Conjecture}

\newtheorem{defn}[thm]{Definition}

\newtheorem{remark}[thm]{Remark}

\newtheorem{ex}[thm]{Example}



%\countstyle[equation]{thm}



%--------------------------------------------------

%       Item references.

%--------------------------------------------------


\newcommand{\exref}[1]{Example-\ref{#1}}

\newcommand{\thmref}[1]{Theorem-\ref{#1}}

\newcommand{\defref}[1]{Definition-\ref{#1}}

\newcommand{\eqnref}[1]{(\ref{#1})}

\newcommand{\secref}[1]{Section-\ref{#1}}

\newcommand{\lemref}[1]{Lemma-\ref{#1}}

\newcommand{\propref}[1]{Prop\-o\-si\-tion-\ref{#1}}

\newcommand{\corref}[1]{Cor\-ol\-lary-\ref{#1}}

\newcommand{\figref}[1]{Fig\-ure-\ref{#1}}

\newcommand{\conjref}[1]{Conjecture-\ref{#1}}


% Normal subgroup or equal.

\providecommand{\normaleq}{\unlhd}

% Normal subgroup.

\providecommand{\normal}{\lhd}

\providecommand{\rnormal}{\rhd}
% Divides, does not divide.

\providecommand{\divides}{\mid}

\providecommand{\ndivides}{\nmid}


\providecommand{\union}{\cup}

\providecommand{\bigunion}{\bigcup}

\providecommand{\intersect}{\cap}

\providecommand{\bigintersect}{\bigcap}










\begin{document}
Suppose $\mathbb{F}$ is a finite field.  Then given a subset $A$ of $\mathbb{F}$
we define the \emph{sum} of $A$ to be the set
   \[A+A=\{a+b:a,b\in A\}\]
and the product to be the set
   \[A\cdot A=\{a\cdot b:a,b\in A\}.\]
We concern ourselves with estimating the size of $A+A$ and $A\cdot A$ relative to the size of $A$, and ultimately also to the size of $\mathbb{F}$.

If $A$ is empty then $A+A$ is empty as is $A\cdot A$ and so 
$|A|=|A+A|=|A\cdot A|$.  Now suppose $A$ is non-empty then let $a\in A$.
Then
   \[a+A=\{a+b:b\in A\}\subset A+A\]
so $|A|\leq |A+A|$.  If $A=\{0\}$ then $A\cdot A=A$ so finally assume
$a\in A$, $a\neq 0$.  Then we have
   \[a\cdot A=\{a\cdot b:b\in A\}\subset A\cdot A\]
so in any case it always follows that 
\begin{equation}\label{eq:lower}
   |A|\leq |A+A|, |A\cdot A|.
\end{equation}

Now if $\mathbb{F}$ has a proper subfield $\mathbb{F}_0$ -- for instance
$\mathbb{F}=GF(p^2)$ and $\mathbb{F}_0=GF(P)$ -- then setting $A=\mathbb{F}_0$
makes $A=A+A=A\cdot A$ and so in this situation the bound in (\ref{eq:lower})
is tight, that is, $|A|=|A+A|=|A\cdot A|$.   So we insist now that $\mathbb{F}$
is a prime field, so it has no proper subfields.

We would like to understand what size $A$ must have to ensure that either
$A+A$ or $A\cdot A$ is larger than $A$.  (Note this is not the same as asking
if $A\neq A+A$ or $A\cdot A$ as we are concerned only with growth in size not the change in the elements of the set.)  Clearly $A=\{0\}$ fails, as does
$A=\mathbb{F}$ and with some intuition as guidance it is safe to presume that
$A$ must be large enough to have enough elements to produce many elements as a sum or product but also small enough that these these new elements outgrow
the size of $A$.  This is the content of the following important result.

\begin{thm}[Sum-Product estimate:Bourgain-Katz-Tao (2003)]
Let $\mathbb{F}=\mathbb{Z}_p$ be the field of prime order $p$.  Let $A$
be any subset of $\mathbb{F}$ such that
    \[|\mathbb{F}|^{\delta} < |A| < |\mathbb{F}|^{1-\delta}\]
for some $\delta>0$.  Then 
    \[\max\{|A+A|,|A\cdot A|\}\geq C |A|^{1+\varepsilon}\]
for some $\varepsilon>0$ which depends on $\delta$ and some constant
$C$ which also depends on $\delta$.
\end{thm}

The proof is non-trivial.  Jean Bourgain was awarded the Fields' medal in 1994,
Terence Tao in 2006 with the prize in part due to his various contributions in additive number theory.
\\
\\
\PMlinkexternal{
Bourgain, Katz, and Tao, \emph{A Sum-Product estimate in finite fields, and applications}, (preprint) arXiv:math/CO/0301343 v2, 2003.}{http://www.arxiv.org/abs/math/0301343}

%%%%%
%%%%%
\end{document}
