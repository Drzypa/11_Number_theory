\documentclass[12pt]{article}
\usepackage{pmmeta}
\pmcanonicalname{Pollardsrho}
\pmcreated{2013-03-22 12:34:58}
\pmmodified{2013-03-22 12:34:58}
\pmowner{yark}{2760}
\pmmodifier{yark}{2760}
\pmtitle{Pollard's $\rho$}
\pmrecord{12}{32832}
\pmprivacy{1}
\pmauthor{yark}{2760}
\pmtype{Algorithm}
\pmcomment{trigger rebuild}
\pmclassification{msc}{11A51}
\pmclassification{msc}{11Y05}
\pmsynonym{Pollard rho}{Pollardsrho}
\pmrelated{CryptographyAndNumberTheory}

\endmetadata

\usepackage{amssymb}
\usepackage{amsmath}
\usepackage{amsfonts}

\begin{document}
\PMlinkescapeword{difference}
\PMlinkescapeword{even}
\PMlinkescapeword{generate}
\PMlinkescapeword{length}
\PMlinkescapeword{loop}
\PMlinkescapeword{part}
\PMlinkescapeword{place}
\PMlinkescapeword{point}
\PMlinkescapeword{restrictions}
\PMlinkescapeword{time}

\emph{Pollard's $\rho$ method} is an algorithm for factoring numbers, better than trial and error for larger numbers. 

Let $n$ a non prime number and $d$ a non trivial factor.  The actual value of the factors are unknown at this stage, and Pollard's $\rho$ provides a way to find them. We do know, however, than $d$ is not larger than $n$. In fact, we know at least one of the factors must hold $d\leq \sqrt{n}$, so we assume this condition. 

So, how does this help?  If you start picking numbers at random
(keeping your numbers greater or equal to zero and strictly less
than $n$), then the only time you will get $a \equiv b \mod n$ is
when $a$ and $b$ are identical.  However, since $d$ is smaller than
$n$, there is a good chance that $a \equiv b \mod d$ sometimes when
$a \neq b$.

Well, if $a \equiv b \mod d$, that means that $(a-b)$ is a multiple of
$d$.  Since $n$ is also a multiple of $d$, the greatest common
divisor of $(a-b)$ and $n$ is a positive, integer multiple of $d$.
We can keep picking numbers randomly until the greatest common
divisor of $n$ and the difference of two of our random numbers is
greater than one.  Then, we can divide $n$ by whatever this greatest
common divisor turned out to be.  In doing so, we have broken down
$n$ into two factors.  If we suspect that the factors may be
composite, we can continue trying to break them down further by
doing the algorithm again on each half.

The amazing thing here is that through all of this, we just knew
there had to be some divisor of $n$.  We were able to use properties
of that divisor to our advantage \emph{before} we even knew what the
divisor was!

This is at the heart of Pollard's rho method.  Pick a random number
$a$.  Pick another random number $b$.  See if the greatest common
divisor of $(a-b)$ and $n$ is greater than one.  If not, pick
another random number $c$.  Now, check the greatest common divisor
of $(c-b)$ and $n$.  If that is not greater than one, check the
greatest common divisor of $(c-a)$ and $n$.  If that doesn't work,
pick another random number $d$.  Check $(d-c)$, $(d-b)$, and $(d-a)$.
Continue in this way until you find a factor.

As you can see from the above paragraph, this could get quite
cumbersome quite quickly.  By the $k$-th iteration, you will have
to do $(k-1)$ greatest common divisor checks.  Fortunately, there
is way around that.  By structuring the way in which you pick
``random'' numbers, you can avoid this buildup.

We use an appropiate polynomial $f(x)$ to generate pseudorandom numbers. Because we're only concerned with numbers from
zero up to (but not including) $n$, we will take all of the values
of $f(x)$ modulo $n$.  We start with some $x_1$.  We then pick our
numbers by taking $x_{k+1} = ( f(x_k) \mod n )$.

Now, say for example we get to some point $k$ where $x_k \equiv
x_j \mod d$ with $k < j$.  Then, because of the way that modulo
arithmetic works, $f(x_k)$ will be congruent to $f(x_j)$ modulo
$d$.  So, once we hit upon $x_k$ and $x_j$, then each element in
the sequence starting with $x_k$ will be congruent modulo $d$ to
the corresponding element in the sequence starting at $x_j$.  Thus,
once the sequence gets to $x_k$ it has looped back upon itself to
match up with $x_j$ (when considering them modulo $d$).

This looping is what gives the $\rho$ method its name.  If you go back
through (once you determine $d$) and look at the sequence of random
numbers that you used (looking at them modulo $d$), you will see
that they start off just going along by themselves for a bit.
Then, they start to come back upon themselves.  They don't typically
loop the whole way back to the first number of your sequence.  So,
they have a bit of a tail and a loop---just like the Greek letter
rho ($\rho$).

Before we see why that looping helps, we will first speak to why
it has to happen.  When we consider a number modulo $d$, we are
only considering the numbers greater than or equal to zero and
strictly less than $d$.  This is a very finite set of numbers.
Your random sequence cannot possibly go on for more than $d$ numbers
without having some number repeat modulo $d$.  And, if the function
$f(x)$ is well-chosen, you can probably loop back a great deal
sooner.

The looping helps because it means that we can get away without
accumulating the number of greatest common divisor steps we need
to perform with each new random number.  In fact, it makes it so
that we only need to do one greatest common divisor check for every
second random number that we pick.

Now, why is that? Let's assume that the loop is of length $t$ and
starts at the $j$-th random number.  Say that we are on the $k$-th
element of our random sequence.  Furthermore, say that $k$ is
greater than or equal to $j$ and $t$ divides $k$.  Because $k$ is
greater than $j$ we know it is inside the looping part of the
$\rho$.  We also know that if $t$ divides $k$, then $t$ also divides
$2k$.  What this means is that $x_{2k}$ and $x_k$ will be congruent
modulo $d$ because they correspond to the same point on the loop.
Because they are congruent modulo $d$, their difference is a multiple
of $d$.  So, if we check the greatest common divisor of $(x_k-x_{k/2})$
with $n$ every time we get to an even $k$, we will find some factor
of $n$ without having to do $k-1$ greatest common divisor calculations
every time we come up with a new random number.  Instead, we only
have to do one greatest common divisor calculation for every second
random number.

The only open question is what to use for a polynomial
$f(x)$ to get some random numbers which don't have too
many choices modulo $d$.  Since we don't usually know
much about $d$, we really can't tailor the polynomial
too much.  A typical choice of polynomial is 
$$f(x) = x^2 + a$$ where $a$ is
some constant which isn't congruent to $0$ or $-2$
modulo $n$.  If you don't place those restrictions on
$a$, then you will end up degenerating into the sequence
$\{ 1, 1, 1, 1, ... \}$ as soon as you hit upon some $x$
which is congruent to either $1$ or $-1$ modulo
$n$.

Let's use the algorithm now to factor our number $16843009$.  We
will use the sequence $x_1=1$ with $x_{n+1} = ( 1024x_n^2 + 32767
\mod n )$.  [ I also tried it with the very basic polynomial $f(x)
= x^2 + 1$, but that one went 80 rounds before stopping so I didn't
include the table here.]

\begin{center}
\begin{tabular}{ccc}
$k$ & $x_k$ & $\gcd( n, x_k - x_{k/2} )$ \\
\hline \\
1 & 1 & \\
2 & 33791 & 1 \\
3 & 10832340 & \\
4 & 12473782 & 1 \\
5 & 4239855 & \\
6 & 309274 & $n$ \\
7 & 11965503 & \\
8 & 15903688 & 1 \\
9 & 3345998 & \\
10 & 2476108 & $n$ \\
11 & 11948879 & \\
12 & 9350010 & 1 \\
13 & 4540646 & \\
14 & 858249 & $n$ \\
15 & 14246641 & \\
16 & 4073290 & $n$ \\
17 & 4451768 & \\
18 & 14770419 & 257 \\
\end{tabular}
\end{center}
and so we have discovered the factor $257$. 

Let's try to factor again with a different random number schema.
We will use the sequence $x_1=1$ with $x_{n+1} = ( 2048x_n^2 +
32767 \mod n )$.

\begin{center}
\begin{tabular}{ccc}
$k$ & $x_k$ & $\gcd( n, x_k - x_{k/2} )$ \\
\hline \\
1 & 1 & \\
2 & 34815 & 1 \\
3 & 9016138 & \\
4 & 4752700 & 1 \\
5 & 1678844 & \\
6 & 14535213 & 257 \\
\end{tabular}
\end{center}
Again, the factor $257$ shows up.

Pollard's $\rho$ can also be applied to other finite groups besides integers,  providing one of the best known methods to computing discrete logarithms on arbitrary groups.
%%%%%
%%%%%
\end{document}
