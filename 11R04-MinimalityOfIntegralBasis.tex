\documentclass[12pt]{article}
\usepackage{pmmeta}
\pmcanonicalname{MinimalityOfIntegralBasis}
\pmcreated{2013-03-22 15:20:38}
\pmmodified{2013-03-22 15:20:38}
\pmowner{Mathprof}{13753}
\pmmodifier{Mathprof}{13753}
\pmtitle{minimality of integral basis}
\pmrecord{9}{37164}
\pmprivacy{1}
\pmauthor{Mathprof}{13753}
\pmtype{Theorem}
\pmcomment{trigger rebuild}
\pmclassification{msc}{11R04}
\pmrelated{CanonicalBasis}
\pmrelated{PropertiesOfDiscriminantInAlgebraicNumberField}
\pmdefines{fundamental number}
\pmdefines{discriminant of field}

\endmetadata

% this is the default PlanetMath preamble.  as your knowledge
% of TeX increases, you will probably want to edit this, but
% it should be fine as is for beginners.

% almost certainly you want these
\usepackage{amssymb}
\usepackage{amsmath}
\usepackage{amsfonts}

% used for TeXing text within eps files
%\usepackage{psfrag}
% need this for including graphics (\includegraphics)
%\usepackage{graphicx}
% for neatly defining theorems and propositions
 \usepackage{amsthm}
% making logically defined graphics
%%%\usepackage{xypic}

% there are many more packages, add them here as you need them

% define commands here

\theoremstyle{definition}
\newtheorem*{thmplain}{Theorem}
\begin{document}
The discriminant\, $\Delta := \Delta(\alpha_1,\,\alpha_2,\,\ldots,\,\alpha_s)$\, of the set\, $\{\alpha_1,\,\alpha_2,\,\ldots,\,\alpha_s\}$\, of integers of an algebraic number field $K$ is a rational integer.\, If this set is an integral basis of $K$, then $|\Delta|$ has the least possible (positive integer) value in the field $K$, and conversely.\, The value\, $d = \Delta$\, is equal for all integral bases of $K$, and it is called the {\em discriminant} or {\em fundamental number} of the field.
%%%%%
%%%%%
\end{document}
