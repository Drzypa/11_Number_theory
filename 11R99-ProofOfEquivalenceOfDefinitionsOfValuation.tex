\documentclass[12pt]{article}
\usepackage{pmmeta}
\pmcanonicalname{ProofOfEquivalenceOfDefinitionsOfValuation}
\pmcreated{2013-03-22 14:56:00}
\pmmodified{2013-03-22 14:56:00}
\pmowner{rspuzio}{6075}
\pmmodifier{rspuzio}{6075}
\pmtitle{proof of equivalence of definitions of valuation}
\pmrecord{12}{36622}
\pmprivacy{1}
\pmauthor{rspuzio}{6075}
\pmtype{Proof}
\pmcomment{trigger rebuild}
\pmclassification{msc}{11R99}
\pmclassification{msc}{12J20}
\pmclassification{msc}{13A18}
\pmclassification{msc}{13F30}

\endmetadata

% this is the default PlanetMath preamble.  as your knowledge
% of TeX increases, you will probably want to edit this, but
% it should be fine as is for beginners.

% almost certainly you want these
\usepackage{amssymb}
\usepackage{amsmath}
\usepackage{amsfonts}

% used for TeXing text within eps files
%\usepackage{psfrag}
% need this for including graphics (\includegraphics)
%\usepackage{graphicx}
% for neatly defining theorems and propositions
%\usepackage{amsthm}
% making logically defined graphics
%%%\usepackage{xypic}

% there are many more packages, add them here as you need them

% define commands here
\begin{document}
We will start with a lemma:

\textbf{Lemma}  Let $| \cdot |$ be a valuation according to the definition of the parent entry with constant $C \le 2$.  Then $\left| \sum_{i=1}^n x_i \right| \le 2n \max_{i=1}^n |x_i|$.

\emph{Proof:} \, We will start with the case where $n$ is a power of two: $n = 2^r$  We shall prove that $\left| \sum_{i=1}^{2^r} x_i \right| \le 2^r \max_{i=1}^n |x_i|$ by induction.  The assertion is certainly true when $n = 2$ by hypothesis.  Assume that it also holds when $n = 2^{r-1}$.  Then we have
 $$\left| \sum_{i=1}^n x_i \right| = \left| \sum_{i=1}^{2^{r-1}} x_i + \sum_{i=2^{r-1}+1}^{2^r} x_i \right| \le 2 \cdot 2^{r-1} \max_{i=1}^{2^{r-1}} |x_i| + 2 \cdot 2^{r-1} \max_{i=2^{r-1}+1}^{2^r} |x_i| \le 2 \cdot 2^r \max_{i=1}^{2^r} |x_i|$$

To deal with the case where $n$ is not a power of two, we shall pad the sum with zeros.  That is to say, we shall define $x_i = 0$ when $i > n$.  Let $r$ be the greatest integer such that $2^r \le n$.  Then $2 n > 2^r$ and we have 
 $$\left| \sum_{i=1}^n x_i \right| = \left| \sum_{i=1}^{2^r} x_i \right| \le 2^r \max_{i=1}^{2^r} |x_i| \le 2n \max_{i=1}^n |x_i|$$
\rightline{Q.E.D.}

\textbf{Corollary}  Let $| \cdot |$ be a valuation according to the definition of the parent entry with constant $C \le 2$.  Then, if $n$ is a positive integer, $|n| < 2 n$.

\emph{Proof:} \, Write $n = \sum_{i=1} ^n 1$.  Then, by the lemma,
 $$|n| = \left| \sum_{i=1} ^n 1 \right| < 2n |1|$$
However, since $| \cdot |$ is a valuation $|1| \neq 0$ since $1 \neq 0$ and $|1| = |1| \cdot |1|$ so $|1| = 1$, hence
 $$|n| \le 2n.$$
\rightline{Q.E.D.}

Having established this lemma, we will now use it to prove the main theorem:

\textbf{Theorem}  If $| \cdot |$ is a valuation according to the definition of the parent entry with constant $C \le 2$, then $| \cdot |$ satisfies the identity
 $$| x + y | \le |x| + |y|.$$

\emph{Proof:}  \, Let $n$ be a positive integer.  Then we have
 $$|x + y|^n = |(x+y)^n| = \left| \sum_{m=0}^n {n \choose m} x^m y^{n-m} \right|$$
Using the lemma, we can bound this:
 $$|x + y|^n \le 2 (n+1) \max_{m=0}^n \left| {n \choose m} x^m y^{n-m} \right|$$
Using the corollary to the lemma, we can bound the binomial coefficient to obtain
 $$|x + y|^n \le 4 (n+1) \max_{m=0}^n {n \choose m} |x|^m |y|^{n-m}$$
Using the obvious inequality
 $$\max_{m=0}^n {n \choose m} |x|^m |y|^{n-m} \le \sum_{m=0}^n {n \choose m} |x|^m |y|^{n-m},$$
we obtain
 $$|x + y|^n \le \sum_{m=0}^n 4 (n+1) {n \choose m} |x|^m |y|^{n-m} = 4 (n+1) (|x| + |y|)^n.$$

If either $x = 0$ or $y = 0$, then the theorem to be proven is trivial.  If not, then $|x| + |y| \neq 0$ and we can divide by $(|x| + |y|)^n$ to obtain
 $$\left( {|x+y| \over |x| + |y|} \right)^n \le 4 (n+1)$$

By the theorem on the growth of exponential function, it follows that this inequality could not hold for all $n$ unless
 $${|x+y| \over |x| + |y|} \le 1,$$
in other word, unless $|x + y| \le |x| + |y|$.
\rightline{Q.E.D.}
%%%%%
%%%%%
\end{document}
