\documentclass[12pt]{article}
\usepackage{pmmeta}
\pmcanonicalname{LogarithmicDensity}
\pmcreated{2013-03-22 15:31:54}
\pmmodified{2013-03-22 15:31:54}
\pmowner{kompik}{10588}
\pmmodifier{kompik}{10588}
\pmtitle{logarithmic density}
\pmrecord{6}{37421}
\pmprivacy{1}
\pmauthor{kompik}{10588}
\pmtype{Definition}
\pmcomment{trigger rebuild}
\pmclassification{msc}{11B05}
\pmrelated{InequalityOfLogarithmicAndAsymptoticDensity}
\pmdefines{upper logarithmic density}
\pmdefines{lower logarithmic density}

% this is the default PlanetMath preamble. as your knowledge
% of TeX increases, you will probably want to edit this, but
% it should be fine as is for beginners.

% almost certainly you want these
\usepackage{amssymb}
\usepackage{amsmath}
\usepackage{amsfonts}
\usepackage{amsthm}

% used for TeXing text within eps files
%\usepackage{psfrag}
% need this for including graphics (\includegraphics)
%\usepackage{graphicx}
% for neatly defining theorems and propositions
%
% making logically defined graphics
%%%\usepackage{xypic}

% there are many more packages, add them here as you need them

% define commands here

\newcommand{\sR}[0]{\mathbb{R}}
\newcommand{\sC}[0]{\mathbb{C}}
\newcommand{\sN}[0]{\mathbb{N}}
\newcommand{\sZ}[0]{\mathbb{Z}}

\usepackage{bbm}
\newcommand{\Z}{\mathbbmss{Z}}
\newcommand{\C}{\mathbbmss{C}}
\newcommand{\R}{\mathbbmss{R}}
\newcommand{\Q}{\mathbbmss{Q}}
\newcommand{\N}[0]{\mathbb{N}}

\newcommand*{\abs}[1]{| #1 |}

\newcommand{\Map}[3]{#1:#2\to#3}
\newcommand{\Emb}[3]{#1:#2\hookrightarrow#3}
\newcommand{\Mor}[3]{#2\overset{#1}\to#3}

\newcommand{\Cat}[1]{\mathcal{#1}}
\newcommand{\Kat}[1]{\mathbf{#1}}
\newcommand{\Func}[3]{\Map{#1}{\Cat{#2}}{\Cat{#3}}}
\newcommand{\Funk}[3]{\Map{#1}{\Kat{#2}}{\Kat{#3}}}

\newcommand{\vp}{\varphi}
\newcommand{\ve}{\varepsilon}

\newcommand{\Invimg}[2]{\inv{#1}(#2)}
\newcommand{\Img}[2]{#1[#2]}
\newcommand{\ol}[1]{\overline{#1}}
\newcommand{\ul}[1]{\underline{#1}}
\newcommand{\inv}[1]{#1^{-1}}
\newcommand{\limti}[1]{\lim\limits_{#1\to\infty}}
\begin{document}
For any $A\subseteq\N$ we denote $S(n):=\sum\limits_{k=1}^n \frac 1k$. The values
$$\ol \delta(A)=\liminf_{n\to\infty} \frac{\sum\limits_{k\in A; k\leq n} \frac 1k}{S(n)} \qquad \ul \delta(A) = \limsup_{n\to\infty}
\frac{\sum\limits_{k\in A; k\leq n} \frac 1k}{S(n)}$$ are called
\emph{lower} and \emph{upper logarithmic density} of $A$. If
$\ol\delta(A)=\ul\delta(A)$ we denote this value by $\delta(A)$
and call it the \emph{logarithmic density} of $A$.

Logarithmic density can be equivalently defined as follows: If the limit
$$\delta(A)=\limti n \frac{\sum\limits_{k\in A; k\leq n} \frac 1k}{S(n)},$$
exists, then it is called logarithmic density of $A$.

By the well-known \PMlinkescapetext{formula} $\gamma = \limti n S(n) - \ln n$
defining Euler's constant, we can see that the denominator in the
above definitions can be replaced by $\ln n$.

\begin{thebibliography}{1}

\bibitem{kolibiar}
M.~Kolibiar, A.~Leg\'e\v{n}, T.~\v{S}al\'at, and \v{S}. Zn\'am.
\newblock {\em Algebra a pr\'\i buzn\'e discipl\'\i ny}.
\newblock Alfa, Bratislava, 1992. (in Slovak)

\bibitem{ostmann}
H.~H. Ostmann.
\newblock {\em Additive {Z}ahlentheorie {I}}.
\newblock Springer-Verlag, Berlin-G\"ottingen-Heidelberg, 1956.

\bibitem{steuding}
J.~Steuding.
\newblock \PMlinkexternal{Probabilistic number theory}{http://www.math.uni-frankfurt.de/~steuding/steuding/prob.pdf}.

\bibitem{tenenbaum}
G.~Tenenbaum.
\newblock {\em {Introduction to analytic and probabilistic number theory}}.
\newblock {Cambridge Univ. Press}, Cambridge, 1995.

\end{thebibliography}
%%%%%
%%%%%
\end{document}
