\documentclass[12pt]{article}
\usepackage{pmmeta}
\pmcanonicalname{ProthPrime}
\pmcreated{2013-03-22 17:21:11}
\pmmodified{2013-03-22 17:21:11}
\pmowner{PrimeFan}{13766}
\pmmodifier{PrimeFan}{13766}
\pmtitle{Proth prime}
\pmrecord{7}{39711}
\pmprivacy{1}
\pmauthor{PrimeFan}{13766}
\pmtype{Definition}
\pmcomment{trigger rebuild}
\pmclassification{msc}{11A51}

% this is the default PlanetMath preamble.  as your knowledge
% of TeX increases, you will probably want to edit this, but
% it should be fine as is for beginners.

% almost certainly you want these
\usepackage{amssymb}
\usepackage{amsmath}
\usepackage{amsfonts}

% used for TeXing text within eps files
%\usepackage{psfrag}
% need this for including graphics (\includegraphics)
%\usepackage{graphicx}
% for neatly defining theorems and propositions
%\usepackage{amsthm}
% making logically defined graphics
%%%\usepackage{xypic}

% there are many more packages, add them here as you need them

% define commands here

\begin{document}
A {\em Proth prime} is a Proth number that is also a prime number. Given a Proth number $p$, if one can find a coprime integer $b$ such that $$b^{\frac{p - 1}{2}} \equiv -1 \mod p$$ then $p$ is a prime, and specifically a Proth prime (this is a theorem first stated by Fran\c{c}ois Proth). Thanks to this theorem, Yves Gallot created an algorithm used in a primality-testing program employed by the Seventeen or Bust project. The first few Proth primes are 3, 5, 13, 17, 41, 97, 113, 193, 241, 257, 353, 449, 577, 641, 673, 769, 929, etc. (listed in A080076 of Sloane's OEIS). Konstantin Agafonov's discovery of the Proth prime $19249 \times 2^{13018586} + 1 \approx 1.484360328715661 \times 10^{3918989}$ currently makes for the largest known prime that is not a Mersenne prime.
%%%%%
%%%%%
\end{document}
