\documentclass[12pt]{article}
\usepackage{pmmeta}
\pmcanonicalname{ExamplesOfSmoothNumbers}
\pmcreated{2013-03-22 18:10:25}
\pmmodified{2013-03-22 18:10:25}
\pmowner{PrimeFan}{13766}
\pmmodifier{PrimeFan}{13766}
\pmtitle{examples of smooth numbers}
\pmrecord{4}{40738}
\pmprivacy{1}
\pmauthor{PrimeFan}{13766}
\pmtype{Example}
\pmcomment{trigger rebuild}
\pmclassification{msc}{11A05}

% this is the default PlanetMath preamble.  as your knowledge
% of TeX increases, you will probably want to edit this, but
% it should be fine as is for beginners.

% almost certainly you want these
\usepackage{amssymb}
\usepackage{amsmath}
\usepackage{amsfonts}

% used for TeXing text within eps files
%\usepackage{psfrag}
% need this for including graphics (\includegraphics)
%\usepackage{graphicx}
% for neatly defining theorems and propositions
%\usepackage{amsthm}
% making logically defined graphics
%%%\usepackage{xypic}

% there are many more packages, add them here as you need them

% define commands here

\begin{document}
If a number is $k$-smooth, it is also a smooth number for smaller $k$. So, to give just a handful of examples of smooth numbers for a few different values of $k$, the following table only goes up to 100, and for $k > 2$, only those numbers not already listed for smaller $k$ are given. Also, the $k$ have been limited to prime numbers. Obviously the 2-smooth numbers are the powers of two.

\begin{tabular}{|r|l|}
2 & 1, 2, 4, 8, 16, 32, 64 \\
3 & 3, 6, 9, 12, 18, 24, 27, 36, 48, 54, 64, 72, 81, 96 \\
5 & 5, 10, 15, 20, 25, 30, 40, 45, 50, 60, 75, 80, 90, 100 \\
7 & 7, 14, 21, 28, 35, 42, 49, 56, 63, 70, 84, 98 \\
11 & 11, 22, 33, 44, 55, 66, 77, 88, 99 \\
13 & 13, 26, 39, 52, 65, 78, 91 \\
17 & 17, 34, 51, 68, 85 \\
19 & 19, 38, 57, 57, 76, 95 \\
\end{tabular}
%%%%%
%%%%%
\end{document}
