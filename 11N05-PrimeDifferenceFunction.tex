\documentclass[12pt]{article}
\usepackage{pmmeta}
\pmcanonicalname{PrimeDifferenceFunction}
\pmcreated{2013-03-22 12:49:16}
\pmmodified{2013-03-22 12:49:16}
\pmowner{XJamRastafire}{349}
\pmmodifier{XJamRastafire}{349}
\pmtitle{prime difference function}
\pmrecord{6}{33143}
\pmprivacy{1}
\pmauthor{XJamRastafire}{349}
\pmtype{Definition}
\pmcomment{trigger rebuild}
\pmclassification{msc}{11N05}
\pmclassification{msc}{11A25}
\pmclassification{msc}{11A41}
%\pmkeywords{number theory}
%\pmkeywords{arithmetic function}

\endmetadata

% this is the default PlanetMath preamble.  as your knowledge
% of TeX increases, you will probably want to edit this, but
% it should be fine as is for beginners.

% almost certainly you want these
\usepackage{amssymb}
\usepackage{amsmath}
\usepackage{amsfonts}

% used for TeXing text within eps files
%\usepackage{psfrag}
% need this for including graphics (\includegraphics)
%\usepackage{graphicx}
% for neatly defining theorems and propositions
%\usepackage{amsthm}
% making logically defined graphics
%%%\usepackage{xypic}

% there are many more packages, add them here as you need them

% define commands here
\begin{document}
The {\it prime difference function} is an arithmetic function for any positive integer $n$, denoted as $d_{n}$ and gives the difference between two consecutive primes $p_{n}$ and $p_{n+1}$:

$$ d_{n} \equiv p_{n+1} - p_{n} \; . $$

For example:
\begin{itemize}
\item $d_{1} = p_{2} - p_{1} = 3 - 2 = 1$, 
\item $d_{10} = p_{11} - p_{10} = 31 - 29 = 2$, 
\item $d_{100} = p_{101} - p_{100} = 547 - 541 = 6$, 
\item $d_{1000} = p_{1001} - p_{1000} = 7927 - 7919 = 8$, 
\item $d_{10000} = p_{10001} - p_{10000} = 104743 - 104729 = 14$ and so forth.
\end{itemize}

The first few values of $d_{n}$ for $n = 1, 2, 3, \ldots $ are 
$1, 2, 2, 4, 2, 4, 2, 4, 6, 2, 6, 4, 2, 4, 6, 6, 2, 6, 4, 2, \ldots $
(\PMlinkexternal{OEIS A001223}{http://www.research.att.com/~njas/sequences/eisA.cgi?Anum=001223}).

%%%%%
%%%%%
\end{document}
