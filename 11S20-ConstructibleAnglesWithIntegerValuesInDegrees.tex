\documentclass[12pt]{article}
\usepackage{pmmeta}
\pmcanonicalname{ConstructibleAnglesWithIntegerValuesInDegrees}
\pmcreated{2013-03-22 14:16:36}
\pmmodified{2013-03-22 14:16:36}
\pmowner{PrimeFan}{13766}
\pmmodifier{PrimeFan}{13766}
\pmtitle{constructible angles with integer values in degrees}
\pmrecord{9}{35728}
\pmprivacy{1}
\pmauthor{PrimeFan}{13766}
\pmtype{Theorem}
\pmcomment{trigger rebuild}
\pmclassification{msc}{11S20}
\pmclassification{msc}{11R32}
\pmclassification{msc}{51M15}
\pmclassification{msc}{13B05}
\pmsynonym{constructible angle}{ConstructibleAnglesWithIntegerValuesInDegrees}
\pmrelated{ExactTrigonometryTables}
\pmrelated{TheoremOnConstructibleAngles}
\pmrelated{ClassicalProblemsOfConstructibility}

\usepackage{graphicx}
%%%\usepackage{xypic} 
\usepackage{bbm}
\newcommand{\Z}{\mathbbmss{Z}}
\newcommand{\C}{\mathbbmss{C}}
\newcommand{\R}{\mathbbmss{R}}
\newcommand{\Q}{\mathbbmss{Q}}
\newcommand{\mathbb}[1]{\mathbbmss{#1}}
\newcommand{\figura}[1]{\begin{center}\includegraphics{#1}\end{center}}
\newcommand{\figuraex}[2]{\begin{center}\includegraphics[#2]{#1}\end{center}}
\newtheorem{lem}{Lemma}
\newtheorem{thm}{Theorem}
\begin{document}
The aim is to characterize all constructible angles with straightedge and compass whose value is an integer number of degrees (like $60\sp\circ$ or $36\sp\circ$).
From now on, every time we refer to the measurement of an angle, it is meant to be in degrees, not radians.

We need two short lemmas:

\begin{lem}
If an angle measuring $x$ degrees can be constructed, then angles measuring
\[\frac x2,\frac x4,\frac x8,\ldots,\frac{x}{2\sp k}\]
can be constructed.
\end{lem}

Notice that we are not stating all of them have integer values, only constructibility. The proof follows almost inmediately by knowing any angle can be bisected with ruler and compass.

\begin{lem}
If an angle measuring $x$ degrees can be constructed, then angles measuring any integer multiple of $x$, that is, $2x,3x,4x,\ldots$ can be constructed
\end{lem}

If you can construct $x$, you can construct again an adjacent angle with the same value and you will have constructed an angle measuring $2x$. Repeat the procedure and you get $3x,4x,\ldots$.

\bigskip

Now, a theorem.

\begin{thm}
The angle measuring $3\sp\circ$ can be constructed.
\end{thm}

It is well known that both regular pentagon and equilateral triangle can be built with ruler and compass. That allows us to construct angles measuring $72\sp\circ$ and $60\sp\circ$.


By first lemma we can construct then 
\[72\sp\circ, \frac{72\sp\circ}{2}=36\sp\circ, \frac{36\sp\circ}{2}=18\sp\circ,
\frac{18\sp\circ}{2}=9\sp\circ,\
\frac{9\sp\circ}{2}=4.5\sp\circ=4\sp\circ\, 30'
\]
and also we can construct
\[
60\sp\circ,\frac{60\sp\circ}{2}=30\sp\circ,
\frac{30\sp\circ}{2}=15\sp\circ,
\frac{15\sp\circ}{2}=7.5\sp\circ = 7\sp\circ\, 30'
\]

But if we can construct $4\sp\circ\ 30'$ and $7\sp\circ\ 30'$ we can then construct their difference, which is exactly $3\sp\circ$.


Alternative (J. Pahikkala): Since $72\sp\circ$ and $60\sp\circ$ can be constructed, $12\sp\circ=72\sp\circ-60\sp\circ$ can be also constructed. Bisecting $12\sp\circ$ gives $6\sp\circ$ and bisecting again shows that $3\sp\circ$ can be constructed.

\begin{thm}
We can construct any angle measuring an integer multiple of $3\sp\circ$.
\end{thm}

The proof follows directly from the second Lemma.

\begin{thm}
The only constructible angles measuring an integer number of degrees are precisely the multiples of $3\sp\circ$.
\end{thm}

We are only left to prove we cannot construct any other integer value. We will work by contradiction.

Suppose we are able to construct with ruler and compass an angle measuring $t\sp\circ$ with $t$ integer and $t$ not multiple of $3$.

Since $3$ does not divide $t$ and $3$ is prime, it follows that $3$ and $t$ are coprime, that is, $ \gcd(3,t)=1$.

But then, by Euclid's algorithm we can find integers $m,n$ so that
$3m-tn=1$ ($n$ or $m$ could be negative).

By the second lemma, we can construct both $3m\sp\circ$ and $tn\sp\circ$, so we can construct their sum (or difference), which would prove $1\sp\circ$ is constructible, and therefore any angle equal to an integer number of degrees could be constructed with ruler and compass.

However, the standard proof of the impossibility of trisecting an arbitrary angle goes by proving $20\sp\circ$ cannot be constructed with ruler and compass, this contradicts what we just showed, and therefore only angles being an integer multiple of $3\sp\circ$ can be constructed.

Q.E.D.

For a more general proof for other real values besides integers, see the theorem on constructible angles.
%%%%%
%%%%%
\end{document}
