\documentclass[12pt]{article}
\usepackage{pmmeta}
\pmcanonicalname{ThereAreAnInfiniteNumberOfPrimesequiv1modM}
\pmcreated{2013-03-22 17:43:02}
\pmmodified{2013-03-22 17:43:02}
\pmowner{rm50}{10146}
\pmmodifier{rm50}{10146}
\pmtitle{there are an infinite number of primes $\equiv 1\mod m$}
\pmrecord{7}{40162}
\pmprivacy{1}
\pmauthor{rm50}{10146}
\pmtype{Theorem}
\pmcomment{trigger rebuild}
\pmclassification{msc}{11N13}
\pmrelated{SpecialCaseOfDirichletsTheoremOnPrimesInArithmeticProgressions}

\endmetadata

% this is the default PlanetMath preamble.  as your knowledge
% of TeX increases, you will probably want to edit this, but
% it should be fine as is for beginners.

% almost certainly you want these
\usepackage{amssymb}
\usepackage{amsmath}
\usepackage{amsfonts}

% used for TeXing text within eps files
%\usepackage{psfrag}
% need this for including graphics (\includegraphics)
%\usepackage{graphicx}
% for neatly defining theorems and propositions
%\usepackage{amsthm}
% making logically defined graphics
%%%\usepackage{xypic}

% there are many more packages, add them here as you need them

% define commands here
\newcommand{\Ints}{\mathbb{Z}}
\begin{document}
This article proves a special case of Dirichlet's theorem, namely that for any integer $m>1$, there are an infinite number of primes $p\equiv 1\pmod m$. 

Let $p$ be an odd prime not dividing $m$, let $\Phi_k(x)$ be the $k^{\mathrm{th}}$ cyclotomic polynomial, and note that
\[x^m-1=\Phi_m(x)\cdot\prod_{\substack{d\mid m\\d<m}} \Phi_d(x)\]
If $a\in\Ints$ with $p\mid\Phi_m(a)$, then clearly $p\mid a^m-1$ and thus $\gcd(a,p)=1$. In fact, the \PMlinkname{order}{OrderGroup} of $a\mod p$ is precisely $m$, for if it were not, say $a^d\equiv 1\pmod p$ for $d<m$, then $a$ would be a root $\mod p$ of $\Phi_d(x)$ and thus $x^m-1$ would have multiple roots $\mod p$, which is a contradiction. But then, by Fermat's little theorem, we have $a^{p-1}\equiv 1\pmod p$, so since $m$ is the least integer with this property, we have $m\mid p-1$ so that $p\equiv 1\pmod m$.

We have thus shown that if $p\nmid m$ and $p\mid \Phi_m(a)$, then $p\equiv 1\pmod m$. The result then follows from the following claim: if $f(x)\in \Ints[x]$ is any polynomial of degree at least one, then the factorizations of
\[f(1),f(2),f(3),\ldots\]
contain infinitely many primes. The proof is \PMlinkescapetext{similar} to Euclid's proof of the infinitude of primes. Assume not, and let $p_1,\ldots,p_k$ be all of the primes. Since $f$ is nonconstant, choose $n$ with $f(n)=a\neq 0$. Then $f(n+ap_1\cdot p_2\cdots p_kx)$ is clearly divisible by $a$, so $g(x)=a^{-1}f(n+ap_1\cdot p_2\cdots p_kx)\in\Ints[x]$, and $g(m)\equiv 1\pmod {p_1\cdots p_k}$ for each $m\in\Ints$. $g$ is nonconstant, so choose $m$ such that $g(m)\neq 1$. Then $g(m)$ is clearly divisible by some prime other that the $p_i$ and thus $f(n+ap_1\cdot p_2\cdots p_kx)$ is as well. Contradiction.

Thus the set $\Phi_m(1),\Phi_m(2),\ldots$ contains an infinite number of primes in their factorizations, only a finite number of which can divide $m$. The remainder must be primes $p\equiv 1\pmod m$.
%%%%%
%%%%%
\end{document}
