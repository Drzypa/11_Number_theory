\documentclass[12pt]{article}
\usepackage{pmmeta}
\pmcanonicalname{DeterminationOfEvenAbundantNumbersWithOneOddPrimeFactor}
\pmcreated{2013-03-22 16:47:51}
\pmmodified{2013-03-22 16:47:51}
\pmowner{rspuzio}{6075}
\pmmodifier{rspuzio}{6075}
\pmtitle{determination of even abundant numbers with one odd prime factor}
\pmrecord{24}{39031}
\pmprivacy{1}
\pmauthor{rspuzio}{6075}
\pmtype{Definition}
\pmcomment{trigger rebuild}
\pmclassification{msc}{11A05}

% this is the default PlanetMath preamble.  as your knowledge
% of TeX increases, you will probably want to edit this, but
% it should be fine as is for beginners.

% almost certainly you want these
\usepackage{amssymb}
\usepackage{amsmath}
\usepackage{amsfonts}

% used for TeXing text within eps files
%\usepackage{psfrag}
% need this for including graphics (\includegraphics)
%\usepackage{graphicx}
% for neatly defining theorems and propositions
\usepackage{amsthm}
% making logically defined graphics
%%%\usepackage{xypic}

% there are many more packages, add them here as you need them

% define commands here

\newtheorem{thm}{Theorem}
\begin{document}
In this entry, we will use the criterion of the parent entry to determine
the first few even abundant numbers.  To keep things more managable, 
we shall take advantage of the fact that a multiple of an abundant number 
is abundant and only look for abundant numbers none of whose proper divisors 
are abundant.  Once we know these numbers, it becomes a rather easy matter
to find the rest of the abundant numbers by taking multiples.

To begin, we look at the criterion of the second thorem.  Since $2 / (2 - 1) = 2$
and, for any $p > 2$, we have $1 < p / (p - 1) < 2$, it follows that, for
every prime $p$, there will exist abundant numbers of the form $2^m p^n$.
By the first theorem, for such a number to be abundant, we must have
\[
{(1 - 2^{-m-1}) (1 - p^{-n-1}) \over \frac{1}{2} (1 - p^{-1})} > 2
\]
or, after a little algebraic simplification,
\[
(1 - 2^{-m-1}) (1 - p^{-n-1}) > 1 - p^{-1}.
\]
From this inequality, we can deduce a description of abundant numbers 
of the form $2^m p^n$.

\begin{thm}
Let $p>2$ be prime.  Then, for all $n \ge 0$, we find that $2^m p^n$ 
is abundant for $m$ sufficiently large.
\end{thm}

\begin{proof}
When $n \ge 0$, we have
\[
1 - p^{-n-1} > 1 - p^{-1}.
\]
Since $\lim_{m \to \infty} 2^{-m-1} = 0$, it follows that 
\[
(1 - 2^{-m-1}) (1 - p^{-n-1}) > 1 - p^{-1}.
\]
for $m$ sufficiently large, hence $2^m p^n$ 
will be abundant for $m$ sufficiently large.
\end{proof}

\begin{thm}
Let $p>2$ be prime.  Then there exists $m_0$ such that:
\begin{enumerate}
\item If $m < m_0$, then $2^m p^n$ is not abundant for any $n$.
\item if $m \ge m_0$, then $2^m p^n$ is abundant for $n$ 
sufficiently large.
\end{enumerate}
\end{thm}

\begin{proof}
Set $m_0$ to be the smallest integer such that $2^{m_0 + 1} > p$.
This inequality is equivalent to
\[
1 - 2^{-m_0 - 1} > 1 - p^{-1}.
\]
On the one hand, if $m < m_0$, then it will be impossible to satisfy 
our criterion for any choice of $n$.  On the other hand, if $m \ge m_0$,
then, by ther same sort of continuity argument employed previously, 
the criterion will be satisfied for $m$ sufficiently large.
\end{proof}

\begin{thm}
Let $p>2$ be prime and let $m_0$ be the unique integer such that
\[
2^{m_0 + 1} > p > 2^{m_0}.
\]
Then $2^{m_0} p$ is either perfect or abundant and every abundant
number of the form $2^m p^n$ is a multiple of $2^{m_0} p$.
\end{thm}

\begin{proof}
We begin with the equation
\[
p + 1 \le 2^{m_0 +1}.
\]
Making some algebraic manipulations, we obtain the following:
\begin{align*}
p &\le {2^{m_0} + 1 \over 1 + p^{-1}} \\
p^{-1} &\ge 2^{-m_0 - 1} (1 + p^{-1}) \\
p^{-1} - 2^{-m_0 - 1} (1 + p^{-1}) &\ge 0 \\
1 + p^{-1} - 2^{-m_0 - 1} (1 + p^{-1}) &\ge 1 \\
(1 - 2^{-m_0 - 1}) (1 + p^{-1}) &\ge 1 \\
(1 - 2^{-m_0 - 1}) (1 + p^{-2}) &\ge 1 - p^{-1}
\end{align*}
According to our earlier inequality, this means that
$2^{m_0} p$ is either perfect or abundant.  

Suppose that $2^m p^n$ is abundant.  Then, by the previous
result, $m \le m_0$; since powers of $2$ are deficient, $n > 0$,
so $2^{m_0} p \mid 2^m p^n$.
\end{proof}

This result makes it rather easy to draw up a list of abundant 
numbers with one odd prime factor none of whose proper factors 
are abundant starting with a table of prime numbers, as has
been done below.  Note that, in the case where $2^{m_0} p$ is
perfect, we have listed $2^{m_0 + 1} p$ and $2^{m_0} p^2$ as
these numbers are abundant.

\begin{tabular}
{| c | l |}
$3$ & $12$ $(2^2 \cdot 3)$, $18$ $(2 \cdot 3^2)$ \\
$5$ & $20$ $(2^2 \cdot 5)$ \\
$7$ & $56$ $(2^3 \cdot 7)$, $196$ $(2^2 \cdot 7^2)$ \\
$11$ & $88$ $(2^3 \cdot 11)$ \\
$13$ & $104$ $(2^3 \cdot 13)$ \\
$17$ & $272$ $(2^4 \cdot 17)$ \\
$19$ & $304$ $(2^4 \cdot 19)$ \\
$23$ & $368$ $(2^4 \cdot 23)$ \\
$29$ & $464$ $(2^4 \cdot 29)$ \\
$31$ & $992$ $(2^5 \cdot 31)$, $7688$ $(2^5 \cdot 31^2)$ \\
$37$ & $1184$ $(2^5 \cdot 37)$ \\
$41$ & $1312$ $(2^5 \cdot 41)$ \\
$43$ & $1376$ $(2^5 \cdot 43)$ \\
$47$ & $1504$ $(2^5 \cdot 47)$ \\
$53$ & $1696$ $(2^5 \cdot 53)$ \\
$59$ & $1888$ $(2^5 \cdot 59)$ \\
$61$ & $1952$ $(2^5 \cdot 61)$ \\
$67$ & $4208$ $(2^6 \cdot 67)$ \\
$71$ & $4544$ $(2^6 \cdot 71)$ \\
$73$ & $4672$ $(2^6 \cdot 73)$ \\
$79$ & $5056$ $(2^6 \cdot 79)$ \\
$83$ & $5312$ $(2^6 \cdot 83)$ \\
$89$ & $5696$ $(2^6 \cdot 89)$ \\
$97$ & $6208$ $(2^6 \cdot 97)$ \\
$101$ & $6464$ $(2^6 \cdot 101)$ \\
$103$ & $6592$ $(2^6 \cdot 103)$ \\
$107$ & $6848$ $(2^6 \cdot 107)$ \\
$109$ & $6976$ $(2^6 \cdot 109)$ \\
$113$ & $7232$ $(2^6 \cdot 113)$ \\
$127$ & $16256$ $(2^7 \cdot 127)$, $1032256$ $(2^6 \cdot 127^2)$ \\
$131$ & $16768$ $(2^7 \cdot 131)$ \\
$137$ & $17536$ $(2^7 \cdot 137)$ \\
$139$ & $17792$ $(2^7 \cdot 139)$ \\
$149$ & $19072$ $(2^7 \cdot 149)$ \\
$151$ & $19328$ $(2^7 \cdot 151)$ \\
$157$ & $20096$ $(2^7 \cdot 157)$ \\
$163$ & $20864$ $(2^7 \cdot 163)$ \\ 
$167$ & $21376$ $(2^7 \cdot 167)$ \\
$173$ & $22144$ $(2^7 \cdot 173)$ \\
$179$ & $22912$ $(2^7 \cdot 179)$ \\
$181$ & $23168$ $(2^7 \cdot 181)$ \\
$191$ & $24448$ $(2^7 \cdot 191)$ \\
$193$ & $24704$ $(2^7 \cdot 193)$ \\
$197$ & $25216$ $(2^7 \cdot 197)$ \\
$199$ & $25472$ $(2^7 \cdot 199)$ \\
$211$ & $27008$ $(2^7 \cdot 211)$ \\
$223$ & $28544$ $(2^7 \cdot 223)$ \\
$227$ & $29056$ $(2^7 \cdot 227)$ \\
$229$ & $29312$ $(2^7 \cdot 229)$ \\
$233$ & $29824$ $(2^7 \cdot 233)$ \\
$239$ & $30592$ $(2^7 \cdot 239)$ \\
$241$ & $30848$ $(2^7 \cdot 241)$ \\
$251$ & $32128$ $(2^7 \cdot 251)$
\end{tabular}
%%%%%
%%%%%
\end{document}
