\documentclass[12pt]{article}
\usepackage{pmmeta}
\pmcanonicalname{IndependenceOfValuations}
\pmcreated{2013-03-22 14:11:44}
\pmmodified{2013-03-22 14:11:44}
\pmowner{pahio}{2872}
\pmmodifier{pahio}{2872}
\pmtitle{independence of valuations}
\pmrecord{22}{35626}
\pmprivacy{1}
\pmauthor{pahio}{2872}
\pmtype{Theorem}
\pmcomment{trigger rebuild}
\pmclassification{msc}{11R99}
\pmsynonym{approximation theorem}{IndependenceOfValuations}
\pmrelated{TrivialValuation}
\pmrelated{EquivalentValuations}
\pmrelated{WeakApproximationTheorem}

\endmetadata

% this is the default PlanetMath preamble.  as your knowledge
% of TeX increases, you will probably want to edit this, but
% it should be fine as is for beginners.

% almost certainly you want these
\usepackage{amssymb}
\usepackage{amsmath}
\usepackage{amsfonts}

% used for TeXing text within eps files
%\usepackage{psfrag}
% need this for including graphics (\includegraphics)
%\usepackage{graphicx}
% for neatly defining theorems and propositions
%\usepackage{amsthm}
% making logically defined graphics
%%%\usepackage{xypic}

% there are many more packages, add them here as you need them

% define commands here
\begin{document}
Let $|\cdot|_1$, \ldots, $|\cdot|_n$ be {\em non-trivial} (i.e., they all have also other values than 0 and 1) and pairwise non-equivalent valuations of a field $K$, all with values real numbers. \,If $a_1$, ..., $a_n$ are some elements of this field and $\varepsilon$ is an arbitrary positive number, then there exists in $K$ an element $y$ which satisfies the conditions
\begin{align*}
\begin{cases}
                |y-a_1|_1 < \varepsilon,\\
                \qquad     \vdots \qquad \\  
                |y-a_n|_n < \varepsilon.\\
\end{cases}
\end{align*}
%%%%%
%%%%%
\end{document}
