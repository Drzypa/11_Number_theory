\documentclass[12pt]{article}
\usepackage{pmmeta}
\pmcanonicalname{ErdHosStrausConjecture}
\pmcreated{2013-03-22 16:28:04}
\pmmodified{2013-03-22 16:28:04}
\pmowner{CompositeFan}{12809}
\pmmodifier{CompositeFan}{12809}
\pmtitle{Erd\H{o}s-Straus conjecture}
\pmrecord{8}{38628}
\pmprivacy{1}
\pmauthor{CompositeFan}{12809}
\pmtype{Conjecture}
\pmcomment{trigger rebuild}
\pmclassification{msc}{11A67}
\pmsynonym{Erd\"os-Straus conjecture}{ErdHosStrausConjecture}
\pmsynonym{Erdos-Straus conjecture}{ErdHosStrausConjecture}
\pmsynonym{Erdos-Strauss conjecture}{ErdHosStrausConjecture}

% this is the default PlanetMath preamble.  as your knowledge
% of TeX increases, you will probably want to edit this, but
% it should be fine as is for beginners.

% almost certainly you want these
\usepackage{amssymb}
\usepackage{amsmath}
\usepackage{amsfonts}

% used for TeXing text within eps files
%\usepackage{psfrag}
% need this for including graphics (\includegraphics)
%\usepackage{graphicx}
% for neatly defining theorems and propositions
%\usepackage{amsthm}
% making logically defined graphics
%%%\usepackage{xypic}

% there are many more packages, add them here as you need them

% define commands here

\begin{document}
In 1948, Paul Erd\H{o}s and Ernst Straus conjectured that for an integer $n > 1$ there is always a solution to $$\frac{4}{n} = \frac{1}{a} + \frac{1}{b} + \frac{1}{c}$$ where $a$, $b$ and $c$ are integers in the relation $0 < a \le b \le c$. This is the {\em Erd\H{o}s-Straus \PMlinkescapetext{conjecture}}. Put another way, $\frac{4}{n}$ can be rewritten as a sum of three unit fractions. The three unit fractions need not be distinct, but some people consider solutions with distinct unit fractions to be more elegant. It is believed that for all $n > 4$ solutions with distinct unit fractions are possible.

As with any conjecture, a single counterexample is enough to disprove, but no multitude of examples is enough to prove. With brute force computer calculations, Allan Swett has obtained examples for all $n < 10^{14}$. Because of the Hasse principle of Diophantine equations, we can be sure that for semiprimes $pq$ (where $p$ and $q$ are distinct primes) a solution can be found by looking at $\frac{4}{p}$ or $\frac{4}{q}$. Researchers are therefore certain that if a counterexample exists, it is surely a prime number. Thus Swett has only made available the raw data only for selected prime $n$ rather than for all $n$ he tested.
%%%%%
%%%%%
\end{document}
