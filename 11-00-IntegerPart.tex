\documentclass[12pt]{article}
\usepackage{pmmeta}
\pmcanonicalname{IntegerPart}
\pmcreated{2013-03-22 16:14:11}
\pmmodified{2013-03-22 16:14:11}
\pmowner{Wkbj79}{1863}
\pmmodifier{Wkbj79}{1863}
\pmtitle{integer part}
\pmrecord{6}{38337}
\pmprivacy{1}
\pmauthor{Wkbj79}{1863}
\pmtype{Definition}
\pmcomment{trigger rebuild}
\pmclassification{msc}{11-00}
\pmclassification{msc}{26A09}
\pmrelated{FractionalPart}

\endmetadata

\usepackage{amssymb}
\usepackage{amsmath}
\usepackage{amsfonts}

\usepackage{psfrag}
\usepackage{graphicx}
\usepackage{amsthm}
%%\usepackage{xypic}

\begin{document}
The \emph{integer part} of a real number is the part of the number that appears before the decimal \PMlinkescapetext{point}.  For example, the integer part of $\pi$ is $3$, and the integer part of $-\sqrt{2}$ is $-1$.

To be more precise, for $x \in \mathbb{R}$, the integer part of $x$, denoted as $[x]$, is given by

$$[x]=\begin{cases}
\lfloor x \rfloor \text{ if } x \ge 0 \\
\lceil x \rceil \text{ if } x<0, \end{cases}$$

where $\lfloor x \rfloor$ and $\lceil x \rceil$ denote the floor and ceiling of $x$, respectively.
%%%%%
%%%%%
\end{document}
