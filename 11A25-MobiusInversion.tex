\documentclass[12pt]{article}
\usepackage{pmmeta}
\pmcanonicalname{MobiusInversion}
\pmcreated{2013-03-22 11:46:58}
\pmmodified{2013-03-22 11:46:58}
\pmowner{mathcam}{2727}
\pmmodifier{mathcam}{2727}
\pmtitle{M\"obius inversion}
\pmrecord{25}{30252}
\pmprivacy{1}
\pmauthor{mathcam}{2727}
\pmtype{Topic}
\pmcomment{trigger rebuild}
\pmclassification{msc}{11A25}
\pmsynonym{Moebius inversion}{MobiusInversion}
%\pmkeywords{number theory}
\pmrelated{MoebiusFunction}
\pmdefines{Mobius inversion formula}
\pmdefines{Mobius transform}
\pmdefines{Mobius function}
\pmdefines{Mobius-Rota inversion}

\usepackage{amssymb}
\usepackage{amsmath}
\usepackage{amsfonts}
\usepackage{amsthm}
\newtheorem{Theo}{Theorem}
\newcommand{\Z}{\mathbb{Z}}
\newcommand{\Nstar}{\mathbb{N}^{*}}
\begin{document}
\PMlinkescapeword{proposition} \PMlinkescapeword{clearly}
\PMlinkescapeword{fix} \PMlinkescapeword{connection}
\PMlinkescapeword{inversion}

\begin{Theo}[Moebius Inversion]
Let $\mu$ be the Moebius function, and let $f$ and $g$ be two functions on the positive integers.  Then the following two conditions are equivalent:
\begin{eqnarray}
f(n)&=&\sum_{d|n}g(d)\textrm{ for all }n\in\Nstar \\
g(n)&=&\sum_{d|n}\mu(d)f\left(\frac{n}{d}\right)\textrm{ for all }n\in\Nstar
\end{eqnarray}
\end{Theo}

\noindent
\textbf{Proof:} Fix some $n\in\Nstar$. Assuming (1), we have
\begin{eqnarray*}
\sum_{d|n}\mu(d)f\left(\frac{n}{d}\right)
&=&\sum_{d|n}\mu(d)\sum_{e|n/d}g(e) \\
&=&\sum_{k|n}\sum_{d|k}\mu(d)g\left(\frac{n}{k}\right) \\
&=&\sum_{k|n}g\left(\frac{n}{k}\right)\sum_{d|k}\mu(d) \\
&=&g(n),
\end{eqnarray*}
the last step following from the identity given in the Mobius function entry.

Conversely, assuming (2), we get
\begin{eqnarray*}
\sum_{d|n}g(d)
&=&\sum_{d|n}\sum_{e|d}\mu(e)f\left(\frac{d}{e}\right) \\
&=&\sum_{k|n}f\left(\frac{n}{k}\right)\sum_{e|k}\mu\left(\frac{k}{e}\right) \\
&=&f(n)\qquad\text{(by the same identity)}
\end{eqnarray*}
as claimed.

\textbf{Definitions: }
In the notation above, $f$ is called the M\"obius transform
of $g$, and formula (2) is called the M\"obius inversion formula.

\section*{M\"obius-Rota inversion}
G.-C. Rota has described a generalization of the M\"obius formalism.
In it, the set $\Nstar$, ordered by the relation $x|y$ between elements
$x$ and $y$, is replaced by a more general ordered set, and $\mu$
is replaced by a function of two variables.

Let $(S,\le)$ be a locally finite ordered set, i.e. an ordered set such that
$\{z\in S|x\le z\le y\}$ is a finite set for all $x,y\in S$. Let $A$ be the set
of functions $\alpha:S\times S\to\Z$ such that
\begin{eqnarray}
\alpha(x,x)&=&1\textrm{ for all }x\in S \\
\alpha(x,y)&\neq&0\textrm{ implies }x\leq y
\end{eqnarray}
$A$ becomes a monoid if we define the product of any two of its
elements, say $\alpha$ and $\beta$, by
$$(\alpha\beta)(x,y)=\sum_{t\in S}\alpha(x,t)\beta(t,y).$$
The sum makes sense because $\alpha(x,t)\beta(t,y)$ is nonzero
for only finitely many values of $t$.
(Clearly this definition is akin to the definition of the product
of two square matrices.)

Consider the element $\iota$ of $A$ defined simply by
\[\iota(x,y)=
\begin{cases}
1 & \textrm{ if $x\le y$} \\ 0 & \text{otherwise.}
\end{cases}\]
The function $\iota$, regarded as a matrix over $\Z$, has an inverse
matrix, say $\nu$. That means
\[
\sum_{t\in S}\iota(x,t)\nu(t,y)=
\begin{cases}
1 & \text{if $x=y$,} \\ 0 & \text{otherwise.}
\end{cases}
\]
Thus for any $f,g\in A$, the equations
\begin{eqnarray}
f&=&\iota g \\
g&=&\nu f
\end{eqnarray}
are equivalent.

Now let's sketch out how the traditional M\"obius inversion is a special
case of Rota's notion. Let $S$ be the set $\Nstar$, ordered by the relation
$x|y$ between elements $x$ and $y$. In this case, $\nu$ is essentially
a function of only one variable:

\textbf{Proposition 3:} With the above notation, $\nu(x,y)=\mu(y/x)$
for all $x,y\in\Nstar$ such that $x|y$.

The proof is fairly straightforward, by induction on the number
of elements of the interval $\{z\in S|x\le z\le y\}$.

Now let $g$ be a function from $\Nstar$ to some additive group,
and write $\overline{g}(x,y)=g(y/x)$ for all pairs $(x,y)$ such
that $x|y$. 

\textbf{Example: }Let $E$ be a set, and let $S$ be the set of all
finite subsets of $E$, ordered by inclusion. The ordered set $S$ is
left-finite, and for any $x,y\in S$ such that $x\subset y$,
we have $\nu(x,y)=(-1)^{|y-x|}$, where $|z|$ denotes the cardinal
of the finite set $z$.

A slightly more sophisticated example comes up in
connection with the chromatic polynomial of a graph or matroid.

\section*{An Additional Generalization}

A final generalization of Moebius inversion occurs when the sum is taken over all integers less than some real value $x$ rather than over the divisors of an integer.  Specifically, let $f:\mathbb{R}\rightarrow\mathbb{C}$ and define $F:\mathbb{R}\rightarrow \mathbb{C}$ by $F(x)=\sum_{n\leq x} f(\frac{x}{n})$.  

Then

\begin{align*}
f(x)=\sum_{n\leq x} \mu(n) F\left(\frac{x}{n}\right).
\end{align*}
%%%%%
%%%%%
%%%%%
%%%%%
\end{document}
