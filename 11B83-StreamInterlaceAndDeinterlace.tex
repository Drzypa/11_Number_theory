\documentclass[12pt]{article}
\usepackage{pmmeta}
\pmcanonicalname{StreamInterlaceAndDeinterlace}
\pmcreated{2013-03-22 15:37:08}
\pmmodified{2013-03-22 15:37:08}
\pmowner{michal}{7107}
\pmmodifier{michal}{7107}
\pmtitle{stream interlace and deinterlace}
\pmrecord{16}{37543}
\pmprivacy{1}
\pmauthor{michal}{7107}
\pmtype{Theorem}
\pmcomment{trigger rebuild}
\pmclassification{msc}{11B83}
\pmsynonym{interlace deinterlace}{StreamInterlaceAndDeinterlace}
\pmrelated{FraenkelsPartitionTheorem}
\pmrelated{BeattySequence}
\pmrelated{DataStream}
\pmdefines{stream junction method}

\endmetadata

% this is the default PlanetMath preamble.  as your knowledge
% of TeX increases, you will probably want to edit this, but
% it should be fine as is for beginners.

% almost certainly you want these
\usepackage{amssymb}
\usepackage{amsmath}
\usepackage{amsfonts}

% used for TeXing text within eps files
%\usepackage{psfrag}
% need this for including graphics (\includegraphics)
%\usepackage{graphicx}
% for neatly defining theorems and propositions
%\usepackage{amsthm}
% making logically defined graphics
%%%\usepackage{xypic}

% there are many more packages, add them here as you need them

% define commands here
\begin{document}
{\it Interlace} is the method to create a new data stream from
two data streams, each of which has a constant time interval sequence.
Formally, suppose $A = (a , \Delta_{a} )$ and $B = (b, \Delta_{b})$
are two data streams, each have a constant time sequence. For convenience,
we use $\Delta_{a}$ and $\Delta_{b}$ to also denote the constant terms
of each of those sequences. We construct a new data stream $C = (c, \Delta_{c})$, also having constant time interval, as follows:

$
c_{n}=\left\{
\begin{array}{cc}
b_{n-\left\lfloor n z \right\rfloor } & \left\lfloor n z
\right\rfloor =\left\lfloor \left( n+1\right) z \right\rfloor \\
a_{\left\lfloor n z \right\rfloor } & \left\lfloor n z \right\rfloor
\neq \left\lfloor \left( n+1\right) z \right\rfloor
\end{array}
\right. , z =\frac{\Delta _{b}}{\Delta _{a}+\Delta _{b}},\Delta _{c}=
\frac{\Delta _{a}\Delta _{b}}{\Delta _{a}+\Delta _{b}} \label{interlace}
$

{\it Deinterlace } is the method of constructing two data streams , $A$ and
$B$,
each having constant time interval, from a given data stream $C$ and primary interlace value $\Delta $ of computed stream, where
$C$ has constant time interval.

$
a_{n} = c_{n+ \left\lceil \frac{(n+1)\Delta _{a}}{\Delta _{b}} \right\rceil }\
,\ \Delta _{a}=\frac{\Delta _{c}\Delta _{b}}{\left\vert \Delta _{c}-\Delta _{b}\right\vert }
\label{deinterlace_a}
$
and
$
b_{n} = c_{n+\left\lfloor \frac{n\Delta _{b}}{\Delta _{a}}\right\rfloor}
,\ \Delta _{b}=\frac{\Delta _{c}\Delta _{a}}{\left\vert \Delta _{c}-\Delta_{a}\right\vert } \label{deintrlace_b}
$

This sequences are the Fraenkel partition theorem instance.

{\Huge \bf References}

\begin{description}
\item[ [1] ] Aviezri S. Fraenkel, {\em The bracket function and complementary sets of integers}, Canad. J.
Math. {\bf 21} (1969), 6--27. {\bf \PMlinkexternal{MR
38:3214}{http://www.ams.org/mathscinet-getitem?mr=38:3214}}
\item[ [2] ] Michal Widera, {\em Deterministic method of data sequence processing}, Vol. IV, ISSN 1732-1360, Annales UMCS (2006), 314--331. {\bf \PMlinkexternal{UMCS Annales AI}{http://www.annales.umcs.lublin.pl/AI/index.html}}

\end{description}
%%%%%
%%%%%
\end{document}
