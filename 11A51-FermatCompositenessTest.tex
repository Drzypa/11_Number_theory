\documentclass[12pt]{article}
\usepackage{pmmeta}
\pmcanonicalname{FermatCompositenessTest}
\pmcreated{2013-03-22 13:17:36}
\pmmodified{2013-03-22 13:17:36}
\pmowner{bbukh}{348}
\pmmodifier{bbukh}{348}
\pmtitle{Fermat compositeness test}
\pmrecord{15}{33781}
\pmprivacy{1}
\pmauthor{bbukh}{348}
\pmtype{Algorithm}
\pmcomment{trigger rebuild}
\pmclassification{msc}{11A51}
%\pmkeywords{primality}
%\pmkeywords{prime}
%\pmkeywords{composite}
%\pmkeywords{number theory}
\pmrelated{MillerRabinPrimeTest}
\pmdefines{pseudoprime}
\pmdefines{witness}
\pmdefines{Carmichael numbers}

% We include the normal PlanetMath packages
\usepackage{amssymb}
\usepackage{amsmath}
\usepackage{amsfonts}

% and ``amsthm'' for neatly the proof enviroment.
\usepackage{amsthm}

% Finnally, we just need to provide a definition for \nequiv
\newcommand{\nequiv}{\not\equiv}
\begin{document}
The \emph{Fermat compositeness test} is a primality test based on the observation that by Fermat's little theorem if $b^{n-1} \nequiv 1\pmod n$ and $b\nequiv 0\pmod n$, then $n$ is composite. The Fermat compositeness test consists of checking whether $b^{n-1} \equiv 1\pmod n$ for a handful of values of $b$. If a $b$ with $b^{n-1} \nequiv 1\pmod n$ is found, then $n$ is composite. 

A value of $b$ for which $b^{n-1} \nequiv 1\pmod n$ is called a \emph{witness} to $n$'s compositeness. If $b^{n-1} \equiv 1\pmod n$, then $n$ is said to be \emph{pseudoprime} base $b$.

It can be proven that most composite numbers can be shown to be composite by testing only a few values of $b$. However, there are infinitely many composite numbers that are pseudoprime in every base. These are \emph{Carmichael numbers} (see OEIS sequence \PMlinkexternal{A002997}{http://www.research.att.com/cgi-bin/access.cgi/as/njas/sequences/eisA.cgi?Anum=A002997} for a list of first few Carmichael numbers).
%%%%%
%%%%%
\end{document}
