\documentclass[12pt]{article}
\usepackage{pmmeta}
\pmcanonicalname{VandiversConjecture}
\pmcreated{2013-03-22 15:01:11}
\pmmodified{2013-03-22 15:01:11}
\pmowner{mathcam}{2727}
\pmmodifier{mathcam}{2727}
\pmtitle{Vandiver's conjecture}
\pmrecord{5}{36727}
\pmprivacy{1}
\pmauthor{mathcam}{2727}
\pmtype{Conjecture}
\pmcomment{trigger rebuild}
\pmclassification{msc}{11R29}
\pmrelated{ClassNumbersAndDiscriminantsTopicsOnClassGroups}

\endmetadata

% this is the default PlanetMath preamble.  as your knowledge
% of TeX increases, you will probably want to edit this, but
% it should be fine as is for beginners.

% almost certainly you want these
\usepackage{amssymb}
\usepackage{amsmath}
\usepackage{amsfonts}
\usepackage{amsthm}

% used for TeXing text within eps files
%\usepackage{psfrag}
% need this for including graphics (\includegraphics)
%\usepackage{graphicx}
% for neatly defining theorems and propositions
%\usepackage{amsthm}
% making logically defined graphics
%%%\usepackage{xypic}

% there are many more packages, add them here as you need them

% define commands here

\newcommand{\mc}{\mathcal}
\newcommand{\mb}{\mathbb}
\newcommand{\mf}{\mathfrak}
\newcommand{\ol}{\overline}
\newcommand{\ra}{\rightarrow}
\newcommand{\la}{\leftarrow}
\newcommand{\La}{\Leftarrow}
\newcommand{\Ra}{\Rightarrow}
\newcommand{\nor}{\vartriangleleft}
\newcommand{\Gal}{\text{Gal}}
\newcommand{\GL}{\text{GL}}
\newcommand{\Z}{\mb{Z}}
\newcommand{\R}{\mb{R}}
\newcommand{\Q}{\mb{Q}}
\newcommand{\C}{\mb{C}}
\newcommand{\<}{\langle}
\renewcommand{\>}{\rangle}
\begin{document}
Let $K=\mathbb{Q}(\zeta_p)^+$, the maximal real subfield of the $p$-th cyclotomic field.  Vandiver's conjecture states that $p$ does not divide $h_K$, the class number of $K$.

For comparison, see the entries on regular primes and irregular primes.

A proof of Vandiver's conjecture would be a landmark in algebraic number theory, as many theorems hinge on the assumption that this conjecture is true.  For example, it is known that if Vandiver's conjecture holds, that the $p$-rank of the ideal class group of $\mathbb{Q}(\zeta_p)$ equals the number of Bernoulli numbers divisible by $p$ (a remarkable strengthening of Herbrand's theorem).
%%%%%
%%%%%
\end{document}
