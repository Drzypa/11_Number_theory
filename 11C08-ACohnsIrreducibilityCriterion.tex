\documentclass[12pt]{article}
\usepackage{pmmeta}
\pmcanonicalname{ACohnsIrreducibilityCriterion}
\pmcreated{2013-03-22 14:37:02}
\pmmodified{2013-03-22 14:37:02}
\pmowner{Mathprof}{13753}
\pmmodifier{Mathprof}{13753}
\pmtitle{A. Cohn's irreducibility criterion}
\pmrecord{17}{36194}
\pmprivacy{1}
\pmauthor{Mathprof}{13753}
\pmtype{Theorem}
\pmcomment{trigger rebuild}
\pmclassification{msc}{11C08}

\endmetadata

% this is the default PlanetMath preamble.  as your knowledge
% of TeX increases, you will probably want to edit this, but
% it should be fine as is for beginners.

% almost certainly you want these
\usepackage{amssymb}
\usepackage{amsmath}
\usepackage{amsfonts}

% used for TeXing text within eps files
%\usepackage{psfrag}
% need this for including graphics (\includegraphics)
%\usepackage{graphicx}
% for neatly defining theorems and propositions
\usepackage{amsthm}
% making logically defined graphics
%%%\usepackage{xypic}

% there are many more packages, add them here as you need them

% define commands here
\newtheorem*{theorem}{Theorem}
\begin{document}
 \begin{theorem} 
Assume $n  \geq 2$ is an integer  and that $P$ is a polynomial with coefficients in $\{0,1,\ldots,n-1\}$. If $P(n)$ is prime then $P(x)$ is \PMlinkname{irreducible}{IrreduciblePolynomial2} in $\mathbb{Z}[x]$.
 \end{theorem}
A proof is given in \cite{MRM}.

 A. Cohn \cite{PZ} proved this theorem for the case $n=10$.



This special case of the above theorem is sketched as problem 128, Part VIII, in \cite{PZ}.


 \begin{thebibliography}{0}
 \bibitem[PZ]{PZ}
George Pólya, Gabor Szego,
 {\it Problems and Theorems in Analysis II},
 Classics in Mathematics 1998.
\bibitem[MRM]{MRM}
M. Ram Murty, {\it Prime Numbers and Irreducible Polynomials}, American
Mathematical Monthly, vol. 109, (2002), 452-458.
 \end{thebibliography}
%%%%%
%%%%%
\end{document}
