\documentclass[12pt]{article}
\usepackage{pmmeta}
\pmcanonicalname{ExistenceOfSquareRootsOfNonnegativeRealNumbers}
\pmcreated{2013-03-22 16:32:42}
\pmmodified{2013-03-22 16:32:42}
\pmowner{PrimeFan}{13766}
\pmmodifier{PrimeFan}{13766}
\pmtitle{existence of square roots of non-negative real numbers}
\pmrecord{8}{38727}
\pmprivacy{1}
\pmauthor{PrimeFan}{13766}
\pmtype{Theorem}
\pmcomment{trigger rebuild}
\pmclassification{msc}{11A25}
%\pmkeywords{root}
%\pmkeywords{square root}
%\pmkeywords{supremum}
%\pmkeywords{completeness axiom}
\pmrelated{AxiomOfAnalysis}
\pmrelated{ArchimedeanProperty}
\pmrelated{Supremum}
\pmrelated{ExistenceOfNthRoot}

% this is the default PlanetMath preamble.  as your knowledge
% of TeX increases, you will probably want to edit this, but
% it should be fine as is for beginners.

% almost certainly you want these
\usepackage{amssymb}
\usepackage{amsmath}
\usepackage{amsfonts}
\usepackage{amsthm}

% used for TeXing text within eps files
%\usepackage{psfrag}
% need this for including graphics (\includegraphics)
%\usepackage{graphicx}
% for neatly defining theorems and propositions
%\usepackage{amsthm}
% making logically defined graphics
%%%\usepackage{xypic}

% there are many more packages, add them here as you need them

% define commands here
\theoremstyle{plain}
\newtheorem*{thm}{Theorem}
\newtheorem*{lem}{Lemma}
\newtheorem*{cor}{Corollary}


\begin{document}
\begin{thm}
Every non-negative real number has a square root. 
\end{thm}
\begin{proof}
Let $x\geq 0\in\mathbb{R}$. If $x=0$ then the result is trivial, so suppose $x>0$ and define $S=\{y\in\mathbb{R}:y> 0\text{ and }y^2<x\}$. $S$ is nonempty, for if $0<y<\min\{x,1\}$, then $y^2< y<x$, and $y\in S$. $S$ is also bounded above, for if $y>\max\{x,1\}$, then $y^2>y>x$, so such a $y$ is an upper bound of $S$. Thus $S$ is nonempty and bounded, and hence has a supremum which we denote $L$. We will show that $L^2=x$. First suppose $L^2<x$. By the Archimedean Principle there exists some $n\in\mathbb{N}$ such that $n>(2L+1)/(x-L^2)$. Then we have
\begin{equation}
\bigg(L+\dfrac{1}{n}\bigg)^2=L^2+\dfrac{2L}{n}+\dfrac{1}{n^2}
<L^2+\dfrac{2L}{n}+\dfrac{1}{n}<x\text{.}
\end{equation}
So $L+1/n$ is a member of $S$ strictly greater than $L$, contrary to assumption. Now suppose that $L^2>x$. Again by the Archimedean Principle there exists some $n\in\mathbb{N}$ such that $1/n<(L^2-x)/2L$ and $1/n<L$. Then we have
\begin{equation}
\bigg(L-\dfrac{1}{n}\bigg)^2=L^2-\dfrac{2L}{n}+\dfrac{1}{n^2}
>L^2-\dfrac{2L}{n}>x\text{.}
\end{equation}
But there must exist some $y\in S$ such that $L-1/n<y<L$, which gives $x<\big(L-1/n\big)^2<y^2$, so that $y\notin S$, a contradiction. Thus it must be that $L^2=x$. 
\end{proof}

%%%%%
%%%%%
\end{document}
