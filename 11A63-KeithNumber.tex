\documentclass[12pt]{article}
\usepackage{pmmeta}
\pmcanonicalname{KeithNumber}
\pmcreated{2013-03-22 16:00:20}
\pmmodified{2013-03-22 16:00:20}
\pmowner{PrimeFan}{13766}
\pmmodifier{PrimeFan}{13766}
\pmtitle{Keith number}
\pmrecord{5}{38035}
\pmprivacy{1}
\pmauthor{PrimeFan}{13766}
\pmtype{Definition}
\pmcomment{trigger rebuild}
\pmclassification{msc}{11A63}
\pmsynonym{repfigit number}{KeithNumber}

% this is the default PlanetMath preamble.  as your knowledge
% of TeX increases, you will probably want to edit this, but
% it should be fine as is for beginners.

% almost certainly you want these
\usepackage{amssymb}
\usepackage{amsmath}
\usepackage{amsfonts}

% used for TeXing text within eps files
%\usepackage{psfrag}
% need this for including graphics (\includegraphics)
%\usepackage{graphicx}
% for neatly defining theorems and propositions
%\usepackage{amsthm}
% making logically defined graphics
%%%\usepackage{xypic}

% there are many more packages, add them here as you need them

% define commands here

\begin{document}
Given a base $b$ integer $$n = \sum_{i = 1}^k d_ib^{i - 1}$$ where $d_1$ is the least significant digit and $d_k$ is the most significant, construct the sequence $a_1 = d_k, \ldots a_k = d_1$, and for $m > k$, $$a_m = \sum_{i = 1}^k a_{m - i}.$$ If there is an $x$ such that $a_x = n$, then $n$ is a {\em Keith number} or {\em repfigit number}.

In base 10, the first few Keith numbers below ten thousand are: 14, 19, 28, 47, 61, 75, 197, 742, 1104, 1537, 2208, 2580, 3684, 4788, 7385, 7647, 7909 (see A007629 in Sloane's OEIS for a longer listing). 47 is a base 10 Keith number because it is contained the Fibonacci-like recurrence started from its base 10 digits: 4, 7, 11, 18, 29, 47, etc.

\begin{thebibliography}{1}
\bibitem{mk} M. Keith, ``Repfigit Numbers" {\it J. Rec. Math.} 19 (1987), 41 - 42.
\end{thebibliography}
%%%%%
%%%%%
\end{document}
