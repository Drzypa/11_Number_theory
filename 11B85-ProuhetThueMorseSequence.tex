\documentclass[12pt]{article}
\usepackage{pmmeta}
\pmcanonicalname{ProuhetThueMorseSequence}
\pmcreated{2013-03-22 14:27:17}
\pmmodified{2013-03-22 14:27:17}
\pmowner{Mathprof}{13753}
\pmmodifier{Mathprof}{13753}
\pmtitle{Prouhet-Thue-Morse sequence}
\pmrecord{24}{35973}
\pmprivacy{1}
\pmauthor{Mathprof}{13753}
\pmtype{Definition}
\pmcomment{trigger rebuild}
\pmclassification{msc}{11B85}
\pmclassification{msc}{68R15}
\pmsynonym{Thue-Morse sequence}{ProuhetThueMorseSequence}
\pmrelated{ProuhetThueMorseConstant}

\endmetadata

% this is the default PlanetMath preamble.  as your knowledge
% of TeX increases, you will probably want to edit this, but
% it should be fine as is for beginners.

% almost certainly you want these
\usepackage{amssymb}
\usepackage{amsmath}
\usepackage{amsfonts}

% used for TeXing text within eps files
%\usepackage{psfrag}
% need this for including graphics (\includegraphics)
%\usepackage{graphicx}
% for neatly defining theorems and propositions
%\usepackage{amsthm}
% making logically defined graphics
%%%\usepackage{xypic}

% there are many more packages, add them here as you need them

% define commands here
\begin{document}
The {\em Prouhet-Thue-Morse sequence} is a binary sequence which begins as follows:

\[ 0, 1, 1, 0, 1, 0, 0, 1, 1, 0, 0, 1, 0, 1, 1, 0, \ldots \]

The $n$th term is defined to be the number of $1$s in the binary expansion of $n$, modulo 2.  That is, $t_n = 0$ if the number of $1$s in the binary expansion of $n$ is even, and $t_n = 1$ if it is odd.

The sequence satisfies the following recurrence relation, with $t_0=0$:
\[ \begin{array}{ccc} t_{2n} &=& t_n \\
                      t_{2n+1} &=& 1 - t_n \end{array} \]

The Prouhet-Thue-Morse sequence is an automatic sequence.  It has been shown to be 
\PMlinkescapetext{cube-free} 
(no three consecutive identical blocks)  and overlap-free 
i.e no sub-block of the form $awawa$, where $a \in \{0,1\}$, when viewed as a word of infinite length over the binary alphabet $\{0,1\}$.

\subsubsection*{Generating function}

The generating function $T(x)=\sum_{n=0}^{\infty} t_nx^n$ for the sequence satisfies the relation

\[ T(x) = T(x^2)(1-x) + \frac{x}{1-x^2} \]

\subsubsection*{History}

The Thue-Morse sequence was independently discovered by P. Prouhet, Axel Thue, and Marston Morse, and has since been rediscovered by many others.

\subsubsection*{References}
\begin{itemize}
\item Allouche, J.-P.; Shallit, J. O.  \PMlinkexternal{The ubiquitous Prouhet-Thue-Morse Sequence}{http://www.cs.uwaterloo.ca/~shallit/Papers/ubiq.ps} [postscript]
\item 
Sloane, N. J. A.  { \it Sequence A010060}, \PMlinkexternal{The On-Line Encyclopedia of Integer Sequences}{http://www.research.att.com/~njas/sequences/}.
\end{itemize}
%%%%%
%%%%%
\end{document}
