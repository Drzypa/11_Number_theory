\documentclass[12pt]{article}
\usepackage{pmmeta}
\pmcanonicalname{FibonacciFraction}
\pmcreated{2013-03-22 18:03:18}
\pmmodified{2013-03-22 18:03:18}
\pmowner{PrimeFan}{13766}
\pmmodifier{PrimeFan}{13766}
\pmtitle{Fibonacci fraction}
\pmrecord{4}{40584}
\pmprivacy{1}
\pmauthor{PrimeFan}{13766}
\pmtype{Definition}
\pmcomment{trigger rebuild}
\pmclassification{msc}{11B39}

% this is the default PlanetMath preamble.  as your knowledge
% of TeX increases, you will probably want to edit this, but
% it should be fine as is for beginners.

% almost certainly you want these
\usepackage{amssymb}
\usepackage{amsmath}
\usepackage{amsfonts}

% used for TeXing text within eps files
%\usepackage{psfrag}
% need this for including graphics (\includegraphics)
%\usepackage{graphicx}
% for neatly defining theorems and propositions
%\usepackage{amsthm}
% making logically defined graphics
%%%\usepackage{xypic}

% there are many more packages, add them here as you need them

% define commands here

\begin{document}
A {\em Fibonacci fraction} is a rational number of the form $\frac{F_n}{F_m}$ where $F_i$ is the $i$th number of the Fibonacci sequence and $n$ and $m$ are integers in the relation $n < m$. In the Fibonacci fractional series, each $m = n + 2$:

$$\frac{1}{2}, \frac{1}{3}, \frac{2}{5}, \frac{3}{8}, \frac{5}{13}, \frac{8}{21}, \frac{13}{34}, \frac{21}{55}, \frac{34}{89}, \ldots$$

The most important application of Fibonacci fractions is in botany: plants arrange the leaves on their stems (phyllotaxy) in many different ways, but ``only those conforming to a Fibonacci fraction allow for efficient packing of leaf primordia on the meristem surface.'' There is also an application in optics.

\begin{thebibliography}{3}
\bibitem{pd} P. A. David ``Leaf Position in Ailanthus Altissima in Relation to the Fibonacci Series'' {\it American Journal of Botany} {\bf 26} 2 (1939): 67
\bibitem{rp} R. W Pearcy \& W Yang ``The functional morphology of light capture and carbon gain in the Redwood forest understorey plant Adenocaulon bicolor Hook'' {\it Functional Ecology} {\bf 12} 4 (1998): 551
\bibitem{hr} H. C. Rosu, J. P. Trevino, H. Cabrera \& J. S. Murguia, ``Self-image effects for diffraction and dispersion'' {\it Electromagnetic Phenomena} {\bf 6} 2 (2006): 204 - 211
\end{thebibliography}
%%%%%
%%%%%
\end{document}
