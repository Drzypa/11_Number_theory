\documentclass[12pt]{article}
\usepackage{pmmeta}
\pmcanonicalname{FieldOfAlgebraicNumbers}
\pmcreated{2015-11-18 14:30:41}
\pmmodified{2015-11-18 14:30:41}
\pmowner{pahio}{2872}
\pmmodifier{pahio}{2872}
\pmtitle{field of algebraic numbers}
\pmrecord{14}{42024}
\pmprivacy{1}
\pmauthor{pahio}{2872}
\pmtype{Definition}
\pmcomment{trigger rebuild}
\pmclassification{msc}{11R04}
\pmrelated{AlgebraicSumAndProduct}
\pmrelated{SubfieldCriterion}
\pmrelated{AlgebraicNumbersAreCountable}
\pmrelated{RingWithoutIrreducibles}
\pmrelated{AllAlgebraicNumbersInASequence}
\pmdefines{ring of algebraic integers}

\endmetadata

% this is the default PlanetMath preamble.  as your knowledge
% of TeX increases, you will probably want to edit this, but
% it should be fine as is for beginners.

% almost certainly you want these
\usepackage{amssymb}
\usepackage{amsmath}
\usepackage{amsfonts}

% used for TeXing text within eps files
%\usepackage{psfrag}
% need this for including graphics (\includegraphics)
%\usepackage{graphicx}
% for neatly defining theorems and propositions
 \usepackage{amsthm}
% making logically defined graphics
%%%\usepackage{xypic}

% there are many more packages, add them here as you need them

% define commands here

\theoremstyle{definition}
\newtheorem*{thmplain}{Theorem}

\begin{document}
As special cases of the theorem of the parent 
``\PMlinkname{polynomial equation with algebraic coefficients}{polynomialequationwithalgebraiccoefficients}'' of this entry, one obtains the 

\textbf{Corollary.}\, If $\alpha$ and $\beta$ are algebraic numbers, then also $\alpha\!+\!\beta$, $\alpha\!-\!\beta$, 
$\alpha\beta$ and $\displaystyle\frac{\alpha}{\beta}$ (provided\, $\beta \neq 0$) are algebraic numbers.\; If $\alpha$ and $\beta$ are algebraic integers, then also $\alpha\!+\!\beta$, $\alpha\!-\!\beta$ and
$\alpha\beta$ are algebraic integers.\\

The case of $\displaystyle\frac{\alpha}{\beta}$ needs an additional consideration:\, If 
$x^m+b_1x^{m-1}+\ldots+b_{m-1}x+b_m$ is the minimal polynomial of $\beta$, the equation 
\,$\beta^m+b_1\beta^{m-1}+\ldots+b_{m-1}\beta+b_m = 0$\, implies
$$\left(\frac{1}{\beta}\right)^m\!+\frac{b_{m-1}}{b_m}\!\left(\frac{1}{\beta}\right)^{m-1}\!+\ldots+
\frac{b_1}{b_m}\!\cdot\!\frac{1}{\beta}+\frac{1}{b_m} \;=\; 0.$$
Hence $\displaystyle\frac{1}{\beta}$ is an algebraic number, and therefore also 
$\displaystyle\alpha\!\cdot\!\frac{1}{\beta}$.\\



It follows from the corollary that the set of all algebraic numbers is a field and the set of all algebraic integers is a ring (an integral domain, too).\, Moreover, the mentioned theorem implies that the \emph{field of algebraic numbers} is algebraically closed and the \emph{ring of algebraic integers} integrally closed.\, The field of algebraic numbers, which is sometimes denoted by $\mathbb{A}$, contains for example the complex numbers obtained from rational numbers by using arithmetic operations and taking \PMlinkid{roots}{5667} (these numbers form a subfield of $\mathbb{A}$).
%%%%%
%%%%%
\end{document}
