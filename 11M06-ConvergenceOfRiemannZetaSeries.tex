\documentclass[12pt]{article}
\usepackage{pmmeta}
\pmcanonicalname{ConvergenceOfRiemannZetaSeries}
\pmcreated{2015-08-22 13:15:14}
\pmmodified{2015-08-22 13:15:14}
\pmowner{pahio}{2872}
\pmmodifier{pahio}{2872}
\pmtitle{convergence of Riemann zeta series}
\pmrecord{8}{39853}
\pmprivacy{1}
\pmauthor{pahio}{2872}
\pmtype{Definition}
\pmcomment{trigger rebuild}
\pmclassification{msc}{11M06}
\pmclassification{msc}{30A99}
%\pmkeywords{Riemann zeta function}
\pmrelated{ModulusOfComplexNumber}
\pmrelated{ComplexExponentialFunction}

% this is the default PlanetMath preamble.  as your knowledge
% of TeX increases, you will probably want to edit this, but
% it should be fine as is for beginners.

% almost certainly you want these
\usepackage{amssymb}
\usepackage{amsmath}
\usepackage{amsfonts}

% used for TeXing text within eps files
%\usepackage{psfrag}
% need this for including graphics (\includegraphics)
%\usepackage{graphicx}
% for neatly defining theorems and propositions
 \usepackage{amsthm}
% making logically defined graphics
%%%\usepackage{xypic}

% there are many more packages, add them here as you need them

% define commands here

\theoremstyle{definition}
\newtheorem*{thmplain}{Theorem}

\begin{document}
The series
\begin{align}
\sum_{n=1}^\infty\frac{1}{n^s}
\end{align}
converges absolutely for all $s$ with real part greater than 1.\\

{\em Proof.}  Let\, $s = \sigma+it$\, where\, $\sigma$ and $t$ are 
real numbers and\, $\sigma > 1$.\, Then
$$\left|\frac{1}{n^s}\right| = \frac{1}{|e^{s\log{n}}|} = 
\frac{1}{e^{\sigma\log{n}}} = \frac{1}{n^\sigma}.$$
Since the series \,$\sum_{n=1}^\infty\frac{1}{n^\sigma}$ converges, by the \PMlinkname{$p$-test}{PTest}, for\, $\sigma > 1$, we conclude that the series (1) is absolutely convergent in the half-plane \,$\sigma > 1$.  
%%%%%
%%%%%
\end{document}
