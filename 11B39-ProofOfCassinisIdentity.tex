\documentclass[12pt]{article}
\usepackage{pmmeta}
\pmcanonicalname{ProofOfCassinisIdentity}
\pmcreated{2013-03-22 14:44:40}
\pmmodified{2013-03-22 14:44:40}
\pmowner{yark}{2760}
\pmmodifier{yark}{2760}
\pmtitle{proof of Cassini's identity}
\pmrecord{24}{36382}
\pmprivacy{1}
\pmauthor{yark}{2760}
\pmtype{Proof}
\pmcomment{trigger rebuild}
\pmclassification{msc}{11B39}
\pmrelated{CatalansIdentity}

\endmetadata

% this is the default PlanetMath preamble.  as your knowledge
% of TeX increases, you will probably want to edit this, but
% it should be fine as is for beginners.

% almost certainly you want these
\usepackage{amssymb}
\usepackage{amsmath}
\usepackage{amsfonts}

% used for TeXing text within eps files
%\usepackage{psfrag}
% need this for including graphics (\includegraphics)
%\usepackage{graphicx}
% for neatly defining theorems and propositions
%\usepackage{amsthm}
% making logically defined graphics
%%%\usepackage{xypic}

% there are many more packages, add them here as you need them

% define commands here
\usepackage{graphicx}
\usepackage{amssymb}
\usepackage{amsmath}
\usepackage{amsfonts}
\begin{document}
\PMlinkescapeword{identity}

For all positive integers $i$, let $F_i$ denote the $i^{th}$
Fibonacci number, with $F_1 = F_2 = 1$. We will show by
induction that the identity
$$F_{n+1}F_{n-1} - F_n^2 = (-1)^n$$
holds for all positive integers $n\geq 2$.
When $n = 2$, we can substitute in the values for $F_1$, $F_2$
and $F_{3}$ yielding the statement $2\times 1 - 1^2 = (-1)^2$, which is true.
Now suppose that the theorem is true when $n=m$,
for some integer $m \geq 2$.
Recalling the recurrence relation for the Fibonacci numbers,
$F_{i+1} = F_i + F_{i-1}$, we have

\begin{eqnarray*}
 F_{m+2}F_m - F_{m+1}^2 &=&(F_{m+1}+F_m)F_m -(F_m+F_{m-1})^2\\
 &=& F_{m+1}F_m+F_m^2 - F_m^2 -2F_mF_{m-1}-F_{m-1}^2\\ 
 &=& F_{m+1}F_m -2F_mF_{m-1}-F_{m-1}^2\\ 
 &=& (F_m+F_{m-1})F_m -2F_mF_{m-1}-F_{m-1}^2\\ 
 &=& F_m^2 + F_{m-1}F_m -2F_mF_{m-1} - F_{m-1}^2 \\
 &=& F_m^2 -F_mF_{m-1} - F_{m-1}^2 \\
 &=& F_m^2 - (F_m+F_{m-1})F_{m-1}\\
 &=& F_m^2 - F_{m+1}F_{m-1}\\ 
 &=& -(-1)^m
\end{eqnarray*}
by the induction hypothesis.
So we get $ F_{m+2}F_m - F_{m+1}^2 = (-1)^{m+1} $,
and the result is thus true for $n=m+1$.
The theorem now follows by induction.
%%%%%
%%%%%
\end{document}
