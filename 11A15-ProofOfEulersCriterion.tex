\documentclass[12pt]{article}
\usepackage{pmmeta}
\pmcanonicalname{ProofOfEulersCriterion}
\pmcreated{2013-03-22 12:41:53}
\pmmodified{2013-03-22 12:41:53}
\pmowner{Koro}{127}
\pmmodifier{Koro}{127}
\pmtitle{proof of Euler's criterion}
\pmrecord{7}{32980}
\pmprivacy{1}
\pmauthor{Koro}{127}
\pmtype{Proof}
\pmcomment{trigger rebuild}
\pmclassification{msc}{11A15}

\endmetadata

\usepackage{amssymb}
\usepackage{amsmath}
\usepackage{amsfonts}
\usepackage{amsthm}

\newcommand{\R}{\mathbb{R}}
\newcommand{\C}{\mathbb{C}}
\newcommand{\F}{\mathbb{F}}
\newcommand{\Z}{\mathbb{Z}}
\newcommand{\Q}{\mathbb{Q}}
\newcommand{\N}{\mathbb{N}}
\newcommand{\im}{\mathit{i}}

\newcommand{\angbr}[1]{\left< #1 \right>}
\newcommand{\rndbr}[1]{\left( #1 \right)}
\newcommand{\sqrbr}[1]{\left[ #1 \right]}
\newcommand{\barbr}[1]{\left| #1 \right|}
\newcommand{\curbr}[1]{\left\{ #1 \right\}}
\newcommand{\abs}[1]{{\left| #1 \right|}}
\newcommand{\setof}[1]{\curbr{\ #1\ }}
\newcommand{\floor}[1]{\left\lfloor #1 \right\rfloor}
\newcommand{\ceil}[1]{\left\lceil #1 \right\rceil}
\newcommand{\divides}{\mid}
\newcommand{\union}{\cup}
\newcommand{\intersection}{\cap}
\newcommand{\bigunion}{\bigcup}
\newcommand{\bigintersection}{\bigcap}

\newcommand{\ra}{\rightarrow}
\newcommand{\Ra}{\Rightarrow}

\newcommand{\st}{\textrm{such that}}
\newcommand{\WLoG}{\textrm{ \scriptsize W.L.O.G. }}
\newcommand{\ord}[1]{\Theta\rndbr{ #1 }}

\renewcommand{\mod}[1]{\;\rndbr{\mathrm{mod}\ #1}}
\renewcommand{\leq}{\leqslant}
\renewcommand{\geq}{\geqslant}
\begin{document}
\emph{(All congruences are modulo $p$ for the proof; omitted for clarity.)}

Let
\[ x = a^{(p-1)/2} \]

\noindent Then $x^2 \equiv 1$ by Fermat's Little Theorem. Thus:
\[ x \equiv \pm 1 \]

Now consider the two possibilities:

\begin{itemize}

\item If $a$ is a quadratic residue then by definition, $a \equiv b^2$ for some
$b$. Hence:
\[ x \equiv a^{(p-1)/2} \equiv b^{p-1} \equiv 1 \]

\item It remains to show that $a^{(p-1)/2} \equiv -1$ if $a$ is a quadratic
non-residue. We can proceed in two ways:

\begin{description}

\item[Proof (a)] Partition the set $\setof{1, \ldots, p - 1}$ into
pairs $\setof{c, d}$ \st $c d \equiv a$. Then $c$ and $d$ must
always be distinct since $a$ is a non-residue. Hence, the product of
the union of the partitions is:
\[ (p - 1)! \equiv a^{(p-1)/2} \equiv -1 \]

\noindent and the result follows by Wilson's Theorem.

\item[Proof (b)] The equation:
\[ z^{(p-1)/2} \equiv 1 \]

\noindent has at most $(p-1)/2$ roots. But we already know of $(p-1)/2$
distinct roots of the above equation, these being the quadratic residues modulo
$p$. So $a$ can't be a root, yet $a^{(p-1)/2} \equiv \pm 1$. Thus we must have:
\[ a^{(p-1)/2} \equiv -1 \]

\end{description}

\end{itemize}

\noindent QED.
%%%%%
%%%%%
\end{document}
