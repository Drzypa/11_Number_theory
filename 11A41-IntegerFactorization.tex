\documentclass[12pt]{article}
\usepackage{pmmeta}
\pmcanonicalname{IntegerFactorization}
\pmcreated{2013-03-22 16:39:09}
\pmmodified{2013-03-22 16:39:09}
\pmowner{PrimeFan}{13766}
\pmmodifier{PrimeFan}{13766}
\pmtitle{integer factorization}
\pmrecord{8}{38857}
\pmprivacy{1}
\pmauthor{PrimeFan}{13766}
\pmtype{Definition}
\pmcomment{trigger rebuild}
\pmclassification{msc}{11A41}
\pmsynonym{prime factorization}{IntegerFactorization}

% this is the default PlanetMath preamble.  as your knowledge
% of TeX increases, you will probably want to edit this, but
% it should be fine as is for beginners.

% almost certainly you want these
\usepackage{amssymb}
\usepackage{amsmath}
\usepackage{amsfonts}

% used for TeXing text within eps files
%\usepackage{psfrag}
% need this for including graphics (\includegraphics)
%\usepackage{graphicx}
% for neatly defining theorems and propositions
%\usepackage{amsthm}
% making logically defined graphics
%%%\usepackage{xypic}

% there are many more packages, add them here as you need them

% define commands here

\begin{document}
Given an integer $n$, its {\em integer factorization} (or {\em prime factorization}) consists of the primes $p_i$ which multiplied together give $n$ as a result. To put it algebraically, $$n = \prod_{i = 1}^{\omega(n)} {p_i}^{a_i},$$ with each $p_i$ distinct, all $a_i > 0$ but not necessarily distinct, and $\omega(n)$ being the value of the number of distinct prime factors function. Theoretically, an integer is a product of all the prime numbers, $$n = \prod_{i = 1}^{\infty} {p_i}^{a_i},$$ with many $a_i = 0$.

For example, the factorization of 32851 is $7 \times 13 \times 19 \times 19$, more usually expressed as $7 \times 13 \times 19^2$. Because of the commutative property of multiplication, it does not matter in what order the prime factors are stated in, but it is customary to give them in \PMlinkname{ascending order}{AscendingOrder}, and to group them together by the use of exponents.

The factorization of a positive integer is unique (this is the fundamental theorem of arithmetic). For a negative number $n < 0$ one could take the factorization of $|n|$ and randomly give negative signs to one (or any odd number) of the prime factors. Alternatively, the factorization can be given as $-1 \cdot {p_1}^{a_1} \cdot \ldots$ (this is what Mathematica opts for).

The term ``factorization'' is often used to refer to the actual process of determining the prime factors. There are several algorithms to choose from, with trial division being the simplest to implement.
%%%%%
%%%%%
\end{document}
