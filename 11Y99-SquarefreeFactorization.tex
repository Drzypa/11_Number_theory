\documentclass[12pt]{article}
\usepackage{pmmeta}
\pmcanonicalname{SquarefreeFactorization}
\pmcreated{2013-03-22 14:54:20}
\pmmodified{2013-03-22 14:54:20}
\pmowner{mathwizard}{128}
\pmmodifier{mathwizard}{128}
\pmtitle{squarefree factorization}
\pmrecord{5}{36590}
\pmprivacy{1}
\pmauthor{mathwizard}{128}
\pmtype{Algorithm}
\pmcomment{trigger rebuild}
\pmclassification{msc}{11Y99}
\pmclassification{msc}{11C99}
\pmclassification{msc}{13P05}
\pmclassification{msc}{13M10}
\pmrelated{CantorZassenhausSplit}

\endmetadata

% this is the default PlanetMath preamble.  as your knowledge
% of TeX increases, you will probably want to edit this, but
% it should be fine as is for beginners.

% almost certainly you want these
\usepackage{amssymb}
\usepackage{amsmath}
\usepackage{amsfonts}

% used for TeXing text within eps files
%\usepackage{psfrag}
% need this for including graphics (\includegraphics)
%\usepackage{graphicx}
% for neatly defining theorems and propositions
%\usepackage{amsthm}
% making logically defined graphics
%%%\usepackage{xypic}

% there are many more packages, add them here as you need them

% define commands here
\begin{document}
Given a polynomial $A\in\mathbb{F}_p[X]$, where $p$ is a prime and $\mathbb{F}_p$ is the field with $p$ elements, we want to find a decomposition
$$A=\prod_{i=1}^nA_i^i$$
with squarefree polynomials $A_i$, which are pairwise coprime. Since we are in a field, we can assume that $A$ is monic.
Since $A$ has a unique factorization we can take $A_i$ to be the product of all irreducible divisors $P$ of $A$ with $v_P(A)=i$, where $v_P(A)$ is the number such that $P^{v_P(A)}$ divides $A$ but $P^{v_P(A)+1}$ does not, so such a decomposition exists.

If $A$ is in the desired form, we have for the derivative $A^\prime$ of $A$:
\begin{equation}\label{eq:aprime}
A^\prime=\sum_{i=1}^n\prod_{j\neq i}iA_j^jA_i^\prime A_i^{i-1}.
\end{equation}
Now let $T:=\gcd(A,A^\prime)$. For every irreducible polynomial $P$ dividing $T$ we can determine $v_P(T)$ in the following way: $P$ must divide $A_m$ for some value of $m$ (then it does not divide any other $A_i$). Then for all $i\neq m$ the $v_P$ of the $i$th summand in equation (\ref{eq:aprime}) is at least $m$ and for the $m$th summand it is $m-1$ ($A_m^\prime$ is not divisible by $P$ since $A_m$ is squarefree) if $p\nmid m$ and 0 if $p|m$ (because of the factor $m$ in that summand). So we find $v_P(T)=m-1$ if $p\nmid m$ and $v_P(T)=m$ if $p|m$ (equality holds because $T|A$). So we obtain
$$T=\gcd(A,A^\prime)=\prod_{p\nmid i}A_i^{i-1}\prod_{p|i}A_i^i.$$
Now we define two sequences $(T_i)$ and $(V_i)$ by $T_1=T$ and 
$$V_1=\frac{A}{T}=\prod_{p\nmid i}A_i.$$
Then set $V_{k+1}=\gcd(T_k,V_k)$ if $p\nmid k$ and $V_{k+1}=V_k$ if $p|k$ and set $T_{k+1}=\frac{T_k}{V_{k+1}}$. By induction one finds
\begin{align*}
V_k&=\prod_{i\geq k,p\nmid i}A_i;\\
T_k&=\prod_{i\geq k-1,p\nmid i}A_i^{i-k}\prod_{p|i}A_i^i.
\end{align*}
So for $p\nmid k$ it follows $A_k=\frac{V_k}{V_{k+1}}$, so for $p\nmid k$ we can continue as long as $V_k$ is non-constant. When $V_k$ is constant, we have
$$T_{k-1}=\prod_{p|i}A_i^i.$$
Assume $A_i(X)=a_kX^k+\dots+a_1X+a_0$, then with $i=lp$ $$A_i^{lp}=a_k^{lp}X^{klp}+\dots+a_1^{lp}X^{lp}+a_0^{lp}=a_kX^{klp}+...+a_1X^{lp}+a_0,$$
so we can set $Y:=X^p$ and have $T_{k-1}=U^p(X)=U(Y)$. Now we can decompose $U$ by using the algorithm recursively.

In practice one can easily compute $T$ using Euclid's algorithm. Then one computes the sequences $(V_i)$ and $(T_i)$ to get the $A_k$. The algorithm is needed as a first step when one wants to find the prime decomposition of a polynomial, because it reduces the problem to the problem of factoring a squarefree polynomial.
%%%%%
%%%%%
\end{document}
