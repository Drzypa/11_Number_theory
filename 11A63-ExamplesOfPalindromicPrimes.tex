\documentclass[12pt]{article}
\usepackage{pmmeta}
\pmcanonicalname{ExamplesOfPalindromicPrimes}
\pmcreated{2013-03-22 15:55:33}
\pmmodified{2013-03-22 15:55:33}
\pmowner{PrimeFan}{13766}
\pmmodifier{PrimeFan}{13766}
\pmtitle{examples of palindromic primes}
\pmrecord{6}{37933}
\pmprivacy{1}
\pmauthor{PrimeFan}{13766}
\pmtype{Example}
\pmcomment{trigger rebuild}
\pmclassification{msc}{11A63}

% this is the default PlanetMath preamble.  as your knowledge
% of TeX increases, you will probably want to edit this, but
% it should be fine as is for beginners.

% almost certainly you want these
\usepackage{amssymb}
\usepackage{amsmath}
\usepackage{amsfonts}

% used for TeXing text within eps files
%\usepackage{psfrag}
% need this for including graphics (\includegraphics)
%\usepackage{graphicx}
% for neatly defining theorems and propositions
%\usepackage{amsthm}
% making logically defined graphics
%%%\usepackage{xypic}

% there are many more packages, add them here as you need them

% define commands here

\begin{document}
The first few palindromic primes in base 10 are 2, 3, 5, 7, 11, 101, 131, 151, 181, 191, 313, 353, 373, 383, 727 these are listed in A002385 of Sloane's OEIS.

In binary, all Mersenne primes are also palindromic primes, and the same is true of Fermat primes. Some other binary palindromic primes are 73, 107, 313, 443, 1193, 1453, 1571, 1619, 1787, 1831, 1879.

In factorial base, the most significant digit $d_k$ of a palindromic number has to be 1, thus a prime $p$ must fall in the range $k! < p < 2k!$ or else it is not a palindromic prime in factorial base. The first few factorial base palindromic primes are 3, 7, 11, 41, 127, 139, 173, 179, 191, 751, 811.
%%%%%
%%%%%
\end{document}
