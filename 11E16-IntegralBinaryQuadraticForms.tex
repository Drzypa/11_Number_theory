\documentclass[12pt]{article}
\usepackage{pmmeta}
\pmcanonicalname{IntegralBinaryQuadraticForms}
\pmcreated{2013-03-22 16:55:44}
\pmmodified{2013-03-22 16:55:44}
\pmowner{rm50}{10146}
\pmmodifier{rm50}{10146}
\pmtitle{integral binary quadratic forms}
\pmrecord{10}{39194}
\pmprivacy{1}
\pmauthor{rm50}{10146}
\pmtype{Topic}
\pmcomment{trigger rebuild}
\pmclassification{msc}{11E16}
\pmclassification{msc}{11E12}
\pmrelated{RepresentationOfIntegersByEquivalentIntegralBinaryQuadraticForms}
\pmrelated{ReducedIntegralBinaryQuadraticForms}

\endmetadata

% this is the default PlanetMath preamble.  as your knowledge
% of TeX increases, you will probably want to edit this, but
% it should be fine as is for beginners.

% almost certainly you want these
\usepackage{amssymb}
\usepackage{amsmath}
\usepackage{amsfonts}

% used for TeXing text within eps files
%\usepackage{psfrag}
% need this for including graphics (\includegraphics)
%\usepackage{graphicx}
% for neatly defining theorems and propositions
\usepackage{amsthm}
% making logically defined graphics
%%%\usepackage{xypic}
\usepackage{array}

% there are many more packages, add them here as you need them

% define commands here
% Some sets
\newcommand{\Nats}{\mathbb{N}}
\newcommand{\Ints}{\mathbb{Z}}
\newcommand{\BZ}{\mathbb{Z}}
\newcommand{\Reals}{\mathbb{R}}
\newcommand{\Complex}{\mathbb{C}}
\newcommand{\Rats}{\mathbb{Q}}
\newcommand{\Gal}{\operatorname{Gal}}
\newcommand{\Cl}{\operatorname{Cl}}
\newcommand{\Alg}{\mathcal{O}}
\newcommand{\ol}{\overline}
\newcommand{\Leg}[2]{\left(\frac{#1}{#2}\right)}
%
%% \theoremstyle{plain} %% This is the default
\newtheorem{thm}{Theorem}
\newtheorem{cor}[thm]{Corollary}
\newtheorem{lem}[thm]{Lemma}
\newtheorem{prop}[thm]{Proposition}
\newtheorem{ax}{Axiom}

\theoremstyle{definition}
\newtheorem{defn}{Definition}
\begin{document}
An integral binary quadratic form is a quadratic form (q.v.) in two variables over $\BZ$, i.e. a polynomial
\[
  F(x,y)=ax^2+bxy+cy^2, a,b,c\in\BZ
\]
$F$ is said to be \emph{primitive} if its coefficients are relatively prime, i.e. $\gcd(a,b,c)=1$, and is said to \emph{represent} an integer $n$ if there are $r,s\in\BZ$ such that $F(r,s)=n$. If $\gcd(r,s)=1$, $F$ is said to represent $n$ \emph{properly}. The theory of integral binary quadratic forms was developed by Gauss, Lagrange, and Legendre.

In what follows, ``form'' means ``integral binary quadratic form''.

Following the article on quadratic forms, two such forms $F(x,y)$ and $G(x,y)$ are \emph{equivalent} if there is a matrix $M\in GL(2,\BZ)$ such that
\[
  G(x,y) = F(M(x,y)^T)
\]
Matrices in $GL(2,\BZ)$ are matrices with determinant $\pm 1$. So if $\alpha,\beta,\gamma, \delta\in\BZ$ and
\begin{displaymath}
det\left( \begin{array}{cc}
\alpha & \gamma \\
\beta & \delta
\end{array}\right) = \pm 1
\end{displaymath}
then if
\[G(x,y)=F(\alpha x + \beta y, \gamma x+\delta y)\]
it follows that $G$ is equivalent to $F$. If $M\in SL(2,\BZ)$ (i.e. $\det M = 1$), we say that $F$ and $G$ are \emph{properly equivalent}, written $F\sim G$; otherwise, they are \emph{improperly equivalent}.

Note that while both equivalence and proper equivalence are equivalence relations, improper equivalence is not. For if $F$ is improperly equivalent to $G$ and $G$ is improperly equivalent to $H$, then the product of the transformation matrices has determinant $1$, so that $F$ is properly equivalent to $H$. Since proper equivalence is an equivalence relation, we will say that two forms are in the same \emph{class} if they are properly equivalent.

$GL(2,\BZ)$ is generated as a multiplicative group by the two matrices
\[\begin{pmatrix}1&0\\1&1\end{pmatrix},\quad\begin{pmatrix}0&1\\1&0\end{pmatrix}\]
so in particular we see that we can construct all equivalence transformations by composing the following three transformations:
\begin{center}
\begin{tabular}{c|c|c}
Transformation & Matrix & Determinant\\
	\hline
$(x,y)\mapsto(y,x)$ & $\left(\begin{array}{cc}0&1\\1&0\end{array}\right)$ & $-1$\\[1.5em]
$(x,y)\mapsto(y,-x)$ & $\left(\begin{array}{cc}0&-1\\1&0\end{array}\right)$ & $1$\\[1.5em]
$(x,y)\mapsto(x+dy,y)$ & $\left(\begin{array}{cc}1&0\\d&1\end{array}\right)$ & $1$
\end{tabular}
\end{center}

\textbf{Example}: Let $F(x,y)=x^2+xy+6y^2$, $G(x,y)=82x^2+51xy+8y^2$. Then
\[G(x,y)=F(\alpha x + \beta y, \gamma x+\delta y)=F(4x+y,3x+y)\]
so
\begin{displaymath}
\left( \begin{array}{cc}
\alpha & \gamma \\
\beta & \delta
\end{array}\right)
=
\left( \begin{array}{cc}
4 & 3 \\
1 & 1
\end{array}\right)
\end{displaymath}
The transformations to map $F$ into $G$ are
\[
\begin{array}{rl}
x^2+xy+6y^2&(x,y)\mapsto(x+y,y)\\[2pt]
x^2+3xy+8y^2&(x,y)\mapsto(y,x)\\[2pt]
8x^2+3xy+y^2&(x,y)\mapsto(x+3y,y)\\[2pt]
8x^2+51xy+82y^2&(x,y)\mapsto(y,x)\\[2pt]
82x^2+51xy+8y^2
\end{array}
\]
and
\[
\left( \begin{array}{cc}
4 & 3 \\
1 & 1
\end{array}\right)=
\left( \begin{array}{cc}
0 & 1 \\
1 & 0
\end{array}\right)
\left( \begin{array}{cc}
1 & 0 \\
3 & 1
\end{array}\right)
\left( \begin{array}{cc}
0 & 1 \\
1 & 0
\end{array}\right)
\left( \begin{array}{cc}
1 & 0 \\
1 & 1
\end{array}\right)
\]
%%%%%
%%%%%
\end{document}
