\documentclass[12pt]{article}
\usepackage{pmmeta}
\pmcanonicalname{BinomialEquation}
\pmcreated{2013-03-22 17:46:05}
\pmmodified{2013-03-22 17:46:05}
\pmowner{pahio}{2872}
\pmmodifier{pahio}{2872}
\pmtitle{binomial equation}
\pmrecord{9}{40223}
\pmprivacy{1}
\pmauthor{pahio}{2872}
\pmtype{Definition}
\pmcomment{trigger rebuild}
\pmclassification{msc}{11C08}
\pmclassification{msc}{12E05}
\pmrelated{Binomial}
\pmrelated{CalculatingTheNthRootsOfAComplexNumber}
\pmrelated{RootOfUnity}
\pmdefines{cyclotomic equation}

\endmetadata

% this is the default PlanetMath preamble.  as your knowledge
% of TeX increases, you will probably want to edit this, but
% it should be fine as is for beginners.

% almost certainly you want these
\usepackage{amssymb}
\usepackage{amsmath}
\usepackage{amsfonts}

% used for TeXing text within eps files
%\usepackage{psfrag}
% need this for including graphics (\includegraphics)
%\usepackage{graphicx}
% for neatly defining theorems and propositions
 \usepackage{amsthm}
% making logically defined graphics
%%%\usepackage{xypic}

% there are many more packages, add them here as you need them

% define commands here

\theoremstyle{definition}
\newtheorem*{thmplain}{Theorem}

\begin{document}
{\em Binomial equation} is an algebraic equation of the \PMlinkescapetext{simple type}
$$x^n-a = 0$$
where $n$ is a positive integer,  $a$ belongs to a certain field (or sometimes to a certain ring) and $x$ is the unknown (or the indeterminate) of the equation.\, Solving the binomial equation means taking the \PMlinkname{$n$th root}{NthRoot} of $a$.

The binomial equation is written also
$$x^n = a.$$
Such a binomial equation may be examined in a group, too.

A special case of the binomial equation is the {\em cyclotomic equation}
$$x^n-1 = 0.$$
This name comes from the fact that the \PMlinkname{roots}{Equation} of the equation \PMlinkescapetext{divide} the unit circle in the complex plane into $n$ equally long arcs (Greek  $\varkappa\acute{\upsilon}\varkappa\lambda{o}\varsigma$ `circle', \,$\tau\acute{o}\mu{o}\varsigma$ `part').
%%%%%
%%%%%
\end{document}
