\documentclass[12pt]{article}
\usepackage{pmmeta}
\pmcanonicalname{PandigitalNumber}
\pmcreated{2013-03-22 16:04:28}
\pmmodified{2013-03-22 16:04:28}
\pmowner{PrimeFan}{13766}
\pmmodifier{PrimeFan}{13766}
\pmtitle{pandigital number}
\pmrecord{4}{38131}
\pmprivacy{1}
\pmauthor{PrimeFan}{13766}
\pmtype{Definition}
\pmcomment{trigger rebuild}
\pmclassification{msc}{11A63}
\pmdefines{zeroless pandigital number}

\endmetadata

% this is the default PlanetMath preamble.  as your knowledge
% of TeX increases, you will probably want to edit this, but
% it should be fine as is for beginners.

% almost certainly you want these
\usepackage{amssymb}
\usepackage{amsmath}
\usepackage{amsfonts}

% used for TeXing text within eps files
%\usepackage{psfrag}
% need this for including graphics (\includegraphics)
%\usepackage{graphicx}
% for neatly defining theorems and propositions
%\usepackage{amsthm}
% making logically defined graphics
%%%\usepackage{xypic}

% there are many more packages, add them here as you need them

% define commands here

\begin{document}
Given a base $b$ integer $$n = \sum_{i = 1}^k d_ib^{i - 1}$$ where $d_1$ is the least significant digit and $d_k$ is the most significant, and $k \ge b$, if for each $-1 < m < b$ there is at least one $d_x = m$ among the digits of $n$, then $n$ is a {\em pandigital number} in base $b$.

The smallest pandigital number in base $b$ is $$b^{b - 1} + \sum_{d = 2}^{b - 1} db^{(b - 1) - d},$$ while the largest (with only one instance of each digit) is $$\sum_{d = 1}^{b - 1} db^d.$$

There are infinitely many pandigital numbers with more than one instance of one or more digits.

If $b$ is not prime, a pandigital number must have at least $b + 1$ digits to be prime. With $k = b$ for the length of digits of a pandigital number $n$, it follows from the divisibility rules in that base that $(b - 1)|n$.

Sometimes a number with at least one instance each of the digits 1 through $b - 1$ but no instances of 0 is called a {\em zeroless pandigital number}.
%%%%%
%%%%%
\end{document}
