\documentclass[12pt]{article}
\usepackage{pmmeta}
\pmcanonicalname{AlternateProofOfMobiusInversionFormula}
\pmcreated{2013-03-22 16:30:31}
\pmmodified{2013-03-22 16:30:31}
\pmowner{rm50}{10146}
\pmmodifier{rm50}{10146}
\pmtitle{alternate proof of M\"obius inversion formula}
\pmrecord{4}{38684}
\pmprivacy{1}
\pmauthor{rm50}{10146}
\pmtype{Proof}
\pmcomment{trigger rebuild}
\pmclassification{msc}{11A25}

\endmetadata

% this is the default PlanetMath preamble.  as your knowledge
% of TeX increases, you will probably want to edit this, but
% it should be fine as is for beginners.

% almost certainly you want these
\usepackage{amssymb}
\usepackage{amsmath}
\usepackage{amsfonts}

% used for TeXing text within eps files
%\usepackage{psfrag}
% need this for including graphics (\includegraphics)
%\usepackage{graphicx}
% for neatly defining theorems and propositions
%\usepackage{amsthm}
% making logically defined graphics
%%%\usepackage{xypic}

% there are many more packages, add them here as you need them

% define commands here

\begin{document}
The M\"obius inversion theorem can also be proved elegantly using the fact that arithmetic functions form a ring under $+$ and $*$. 

Let $I$ be the arithmetic function that is everywhere $1$. Then obviously if $\mu$ is the M\"obius function,
\[(\mu *I)(n) = \sum_{d|n}\mu(d)I\left(\frac{n}{d}\right) = \sum_{d|n}\mu(d) = \begin{cases}1&$n=1$\\0&\text{otherwise}\end{cases}\]
and thus $I*\mu=e$, where $e$ is the identity of the ring.

But then
\[f(n)=\sum_{d|n}g(d)=\sum_{d|n}g(d)I\left(\frac{n}{d}\right)\]
and so $f=g*I$. Thus $f*\mu=g*I*\mu=g$. But $g=f*\mu$ means precisely that
\[g(n)=\sum_{d|n}\mu(d)f\left(\frac{n}{d}\right)\]
and we are done.

The reverse equivalence is similar ($f*\mu=g\Rightarrow f*\mu*I=g*I\Rightarrow f=g*I$).
%%%%%
%%%%%
\end{document}
