\documentclass[12pt]{article}
\usepackage{pmmeta}
\pmcanonicalname{Nontotient}
\pmcreated{2013-03-22 15:51:20}
\pmmodified{2013-03-22 15:51:20}
\pmowner{PrimeFan}{13766}
\pmmodifier{PrimeFan}{13766}
\pmtitle{nontotient}
\pmrecord{6}{37840}
\pmprivacy{1}
\pmauthor{PrimeFan}{13766}
\pmtype{Definition}
\pmcomment{trigger rebuild}
\pmclassification{msc}{11A25}
\pmrelated{Noncototient}

% this is the default PlanetMath preamble.  as your knowledge
% of TeX increases, you will probably want to edit this, but
% it should be fine as is for beginners.

% almost certainly you want these
\usepackage{amssymb}
\usepackage{amsmath}
\usepackage{amsfonts}

% used for TeXing text within eps files
%\usepackage{psfrag}
% need this for including graphics (\includegraphics)
%\usepackage{graphicx}
% for neatly defining theorems and propositions
%\usepackage{amsthm}
% making logically defined graphics
%%%\usepackage{xypic}

% there are many more packages, add them here as you need them

% define commands here
\begin{document}
An integer $n > 0$ is called a {\it nontotient} if $N_{\phi}(n) = 0$, where $N_{\phi}$ is the totient valence function. This is the case of any odd integer greater than 1. This is also the case of twice a prime that isn't a Sophie Germain prime, one more than a prime, an oblong number that is the product of a prime and one more or less than that prime, etc.

Sloane's OEIS lists even values in A005277 and all values together in A007617.
%%%%%
%%%%%
\end{document}
