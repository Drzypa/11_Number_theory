\documentclass[12pt]{article}
\usepackage{pmmeta}
\pmcanonicalname{RareNumbers}
\pmcreated{2013-03-22 19:20:33}
\pmmodified{2013-03-22 19:20:33}
\pmowner{Kausthub}{26471}
\pmmodifier{Kausthub}{26471}
\pmtitle{Rare numbers}
\pmrecord{4}{42291}
\pmprivacy{1}
\pmauthor{Kausthub}{26471}
\pmtype{Definition}
\pmcomment{trigger rebuild}
\pmclassification{msc}{11A25}
\pmdefines{Numbers when added or subtracted to its reverse gives a perfect square}

\endmetadata

% this is the default PlanetMath preamble.  as your knowledge
% of TeX increases, you will probably want to edit this, but
% it should be fine as is for beginners.

% almost certainly you want these
\usepackage{amssymb}
\usepackage{amsmath}
\usepackage{amsfonts}

% used for TeXing text within eps files
%\usepackage{psfrag}
% need this for including graphics (\includegraphics)
%\usepackage{graphicx}
% for neatly defining theorems and propositions
%\usepackage{amsthm}
% making logically defined graphics
%%%\usepackage{xypic}

% there are many more packages, add them here as you need them

% define commands here

\begin{document}
Rare numbers are the non palindromic numbers when added or subtracted to its reverse gives a perfect square. For example 65 is a rare number because 65 + 56 = 11<sup>2</sup> and 65 - 56 = 3<sup>2</sup>. Other examples of rare numbers are 65, 621770, 281089082, 2022652202, 868591084757, 872546974178 ... (Sequence A035519 of OEIS). If we consider palindromic rare numbers, there are infinitely many rare numbers. For example, the numbers in the series 242, 20402, 2004002, 200040002, 20000400002 ... are palindromic rare numbers. There are 84 rare numbers less than 10<sup>20</sup>.

Rare numbers exhibit certain properties. Rare numbers start always with an even digit and end in the digits 0, 2, 3, 7 and 8. Digital root of a rare number is always 2, 5, 8 and 9. Odd rare numbers are much fewer than even rare numbers. Also rare numbers with odd number of digits is fewer than rare numbers with even number of digits. After investigating rare numbers upto 10<sup>20</sup>, Shyam Sunder Gupta conjectured that there are no prime numbers. Also it is not known whether there are infinitely many rare numbers.
%%%%%
%%%%%
\end{document}
