\documentclass[12pt]{article}
\usepackage{pmmeta}
\pmcanonicalname{ExistenceAndUniquenessOfTheGcdOfTwoIntegers}
\pmcreated{2013-03-22 16:28:44}
\pmmodified{2013-03-22 16:28:44}
\pmowner{alozano}{2414}
\pmmodifier{alozano}{2414}
\pmtitle{existence and uniqueness of the gcd of two integers}
\pmrecord{5}{38645}
\pmprivacy{1}
\pmauthor{alozano}{2414}
\pmtype{Theorem}
\pmcomment{trigger rebuild}
\pmclassification{msc}{11-00}
%\pmkeywords{divisor}
%\pmkeywords{divisibility}
%\pmkeywords{gcd}
%\pmkeywords{greatest common divisor}
%\pmkeywords{division algorithm}
%\pmkeywords{well-ordering principle}
\pmrelated{GreatestCommonDivisor}
\pmrelated{DivisibilityInRings}
\pmrelated{DivisionAlgorithmForIntegers}
\pmrelated{WellOrderingPrinciple}

% this is the default PlanetMath preamble.  as your knowledge
% of TeX increases, you will probably want to edit this, but
% it should be fine as is for beginners.

% almost certainly you want these
\usepackage{amssymb}
\usepackage{amsmath}
\usepackage{amsfonts}
\usepackage{amsthm}

% used for TeXing text within eps files
%\usepackage{psfrag}
% need this for including graphics (\includegraphics)
%\usepackage{graphicx}
% for neatly defining theorems and propositions
%\usepackage{amsthm}
% making logically defined graphics
%%%\usepackage{xypic}

% there are many more packages, add them here as you need them

% define commands here
\theoremstyle{plain}
\newtheorem*{thm}{Theorem}
\newtheorem*{lem}{Lemma}
\newtheorem*{cor}{Corollary}


\begin{document}
\begin{thm}
Given two integers, at least one different from zero, there exists a unique natural number satisfying the definition of the greatest common divisor. 
\end{thm}
\begin{proof}
Let $a,b\in\mathbb{Z}$, where at least one of $a,b$ is nonzero. First we show existence. Define 
$S=\{ma+nb:m,n\in\mathbb{Z}\text{ and }ma+nb>0\}$. Now clearly $S$ is a subset of natural numbers, and $S$ is also nonempty, for depending upon the signs of $a$ and $b$, we may take $m=n=\pm 1$ to have $ma+nb\in S$. So, by the well-ordering principle for natural numbers, $S$ has a smallest element which we denote $g$.
Note that, by construction, $g=m_0a+n_0b$ for some $m_0,n_0\in\mathbb{Z}$. We will show that $g$ is a greatest common divisor of $a$ and $b$. Suppose first that $g\nmid a$. Then, by the division algorithm for integers, there exist unique $q,r\in\mathbb{Z}$, where $0\leq r< g$, such that $a=qg+r$. By assumption, $g\nmid a$, so we have
\begin{equation*}
0<r=a-qg=a-q(m_0a+n_0b)=a-qm_0a-qn_0b=(1-qm_0)a-qn_0b<g\text{.}
\end{equation*}
But then $r$ is an element of $S$ strictly less than $g$, contrary to assumption. Thus it must be that $g\mid a$. Similarly it can be shown that $g\mid b$. Now suppose $h\in\mathbb{N}$ is a divisor of both $a$ and $b$. Then there exist $k,l\in\mathbb{Z}$ such that $a=kh$ and $b=lh$, and we have
\begin{equation*}
g=m_0a+n_0b=m_0kh+n_0lh=(m_0k+n_0l)h\text{,}
\end{equation*}
so $h\mid g$. 
Thus $g$ is a greatest common divisor of $a$ and $b$. To see that $g$ is unique, suppose that $g'\in\mathbb{N}$ is also a greatest common divisor of $a$ and $b$. Then we have $g'\mid g$ and $g\mid g'$, whence $g=\pm g'$, and since $g,g'>0$, $g=g'$. 
\end{proof}
%%%%%
%%%%%
\end{document}
