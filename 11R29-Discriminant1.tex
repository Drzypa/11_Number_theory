\documentclass[12pt]{article}
\usepackage{pmmeta}
\pmcanonicalname{Discriminant1}
\pmcreated{2013-03-22 12:37:57}
\pmmodified{2013-03-22 12:37:57}
\pmowner{djao}{24}
\pmmodifier{djao}{24}
\pmtitle{discriminant}
\pmrecord{12}{32895}
\pmprivacy{1}
\pmauthor{djao}{24}
\pmtype{Definition}
\pmcomment{trigger rebuild}
\pmclassification{msc}{11R29}
\pmrelated{IntegralBasis}
\pmrelated{PolynomialDiscriminant}
\pmrelated{ModularDiscriminant}
\pmdefines{discriminant ideal}

\endmetadata

% this is the default PlanetMath preamble.  as your knowledge
% of TeX increases, you will probably want to edit this, but
% it should be fine as is for beginners.

% almost certainly you want these
\usepackage{amssymb}
\usepackage{amsmath}
\usepackage{amsfonts}

% used for TeXing text within eps files
%\usepackage{psfrag}
% need this for including graphics (\includegraphics)
%\usepackage{graphicx}
% for neatly defining theorems and propositions
%\usepackage{amsthm}
% making logically defined graphics
%%%\usepackage{xypic}

% there are many more packages, add them here as you need them

% define commands here
\newcommand{\abs}[1]{\left| #1 \right|}
\newcommand{\Tr}{\operatorname{Tr}}
\begin{document}
\section{Definitions}

Let $R$ be any Dedekind domain with field of fractions $K$. Fix a finite dimensional field extension $L/K$ and let $S$ denote the integral closure of $R$ in $L$. For any basis $x_1, \ldots, x_n$ of $L$ over $K$, the determinant
$$
\Delta(x_1,\ldots,x_n) := \det[\Tr(x_i x_j)],
$$
whose entries are the trace of $x_i x_j$ over all pairs $i,j$, is called the {\em discriminant} of the basis $x_1,\ldots,x_n$. The ideal in $R$ generated by all discriminants of the form
$$
\Delta(x_1,\ldots,x_n),\ \ x_i \in S
$$
is called the {\em discriminant ideal} of $S$ over $R$, and denoted $\Delta(S/R)$.

In the special case where $S$ is a free $R$--module, the discriminant ideal $\Delta(S/R)$ is always a principal ideal, generated by any discriminant of the form $\Delta(x_1,\ldots,x_n)$ where $x_1,\ldots,x_n$ is a basis for $S$ as an $R$--module. In particular, this situation holds whenever $K$ and $L$ are number fields.

\section{Alternative notations}

The discriminant is sometimes denoted with $\operatorname{disc}$ instead of $\Delta$. In the context of number fields, one often writes $\operatorname{disc}(L/K)$ for $\operatorname{disc}(\mathcal{O}_L/\mathcal{O}_K)$ where $\mathcal{O}_L$ and $\mathcal{O}_K$ are the rings of algebraic integers of $L$ and $K$. If $K$ or $\mathcal{O}_K$ is omitted, it is typically assumed to be $\mathbb{Q}$ or $\mathbb{Z}$.

\section{Properties}

The discriminant is so named because it allows one to determine which ideals of $R$ are ramified in $S$. Specifically, the prime ideals of $R$ that ramify in $S$ are precisely the ones that contain the discriminant ideal $\Delta(S/R)$. In the case $R = \mathbb{Z}$, \PMlinkname{a theorem of Minkowski}{MinkowskisConstant} \PMlinkescapetext{states} that any ring of integers $S$ of a number field larger than $\mathbb{Q}$ has discriminant strictly smaller than $\mathbb{Z}$ itself, and this fact combined with the previous result shows that any number field $K \neq \mathbb{Q}$ admits at least one ramified prime over $\mathbb{Q}$.

\section{Other types of discriminants}

In the special case where $L = K[x]$ is a primitive separable field extension of degree $n$, the discriminant $\Delta(1,x,\ldots,x^{n-1})$ is equal to the \PMlinkname{polynomial discriminant}{PolynomialDiscriminant} of the minimal polynomial $f(X)$ of $x$ over $K[X]$.

The discriminant of an elliptic curve can be obtained by taking the polynomial discrimiant of its Weierstrass polynomial, and the modular discriminant of a complex lattice equals the discriminant of the elliptic curve represented by the corresponding lattice quotient.
%%%%%
%%%%%
\end{document}
