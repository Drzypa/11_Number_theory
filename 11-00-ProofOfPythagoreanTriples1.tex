\documentclass[12pt]{article}
\usepackage{pmmeta}
\pmcanonicalname{ProofOfPythagoreanTriples1}
\pmcreated{2013-03-22 17:44:34}
\pmmodified{2013-03-22 17:44:34}
\pmowner{rm50}{10146}
\pmmodifier{rm50}{10146}
\pmtitle{proof of Pythagorean triples}
\pmrecord{5}{40192}
\pmprivacy{1}
\pmauthor{rm50}{10146}
\pmtype{Proof}
\pmcomment{trigger rebuild}
\pmclassification{msc}{11-00}

\endmetadata

% this is the default PlanetMath preamble.  as your knowledge
% of TeX increases, you will probably want to edit this, but
% it should be fine as is for beginners.

% almost certainly you want these
\usepackage{amssymb}
\usepackage{amsmath}
\usepackage{amsfonts}

% used for TeXing text within eps files
%\usepackage{psfrag}
% need this for including graphics (\includegraphics)
%\usepackage{graphicx}
% for neatly defining theorems and propositions
%\usepackage{amsthm}
% making logically defined graphics
%%%\usepackage{xypic}

% there are many more packages, add them here as you need them

% define commands here
\newcommand{\Rats}{\mathbb{Q}}
\newcommand{\Ints}{\mathbb{Z}}
\DeclareMathOperator{\N}{N}
\begin{document}
\PMlinkescapeword{right}
\PMlinkescapeword{side}
Suppose that $a^2+b^2=1$ where $a,b\in\Rats$. $a^2+b^2=N_{\Rats(i)/\Rats}(a+bi)$ (here $\N$ is the norm), so $a^2+b^2=1$ if and only if $N_{\Rats(i)/\Rats}(a+bi)=1$. $\Rats(i)$ is cyclic over $\Rats$ with Galois group isomorphic to $\Ints/2\Ints$, so by Hilbert's Theorem 90, there is some element $s+ti\in\Rats(i)$ such that
\[a+bi=\frac{s+ti}{\sigma(s+ti)}=\frac{s+ti}{s-ti}=\frac{s^2-t^2+2sti}{s^2+t^2}\]
so that
\[a=\frac{s^2-t^2}{s^2+t^2}, \qquad b=\frac{2st}{s^2+t^2}\]

Now, given any integer right triangle $p,q,r$ with $p^2+q^2=r^2$, we have
\[\left(\frac{p}{r}\right)^2+\left(\frac{q}{r}\right)^2=1\]
where $p/r, q/r\in\Rats$, so for some $s,t\in\Rats$,
\[\frac{p}{r}=\frac{s^2-t^2}{s^2+t^2}, \qquad \frac{q}{r}=\frac{2st}{s^2+t^2}\]
Clearing fractions on the right hand side of these equations by multiplying numerator and denominator by the square of the least common multiple of the denominators of $s, t$, we get
\[\frac{p}{r}=\frac{m^2-n^2}{m^2+n^2}, \qquad \frac{q}{r}=\frac{2mn}{m^2+n^2}\]
for $m,n\in\Ints$. Thus for some $d\in\Ints$,
\[p=d(m^2-n^2), \qquad q=2mnd, \qquad r=d(m^2+n^2)\]

%%%%%
%%%%%
\end{document}
