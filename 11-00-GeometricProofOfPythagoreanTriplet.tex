\documentclass[12pt]{article}
\usepackage{pmmeta}
\pmcanonicalname{GeometricProofOfPythagoreanTriplet}
\pmcreated{2013-03-22 18:36:20}
\pmmodified{2013-03-22 18:36:20}
\pmowner{rm50}{10146}
\pmmodifier{rm50}{10146}
\pmtitle{geometric proof of Pythagorean triplet}
\pmrecord{6}{41337}
\pmprivacy{1}
\pmauthor{rm50}{10146}
\pmtype{Proof}
\pmcomment{trigger rebuild}
\pmclassification{msc}{11-00}
\pmrelated{RationalSineAndCosine}

% this is the default PlanetMath preamble.  as your knowledge
% of TeX increases, you will probably want to edit this, but
% it should be fine as is for beginners.

% almost certainly you want these
\usepackage{amssymb}
\usepackage{amsmath}
\usepackage{amsfonts}

% used for TeXing text within eps files
%\usepackage{psfrag}
% need this for including graphics (\includegraphics)
%\usepackage{graphicx}
% for neatly defining theorems and propositions
%\usepackage{amsthm}
% making logically defined graphics
%%%\usepackage{xypic}
\usepackage{pstricks}

% there are many more packages, add them here as you need them

% define commands here

\begin{document}
If $x^2+y^2=z^2$ for $x,y,z$ positive integers is a pythagorean triple, then dividing through by $z^2$, we can write this in the form $r^2+s^2=1$ for positive rational numbers $r,s$. There is thus a 1-1 correspondence between primitive pythagorean triples (i.e. those for which $x,y$, and $z$ are pairwise coprime) and rational points in the first quadrant on the unit circle.

To find all such points on the unit circle, consider the following diagram:
\begin{center}
\begin{pspicture}
\psframe(-5,-1)(5,5.2)
\psline(0,-1)(0,5)
\psline(-5.2,0)(5.2,0)
\psarc(0,0){4}{350}{190}
\uput{.2}[135](0,2){1}
\psdot*(0,0)
\uput{.2}[225](0,0){$O$}
\psdot*(-4,0)
\uput{.2}[315](-4,0){$A=(-1,0)$}
\psdot*(3.4641,2)
\uput{.2}[45](3.4641,2){$B=(x,t(x+1))$}
\psdot*(3.4641,0)
\uput{.2}[270](3.4641,0){$(x,0)$}
\psline(3.4641,2)(3.4641,0)
\psline(-4,0)(3.4641,2)
\rput(0,5){.}
%\rput(-5.2,0){.}
\rput(7,0){.}
\rput(0,-1){.}
\end{pspicture}
\end{center}
The line from $A$ to $B$ is the line $y=t(x+1)$; we parametrize this as $t$ ranges over $[0,1]$ to capture all points in the first quadrant.

Substituting $y=t(x+1)$ back into the equation for the unit circle, we get
\begin{gather*}
  1=x^2+y^2 = x^2 + t^2(x+1)^2 = (1+t^2)x^2 + 2t^2x+t^2\\
  (1+t^2)x^2 + 2t^2x + (t^2-1)=0
\end{gather*}
Solving for $x$ using the quadratic formula (or, alternatively, dividing this polynomial by the known factor $x+1)$, and computing $y$ using the equation of the line, we get
\[
  x = \frac{1-t^2}{1+t^2}=1-\frac{2t^2}{1+t^2},\quad y = \frac{2t}{1+t^2}
\]
So if both $x$ and $y$ are to be rational, we must have that both
\[
  \frac{2t^2}{1+t^2}\quad\text{and}\quad\frac{2t}{1+t^2}
\]
are rational, and thus their quotient $t$ must be rational. Writing $t=\frac{q}{p}$, we get
\[
  x = \frac{1-t^2}{1+t^2} = \frac{1-\left(\frac{q}{p}\right)^2}{1+\left(\frac{q}{p}\right)^2}
    = \frac{p^2-q^2}{p^2+q^2},
  \quad
  y = \frac{2\frac{q}{p}}{1+\left(\frac{q}{p}\right)^2}=\frac{2pq}{p^2+q^2}
\]
and then $x^2+y^2=1$ becomes
\[
  (p^2+q^2)^2 = (p^2-q^2)^2 + (2pq)^2
\]
which is the desired parametrization of the pythagorean triple.
%%%%%
%%%%%
\end{document}
