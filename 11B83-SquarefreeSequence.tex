\documentclass[12pt]{article}
\usepackage{pmmeta}
\pmcanonicalname{SquarefreeSequence}
\pmcreated{2013-03-22 11:55:36}
\pmmodified{2013-03-22 11:55:36}
\pmowner{akrowne}{2}
\pmmodifier{akrowne}{2}
\pmtitle{square-free sequence}
\pmrecord{12}{30640}
\pmprivacy{1}
\pmauthor{akrowne}{2}
\pmtype{Definition}
\pmcomment{trigger rebuild}
\pmclassification{msc}{11B83}
\pmsynonym{square free sequence}{SquarefreeSequence}

\usepackage{amssymb}
\usepackage{amsmath}
\usepackage{amsfonts}
\usepackage{graphicx}
%%%%\usepackage{xypic}
\begin{document}
A square-free sequence is a sequence which has no adjacent repeating subsequences of any length.  

The name ``square-free'' comes from notation: Let $\{s\}$ be a sequence.    Then $\{s,s\}$ is also a sequence, which we write ``compactly'' as $\{s^2\}$.  In the rest of this entry we use a compact notation, lacking commas or braces.  This notation is commonly used when dealing with sequences in the capacity of strings. Hence we can write $\{s,s\} = ss = s^2$.

Some examples:

\begin{itemize}
\item $xabcabcx = x(abc)^2x$, not a square-free sequence.
\item $abcdabc$ cannot have any subsequence written in square notation, hence it is a square-free sequence.
\item $ababab = (ab)^3 = ab(ab)^2$, not a square-free sequence.
\end{itemize}

Note that, while notationally similar to the number-theoretic sense of ``square-free,'' the two concepts are distinct.  For example, for integers $a$ and $b$ the product $aba = a^2b$, a square.   But as a sequence, $aba = \{a,b,a\}$; clearly lacking any commutativity that might allow us to shift elements.  Hence, the sequence $aba$ is square-free.
%%%%%
%%%%%
%%%%%
%%%%%
\end{document}
