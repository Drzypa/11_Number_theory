\documentclass[12pt]{article}
\usepackage{pmmeta}
\pmcanonicalname{ExamplesOfTrimorphicNumbers}
\pmcreated{2013-03-22 16:21:35}
\pmmodified{2013-03-22 16:21:35}
\pmowner{PrimeFan}{13766}
\pmmodifier{PrimeFan}{13766}
\pmtitle{examples of trimorphic numbers}
\pmrecord{4}{38496}
\pmprivacy{1}
\pmauthor{PrimeFan}{13766}
\pmtype{Example}
\pmcomment{trigger rebuild}
\pmclassification{msc}{11A63}

\endmetadata

% this is the default PlanetMath preamble.  as your knowledge
% of TeX increases, you will probably want to edit this, but
% it should be fine as is for beginners.

% almost certainly you want these
\usepackage{amssymb}
\usepackage{amsmath}
\usepackage{amsfonts}

% used for TeXing text within eps files
%\usepackage{psfrag}
% need this for including graphics (\includegraphics)
%\usepackage{graphicx}
% for neatly defining theorems and propositions
%\usepackage{amsthm}
% making logically defined graphics
%%%\usepackage{xypic}

% there are many more packages, add them here as you need them

% define commands here

\begin{document}
As with examples of 1-automorphic numbers, we shall limit ourselves to the range specified by the iterator $0 < i < b^3 - 1$ and the bases $1 < b < 17$. Again it is obvious that 1 is trimorphic regardless of the base.

It appears that a given base has more trimorphic numbers than 1-automorphic numbers, at least in the range observed. For binary, 1, 3, 5, 7 are trimorphic; 1, 2, 8, 26 in ternary; 1, 3, 7, 9, 15, 31, 33, 63 in quartal; 1, 4, 24, 124 in base 5. So far we've looked at four bases where the only 1-automorphic number in the range given is 1.

In base 6 we have: 1, 2, 3, 4, 5, 8, 9, 17, 19, 27, 28, 35, 53, 55, 80, 81, 107, 109, 135, 136, 161, 163, 189, 215, a veritable embarrasse de richesse.

Skipping ahead to base 10, it might find it useful to list only those trimorphic numbers that are not 1-automorphic: 4, 9, 24, 49, 51, 75, 99, 125, 249, 251, 375, 499 (see A056032 in Sloane's OEIS for more of these).

Like base 6, duodecimal also has a lot of trimorphic numbers in the given range: 1, 3, 4, 5, 7, 8, 9, 11, 17, 55, 63, 64, 71, 73, 80, 81, 89, 127, 135, 143, 161, 351, 512, 513, 703, 863, 865, 1025, 1215, 1216, 1377, 1567, 1727.

In hexadecimal, the trimorphic numbers in the given range are: 1, 7, 9, 15, 127, 129, 255, 2047, 2049, 4095. In hexadecimal these all end in 1, 7, 9 or F, and if we extend our search to $b^5 - 1$, we find that this still holds true.
%%%%%
%%%%%
\end{document}
