\documentclass[12pt]{article}
\usepackage{pmmeta}
\pmcanonicalname{ValuesOfDedekindZetaFunctionsOfRealQuadraticNumberFieldsAtNegativeIntegers}
\pmcreated{2013-03-22 16:01:27}
\pmmodified{2013-03-22 16:01:27}
\pmowner{alozano}{2414}
\pmmodifier{alozano}{2414}
\pmtitle{values of Dedekind zeta functions of real quadratic number fields at negative integers}
\pmrecord{5}{38064}
\pmprivacy{1}
\pmauthor{alozano}{2414}
\pmtype{Application}
\pmcomment{trigger rebuild}
\pmclassification{msc}{11R42}
\pmclassification{msc}{11M06}
\pmrelated{FactorizationOfTheDedekindZetaFunctionOfAnAbelianNumberField}

\endmetadata

% this is the default PlanetMath preamble.  as your knowledge
% of TeX increases, you will probably want to edit this, but
% it should be fine as is for beginners.

% almost certainly you want these
\usepackage{amssymb}
\usepackage{amsmath}
\usepackage{amsthm}
\usepackage{amsfonts}

% used for TeXing text within eps files
%\usepackage{psfrag}
% need this for including graphics (\includegraphics)
%\usepackage{graphicx}
% for neatly defining theorems and propositions
%\usepackage{amsthm}
% making logically defined graphics
%%%\usepackage{xypic}

% there are many more packages, add them here as you need them

% define commands here

\newtheorem{thm}{Theorem}
\newtheorem{defn}{Definition}
\newtheorem{prop}{Proposition}
\newtheorem{lemma}{Lemma}
\newtheorem{cor}{Corollary}

\theoremstyle{definition}
\newtheorem{exa}{Example}

% Some sets
\newcommand{\Nats}{\mathbb{N}}
\newcommand{\Ints}{\mathbb{Z}}
\newcommand{\Reals}{\mathbb{R}}
\newcommand{\Complex}{\mathbb{C}}
\newcommand{\Rats}{\mathbb{Q}}
\newcommand{\Gal}{\operatorname{Gal}}
\newcommand{\Cl}{\operatorname{Cl}}
\begin{document}
Let $K$ be a real quadratic number field of discriminant $D_K$ and let $\zeta(s,K)$ be the Dedekind zeta function associated to $K$. By the Siegel-Klingen Theorem, if $n>0$ then $\zeta(-n,K)$ is a rational number. On the other hand, $K$ is obviously an abelian number field, thus the factorization of the Dedekind zeta function of an abelian number field tells us that:
$$\zeta(s,K)=\zeta(s)L(s,\chi)$$
where $\zeta(s)$ is the famous Riemann zeta function and $L(s,\chi)$ is the Dirichlet L-function associated to $\chi$, where $\chi$ is the unique Dirichlet character with conductor $D_K$ such that the group of characters of $K/\Rats$ is $\{ \chi_0, \chi \}$ and $\chi_0$ is the trivial character. In fact, the values of $\chi$ are simply given by $$\chi(a)=\left(\frac{D_K}{a}\right)$$ where the parentheses denote the Kronecker symbol.

Furthermore, if $k$ is a positive integer then:
\begin{enumerate}
\item Putting the values of the Riemann zeta function in terms of Bernoulli numbers one gets: $$\zeta(1-k)=-\frac{B_k}{k}$$
where $B_k$ is the $k$th Bernoulli number;

\item The values of Dirichlet L-series at negative integers can be written in terms of generalized Bernoulli numbers as follows:
$$L(1-k,\chi)= -\frac{B_{k,\chi}}{k}$$
where $B_{k,\chi}$ is the $k$th generalized Bernoulli number associated to $\chi$.
\end{enumerate}

Therefore:
$$\zeta(1-k,K)=\zeta(1-k)L(1-k,\chi)=\frac{B_k \cdot B_{k,\chi}}{k^2}.$$
The interested reader can find tables of values at the \PMlinkexternal{author's personal website}{http://www.math.cornell.edu/~alozano/dedekind-values/index.html}.
%%%%%
%%%%%
\end{document}
