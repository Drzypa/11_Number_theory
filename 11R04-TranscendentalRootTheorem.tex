\documentclass[12pt]{article}
\usepackage{pmmeta}
\pmcanonicalname{TranscendentalRootTheorem}
\pmcreated{2013-03-22 14:04:23}
\pmmodified{2013-03-22 14:04:23}
\pmowner{mathcam}{2727}
\pmmodifier{mathcam}{2727}
\pmtitle{transcendental root theorem}
\pmrecord{8}{35431}
\pmprivacy{1}
\pmauthor{mathcam}{2727}
\pmtype{Theorem}
\pmcomment{trigger rebuild}
\pmclassification{msc}{11R04}

% this is the default PlanetMath preamble.  as your knowledge
% of TeX increases, you will probably want to edit this, but
% it should be fine as is for beginners.

% almost certainly you want these
\usepackage{amssymb}
\usepackage{amsmath}
\usepackage{amsfonts}
\usepackage{amsthm}
% used for TeXing text within eps files
%\usepackage{psfrag}
% need this for including graphics (\includegraphics)
%\usepackage{graphicx}
% for neatly defining theorems and propositions
%\usepackage{amsthm}
% making logically defined graphics
%%%\usepackage{xypic}

% there are many more packages, add them here as you need them

% define commands here
\begin{document}
Suppose a constant $x$ is transcendental over some field $F$. Then $\sqrt[n]{x}$ is also transcendental over $F$ for any $n\geq 1$. 

\begin{proof}
Let $\overline{F}$ denote an algebraic closure of $F$.  Assume for the sake of contradiction that $\sqrt[n]{x}\in\overline{F}$.  Then since algebraic numbers are closed under multiplication (and thus exponentiation by positive integers), we have $(\sqrt[n]{x})^n=x\in \overline{F}$, so that $x$ is algebraic over $F$, creating a contradiction.
\end{proof}
%%%%%
%%%%%
\end{document}
