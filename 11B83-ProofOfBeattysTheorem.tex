\documentclass[12pt]{article}
\usepackage{pmmeta}
\pmcanonicalname{ProofOfBeattysTheorem}
\pmcreated{2013-03-22 13:18:58}
\pmmodified{2013-03-22 13:18:58}
\pmowner{lieven}{1075}
\pmmodifier{lieven}{1075}
\pmtitle{proof of Beatty's theorem}
\pmrecord{8}{33823}
\pmprivacy{1}
\pmauthor{lieven}{1075}
\pmtype{Proof}
\pmcomment{trigger rebuild}
\pmclassification{msc}{11B83}

% this is the default PlanetMath preamble.  as your knowledge
% of TeX increases, you will probably want to edit this, but
% it should be fine as is for beginners.

% almost certainly you want these
\usepackage{amssymb}
\usepackage{amsmath}
\usepackage{amsfonts}

% used for TeXing text within eps files
%\usepackage{psfrag}
% need this for including graphics (\includegraphics)
%\usepackage{graphicx}
% for neatly defining theorems and propositions
%\usepackage{amsthm}
% making logically defined graphics
%%%\usepackage{xypic}

% there are many more packages, add them here as you need them

% define commands here
\begin{document}
We define $a_n:=np$ and $b_n:=nq$. Since $p$ and $q$ are irrational, so are $a_n$ and $b_n$.

It is also the case that $a_n\neq b_m$ for all $m$ and $n$, for if $np=mq$ then $q=1+\frac{n}{m}$ would be rational.

The theorem is equivalent with the statement that for each integer $N\geq 1$ exactly $1$ element of $\{a_n\}\cup\{b_n\}$ lies in $(N,N+1)$.

Choose $N$ integer. Let $s(N)$ be the number of elements of $\{a_n\}\cup\{b_n\}$ less than $N$. 

$$a_n<N\Leftrightarrow np<N\Leftrightarrow n<\frac{N}{p}$$

So there are $\lfloor{\frac{N}{p}}\rfloor$ elements of $\{a_n\}$ less than $N$ and likewise $\lfloor{\frac{N}{q}}\rfloor$ elements of $\{b_n\}$.

By definition,

$$
\begin{array}{ccccc}
\frac{N}{p}-1&<& \lfloor {\frac{N}{p}}\rfloor&<&\frac{N}{p}\\
\frac{N}{q}-1&<& \lfloor {\frac{N}{q}}\rfloor&<&\frac{N}{q}\\
\end{array}
$$

and summing these inequalities gives $N-2<s(N)<N$ which gives that $s(N)=N-1$ since $s(N)$ is integer.

The number of elements of $\{a_n\}\cup\{b_n\}$ lying in $(N,N+1)$ is then $s(N+1)-s(N)=1$.
%%%%%
%%%%%
\end{document}
