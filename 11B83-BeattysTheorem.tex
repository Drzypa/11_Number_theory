\documentclass[12pt]{article}
\usepackage{pmmeta}
\pmcanonicalname{BeattysTheorem}
\pmcreated{2013-03-22 11:56:34}
\pmmodified{2013-03-22 11:56:34}
\pmowner{drini}{3}
\pmmodifier{drini}{3}
\pmtitle{Beatty's theorem}
\pmrecord{6}{30695}
\pmprivacy{1}
\pmauthor{drini}{3}
\pmtype{Theorem}
\pmcomment{trigger rebuild}
\pmclassification{msc}{11B83}
\pmrelated{Sequence}
\pmrelated{Irrational}
\pmrelated{Partition}
\pmrelated{Floor}
\pmrelated{Ceiling}
\pmrelated{BeattySequence}
\pmrelated{FraenkelsPartitionTheorem}
\pmrelated{FraenkelsPartitionTheorem2}
\pmrelated{ConjugateIndex}

\endmetadata

\usepackage{amssymb}
\usepackage{amsmath}
\usepackage{amsfonts}
\usepackage{graphicx}
%%%\usepackage{xypic}
\begin{document}
If $p$ and $q$ are positive irrationals such that
$$\frac{1}{p}+\frac{1}{q}=1$$
then the sequences
\begin{eqnarray*}
\{\lfloor np\rfloor\}_{n=1}^\infty&=&\lfloor p\rfloor,\lfloor 2p\rfloor,\lfloor 3p\rfloor,\ldots\\
\{\lfloor nq\rfloor\}_{n=1}^\infty&=&\lfloor q\rfloor,\lfloor 2q\rfloor,\lfloor 3q\rfloor,\ldots\\
\end{eqnarray*}
where $\lfloor x\rfloor$ denotes the floor (or greatest integer function) of $x$, constitute a partition of the set of positive integers.

That is, every positive integer is a member exactly once of one of the two sequences and the two sequences have no common terms.
%%%%%
%%%%%
%%%%%
\end{document}
