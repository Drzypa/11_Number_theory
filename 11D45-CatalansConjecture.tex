\documentclass[12pt]{article}
\usepackage{pmmeta}
\pmcanonicalname{CatalansConjecture}
\pmcreated{2014-12-16 16:16:07}
\pmmodified{2014-12-16 16:16:07}
\pmowner{pahio}{2872}
\pmmodifier{pahio}{2872}
\pmtitle{Catalan's conjecture}
\pmrecord{8}{40198}
\pmprivacy{1}
\pmauthor{pahio}{2872}
\pmtype{Conjecture}
\pmcomment{trigger rebuild}
\pmclassification{msc}{11D45}
\pmclassification{msc}{11D61}
\pmsynonym{Mihailescu's theorem}{CatalansConjecture}
\pmrelated{FermatsLastTheorem}
\pmrelated{SolutionsOfXyYx}

% this is the default PlanetMath preamble.  as your knowledge
% of TeX increases, you will probably want to edit this, but
% it should be fine as is for beginners.

% almost certainly you want these
\usepackage{amssymb}
\usepackage{amsmath}
\usepackage{amsfonts}

% used for TeXing text within eps files
%\usepackage{psfrag}
% need this for including graphics (\includegraphics)
%\usepackage{graphicx}
% for neatly defining theorems and propositions
 \usepackage{amsthm}
% making logically defined graphics
%%%\usepackage{xypic}

% there are many more packages, add them here as you need them

% define commands here

\theoremstyle{definition}
\newtheorem*{thmplain}{Theorem}

\begin{document}
The successive positive integers 8 and 9 are integer powers of positive integers ($2^3$ and $3^2$), with exponents greater than 1. {\em Catalan's conjecture} (1844) said that there are no other such successive positive integers, i.e. that the only integer solution of the Diophantine equation
$$x^m-y^n = 1$$
with\; $x > 1$,\, $y > 1$,\, $m > 1$,\, $n > 1$\; is
$$x = n = 3, \quad y = m = 2.$$

It took more than 150 years before the conjecture was proven.  Mihailescu gave in 2002 a proof in which he used the theory of cyclotomic fields and Galois modules.

For details, see e.g. 
\PMlinkexternal{this article}{http://www.ams.org/journals/bull/2004-41-01/S0273-0979-03-00993-5/S0273-0979-03-00993-5.pdf}.

See also a related problem concerning the equation 
\PMlinkname{$x^y = y^x$}{solutionsofxyyx}.
%%%%%
%%%%%
\end{document}
