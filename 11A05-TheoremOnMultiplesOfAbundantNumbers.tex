\documentclass[12pt]{article}
\usepackage{pmmeta}
\pmcanonicalname{TheoremOnMultiplesOfAbundantNumbers}
\pmcreated{2013-03-22 16:05:46}
\pmmodified{2013-03-22 16:05:46}
\pmowner{CompositeFan}{12809}
\pmmodifier{CompositeFan}{12809}
\pmtitle{theorem on multiples of abundant numbers}
\pmrecord{15}{38158}
\pmprivacy{1}
\pmauthor{CompositeFan}{12809}
\pmtype{Theorem}
\pmcomment{trigger rebuild}
\pmclassification{msc}{11A05}
\pmrelated{APositiveMultipleOfAnAbundantNumberIsAbundant}

% this is the default PlanetMath preamble.  as your knowledge
% of TeX increases, you will probably want to edit this, but
% it should be fine as is for beginners.

% almost certainly you want these
\usepackage{amssymb}
\usepackage{amsmath}
\usepackage{amsfonts}

% used for TeXing text within eps files
%\usepackage{psfrag}
% need this for including graphics (\includegraphics)
%\usepackage{graphicx}
% for neatly defining theorems and propositions
%\usepackage{amsthm}
% making logically defined graphics
%%%\usepackage{xypic}

% there are many more packages, add them here as you need them

% define commands here

\begin{document}
Theorem. The product $nm$ of an abundant number $n$ and any integer $m > 0$ is also an abundant number, regardless of the abundance or deficiency of $m$.

Proof. Choose an abundant number $n$ with $k$ divisors $d_1, \ldots ,d_k$ (where the divisors are sorted in ascending order and $d_1 = 1$, $d_k = n$) that add up to $2n + a$, where $a > 0$ is the abundance of $n$. For maximum flair, set $a = 1$, the bare minimum for abundance (that is, a quasiperfect number). Next, for $m$ choose a spectacularly deficient number such that $\gcd(m, n) = 1$, preferably some large prime number. If we choose a prime number, its divisors will only add up to $m + 1$. However, the divisors of $nm$ will include each $d_im$, where $d_i$ is a divisor of $n$ and $0 < i \le k$. Therefore, the divisors of $nm$ will add up to $$\sum_{i = 1}^k d_i + \sum_{i = 1}^k d_im = 2nm + a(m + 1) + 2n.$$

It now becomes obvious that by insisting that $m$ and $n$ be coprime we are guaranteeing that \PMlinkescapetext{even} if $m$ is itself prime, it will bring at least $k$ new divisors to the table. But what if $\gcd(m, n) > 1$, or in the most extreme case, $m = n$? In such a case, we just can't use the same formula for the sum of divisors of $nm$ that we used when $m$ and $n$ were coprime, as that would count some divisors twice. However, $m = n$ still brings new divisors to the table, and those new divisors add up to $$\sum_{i = 2}^k d_id_k = 2n^2 + 2a^2 + a.$$

Having proven these extreme cases, it is obvious that $nm$ will be abundant in other cases, such as $m$ being a composite deficient number, a perfect number, an abundant number sharing some but not all prime factors with $n$, etc.
%%%%%
%%%%%
\end{document}
