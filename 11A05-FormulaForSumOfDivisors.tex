\documentclass[12pt]{article}
\usepackage{pmmeta}
\pmcanonicalname{FormulaForSumOfDivisors}
\pmcreated{2013-03-22 16:47:35}
\pmmodified{2013-03-22 16:47:35}
\pmowner{rspuzio}{6075}
\pmmodifier{rspuzio}{6075}
\pmtitle{formula for sum of divisors}
\pmrecord{14}{39026}
\pmprivacy{1}
\pmauthor{rspuzio}{6075}
\pmtype{Theorem}
\pmcomment{trigger rebuild}
\pmclassification{msc}{11A05}

% this is the default PlanetMath preamble.  as your knowledge
% of TeX increases, you will probably want to edit this, but
% it should be fine as is for beginners.

% almost certainly you want these
\usepackage{amssymb}
\usepackage{amsmath}
\usepackage{amsfonts}

% used for TeXing text within eps files
%\usepackage{psfrag}
% need this for including graphics (\includegraphics)
%\usepackage{graphicx}
% for neatly defining theorems and propositions
\usepackage{amsthm}
% making logically defined graphics
%%%\usepackage{xypic}

% there are many more packages, add them here as you need them

% define commands here

\newtheorem{thm}{Theorem}
\newtheorem{Definition}{def}
\begin{document}
\PMlinkescapeword{multiplicities} \PMlinkescapeword{multiplicity} 
If one knows the factorization of a number,
one can compute the sum of the positive divisors of
that number without having to write down
all the divisors of that number.  To do 
this, one can use a formula which is obtained
by summing a geometric series.\\

\begin{thm}
Suppose that $n$ is a positive integer whose factorization
into prime factors is $\prod_{i=1}^k p_i^{m_i}$,
where the $p_i$'s are distinct primes and the
multiplicities $m_i$ are all at least $1$.  Then
the sum of the divisors of $n$ equals
\[
\prod_{i=1}^k
{p_i^{m_i + 1} - 1 \over p_i - 1}
\]
and the sum of the proper divisors of $n$ equals
\[
\prod_{i=1}^k
{p_i^{m_i + 1} - 1 \over p_i - 1} -
\prod_{i=1}^k p_i^{m_i}.
\]
\end{thm}

\begin{proof}
A number will divide $n$ if and only if prime
factors are also prime factors of $n$ and
their multiplicity is less than to or equal
to their multiplicities in $n$.  In other
words, a divisors $n$ can be expressed
as $\prod_{i=1}^k p_i^{\mu_i}$ where $0 \le
\mu_i \le m_i$.  Then the sum over all divisors
becomes the sum over all possible choices
for the $\mu_i$'s:
\[
\sum_{d \mid n} d =
\sum_{0 \le \mu_i \le m_i}
\prod_{i=1}^k p_i^{\mu_i}
\]
This sum may be expressed as a multiple 
sum like so:
\[
\sum_{\mu_1 = 0}^{m_1} 
\sum_{\mu_2 = 0}^{m_2} \cdots
\sum_{\mu_k = 0}^{m_k}
\prod_{i=1}^k p_i^{\mu_i}
\]
This sum of products may be factored into
a product of sums:
\[
\prod_{i=1}^k
\left( 
\sum_{\mu_i = 0}^{m_i}
p_i^{\mu_i}
\right)
\]
Each of these sums is a geometric series;
hence we may use the formula for sum of a
geometric series to conclude
\[
\sum_{d \mid n} d =
\prod_{i=1}^k
{p_i^{m_i + 1} - 1 \over p_i - 1}.
\]
If we want only proper divisors, we should
not include $n$ in the sum, so we obtain
the formula for proper divisors by subtracting
$n$ from our formula.\\
\end{proof}

As an illustration, let us compute the sum
of the divisors of $1800$.  The factorization
of our number is $2^3 \cdot 3^2 \cdot 5^2$.
Therefore, the sum of its divisors equals
\[
\left( {2^4 - 1} \over {2 - 1} \right)
\left( {3^3 - 1} \over {3 - 1} \right)
\left( {5^3 - 1} \over {5 - 1} \right) =
{15 \cdot 26 \cdot 124 \over 2 \cdot 4} =
6045.
\]
The sum of the proper divisors equals
$6045 - 1800 = 4245$,\, so we see that
$1800$ is an abundant number.
%%%%%
%%%%%
\end{document}
