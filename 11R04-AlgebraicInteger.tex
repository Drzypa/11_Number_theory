\documentclass[12pt]{article}
\usepackage{pmmeta}
\pmcanonicalname{AlgebraicInteger}
\pmcreated{2013-03-22 11:45:41}
\pmmodified{2013-03-22 11:45:41}
\pmowner{KimJ}{5}
\pmmodifier{KimJ}{5}
\pmtitle{algebraic integer}
\pmrecord{13}{30210}
\pmprivacy{1}
\pmauthor{KimJ}{5}
\pmtype{Definition}
\pmcomment{trigger rebuild}
\pmclassification{msc}{11R04}
\pmclassification{msc}{62-01}
\pmclassification{msc}{03-01}
%\pmkeywords{algebraic number theory}
\pmrelated{IntegralBasis}
\pmrelated{CyclotomicUnitsAreAlgebraicUnits}
\pmrelated{FundamentalUnits}
\pmrelated{Monic2}
\pmrelated{RingWithoutIrreducibles}

\usepackage{amssymb}
\usepackage{amsmath}
\usepackage{amsfonts}
\usepackage{graphicx}
%%%%\usepackage{xypic}
\begin{document}
Let $K$ be an \PMlinkname{extension}{ExtensionField} of $\mathbb{Q}$ contained in $\mathbb{C}$.  A number $\alpha \in K$ is called an \emph{algebraic integer} of $K$ if it is the root of a monic polynomial with coefficients in $\mathbb{Z}$, i.e., an element of $K$ that is integral over $\mathbb{Z}$. Every algebraic integer is an algebraic number (with $K = \mathbb{C}$), but the converse is false.
%%%%%
%%%%%
%%%%%
%%%%%
\end{document}
