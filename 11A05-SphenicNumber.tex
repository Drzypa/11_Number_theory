\documentclass[12pt]{article}
\usepackage{pmmeta}
\pmcanonicalname{SphenicNumber}
\pmcreated{2013-03-22 16:10:33}
\pmmodified{2013-03-22 16:10:33}
\pmowner{CompositeFan}{12809}
\pmmodifier{CompositeFan}{12809}
\pmtitle{sphenic number}
\pmrecord{7}{38262}
\pmprivacy{1}
\pmauthor{CompositeFan}{12809}
\pmtype{Definition}
\pmcomment{trigger rebuild}
\pmclassification{msc}{11A05}

\endmetadata

% this is the default PlanetMath preamble.  as your knowledge
% of TeX increases, you will probably want to edit this, but
% it should be fine as is for beginners.

% almost certainly you want these
\usepackage{amssymb}
\usepackage{amsmath}
\usepackage{amsfonts}

% used for TeXing text within eps files
%\usepackage{psfrag}
% need this for including graphics (\includegraphics)
%\usepackage{graphicx}
% for neatly defining theorems and propositions
%\usepackage{amsthm}
% making logically defined graphics
%%%\usepackage{xypic}

% there are many more packages, add them here as you need them

% define commands here

\begin{document}
Given three primes $p < q < r$, the composite integer $pqr$ is a {\em sphenic number}. The first few sphenic numbers are 30, 42, 66, 70, 78, 102, 105, 110, 114, 130, 138, 154, $\ldots$ listed in A007304 of Sloane's OEIS.

The divisors of a sphenic number therefore are $1, p, q, r, pq, pr, qr, pqr$. Furthermore, $\mu(pqr) = (-1)^3$ (where $\mu$ is the M\"obius function), $\tau(pqr) = 8$ (where $\tau$ is the divisor function) and $\Omega(pqr) = \omega(pqr) = 3$ (where $\Omega$ and $\omega$ are the \PMlinkname{number of (nondistinct) prime factors function}{NumberOfNondistinctPrimeFactorsFunction} and the number of distinct prime factors function, respectively).

The largest known sphenic number at any time is usually the product of the three largest known Mersenne primes.
%%%%%
%%%%%
\end{document}
