\documentclass[12pt]{article}
\usepackage{pmmeta}
\pmcanonicalname{EliminatedSierpinskiNumberCandidates}
\pmcreated{2013-03-22 17:21:14}
\pmmodified{2013-03-22 17:21:14}
\pmowner{PrimeFan}{13766}
\pmmodifier{PrimeFan}{13766}
\pmtitle{eliminated Sierpi\'nski number candidates}
\pmrecord{6}{39712}
\pmprivacy{1}
\pmauthor{PrimeFan}{13766}
\pmtype{Example}
\pmcomment{trigger rebuild}
\pmclassification{msc}{11A51}

\endmetadata

% this is the default PlanetMath preamble.  as your knowledge
% of TeX increases, you will probably want to edit this, but
% it should be fine as is for beginners.

% almost certainly you want these
\usepackage{amssymb}
\usepackage{amsmath}
\usepackage{amsfonts}

% used for TeXing text within eps files
%\usepackage{psfrag}
% need this for including graphics (\includegraphics)
%\usepackage{graphicx}
% for neatly defining theorems and propositions
%\usepackage{amsthm}
% making logically defined graphics
%%%\usepackage{xypic}

% there are many more packages, add them here as you need them

% define commands here

\begin{document}
Most numbers $k$ are very easy to eliminate as Sierpi\'nski number candidates, as it is very easy to come up with sequences of primes of the form $k2^n + 1$. For example, for $k = 1$ it is enough to mention the Fermat primes. For some of the seventeen Sierpi\'nski number candidates when the Seventeen or Bust project began, only a single Proth prime, and it is often quite large. Eight candidates remain to be eliminated. The primes listed below were discovered by the Seventeen or Bust project, with the latest being for 19249, discovered by user Konstantin Agafonov of team TSC! Russia.

\begin{tabular}{|r|l|l|}
$k$ & Exponent $n$ which gives prime & Prime in base 10 scientific notation \\
4847 & 3321063 & $1.844857508381060 \times 10^{999743}$ \\
5359 & 5054502 & $2.781168752802502 \times 10^{1521560}$ \\
10223 & Still a candidate & N/A \\
19249 & 13018586 & $1.484360328715661 \times 10^{3918989}$ \\
21181 & Still a candidate & N/A \\
22699 & Still a candidate & N/A \\
24737 & Still a candidate & N/A \\
27653 & 9167433 & $5.727724120920733 \times 10^{2759676}$ \\
28433 & 7830457 & $7.772839072447348 \times 10^{2357206}$ \\
44131 & Still a candidate & N/A \\
46157 & Still a candidate & N/A \\
54767 & Still a candidate & N/A \\
55459 & 995972 & $1.234767571821004 \times 10^{299822}$ \\
65567 & 1013803 & $8.499227304459893 \times 10^{305189}$ \\
67607 & Still a candidate & N/A \\
69109 & 1157446 & $6.366429016367452 \times 10^{348430}$ \\
\end{tabular}

%%%%%
%%%%%
\end{document}
