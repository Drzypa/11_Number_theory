\documentclass[12pt]{article}
\usepackage{pmmeta}
\pmcanonicalname{Totative}
\pmcreated{2013-03-22 16:58:16}
\pmmodified{2013-03-22 16:58:16}
\pmowner{CompositeFan}{12809}
\pmmodifier{CompositeFan}{12809}
\pmtitle{totative}
\pmrecord{9}{39246}
\pmprivacy{1}
\pmauthor{CompositeFan}{12809}
\pmtype{Definition}
\pmcomment{trigger rebuild}
\pmclassification{msc}{11A25}
\pmrelated{ResidueSystems}

% this is the default PlanetMath preamble.  as your knowledge
% of TeX increases, you will probably want to edit this, but
% it should be fine as is for beginners.

% almost certainly you want these
\usepackage{amssymb}
\usepackage{amsmath}
\usepackage{amsfonts}

% used for TeXing text within eps files
%\usepackage{psfrag}
% need this for including graphics (\includegraphics)
%\usepackage{graphicx}
% for neatly defining theorems and propositions
%\usepackage{amsthm}
% making logically defined graphics
%%%\usepackage{xypic}

% there are many more packages, add them here as you need them

% define commands here

\begin{document}
Given a positive integer $n$, an integer $0 < m < n$ is a {\em totative} of $n$ if $\gcd(m, n) = 1$. Put another way, all the smaller integers than $n$ that are coprime to $n$ are totatives of $n$.

For example, the totatives of 21 are 1, 2, 4, 5, 8, 10, 11, 13, 16, 17, 19 and 20.

The count of totatives of $n$ is Euler's totient function $\phi(n)$. The set of totatives of $n$ forms a reduced residue system modulo $n$. The word ``totative'' was coined by James Joseph Sylvester, who also coined ``totient'' (though despite occasional usage in some papers and books, the term ``totative'' has not caught on the way ``totient'' has).
%%%%%
%%%%%
\end{document}
