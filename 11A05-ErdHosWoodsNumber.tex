\documentclass[12pt]{article}
\usepackage{pmmeta}
\pmcanonicalname{ErdHosWoodsNumber}
\pmcreated{2013-03-22 17:37:14}
\pmmodified{2013-03-22 17:37:14}
\pmowner{PrimeFan}{13766}
\pmmodifier{PrimeFan}{13766}
\pmtitle{Erd\H{o}s-Woods number}
\pmrecord{4}{40040}
\pmprivacy{1}
\pmauthor{PrimeFan}{13766}
\pmtype{Definition}
\pmcomment{trigger rebuild}
\pmclassification{msc}{11A05}
\pmsynonym{Erdos-Woods number}{ErdHosWoodsNumber}
\pmsynonym{Erd\"os-Woods number}{ErdHosWoodsNumber}

% this is the default PlanetMath preamble.  as your knowledge
% of TeX increases, you will probably want to edit this, but
% it should be fine as is for beginners.

% almost certainly you want these
\usepackage{amssymb}
\usepackage{amsmath}
\usepackage{amsfonts}

% used for TeXing text within eps files
%\usepackage{psfrag}
% need this for including graphics (\includegraphics)
%\usepackage{graphicx}
% for neatly defining theorems and propositions
%\usepackage{amsthm}
% making logically defined graphics
%%%\usepackage{xypic}

% there are many more packages, add them here as you need them

% define commands here

\begin{document}
An integer $k$ is an {\em Erd\H{o}s-Woods number} if there is an integer $n$ such that each of the consecutive integers $n + i$ for $0 < i < k$ shares at least one prime factor with either $n$ or $n + k$. In other words, if for a $k$ there is an $n$ such that each evaluation of $\gcd(n, n + i) > 1$ or $\gcd(n + k, n + i) > 1$ returns true, then $k$ is an Erd\H{o}s-Woods number.

For example, one $n$ for $k = 16$ is 2184. 2184 is $2^3 \times 3 \times 7 \times 13$, while $2184 + 16 = 2200 = 2^3 \times 5^2 \times 11$. We then verify that

\begin{itemize}
\item 2185 is clearly divisible by 5 and thus shares that odd prime as a factor with 2200.
\item 2186 is even and so shares 2 as a factor with both 2184 and 2200.
\item 2187 is 3 more than 2184 and therefore must also be divisible by 3. In fact, it is $3^7$.
\item 2188 is even and so shares 2 as a factor with both 2184 and 2200, suggesting we needn't look at any other even numbers in this range.
\item 2189 is 11 less than 2200 and therefore must be divisible by 11. In base 10 we can quickly verify that 2 + 8 = 10 and 1 + 9 is also 10.
\item 2191 is 7 more than 2184 and thus must be divisible by 7.
\item 2193 is 9 more than 2184 and thus divisible by 3.
\item 2195 is obviously divisible by 5.
\item 2197 is 13 more than 2184 and thus must be divisible by 13. In fact, it is $13^3$.
\item 2199 is 15 more than 2184 and thus divisible by 3.
\end{itemize}

Knowing one $n$ for a given $k$ one can find other $n$ by multiplying the odd prime factors of $n + k$ (just once each, let's call that product ``$q$'') and then calculating $n(jq + 1)$ with $j$ any positive integer of one's choice. To give one example: with $n = 2184$ and $j =  17$, we get another $n$, namely 2044224. The range 2044224 to 2044240 displays the same patterns of factorization as described above, except that 2044227 and 2044237 are a semiprime and a sphenic number respectively, as opposed to 2187 and 2197 which are both prime powers.

Other Erd\H{o}s-Woods numbers are 22, 34, 36, 46, 56, 64, 66, 70, 76, 78, 86, 88, 92, 94, 96, 100, etc., listed in A059756 of Sloane's OEIS (the smallest odd Erd\H{o}s-Woods number is 903), while A059757 lists the smallest matching $n$ for each of those $k$.

\begin{thebibliography}{2}
\bibitem{rg} R. K. Guy, {\it Unsolved Problems in Number Theory} New York: Springer-Verlag 2004: B28
\end{thebibliography}
%%%%%
%%%%%
\end{document}
