\documentclass[12pt]{article}
\usepackage{pmmeta}
\pmcanonicalname{ProperDivisor}
\pmcreated{2013-03-22 15:52:00}
\pmmodified{2013-03-22 15:52:00}
\pmowner{PrimeFan}{13766}
\pmmodifier{PrimeFan}{13766}
\pmtitle{proper divisor}
\pmrecord{7}{37859}
\pmprivacy{1}
\pmauthor{PrimeFan}{13766}
\pmtype{Definition}
\pmcomment{trigger rebuild}
\pmclassification{msc}{11A51}
\pmsynonym{aliquot part}{ProperDivisor}
\pmsynonym{restricted divisor}{ProperDivisor}

% this is the default PlanetMath preamble.  as your knowledge
% of TeX increases, you will probably want to edit this, but
% it should be fine as is for beginners.

% almost certainly you want these
\usepackage{amssymb}
\usepackage{amsmath}
\usepackage{amsfonts}

% used for TeXing text within eps files
%\usepackage{psfrag}
% need this for including graphics (\includegraphics)
%\usepackage{graphicx}
% for neatly defining theorems and propositions
%\usepackage{amsthm}
% making logically defined graphics
%%%\usepackage{xypic}

% there are many more packages, add them here as you need them

% define commands here
\begin{document}
If a divisor $d$ of $n$ (that is, $d|n$) satisfies $0 < |d| < |n|$, then $d$ is a {\em proper divisor} of $n$. In the realm of real positive integers, it is usually considered sufficient to list the positive divisors. For example, the proper divisors of 42 are 1, 2, 3, 6, 7, 14, 21.

By restricting the sum of divisors to proper divisors, some $n$ will be less than this sum (deficient numbers, including prime numbers), some will be equal (perfect numbers) and some will be greater (abundant numbers). The term {\em restricted divisor} is sometimes used to further distinguish divisors in the range $1 < |d| < |n|$ (and sometimes it used to mean the same thing as proper divisor). Thus, in our example, the list would be shortened to 2, 3, 6, 7, 14, 21.
%%%%%
%%%%%
\end{document}
