\documentclass[12pt]{article}
\usepackage{pmmeta}
\pmcanonicalname{ProofThatEulervarphiFunctionIsMultiplicative}
\pmcreated{2013-03-22 15:03:40}
\pmmodified{2013-03-22 15:03:40}
\pmowner{Wkbj79}{1863}
\pmmodifier{Wkbj79}{1863}
\pmtitle{proof that Euler $\varphi$ function is multiplicative}
\pmrecord{12}{36781}
\pmprivacy{1}
\pmauthor{Wkbj79}{1863}
\pmtype{Proof}
\pmcomment{trigger rebuild}
\pmclassification{msc}{11A25}
\pmrelated{EulerPhiFunction}
\pmrelated{MultiplicativeFunction}
\pmrelated{EulerPhifunction}

\endmetadata

\usepackage{amsmath}
\begin{document}
\PMlinkescapeword{words}

Suppose that $t=mn$ where $m,n$ are coprime.  The \PMlinkname{Chinese remainder theorem}{ChineseRemainderTheorem} states that $\gcd(a,t)=1$ if and only if $\gcd(a,m)=1$ and $\gcd(a,n)=1$. 

In other words, there is a bijective correspondence between these two sets:

\begin{itemize}
\item $\{a : a\equiv 1\pmod{t}\}$
\item $\{a : a\equiv1\pmod{m}\text{ and } a\equiv1\pmod{ n} \}$
\end{itemize}

Now the number of positive integers not greater than $t$ and coprime with $t$ is precisely $\varphi(t)$, but it is also the number of pairs $(u,v)$, where $u$ not greater than $m$ and coprime with $m$, and $v$ not greater than $n$ and coprime with $n$.  Thus, $\varphi(mn)=\varphi(m)\varphi(n)$.
%%%%%
%%%%%
\end{document}
