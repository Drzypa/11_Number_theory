\documentclass[12pt]{article}
\usepackage{pmmeta}
\pmcanonicalname{ViswanathsConstant}
\pmcreated{2013-03-22 18:09:32}
\pmmodified{2013-03-22 18:09:32}
\pmowner{PrimeFan}{13766}
\pmmodifier{PrimeFan}{13766}
\pmtitle{Viswanath's constant}
\pmrecord{6}{40716}
\pmprivacy{1}
\pmauthor{PrimeFan}{13766}
\pmtype{Definition}
\pmcomment{trigger rebuild}
\pmclassification{msc}{11B39}
\pmsynonym{Viswanath constant}{ViswanathsConstant}

% this is the default PlanetMath preamble.  as your knowledge
% of TeX increases, you will probably want to edit this, but
% it should be fine as is for beginners.

% almost certainly you want these
\usepackage{amssymb}
\usepackage{amsmath}
\usepackage{amsfonts}

% used for TeXing text within eps files
%\usepackage{psfrag}
% need this for including graphics (\includegraphics)
%\usepackage{graphicx}
% for neatly defining theorems and propositions
%\usepackage{amsthm}
% making logically defined graphics
%%%\usepackage{xypic}

% there are many more packages, add them here as you need them

% define commands here

\begin{document}
{\em Viswanath's constant} $V \approx 1.1319882487943$ is a real number whose $n$th power approximates the absolute value of the $n$th term of some random Fibonacci sequences, especially as $n$ gets larger. In his 2000 paper, Divakar  Viswanath gave the value of the function to just eight decimal places as 1.13198824. Viswanath believed the logarithm of the constant to lie between 0.123975598 and 0.1239755995. Oliveira and Figuereido in 2002 computed the value again using interval arithmetic instead of Viswanath's ``detailed rounding-error analysis,'' in order to obtain ``slightly better results.'' Using Mathematica, Eric Weisstein computed a different value: 1.1321506910656020459.

The continued fraction of Viswanath's constant, which is not periodic, begins

$$1 + \frac{1}{1 + \frac{1}{7 + \frac{1}{1 + \frac{1}{\ddots}}}},$$

and aside from some instances of 2s, is thought to contain mostly odd numbers.

\begin{thebibliography}{1}
\bibitem{sf} S. R. Finch, {\it Mathematical Constants}. Cambridge: Cambridge University Press (2003): 1.2.4
\bibitem{jo} Jo\~ao Batista Oliveira \& Luiz Henrique de Figuereido, ``Interval Computation of Viswanath's Constant'' {\it Reliable Computing} {\bf 8} 2 (2002): 131 - 138
\bibitem{dv} Divakar Viswanath ``Random Fibonacci sequences and the number 1.13198824....'' {\it Mathematics of Computation} {\bf 69} 231 (2000): 1131 - 1155
\end{thebibliography}
%%%%%
%%%%%
\end{document}
