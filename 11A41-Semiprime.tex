\documentclass[12pt]{article}
\usepackage{pmmeta}
\pmcanonicalname{Semiprime}
\pmcreated{2013-03-22 12:49:22}
\pmmodified{2013-03-22 12:49:22}
\pmowner{drini}{3}
\pmmodifier{drini}{3}
\pmtitle{semiprime}
\pmrecord{9}{33145}
\pmprivacy{1}
\pmauthor{drini}{3}
\pmtype{Definition}
\pmcomment{trigger rebuild}
\pmclassification{msc}{11A41}
\pmsynonym{semi-prime}{Semiprime}
\pmsynonym{2-almost prime}{Semiprime}
%\pmkeywords{number theory}
%\pmkeywords{primes}
\pmdefines{almost prime}

\endmetadata

% this is the default PlanetMath preamble.  as your knowledge
% of TeX increases, you will probably want to edit this, but
% it should be fine as is for beginners.

% almost certainly you want these
\usepackage{amssymb}
\usepackage{amsmath}
\usepackage{amsfonts}

% used for TeXing text within eps files
%\usepackage{psfrag}
% need this for including graphics (\includegraphics)
%\usepackage{graphicx}
% for neatly defining theorems and propositions
%\usepackage{amsthm}
% making logically defined graphics
%%%\usepackage{xypic}

% there are many more packages, add them here as you need them

% define commands here
\begin{document}
A composite number which is the product of two (possibly equal) primes is called  \emph{semiprime}. Such numbers are sometimes also called 2-\emph{almost primes}. For example:

\begin{itemize}
\item 1 is not a semiprime because it is not a composite number or a prime, 
\item 2 is not a semiprime, as it is a prime,
\item 4 is a semiprime, since $4 = 2\cdot 2$,
\item 8 is not a semiprime, since it is a product of three primes ($8 = 2\cdot 2\cdot 2$),
\item 2003 is not a semiprime, as it is a prime,
\item 2005 is a semiprime, since $2005 = 5\cdot 401$,
\item 2007 is not a semiprime, since it is a product of three primes ($2007 = 3\cdot 3\cdot 223$).
\end{itemize} 

The first few semiprimes are $4, 6, 9, 10, 14, 15, 21, 22, 25, 26, 33, 34, 35, 38, 39, 46, 49, 51, 55, 57, 58, 62, \ldots$ (\PMlinkexternal{Sloane's sequence A001358}{http://www.research.att.com/cgi-bin/access.cgi/as/njas/sequences/eisA.cgi?Anum=001358}
). The Moebius function $\mu(n)$ for semiprimes can be only equal to 0 or 1. If we form an integer sequence of values of $\mu(n)$ for semiprimes we get a binary sequence: $0, 1, 0, 1, 1, 1, 1, 1, 0, 1, 1, 1, 1, 1, 1, 1, 0, 1, 1, 1, 1, 1, \ldots$. (\PMlinkexternal{Sloane's sequence A072165}{http://www.research.att.com/cgi-bin/access.cgi/as/njas/sequences/eisA.cgi?Anum=072165}
).

All the squares of primes are also semiprimes. The first few squares of primes are then $4, 9, 25, 49, 121, 169, 289, 361, 529, 841, 961, 1369, 1681, 1849, 2209, 2809, 3481, 3721, 4489, 5041, \ldots$. (\PMlinkexternal{Sloane's sequence A001248}{http://www.research.att.com/cgi-bin/access.cgi/as/njas/sequences/eisA.cgi?Anum=001248}
). The Moebius function $\mu(n)$ for the squares of primes is always equal to 0 as it is equal to 0 for all squares.
%%%%%
%%%%%
\end{document}
