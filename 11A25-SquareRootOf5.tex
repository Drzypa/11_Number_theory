\documentclass[12pt]{article}
\usepackage{pmmeta}
\pmcanonicalname{SquareRootOf5}
\pmcreated{2013-03-22 17:28:12}
\pmmodified{2013-03-22 17:28:12}
\pmowner{MathNerd}{17818}
\pmmodifier{MathNerd}{17818}
\pmtitle{square root of 5}
\pmrecord{6}{39855}
\pmprivacy{1}
\pmauthor{MathNerd}{17818}
\pmtype{Definition}
\pmcomment{trigger rebuild}
\pmclassification{msc}{11A25}

\endmetadata

% this is the default PlanetMath preamble.  as your knowledge
% of TeX increases, you will probably want to edit this, but
% it should be fine as is for beginners.

% almost certainly you want these
\usepackage{amssymb}
\usepackage{amsmath}
\usepackage{amsfonts}

% used for TeXing text within eps files
%\usepackage{psfrag}
% need this for including graphics (\includegraphics)
%\usepackage{graphicx}
% for neatly defining theorems and propositions
%\usepackage{amsthm}
% making logically defined graphics
%%%\usepackage{xypic}

% there are many more packages, add them here as you need them

% define commands here

\begin{document}
The \emph{square root of 5} is an irrational number involved in the formula for the golden ratio. It is also used in statistics when dealing with 5-business day weeks. Its decimal expansion begins 2.2360679774997896964, see \PMlinkexternal{sequence A002163}{http://www.research.att.com/~njas/sequences/A002163} in Sloane's OEIS. Its simple continued fraction is 2, 4, 4, 4, 4, 4, 4, 4, 4, 4, 4, 4, 4, 4, ... One formula for the square root of 5 involves some of the same numbers as in Euler's identity (but with a 2 instead of the 1): $e^{i\pi}+2\phi$. The square root of 5 modulo a prime number is employed in some ECM algorithms.

A rectangle with unit height and $\sqrt{5}$ width can be split into two golden rectangles of the same size and a square, or into two golden rectangles of different sizes.

The conjecture stating ``that any abelian surface with RM by $Q(\sqrt{5})$ is isogenous to a simple factor of the Jacobian of a modular curve $X_0(N)$ for some $N$'' still stands. John Wilson has produced equations for curves of genus 2 with Jacobians of the specified RM.

\begin{thebibliography}{1}
\bibitem{morain} Francois Morain. {\it Primality Proving Using Elliptic Curves: An Update}. Springer: Berlin (2004)
\bibitem{nemiroff} Robert Nemiroff and Jerry Bonnell. {\it A million digits of sqrt(5)} at Project Gutenberg \PMlinkexternal{http://www.gutenberg.org/dirs/etext96/5sqrt10.txt}{http://www.gutenberg.org/dirs/etext96/5sqrt10.txt}
\bibitem{pickover} Clifford Pickover. {\it Wonders of Numbers}, Oxford: Oxford University Press (2001) p. 106.
\bibitem{wilson} John Wilson  ``Curves of genus 2 with real multiplication by a square root of 5'' p. i Dissertation, Oxford University, Oxford (1998) \PMlinkexternal{http://eprints.maths.ox.ac.uk/32/01/wilson.pdf}{http://eprints.maths.ox.ac.uk/32/01/wilson.pdf}
\end{thebibliography}
%%%%%
%%%%%
\end{document}
