\documentclass[12pt]{article}
\usepackage{pmmeta}
\pmcanonicalname{SequenceOfBoundedVariation}
\pmcreated{2014-11-28 21:01:47}
\pmmodified{2014-11-28 21:01:47}
\pmowner{pahio}{2872}
\pmmodifier{pahio}{2872}
\pmtitle{sequence of bounded variation}
\pmrecord{7}{88202}
\pmprivacy{1}
\pmauthor{pahio}{2872}
\pmtype{Theorem}

\endmetadata

% this is the default PlanetMath preamble.  as your knowledge
% of TeX increases, you will probably want to edit this, but
% it should be fine as is for beginners.

% almost certainly you want these
\usepackage{amssymb}
\usepackage{amsmath}
\usepackage{amsfonts}

% need this for including graphics (\includegraphics)
\usepackage{graphicx}
% for neatly defining theorems and propositions
\usepackage{amsthm}

% making logically defined graphics
%\usepackage{xypic}
% used for TeXing text within eps files
%\usepackage{psfrag}

% there are many more packages, add them here as you need them

% define commands here

\begin{document}
The sequence 
\begin{align}
a_1, a_2, a_3, \ldots
\end{align}
of complex numbers is said to be {\it of bounded variation}, iff it satisfies 
$$\sum_{n=1}^\infty|a_n\!-\!a_{n+1}| \;<\; \infty.$$
Cf. function of bounded variation. See also 
\PMlinkname{contractive sequence}{ContractiveSequence}.\\

\textbf{Theorem.}\, Every sequence of bounded variation is
\PMlinkname{convergent}{ConvergentSequence}.\\

{\it Proof.}\, Let's have a sequence (1) of bounded variation.\, 
When $m < n$, we form the telescoping sum
$$a_m-a_n = \sum_{i=m}^{n-1}(a_i-a_{i+1})$$
from which we see that
$$|a_m-a_n| \;\leqq\; \sum_{i=m}^{n-1}|a_i-a_{i+1}|.$$
This inequality shows, by the Cauchy criterion for convergence of 
series, that the sequence (1) is a Cauchy sequence and thus 
converges. \Box\\

One kind of sequences of bounded variation is formed by the 
bounded monotonic sequences of real numbers (those sequences 
are convergent, as is well known).\, Indeed, if (1) is a bounded 
and e.g. monotonically nondecreasing sequence, then 
$$a_i \leqq a_{i+1} \quad\mbox{  for each }i,$$
whence 
\begin{align}
\sum_{i=1}^n|a_{i+1}-a_i| = \sum_{i=1}^n(a_{i+1}-a_i) 
                            = a_{n+1}-a_1.
\end{align}                            
The boundedness of (1) thus implies that the partial sums (2) of 
the series $\sum_{i=1}^\infty|a_{i+1}-a_i|$ with nonnegative 
terms are bounded.\, Therefore the last series is convergent, i.e. 
our sequence (1) is of bounded variarion.

\begin{thebibliography}{8}
\bibitem{loya}{\sc Paul Loya}: {\it Amazing and Aesthetic
Aspects of Analysis: On the incredible infinite}.\, A Course in Undergraduate Analysis, Fall 2006.\; 
Available in http://www.math.binghamton.edu/dennis/478.f07/EleAna.pdf
\end{thebibliography}
\end{document}
