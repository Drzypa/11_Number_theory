\documentclass[12pt]{article}
\usepackage{pmmeta}
\pmcanonicalname{ProbabilityThatTwoPositiveIntegersAreRelativelyPrime}
\pmcreated{2013-03-22 14:56:08}
\pmmodified{2013-03-22 14:56:08}
\pmowner{mps}{409}
\pmmodifier{mps}{409}
\pmtitle{probability that two positive integers are relatively prime}
\pmrecord{21}{36625}
\pmprivacy{1}
\pmauthor{mps}{409}
\pmtype{Result}
\pmcomment{trigger rebuild}
\pmclassification{msc}{11A41}
\pmclassification{msc}{11A05}
\pmclassification{msc}{11A51}
%\pmkeywords{Divisibility}
%\pmkeywords{Probability}
%\pmkeywords{Pi}

\endmetadata

% this is the default PlanetMath preamble.  as your knowledge
% of TeX increases, you will probably want to edit this, but
% it should be fine as is for beginners.

% almost certainly you want these
\usepackage{amssymb}
\usepackage{amsmath}
\usepackage{amsfonts}

% used for TeXing text within eps files
%\usepackage{psfrag}
% need this for including graphics (\includegraphics)
%\usepackage{graphicx}
% for neatly defining theorems and propositions
%\usepackage{amsthm}
% making logically defined graphics
%%%\usepackage{xypic}

% there are many more packages, add them here as you need them

% define commands here
\begin{document}
\PMlinkescapeword{event}
\PMlinkescapeword{language}
The probability that two positive integers chosen randomly are 
relatively prime is 
\[
\frac{6}{\pi^{2}} = 0.60792710185\dots.
\]

At first glance this ``naked'' result is beautiful, but no 
suitable definition is there: there isn't a probability space 
defined. Indeed, the word ``probability'' here is an abuse of
language.
So, now, let's write the mathematical statement.

For each $n\in\mathbb{Z}^+$, let $S_n$ be the set $\{1,2,\dots,n\}\times\{1,2,\dots,n\}$ and define
$\Sigma_n$ to be the powerset of $S_n$.  Define 
$\mu\colon\Sigma_n\to\mathbb{R}$ by $\mu(E)=|E|/|S_n|$.  This makes
$(S_n,\Sigma_n,\mu)$ into a probability space.

We wish to consider the event of some $(x,y)\in S_n$ also being 
in the set
$A_n=\{(a,b)\in S_n\colon\gcd(a,b)=1\}$.
The probability of this event is 
\[
P((x,y)\in A_n)=\int_{S_n} \chi_{A_n} \,d\mu=\frac{|A_n|}{|S_n|}.
\]
Our statement is thus the following.  For each $n\in\mathbb{Z}^+$,
select random integers $x_n$ and $y_n$ with $1\le x_n, y_n\le n$.
Then the limit $\lim_{n\to\infty}P((x_n,y_n)\in A_n)$ exists and
\[
\lim_{n\to\infty}P((x_n,y_n)\in A_n)=\frac{6}{\pi^2}.
\]
In other words, as $n$ gets large, the fraction of $|S_n|$ consisting of relatively prime pairs of positive integers tends to $6/\pi^2$.

\begin{thebibliography}{9}
\bibitem{book} Challenging Mathematical Problems with Elementary Solutions, A.M. Yaglom and I.M. Yaglom, Vol. 1, Holden-Day, 1964. (See Problems 92 and 93)
\end{thebibliography}
%%%%%
%%%%%
\end{document}
