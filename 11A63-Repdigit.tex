\documentclass[12pt]{article}
\usepackage{pmmeta}
\pmcanonicalname{Repdigit}
\pmcreated{2013-03-22 16:20:14}
\pmmodified{2013-03-22 16:20:14}
\pmowner{CompositeFan}{12809}
\pmmodifier{CompositeFan}{12809}
\pmtitle{repdigit}
\pmrecord{5}{38468}
\pmprivacy{1}
\pmauthor{CompositeFan}{12809}
\pmtype{Definition}
\pmcomment{trigger rebuild}
\pmclassification{msc}{11A63}

\endmetadata

% this is the default PlanetMath preamble.  as your knowledge
% of TeX increases, you will probably want to edit this, but
% it should be fine as is for beginners.

% almost certainly you want these
\usepackage{amssymb}
\usepackage{amsmath}
\usepackage{amsfonts}

% used for TeXing text within eps files
%\usepackage{psfrag}
% need this for including graphics (\includegraphics)
%\usepackage{graphicx}
% for neatly defining theorems and propositions
%\usepackage{amsthm}
% making logically defined graphics
%%%\usepackage{xypic}

% there are many more packages, add them here as you need them

% define commands here

\begin{document}
Given base $b$, a number of the form $d({{b^n - 1} \over {b - 1}})$ for $n > 0$ and $0 < d < b$ is written using using the digit $d$ only, $n$ times in that base and is therefore a {\em repdigit}. The term, short for "repeated digit," is credited to Beiler's book {\it Recreations in the theory of numbers}, in chapter 11.

When $d = 1$, the resulting repdigit is called a repunit. Only repunits can also be prime (and \PMlinkescapetext{even} then they are rare). No other repdigit can be prime since it is obvious that it is a multiple of a repunit.

In a trivial way, all repdigits are palindromic numbers.
%%%%%
%%%%%
\end{document}
