\documentclass[12pt]{article}
\usepackage{pmmeta}
\pmcanonicalname{CyclicNumber}
\pmcreated{2013-03-22 16:46:18}
\pmmodified{2013-03-22 16:46:18}
\pmowner{PrimeFan}{13766}
\pmmodifier{PrimeFan}{13766}
\pmtitle{cyclic number}
\pmrecord{5}{39000}
\pmprivacy{1}
\pmauthor{PrimeFan}{13766}
\pmtype{Definition}
\pmcomment{trigger rebuild}
\pmclassification{msc}{11N05}

% this is the default PlanetMath preamble.  as your knowledge
% of TeX increases, you will probably want to edit this, but
% it should be fine as is for beginners.

% almost certainly you want these
\usepackage{amssymb}
\usepackage{amsmath}
\usepackage{amsfonts}

% used for TeXing text within eps files
%\usepackage{psfrag}
% need this for including graphics (\includegraphics)
%\usepackage{graphicx}
% for neatly defining theorems and propositions
%\usepackage{amsthm}
% making logically defined graphics
%%%\usepackage{xypic}

% there are many more packages, add them here as you need them

% define commands here

\begin{document}
A {\em cyclic number} $n$ is an integer which in a given base $b$ maintains the same digits after repeated multiplications. For example, twice 142857 is 285714; thrice is 428571; four times is 571428; etc. Cyclic numbers are tied to the full reptend primes $p$ thus: $$n = \frac{b^{p - 1} - 1}{p}.$$ From this it follows that the base $b$ representation of the number $np$ consists of $p - 1$ instances of the digit $b - 1$. For $mn$ with $m > p$ the multiples gradually begin to exhibit greater variety in their digits, with the occasional number nearly full of digits $b - 1$.

In base 10, the first few cyclic numbers are 142857, 588235294117647, 52631578947368421, 434782608695652173913, 344827586206896551724137931, 212765957446808510638297872340425531914893617, etc.

There are no cyclic numbers in factorial base or primorial base because not every digit can occupy every significant position. For example, take the multiples of 142857 in factorial base: 34222111, 70444300, 115012011, 151234200, 185501311, 232024100, 266251211. The digit 4 can only occur to the left of $d_3$.

Confusingly, Sloane's OEIS uses this term to refer to the full reptend primes instead of the cyclic numbers they generate.
%%%%%
%%%%%
\end{document}
