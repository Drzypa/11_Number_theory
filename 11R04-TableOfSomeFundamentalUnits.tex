\documentclass[12pt]{article}
\usepackage{pmmeta}
\pmcanonicalname{TableOfSomeFundamentalUnits}
\pmcreated{2013-03-22 17:58:39}
\pmmodified{2013-03-22 17:58:39}
\pmowner{pahio}{2872}
\pmmodifier{pahio}{2872}
\pmtitle{table of some fundamental units}
\pmrecord{9}{40486}
\pmprivacy{1}
\pmauthor{pahio}{2872}
\pmtype{Result}
\pmcomment{trigger rebuild}
\pmclassification{msc}{11R04}
\pmclassification{msc}{11R11}
\pmclassification{msc}{11R27}
\pmrelated{UnitsOfQuadraticFields}
\pmrelated{QuadraticField}
\pmrelated{IntegralBasisOfQuadraticField}
\pmrelated{AlgebraicNumberTheory}

% this is the default PlanetMath preamble.  as your knowledge
% of TeX increases, you will probably want to edit this, but
% it should be fine as is for beginners.

% almost certainly you want these
\usepackage{amssymb}
\usepackage{amsmath}
\usepackage{amsfonts}

% used for TeXing text within eps files
%\usepackage{psfrag}
% need this for including graphics (\includegraphics)
%\usepackage{graphicx}
% for neatly defining theorems and propositions
 \usepackage{amsthm}
% making logically defined graphics
%%%\usepackage{xypic}

% there are many more packages, add them here as you need them

% define commands here

\theoremstyle{definition}
\newtheorem*{thmplain}{Theorem}

\begin{document}
Below, we tabulate the fundamental units $\eta$ of first real quadratic fields $\mathbb{Q}(\sqrt{d})$; the number $\omega$ is $\displaystyle\frac{1\!+\!\sqrt{d}}{2}$ for\, 
$d \equiv 1 \pmod{4}$\, and $\sqrt{d}$ for\, $d \not\equiv 1 \pmod{4}$.


\begin{center}
\begin{tabular}{||c|c||c|c||}
\hline\hline
$d$ & $\eta$ & $d$ & $\eta$\\

\hline\hline
$2$ & $1+\omega$ & $47$ & $48+7\omega$\\
\hline
$3$ & $2+\omega$ & $51$ & $50+7\omega$\\
\hline
$5$ & $\omega$ & $53$ & $3+\omega$\\
\hline
$6$ & $5+2\omega$ & $55$ & $89+12\omega$\\
\hline
$7$ & $8+3\omega$ & $57$ & $131+40\omega$\\
\hline
$10$ & $3+\omega$ & $58$ & $99+13\omega$\\
\hline
$11$ & $10+3\omega$ & $59$ & $530+69\omega$\\
\hline
$13$ & $1+\omega$ & $61$ & $17+5\omega$\\
\hline
$14$ & $15+4\omega$ & $62$ & $63+8\omega$\\
\hline
$15$ & $4+\omega$ & $65$ & $7+2\omega$\\
\hline
$17$ & $3+2\omega$ & $66$ & $65+8\omega$\\
\hline
$19$ & $170+39\omega$ & $67$ & $48842+5967\omega$\\
\hline
$21$ & $2+\omega$ & $69$ & $11+3\omega$\\
\hline
$22$ & $197+42\omega$ & $70$ & $251+30\omega$\\
\hline
$23$ & $24+5\omega$ & $71$ & $3480+413\omega$\\
\hline
$26$ & $5+\omega$ & $73$ & $943+250\omega$\\
\hline
$29$ & $2+\omega$ & $74$ & $43+5\omega$\\
\hline
$30$ & $11+2\omega$ & $77$ & $4+\omega$\\
\hline
$31$ & $1520+273\omega$ & $78$ & $53+6\omega$\\
\hline
$33$ & $19+8\omega$ & 79 & $80+9\omega$\\
\hline
$34$ & $35+6\omega$ & $82$ & $9+\omega$\\
\hline
$35$ & $6+\omega$ & $83$ & $82+9\omega$\\
\hline
$37$ & $5+2\omega$ & $85$ & $4+\omega$\\
\hline
$38$ & $37+6\omega$ & $86$ & $10405+1122\omega$\\
\hline
$39$ & $25+4\omega$ & $87$ & $28+3\omega$\\
\hline
$41$ & $27+10\omega$ & $89$ & $447+106\omega$\\
\hline
$42$ & $13+2\omega$ & $91$ & $1574+165\omega$\\
\hline
$43$ & $3482+531\omega$ & $93$ & $13+3\omega$\\
\hline
$46$ & $24335+3588\omega$ & $94$ & $2143295+221064\omega$\\
\hline
\end{tabular}
\end{center}

\begin{thebibliography}{9}
\bibitem{BS}{\sc S. Borewicz \& I. Safarevic}: {\em Zahlentheorie}.\, Birkh\"auser Verlag. Basel und Stuttgart (1966).
\end{thebibliography}

%%%%%
%%%%%
\end{document}
