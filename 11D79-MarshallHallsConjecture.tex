\documentclass[12pt]{article}
\usepackage{pmmeta}
\pmcanonicalname{MarshallHallsConjecture}
\pmcreated{2013-03-22 18:15:36}
\pmmodified{2013-03-22 18:15:36}
\pmowner{PrimeFan}{13766}
\pmmodifier{PrimeFan}{13766}
\pmtitle{Marshall Hall's conjecture}
\pmrecord{4}{40857}
\pmprivacy{1}
\pmauthor{PrimeFan}{13766}
\pmtype{Conjecture}
\pmcomment{trigger rebuild}
\pmclassification{msc}{11D79}
\pmsynonym{Marshall Hall conjecture}{MarshallHallsConjecture}
\pmrelated{PerfectPower}

% this is the default PlanetMath preamble.  as your knowledge
% of TeX increases, you will probably want to edit this, but
% it should be fine as is for beginners.

% almost certainly you want these
\usepackage{amssymb}
\usepackage{amsmath}
\usepackage{amsfonts}

% used for TeXing text within eps files
%\usepackage{psfrag}
% need this for including graphics (\includegraphics)
%\usepackage{graphicx}
% for neatly defining theorems and propositions
%\usepackage{amsthm}
% making logically defined graphics
%%%\usepackage{xypic}

% there are many more packages, add them here as you need them

% define commands here

\begin{document}
Conjecture. (Marshall Hall, Jr.). With the exception of $n^2$ being a perfect sixth power, for any positive integer $n$, the inequality $|n^2 - m^3| > C \sqrt{m}$, (with $m$ also being a positive integer and $C$ being a number less than 1 that nears 1 as $n$ tends to infinity) always holds.

The reason for the exception of perfect sixth powers (those cases of $n$ for which there is a solution to $n^2 = h^6$ in integers) is a simple consequence of associativity: if $n^2 = h^6$, then $h^6 = h^2 h^2 h^2 = h^3 h^3$. Then $m = h$ and $n^2 - m^3 = 0$. For example, $8^2 - 4^3 = 0$.

For small $n$, $C$ can't be exactly 1. For example, $3^2 - 2^3 = 1$, and $\sqrt{2} > 1$. But even among the smaller numbers, the conjecture generally holds even with $C = 1$. After $n = 3$, the next counterexample (that is not a perfect sixth power) to $C = 1$ is $n = 378661$, with the corresponding $m = 5234$ producing a difference of just 17. A078933 in Sloane's OEIS lists smaller values of $m$ with cubes being at a distance from the nearest square that is less than $\sqrt{m}$. Noam Elkies has found some fairly large counterexamples to setting $C = 1$, such as $n = 447884928428402042307918$ and $m = 5853886516781223$, the difference between the square of the former and the cube of the latter being a relatively small 1641843. 

\begin{thebibliography}{1}
\bibitem{rg} R. K. Guy, {\it Unsolved Problems in Number Theory} New York: Springer-Verlag 2004: D9
\end{thebibliography}
%%%%%
%%%%%
\end{document}
