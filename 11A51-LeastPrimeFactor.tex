\documentclass[12pt]{article}
\usepackage{pmmeta}
\pmcanonicalname{LeastPrimeFactor}
\pmcreated{2013-03-22 17:40:03}
\pmmodified{2013-03-22 17:40:03}
\pmowner{PrimeFan}{13766}
\pmmodifier{PrimeFan}{13766}
\pmtitle{least prime factor}
\pmrecord{5}{40104}
\pmprivacy{1}
\pmauthor{PrimeFan}{13766}
\pmtype{Definition}
\pmcomment{trigger rebuild}
\pmclassification{msc}{11A51}

% this is the default PlanetMath preamble.  as your knowledge
% of TeX increases, you will probably want to edit this, but
% it should be fine as is for beginners.

% almost certainly you want these
\usepackage{amssymb}
\usepackage{amsmath}
\usepackage{amsfonts}

% used for TeXing text within eps files
%\usepackage{psfrag}
% need this for including graphics (\includegraphics)
%\usepackage{graphicx}
% for neatly defining theorems and propositions
%\usepackage{amsthm}
% making logically defined graphics
%%%\usepackage{xypic}

% there are many more packages, add them here as you need them

% define commands here
\newcommand{\lpf}{\textrm{lpf}}
\begin{document}
The {\em least prime factor} of a positive integer $n$ is the smallest positive prime number dividing $n$. Sometimes expressed as a function, $\lpf(n)$. For example, $\lpf(91) = 7$. For a prime number $p$, clearly $\lpf(p) = p$, while for any composite number (except squares of primes) $(\lpf(n))^2 < n$. (The function would be quite useless if 1 is considered a prime, therefore $\lpf(1)$ is undefined --- though we could make an argument for $\lpf(0) = 2$). In the sequence of least prime factors for each integer in turn, each prime occurs first at the index for itself then not again until its square.

In Mathematica, one can use \verb=LeastPrimeFactor[n]= after loading a number theory package, or much more simply by using the command \verb=FactorInteger[n][[1,1]]= (of course substituting \verb=n= as necessary).
%%%%%
%%%%%
\end{document}
