\documentclass[12pt]{article}
\usepackage{pmmeta}
\pmcanonicalname{CriticalStrip}
\pmcreated{2013-03-22 16:07:21}
\pmmodified{2013-03-22 16:07:21}
\pmowner{Wkbj79}{1863}
\pmmodifier{Wkbj79}{1863}
\pmtitle{critical strip}
\pmrecord{11}{38191}
\pmprivacy{1}
\pmauthor{Wkbj79}{1863}
\pmtype{Definition}
\pmcomment{trigger rebuild}
\pmclassification{msc}{11M06}
\pmrelated{FormulaeForZetaInTheCriticalStrip}
\pmrelated{ValueOfTheRiemannZetaFunctionAtS0}
\pmrelated{AnalyticContinuationOfRiemannZeta}

\endmetadata

% this is the default PlanetMath preamble.  as your knowledge
% of TeX increases, you will probably want to edit this, but
% it should be fine as is for beginners.

% almost certainly you want these
\usepackage{amssymb}
\usepackage{amsmath}
\usepackage{amsfonts}

% used for TeXing text within eps files
%\usepackage{psfrag}
% need this for including graphics (\includegraphics)
%\usepackage{graphicx}
% for neatly defining theorems and propositions
%\usepackage{amsthm}
% making logically defined graphics
%%%\usepackage{xypic}

% there are many more packages, add them here as you need them

% define commands here

\begin{document}
The {\em critical strip} of the Riemann zeta function is $\{ s \in \mathbb{C}: 0 \le \operatorname{Re}(s) \le 1 \}$.  The zeroes of the Riemann zeta function outside of the critical strip are exactly the set of all negative even integers.  The location of the zeroes of the Riemann zeta function inside the critical strip is not totally known; \PMlinkescapetext{information} about these zeroes is crucial in analytic number theory and the \PMlinkescapetext{distribution} of primes.  The Riemann hypothesis asserts that all zeroes of the Riemann zeta function that are in the critical strip lie on the line $\operatorname{Re}(s)=\frac{1}{2}$.  This is all explained in more detail in the entry \PMlinkname{Riemann zeta function}{RiemannZetaFunction}.

It is well known that no zeroes of the Riemann zeta function lie on either of the lines $\operatorname{Re}(s)=0$ and $\operatorname{Re}(s)=1$.  (See \PMlinkname{this entry}{RiemannZetaFunctionHasNoZerosOnReS01} for a proof.)  Therefore, some people use the \PMlinkescapetext{term} ``critical strip'' to refer to the \PMlinkname{region}{Region} $\{ s \in \mathbb{C}: 0<\operatorname{Re}(s)<1 \}$.  (Note that this is the interior of the critical strip as defined above.)  For example, this usage \PMlinkescapetext{occurs} in the title of the entry formulae for zeta in the critical strip.
%%%%%
%%%%%
\end{document}
