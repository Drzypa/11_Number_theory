\documentclass[12pt]{article}
\usepackage{pmmeta}
\pmcanonicalname{HogattsTheorem}
\pmcreated{2013-03-22 13:43:24}
\pmmodified{2013-03-22 13:43:24}
\pmowner{mathcam}{2727}
\pmmodifier{mathcam}{2727}
\pmtitle{Hogatt's theorem}
\pmrecord{8}{34408}
\pmprivacy{1}
\pmauthor{mathcam}{2727}
\pmtype{Theorem}
\pmcomment{trigger rebuild}
\pmclassification{msc}{11B39}
%\pmkeywords{Fibonacci sum}
\pmrelated{FibonacciSequence}

\endmetadata

% this is the default PlanetMath preamble.  as your knowledge
% of TeX increases, you will probably want to edit this, but
% it should be fine as is for beginners.

% almost certainly you want these
\usepackage{amssymb}
\usepackage{amsmath}
\usepackage{amsfonts}
\usepackage{amsthm}

% used for TeXing text within eps files
%\usepackage{psfrag}
% need this for including graphics (\includegraphics)
%\usepackage{graphicx}
% for neatly defining theorems and propositions
%\usepackage{amsthm}
% making logically defined graphics
%%%\usepackage{xypic}

% there are many more packages, add them here as you need them

% define commands here

\newcommand{\mc}{\mathcal}
\newcommand{\mb}{\mathbb}
\newcommand{\mf}{\mathfrak}
\newcommand{\ol}{\overline}
\newcommand{\ra}{\rightarrow}
\newcommand{\la}{\leftarrow}
\newcommand{\La}{\Leftarrow}
\newcommand{\Ra}{\Rightarrow}
\newcommand{\nor}{\vartriangleleft}
\newcommand{\Gal}{\text{Gal}}
\newcommand{\GL}{\text{GL}}
\newcommand{\Z}{\mb{Z}}
\newcommand{\R}{\mb{R}}
\newcommand{\Q}{\mb{Q}}
\newcommand{\C}{\mb{C}}
\newcommand{\<}{\langle}
\renewcommand{\>}{\rangle}
\begin{document}
Hogatt's theorem states that every positive integer can be expressed as 
a sum of distinct Fibonacci numbers.

For any positive integer, $k\in\Z^+$, there exists a unique positive integer 
$n$ so that $F_{n-1} < k \leq F_n$. We proceed by strong induction on $n$.  For $k=0,1,2,3$, the property is true as $0,1,2,3$ are themselves Fibonacci numbers.  Suppose $k\geq 4$ and that every integer less than $k$ is a sum of distinct Fibonacci numbers.  Let $n$ be the largest positive integer such that $F_n<k$. We first note that if $k-F_n > F_{n-1}$ then
$$F_{n+1} \geq k > F_n + F_{n-1} = F_{n+1}, $$
giving us a contradiction.  Hence $k-F_n \leq F_{n-1}$ and consequently the positive integer $(k-F_n)$ can be expressed as a sum of distinct Fibonacci numbers.  Moreover, this sum does not contain the term $F_n$ as $k-F_n \leq F_{n-1} < F_n$.
Hence,$k = (k-F_n) + F_n$ is a sum of distinct Fibonacci numbers and Hogatt's theorem is proved by induction.
%%%%%
%%%%%
\end{document}
