\documentclass[12pt]{article}
\usepackage{pmmeta}
\pmcanonicalname{ProuhetThueMorseConstant}
\pmcreated{2013-03-22 14:28:23}
\pmmodified{2013-03-22 14:28:23}
\pmowner{mathcam}{2727}
\pmmodifier{mathcam}{2727}
\pmtitle{Prouhet-Thue-Morse constant}
\pmrecord{7}{35994}
\pmprivacy{1}
\pmauthor{mathcam}{2727}
\pmtype{Definition}
\pmcomment{trigger rebuild}
\pmclassification{msc}{11J81}
\pmclassification{msc}{11B85}
\pmrelated{ProuhetThueMorseSequence}
\pmdefines{Thue-Morse constant}

% this is the default PlanetMath preamble.  as your knowledge
% of TeX increases, you will probably want to edit this, but
% it should be fine as is for beginners.

% almost certainly you want these
\usepackage{amssymb}
\usepackage{amsmath}
\usepackage{amsfonts}

% used for TeXing text within eps files
%\usepackage{psfrag}
% need this for including graphics (\includegraphics)
%\usepackage{graphicx}
% for neatly defining theorems and propositions
%\usepackage{amsthm}
% making logically defined graphics
%%%\usepackage{xypic}

% there are many more packages, add them here as you need them

% define commands here
\begin{document}
The {\em Prouhet-Thue-Morse constant} is the number $\tau$ whose binary expansion is the Prouhet-Thue-Morse sequence.  That is,
\[ \tau = \sum_{i=0}^{\infty} \frac{t_i}{2^{i+1}} = 0.412454033640 \ldots \]

where $t_i$ is the Prouhet-Thue-Morse sequence

Another expression, not in \PMlinkescapetext{terms} of $t_i$, is

\[ \tau = \frac{1}{2} - \frac{1}{4} \prod_{n=0}^{\infty} ( 1 - 2^{-2^i} ) \]

The number $\tau$ has been shown to be transcendental.
%%%%%
%%%%%
\end{document}
