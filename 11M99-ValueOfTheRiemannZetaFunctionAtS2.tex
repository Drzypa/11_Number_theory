\documentclass[12pt]{article}
\usepackage{pmmeta}
\pmcanonicalname{ValueOfTheRiemannZetaFunctionAtS2}
\pmcreated{2013-03-22 13:57:16}
\pmmodified{2013-03-22 13:57:16}
\pmowner{alozano}{2414}
\pmmodifier{alozano}{2414}
\pmtitle{value of the Riemann zeta function at $s=2$}
\pmrecord{15}{34719}
\pmprivacy{1}
\pmauthor{alozano}{2414}
\pmtype{Theorem}
\pmcomment{trigger rebuild}
\pmclassification{msc}{11M99}
\pmclassification{msc}{42A16}
\pmrelated{ExampleOfFourierSeries}
\pmrelated{PersevalEquality}
\pmrelated{ValuesOfTheRiemannZetaFunctionInTermsOfBernoulliNumbers}
\pmrelated{ValueOfRiemannZetaFunctionAtS4}
\pmrelated{ValueOfDirichletEtaFunctionAtS2}
\pmrelated{APathologicalFunctionOfRiemann}
\pmrelated{KummersAccelerationMethod}

% this is the default PlanetMath preamble.  as your knowledge
% of TeX increases, you will probably want to edit this, but
% it should be fine as is for beginners.

% almost certainly you want these
\usepackage{amssymb}
\usepackage{amsmath}
\usepackage{amsthm}
\usepackage{amsfonts}

% used for TeXing text within eps files
%\usepackage{psfrag}
% need this for including graphics (\includegraphics)
%\usepackage{graphicx}
% for neatly defining theorems and propositions
%\usepackage{amsthm}
% making logically defined graphics
%%%\usepackage{xypic}

% there are many more packages, add them here as you need them

% define commands here

\newtheorem{thm}{Theorem}
\newtheorem{defn}{Definition}
\newtheorem{prop}{Proposition}
\newtheorem{lemma}{Lemma}
\newtheorem{cor}{Corollary}

% Some sets
\newcommand{\Nats}{\mathbb{N}}
\newcommand{\Ints}{\mathbb{Z}}
\newcommand{\Reals}{\mathbb{R}}
\newcommand{\Complex}{\mathbb{C}}
\newcommand{\Rats}{\mathbb{Q}}
\begin{document}
Here we present an application of Parseval's equality to number
theory. Let $\zeta(s)$ denote the Riemann zeta function. We will
compute the value
$$\zeta(2)$$
with the help of Fourier analysis.

{\bf Example:}

Let $f\colon \Reals \to \Reals$ be the ``identity'' function,
defined by
$$f(x)=x, \text{ for all }x\in\Reals.$$

The Fourier series of this function has been computed in the entry
example of Fourier series.

Thus
\begin{eqnarray*}
 f(x)=\ x&=& a_0^f +
\sum_{n=1}^{\infty}(a_n^f\cos(nx)+b_n^f\sin(nx)) \\
&=& \sum_{n=1}^{\infty}(-1)^{n+1}\frac{2}{n} \sin(nx), \quad \forall x\in
(-\pi,\pi).
\end{eqnarray*}

Parseval's theorem asserts that:

$$\frac{1}{\pi}\int_{-\pi}^{\pi}f^2(x)dx = 2(a_0^f)^2 + \sum_{k=1}^{\infty}[(a_k^f)^2+(b_k^f)^2].$$

So we apply this to the function $f(x)= x $:
$$2(a_0^f)^2 +
\sum_{k=1}^{\infty}[(a_k^f)^2+(b_k^f)^2]= \sum_{n=1}^{\infty}
\frac{4}{n^2}= 4\sum_{n=1}^{\infty}\frac{1}{n^2}$$ and
$$\frac{1}{\pi}\int_{-\pi}^{\pi}f^2(x)dx = \frac{1}{\pi}\int_{-\pi}^{\pi}x^2dx= \frac{2\pi^2}{3}.$$

Hence by Parseval's equality
$$4\sum_{n=1}^{\infty}\frac{1}{n^2}=\frac{2\pi^2}{3}$$
and hence
$$\zeta(2)=\sum_{n=1}^{\infty}\frac{1}{n^2}=\frac{\pi^2}{6}.$$
%%%%%
%%%%%
\end{document}
