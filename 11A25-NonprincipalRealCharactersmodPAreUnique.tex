\documentclass[12pt]{article}
\usepackage{pmmeta}
\pmcanonicalname{NonprincipalRealCharactersmodPAreUnique}
\pmcreated{2013-03-22 16:34:51}
\pmmodified{2013-03-22 16:34:51}
\pmowner{rm50}{10146}
\pmmodifier{rm50}{10146}
\pmtitle{nonprincipal real characters $\mod p$ are unique}
\pmrecord{6}{38772}
\pmprivacy{1}
\pmauthor{rm50}{10146}
\pmtype{Theorem}
\pmcomment{trigger rebuild}
\pmclassification{msc}{11A25}

\endmetadata

% this is the default PlanetMath preamble.  as your knowledge
% of TeX increases, you will probably want to edit this, but
% it should be fine as is for beginners.

% almost certainly you want these
\usepackage{amssymb}
\usepackage{amsmath}
\usepackage{amsfonts}

% used for TeXing text within eps files
%\usepackage{psfrag}
% need this for including graphics (\includegraphics)
%\usepackage{graphicx}
% for neatly defining theorems and propositions
%\usepackage{amsthm}
% making logically defined graphics
%%%\usepackage{xypic}

% there are many more packages, add them here as you need them

% define commands here
\newcommand{\Ints}{\mathbb{Z}}
\newcommand{\Leg}[2]{\left(\frac{#1}{#2}\right)}
\newtheorem{thm}{Theorem}
\begin{document}
\begin{thm} Let $p$ be a prime. Then there is a unique nonprincipal real Dirichlet character $\chi \mod p$, given by
\[\chi(n)=\Leg{n}{p}\]
\end{thm}
\textbf{Proof. }
Note first that $\chi(n)=\Leg{n}{p}$ is obviously a nonprincipal real character $\mod p$. Now, suppose $\chi$ is any nonprincipal real character $\mod p$. Choose some generator, $a$, of $(\Ints/p\Ints)^*$. Clearly $\chi(a)=-1$ (since otherwise $\chi$ is principal), and thus $\chi(a^k)=(-1)^k$. Since $a^{p-1}=1$, and no lower power of $a$ is $1$, it follows that $\chi$ is $-1$ on exactly $(p-1)/2$ elements of $(\Ints/p\Ints)^*$ and is $1$ on exactly $(p-1)/2$ elements. However, since $\chi(x^2)=\chi(x)^2=1$, $\chi$ is $1$ on each of the $(p-1)/2$ squares in $(\Ints/p\Ints)^*$. Thus $\chi$ is $1$ on squares and $-1$ on nonsquares, so $\chi(n)=\Leg{n}{p}$.

Note that this result is not true if $p$ is not prime. For example, the following is a table of Dirichlet characters modulo $8$, all of which are real:
\begin{center}
\begin{tabular}{c|rrrr}
&$\chi_1$&$\chi_2$&$\chi_3$&$\chi_4$\\
\hline\\
$1$&$1$&$1$&$1$&$1$\\
$3$&$1$&$1$&$-1$&$-1$\\
$5$&$1$&$-1$&$1$&$-1$\\
$7$&$1$&$-1$&$-1$&$1$
\end{tabular}
\end{center}

%%%%%
%%%%%
\end{document}
