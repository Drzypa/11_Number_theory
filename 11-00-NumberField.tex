\documentclass[12pt]{article}
\usepackage{pmmeta}
\pmcanonicalname{NumberField}
\pmcreated{2013-03-22 12:04:09}
\pmmodified{2013-03-22 12:04:09}
\pmowner{alozano}{2414}
\pmmodifier{alozano}{2414}
\pmtitle{number field}
\pmrecord{17}{31128}
\pmprivacy{1}
\pmauthor{alozano}{2414}
\pmtype{Definition}
\pmcomment{trigger rebuild}
\pmclassification{msc}{11-00}
\pmsynonym{algebraic number field}{NumberField}
\pmrelated{AlgebraicNumberTheory}
\pmrelated{ExamplesOfPrimeIdealDecompositionInNumberFields}
\pmrelated{ExamplesOfFields}
\pmrelated{AbelianExtensionsOfQuadraticImaginaryNumberFields}
\pmrelated{NumberTheory}
\pmrelated{ResidueDegree}
\pmrelated{Regulator}
\pmrelated{DiscriminantIdeal}
\pmrelated{ClassNumber2}
\pmrelated{ExistenceOfHilbertClassField}
\pmrelated{Multiplicat}
\pmdefines{quadratic number field}
\pmdefines{quadratic field}

\usepackage{amssymb}
\usepackage{amsmath}
\usepackage{amsthm}
\usepackage{amsfonts}
\usepackage{graphicx}
%%%\usepackage{xypic}

\newtheorem{defn}{Definition}

\theoremstyle{definition}
\newtheorem{exa}{Example}

\newcommand{\Rats}{\mathbb{Q}}
\begin{document}
\begin{defn}
A field which is a finite extension of $\mathbb{Q}$, the rational numbers, is called a {\bf number field} (sometimes called {\em algebraic number field}). If the degree of the extension $K/\Rats$ is $n$ then we say that $K$ is a number field of degree $n$ (over $\Rats$).\\
\end{defn}

\begin{exa}
The field of rational numbers $\Rats$ is a number field.\\
\end{exa}

\begin{exa}
Let $K=\Rats(\sqrt{d})$, where $d\neq 1$ is a square-free non-zero integer and $\sqrt{d}$ stands for any of the roots of $x^2-d=0$ (note that if $\sqrt{d}\in K$ then $-\sqrt{d}\in K$ as well). Then $K$ is a number field and $[K:\Rats]=2$. We can explictly describe all elements of $K$ as follows:
$$K=\{ t+s\sqrt{d} : t,s \in \Rats \}.$$ 
\end{exa}

\begin{defn}
A number field $K$ such that the degree of the extension $K/\Rats$ is $2$ is called a {\bf quadratic number field}.
\end{defn}

In fact, if $K$ is a quadratic number field, then it is easy to show that $K$ is one of the fields described in Example $2$.

\begin{exa}
Let $K_n=\Rats(\zeta_n)$ be a cyclotomic extension of $\Rats$, where $\zeta_n$ is a primitive $n$th root of unity. Then $K$ is a number field and 
$$[K:\Rats]=\varphi(n)$$
where $\varphi(n)$ is the Euler phi function. In particular, $\varphi(3)=2$, therefore $K_3$ is a quadratic number field (in fact $K_3=\Rats(\sqrt{-3})$). We can explicitly describe all elements of $K$ as follows:
$$K_n=\{ q_0+q_1\zeta_n+q_2\zeta_n^2+\ldots+q_{n-1}\zeta_n^{n-1} : q_i\in \Rats \}.$$
In fact, one can do better. Every element of $K_n$ can be uniquely expressed as a rational combination of the $\varphi(n)$ elements $\{\zeta_n^a : \gcd(a,n)=1,\ 1\leq a < n\}$.
\end{exa}

\begin{exa}
Let $K$ be a number field. Then any subfield $L$ with $\Rats \subseteq L \subseteq K$ is also a number field. For example, let $p$ be a prime number and let $F=\Rats(\zeta_p)$ be a cyclotomic extension of $\Rats$, where $\zeta_p$ is a primitive $p$th root of unity. Let $F^+$ be the maximal real subfield of $F$. $F^{+}$ is a number field and it can be shown that:
$$F^+=\Rats(\zeta_p+\zeta_p^{-1}).$$
\end{exa}
%%%%%
%%%%%
%%%%%
\end{document}
