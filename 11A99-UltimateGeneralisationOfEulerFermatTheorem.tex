\documentclass[12pt]{article}
\usepackage{pmmeta}
\pmcanonicalname{UltimateGeneralisationOfEulerFermatTheorem}
\pmcreated{2013-03-22 19:35:04}
\pmmodified{2013-03-22 19:35:04}
\pmowner{akdevaraj}{13230}
\pmmodifier{akdevaraj}{13230}
\pmtitle{ultimate generalisation of Euler-Fermat theorem}
\pmrecord{5}{42572}
\pmprivacy{1}
\pmauthor{akdevaraj}{13230}
\pmcomment{trigger rebuild}
\pmclassification{msc}{11A99}

\endmetadata

% this is the default PlanetMath preamble.  as your knowledge
% of TeX increases, you will probably want to edit this, but
% it should be fine as is for beginners.

% almost certainly you want these
\usepackage{amssymb}
\usepackage{amsmath}
\usepackage{amsfonts}

% used for TeXing text within eps files
%\usepackage{psfrag}
% need this for including graphics (\includegraphics)
%\usepackage{graphicx}
% for neatly defining theorems and propositions
%\usepackage{amsthm}
% making logically defined graphics
%%%\usepackage{xypic}

% there are many more packages, add them here as you need them

% define commands here

\begin{document}
        
Let $a^b+u = m$ where $a,\,b,\,u,\,m$ are positive integers.  Then\,  
$$ a^{b+k\varphi(m)}+u  \;\equiv\; 0 \pmod m, $$
by the result in ``Euler's  generalisation of Fermat's theorem -- a further generalisation''. {\it Proceedings of Hawaii  Intl. conference on maths \& statistics} \textbf{2004} (ISSN 1550--3747).  Here, $k$ is a positive integer.  Next,  
$$ a^{b^{1+k\varphi(\varphi(m))}}+u \;\equiv\; 0 \pmod m. $$
(This is a corollary of ``Euler's generalisation of Fermat's theorem -- a further generalisation''.)
We can proceed in a like manner till we reach
$$ a^{b^{c^{\vdots^{t^{1+k\varphi(\varphi(\varphi(\ldots \varphi(2)\ldots)))}}}}}. $$ 
At this stage onwards the function generates only multiples of $m$ and no prime number is generated.  This is the ultimate generalisation of Fermat's theorem.  Please note that each step of multiple exponentiation in the above is a corollary  of the theorem referred to.


%%%%%
%%%%%
\end{document}
