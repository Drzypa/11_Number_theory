\documentclass[12pt]{article}
\usepackage{pmmeta}
\pmcanonicalname{HerbrandsTheorem}
\pmcreated{2013-03-22 14:12:45}
\pmmodified{2013-03-22 14:12:45}
\pmowner{mathcam}{2727}
\pmmodifier{mathcam}{2727}
\pmtitle{Herbrand's theorem}
\pmrecord{5}{35647}
\pmprivacy{1}
\pmauthor{mathcam}{2727}
\pmtype{Theorem}
\pmcomment{trigger rebuild}
\pmclassification{msc}{11R29}

% this is the default PlanetMath preamble.  as your knowledge
% of TeX increases, you will probably want to edit this, but
% it should be fine as is for beginners.

% almost certainly you want these

\usepackage{amssymb}
\usepackage{amsmath}
\usepackage{amsfonts}
\usepackage{amsthm}

% used for TeXing text within eps files
%\usepackage{psfrag}
% need this for including graphics (\includegraphics)
%\usepackage{graphicx}
% for neatly defining theorems and propositions
%\usepackage{amsthm}
% making logically defined graphics
%%%\usepackage{xypic}

% there are many more packages, add them here as you need them

% define commands here

\newtheorem{Theo}{Theorem}
\newcommand{\mc}{\mathcal}
\newcommand{\mb}{\mathbb}
\newcommand{\mf}{\mathfrak}
\newcommand{\ol}{\overline}
\newcommand{\ra}{\rightarrow}
\newcommand{\la}{\leftarrow}
\newcommand{\La}{\Leftarrow}
\newcommand{\Ra}{\Rightarrow}
\newcommand{\nor}{\vartriangleleft}
\newcommand{\Gal}{\text{Gal}}
\newcommand{\GL}{\text{GL}}
\newcommand{\Z}{\mb{Z}}
\newcommand{\R}{\mb{R}}
\newcommand{\Q}{\mb{Q}}
\newcommand{\C}{\mb{C}}
\newcommand{\<}{\langle}
\renewcommand{\>}{\rangle}
\begin{document}
Let $\mb{Q}(\zeta_p)$ be a cyclotomic extension of $\Q$, with $p$ an odd prime, let $A$ be the Sylow $p$-subgroup of the ideal class group of $\mb{Q}(\zeta_p)$, and let $G$ be the Galois group of this extension.  Note that the character  group of $G$, denoted $\hat{G}$, is given by
\begin{align*}
\hat{G}=\{\chi^i\mid0\leq i\leq p-2\}
\end{align*}

For each $\chi\in\hat{G}$, let $\varepsilon_\chi$ denote the corresponding orthogonal idempotent of the group ring, and note that the $p$-Sylow subgroup of the ideal class group is a $\mathbb{Z}[G]$-module under the typical multiplication.  Thus, using the orthogonal idempotents, we can decompose the module $A$ via $A=\sum_{i=0}^{p-2}A_{\omega^i}\equiv\sum_{i=0}^{p-2}A_i$.

Last, let $B_k$ denote the $k$th Bernoulli number.

\begin{Theo}[Herbrand]
Let $i$ be odd with $3\leq i\leq p-2$.  Then $A_i\neq 0 \iff p\mid B_{p-i}$.
\end{Theo}

Only the first direction of this theorem ($\implies$) was proved by Herbrand himself.  The converse is much more intricate, and was proved by Ken Ribet.
%%%%%
%%%%%
\end{document}
