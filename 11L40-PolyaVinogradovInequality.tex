\documentclass[12pt]{article}
\usepackage{pmmeta}
\pmcanonicalname{PolyaVinogradovInequality}
\pmcreated{2013-03-22 12:46:23}
\pmmodified{2013-03-22 12:46:23}
\pmowner{djao}{24}
\pmmodifier{djao}{24}
\pmtitle{P\'olya-Vinogradov inequality}
\pmrecord{7}{33084}
\pmprivacy{1}
\pmauthor{djao}{24}
\pmtype{Theorem}
\pmcomment{trigger rebuild}
\pmclassification{msc}{11L40}

\endmetadata

% this is the default PlanetMath preamble.  as your knowledge
% of TeX increases, you will probably want to edit this, but
% it should be fine as is for beginners.

% almost certainly you want these
\usepackage{amssymb}
\usepackage{amsmath}
\usepackage{amsfonts}

% used for TeXing text within eps files
%\usepackage{psfrag}
% need this for including graphics (\includegraphics)
%\usepackage{graphicx}
% for neatly defining theorems and propositions
\usepackage{amsthm}
% making logically defined graphics
%%%\usepackage{xypic} 

% there are many more packages, add them here as you need them

% define commands here

\newtheorem{theorem}{Theorem}
\newtheorem{proposition}[theorem]{Proposition}
\newtheorem{lemma}[theorem]{Lemma}
\newtheorem{corollary}[theorem]{Corollary}

\theoremstyle{definition}
\newtheorem{definition}[theorem]{Definition}
\newtheorem{example}[theorem]{Example}
\begin{document}
\begin{theorem}
For $m,n \in {\bf N}$ and $p$ a positive odd rational
prime,
$$\left|\sum_{t=m}^{m+n} \left(\frac{t}{p}\right)\right| < \sqrt{p} \,
\ln p.$$
\end{theorem}

\begin{proof}
Start with the following manipulations:
$$
\sum_{t=m}^{m+n} \left(\frac{t}{p}\right) = \frac{1}{p}
\sum_{t=0}^{p-1} \sum_{x=m}^{m+n} \sum_{a=0}^{p-1}
\left(\frac{t}{p} \right) e^{2 \pi i a (x-t)/p} = \frac{1}{p}
\sum_{a=1}^{p-1} \sum_{x=m}^{m+n} e^{2 \pi i a x/p}
\sum_{t=0}^{p-1} \left(\frac{t}{p} \right) e^{-2 \pi i a t/p}
$$

The expression \begin{math}\sum_{t=0}^{p-1} (\frac{t}{p}) e^{-2 \pi i
a t/p}\end{math} is just a Gauss sum, and has magnitude 
\begin{math}\sqrt{p} \,\end{math}. Hence

\begin{eqnarray*}
\left|\sum_{t=m}^{m+n} \left(\frac{t}{p}\right)\right| & \leq &
\left|\frac{\sqrt{p}}{p} \sum_{a=1}^{p-1}
\sum_{x=m}^{m+n} e^{2 \pi a i x/p}\right|
= \left|\frac{\sqrt{p}}{p} \sum_{a=1}^{p-1} e^{2 \pi i a m/p}
\sum_{x=0}^{n} e^{2 \pi i a x/p}\right| \leq
\left|\frac{\sqrt{p}}{p} \sum_{a=1}^{p-1} 
\frac{e^{2 \pi i a n/p}-1}{e^{2 \pi ia/p}-1}\right|
\\
& = & \left|\frac{\sqrt{p}}{p} \sum_{a=1}^{p-1}
\frac{e^{\pi ian/p} \sin(\pi a n/p)}{e^{\pi ia/p} \sin(\pi a/p)}\right|
\leq \frac{\sqrt{p}}{p} \sum_{a=1}^{p-1} \left|\frac{1}{\sin(\pi
\langle a/p \rangle)}\right|
\leq \frac{\sqrt{p}}{p} \sum_{a=1}^{p-1} \frac{1}{2 \langle a/p \rangle}
\end{eqnarray*}

Here $\langle x \rangle$ denotes the absolute value of the difference
between $x$ and the closest integer to $x$, i.e. $\langle x \rangle = 
\inf_{z \in {\bf Z}} \{|x-z|\}$.

Since $p$ is odd, we have

$$
\frac{1}{2} \sum_{a=1}^{p-1} \frac{1}{\langle a/p \rangle} = 
\sum_{0<a<\frac{p}{2}} \frac{p}{a} = 
p \sum_{a=1}^{\frac{p-1}{2}} \frac{1}{a}
$$

Now $\ln \frac{2x+1}{2x-1} > \frac{1}{x}$ for $x>1$; to prove this, it
suffices to show that the function $f:[1,\infty)\rightarrow{\bf R}$
given by $f(x)=x \ln \frac{2x+1}{2x-1}$
is decreasing and approaches 1 as $x\rightarrow \infty$. To prove the
latter statement, substitute $v=1/x$ and take the limit as $v
\rightarrow 0$ using L'H\^opital's rule. To prove the former statement,
it will suffice to show that $f'$ is less than zero on the interval
$[1,\infty)$. But $f'(x)\rightarrow 0$ as $x \rightarrow \infty$ and
$f'$ is increasing on $[1,\infty)$, since $f''(x) = 
\frac{-4}{4 x^2-1} (1-\frac{4x^2+1}{4x^2-1}) > 0$ for $x>1$, so $f'$
is less than zero for $x>1$.

With this in hand, we have

$$
\left|\sum_{t=m}^{m+n} \left(\frac{t}{p}\right)\right| \leq
\frac{\sqrt{p}}{p} \cdot p \sum_{a=1}^{\frac{p-1}{2}} \frac{1}{a} < 
\sqrt{p} \sum_{a=1}^{\frac{p-1}{2}} \ln \frac{2a+1}{2a-1} = 
\sqrt{p} \, \ln p.
$$
\end{proof}
\begin{thebibliography}{9}
\bibitem{v} Vinogradov, I. M., {\em Elements of Number Theory}, 5th rev. ed., Dover, 1954.
\end{thebibliography}
%%%%%
%%%%%
\end{document}
