\documentclass[12pt]{article}
\usepackage{pmmeta}
\pmcanonicalname{SternBrocotTree}
\pmcreated{2013-03-22 13:41:19}
\pmmodified{2013-03-22 13:41:19}
\pmowner{bbukh}{348}
\pmmodifier{bbukh}{348}
\pmtitle{Stern-Brocot tree}
\pmrecord{9}{34361}
\pmprivacy{1}
\pmauthor{bbukh}{348}
\pmtype{Definition}
\pmcomment{trigger rebuild}
\pmclassification{msc}{11A05}
\pmclassification{msc}{11A55}
%\pmkeywords{mediant}
\pmrelated{FareySequence}
\pmrelated{ContinuedFraction}

\endmetadata

\usepackage{amssymb}
\usepackage{amsmath}
\usepackage{amsfonts}

%%\usepackage{xypic}

\makeatletter
\@ifundefined{bibname}{}{\renewcommand{\bibname}{References}}
\makeatother
\begin{document}
If we start with irreducible fractions representing zero and
infinity,
\begin{equation*}
\frac{0}{1}, \frac{1}{0},
\end{equation*}
and then between adjacent fractions $\frac{m}{n}$ and
$\frac{m'}{n'}$ we insert fraction $\frac{m+m'}{n+n'}$, then we
obtain
\begin{equation*}
\frac{0}{1}, \frac{1}{1}, \frac{1}{0}.
\end{equation*}
Repeating the process, we get
\begin{equation*}
\frac{0}{1}, \frac{1}{2}, \frac{1}{1}, \frac{2}{1}, \frac{1}{0},
\end{equation*}
and then
\begin{equation*}
\frac{0}{1}, \frac{1}{3}, \frac{1}{2}, \frac{2}{3}, \frac{1}{1},
\frac{3}{2}, \frac{2}{1}, \frac{3}{1}, \frac{1}{0},
\end{equation*}
and so forth. It can be proven that every irreducible fraction
appears at some iteration and no fraction ever appears twice \cite{cite:knuth_concmath}. The process
can be represented graphically by means of so-called
\emph{Stern-Brocot tree}, named after its discoverers, Moritz Stern
and Achille Brocot.

% Remove *!C\xybox when Bug 122 is fixed
\begin{equation*}
\begin{xy}
*!C\xybox{\xymatrix@C=0.4em{
\frac{0}{1}\ar@{.}[d] & & & & & &
&&\frac{1}{1}\ar@{-}[lllld]\ar@{.}[d]\ar@{-}[rrrrd]& & & & & & & &
\frac{1}{0}\ar@{.}[d]\\
\frac{0}{1}\ar@{.}[d] &
&&&\frac{1}{2}\ar@{-}[lld]\ar@{.}[d]\ar@{-}[rrd]&&&&
\frac{1}{1}\ar@{.}[d]&&&&\frac{2}{1}\ar@{-}[lld]\ar@{.}[d]\ar@{-}[rrd]&&
&&\frac{1}{0}\ar@{.}[d] \\
\frac{0}{1}\ar@{.}[d] &
&\frac{1}{3}\ar@{-}[ld]\ar@{.}[d]\ar@{-}[rd]&&\frac{1}{2}\ar@{.}[d]&&
\frac{2}{3}\ar@{-}[ld]\ar@{.}[d]\ar@{-}[rd]&&\frac{1}{1}\ar@{.}[d]
& & \frac{3}{2}\ar@{-}[ld]\ar@{.}[d]\ar@{-}[rd] & &
\frac{2}{1}\ar@{.}[d] &
&\frac{3}{1}\ar@{-}[ld]\ar@{.}[d]\ar@{-}[rd]&&\frac{1}{0}\ar@{.}[d] \\
\frac{0}{1}\ar@{.}+<0em,-5ex> &
\frac{1}{4}\ar@{-}+<-0.5em,-5ex>\ar@{.}+<0em,-5ex>\ar@{-}+<0.5em,-5ex>&\frac{1}{3}\ar@{.}+<0em,-5ex>&
\frac{2}{5}\ar@{-}+<-0.5em,-5ex>\ar@{.}+<0em,-5ex>\ar@{-}+<0.5em,-5ex>&\frac{1}{2}\ar@{.}+<0em,-5ex>&
\frac{3}{5}\ar@{-}+<-0.5em,-5ex>\ar@{.}+<0em,-5ex>\ar@{-}+<0.5em,-5ex>&\frac{2}{3}\ar@{.}+<0em,-5ex>&
\frac{3}{4}\ar@{-}+<-0.5em,-5ex>\ar@{.}+<0em,-5ex>\ar@{-}+<0.5em,-5ex>&\frac{1}{1}\ar@{.}+<0em,-5ex>&
\frac{4}{3}\ar@{-}+<-0.5em,-5ex>\ar@{.}+<0em,-5ex>\ar@{-}+<0.5em,-5ex>&\frac{3}{2}\ar@{.}+<0em,-5ex>&
\frac{5}{3}\ar@{-}+<-0.5em,-5ex>\ar@{.}+<0em,-5ex>\ar@{-}+<0.5em,-5ex>&\frac{2}{1}\ar@{.}+<0em,-5ex>&
\frac{5}{2}\ar@{-}+<-0.5em,-5ex>\ar@{.}+<0em,-5ex>\ar@{-}+<0.5em,-5ex>&\frac{3}{1}\ar@{.}+<0em,-5ex>&
\frac{4}{1}\ar@{-}+<-0.5em,-5ex>\ar@{.}+<0em,-5ex>\ar@{-}+<0.5em,-5ex>&\frac{1}{0}\ar@{.}+<0em,-5ex>
}}
\end{xy}
\end{equation*}

If we specify position of a fraction in the tree as a path consisting of L(eft) an R(ight) moves along the tree starting from the top (fraction $\frac{1}{1}$), and also define matrices 
\begin{equation*}
L=\begin{bmatrix} 1&1\\0&1 \end{bmatrix},\qquad
R=\begin{bmatrix} 1&0\\1&1 \end{bmatrix},
\end{equation*}
then product of the matrices corresponding to the path is matrix $\left[\begin{smallmatrix}n&n'\\m&m'\end{smallmatrix}\right]$ whose entries are numerators and denominators of parent fractions. For example, the path leading to fraction $\frac{3}{5}$ is LRL. The corresponding matrix product is
\begin{equation*}
LRL=\begin{bmatrix} 1&1\\0&1 \end{bmatrix}\begin{bmatrix} 1&0\\1&1 \end{bmatrix}\begin{bmatrix} 1&1\\0&1 \end{bmatrix}=\begin{bmatrix} 2&3\\1&2 \end{bmatrix},
\end{equation*}
and the parents of $\frac{3}{5}$ are $\frac{1}{2}$ and $\frac{2}{3}$.

Continued fractions and Stern-Brocot tree are closely related. A continued fraction $\langle a_0;a_1,a_2,\dotsc\rangle$ corresponds to fraction whose path from the top is $R^{a_0}L^{a_1}R^{a_2}\dotsb$ with the last element removed. For example,
\begin{align*}
\frac{5}{3}&=1+\frac{1}{1+\frac{1}{2}}=\langle 1;1,2\rangle=R^1L^1R^{2-1}=RLR,\\
\frac{4}{11}&=0+\frac{1}{2+\frac{1}{1+\frac{1}{3}}}=\langle 0;2,1,3\rangle=R^0L^2R^1L^{3-1}=LLRLL.
\end{align*}
The irrational numbers correspond to the infinite paths in the Stern-Brocot tree. For example, the golden mean $\phi=\langle 1;1,1,1,\dotsc\rangle$ corresponds to the infinite path $RLRLR\dotsb$. In particular, the denominator of the fraction corresponding to the string $RLR\dotsb$ of length $n$ is the $n$'th Fibonacci number.

\begin{thebibliography}{1}

\bibitem{cite:knuth_concmath}
Ronald~L. Graham, Donald~E. Knuth, and Oren Patashnik.
\newblock {\em Concrete Mathematics}.
\newblock Addison-Wesley, 1998.
\newblock \PMlinkexternal{Zbl 0836.00001}{http://www.emis.de/cgi-bin/zmen/ZMATH/en/quick.html?type=html&an=0836.00001}.

\end{thebibliography}
%%%%%
%%%%%
\end{document}
