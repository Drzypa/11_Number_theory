\documentclass[12pt]{article}
\usepackage{pmmeta}
\pmcanonicalname{sqrtn2IsIrrationalForNge3proofUsingFermatsLastTheorem}
\pmcreated{2013-03-22 13:38:32}
\pmmodified{2013-03-22 13:38:32}
\pmowner{matte}{1858}
\pmmodifier{matte}{1858}
\pmtitle{$\sqrt[n]{2}$ is irrational for $n\ge 3$ (proof using Fermat's last theorem)}
\pmrecord{13}{34292}
\pmprivacy{1}
\pmauthor{matte}{1858}
\pmtype{Proof}
\pmcomment{trigger rebuild}
\pmclassification{msc}{11J72}

% this is the default PlanetMath preamble.  as your knowledge
% of TeX increases, you will probably want to edit this, but
% it should be fine as is for beginners.

% almost certainly you want these
\usepackage{amssymb}
\usepackage{amsmath}
\usepackage{amsfonts}
\usepackage{amsthm}

\usepackage{mathrsfs}

% used for TeXing text within eps files
%\usepackage{psfrag}
% need this for including graphics (\includegraphics)
%\usepackage{graphicx}
% for neatly defining theorems and propositions
%
% making logically defined graphics
%%%\usepackage{xypic}

% there are many more packages, add them here as you need them

% define commands here

\newcommand{\sR}[0]{\mathbb{R}}
\newcommand{\sC}[0]{\mathbb{C}}
\newcommand{\sN}[0]{\mathbb{N}}
\newcommand{\sZ}[0]{\mathbb{Z}}

 \usepackage{bbm}
 \newcommand{\Z}{\mathbbmss{Z}}
 \newcommand{\C}{\mathbbmss{C}}
 \newcommand{\R}{\mathbbmss{R}}
 \newcommand{\Q}{\mathbbmss{Q}}



\newcommand*{\norm}[1]{\lVert #1 \rVert}
\newcommand*{\abs}[1]{| #1 |}



\newtheorem{thm}{Theorem}
\newtheorem{defn}{Definition}
\newtheorem{prop}{Proposition}
\newtheorem{lemma}{Lemma}
\newtheorem{cor}{Corollary}
\begin{document}
\PMlinkescapeword{states}

\begin{thm} If $n\ge 3$, then $\sqrt[n]{2}$ is irrational.
\end{thm}

The below proof can be seen as an example of a pathological proof. 
It gives no information to ``why" the result holds, or 
how non-trivial the result is. 
Yet, assuming Wiles' proof does not use the above theorem anywhere, 
it proves the statement. Otherwise, the below proof would be an
example of a circular argument. 

\begin{proof}
Suppose $\sqrt[n]{2}=a/b$ for some positive integers $a,b$. It follows that
$2=a^n/b^n$, or
\begin{eqnarray}
\label{fermat}
b^n + b^n &=& a^n.
\end{eqnarray}
We can now apply a recent result of Andrew Wiles \cite{wiles},
which states that there are no non-zero integers $a$, $b$ satisfying equation \eqref{fermat}.
Thus $\sqrt[n]{2}$ is irrational. 
\end{proof}
 
The above proof is given in \cite{abobs}, where it is attributed to W.H. Schultz.
 
\begin{thebibliography}{99}
\bibitem{wiles} A. Wiles, \emph{Modular elliptic curves and Fermat's last theorem},
Annals of Mathematics, Volume 141, No. 3 May, 1995, 443-551. 
\bibitem{abobs} W.H. Schultz, \emph{An observation},
American Mathematical Monthly, Vol. 110, Nr. 5, May 2003.
(submitted by R. Ehrenborg).
\end{thebibliography}
%%%%%
%%%%%
\end{document}
