\documentclass[12pt]{article}
\usepackage{pmmeta}
\pmcanonicalname{BombieriVinogradovTheorem}
\pmcreated{2013-03-22 16:25:36}
\pmmodified{2013-03-22 16:25:36}
\pmowner{Mravinci}{12996}
\pmmodifier{Mravinci}{12996}
\pmtitle{Bombieri-Vinogradov theorem}
\pmrecord{4}{38578}
\pmprivacy{1}
\pmauthor{Mravinci}{12996}
\pmtype{Theorem}
\pmcomment{trigger rebuild}
\pmclassification{msc}{11A25}
\pmsynonym{Bombieri's theorem}{BombieriVinogradovTheorem}

% this is the default PlanetMath preamble.  as your knowledge
% of TeX increases, you will probably want to edit this, but
% it should be fine as is for beginners.

% almost certainly you want these
\usepackage{amssymb}
\usepackage{amsmath}
\usepackage{amsfonts}

% used for TeXing text within eps files
%\usepackage{psfrag}
% need this for including graphics (\includegraphics)
%\usepackage{graphicx}
% for neatly defining theorems and propositions
%\usepackage{amsthm}
% making logically defined graphics
%%%\usepackage{xypic}

% there are many more packages, add them here as you need them

% define commands here

\begin{document}
The \emph{Bombieri-Vinogradov theorem}, sometimes called \emph{Bombieri's theorem}, states that for a positive real number $A$, if $x^{\frac{1}{2}}\log^{-A}x\leq Q\leq x^{\frac{1}{2}}$ then $$\sum_{q\leq Q}\max_{y\leq x}\max_{1\le a\le q\atop (a,q) = 1}\left|\psi(x;q,a) - {x\over\phi(q)}\right| = O\left(x^{\frac{1}{2}}Q(\log x)^5\right),$$ where $\phi(q)$ is Euler's totient function and $$\psi(x;q,a)=\sum_{n\le x\atop n\equiv a\mod q}\Lambda(n),$$ where $\Lambda(n)$ is the Mangoldt function.
%%%%%
%%%%%
\end{document}
