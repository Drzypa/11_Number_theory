\documentclass[12pt]{article}
\usepackage{pmmeta}
\pmcanonicalname{HenselsLemmaForIntegers}
\pmcreated{2013-04-08 19:35:26}
\pmmodified{2013-04-08 19:35:26}
\pmowner{pahio}{2872}
\pmmodifier{pahio}{2872}
\pmtitle{Hensel's lemma for integers}
\pmrecord{7}{87337}
\pmprivacy{1}
\pmauthor{pahio}{2872}
\pmtype{Definition}
\pmclassification{msc}{11A07}

% this is the default PlanetMath preamble.  as your knowledge
% of TeX increases, you will probably want to edit this, but
% it should be fine as is for beginners.

% almost certainly you want these
\usepackage{amssymb}
\usepackage{amsmath}
\usepackage{amsfonts}

% need this for including graphics (\includegraphics)
\usepackage{graphicx}
% for neatly defining theorems and propositions
\usepackage{amsthm}

% making logically defined graphics
%\usepackage{xypic}
% used for TeXing text within eps files
%\usepackage{psfrag}

% there are many more packages, add them here as you need them

% define commands here

\begin{document}
Let $f(x)$ be a polynomial with integer coefficients, $p$ a prime number, and $n$ a positive integer.\, Assume that an integer $a$ (and naturally its whole residue class modulo $p^n$) satisfies the congruence
\begin{align}
f(x) \;\equiv\; 0 \;\; \pmod{p^n}.
\end{align}
The solution\, $x = a$\, of (1) may be refined in its residue class modulo $p^n$ to a solution\, $x = a\!+\!rp^n$\, of the congruence
\begin{align}
f(x) \;\equiv\; 0 \;\; \pmod{p^{n+1}}.
\end{align}
This refinement is unique modulo $p^{n+1}$ iff\, $f'(a) \not\equiv 0\, \pmod{p}$.\\

{\it Proof.}\; Now we have\, $f(a) = sp^n$.\, We have to find an $r$ such that
$$f(a\!+\!rp^n) \;\equiv\; 0 \;\; \pmod{p^{n+1}}.$$
The short Taylor theorem requires that
$$f(a\!+\!rp^n) \;\equiv\; f(a)+rf'(a)p^n \;\; 
\pmod{r^2p^{2n}}$$
where\, $2n \ge n\!+\!1$, whence this congruence can be simplified to
$$sp^n+rf'(a)p^n \;\equiv\; 0 \;\; \pmod{p^{n+1}}.$$
Thus the integer $r$ must satisfy the linear congruence
$$s+rf'(a) \;\equiv\; 0 \;\; \pmod{p}.$$
When\, $f'(a) \not\equiv 0$,\, this congruence has a unique solution modulo $p$ (see linear congruence); thus we have the refinement\, 
$a' = a\!+\!rp^n$\, which is unique modulo $p^{n+1}$.

When\, $f'(a) \equiv 0$\, and\, $s \not\equiv 0 \; \pmod{p}$,\, the congruence evidently is impossible.

In the case\, $f'(a) \equiv s \equiv 0 \; \pmod{p}$\, the congruence (2) is identically true in the residue class of $a$ modulo $p^n$. \qquad \Box




\begin{thebibliography}{8}
\bibitem{hack}{\sc Peter Hackman}: {\em Elementary Number Theory}. HHH Productions, Link\"oping (2009).
\end{thebibliography} 
\end{document}
