\documentclass[12pt]{article}
\usepackage{pmmeta}
\pmcanonicalname{SummatoryFunctionOfArithmeticFunction}
\pmcreated{2013-03-22 19:31:53}
\pmmodified{2013-03-22 19:31:53}
\pmowner{pahio}{2872}
\pmmodifier{pahio}{2872}
\pmtitle{summatory function of arithmetic function}
\pmrecord{7}{42508}
\pmprivacy{1}
\pmauthor{pahio}{2872}
\pmtype{Theorem}
\pmcomment{trigger rebuild}
\pmclassification{msc}{11A25}
\pmsynonym{summatory function}{SummatoryFunctionOfArithmeticFunction}
%\pmkeywords{Dirichlet convolution}
\pmrelated{EulerPhifunction}
\pmrelated{PrimeClass}

% this is the default PlanetMath preamble.  as your knowledge
% of TeX increases, you will probably want to edit this, but
% it should be fine as is for beginners.

% almost certainly you want these
\usepackage{amssymb}
\usepackage{amsmath}
\usepackage{amsfonts}

% used for TeXing text within eps files
%\usepackage{psfrag}
% need this for including graphics (\includegraphics)
%\usepackage{graphicx}
% for neatly defining theorems and propositions
 \usepackage{amsthm}
% making logically defined graphics
%%%\usepackage{xypic}

% there are many more packages, add them here as you need them

% define commands here

\theoremstyle{definition}
\newtheorem*{thmplain}{Theorem}

\begin{document}
\textbf{Definition.}\, The \emph{summatory function} $F$ of an arithmetic function $f$ is the Dirichlet convolution of $F$ and the constant function 1, i.e. 
$$F(n) \;=:\; \sum_{d\,\mid\,n}f(d)$$
where $d$ runs the positive divisors of the integer $n$.

It may be proved that the summatory function of a multiplicative function is multiplicative.\\

\textbf{Theorem.}\, The summatory function of the Euler phi function is the identity function:
$$\sum_{d\,\mid\,n}\varphi(d) \;=\; \sum_{d\,\mid\,n}\varphi\left(\frac{n}{d}\right) \;=\; n 
\quad \mbox{for all }\, n \in \mathbb{Z}_+.$$

\emph{Proof.}\, The first equality follows from the fact that any positive divisor of
$n$ is got from $n/d$ where $d$ is a divisor of $n$.
Further, let\, $1 \le m \le n$\, where\, $\gcd(m,\,n) = d$.\, Then\, $\gcd(m/d,\,n/d) = 1$\, and\,
$1 \le m/d \le n/d$.\, This defines a bijection between the prime classes modulo $n/d$ and such values of $m$ in $\{1,\,2,\,\ldots,\,n\!-\!1\}$ for which\, $\gcd(m,\,n) = d$.\, The number of the latters $\varphi(n/d)$.
Furthermore, the only $m$ with\, $1 \le m \le n$\, and\, $\gcd(m,\,n) = n$ is\, $m := n$,\, and\, $\varphi(n/n) = \varphi(1)$, by definition.\, Summing then over all possible values $d$ yields the second equality.

\begin{thebibliography}{8}
\bibitem{H}{\sc Peter Hackman}: {\em Elementary number theory}.\, HHH productions, Link\"oping (2009).
\end{thebibliography} 

%%%%%
%%%%%
\end{document}
