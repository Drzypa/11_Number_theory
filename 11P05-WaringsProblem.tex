\documentclass[12pt]{article}
\usepackage{pmmeta}
\pmcanonicalname{WaringsProblem}
\pmcreated{2013-03-22 13:19:46}
\pmmodified{2013-03-22 13:19:46}
\pmowner{bbukh}{348}
\pmmodifier{bbukh}{348}
\pmtitle{Waring's problem}
\pmrecord{13}{33841}
\pmprivacy{1}
\pmauthor{bbukh}{348}
\pmtype{Theorem}
\pmcomment{trigger rebuild}
\pmclassification{msc}{11P05}
\pmclassification{msc}{11B13}
\pmrelated{LagrangesFourSquareTheorem}
\pmrelated{Basis2}

\usepackage{amssymb}
\usepackage{amsmath}
\usepackage{amsfonts}

\makeatletter
\@ifundefined{bibname}{}{\renewcommand{\bibname}{References}}
\makeatother
\begin{document}
Waring asked whether it is possible to represent every natural number as a sum of \PMlinkname{bounded}{BoundedInterval} number of nonnegative $k$'th powers, that is, whether the set $\{\,n^k \mid n \in \mathbb{Z_+}\,\}$ is an \PMlinkname{additive basis}{Basis2}. He was led to this conjecture by \PMlinkname{Lagrange's theorem}{LagrangesFourSquareTheorem} which asserted that every natural number can be represented as a sum of four squares. 

Hilbert \cite{cite:hilbert_waring} was the first to prove the conjecture for all $k$. In his paper he did not give an explicit bound on $g(k)$, the number of powers needed, but later it was proved that 
\begin{equation*}
g(k)=2^k+\left\lfloor\left(\frac{3}{2}\right)^k\right\rfloor-2
\end{equation*}
except possibly finitely many exceptional $k$, none of which are known.

Wooley\cite{cite:wooley_waring}, improving the result of Vinogradov\cite{cite:vinogradov_waringG}, proved that the number of $k$'th powers needed to represent all \emph{sufficiently large} integers is
\begin{equation*}
G(k)\leq k (\ln k + \ln \ln k + O(1)).
\end{equation*}

\begin{thebibliography}{1}

\bibitem{cite:hilbert_waring}
David Hilbert.
\newblock Beweis f{\"u}r {D}arstellbarkeit der ganzen {Z}ahlen durch eine feste
  {A}nzahl $n$-ter {P}otenzen ({W}aringsches {P}roblem).
\newblock {\em Math. Ann.}, pages 281--300, 1909.
\newblock Available electronically from
  \PMlinkexternal{GDZ}{http://gdz.sub.uni-goettingen.de/en/index.html}.
  %\url{http://gdz.sub.uni-goettingen.de/en/index.html}.

\bibitem{cite:vaughan_circlemethod}
Robert~C. Vaughan.
\newblock {\em The Hardy-Littlewood method}.
\newblock Cambridge University Press, 1981.
\newblock \PMlinkexternal{Zbl 0868.11046}{http://www.emis.de/cgi-bin/zmen/ZMATH/en/quick.html?type=html&an=0868.11046}.

\bibitem{cite:vinogradov_waringG}
I.~M. Vinogradov.
\newblock On an upper bound for {$G(n)$}.
\newblock {\em Izv. Akad. Nauk SSSR. Ser. Mat.}, 23:637--642, 1959.
\newblock \PMlinkexternal{Zbl 0089.02703}{http://www.emis.de/cgi-bin/zmen/ZMATH/en/quick.html?type=html&an=0089.02703}.
%\newblock \PMlinkexternal{MR 22
%  \#699}{http://www.ams.org/mathscinet-getitem?mr=0109814}.


\bibitem{cite:wooley_waring}
Trevor~D. Wooley.
\newblock Large improvements in Waring's problem.
\newblock {\em Ann. Math}, 135(1):131--164, 1992.
\newblock \PMlinkexternal{Zbl 0754.11026}{http://www.emis.de/cgi-bin/zmen/ZMATH/en/quick.html?type=html&an=0754.11026}.
\newblock \PMlinkexternal{Available online}{http://links.jstor.org/sici?sici=0003-486X\%28199201\%292\%3A135\%3A1\%3C131\%3ALIIWP\%3E2.0.CO\%3B2-O} at \PMlinkexternal{JSTOR}{http://www.jstor.org}.


\end{thebibliography}
%
%\bibitem{cite:hilbert_waring}
%David Hilbert.
%\newblock Beweis f{\"u}r {D}arstellbarkeit der ganzen {Z}ahlen durch eine feste
%  {A}nzahl $n$-ter {P}otenzen ({W}aringsches {P}roblem).
%\newblock {\em Math. Ann.}, pages 281--300, 1909.
%\newblock Available electronically from
%  \url{http://gdz.sub.uni-goettingen.de/en/index.html}.
%
%%%%%
%%%%%
\end{document}
