\documentclass[12pt]{article}
\usepackage{pmmeta}
\pmcanonicalname{SquareRootOf2}
\pmcreated{2013-03-22 17:29:12}
\pmmodified{2013-03-22 17:29:12}
\pmowner{MathNerd}{17818}
\pmmodifier{MathNerd}{17818}
\pmtitle{square root of 2}
\pmrecord{10}{39874}
\pmprivacy{1}
\pmauthor{MathNerd}{17818}
\pmtype{Definition}
\pmcomment{trigger rebuild}
\pmclassification{msc}{11A25}
\pmsynonym{Pythagoras' constant}{SquareRootOf2}
\pmrelated{Surd}

\endmetadata

% this is the default PlanetMath preamble.  as your knowledge
% of TeX increases, you will probably want to edit this, but
% it should be fine as is for beginners.

% almost certainly you want these
\usepackage{amssymb}
\usepackage{amsmath}
\usepackage{amsfonts}

% used for TeXing text within eps files
%\usepackage{psfrag}
% need this for including graphics (\includegraphics)
%\usepackage{graphicx}
% for neatly defining theorems and propositions
%\usepackage{amsthm}
% making logically defined graphics
%%%\usepackage{xypic}

% there are many more packages, add them here as you need them

% define commands here

\begin{document}
The \emph{square root of 2} is an irrational number, the first to have been proved irrational. Its decimal expansion begins 1.41421356237309504880168872420969807856... (sequence \PMlinkexternal{A002194}{http://www.research.att.com/~njas/sequences/A002193} in Sloane's OEIS) Its simple continued fraction is $$1 + \frac{1}{2 + \frac{1}{2 + \frac{1}{2 + \frac{1}{2 + \ldots}}}},$$ periodically repeating the 2. Some call this number \emph{Pythagoras' constant}.

There are several different ways to express $\sqrt{2}$ as an infinite product. One way is $$\sqrt{2} = \prod_{i=0}^\infty\frac{(4i+2)^2}{(4i+1)(4i+3)},$$ another is $$\sqrt{2} = \sum_{i=0}^\infty (-1)^{i+1} \frac{(2i-3)!!}{(2i)!!}.$$

\begin{thebibliography}{1}
\bibitem{flannery} Flannery, David. {\it The square root of 2 : a dialogue concerning a number and a sequence}. New York: Copernicus, 2006.
\end{thebibliography}

%%%%%
%%%%%
\end{document}
