\documentclass[12pt]{article}
\usepackage{pmmeta}
\pmcanonicalname{UnitsOfQuadraticFields}
\pmcreated{2013-03-22 14:16:30}
\pmmodified{2013-03-22 14:16:30}
\pmowner{pahio}{2872}
\pmmodifier{pahio}{2872}
\pmtitle{units of quadratic fields}
\pmrecord{39}{35726}
\pmprivacy{1}
\pmauthor{pahio}{2872}
\pmtype{Application}
\pmcomment{trigger rebuild}
\pmclassification{msc}{11R04}
\pmclassification{msc}{11R27}
\pmsynonym{quadratic unit}{UnitsOfQuadraticFields}
\pmrelated{Unit}
\pmrelated{NumberField}
\pmrelated{ImaginaryQuadraticField}
\pmrelated{SomethingRelatedToFundamentalUnits}

% this is the default PlanetMath preamble.  as your knowledge
% of TeX increases, you will probably want to edit this, but
% it should be fine as is for beginners.

% almost certainly you want these
\usepackage{amssymb}
\usepackage{amsmath}
\usepackage{amsfonts}

% used for TeXing text within eps files
%\usepackage{psfrag}
% need this for including graphics (\includegraphics)
%\usepackage{graphicx}
% for neatly defining theorems and propositions
%\usepackage{amsthm}
% making logically defined graphics
%%%\usepackage{xypic}

% there are many more packages, add them here as you need them

% define commands here
\begin{document}
\PMlinkescapeword{primitive}
Dirichlet's unit theorem gives all units of an algebraic number field $\mathbb{Q}(\vartheta)$ in the unique form
   $$\varepsilon = \zeta^{n}\eta_1^{k_1}\eta_2^{k_2}...\eta_t^{k_t},$$
where $\zeta$ is a primitive $w^\mathrm{th}$ root of unity in $\mathbb{Q}(\vartheta)$, the $\eta_j$'s are the fundamental units of $\mathbb{Q}(\vartheta)$,\, $0 \leqq n \leqq w\!-\!1$,\, $k_j \in \mathbb{Z}$\, $\forall j$,\, $t = r\!+\!s\!-\!1$.

\begin{itemize}

\item The case of a real quadratic field $\mathbb{Q}(\sqrt{m})$, the square-free \,$m > 1$:\, $r = 2$,\, $s = 0$,\, $t = r\!+\!s\!-\!1 = 1$.\, So we obtain
     $$\varepsilon = \zeta^{n}\eta^{k} = \pm\eta^{k},$$ 
because\, $\zeta= -1$\, is the only real primitive root of unity ($w = 2$).\, Thus, every real quadratic field has infinitely many units and a unique fundamental unit $\eta$.

Examples:\, If\, $m = 3$,\, then\, $\eta = 2\!+\!\sqrt{3}$;\, if\, $m = 421$,\, then\, $\eta = \frac{444939+21685\sqrt{421}}{2}$.

\item The case of any imaginary quadratic field $\mathbb{Q}(\vartheta)$; here\, $\vartheta = \sqrt{m}$,\, the square-free\, $m < 0$:\, The conjugates of $\vartheta$ are the pure imaginary numbers $\pm\sqrt{m}$, hence\, $r = 0$,\, $2s = 2$,\, $t = r\!+\!s\!-\!1 = 0$.\, Thus we see that all units are 
                     $$\varepsilon = \zeta^{n}.$$

1) $m = -1$.\, The field contains the primitive fourth root of unity, e.g. $i$, and therefore all units in the {\em \PMlinkescapetext{Gaussian} field} $\mathbb{Q}(i)$ are $i^n$, where\, $n = 0,\,1,\,2,\,3$.

2) $m = -3$.\, The field in question is a \PMlinkname{cyclotomic field}{CyclotomicExtension} containing the primitive third root of unity and also the primitive sixth root of unity, namely
           $$\zeta = \cos{\frac{2\pi}{6}}+i\sin{\frac{2\pi}{6}};$$
hence all units are\, $\varepsilon = (\frac{1+\sqrt{-3}}{2})^{n}$,\, where\, $n = 0,\,1,\,\ldots,\,5$, or, equivalently,
\, $\varepsilon = \pm(\frac{-1+\sqrt{-3}}{2})^{n}$,\, where\, 
$n = 0,\,1,\,2$.

3) $m = -2$,\, $m <-3$. \, The only roots of unity in the field are $\pm 1$; hence\, $\zeta = -1$,\, $w = 2$,\, and the units of the field are simply\,
 $(-1)^{n}$, where\, $n = 0,\,1$.
                   
\end{itemize}
%%%%%
%%%%%
\end{document}
