\documentclass[12pt]{article}
\usepackage{pmmeta}
\pmcanonicalname{ExamplesOfProbablePrimes}
\pmcreated{2013-03-22 15:53:49}
\pmmodified{2013-03-22 15:53:49}
\pmowner{PrimeFan}{13766}
\pmmodifier{PrimeFan}{13766}
\pmtitle{examples of probable primes}
\pmrecord{5}{37899}
\pmprivacy{1}
\pmauthor{PrimeFan}{13766}
\pmtype{Example}
\pmcomment{trigger rebuild}
\pmclassification{msc}{11A41}

\endmetadata

% this is the default PlanetMath preamble.  as your knowledge
% of TeX increases, you will probably want to edit this, but
% it should be fine as is for beginners.

% almost certainly you want these
\usepackage{amssymb}
\usepackage{amsmath}
\usepackage{amsfonts}

% used for TeXing text within eps files
%\usepackage{psfrag}
% need this for including graphics (\includegraphics)
%\usepackage{graphicx}
% for neatly defining theorems and propositions
%\usepackage{amsthm}
% making logically defined graphics
%%%\usepackage{xypic}

% there are many more packages, add them here as you need them

% define commands here

\begin{document}
To give an example of a probable prime relative to a base: $4^{341233} - 3^{341233}$ has passed preliminary primality tests relative to bases 2, 3, 5, 7, 11, 13 and 101. Its square root is approximately $2.3362 \cdot 10^{102721}$, which makes a conclusive primality test by trial division in a reasonable time period impractical.

To give an example of a probable prime by a pattern: this pattern

$$2^2 - 1 = 3, 2^3 - 1 = 7, 2^7 - 1 = 127$$

$$2^{127} - 1 = 170141183460469231731687303715884105727$$

suggests that $2^{170141183460469231731687303715884105727} - 1$ might be a Mersenne prime. But since this is larger than the largest known Mersenne prime $2^{30402457} - 1$ (as of 2005), a Lucas-Lehmer test might take longer than the average human lifetime.

On the other hand, $123456789 \cdot 10^{123456789} + 123456789$ is not a probable prime, because even though it is much larger than either of the probable primes given above, it is clearly divisible by $3^2$.
%%%%%
%%%%%
\end{document}
