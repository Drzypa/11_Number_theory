\documentclass[12pt]{article}
\usepackage{pmmeta}
\pmcanonicalname{JacobiSymbol}
\pmcreated{2013-03-22 12:36:26}
\pmmodified{2013-03-22 12:36:26}
\pmowner{mathwizard}{128}
\pmmodifier{mathwizard}{128}
\pmtitle{Jacobi symbol}
\pmrecord{9}{32863}
\pmprivacy{1}
\pmauthor{mathwizard}{128}
\pmtype{Definition}
\pmcomment{trigger rebuild}
\pmclassification{msc}{11A07}
\pmclassification{msc}{11A15}
\pmrelated{LegendreSymbol}
\pmrelated{KroneckerSymbol}

\endmetadata

% this is the default PlanetMath preamble.  as your knowledge
% of TeX increases, you will probably want to edit this, but
% it should be fine as is for beginners.

% almost certainly you want these
\usepackage{amssymb}
\usepackage{amsmath}
\usepackage{amsfonts}

% used for TeXing text within eps files
%\usepackage{psfrag}
% need this for including graphics (\includegraphics)
%\usepackage{graphicx}
% for neatly defining theorems and propositions
%\usepackage{amsthm}
% making logically defined graphics
%%%\usepackage{xypic}

% there are many more packages, add them here as you need them

% define commands here
\begin{document}
The {\bf Jacobi symbol} is a generalization of the Legendre symbol to all odd positive integers.

Let $n$ be an odd positive integer, with prime factorization ${p_1}^{e_1} \cdots {p_k}^{e_k}$.  Let $a \geq 0$ be an integer.  The {\em Jacobi symbol}  $\left(\frac{a}{n}\right)$ is defined to be
\[ \left(\frac{a}{n}\right) = \prod_{i=1}^k \left(\frac{a}{p_i}\right)^{e_i} \]
where $\left(\frac{a}{p_i}\right)$ is the Legendre symbol of $a$ and $p_i$.

A further generalization of the Legendre symbol, due to Kronecker, is the Kronecker symbol.
%%%%%
%%%%%
\end{document}
