\documentclass[12pt]{article}
\usepackage{pmmeta}
\pmcanonicalname{ZeiselNumber}
\pmcreated{2013-03-22 17:44:03}
\pmmodified{2013-03-22 17:44:03}
\pmowner{CompositeFan}{12809}
\pmmodifier{CompositeFan}{12809}
\pmtitle{Zeisel number}
\pmrecord{6}{40183}
\pmprivacy{1}
\pmauthor{CompositeFan}{12809}
\pmtype{Definition}
\pmcomment{trigger rebuild}
\pmclassification{msc}{11A25}
\pmdefines{Brown-Zeisel number}

\endmetadata

% this is the default PlanetMath preamble.  as your knowledge
% of TeX increases, you will probably want to edit this, but
% it should be fine as is for beginners.

% almost certainly you want these
\usepackage{amssymb}
\usepackage{amsmath}
\usepackage{amsfonts}

% used for TeXing text within eps files
%\usepackage{psfrag}
% need this for including graphics (\includegraphics)
%\usepackage{graphicx}
% for neatly defining theorems and propositions
%\usepackage{amsthm}
% making logically defined graphics
%%%\usepackage{xypic}

% there are many more packages, add them here as you need them

% define commands here

\begin{document}
Given a squarefree integer $$n = \prod_{i = 0}^{\omega(n)} p_i,$$ with $\omega(n) > 2$ (in which $\omega(n)$ is number of distinct prime factors function, and all the $p_i$ are prime divisors of $n$, except $p_0 = 1$ purely as a notational convenience, and are sorted in ascending order) if each prime $p_i$ fits into the recurrence relation $p_i = mp_{i - 1} + a$, with $m$ being some fixed integer multiplicand, and $a$ being some fixed integer addend, then $n$ is called a {\em Zeisel number}.

For example, $1419 = 1 \times 3 \times 11 \times 43$. Say that $m = 4$ and $a = -1$. This checks out: $3 = m + a$, $11 = 3m + a$ and $43 = 11m + a$. 1419 is a Zeisel number. The first few Zeisel numbers are 105, 1419, 1729, 1885, 4505, 5719, ... listed in A051015 of Sloane's OEIS. The Carmichael numbers of the form $(6n + 1)(12n + 1)(18n + 1)$ are a subset of the Zeisel numbers; the constants are then $m = 1$ and $a = 6n$.

These numbers were first studied by Kevin Brown, who was searching for prime solutions to $2^{n - 1} + n$. Helmut Zeisel replied that 1885 is such an $n$. Brown discovered that its prime factors fit the recurrrence relation with $m = 2$ and $a = 3$. He called numbers fitting such a recurrence relation ``Zeisel numbers'' and the term has stuck, being taken up by the OEIS, MathWorld and Wikipedia. Zeisel himself has suggested the term ``Brown-Zeisel number'' but this has not caught on. There is a different concept of the Zeisel number used in chemistry.
%%%%%
%%%%%
\end{document}
