\documentclass[12pt]{article}
\usepackage{pmmeta}
\pmcanonicalname{UnitsOfRealCubicFieldsWithExactlyOneRealEmbedding}
\pmcreated{2013-03-22 16:02:25}
\pmmodified{2013-03-22 16:02:25}
\pmowner{Wkbj79}{1863}
\pmmodifier{Wkbj79}{1863}
\pmtitle{units of real cubic fields with exactly one real embedding}
\pmrecord{13}{38090}
\pmprivacy{1}
\pmauthor{Wkbj79}{1863}
\pmtype{Application}
\pmcomment{trigger rebuild}
\pmclassification{msc}{11R27}
\pmclassification{msc}{11R16}
\pmclassification{msc}{11R04}
\pmrelated{NormAndTraceOfAlgebraicNumber}

\endmetadata

% this is the default PlanetMath preamble.  as your knowledge
% of TeX increases, you will probably want to edit this, but
% it should be fine as is for beginners.

% almost certainly you want these
\usepackage{amssymb}
\usepackage{amsmath}
\usepackage{amsfonts}

% used for TeXing text within eps files
%\usepackage{psfrag}
% need this for including graphics (\includegraphics)
%\usepackage{graphicx}
% for neatly defining theorems and propositions
%\usepackage{amsthm}
% making logically defined graphics
%%%\usepackage{xypic}

% there are many more packages, add them here as you need them

% define commands here

\begin{document}
Let $K \subseteq \mathbb{R}$ be a number field with $[K\!:\!\mathbb{Q}]=3$ such that $K$ has exactly one real embedding.  Thus, $r=1$ and $s=1$.  Let ${\mathcal{O}_K}^*$ denote the group of units of the ring of integers of $K$.  By Dirichlet's unit theorem, ${\mathcal{O}_K}^* \cong \mu(K) \times \mathbb{Z}$ since $r+s-1=1$.  The only roots of unity in $K$ are $1$ and $-1$ because $K \subseteq \mathbb{R}$.  Thus, $\mu(K)=\{1,-1\}$.  Therefore, there exists $u \in {\mathcal{O}_K}^*$ with $u>1$, such that every element of ${\mathcal{O}_K}^*$ is of the form $\pm u^n$ for some $n \in \mathbb{Z}$.

Let $\rho>0$ and $0<\theta<\pi$ such that the conjugates of $u$ are $\rho e^{i\theta}$ and $\rho e^{-i\theta}$.  Since $u$ is a unit, $N(u)=\pm 1$.  Thus, $\pm 1=N(u)=u(\rho e^{i\theta})(\rho e^{-i\theta})=u\rho^2$.  Since $u>0$ and $\rho^2>0$, it must be the case that $u\rho^2=1$.  Thus, $\displaystyle u=\frac{1}{\rho^2}$.  One can then deduce that $\displaystyle \operatorname{disc}u=-4\sin^2\theta\left(\rho^3+\frac{1}{\rho^3}- 2\cos\theta\right)^2$.  Since the maximum value of the polynomial $4\sin^2\theta(x-2\cos\theta)^2-4x^2$ is at most $16$, one can deduce that $\displaystyle |\operatorname{disc}u| \le 4\left(u^3+\frac{1}{u^3}+4\right)$.  Define $d=|\operatorname{disc}\mathcal{O}_K|$.  Then $\displaystyle d\le|\operatorname{disc}u| \le 4\left(u^3+\frac{1}{u^3}+4\right)$.  Thus, $\displaystyle u^3 \ge \frac{d}{4}-4-\frac{1}{u^3}$.  From this, one can obtain that $\displaystyle u^3 \ge \frac{d-16+\sqrt{d^2-32d+192}}{8}$.  (Note that a higher lower bound on $u^3$ is desirable, and the one stated here is much higher than that stated in Marcus.)  Thus, $\displaystyle u^2 \ge \left( \frac{d-16+\sqrt{d^2-32d+192}}{8} \right)^{\frac{2}{3}}$.  Therefore, if an element $x \in {\mathcal{O}_K}^*$ can be found such that $\displaystyle 1<x<\left( \frac{d-16+\sqrt{d^2-32d+192}}{8} \right)^{\frac{2}{3}}$, then $x=u$.

Following are some applications:

\begin{itemize}
\item The above is most applicable for finding the fundamental unit of a ring of integers of a pure cubic field.  For example, if $K=\mathbb{Q}(\sqrt[3]{2})$, then $d=108$, and the lower bound on $u^2$ is $\displaystyle \left( \frac{23+10\sqrt{21}}{2} \right)^{\frac{2}{3}}$, which is larger than $9$.  Note that $\displaystyle \left( \sqrt[3]{4}+\sqrt[3]{2}+1 \right) \left( \sqrt[3]{2}-1 \right)=2-1=1$.  Since $1<\sqrt[3]{4}+\sqrt[3]{2}+1<9$, it follows that $\sqrt[3]{4}+\sqrt[3]{2}+1$ is the fundamental unit of $\mathcal{O}_K$.
\item The above can also be used for any number field $K$ with $[K\!:\!\mathbb{Q}]=3$ such that $K$ has exactly one real embedding.  Let $\sigma$ be the real embedding.  Then the above produces the fundamental unit $u$ of $\sigma(K)$.  Thus, $\sigma^{-1}(u)$ is a fundamental unit of $K$.
\end{itemize}

\begin{thebibliography}{9}
\bibitem{marcus} Marcus, Daniel A. {\em Number Fields}. New York: Springer-Verlag, 1977.
\end{thebibliography}
%%%%%
%%%%%
\end{document}
