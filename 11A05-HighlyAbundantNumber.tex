\documentclass[12pt]{article}
\usepackage{pmmeta}
\pmcanonicalname{HighlyAbundantNumber}
\pmcreated{2013-03-22 18:20:55}
\pmmodified{2013-03-22 18:20:55}
\pmowner{CompositeFan}{12809}
\pmmodifier{CompositeFan}{12809}
\pmtitle{highly abundant number}
\pmrecord{4}{40984}
\pmprivacy{1}
\pmauthor{CompositeFan}{12809}
\pmtype{Definition}
\pmcomment{trigger rebuild}
\pmclassification{msc}{11A05}

\endmetadata

% this is the default PlanetMath preamble.  as your knowledge
% of TeX increases, you will probably want to edit this, but
% it should be fine as is for beginners.

% almost certainly you want these
\usepackage{amssymb}
\usepackage{amsmath}
\usepackage{amsfonts}

% used for TeXing text within eps files
%\usepackage{psfrag}
% need this for including graphics (\includegraphics)
%\usepackage{graphicx}
% for neatly defining theorems and propositions
%\usepackage{amsthm}
% making logically defined graphics
%%%\usepackage{xypic}

% there are many more packages, add them here as you need them

% define commands here

\begin{document}
An integer $n$ is a {\em highly abundant number} if $\sigma(n) > \sigma(m)$ for all $m < n$ (with $\sigma$ being the sum of divisors function). The highly abundant numbers less than 100 are 1, 2, 3, 4, 6, 8, 10, 12, 16, 18, 20, 24, 30, 36, 42, 48, 60, 72, 84, 90, 96 (see A002093 in Sloane's OEIS). Highly abundant numbers are like highly composite numbers except the definition for the latter uses the divisor function $\tau$ instead of $\sigma$. The highly abundant numbers grow much more slowly than the highly composite numbers.

Though the first eight factorials are highly abundant, not all factorials are highly abundant. Two examples: 360360 is more abudant than 362880; and 3492720, 3538080, 3598560, 3603600 are all more abundant than 3628800.
%%%%%
%%%%%
\end{document}
