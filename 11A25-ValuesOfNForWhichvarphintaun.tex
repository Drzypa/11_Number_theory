\documentclass[12pt]{article}
\usepackage{pmmeta}
\pmcanonicalname{ValuesOfNForWhichvarphintaun}
\pmcreated{2013-03-22 18:03:48}
\pmmodified{2013-03-22 18:03:48}
\pmowner{Wkbj79}{1863}
\pmmodifier{Wkbj79}{1863}
\pmtitle{values of $n$ for which $\varphi(n)=\tau(n)$}
\pmrecord{12}{40595}
\pmprivacy{1}
\pmauthor{Wkbj79}{1863}
\pmtype{Feature}
\pmcomment{trigger rebuild}
\pmclassification{msc}{11A25}
\pmrelated{EulerPhifunction}
\pmrelated{TauFunction}

\usepackage{amssymb}
\usepackage{amsmath}
\usepackage{amsfonts}
\usepackage{pstricks}
\usepackage{psfrag}
\usepackage{graphicx}
\usepackage{amsthm}
%%\usepackage{xypic}

\newtheorem{lemma}{Lemma}

\newcommand{\ds}{\displaystyle}

\renewcommand{\phi}{\varphi}

\begin{document}
\PMlinkescapeword{consequence}
\PMlinkescapeword{derivative}
\PMlinkescapeword{focus}
\PMlinkescapeword{information}

Within this entry, we use the following notation:
\begin{itemize}
\item $\mathbb{N}$ denotes the natural numbers (positive integers)
\item $n\in\mathbb{N}$
\item $p$ denotes a prime
\item $k\in\mathbb{N}\cup\{0\}$
\item $\phi$ denotes the Euler phi function
\item $\tau$ denotes the divisor function
\item $\mid$ denotes divides
\item $\Vert$ denotes exactly divides
\end{itemize}

Within this entry, we will determine all values of $n$ for which $\phi(n)=\tau(n)$.

Define $\gamma \colon \mathbb{N} \to \mathbb{Q}$ by
\[
\gamma(n)=\frac{\tau(n)}{\phi(n)}.
\]
Note that $\gamma$ is a multiplicative function since both $\phi$ and $\tau$ are.  Thus, we will initially focus on the values of $\gamma$ at prime powers.  We will need specific values of $\gamma(p^k)$.  These are calculated below.

\begin{center}
$\begin{array}{ccccc}
\gamma(1)  &=& \ds \frac{1}{1}  &=& 1 \\ \\
\gamma(2)  &=& \ds \frac{2}{1}  &=& 2 \\ \\
\gamma(4)  &=& \ds \frac{3}{2} \\ \\
\gamma(8)  &=& \ds \frac{4}{4}  &=& 1 \\ \\
\gamma(16) &=& \ds \frac{5}{8} \\ \\
\gamma(32) &=& \ds \frac{6}{16} &=& \ds \frac{3}{8} \\ \\ \\
\gamma(3)  &=& \ds \frac{2}{2}  &=& 1 \\ \\
\gamma(9)  &=& \ds \frac{3}{6}  &=& \ds \frac{1}{2} \\ \\ \\
\gamma(5)  &=& \ds \frac{2}{4}  &=& \ds \frac{1}{2}
\end{array}$
\end{center}

Note that
\begin{center}
$\begin{array}{rl}
\gamma(p^k) & =\ds \frac{\tau(p^k)}{\phi(p^k)} \\ \\
& =\ds \frac{k+1}{p^{k-1}(p-1)}.
\end{array}$
\end{center}
If $p$ is fixed, we can extend this to a continuous function $\Gamma_p \colon \mathbb{R} \to \mathbb{R}$ defined by
\[
\Gamma_p(x)=\frac{x+1}{p^{x-1}(p-1)}.
\]

We investigate the \PMlinkname{derivative}{Derivative} of $\Gamma_p$ for $x\ge 1$:
\begin{center}
$\begin{array}{rl}
{\Gamma_p}'(x) & =\ds \frac{1}{p-1} \cdot \frac{p^{x-1}-(x+1)p^{x-1}\ln p}{(p^{x-1})^2} \\ \\
& =\ds \frac{1-(x+1)\ln p}{p^{x-1}(p-1)} \\ \\
& <\ds \frac{1-2\ln 2}{p^{x-1}(p-1)} \\ \\
& <0.
\end{array}$
\end{center}
Thus, for $p$ fixed and $k\ge 1$, $\gamma(p^k)$ is a strictly decreasing function of $k$.

On the other hand, from the equation
\[
\gamma(p^k)=\frac{k+1}{p^{k-1}(p-1)},
\]
it is clear that, if $k$ is fixed, $\gamma(p^k)$ is a strictly decreasing function of $p$.

Thus, we have proven the following:

\begin{lemma}
Let $p$ be a prime and $k$ be a nonnegative integer with $p^k\notin\{1,2,3,4,8,16\}$.  Then
\[
\gamma(p^k)\le\frac{1}{2}
\]
with equality holding if and only if $p^k\in\{5,9\}$.
\end{lemma}

This lemma has an immediate consequence:

\begin{lemma}
Let $m$ be an odd natural number.  Then
\[
\gamma(m)=1 \text{ or } \gamma(m)\le\frac{1}{2}.
\]
Moreover, $\gamma(m)=1$ if and only if $m=1$ or $m=3$.
\end{lemma}

Now we will examine the general case.  Let $\phi(n)=\tau(n)$.  Then $\gamma(n)=1$.

Suppose that $4 \Vert n$.  Let $m$ be an odd natural number with $n=4m$.  Thus,
\[
1=\gamma(n)=\gamma(4m)=\gamma(4)\gamma(m)=\frac{3}{2}\gamma(m).
\]
Therefore,
\[
\gamma(m)=\frac{2}{3},
\]
which contradicts the second lemma.  Hence, $4 \not\Vert n$.

Suppose that $16 \mid n$.  Let $m$ be an odd natural number with $n=2^km$.  Then $k\ge 4$.  Thus,
\[
1=\gamma(n)=\gamma(2^km)=\gamma(2^k)\gamma(m)\le\gamma(16)\gamma(m)=\frac{5}{8}\gamma(m).
\]
Therefore,
\[
\gamma(m)\ge\frac{8}{5},
\]
which contradicts the second lemma.  Hence, $16 \nmid n$.

Now we deal with the cases that can actually occur.

\begin{itemize}
\item Case I: $n$ is odd

The second lemma immediately applies, yielding $n=1$ or $n=3$.

\item Case II: $2 \Vert n$ and $3 \nmid n$

Let $m$ be an odd natural number with $n=2m$.  Then $3 \nmid m$ and
\[
1=\gamma(n)=\gamma(2m)=\gamma(2)\gamma(m)=2\gamma(m).
\]
Thus,
\[
\gamma(m)=\frac{1}{2}.
\]
By the first lemma, for all $p^k \Vert m$ with $k>0$,
\[
\gamma(p^k)\le\frac{1}{2}
\]
with equality holding if and only if $p^k=5$.  Therefore, $m=5$.  Hence $n=10$.

\item Case III: $2 \Vert n$ and $3 \Vert n$

Let $m$ be an odd natural number with $n=6m$.  Then $3 \nmid m$ and
\[
1=\gamma(n)=\gamma(6m)=\gamma(2)\gamma(3)\gamma(m)=2\gamma(m).
\]
Thus,
\[
\gamma(m)=\frac{1}{2}.
\]
By the first lemma, for all $p^k \Vert m$ with $k>0$,
\[
\gamma(p^k)\le\frac{1}{2}
\]
with equality holding if and only if $p^k=5$.  Therefore, $m=5$.  Hence $n=30$.

\item Case IV: $2 \Vert n$ and $9 \mid n$

Let $m$ be an odd natural number with $3 \nmid m$ such that $n=2\cdot 3^k m$.  Then $k\ge 2$ and
\[
1=\gamma(n)=\gamma(2\cdot 3^k m)=\gamma(2)\gamma(3^k)\gamma(m)=2\gamma(3^k)\gamma(m)\le 2\gamma(9)\gamma(m)=\gamma(m).
\]
Since $3 \nmid m$, the second lemma yields that $m=1$.  Thus,
\[
1=\gamma(n)=\gamma(2\cdot 3^k)=\gamma(2)\gamma(3^k)=2\gamma(3^k).
\]
Therefore,
\[
\gamma(3^k)=\frac{1}{2}.
\]
By the first lemma, $k=2$.  Hence, $n=18$.

\item Case V: $8 \mid n$

Recall that $16 \nmid n$.  Thus, there exists an odd natural number $m$ with $n=8m$.  Then
\[
1=\gamma(n)=\gamma(8m)=\gamma(8)\gamma(m)=\gamma(m).
\]
The second lemma yields that $m=1$ or $m=3$.  Hence, $n=8$ or $n=24$.
\end{itemize}

It follows that
\[
\{n\in\mathbb{N}:\phi(n)=\tau(n)\}=\{1,3,8,10,18,24,30\}.
\]
This list of numbers appears in the OEIS as sequence \PMlinkexternal{A020488}{http://www.research.att.com/~njas/sequences/A020488}.
%%%%%
%%%%%
\end{document}
