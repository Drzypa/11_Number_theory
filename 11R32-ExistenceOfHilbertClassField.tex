\documentclass[12pt]{article}
\usepackage{pmmeta}
\pmcanonicalname{ExistenceOfHilbertClassField}
\pmcreated{2013-03-22 12:36:45}
\pmmodified{2013-03-22 12:36:45}
\pmowner{mathcam}{2727}
\pmmodifier{mathcam}{2727}
\pmtitle{existence of Hilbert class field}
\pmrecord{16}{32870}
\pmprivacy{1}
\pmauthor{mathcam}{2727}
\pmtype{Theorem}
\pmcomment{trigger rebuild}
\pmclassification{msc}{11R32}
\pmclassification{msc}{11R29}
\pmclassification{msc}{11R37}
\pmrelated{IdealClass}
\pmrelated{Group}
\pmrelated{NumberField}
\pmrelated{ClassNumberDivisibilityInExtensions}
\pmrelated{RootDiscriminant}
\pmrelated{ExtensionsWithoutUnramifiedSubextensionsAndClassNumberDivisibility}
\pmrelated{ClassNumbersAndDiscriminantsTopicsOnClassGroups}
\pmdefines{Hilbert class field}

% this is the default PlanetMath preamble.  as your knowledge
% of TeX increases, you will probably want to edit this, but
% it should be fine as is for beginners.

% almost certainly you want these
\usepackage{amssymb}
\usepackage{amsmath}
\usepackage{amsfonts}

% used for TeXing text within eps files
%\usepackage{psfrag}
% need this for including graphics (\includegraphics)
%\usepackage{graphicx}
% for neatly defining theorems and propositions
%\usepackage{amsthm}
% making logically defined graphics
%%%\usepackage{xypic}

% there are many more packages, add them here as you need them

% define commands here
\newcommand{\rai}[1]{\mathcal{O}_{#1}}
\begin{document}
Let $K$ be a number field.  There exists a finite extension $E$ of $K$ with the following properties:
  \begin{enumerate}
  \item $[E:K]=h_K$, where $h_K$ is the class number of $K$.
  \item $E$ is Galois over $K$.
  \item The ideal class group of $K$ is isomorphic to the Galois group of
        $E$ over $K$.
  \item Every ideal of $\rai{K}$ is a principal ideal of the ring extension $\rai{E}$.
  \item Every prime ideal ${\cal P}$ of $\rai{K}$ decomposes into the product of
        $\frac{h_K}{f}$ prime ideals in $\rai{E}$, where $f$ is the \PMlinkname{order}{Order}
        of $[{\cal P}]$ in the ideal class group of $\rai{E}$.
  \end{enumerate}
There is a unique field $E$ satisfying the above five properties, and it is known as the {\em Hilbert class field} of $K$.

The field $E$ may also be characterized as the \PMlinkname{maximal abelian unramified}{AbelianExtension} extension of $K$.  Note that in this context, the term `unramified' is meant not only for the finite places (the classical ideal theoretic \PMlinkescapetext{ interpretation}) but also for the infinite places.  That is, every real embedding of $K$ extends to a real embedding of $E$.  As an example of why this is necessary, consider some real quadratic field.
%%%%%
%%%%%
\end{document}
