\documentclass[12pt]{article}
\usepackage{pmmeta}
\pmcanonicalname{EIsNotAQuadraticIrrational}
\pmcreated{2013-03-22 14:04:06}
\pmmodified{2013-03-22 14:04:06}
\pmowner{mathcam}{2727}
\pmmodifier{mathcam}{2727}
\pmtitle{e is not a quadratic irrational}
\pmrecord{11}{35426}
\pmprivacy{1}
\pmauthor{mathcam}{2727}
\pmtype{Proof}
\pmcomment{trigger rebuild}
\pmclassification{msc}{11J72}
\pmclassification{msc}{26E99}
\pmrelated{EIsIrrationalProof}
\pmrelated{ErIsIrrationalForRinmathbbQsetminus0}
\pmrelated{EIsTranscendental}

\endmetadata

% this is the default PlanetMath preamble.  as your knowledge
% of TeX increases, you will probably want to edit this, but
% it should be fine as is for beginners.

% almost certainly you want these
\usepackage{amssymb}
\usepackage{amsmath}
\usepackage{amsfonts}

% used for TeXing text within eps files
%\usepackage{psfrag}
% need this for including graphics (\includegraphics)
%\usepackage{graphicx}
% for neatly defining theorems and propositions
%\usepackage{amsthm}
% making logically defined graphics
%%%\usepackage{xypic}

% there are many more packages, add them here as you need them

% define commands here
\newcommand*{\abs}[1]{\left\lvert #1\right\rvert}
\begin{document}
We wish to show that $e$ is not a quadratic irrational, i.e. $\mathbb{Q}(e)$ is not a quadratic extension of $\mathbb{Q}$.  To do this, we show that it can not be the root of any quadratic polynomial with integer coefficients.

We begin by looking at the Taylor series for $e^x$:
\begin{equation*}
  e^x=\sum_{k=0}^{\infty}\frac{x^k}{k!}.
\end{equation*}

This converges for every $x\in\mathbb{R}$, so $e=\sum_{k=0}^{\infty}\frac{1}{k!}$ and $e^{-1}=\sum_{k=0}^{\infty}(-1)^k\frac{1}{k!}$. Arguing
by contradiction, assume $ae^2+be+c=0$ for integers
$a$, $b$ and $c$. That is the same as $ae+b+ce^{-1}=0$. 

Fix $n>\abs{a}+\abs{c}$, then $a,c\mid n!$ and $\forall k\le n$, $k!\mid n!\;$. 
Consider
\begin{align*}
0=n!(ae+b+ce^{-1})&=an!\sum_{k=0}^{\infty}\frac{1}{k!}+b+ cn!\sum_{k=0}^{\infty}(-1)^k\frac{1}{k!}\\
&=b+\sum_{k=0}^n (a+c(-1)^k)\frac{n!}{k!}+\sum_{k=n+1}^\infty (a+c(-1)^k)\frac{n!}{k!}
\end{align*}
Since $k!\mid n!$ for $k\le n$, the first two terms are integers. So the third term should be an integer. However,
\begin{align*}
\abs{\sum_{k=n+1}^\infty (a+c(-1)^k)\frac{n!}{k!}}&\le (\abs{a}+\abs{c})\sum_{k=n+1}^\infty \frac{n!}{k!}\\
&=(\abs{a}+\abs{c})\sum_{k=n+1}^\infty \frac{1}{(n+1)(n+2)\dotsb k}\\
&\le (\abs{a}+\abs{c})\sum_{k=n+1}^\infty (n+1)^{n-k}\\
&=(\abs{a}+\abs{c})\sum_{t=1}^\infty (n+1)^{-t}\\
&=(\abs{a}+\abs{c})\frac{1}{n}
\end{align*}
is less than $1$ by our assumption that $n>\abs{a}+\abs{c}$. Since there is only one integer which is less than $1$ in absolute value, this means that $\sum_{k=n+1}^\infty (a+c(-1)^k)\frac{1}{k!}=0$ for every sufficiently large $n$ which is not the case because
\begin{equation*}
\sum_{k=n+1}^\infty (a+c(-1)^k)\frac{1}{k!}-\sum_{k=n+2}^\infty (a+c(-1)^k)\frac{1}{k!}=(a+c(-1)^{n+1})\frac{1}{(n+1)!}
\end{equation*}
is not identically zero. The contradiction completes the proof.
%%%%%
%%%%%
\end{document}
