\documentclass[12pt]{article}
\usepackage{pmmeta}
\pmcanonicalname{TestForPrimalityOfMersenneNumbers}
\pmcreated{2013-03-22 13:39:48}
\pmmodified{2013-03-22 13:39:48}
\pmowner{bbukh}{348}
\pmmodifier{bbukh}{348}
\pmtitle{test for primality of Mersenne numbers}
\pmrecord{8}{34320}
\pmprivacy{1}
\pmauthor{bbukh}{348}
\pmtype{Algorithm}
\pmcomment{trigger rebuild}
\pmclassification{msc}{11A41}
\pmclassification{msc}{11Y11}
\pmclassification{msc}{11A51}

\usepackage{amssymb}
\usepackage{amsmath}
\usepackage{amsfonts}


%%%\usepackage{xypic}

\makeatletter
\@ifundefined{bibname}{}{\renewcommand{\bibname}{References}}
\makeatother
\begin{document}
Suppose $p$ is an odd prime, and define a sequence $L_n$ recursively as
\begin{equation*}
L_0=4,\qquad L_{n+1}=(L_n^2-2) \bmod (2^p-1).
\end{equation*}
The number $2^p-1$ is prime if and only if $L_{p-2}=0$.

\begin{thebibliography}{1}

\bibitem{knuth_art_prog:2}
Donald~E. Knuth.
\newblock {\em The Art of Computer Programming}, volume~2.
\newblock Addison-Wesley, 1969.

\end{thebibliography}

%@BOOK{knuth_art_prog:2,
% author     = {Donald E. Knuth},
% title      = {The Art of Computer Programming},
% publisher  = {Addison-Wesley},
% year       = {1969},
% volume     = {2}
%}
%%%%%
%%%%%
\end{document}
