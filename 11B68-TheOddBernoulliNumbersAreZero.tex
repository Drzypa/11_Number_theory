\documentclass[12pt]{article}
\usepackage{pmmeta}
\pmcanonicalname{TheOddBernoulliNumbersAreZero}
\pmcreated{2013-03-22 15:12:04}
\pmmodified{2013-03-22 15:12:04}
\pmowner{alozano}{2414}
\pmmodifier{alozano}{2414}
\pmtitle{the odd Bernoulli numbers are zero}
\pmrecord{5}{36959}
\pmprivacy{1}
\pmauthor{alozano}{2414}
\pmtype{Theorem}
\pmcomment{trigger rebuild}
\pmclassification{msc}{11B68}
%\pmkeywords{Bernoulli number}
\pmrelated{KummersCongruence}
\pmrelated{CongruenceOfClausenAndVonStaudt}

\endmetadata

% this is the default PlanetMath preamble.  as your knowledge
% of TeX increases, you will probably want to edit this, but
% it should be fine as is for beginners.

% almost certainly you want these
\usepackage{amssymb}
\usepackage{amsmath}
\usepackage{amsthm}
\usepackage{amsfonts}

% used for TeXing text within eps files
%\usepackage{psfrag}
% need this for including graphics (\includegraphics)
%\usepackage{graphicx}
% for neatly defining theorems and propositions
%\usepackage{amsthm}
% making logically defined graphics
%%%\usepackage{xypic}

% there are many more packages, add them here as you need them

% define commands here

\newtheorem{thm}{Theorem}
\newtheorem{defn}{Definition}
\newtheorem{prop}{Proposition}
\newtheorem*{lemma}{Lemma}
\newtheorem{cor}{Corollary}

\theoremstyle{definition}
\newtheorem{exa}{Example}

% Some sets
\newcommand{\Nats}{\mathbb{N}}
\newcommand{\Ints}{\mathbb{Z}}
\newcommand{\Reals}{\mathbb{R}}
\newcommand{\Complex}{\mathbb{C}}
\newcommand{\Rats}{\mathbb{Q}}
\newcommand{\Gal}{\operatorname{Gal}}
\newcommand{\Cl}{\operatorname{Cl}}
\begin{document}
Recall that, for $k\geq 0$, the Bernoulli numbers $B_k$ are defined as the coefficients in the Taylor expansion:
\begin{eqnarray}
\label{ber} \frac{t}{e^t-1}=\sum_{k\geq 0} B_k \frac{t^k}{k!}.
\end{eqnarray}
Just to name a few:
$$B_0=1,\quad B_1=-\frac{1}{2},\quad B_2=\frac{1}{6},\quad B_3=0,\quad B_4=-\frac{1}{30},\ B_5=0,\ldots,\ B_{10}=\frac{5}{66},\ldots $$
\begin{lemma}
If $k\geq 3$ is odd then $B_k=0$.
\end{lemma}
\begin{proof}
From the right hand side of (\ref{ber}) we extract the term corresponding to $k=1$:
\begin{eqnarray}
\label{ber2} \frac{t}{e^t-1}=-\frac{t}{2}+\sum_{k\geq 0,\ k\neq 1} B_k \frac{t^k}{k!}.
\end{eqnarray}
Thus:
\begin{eqnarray}
\label{ber21} \frac{t}{e^t-1}+\frac{t}{2}=\sum_{k\geq 0,\ k\neq 1} B_k \frac{t^k}{k!}
\end{eqnarray}
and the left hand side can be rewritten as:
\begin{eqnarray}
\label{ber3} \frac{t}{e^t-1}+\frac{t}{2}= \frac{2t+t(e^t-1)}{2(e^t-1)} =
\frac{t}{2}\cdot \frac{e^t+1}{e^t-1}=\frac{t}{2} \cdot \frac{e^{t/2}+e^{-t/2}}{e^{t/2}-e^{-t/2}}.
\end{eqnarray}
Hence, if one replaces $t$ by $-t$ then (\ref{ber3}) is unchanged. Since (\ref{ber3}) is the left hand side of (\ref{ber21}), the quantity 
$$\sum_{k\geq 0,\ k\neq 1} B_k \frac{t^k}{k!}$$
is also unchanged when $t$ is exchanged by $-t$, and so we must have $B_k=(-1)^kB_k$ for $k\neq 1$. We conclude that if $k\geq 3$ and $k$ is odd, $B_k=0$.
\end{proof}
%%%%%
%%%%%
\end{document}
