\documentclass[12pt]{article}
\usepackage{pmmeta}
\pmcanonicalname{CanonicalBasis}
\pmcreated{2015-02-06 13:12:19}
\pmmodified{2015-02-06 13:12:19}
\pmowner{pahio}{2872}
\pmmodifier{pahio}{2872}
\pmtitle{canonical basis}
\pmrecord{14}{37165}
\pmprivacy{1}
\pmauthor{pahio}{2872}
\pmtype{Theorem}
\pmcomment{trigger rebuild}
\pmclassification{msc}{11R04}
\pmrelated{MinimalityOfIntegralBasis}
\pmrelated{ExamplesOfRingOfIntegersOfANumberField}
\pmrelated{ConditionForPowerBasis}
\pmrelated{IntegralBasisOfQuadraticField}
\pmrelated{CanonicalFormOfElementOfNumberField}
\pmdefines{canonical basis}
\pmdefines{canonical basis of a number field}
\pmdefines{adjusted canonical basis}

\endmetadata

% this is the default PlanetMath preamble.  as your knowledge
% of TeX increases, you will probably want to edit this, but
% it should be fine as is for beginners.

% almost certainly you want these
\usepackage{amssymb}
\usepackage{amsmath}
\usepackage{amsfonts}

% used for TeXing text within eps files
%\usepackage{psfrag}
% need this for including graphics (\includegraphics)
%\usepackage{graphicx}
% for neatly defining theorems and propositions
 \usepackage{amsthm}
% making logically defined graphics
%%%\usepackage{xypic}

% there are many more packages, add them here as you need them

% define commands here

\theoremstyle{definition}
\newtheorem*{thmplain}{Theorem}
\begin{document}
Let $\vartheta$ be an algebraic integer of \PMlinkname{degree}{ExtensionField} $n$.\, The algebraic number field $\mathbb{Q}(\vartheta)$ has always an integral basis of the form 

$\displaystyle\omega_1 = 1,$\\
$\displaystyle\omega_2 = \frac{a_{21}\!+\!\vartheta}{d_2},$\\
$\displaystyle\omega_3 = \frac{a_{31}\!+\!a_{32}\vartheta\!+\!\vartheta^2}{d_3},$\\
$\vdots\,\qquad\vdots\,\qquad\vdots$\\
$\displaystyle\omega_n = \frac{a_{n1}\!+\!a_{n2}\vartheta\!+\ldots+\!a_{n,n-1}\vartheta^{n-2}\!+\!\vartheta^{n-1}}{d_n}$,

where the $a_{ij}$'s and $d_i$'s are rational integers such that
               $$d_2\mid d_3\mid d_4\mid\ldots\mid d_n,$$
i.e.
    $$d_i\mid d_{i+1}\quad \forall\, i = 2,\,3,\,\ldots,\,n\!-\!1.$$

The integral basis\, $\omega_1,\,\omega_2,\,\ldots,\,\omega_n$\ is called a {\em canonical basis} of the number field.

\textbf{Remark.}\, The integers $a_{ij}$ can be reduced so that for all $i$ and $j$, 
         $$-\frac{d_i}{2} < a_{ij} \leqq \frac{d_i}{2}.$$
Then one may speak of an {\em adjusted canonical basis}.\, In the case of a quadratic number field $\mathbb{Q}(\sqrt{d})$ with\, 
$d \equiv 1\, (\mbox{mod}\, 4)$\, we have (see the examples of ring of integers of a number field)
    $$\omega_1 = 1,  \quad \omega_2 = \frac{1\!+\!\sqrt{d}}{2}.$$
The discriminant of this basis is $d$.
%%%%%
%%%%%
\end{document}
