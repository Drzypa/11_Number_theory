\documentclass[12pt]{article}
\usepackage{pmmeta}
\pmcanonicalname{Repunit}
\pmcreated{2013-03-22 16:13:26}
\pmmodified{2013-03-22 16:13:26}
\pmowner{PrimeFan}{13766}
\pmmodifier{PrimeFan}{13766}
\pmtitle{repunit}
\pmrecord{5}{38322}
\pmprivacy{1}
\pmauthor{PrimeFan}{13766}
\pmtype{Definition}
\pmcomment{trigger rebuild}
\pmclassification{msc}{11A63}
\pmdefines{repunit prime}

% this is the default PlanetMath preamble.  as your knowledge
% of TeX increases, you will probably want to edit this, but
% it should be fine as is for beginners.

% almost certainly you want these
\usepackage{amssymb}
\usepackage{amsmath}
\usepackage{amsfonts}

% used for TeXing text within eps files
%\usepackage{psfrag}
% need this for including graphics (\includegraphics)
%\usepackage{graphicx}
% for neatly defining theorems and propositions
%\usepackage{amsthm}
% making logically defined graphics
%%%\usepackage{xypic}

% there are many more packages, add them here as you need them

% define commands here

\begin{document}
Given base $b$, a number of the form ${{b^n - 1} \over {b - 1}}$ for $n > 0$ is written using using only the digit 1 in that base and is therefore a {\em repunit}. The term, short for "repeated unit," is credited to Beiler's book {\it Recreations in the theory of numbers}, in chapter 11.

Regardless of base, a prime number is a prime number, but if in a given base it is a repunit, then it is called a {\em repunit prime} in that base. In binary, the Mersenne numbers are repunits, therefore the Mersenne primes are repunit primes in binary. Repunit primes in base 10 appear to be fewer, with only seven known as of 2006, for $n$ taking on the values 2, 19, 23, 317, 1031, 49081, 86453 (see Sloane's OEIS, A004023, for updates).

In a trivial way, repunit primes are also palindromic primes and permutable primes.
%%%%%
%%%%%
\end{document}
