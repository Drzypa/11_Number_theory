\documentclass[12pt]{article}
\usepackage{pmmeta}
\pmcanonicalname{ContraharmonicDiophantineEquation}
\pmcreated{2013-11-19 21:49:13}
\pmmodified{2013-11-19 21:49:13}
\pmowner{pahio}{2872}
\pmmodifier{pahio}{2872}
\pmtitle{contraharmonic Diophantine equation}
\pmrecord{5}{87925}
\pmprivacy{1}
\pmauthor{pahio}{2872}
\pmtype{Derivation}
\pmclassification{msc}{11D09}
\pmclassification{msc}{11D45}

\endmetadata

% this is the default PlanetMath preamble.  as your knowledge
% of TeX increases, you will probably want to edit this, but
% it should be fine as is for beginners.

% almost certainly you want these
\usepackage{amssymb}
\usepackage{amsmath}
\usepackage{amsfonts}

% need this for including graphics (\includegraphics)
\usepackage{graphicx}
% for neatly defining theorems and propositions
\usepackage{amsthm}

% making logically defined graphics
%\usepackage{xypic}
% used for TeXing text within eps files
%\usepackage{psfrag}

% there are many more packages, add them here as you need them

% define commands here

\begin{document}
We call {\it contraharmonic Diophantine equation} the equation
\begin{align}
u^2+v^2 \;=\; (u+v)c
\end{align}
of the three unknowns $u$, $v$, $c$ required to get only positive integer 
values.\, The equation expresses that $c$ is the contraharmonic 
mean of $u$ and $v$.\, As proved in the article 
``contraharmonic means and Pythagorean 
hypotenuses'', the supposition $u \neq v$ implies that the 
number $c$ must be the hypotenuse in a Pythagorean triple
$(a, b, c)$, and if particularly $u < v$, then
\begin{align}
u \;=\; \frac{c+b-a}{2}, \quad v \;=\; \frac{c+b+a}{2}.
\end{align}
For getting the general solution of the quadratic Diophantine 
equation (1), one can utilise the general formulas for 
Pythagorean triples
\begin{align}
a \;=\; l\!\cdot\!(m^2-n^2), \quad b \;=\; l\!\cdot\!2mn, 
\quad c \;=\; l\!\cdot\!(m^2+n^2)
\end{align}
where the parameters $l$, $m$, $n$ are arbitrary positive 
integers with\, $m > n$.\, Using (3) in (2) one obtains the 
result
\begin{align}
\begin{cases}
  u_1 \;=\; l(m^2-mn),\\
  u_2 \;=\; l(n^2+mn),\\
  v \;=\; l(m^2+mn),\\
  c \;=\; l(m^2+n^2), 
\end{cases}
\end{align}
in which $u_1$ and $u_2$ mean the alternative values for $u$ gotten from (2) by swapping 
the expressions of $a$ and $b$ in (3). 

It's clear that the contraharmonic Diophantine equation has an 
infinite set of solutions (4).\, According to the Proposition 
6 of the article ``integer contraharmonic means'', fixing e.g. 
the variable $u$ allows for the equation only a restricted 
number of pertinent values $v$ and $c$.\, See also the 
alternative expressions (1) and (2) in the article ``sums of 
two squares''.\\



\end{document}
