\documentclass[12pt]{article}
\usepackage{pmmeta}
\pmcanonicalname{ChebyshevFunctions}
\pmcreated{2013-03-22 13:50:15}
\pmmodified{2013-03-22 13:50:15}
\pmowner{Mathprof}{13753}
\pmmodifier{Mathprof}{13753}
\pmtitle{Chebyshev functions}
\pmrecord{11}{34573}
\pmprivacy{1}
\pmauthor{Mathprof}{13753}
\pmtype{Definition}
\pmcomment{trigger rebuild}
\pmclassification{msc}{11A41}
\pmrelated{MangoldtSummatoryFunction}

\usepackage{amssymb}
\usepackage{amsmath}
\usepackage{amsfonts}
\usepackage{amsthm}

\newcommand{\mc}{\mathcal}
\newcommand{\mb}{\mathbb}
\newcommand{\mf}{\mathfrak}
\newcommand{\ol}{\overline}
\newcommand{\ra}{\rightarrow}
\newcommand{\la}{\leftarrow}
\newcommand{\La}{\Leftarrow}
\newcommand{\Ra}{\Rightarrow}
\newcommand{\nor}{\vartriangleleft}
\newcommand{\Gal}{\text{Gal}}
\newcommand{\GL}{\text{GL}}
\newcommand{\Z}{\mb{Z}}
\newcommand{\R}{\mb{R}}
\newcommand{\Q}{\mb{Q}}
\newcommand{\C}{\mb{C}}
\begin{document}
There are two different functions which are collectively known as the \emph{Chebyshev functions}:

\begin{align*}
\vartheta(x)=\sum_{p\leq x}\log p.
\end{align*}
where the notation used indicates the summation over all positive primes $p$ less than or equal to $x$, and
\begin{align*}
\psi(x)=\sum_{p\leq x}k\log p,
\end{align*}
where the same summation notation is used and $k$ denotes the unique integer such that $p^k\leq x$ but $p^{k+1}>x$.  Heuristically, the first of these two functions \PMlinkescapetext{measures} the number of primes less than $x$ and the second does the same, but weighting each prime in accordance with their logarithmic relationship to $x$.

Many innocuous results in number \PMlinkescapetext{theory} owe their proof to a relatively \PMlinkescapetext{simple} analysis of the asymptotics of one or both of these functions.  For example, the fact that for any $n$, we have
\begin{align*}
\prod_{p\leq n}p<4^n
\end{align*}
is equivalent to the statement that $\vartheta(x)<x\log 4$.

A somewhat less innocuous result is that the prime number theorem (i.e., that $\pi(x)\sim \frac{x}{\log x}$) is equivalent to the statement that $\vartheta(x)\sim x$, which in turn, is equivalent to the statement that $\psi(x)\sim x$.

\begin{thebibliography}{9}
\bibitem{IR} Ireland, Kenneth and Rosen, Michael.  A Classical Introduction to Modern Number Theory.  Springer, 1998.

\bibitem{Na} Nathanson, Melvyn B.  Elementary Methods in Number Theory.  Springer, 2000.
\end{thebibliography}
%%%%%
%%%%%
\end{document}
