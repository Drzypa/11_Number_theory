\documentclass[12pt]{article}
\usepackage{pmmeta}
\pmcanonicalname{Sandboxmaximaworkedexample2}
\pmcreated{2013-12-27 13:29:32}
\pmmodified{2013-12-27 13:29:32}
\pmowner{robert_dodier}{1000903}
\pmmodifier{robert_dodier}{1000903}
\pmtitle{sandbox-maxima-worked-example-2}
\pmrecord{1}{}
\pmprivacy{1}
\pmauthor{robert_dodier}{1000903}
\pmtype{Example}

% this is the default PlanetMath preamble.  as your knowledge
% of TeX increases, you will probably want to edit this, but
% it should be fine as is for beginners.

% almost certainly you want these
\usepackage{amssymb}
\usepackage{amsmath}
\usepackage{amsfonts}

% need this for including graphics (\includegraphics)
\usepackage{graphicx}
% for neatly defining theorems and propositions
\usepackage{amsthm}

% making logically defined graphics
%\usepackage{xypic}
% used for TeXing text within eps files
%\usepackage{psfrag}

% there are many more packages, add them here as you need them

% define commands here

\begin{document}
Converting inches to kilometers.
First load the `ezunits' package.
\begin{verbatim}
load (ezunits) $
\end{verbatim}
\begin{verbatim}
foo : 1 ` inch;
\end{verbatim}
$$1\;\mathrm{inch}$$
Units are converted by the double backtick operator; conversions are not carried out automatically.
\begin{verbatim}
bar : foo `` km;
\end{verbatim}
$${{127}\over{5000000}}\;\mathrm{km}$$
As usual, Maxima prefers exact numbers to inexact. 
We can get a float approximation if we wish.
\begin{verbatim}
float (bar);
\end{verbatim}
$$2.54 \times 10^{-5}\;\mathrm{km}$$
A volume conversion. 
Here `baz' is the volume of one inch of water spread over one acre, while `quux' is the same volume expressed in SI units.
\begin{verbatim}
baz : 1 ` (inch * acre);
\end{verbatim}
$$1\;\mathrm{inch}\,\mathrm{acre}$$
\begin{verbatim}
quux : baz `` m^3;
\end{verbatim}
$${{20076201783}\over{195312500}}\;\mathrm{m}^3$$
\begin{verbatim}
float (%);
\end{verbatim}
$$102.79015312896\;\mathrm{m}^3$$

\end{document}
