\documentclass[12pt]{article}
\usepackage{pmmeta}
\pmcanonicalname{ConditionalCongruences}
\pmcreated{2013-03-22 18:52:23}
\pmmodified{2013-03-22 18:52:23}
\pmowner{pahio}{2872}
\pmmodifier{pahio}{2872}
\pmtitle{conditional congruences}
\pmrecord{6}{41720}
\pmprivacy{1}
\pmauthor{pahio}{2872}
\pmtype{Topic}
\pmcomment{trigger rebuild}
\pmclassification{msc}{11A07}
\pmclassification{msc}{11A05}
\pmrelated{LinearCongruence}
\pmrelated{QuadraticCongruence}
\pmdefines{degree of congruence}
\pmdefines{root of congruence}
\pmdefines{root}

% this is the default PlanetMath preamble.  as your knowledge
% of TeX increases, you will probably want to edit this, but
% it should be fine as is for beginners.

% almost certainly you want these
\usepackage{amssymb}
\usepackage{amsmath}
\usepackage{amsfonts}

% used for TeXing text within eps files
%\usepackage{psfrag}
% need this for including graphics (\includegraphics)
%\usepackage{graphicx}
% for neatly defining theorems and propositions
 \usepackage{amsthm}
% making logically defined graphics
%%%\usepackage{xypic}

% there are many more packages, add them here as you need them

% define commands here

\theoremstyle{definition}
\newtheorem*{thmplain}{Theorem}

\begin{document}
\PMlinkescapeword{solution}

Consider \PMlinkname{congruences}{Congruences} of the form
\begin{align}
f(x) \;:=\; a_nx^n+a_{n-1}x^{n-1}+\ldots+a_0 \;\equiv\; 0 \pmod{m}
\end{align}
where the coefficients $a_i$ and $m$ are rational integers.\, \emph{Solving} the congruence means finding all the integer values of $x$ which satisfy (1).

\begin{itemize}

\item If\, $a_i \equiv 0 \pmod{m}$\, for all $i$'s, the congruence is satisfied by each integer, in which case the congruence is identical (cf. the formal congruence).\, Therefore one can assume that at least
$$a_n \not\equiv 0 \pmod{m},$$
since one would otherwise have\, $a_nx^n \equiv 0 \pmod{m}$\, and the first term could be left out of (1).\, Now, we say that the \emph{degree of the congruence} (1) is $n$.

\item If\, $x = x_0$\, is a solution of (1) and\, $x_1 \equiv x_0 \pmod{m}$,\, then by the properties of \PMlinkname{congruences}{Congruences}, 
$$f(x_1) \;\equiv\; f(x_0) \;\equiv\; 0 \pmod{m},$$
and thus also\, $x = x_1$\, is a solution.\, Therefore, one regards as different \emph{roots} of a congruence modulo 
$m$ only such values of $x$ which are incongruent modulo $m$.

\item One can think that the congruence (1) has as many roots as is found in a complete residue system modulo $m$.

\end{itemize}


%%%%%
%%%%%
\end{document}
