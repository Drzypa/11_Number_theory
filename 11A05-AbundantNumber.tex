\documentclass[12pt]{article}
\usepackage{pmmeta}
\pmcanonicalname{AbundantNumber}
\pmcreated{2013-03-22 15:52:21}
\pmmodified{2013-03-22 15:52:21}
\pmowner{CompositeFan}{12809}
\pmmodifier{CompositeFan}{12809}
\pmtitle{abundant number}
\pmrecord{6}{37869}
\pmprivacy{1}
\pmauthor{CompositeFan}{12809}
\pmtype{Definition}
\pmcomment{trigger rebuild}
\pmclassification{msc}{11A05}
\pmrelated{AmicableNumbers}

\endmetadata

% this is the default PlanetMath preamble.  as your knowledge
% of TeX increases, you will probably want to edit this, but
% it should be fine as is for beginners.

% almost certainly you want these
\usepackage{amssymb}
\usepackage{amsmath}
\usepackage{amsfonts}

% used for TeXing text within eps files
%\usepackage{psfrag}
% need this for including graphics (\includegraphics)
%\usepackage{graphicx}
% for neatly defining theorems and propositions
%\usepackage{amsthm}
% making logically defined graphics
%%%\usepackage{xypic}

% there are many more packages, add them here as you need them

% define commands here
\begin{document}
An integer $n$ is an {\em abundant number} if the sum of the proper divisors of $n$ is more than $n$ itself, or the sum of all the divisors is more than twice $n$. That is, $\sigma(n) > 2n$, with $\sigma(n)$ being the sum of divisors function.

For example, the integer 30. Its proper divisors are 1, 2, 3, 5, 6, 10, 15, which add up to 42.

Multiplying a perfect number by some integer $x$ gives an abundant number (as long as $x > 1$).

Given a pair of amicable numbers, the lesser of the two is abundant, its proper divisors adding up to the greater of the two.
%%%%%
%%%%%
\end{document}
