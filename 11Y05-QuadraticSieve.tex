\documentclass[12pt]{article}
\usepackage{pmmeta}
\pmcanonicalname{QuadraticSieve}
\pmcreated{2013-03-22 12:48:14}
\pmmodified{2013-03-22 12:48:14}
\pmowner{patrickwonders}{217}
\pmmodifier{patrickwonders}{217}
\pmtitle{quadratic sieve}
\pmrecord{9}{33121}
\pmprivacy{1}
\pmauthor{patrickwonders}{217}
\pmtype{Algorithm}
\pmcomment{trigger rebuild}
\pmclassification{msc}{11Y05}
\pmclassification{msc}{11A51}

% this is the default PlanetMath preamble.  as your knowledge
% of TeX increases, you will probably want to edit this, but
% it should be fine as is for beginners.

% almost certainly you want these
\usepackage{amssymb}
\usepackage{amsmath}
\usepackage{amsfonts}

% used for TeXing text within eps files
%\usepackage{psfrag}
% need this for including graphics (\includegraphics)
%\usepackage{graphicx}
% for neatly defining theorems and propositions
%\usepackage{amsthm}
% making logically defined graphics
%%%\usepackage{xypic}

% there are many more packages, add them here as you need them

% define commands here
\newcommand{\B}{\ensuremath\mathbf{B}}
\newcommand{\Z}{\ensuremath\mathbb{Z}}
\begin{document}
\paragraph{Algorithm}

    To factor a number $n$ using the quadratic sieve, one seeks
    two numbers $x$ and $y$ which are not congruent modulo $n$
    with $x$ not congruent to $-y$ modulo $n$ but
    have $x^2 \equiv y^2 \pmod n$.  If two such numbers are found,
    one can then say that $(x+y)(x-y) \equiv 0 \pmod n$.  Then,
    $x+y$ and $x-y$ must have non-trivial factors in common with $n$.

    The quadratic sieve method of factoring depends upon being able
    to create a set of numbers whose factorization can be expressed
    as a product of pre-chosen primes.  These factorizations are
    recorded as vectors of the exponents.  Once enough vectors are
    collected to form a set which contains a linear dependence,
    this linear dependence is exploited to find two squares which
    are equivalent modulo $n$.

    To accomplish this, the quadratic sieve method uses a set of
    prime numbers called a factor base.  Then, it searches for
    numbers which can be factored entirely within that factor base.
    If there are $k$ prime numbers in the factor base, then each
    number which can be factored within the factor base is stored
    as a $k$-dimensional vector where the $i$-th component of the
    vector for $y$ gives the exponent of the $i$-th prime from the
    factor base in the factorization of $y$.  For example, if the
    factor base were $\left\{ 2, 3, 5, 7, 11, 13 \right\}$, then
    the number $y = 2^3\cdot3^2\cdot11^5$ would be stored as the
    vector $\left< 3,2,0,0,5,0 \right>$.

    Once $k+1$ of these vectors have been collected, there must be
    a linear dependence among them.  The $k+1$ vectors are taken
    modulo $2$ to form vectors in $\Z_2^k$.  The linear dependence
    among them is used to find a combination of the vectors which
    sum up to the zero vector in $\Z_2^k$.  Summing these vectors
    is equivalent to multiplying the $y$'s to which they correspond.
    And, the zero vector in $\Z_2^k$ signals a perfect square.

    To factor $n$, chose a factor base $\B = \left\{ p_1, p_2,
    \ldots, p_k \right\}$ such that $2 \in \B$ and for each odd
    prime $p_j$ in $\B$, $n$ is a quadratic residue of $p_j$.  Now,
    start picking $x_i$ near $\sqrt{n}$ and calculate $y_i = x_i^2
    - n$.  Clearly $y_i \equiv x_i^2 \pmod n$.  If $y_i$ can be
    completely factored by numbers in $\B$, then it is called
    $\B$-smooth.  If it is not $\B$-smooth, then discard $x_i$ and
    $y_i$ and move on to a new choice of $x_i$.  If it is $\B$-smooth,
    then store $x_i$, $y_i$, and the vector of its exponents for
    the primes in $\B$.  Also, record a copy of the exponent vector
    with each component taken modulo $2$.

    Once $k+1$ vectors have been recorded, there must be a linear
    dependence among them.  Using the copies of the exponent vectors
    that were taken modulo $2$, determine which ones can be added
    together to form the zero vector.  Multiply together the $x_i$
    that correspond to those chosen vectors---call this $x$.  Also,
    add together the original vectors that correspond to the chosen
    vectors to form a new vector $\vec{v}$.  Every component of
    this vector will be even.  Divide each element of $\vec{v}$ by
    $2$.  Form $y = \prod_{i=1}^{k} p_i^{v_i}$.

    Because each $y_i \equiv x_i^2 \pmod n$, $x^2 \equiv y^2 \pmod
    n$.  If $x \equiv y \pmod n$, then find some more $\B$-smooth
    numbers and try again.  If $x$ is not congruent to $y$ modulo
    $n$, then $(x+y)$ and $(x-y)$ are factors of $n$.

\paragraph{Example}

	Consider the number $n = 16843009$  The integer nearest its
	square root is $4104$.  Given the factor base $$\B = \left\{
	2, 3, 5, 7, 13 \right\}$$, the first few $\B$-smooth values
	of $y_i = f( x_i ) = x_i^2 - n$ are:

	\begin{center}
\begin{tabular}{c|c||c|c|c|c|c}  
    $x_i$ &  $y_i = f(x_i)$ &  2 &   3 &   5 &   7 &  13 \\     
\hline
  4122 &   147875 &       0 &   0 &   3 &   1 &   2 \\ 
  4159 &   454272 &       7 &   1 &   0 &   1 &   2 \\ 
  4187 &   687960 &       3 &   3 &   1 &   2 &   1 \\ 
  4241 &  1143072 &       5 &   6 &   0 &   2 &   0 \\ 
  4497 &  3380000 &       5 &   0 &   4 &   0 &   2 \\ 
  4993 &  8087040 &       9 &   5 &   1 &   0 &   1
\end{tabular}
\end{center}

	Using $x_0 = 4241$ and $x_1 = 4497$, one obtains:
$$ y_0 = 1143072 = 2^5 \cdot 3^6 \cdot 5^0 \cdot 7^2 \cdot 13^0 $$
$$ y_1 = 3380000 = 2^5 \cdot 3^0 \cdot 5^4 \cdot 7^0 \cdot 13^2 $$
	Which results in:
$$ x = 4241 \cdot 4497 = 19071777 $$
$$ y = 2^5 \cdot 3^3 \cdot 5^2 \cdot 7^1 \cdot 13^1 = 1965600 $$

	From there:
$$ \gcd( x-y, n ) = 257 $$
$$ \gcd( x+y, n ) = 65537 $$

	It may not be completely obvious why we required that
	$n$ be a quadratic residue of each $p_i$ in the
	factor base $\B$.  One might
	intuitively think that we actually want the $p_i$ to
	be quadratic residues of $n$ instead.  But, that is
	not the case.

	We are trying to express $n$ as:
$$ ( x + y ) ( x - y ) = x^2 - y^2 = n $$
	where
$$ y = \prod_{i=1}^{k} p_i^{v_i} $$
	Because we end up squaring $y$, there is no reason
	that the $p_i$ would need to be quadratic residues
	of $n$.

	So, why do we require that $n$ be a quadratic residue
	of each $p_i$?  We can rewrite the $x^2 - y^2 = n$ as:
$$ x^2 - \prod_{i=1}^{k} p_i^{2v_i} = n$$
	If we take that expression modulo $p_i$ for any $p_i$
	for which the corresponding $v_i$ is non-zero, we are
	left with:
$$ x^2 \equiv n \pmod{p_i} $$
	Thus, in order for $p_i$ to show up in a useful
	solution, $n$ must be a quadratic residue of $p_i$.
	We would be wasting time and space to employ other
	primes in our factoring and linear combinations.
%%%%%
%%%%%
\end{document}
