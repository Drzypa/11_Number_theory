\documentclass[12pt]{article}
\usepackage{pmmeta}
\pmcanonicalname{Palindrome}
\pmcreated{2013-03-22 12:57:05}
\pmmodified{2013-03-22 12:57:05}
\pmowner{yark}{2760}
\pmmodifier{yark}{2760}
\pmtitle{palindrome}
\pmrecord{5}{33311}
\pmprivacy{1}
\pmauthor{yark}{2760}
\pmtype{Definition}
\pmcomment{trigger rebuild}
\pmclassification{msc}{11B83}
\pmrelated{Reversal}

\endmetadata

\usepackage{amssymb}
\usepackage{amsmath}
\usepackage{amsfonts}

%\usepackage{psfrag}
%\usepackage{graphicx}
%%%\usepackage{xypic}
\begin{document}
A \emph{palindrome} is a number which yields itself when its digits are reversed.  Some palindromes (in base 10) are :

\begin{itemize}
\item 121
\item 2002
\item 314159951413
\end{itemize}

Clearly one can construct palindromes of arbitrary length by taking any number and appending to it a reversed copy of itself or of all but the last digit.

The concept of palindromes can also be extended to sequences and strings.
%%%%%
%%%%%
\end{document}
