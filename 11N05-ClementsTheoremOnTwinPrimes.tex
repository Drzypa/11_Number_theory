\documentclass[12pt]{article}
\usepackage{pmmeta}
\pmcanonicalname{ClementsTheoremOnTwinPrimes}
\pmcreated{2013-03-22 17:58:32}
\pmmodified{2013-03-22 17:58:32}
\pmowner{PrimeFan}{13766}
\pmmodifier{PrimeFan}{13766}
\pmtitle{Clement's theorem on twin primes}
\pmrecord{4}{40484}
\pmprivacy{1}
\pmauthor{PrimeFan}{13766}
\pmtype{Theorem}
\pmcomment{trigger rebuild}
\pmclassification{msc}{11N05}

\endmetadata

% this is the default PlanetMath preamble.  as your knowledge
% of TeX increases, you will probably want to edit this, but
% it should be fine as is for beginners.

% almost certainly you want these
\usepackage{amssymb}
\usepackage{amsmath}
\usepackage{amsfonts}

% used for TeXing text within eps files
%\usepackage{psfrag}
% need this for including graphics (\includegraphics)
%\usepackage{graphicx}
% for neatly defining theorems and propositions
%\usepackage{amsthm}
% making logically defined graphics
%%%\usepackage{xypic}

% there are many more packages, add them here as you need them

% define commands here

\begin{document}
{\bf Theorem}. (P. Clement) Given a prime number $p$, $p + 2$ is also a prime (and $p$ and $p + 2$ form a twin prime) if and only if $4(p - 1)! \equiv -4 - p \pmod{p^2 + 2p}$.

Richard Crandall and Carl Pomerance see this theorem as ``a way to connect the notion of twin-prime pairs with the Wilson-Lagrange theorem.''

\begin{thebibliography}{1}
\bibitem{rccp} Richard Crandall \& Carl Pomerance, {\it Prime Numbers: A Computational Perspective}, 2nd Edition. New York: Springer (2005): 65, Exercise 1.57
\end{thebibliography}
%%%%%
%%%%%
\end{document}
