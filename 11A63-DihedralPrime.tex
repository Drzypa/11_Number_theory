\documentclass[12pt]{article}
\usepackage{pmmeta}
\pmcanonicalname{DihedralPrime}
\pmcreated{2013-03-22 16:46:53}
\pmmodified{2013-03-22 16:46:53}
\pmowner{PrimeFan}{13766}
\pmmodifier{PrimeFan}{13766}
\pmtitle{dihedral prime}
\pmrecord{4}{39012}
\pmprivacy{1}
\pmauthor{PrimeFan}{13766}
\pmtype{Definition}
\pmcomment{trigger rebuild}
\pmclassification{msc}{11A63}
\pmsynonym{dihedral calculator prime}{DihedralPrime}

\endmetadata

% this is the default PlanetMath preamble.  as your knowledge
% of TeX increases, you will probably want to edit this, but
% it should be fine as is for beginners.

% almost certainly you want these
\usepackage{amssymb}
\usepackage{amsmath}
\usepackage{amsfonts}

% used for TeXing text within eps files
%\usepackage{psfrag}
% need this for including graphics (\includegraphics)
%\usepackage{graphicx}
% for neatly defining theorems and propositions
%\usepackage{amsthm}
% making logically defined graphics
%%%\usepackage{xypic}

% there are many more packages, add them here as you need them

% define commands here

\begin{document}
A {\em dihedral prime} or {\em dihedral calculator prime} is a prime number that still reads like itself or another prime number when read in a seven-segment display, regardless of orientation (normally or upside down), and surface (actual display or reflection on a mirror). The first few base 10 dihedral primes are 2, 11, 101, 181, 1181, 1811, 18181, 108881, 110881, 118081, 120121, 121021, 121151, 150151, 151051, 151121, 180181, 180811, 181081, etc. These are listed in A038136 of Sloane's OEIS.

The smallest dihedral prime that reads differently with each orientation and surface combination is 120121.

The digits 0, 1 and 8 remain the same regardless of orientation or surface. 2 and 5 turn into each other when viewed upside down or reflected on a mirror. In the display of a scientific calculator that can handle hexadecimal, 3 would become E reflected, but E being an even digit, the 3 can't be used as the first digit because the reflected number will be even.
% TO DO: find or disprove hex numbers dihedral primes with 3 and E
Though 6 and 9 become each other upside down, they are not valid digits when reflected, at least not in any of the numeral systems pocket calculators usually operate in. In binary, all palindromic primes are dihedral.

Strobogrammatic primes that don't use 6 or 9 are dihedral primes.
%%%%%
%%%%%
\end{document}
