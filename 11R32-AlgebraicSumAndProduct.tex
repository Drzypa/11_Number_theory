\documentclass[12pt]{article}
\usepackage{pmmeta}
\pmcanonicalname{AlgebraicSumAndProduct}
\pmcreated{2013-03-22 15:28:03}
\pmmodified{2013-03-22 15:28:03}
\pmowner{pahio}{2872}
\pmmodifier{pahio}{2872}
\pmtitle{algebraic sum and product}
\pmrecord{8}{37320}
\pmprivacy{1}
\pmauthor{pahio}{2872}
\pmtype{Theorem}
\pmcomment{trigger rebuild}
\pmclassification{msc}{11R32}
\pmclassification{msc}{11R04}
\pmclassification{msc}{13B05}
\pmsynonym{sum and product algebraic}{AlgebraicSumAndProduct}
\pmrelated{FiniteExtension}
\pmrelated{TheoryOfAlgebraicNumbers}
\pmrelated{FieldOfAlgebraicNumbers}

\endmetadata

% this is the default PlanetMath preamble.  as your knowledge
% of TeX increases, you will probably want to edit this, but
% it should be fine as is for beginners.

% almost certainly you want these
\usepackage{amssymb}
\usepackage{amsmath}
\usepackage{amsfonts}

% used for TeXing text within eps files
%\usepackage{psfrag}
% need this for including graphics (\includegraphics)
%\usepackage{graphicx}
% for neatly defining theorems and propositions
 \usepackage{amsthm}
% making logically defined graphics
%%%\usepackage{xypic}

% there are many more packages, add them here as you need them

% define commands here

\theoremstyle{definition}
\newtheorem*{thmplain}{Theorem}
\begin{document}
Let $\alpha,\,\beta$ be two elements of an extension field of a given field $K$.\,  Both these elements are algebraic over $K$ if and only if both $\alpha\!+\!\beta$ and $\alpha\beta$ are algebraic over $K$.

{\em Proof.}\, Assume first that $\alpha$ and $\beta$ are algebraic.\, Because
$$[K(\alpha,\,\beta):K] = [K(\alpha,\,\beta):K(\alpha)]\,[K(\alpha):K]$$
and both \PMlinkescapetext{factors} here are \PMlinkname{finite}{ExtendedRealNumbers}, then $[K(\alpha,\,\beta):K]$ is finite.\, So we have a finite field extension $K(\alpha,\,\beta)/K$ which thus is also algebraic, and therefore the elements $\alpha\!+\!\beta$ and $\alpha\beta$ of $K(\alpha,\,\beta)$ are algebraic over $K$.\, Secondly suppose that $\alpha\!+\!\beta$ and $\alpha\beta$ are algebraic over $K$.\, The elements $\alpha$ and $\beta$ are the roots of the quadratic equation\, $x^2-(\alpha\!+\!\beta)x+\alpha\beta = 0$\, (cf. properties of quadratic equation) with the coefficients in $K(\alpha\!+\!\beta,\,\alpha\beta)$.\, Thus
$$[K(\alpha,\,\beta):K] = 
[K(\alpha,\,\beta):K(\alpha\!+\!\beta,\,\alpha\beta)]\,
[K(\alpha\!+\!\beta,\,\alpha\beta):K] \leqq 2 
[K(\alpha\!+\!\beta,\,\alpha\beta):K].$$
Since\, $[K(\alpha\!+\!\beta,\,\alpha\beta):K]$\, is finite,\, then also\, $[K(\alpha,\,\beta):K]$ is, and in the \PMlinkname{finite extension}{FiniteExtension}\, $K(\alpha,\,\beta)/K$\, the elements $\alpha$ and $\beta$ must be algebraic over $K$.
%%%%%
%%%%%
\end{document}
