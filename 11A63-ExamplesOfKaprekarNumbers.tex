\documentclass[12pt]{article}
\usepackage{pmmeta}
\pmcanonicalname{ExamplesOfKaprekarNumbers}
\pmcreated{2013-03-22 16:02:00}
\pmmodified{2013-03-22 16:02:00}
\pmowner{PrimeFan}{13766}
\pmmodifier{PrimeFan}{13766}
\pmtitle{examples of Kaprekar numbers}
\pmrecord{5}{38077}
\pmprivacy{1}
\pmauthor{PrimeFan}{13766}
\pmtype{Example}
\pmcomment{trigger rebuild}
\pmclassification{msc}{11A63}

% this is the default PlanetMath preamble.  as your knowledge
% of TeX increases, you will probably want to edit this, but
% it should be fine as is for beginners.

% almost certainly you want these
\usepackage{amssymb}
\usepackage{amsmath}
\usepackage{amsfonts}

% used for TeXing text within eps files
%\usepackage{psfrag}
% need this for including graphics (\includegraphics)
%\usepackage{graphicx}
% for neatly defining theorems and propositions
%\usepackage{amsthm}
% making logically defined graphics
%%%\usepackage{xypic}

% there are many more packages, add them here as you need them

% define commands here

\begin{document}
Take the integer 142857 in base 10. Its square is 20408122449, an 11-digit number, but for the sake of example let's think of it as 020408122449. If we split it into two 6-digit numbers, 020408 and 122449, and add them up, we get 142857 back. Thus 142857 is a Kaprekar number in base 10.

The Mersenne numbers have plenty of base-dependent properties in binary, and one of them is that they are all Kaprekar numbers. For example, 127, which is 1111111 in binary. Its square is 11111100000001, and sure enough, 1111110 + 0000001 = 1111111.

It has been mentioned that $b^x - 1$ is a Kaprekar number in base $b$. Take 999 for example. Its square is 998001, and it's obvious that 998 + 001 = 999. From these observations we can generalize that $(b^x - 1)^2 = (b^x(b^x - 1) - 1) + 1$.
%%%%%
%%%%%
\end{document}
