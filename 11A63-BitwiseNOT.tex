\documentclass[12pt]{article}
\usepackage{pmmeta}
\pmcanonicalname{BitwiseNOT}
\pmcreated{2013-03-22 17:02:55}
\pmmodified{2013-03-22 17:02:55}
\pmowner{PrimeFan}{13766}
\pmmodifier{PrimeFan}{13766}
\pmtitle{bitwise NOT}
\pmrecord{4}{39338}
\pmprivacy{1}
\pmauthor{PrimeFan}{13766}
\pmtype{Definition}
\pmcomment{trigger rebuild}
\pmclassification{msc}{11A63}
\pmrelated{BitwiseAND}
\pmrelated{BitwiseOR}
\pmrelated{BitwiseXOR}

\endmetadata

% this is the default PlanetMath preamble.  as your knowledge
% of TeX increases, you will probably want to edit this, but
% it should be fine as is for beginners.

% almost certainly you want these
\usepackage{amssymb}
\usepackage{amsmath}
\usepackage{amsfonts}

% used for TeXing text within eps files
%\usepackage{psfrag}
% need this for including graphics (\includegraphics)
%\usepackage{graphicx}
% for neatly defining theorems and propositions
%\usepackage{amsthm}
% making logically defined graphics
%%%\usepackage{xypic}

% there are many more packages, add them here as you need them

% define commands here

\begin{document}
{\em Bitwise NOT} is a bit-level operation on a single binary value that sets the bits that are off and turns off the bits that are on. For example, performing a bitwise NOT on 163 gives 92.

\begin{tabular}{|r|c|c|c|c|c|c|c|c|}
NOT & 1 & 0 & 1 & 0 & 0 & 0 & 1 & 1 \\
  = & 0 & 1 & 0 & 1 & 1 & 1 & 0 & 0 \\
\end{tabular}

Performing a bitwise NOT on a number $n$ has the same effect as performing a bitwise XOR on a Mersenne number of the form $2^k - 1$ (where $k$ is the bit size of the data type in use, e.g., 8 for bytes, 16 for words, 32 for double words, etc.) and $n$. Obviously a bitwise NOT on 0 gives the largest Mersenne number that can fit in the data type in use.

The Windows Calculator offers bitwise NOT in scientific calculator mode, while the Mac OS X Calculator offers it in programmer mode.
%%%%%
%%%%%
\end{document}
