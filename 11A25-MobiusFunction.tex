\documentclass[12pt]{article}
\usepackage{pmmeta}
\pmcanonicalname{MobiusFunction}
\pmcreated{2013-03-22 11:47:03}
\pmmodified{2013-03-22 11:47:03}
\pmowner{mps}{409}
\pmmodifier{mps}{409}
\pmtitle{M\"obius function}
\pmrecord{11}{30253}
\pmprivacy{1}
\pmauthor{mps}{409}
\pmtype{Definition}
\pmcomment{trigger rebuild}
\pmclassification{msc}{11A25}
\pmclassification{msc}{55-00}
\pmclassification{msc}{55-01}
\pmsynonym{Moebius function}{MobiusFunction}
%\pmkeywords{number theory}
\pmrelated{SquareFreeNumber}
\pmrelated{SumOfFracmunn}
\pmrelated{MoebiusInversionFormula}
\pmrelated{ConvolutionMethod}

% this is the default PlanetMath preamble.  as your knowledge
% of TeX increases, you will probably want to edit this, but
% it should be fine as is for beginners.

% almost certainly you want these
\usepackage{amssymb}
\usepackage{amsmath}
\usepackage{amsfonts}

% used for TeXing text within eps files
%\usepackage{psfrag}
% need this for including graphics (\includegraphics)
%\usepackage{graphicx}
% for neatly defining theorems and propositions
%\usepackage{amsthm}
% making logically defined graphics
%%%%%\usepackage{xypic}

% there are many more packages, add them here as you need them

% define commands here
\begin{document}
The {\em M\"obius function} of number theory is the function $\mu:\mathbb{Z}^+\to\{-1,0,1\}$ defined by
\[
\mu (n) = 
\begin{cases}
1, &\text{if $n=1$}\\
0, &\text{if $p^2 | n$ for some prime $p$} \\
(-1)^r, &\text{if $n = p_1 p_2 \cdots p_r$, where the $p_i$ are distinct primes.}
\end{cases}
\]

In other words, $\mu (n) = 0$ if $n$ is not a square-free integer, while $\mu (n) = (-1)^r$ if $n$ is square-free with $r$ prime factors. The function $\mu$ is a multiplicative function, and obeys the identity
\[
\sum_{d | n} \mu(d) = 
\begin{cases}
1 & \text{if $n = 1$}\\
0 & \text{if $n > 1$}
\end{cases}
\]
where $d$ runs through the positive divisors of $n$.
%%%%%
%%%%%
%%%%%
%%%%%
\end{document}
