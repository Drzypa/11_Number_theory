\documentclass[12pt]{article}
\usepackage{pmmeta}
\pmcanonicalname{PierpontPrime}
\pmcreated{2013-03-22 16:52:39}
\pmmodified{2013-03-22 16:52:39}
\pmowner{PrimeFan}{13766}
\pmmodifier{PrimeFan}{13766}
\pmtitle{Pierpont prime}
\pmrecord{4}{39128}
\pmprivacy{1}
\pmauthor{PrimeFan}{13766}
\pmtype{Definition}
\pmcomment{trigger rebuild}
\pmclassification{msc}{11A41}

\endmetadata

% this is the default PlanetMath preamble.  as your knowledge
% of TeX increases, you will probably want to edit this, but
% it should be fine as is for beginners.

% almost certainly you want these
\usepackage{amssymb}
\usepackage{amsmath}
\usepackage{amsfonts}

% used for TeXing text within eps files
%\usepackage{psfrag}
% need this for including graphics (\includegraphics)
%\usepackage{graphicx}
% for neatly defining theorems and propositions
%\usepackage{amsthm}
% making logically defined graphics
%%%\usepackage{xypic}

% there are many more packages, add them here as you need them

% define commands here

\begin{document}
A {\em Pierpont prime} is a prime number of the form $p = 1 + 2^x3^y$ with $-1 < y \le x$. If $x > 0$ and $y = 0$ then the resulting prime is a Fermat prime. In the \PMlinkname{Erd\H{o}s-Selfridge classification of primes}{ErdHosSelfridgeClassificationOfPrimes}, the Pierpont primes are class 1-. The first few Pierpont primes are 2, 3, 5, 7, 13, 17, 19, 37, 73, 97, 109, 163, 193, 257, 433, 487, 577, 769, etc., listed in A005109 of Sloane's OEIS.

In 1988, Gleason showed that an $n$-sided regular polygon can be constructed with ruler and compass if $n$ is the product of two Pierpont primes.
%%%%%
%%%%%
\end{document}
