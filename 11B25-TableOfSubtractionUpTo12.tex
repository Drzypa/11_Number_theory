\documentclass[12pt]{article}
\usepackage{pmmeta}
\pmcanonicalname{TableOfSubtractionUpTo12}
\pmcreated{2013-03-22 16:36:11}
\pmmodified{2013-03-22 16:36:11}
\pmowner{PrimeFan}{13766}
\pmmodifier{PrimeFan}{13766}
\pmtitle{table of subtraction up to 12}
\pmrecord{6}{38798}
\pmprivacy{1}
\pmauthor{PrimeFan}{13766}
\pmtype{Data Structure}
\pmcomment{trigger rebuild}
\pmclassification{msc}{11B25}
\pmclassification{msc}{00A05}
\pmclassification{msc}{00A06}

\endmetadata

% this is the default PlanetMath preamble.  as your knowledge
% of TeX increases, you will probably want to edit this, but
% it should be fine as is for beginners.

% almost certainly you want these
\usepackage{amssymb}
\usepackage{amsmath}
\usepackage{amsfonts}

% used for TeXing text within eps files
%\usepackage{psfrag}
% need this for including graphics (\includegraphics)
%\usepackage{graphicx}
% for neatly defining theorems and propositions
%\usepackage{amsthm}
% making logically defined graphics
%%%\usepackage{xypic}

% there are many more packages, add them here as you need them

% define commands here

\begin{document}
In this table of subtraction, the column operand is first and the row operand is second.

\begin{tabular}{|c|l|l|l|l|l|l|l|l|l|l|l|l|}
$-$ & 1 & 2 & 3 & 4 & 5 & 6 & 7 & 8 & 9 & 10 & 11 & 12 \\
1 & 0 & $-1$ & $-2$ & $-3$ & $-4$ & $-5$ & $-6$ & $-7$ & $-8$ & $-9$ & $-10$ & $-11$ \\
2 & 1 & 0 & $-1$ & $-2$ & $-3$ & $-4$ & $-5$ & $-6$ & $-7$ & $-8$ & $-9$ & $-10$ \\
3 & 2 & 1 & 0 & $-1$ & $-2$ & $-3$ & $-4$ & $-5$ & $-6$ & $-7$ & $-8$ & $-9$ \\
4 & 3 & 2 & 1 & 0 & $-1$ & $-2$ & $-3$ & $-4$ & $-5$ & $-6$ & $-7$ & $-8$ \\
5 & 4 & 3 & 2 & 1 & 0 & $-1$ & $-2$ & $-3$ & $-4$ & $-5$ & $-6$ & $-7$ \\
6 & 5 & 4 & 3 & 2 & 1 & 0 & $-1$ & $-2$ & $-3$ & $-4$ & $-5$ & $-6$ \\
7 & 6 & 5 & 4 & 3 & 2 & 1 & 0 & $-1$ & $-2$ & $-3$ & $-4$ & $-5$ \\
8 & 7 & 6 & 5 & 4 & 3 & 2 & 1 & 0 & $-1$ & $-2$ & $-3$ & $-4$ \\
9 & 8 & 7 & 6 & 5 & 4 & 3 & 2 & 1 & 0 & $-1$ & $-2$ & $-3$ \\
10 & 9 & 8 & 7 & 6 & 5 & 4 & 3 & 2 & 1 & 0 & $-1$ & $-2$ \\
11 & 10 & 9 & 8 & 7 & 6 & 5 & 4 & 3 & 2 & 1 & 0 & $-1$ \\
12 & 11 & 10 & 9 & 8 & 7 & 6 & 5 & 4 & 3 & 2 & 1 & 0\\
\end{tabular}


The longest northwest to southeast diagonal obviously contains zeroes.
%%%%%
%%%%%
\end{document}
