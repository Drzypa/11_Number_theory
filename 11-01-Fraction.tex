\documentclass[12pt]{article}
\usepackage{pmmeta}
\pmcanonicalname{Fraction}
\pmcreated{2013-03-22 12:34:11}
\pmmodified{2013-03-22 12:34:11}
\pmowner{bwebste}{988}
\pmmodifier{bwebste}{988}
\pmtitle{fraction}
\pmrecord{11}{32818}
\pmprivacy{1}
\pmauthor{bwebste}{988}
\pmtype{Definition}
\pmcomment{trigger rebuild}
\pmclassification{msc}{11-01}
\pmrelated{RationalNumber}
\pmrelated{Number}
\pmrelated{CategoryOfAdditiveFractions}
\pmdefines{solidus}
\pmdefines{proper fraction}
\pmdefines{numerator}
\pmdefines{denominator}
\pmdefines{improper fraction}
\pmdefines{lowest terms}

\endmetadata

% this is the default PlanetMath preamble.  as your knowledge
% of TeX increases, you will probably want to edit this, but
% it should be fine as is for beginners.

% almost certainly you want these
\usepackage{amssymb}
\usepackage{amsmath}
\usepackage{amsfonts}

% used for TeXing text within eps files
%\usepackage{psfrag}
% need this for including graphics (\includegraphics)
%\usepackage{graphicx}
% for neatly defining theorems and propositions
%\usepackage{amsthm}
% making logically defined graphics
%%%\usepackage{xypic} 

% there are many more packages, add them here as you need them

% define commands here
\begin{document}
A \emph{fraction} is a rational number expressed in the form $\frac{n}{d}$ or $n/d$, where $n$ is designated the \emph{numerator} and $d$ the \emph{denominator}. The slash between them is known as a \emph{solidus} when the fraction is expressed as $n/d$.

The fraction $n/d$ has value $n \div d$. For instance, $3/2 = 3 \div 2 = 1.5$. 

If $n$ and $d$ are positive, and $n/d < 1$, then $n/d$ is known as a \emph{proper fraction}. Otherwise, it is an \emph{improper fraction}. If $n$ and $d$ are relatively prime, then $n/d$ is said to be in \emph{lowest terms}. Each rational number can be expressed uniquely as a fraction in lowest terms.  To get a fraction in lowest terms, simply divide the numerator and the denominator by their greatest common divisor:
$$\frac{60}{84} = \frac{60 \div 12}{84 \div 12} = \frac{5}{7}.$$

The rules for manipulating fractions are
\begin{eqnarray*}
   \frac{a}{b} & \qquad = \qquad & \frac{ka}{kb}\\
   \frac{a}{b} + \frac{c}{d} & \qquad = & \frac{ad + bc}{bd}\\
   \frac{a}{b} - \frac{c}{d} & \qquad = & \frac{ad - bc}{bd}\\
   \frac{a}{b} \times \frac{c}{d} & \qquad = & \frac{ac}{bd}\\
   \frac{a}{b} \div \frac{c}{d} & \qquad = & \frac{ad}{bc}.
\end{eqnarray*}
%%%%%
%%%%%
\end{document}
