\documentclass[12pt]{article}
\usepackage{pmmeta}
\pmcanonicalname{GoormaghtighConjecture}
\pmcreated{2013-03-22 15:18:55}
\pmmodified{2013-03-22 15:18:55}
\pmowner{mathcam}{2727}
\pmmodifier{mathcam}{2727}
\pmtitle{Goormaghtigh conjecture}
\pmrecord{8}{37120}
\pmprivacy{1}
\pmauthor{mathcam}{2727}
\pmtype{Conjecture}
\pmcomment{trigger rebuild}
\pmclassification{msc}{11J86}
\pmclassification{msc}{11J61}

% this is the default PlanetMath preamble.  as your knowledge
% of TeX increases, you will probably want to edit this, but
% it should be fine as is for beginners.

% almost certainly you want these
\usepackage{amssymb}
\usepackage{amsmath}
\usepackage{amsfonts}

% used for TeXing text within eps files
%\usepackage{psfrag}
% need this for including graphics (\includegraphics)
%\usepackage{graphicx}
% for neatly defining theorems and propositions
%\usepackage{amsthm}
% making logically defined graphics
%%%\usepackage{xypic}

% there are many more packages, add them here as you need them

% define commands here
\begin{document}
For integers $x,y,m $ and $n$ satisfying $x>1$, $y>1$, and $n>m>2$, the equation
$$\frac{x^m - 1}{x-1}=\frac{y^n - 1}{y - 1}$$
has only the solutions $(x,y,m,n)=(5,2,3,5)$ and $(90,2,3,13)$.




See the following paper   for some progress' on the conjecture:

M. Le, Exceptional solutions of the exponential Diophantine equation 
$(x^3-1)/(x-1)=(y^n-1)/(y-1)$. J. Reine Angew. Math. {\bf 543} (2002) 187-192.

See Section 7 of following paper in which the solution of a certain case of the
conjecture (given in the
latter qouted paper) is used to solve a
problem in group theory concerning  the non-cyclic graph of a group. Also see
Proposition 7.6 of 
the folloiwng paper for an slightly special case of  the conjecture which its
solotion has some 
applications in group theory.

Alireza Abdollahi and A. Mohammadi Hassanabadi, Non-cyclic graph of a group, 
Communications in Algebra, {\bf 35} (2007) 2057-2081.
%%%%%
%%%%%
\end{document}
