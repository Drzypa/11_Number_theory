\documentclass[12pt]{article}
\usepackage{pmmeta}
\pmcanonicalname{AFormulaForAmicablePairs}
\pmcreated{2013-03-22 15:52:45}
\pmmodified{2013-03-22 15:52:45}
\pmowner{alozano}{2414}
\pmmodifier{alozano}{2414}
\pmtitle{a formula for amicable pairs}
\pmrecord{6}{37876}
\pmprivacy{1}
\pmauthor{alozano}{2414}
\pmtype{Definition}
\pmcomment{trigger rebuild}
\pmclassification{msc}{11A05}
\pmrelated{ThabitNumber}

% this is the default PlanetMath preamble.  as your knowledge
% of TeX increases, you will probably want to edit this, but
% it should be fine as is for beginners.

% almost certainly you want these
\usepackage{amssymb}
\usepackage{amsmath}
\usepackage{amsthm}
\usepackage{amsfonts}

% used for TeXing text within eps files
%\usepackage{psfrag}
% need this for including graphics (\includegraphics)
%\usepackage{graphicx}
% for neatly defining theorems and propositions
%\usepackage{amsthm}
% making logically defined graphics
%%%\usepackage{xypic}

% there are many more packages, add them here as you need them

% define commands here

\newtheorem*{thm}{Theorem}
\newtheorem{defn}{Definition}
\newtheorem{prop}{Proposition}
\newtheorem{lemma}{Lemma}
\newtheorem{cor}{Corollary}

\theoremstyle{definition}
\newtheorem*{exa}{Example}

% Some sets
\newcommand{\Nats}{\mathbb{N}}
\newcommand{\Ints}{\mathbb{Z}}
\newcommand{\Reals}{\mathbb{R}}
\newcommand{\Complex}{\mathbb{C}}
\newcommand{\Rats}{\mathbb{Q}}
\newcommand{\Gal}{\operatorname{Gal}}
\newcommand{\Cl}{\operatorname{Cl}}
\begin{document}
The following formula is due to Thabit ibn Qurra (836-901), a mathematician who worked in Baghdad's ``House of Wisdom'' translating Greek and Syrian works (such as Apollonius's ``Conics'' or works of Euclid and Archimedes). As he translated the texts, ibn Qurra produced a mathematical body of his own.

\begin{thm}
Let $n\geq 1$ be a natural number and suppose that the numbers 
$$3\cdot 2^n -1, \quad 3\cdot 2^{n-1}-1 \quad \text{ and } \quad 9\cdot 2^{2n-1}-1$$
are all prime. Then the numbers:
$$2^n(3\cdot 2^n-1)(3\cdot 2^{n-1}-1)\quad \text{ and } \quad 2^n(9\cdot 2^{2n-1}-1)$$
are amicable numbers.
\end{thm}

\begin{exa}
When $n=2$ one has:
$$3\cdot 2^2 -1=11, \quad 3\cdot 2^{2-1}-1=5 \quad \text{ and } \quad 9\cdot 2^{4-1}-1=71$$
which are all primes. Thus, the numbers:
$$2^2(3\cdot 2^2-1)(3\cdot 2^{2-1}-1)=220\quad \text{ and } \quad 2^2(9\cdot 2^{4-1}-1)=284$$
form an amicable pair. In fact, this is the smallest amicable pair. For $n=4$ one obtains the amicable pair $17296$ and $18416$.
\end{exa}
%%%%%
%%%%%
\end{document}
