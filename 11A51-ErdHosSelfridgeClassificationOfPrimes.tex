\documentclass[12pt]{article}
\usepackage{pmmeta}
\pmcanonicalname{ErdHosSelfridgeClassificationOfPrimes}
\pmcreated{2013-03-22 16:05:02}
\pmmodified{2013-03-22 16:05:02}
\pmowner{PrimeFan}{13766}
\pmmodifier{PrimeFan}{13766}
\pmtitle{Erd\H{o}s-Selfridge classification of primes}
\pmrecord{5}{38143}
\pmprivacy{1}
\pmauthor{PrimeFan}{13766}
\pmtype{Definition}
\pmcomment{trigger rebuild}
\pmclassification{msc}{11A51}
\pmsynonym{Erdos-Selfridge classification of primes}{ErdHosSelfridgeClassificationOfPrimes}

% this is the default PlanetMath preamble.  as your knowledge
% of TeX increases, you will probably want to edit this, but
% it should be fine as is for beginners.

% almost certainly you want these
\usepackage{amssymb}
\usepackage{amsmath}
\usepackage{amsfonts}

% used for TeXing text within eps files
%\usepackage{psfrag}
% need this for including graphics (\includegraphics)
%\usepackage{graphicx}
% for neatly defining theorems and propositions
%\usepackage{amsthm}
% making logically defined graphics
%%%\usepackage{xypic}

% there are many more packages, add them here as you need them

% define commands here

\begin{document}
Paul Erd\H{o}s and John Selfridge classified primes $p$ thus: If the largest prime factor of $p + 1$ is 2 or 3, then $p$ is in class 1+. Otherwise, assign to $c$ the class of the largest prime factor of $p + 1$, then $p$ is in class $(c + 1)+$. Class 1+ primes are of the form $2^i3^j - 1$ for $i > -1$ and $j > -1$.

According to this scheme, $p < 200$ are sorted thus:

Class 1+: 2, 3, 5, 7, 11, 17, 23, 31, 47, 53, 71, 107, 127, 191 ( listed in A005105 of Sloane's OEIS)

Class 2+: 13, 19, 29, 41, 43, 59, 61, 67, 79, 83, 89, 97, 101, 109, 131, 137, 139, 149, 167, 179, 197, 199 (A005106)

Class 3+: 37, 103, 113, 151, 157, 163, 173, 181, 193 (A005107)

Class 4+: 73

A005113 lists the smallest prime of class $n+$.

Clearly, all Mersenne primes are class 1+. The known Fermat primes show slightly more variety: 257 is class 3+ while 65537 is class 4+.

\begin{thebibliography}{1}
\bibitem{rg} R. K. Guy, {\it Unsolved Problems in Number Theory}. New York: Springer-Verlag (2004)
\end{thebibliography}


%%%%%
%%%%%
\end{document}
