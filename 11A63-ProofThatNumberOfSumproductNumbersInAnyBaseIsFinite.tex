\documentclass[12pt]{article}
\usepackage{pmmeta}
\pmcanonicalname{ProofThatNumberOfSumproductNumbersInAnyBaseIsFinite}
\pmcreated{2013-03-22 15:47:06}
\pmmodified{2013-03-22 15:47:06}
\pmowner{rspuzio}{6075}
\pmmodifier{rspuzio}{6075}
\pmtitle{proof that number of sum-product numbers in any base is finite}
\pmrecord{7}{37743}
\pmprivacy{1}
\pmauthor{rspuzio}{6075}
\pmtype{Proof}
\pmcomment{trigger rebuild}
\pmclassification{msc}{11A63}

\endmetadata

% this is the default PlanetMath preamble.  as your knowledge
% of TeX increases, you will probably want to edit this, but
% it should be fine as is for beginners.

% almost certainly you want these
\usepackage{amssymb}
\usepackage{amsmath}
\usepackage{amsfonts}

% used for TeXing text within eps files
%\usepackage{psfrag}
% need this for including graphics (\includegraphics)
%\usepackage{graphicx}
% for neatly defining theorems and propositions
%\usepackage{amsthm}
% making logically defined graphics
%%%\usepackage{xypic}

% there are many more packages, add them here as you need them

% define commands here
\begin{document}
Let $b$ be the base of numeration.

Suppose that an integer $n$ has $m$ digits when expressed in base $b$
(not counting leading zeros, of course).  Then $n \ge b^{m-1}$.

Since each digit is at most $b-1$, we have that the sum of the digits
is at most $m(b-1)$ and the product is at most $(b-1)^m$, hence the
sum of the digits of $n$ times the product of the digits of $n$ is at
most $m(b-1)^{m+1}$.

If $n$ is a sum-product number, then $n$ equals the sum of its digits
times the product of its digits.  In light of the inequalities of the
last two paragraphs, this implies that $m(b-1)^{m+1} \ge n \ge
b^{m-1}$, so $m(b-1)^{m+1} \ge b^{m-1}$.  Dividing both sides, we
obtain $(b-1)^2 m \ge (b/(b-1))^{m-1}$.  By the growth of exponential
function, there can only be a finite number of values of $m$ for which
this is true.  Hence, there is a finite limit to the number of digits
of $n$, so there can only be a finite number of sum-product numbers to
any given base $b$.
%%%%%
%%%%%
\end{document}
