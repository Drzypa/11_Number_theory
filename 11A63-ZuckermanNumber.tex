\documentclass[12pt]{article}
\usepackage{pmmeta}
\pmcanonicalname{ZuckermanNumber}
\pmcreated{2013-03-22 16:04:36}
\pmmodified{2013-03-22 16:04:36}
\pmowner{CompositeFan}{12809}
\pmmodifier{CompositeFan}{12809}
\pmtitle{Zuckerman number}
\pmrecord{4}{38134}
\pmprivacy{1}
\pmauthor{CompositeFan}{12809}
\pmtype{Definition}
\pmcomment{trigger rebuild}
\pmclassification{msc}{11A63}

% this is the default PlanetMath preamble.  as your knowledge
% of TeX increases, you will probably want to edit this, but
% it should be fine as is for beginners.

% almost certainly you want these
\usepackage{amssymb}
\usepackage{amsmath}
\usepackage{amsfonts}

% used for TeXing text within eps files
%\usepackage{psfrag}
% need this for including graphics (\includegraphics)
%\usepackage{graphicx}
% for neatly defining theorems and propositions
%\usepackage{amsthm}
% making logically defined graphics
%%%\usepackage{xypic}

% there are many more packages, add them here as you need them

% define commands here

\begin{document}
Consider the integer 384. Multiplying its digits, $$3 \times 8 \times 4 = 96$$ and $${{384} \over {96}} = 91.$$

When an integer is divisible by the product of its digits, it's called a {\em Zuckerman number}. That is, given $m$ is the number of digits of $n$ and $d_x$ (for $x \le k$) is an integer of $n$,

$${\prod_{i = 1}^m d_i}|n$$

All 1-digit numbers and the base number itself are Zuckerman numbers.

It is possible for an integer to be divisible by its multiplicative digital root and yet not be a Zuckerman number because it doesn't divide its first digit product evenly (for example, 1728 in base 10 has multiplicative digital root 2 but is not divisible by $1 \times 7 \times 2 \times 8 = 112$). The reverse is also possible (for example, 384 is divisible by 96, as shown above, but clearly not by its multiplicative digital root 0).

\begin{thebibliography}{2}
\bibitem{jt} J. J. Tattersall, {\it Elementary number theory in nine chapters}, p. 86. Cambridge: Cambridge University Press (2005)
\end{thebibliography}

%%%%%
%%%%%
\end{document}
