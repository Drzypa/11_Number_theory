\documentclass[12pt]{article}
\usepackage{pmmeta}
\pmcanonicalname{CauchyDavenportTheorem}
\pmcreated{2013-03-22 13:32:13}
\pmmodified{2013-03-22 13:32:13}
\pmowner{bbukh}{348}
\pmmodifier{bbukh}{348}
\pmtitle{Cauchy-Davenport theorem}
\pmrecord{7}{34133}
\pmprivacy{1}
\pmauthor{bbukh}{348}
\pmtype{Theorem}
\pmcomment{trigger rebuild}
\pmclassification{msc}{11B05}
\pmrelated{Sumset}

\usepackage{amssymb}
\usepackage{amsmath}
\usepackage{amsfonts}

%%%\usepackage{xypic}
\newcommand*{\abs}[1]{\left\lvert #1\right\rvert}
\newcommand*{\integers}{\ensuremath{{\mathbb{Z}}}}
\makeatletter
\@ifundefined{bibname}{}{\renewcommand{\bibname}{References}}
\makeatother
\begin{document}
If $A$ and $B$ are non-empty subsets of $\integers_p$, then
\begin{equation*}
\abs{A+B}\geq \min(\abs{A}+\abs{B}-1, p),
\end{equation*}
where $A+B$ denotes the sumset of $A$ and $B$.

\begin{thebibliography}{1}

\bibitem{cite:nathanson_classicalbases}
Melvyn~B. Nathanson.
\newblock {\em Additive Number Theory: Inverse Problems and Geometry of
  Sumsets}, volume 165 of {\em GTM}.
\newblock Springer, 1996.
\newblock \PMlinkexternal{Zbl 0859.11003}{http://www.emis.de/cgi-bin/zmen/ZMATH/en/quick.html?type=html&an=0859.11003}.


\end{thebibliography}

%@BOOK{cite:nathanson_inverseprob,
% author    = {Melvyn B. Nathanson},
% title     = {Additive Number Theory: Inverse Problems and Geometry of Sumsets},
% series    = {GTM},
% volume    = 165,
% year      = 1996,
% publisher = {Springer}
%}
%%%%%
%%%%%
\end{document}
