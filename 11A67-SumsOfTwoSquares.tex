\documentclass[12pt]{article}
\usepackage{pmmeta}
\pmcanonicalname{SumsOfTwoSquares}
\pmcreated{2013-11-19 16:28:21}
\pmmodified{2013-11-19 16:28:21}
\pmowner{pahio}{2872}
\pmmodifier{pahio}{2872}
\pmtitle{sums of two squares}
\pmrecord{33}{35842}
\pmprivacy{1}
\pmauthor{pahio}{2872}
\pmtype{Theorem}
\pmcomment{trigger rebuild}
\pmclassification{msc}{11A67}
\pmclassification{msc}{11E25}
\pmsynonym{Diophantus' identity}{SumsOfTwoSquares}
\pmsynonym{Brahmagupta's identity}{SumsOfTwoSquares}
\pmsynonym{Fibonacci's identity}{SumsOfTwoSquares}
\pmrelated{EulerFourSquareIdentity}
\pmrelated{TheoremsOnSumsOfSquares}
\pmrelated{DifferenceOfSquares}

% this is the default PlanetMath preamble.  as your knowledge
% of TeX increases, you will probably want to edit this, but
% it should be fine as is for beginners.

% almost certainly you want these
\usepackage{amssymb}
\usepackage{amsmath}
\usepackage{amsfonts}

% used for TeXing text within eps files
%\usepackage{psfrag}
% need this for including graphics (\includegraphics)
%\usepackage{graphicx}
% for neatly defining theorems and propositions
 \usepackage{amsthm}
% making logically defined graphics
%%%\usepackage{xypic}

% there are many more packages, add them here as you need them

% define commands here
\theoremstyle{definition}
\newtheorem*{thmplain}{Theorem}
\begin{document}
\begin{thmplain}
\,\, The set of the sums of two squares of integers is closed under multiplication; in fact we have the identical equation
\begin{align}
(a^2\!+\!b^2)(c^2\!+\!d^2) \;=\; (ac\!-\!bd)^2\!+\!(ad\!+\!bc)^2.
\end{align}
\end{thmplain}

This was presented by Leonardo Fibonacci in 1225 (in 
\emph{Liber quadratorum}), but was known also by Brahmagupta 
and already by Diophantus of Alexandria (III book of his 
\emph{Arithmetica}).

The proof of the equation may utilize Gaussian integers as follows:
\begin{align*}
   (a^2\!+\!b^2)(c^2\!+\!d^2) &\;=\; 
    (a\!+\!ib)(a\!-\!ib)(c\!+\!id)(c\!-\!id)\\
&\;=\;(a\!+\!ib)(c\!+\!id)(a\!-\!ib)(c\!-\!id)\\
&\;=\;[(ac\!-\!bd)\!+\!i(ad\!+\!bc)][(ac\!-\!bd)\!-\!i(ad\!+\!bc)]\\
&\;=\;(ac\!-\!bd)^2\!+\!(ad\!+\!bc)^2
\end{align*}\\


\textbf{Note 1.}\, The equation (1) is the special case\, $n = 2$\, 
of Lagrange's identity.\\

\textbf{Note 2.}\, Similarly as (1), one can derive the identity
\begin{align}
(a^2\!+\!b^2)(c^2\!+\!d^2) \;=\; (ac\!+\!bd)^2\!+\!(ad\!-\!bc)^2.
\end{align}
Thus in most cases, we can get two different nontrivial sum forms 
(i.e. without a zero addend) for a given product of two sums of 
squares.\, For example, the product
$$65 = 5\!\cdot\!13 = (2^2\!+\!1^2)(3^2\!+\!2^2)$$
attains the two forms  $4^2\!+\!7^2$ and $8^2\!+\!1^2$.


%%%%%
%%%%%
\end{document}
