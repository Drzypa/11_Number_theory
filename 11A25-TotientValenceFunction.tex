\documentclass[12pt]{article}
\usepackage{pmmeta}
\pmcanonicalname{TotientValenceFunction}
\pmcreated{2013-03-22 15:50:57}
\pmmodified{2013-03-22 15:50:57}
\pmowner{CompositeFan}{12809}
\pmmodifier{CompositeFan}{12809}
\pmtitle{totient valence function}
\pmrecord{5}{37832}
\pmprivacy{1}
\pmauthor{CompositeFan}{12809}
\pmtype{Definition}
\pmcomment{trigger rebuild}
\pmclassification{msc}{11A25}

\endmetadata

% this is the default PlanetMath preamble.  as your knowledge
% of TeX increases, you will probably want to edit this, but
% it should be fine as is for beginners.

% almost certainly you want these
\usepackage{amssymb}
\usepackage{amsmath}
\usepackage{amsfonts}

% used for TeXing text within eps files
%\usepackage{psfrag}
% need this for including graphics (\includegraphics)
%\usepackage{graphicx}
% for neatly defining theorems and propositions
%\usepackage{amsthm}
% making logically defined graphics
%%%\usepackage{xypic}

% there are many more packages, add them here as you need them

% define commands here
\begin{document}
Given an integer $n$, count how many integers $m$ in the set $\{n + 1, n + 2, \ldots n^2\}$ satisfy $\phi(m) = n$. This is the {\em totient valence} of $n$, usually labelled $N_{\phi}(n)$. (The only two special cases are 2 and 6, for which one has to look a little beyound $n^2$).

Robert Carmichael conjectured that $N_{\phi}(n) = 1$ never. Two sequences in Sloane's OEIS that list numbers with higher totient valences than preceding numbers are A007374 and A097942.
%%%%%
%%%%%
\end{document}
