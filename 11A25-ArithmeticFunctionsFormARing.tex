\documentclass[12pt]{article}
\usepackage{pmmeta}
\pmcanonicalname{ArithmeticFunctionsFormARing}
\pmcreated{2013-03-22 16:30:28}
\pmmodified{2013-03-22 16:30:28}
\pmowner{rm50}{10146}
\pmmodifier{rm50}{10146}
\pmtitle{arithmetic functions form a ring}
\pmrecord{6}{38683}
\pmprivacy{1}
\pmauthor{rm50}{10146}
\pmtype{Theorem}
\pmcomment{trigger rebuild}
\pmclassification{msc}{11A25}
\pmrelated{ConvolutionInversesForArithmeticFunctions}

% this is the default PlanetMath preamble.  as your knowledge
% of TeX increases, you will probably want to edit this, but
% it should be fine as is for beginners.

% almost certainly you want these
\usepackage{amssymb}
\usepackage{amsmath}
\usepackage{amsfonts}

% used for TeXing text within eps files
%\usepackage{psfrag}
% need this for including graphics (\includegraphics)
%\usepackage{graphicx}
% for neatly defining theorems and propositions
%\usepackage{amsthm}
% making logically defined graphics
%%%\usepackage{xypic}

% there are many more packages, add them here as you need them

% define commands here
\newtheorem{thm}{Theorem}

\begin{document}
\begin{thm} The set $\mathcal{S}$ of arithmetic functions forms a commutative ring with unity under the operations of element-by-element addition and Dirichlet convolution, i.e. under
\begin{align*}(f+g)(n)&=f(n)+g(n)\\
(f*g)(n)&=\sum_{d|n} f(d)g\left(\frac{n}{d}\right)
\end{align*}
The $0$ of the ring is the function $z$ such that $z(n)=0$ for all positive integers $n$, the $1$ of the ring is the convolution identity function $\varepsilon$, and the units of the ring are those arithmetic functions $f$ such that $f(1)\neq 0$.
\end{thm}

\textbf{Proof.}
This is essentially a triviality and a little bit of computation.

That $\mathcal{S}$ is an abelian group under $+$ is obvious; the only interesting \PMlinkescapetext{point} is noting that indeed $z$ is the identity of the group (the $0$ of the ring).

Many of the ring identities are also obvious. We will prove that  $\varepsilon$ is the multiplicative identity, that $*$ is commutative and associative, that $*$ distributes over $+$, and that the units of the ring are as stated.

To see that $\varepsilon$ is the multiplicative identity, note that
\[(\varepsilon*f)(n)=\sum_{d|n} \varepsilon(d)f\left(\frac{n}{d}\right)=\varepsilon(1)f(n)=f(n)\]
and thus $\varepsilon*f=f$.

To see that $*$ is commutative, note that $f*g$ can also be written as
\[(f*g)(n)=\sum_{ab=n}f(a)g(b)\]
Commutativity is obvious from this \PMlinkescapetext{representation} of the operation.

Associativity follows similarly. Note that
\[((f*g)*h)(n)= \sum_{ra=n}(f*g)(r)h(a)=\sum_{ra=n}h(a)\sum_{bc=r}f(b)g(c)=\sum_{abc=n}f(b)g(c)h(a)\]
If one expands $(f*(g*h))(n)$ similarly, the resulting sum is identical, so the two are equal.

Distributivity follows since
\begin{multline*}(f*(g+h))(n)=\sum_{d|n}f(d)\left(g+h\right)\left(\frac{n}{d}\right)=\sum_{d|n}f(d)\left(g\left(\frac{n}{d}\right)+h\left(\frac{n}{d}\right)\right)=\\
\sum_{d|n}f(d)g\left(\frac{n}{d}\right)+\sum_{d|n}f(d)h\left(\frac{n}{d}\right)=((f*g)+(f*h))(n)
\end{multline*}

The units of the ring are simply the invertible functions; the entry on convolution inverses for arithmetic functions shows that the invertible functions are those functions $f$ with $f(1)\neq 0$.

%%%%%
%%%%%
\end{document}
