\documentclass[12pt]{article}
\usepackage{pmmeta}
\pmcanonicalname{GroupTheoreticProofOfWilsonsTheorem}
\pmcreated{2013-03-22 13:35:27}
\pmmodified{2013-03-22 13:35:27}
\pmowner{ottocolori}{1519}
\pmmodifier{ottocolori}{1519}
\pmtitle{group theoretic proof of Wilson's theorem}
\pmrecord{10}{34214}
\pmprivacy{1}
\pmauthor{ottocolori}{1519}
\pmtype{Proof}
\pmcomment{trigger rebuild}
\pmclassification{msc}{11-00}
%\pmkeywords{Sylow}
%\pmkeywords{cycles}

% this is the default PlanetMath preamble.  as your knowledge
% of TeX increases, you will probably want to edit this, but
% it should be fine as is for beginners.

% almost certainly you want these
\usepackage{amssymb}
\usepackage{amsmath}
\usepackage{amsfonts}

% used for TeXing text within eps files
%\usepackage{psfrag}
% need this for including graphics (\includegraphics)
%\usepackage{graphicx}
% for neatly defining theorems and propositions
%\usepackage{amsthm}
% making logically defined graphics
%%%\usepackage{xypic}

% there are many more packages, add them here as you need them

% define commands here
\begin{document}
Here we present a group theoretic proof of it.
\\Clearly, it is enough to show that
$ (p-2)!\equiv 1\pmod{p}$ since $ p-1\equiv -1\pmod{p}$.
By Sylow theorems, we have that $p$-Sylow subgroups of $S_p$, the
symmetric group on $p$ elements, have order $p$, and the number $n_p$ of
Sylow subgroups is congruent to 1 modulo $p$. Let $P$ be a Sylow subgroup
of $S_p$. Note that $P$ is generated by a $p$-cycle. There are $(p-1)!$ cycles
of length $p$ in $S_p$. Each $p$-Sylow subgroup contains $p-1$ cycles
of length $p$, hence there are $\frac{(p-1)!}{p-1}=(p-2)!$ different
$p$-Sylow subgrups in $S_p$, i.e. $n_P=(p-2)!$. From Sylow's Second
Theorem, it follows that $(p-2)!\equiv1\pmod{p}$,so $(p-1)!\equiv-1\pmod{p}$.
%%%%%
%%%%%
\end{document}
