\documentclass[12pt]{article}
\usepackage{pmmeta}
\pmcanonicalname{UnusualNumber}
\pmcreated{2013-03-22 18:09:43}
\pmmodified{2013-03-22 18:09:43}
\pmowner{PrimeFan}{13766}
\pmmodifier{PrimeFan}{13766}
\pmtitle{unusual number}
\pmrecord{4}{40721}
\pmprivacy{1}
\pmauthor{PrimeFan}{13766}
\pmtype{Definition}
\pmcomment{trigger rebuild}
\pmclassification{msc}{11A51}

\endmetadata

% this is the default PlanetMath preamble.  as your knowledge
% of TeX increases, you will probably want to edit this, but
% it should be fine as is for beginners.

% almost certainly you want these
\usepackage{amssymb}
\usepackage{amsmath}
\usepackage{amsfonts}

% used for TeXing text within eps files
%\usepackage{psfrag}
% need this for including graphics (\includegraphics)
%\usepackage{graphicx}
% for neatly defining theorems and propositions
%\usepackage{amsthm}
% making logically defined graphics
%%%\usepackage{xypic}

% there are many more packages, add them here as you need them

% define commands here

\begin{document}
An {\em unusual number} or {\em $\sqrt{n}$-rough number} $n$ is an integer with a greatest prime factor exceeding $\sqrt{n}$. For example, the greatest prime factor of 102 is 17, which is greater than 11 (the square root of 102 rounded up to the next higher integer). The first few unusual numbers are 2, 3, 5, 6, 7, 10, 11, 13, 14, 15, 17, 19, 20, 21, 22, 23, 26, 28, 29, 31, 33, 34, 35, 37, 38, 39, 41, 42, 43, 44, 46, 47, etc., listed in A064052 of Sloane's OEIS. In fact, Donald Knuth and Donald Greene, who coined the term ``unusual number,'' remark that these numbers occur so frequently they're not all that unusual. Unusual numbers include all of the prime numbers and many composites. Richard Schroeppel proved in HAKMEM 239 that the probability that a random integer is unusual is $\log 2$ (about 0.69314718).

\begin{thebibliography}{1}
\bibitem{dg} Donald Greene \& Donald Knuth, {\it Mathematics for the Analysis of Algorithms}, 3rd edition. Boston: Birkh\"auser (1990): 95 - 98
\end{thebibliography}

%%%%%
%%%%%
\end{document}
