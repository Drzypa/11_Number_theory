\documentclass[12pt]{article}
\usepackage{pmmeta}
\pmcanonicalname{SloanesConjectureOnMultiplicativeDigitalRoot}
\pmcreated{2013-03-22 16:00:45}
\pmmodified{2013-03-22 16:00:45}
\pmowner{CompositeFan}{12809}
\pmmodifier{CompositeFan}{12809}
\pmtitle{Sloane's conjecture on multiplicative digital root}
\pmrecord{7}{38047}
\pmprivacy{1}
\pmauthor{CompositeFan}{12809}
\pmtype{Conjecture}
\pmcomment{trigger rebuild}
\pmclassification{msc}{11A63}
\pmsynonym{Sloane-Erd\H{o}s conjecture on multiplicative digital root}{SloanesConjectureOnMultiplicativeDigitalRoot}

% this is the default PlanetMath preamble.  as your knowledge
% of TeX increases, you will probably want to edit this, but
% it should be fine as is for beginners.

% almost certainly you want these
\usepackage{amssymb}
\usepackage{amsmath}
\usepackage{amsfonts}

% used for TeXing text within eps files
%\usepackage{psfrag}
% need this for including graphics (\includegraphics)
%\usepackage{graphicx}
% for neatly defining theorems and propositions
%\usepackage{amsthm}
% making logically defined graphics
%%%\usepackage{xypic}

% there are many more packages, add them here as you need them

% define commands here

\begin{document}
It is believed that there is no integer has a multiplicative persistence greater than itself, a conjecture put forth in 1973 by Neil Sloane, and that Sloane meant to limit this conjecture to fixed radix bases.

In 1998, Diamond and Reidpath published a factorial base counterexample, by proving that ``it is possible to find a number in factorial base of arbitrarily large persistence,'' specifically a number of the form $$n!n + \sum_{i = 1}^{n - 1} i!$$ Obviously, this number will have a factorial base multiplicative digital root of $n$ and a persistence of also $n$, suggesting an upper bound for the desired counterexample.

\begin{thebibliography}{2}
\bibitem{md} M. R. Diamond, D. D. Reidpath, ``A Counterexample to Conjectures by Sloane and Erdos Concerning the Persistence of Numbers", {\it J. Rec. Math.} 29 (1998), 89 - 92. 
\bibitem{ns} N. J. A. Sloane, ``The persistence of a number" {\it J. Rec. Math.} 6 (1973), 97 - 98.
\end{thebibliography}
%%%%%
%%%%%
\end{document}
