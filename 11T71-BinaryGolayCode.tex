\documentclass[12pt]{article}
\usepackage{pmmeta}
\pmcanonicalname{BinaryGolayCode}
\pmcreated{2013-03-22 14:23:39}
\pmmodified{2013-03-22 14:23:39}
\pmowner{mathcam}{2727}
\pmmodifier{mathcam}{2727}
\pmtitle{binary Golay code}
\pmrecord{4}{35891}
\pmprivacy{1}
\pmauthor{mathcam}{2727}
\pmtype{Definition}
\pmcomment{trigger rebuild}
\pmclassification{msc}{11T71}
\pmrelated{LeechLattice}
\pmrelated{Hexacode}
\pmdefines{extended binary golay code}

\endmetadata

% this is the default PlanetMath preamble.  as your knowledge
% of TeX increases, you will probably want to edit this, but
% it should be fine as is for beginners.

% almost certainly you want these
\usepackage{amssymb}
\usepackage{amsmath}
\usepackage{amsfonts}
\usepackage{amsthm}

% used for TeXing text within eps files
%\usepackage{psfrag}
% need this for including graphics (\includegraphics)
%\usepackage{graphicx}
% for neatly defining theorems and propositions
%\usepackage{amsthm}
% making logically defined graphics
%%%\usepackage{xypic}

% there are many more packages, add them here as you need them

% define commands here

\newcommand{\mc}{\mathcal}
\newcommand{\mb}{\mathbb}
\newcommand{\mf}{\mathfrak}
\newcommand{\ol}{\overline}
\newcommand{\ra}{\rightarrow}
\newcommand{\la}{\leftarrow}
\newcommand{\La}{\Leftarrow}
\newcommand{\Ra}{\Rightarrow}
\newcommand{\nor}{\vartriangleleft}
\newcommand{\Gal}{\text{Gal}}
\newcommand{\GL}{\text{GL}}
\newcommand{\Z}{\mb{Z}}
\newcommand{\R}{\mb{R}}
\newcommand{\Q}{\mb{Q}}
\newcommand{\C}{\mb{C}}
\newcommand{\<}{\langle}
\renewcommand{\>}{\rangle}
\begin{document}
The \emph{binary Golay Code} $\mc{G}_{23}$ is a perfect linear binary [23,12,7]-code with a plethora of different \PMlinkescapetext{equivalent} constructions.

\subsection*{Sample Constructions}
\begin{itemize}
\item {\bf Lexicographic Construction:}  Let $v_0$ be the all-zero word in $\mb{F}_2^{23}$, and inductively define $v_j$ to be the smallest word (smallest with respect to the lexicographic ordering on $\mb{F}_2^{23}$ that differs from $v_i$ in at least 7 places for all $i<j$.  
\item {\bf \PMlinkescapetext{Quadratic Residue} Construction:}  $\mc{G}_{23}$ is the quadratic residue code of length 23.
\end{itemize}

The \emph{extended binary Golay Code} $\mc{G}_{24}$ is obtained by appending a zero-sum check digit to the end of every word in $\mc{G}_{23}$. 

Both the binary Golay code and the extended binary Golay code have some remarkable \PMlinkescapetext{properties}.

\subsection*{Properties}
\begin{itemize}
\item $\mc{G}_{24}$ has 4096 codewords:  1 of weight 0, 759 of weight 8, 2576 of weight 12, 759 of weight 18, and 1 of weight 24.
\item The automorphism group of $\mc{G}_{24}$ is the Mathieu group $M_{24}$, one of the sporadic groups.
\item The Golay Code is used to define the Leech Lattice, one of the most efficient sphere-packings known to date.
\item The optimal strategy to the mathematical game called Mogul is to always revert the current position to one corresponding to a word of the Golay code.
\item The words of weight 8 in $\mc{G}_{24}$ form a $S(5,8,24)$ Steiner system.  In fact, this property uniquely determines the code.
\end{itemize}
%%%%%
%%%%%
\end{document}
