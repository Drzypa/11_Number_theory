\documentclass[12pt]{article}
\usepackage{pmmeta}
\pmcanonicalname{LiouvilleApproximationTheorem}
\pmcreated{2013-03-22 11:45:45}
\pmmodified{2013-03-22 11:45:45}
\pmowner{KimJ}{5}
\pmmodifier{KimJ}{5}
\pmtitle{Liouville approximation theorem}
\pmrecord{13}{30211}
\pmprivacy{1}
\pmauthor{KimJ}{5}
\pmtype{Theorem}
\pmcomment{trigger rebuild}
\pmclassification{msc}{11J68}
\pmclassification{msc}{46-01}
\pmclassification{msc}{46N40}
%\pmkeywords{number theory}
\pmrelated{ExampleOfTranscendentalNumber}

\usepackage{amssymb}
\usepackage{amsmath}
\usepackage{amsfonts}
\usepackage{graphicx}
%%%%\usepackage{xypic}
\begin{document}
Given $\alpha$, a real algebraic number of degree $n \neq 1$, there is a constant $c = c( \alpha ) > 0$ such that for all rational numbers $p/q, (p,q)=1$, the inequality
\[ \left| \alpha - \frac{p}{q} \right| > \frac{c(\alpha )}{q^n} \] holds.

Many mathematicians have worked at strengthening this theorem:
\begin{itemize}
\item Thue: If $\alpha$ is an algebraic number of degree $n \geq 3$, then there is a constant $c_0 = c_0( \alpha , \epsilon ) > 0$ such that for all rational numbers $p/q$, the inequality
\[ \left| \alpha - \frac{p}{q} \right| > c_0 q^{-1- \epsilon - n/2} \] holds.

\item Siegel: If $\alpha$ is an algebraic number of degree $n \geq 2$, then there is a constant $c_1 = c_1( \alpha , \epsilon ) > 0$ such that for all rational numbers $p/q$, the inequality
\[ \left| \alpha - \frac{p}{q} \right| > c_1 q^{- \lambda}, \qquad \lambda = {\min}_{t=1,\ldots ,n} \left( \frac{n}{t+1} + t \right) + \epsilon \] holds.

\item Dyson: If $\alpha$ is an algebraic number of degree $n > 3$, then there is a constant $c_2 = c_2( \alpha , \epsilon ) > 0$ such that for all rational numbers $p/q$ with $q > c_2$, the inequality
\[ \left| \alpha - \frac{p}{q} \right| > q^{- \sqrt{2n}- \epsilon } \] holds.

\item Roth: If $\alpha$ is an irrational algebraic number and $\epsilon > 0$, then there is a constant $c_3 = c_3( \alpha , \epsilon ) > 0$ such that for all rational numbers $p/q$, the inequality
\[ \left| \alpha - \frac{p}{q} \right| > c_3 q^{-2 - \epsilon } \] holds.

\end{itemize}
%%%%%
%%%%%
%%%%%
%%%%%
\end{document}
