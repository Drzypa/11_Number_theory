\documentclass[12pt]{article}
\usepackage{pmmeta}
\pmcanonicalname{ProofOfInfinitudeOfPrimes}
\pmcreated{2013-12-27 17:11:49}
\pmmodified{2013-12-27 17:11:49}
\pmowner{rspuzio}{6075}
\pmmodifier{rspuzio}{6075}
\pmtitle{proof of infinitude of primes}
\pmrecord{15}{39404}
\pmprivacy{1}
\pmauthor{rspuzio}{6075}
\pmtype{Proof}
\pmcomment{trigger rebuild}
\pmclassification{msc}{11A41}

\endmetadata

% this is the default PlanetMath preamble.  as your knowledge
% of TeX increases, you will probably want to edit this, but
% it should be fine as is for beginners.

% almost certainly you want these
\usepackage{amssymb}
\usepackage{amsmath}
\usepackage{amsfonts}

% used for TeXing text within eps files
%\usepackage{psfrag}
% need this for including graphics (\includegraphics)
%\usepackage{graphicx}
% for neatly defining theorems and propositions
%\usepackage{amsthm}
% making logically defined graphics
%%%\usepackage{xypic}

% there are many more packages, add them here as you need them

% define commands here

\begin{document}
We begin by noting a fact about factorizations.  Suppose that $n > 0$ is an integer
which has a prime factorization
\[
 n = p_1^{k_1} p_2^{k_2} \cdots p_m^{k_m} .
\]
Then, because $2$ is the smallest prime number, we must have
\[
 p_1^{k_1} p_2^{k_2} \cdots p_m^{k_m} \ge
 2^{k_1} 2^{k_2} \cdots 2^{k_m} =
 2^{k_1 + k_2 + \cdots + k_m},
\]
so $n \ge 2^{k_1 + k_2 + \cdots + k_m}$.

Assume that there were only a finite number of prime numbers $p_1, p_2, \ldots p_m$.
By the above-noted fact, given a positive integer $j$, every integer n such that $2^j \ge n > 0$ could be
expressed as
\[
 n = p_1^{k_1} p_2^{k_2} \cdots p_m^{k_m}
\]
with
\[
 k_1 + k_2 + \cdots + k_m \le j .
\]

This, however, leads to a contradiction because it would imply that there exist more
integers than possible factorizations despite the fact that every integer is supposed
to have a prime factorization.  To see this, let us over-count the number of 
factorizations.  A factorization being specified by an $m$-tuplet of integers
$k_1, k_2, \ldots, k_n$ such that $k_1 + k_2 + \cdots + k_m \le j$, the number of
factorizations is equal to the number of such tuplets.  Now, for all $i$ we must have
$0 \le k_i \le j$, so there are not more than $(j+1)^m$ such tuplets available.
However, for all $m$, one can choose $j$ such that $2^j > j^m$.  For such a choice
of $j$ we could not make ends meet --- there are not enough possible factorizations 
available to handle all integers, so we conclude that there must be more than $m$ primes
for any integer $m$, i.e. that the number of primes is infinite.

To make this exposition self-contained, we conclude with a proof that, for every $m$,
there exists a $j$ such that $2^j > (j+1)^m$.  We begin by showing that, 
for every integer $a \ge 1$, we have $2^a > a$.  This is an easy induction.  When $a = 1$, 
we have $2^1 = 2 > 1$.  If $2^a > a$ for some $a > 1$, then $2^{a+1} = 2^a + 2^a > a + a > a+1$.  
Hence, by induction, $2^a > a$ for all $a \ge 1$.

From this starting point, we obtain the desired inequality by algebraic manipulation.
Suppose that $a > 2$.  Multiplying both sides by $2$, we get $2^{a+1} > 2a > a+2$.
Setting $a = m - 2$, we have $2^{m-1} > m$, or $2^m > 2 m$.  Raising both sides to the
$2m$-th power, $2^{2 m^2} > 2^{2m} m^{2m} = 2^m (2 m^2)^m \ge 2 (2 m^2)^m$.  Setting
$j = (2 m^2 - 1)$, this becomes $2^j > (j+1)^2$.

%%%%%
%%%%%
\end{document}
