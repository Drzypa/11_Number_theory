\documentclass[12pt]{article}
\usepackage{pmmeta}
\pmcanonicalname{BinomialTheoremProofOf}
\pmcreated{2013-03-22 15:03:54}
\pmmodified{2013-03-22 15:03:54}
\pmowner{mps}{409}
\pmmodifier{mps}{409}
\pmtitle{binomial theorem, proof of}
\pmrecord{15}{36785}
\pmprivacy{1}
\pmauthor{mps}{409}
\pmtype{Proof}
\pmcomment{trigger rebuild}
\pmclassification{msc}{11B65}

% this is the default PlanetMath preamble.  as your knowledge
% of TeX increases, you will probably want to edit this, but
% it should be fine as is for beginners.

% almost certainly you want these
\usepackage{amssymb}
\usepackage{amsmath}
\usepackage{amsfonts}

% used for TeXing text within eps files
%\usepackage{psfrag}
% need this for including graphics (\includegraphics)
%\usepackage{graphicx}
% for neatly defining theorems and propositions
\usepackage{amsthm}
% making logically defined graphics
%%%\usepackage{xypic}

% there are many more packages, add them here as you need them

% define commands here
\newtheorem*{proposition*}{Proposition}
\newcommand{\fm}[1]{{\it #1}}
\begin{document}
\begin{proposition*}
Let \fm{a} and \fm{b} be commuting elements of some rig.  Then
\[
(a+b)^n = \sum_{k=0}^n \binom{n}{k} a^k b^{n-k},
\]
where the $\binom{n}{k}$ are binomial coefficients.
\end{proposition*}

\begin{proof}
Each term in the expansion of $(a+b)^n$ is obtained by making \fm{n}
decisions of whether to use \fm{a} or \fm{b} as a factor.  Moreover,
any sequence of \fm{n} such decisions yields a term in the expansion.
So the expandsion of $(a+b)^n$ is precisely the sum of all the
\fm{ab}-words of length \fm{n}, where each word appears exactly once.

Since \fm{a} and \fm{b} commute, we can reduce each term via rewrite
rules of the form $b^na\mapsto ab^n$ to a term in which the \fm{a}
factors precede all the \fm{b} factors.  This produces a term of the
form $a^k b^{n-k}$ for some \fm{k}, where we use the expressions $a^0
b^n$ and $a^n b^0$ to denote $b^n$ and $a^n$ respectively.  For
example, reducing the word $babab^2aba$ yields $a^4 b^5$, via the
following reduction.
\[
babab^2aba \mapsto
a ba ba b^2a b \mapsto
a^2babab^3 \mapsto
a^3bab^4 \mapsto
a^4b^5.
\]
After performing this rewriting process, we collect like terms.  Let
us illustrate this with the case \fm{n} = 3.
\begin{align*}
(a+b)^3
&= aaa + aab + aba + abb + baa + bab + bba + bbb \\
&= a^3 + a^2b + a^2 b + ab^2 +a^2 b + ab^2 + ab^2 + b^3 \\
&= a^3 + 3a^2b + 3ab^2 + b^3.
\end{align*}
To determine the coefficient of a reduced term, it suffices to
determine how many \fm{ab}-words have that reduction.  Since reducing
a term only changes the positions of \fm{a}s and \fm{b}s and not their
number, all the \fm{ab}-words where \fm{k} of the letters are \fm{b}s
and \fm{n-k} are \fm{a}s, for $0\le k\le n$, have the same
normalization.  But there are exactly $\binom{n}{k}$ such
\fm{ab}-words, since there are $\binom{n}{k}$ ways to select \fm{k}
positions out of \fm{n} to place \fm{a}s in an \fm{ab}-word of length
\fm{n}.  This shows that the coefficient of the $a^n$ term is
$\binom{n}{0}=1$, the coefficient of the $b^n$ term is
$\binom{n}{n}=1$, and that the coefficient of the $a^k b^{n-k}$ term
is $\binom{n}{k}$.
\end{proof}

\PMlinkescapeword{expansion}
\PMlinkescapeword{length}
\PMlinkescapeword{place}
\PMlinkescapeword{term}
\PMlinkescapeword{terms}

%%%%%
%%%%%
\end{document}
