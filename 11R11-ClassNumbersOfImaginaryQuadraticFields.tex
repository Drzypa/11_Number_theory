\documentclass[12pt]{article}
\usepackage{pmmeta}
\pmcanonicalname{ClassNumbersOfImaginaryQuadraticFields}
\pmcreated{2013-03-22 18:31:20}
\pmmodified{2013-03-22 18:31:20}
\pmowner{pahio}{2872}
\pmmodifier{pahio}{2872}
\pmtitle{class numbers of imaginary quadratic fields}
\pmrecord{10}{41215}
\pmprivacy{1}
\pmauthor{pahio}{2872}
\pmtype{Data Structure}
\pmcomment{trigger rebuild}
\pmclassification{msc}{11R11}
\pmclassification{msc}{11R04}
\pmrelated{LemmaForImaginaryQuadraticFields}
\pmrelated{QuadraticImaginaryEuclideanNumberFields}

% this is the default PlanetMath preamble.  as your knowledge
% of TeX increases, you will probably want to edit this, but
% it should be fine as is for beginners.

% almost certainly you want these
\usepackage{amssymb}
\usepackage{amsmath}
\usepackage{amsfonts}

% used for TeXing text within eps files
%\usepackage{psfrag}
% need this for including graphics (\includegraphics)
%\usepackage{graphicx}
% for neatly defining theorems and propositions
 \usepackage{amsthm}
% making logically defined graphics
%%%\usepackage{xypic}

% there are many more packages, add them here as you need them

% define commands here

\theoremstyle{definition}
\newtheorem*{thmplain}{Theorem}

\begin{document}
We tabulate the \PMlinkname{ideal class numbers}{ClassNumber} $h$ of first imaginary quadratic number fields $\mathbb{Q}(\sqrt{d})$.\, The table \PMlinkescapetext{contains} all the nine cases where the class number is 1.\\

\begin{center}
\begin{tabular}{||c|c||c|c||c|c||c|c||}
\hline\hline
$d$ & $h$ & $d$ & $h$ & $d$ & $h$ & $d$ & $h$\\
\hline\hline
$-1$ & $1!$ & $-47$ & $5$ & $-97$ & $4$& $-146$ & $16$\\
\hline
$-2$ & $1!$ & $-51$ & $2$ & $-101$ & $14$& $-149$ & $14$\\
\hline
$-3$ & $1!$ & $-53$ & $6$ & $-102$ & $4$& $-151$ & $7$\\
\hline
$-5$ & $2$ & $-55$ & $4$ & $-103$ & $5$& $-154$ & $8$\\
\hline
$-6$ & $2$ & $-57$ & $4$ & $-105$ & $8$& $-155$ & $4$\\
\hline
$-7$ & $1!$ & $-58$ & $2$ & $-106$ & $6$& $-157$ & $6$\\
\hline
$-10$ & $2$ & $-59$ & $3$ & $-107$ & $3$& $-158$ & $8$\\
\hline
$-11$ & $1!$ & $-61$ & $6$ & $-109$ & $6$& $-159$ & $10$\\
\hline
$-13$ & $2$ & $-62$ & $8$ & $-110$ & $12$& $-161$ & $16$\\
\hline
$-14$ & $4$ & $-65$ & $8$ & $-111$ & $8$& $-163$ & $1!$\\
\hline
$-15$ & $2$ & $-66$ & $8$ & $-113$ & $8$& $-165$ & $8$\\
\hline
$-17$ & $4$ & $-67$ & $1!$ & $-114$ & $8$& $-166$ & $10$\\
\hline
$-19$ & $1!$ & $-69$ & $8$ & $-115$ & $2$& $-167$ & $11$\\
\hline
$-21$ & $4$ & $-70$ & $4$ & $-118$ & $6$& $-170$ & $12$\\
\hline
$-22$ & $2$ & $-71$ & $7$ & $-119$ & $10$& $-173$ & $14$\\
\hline
$-23$ & $3$ & $-73$ & $4$ & $-122$ & $10$& $-174$ & $12$\\
\hline
$-26$ & $6$ & $-74$ & $10$ & $-123$ & $2$& $-177$ & $4$\\
\hline
$-29$ & $6$ & $-77$ & $8$ & $-127$ & $5$& $-178$ & $8$\\
\hline
$-30$ & $4$ & $-78$ & $4$ & $-129$ & $12$& $-179$ & $5$\\
\hline
$-31$ & $3$ & $-79$ & $5$ & $-130$ & $4$& $-181$ & $10$\\
\hline
$-33$ & $4$ & $-82$ & $4$ & $-131$ & $5$& $-182$ & $12$\\
\hline
$-34$ & $4$ & $-83$ & $3$ & $-133$ & $4$& $-183$ & $8$\\
\hline
$-35$ & $2$ & $-85$ & $4$ & $-134$ & $14$& $-185$ & $16$\\
\hline
$-37$ & $2$ & $-86$ & $10$ & $-137$ & $8$& $-186$ & $12$\\
\hline
$-38$ & $6$ & $-87$ & $6$ & $-138$ & $8$& $-187$ & $2$\\
\hline
$-39$ & $4$ & $-89$ & $12$ & $-139$ & $3$& $-190$ & $4$\\
\hline
$-41$ & $8$ & $-91$ & $2$ & $-141$ & $8$& $-191$ & $13$\\
\hline
$-42$ & $4$ & $-93$ & $4$ & $-142$ & $4$& $-193$ & $4$\\
\hline
$-43$ & $1!$ & $-94$ & $8$ & $-143$ & $10$& $-194$ & $20$\\
\hline
$-46$ & $4$ & $-95$ & $8$ & $-145$ & $8$& $-195$ & $4$\\
\hline
\end{tabular}
\end{center}

The class numbers of $\mathbb{Q}(\sqrt{d})$ for the squarefree $d$'s form Sloane's \PMlinkexternal{sequence A000924}{http://www.research.att.com/~njas/sequences/?q=A000924&sort=0&fmt=0&language=english&go=Search}.

\begin{thebibliography}{9}
\bibitem{BS}{\sc S. Borewicz \& I. Safarevic}: {\em Zahlentheorie}.\, Birkh\"auser Verlag. Basel und Stuttgart (1966).
\end{thebibliography}

%%%%%
%%%%%
\end{document}
