\documentclass[12pt]{article}
\usepackage{pmmeta}
\pmcanonicalname{RationalAndIrrational}
\pmcreated{2013-03-22 14:58:33}
\pmmodified{2013-03-22 14:58:33}
\pmowner{pahio}{2872}
\pmmodifier{pahio}{2872}
\pmtitle{rational and irrational}
\pmrecord{7}{36677}
\pmprivacy{1}
\pmauthor{pahio}{2872}
\pmtype{Result}
\pmcomment{trigger rebuild}
\pmclassification{msc}{11J72}
\pmclassification{msc}{11J82}
\pmrelated{ExamplesOfPeriodicFunctions}
\pmrelated{CommensurableNumbers}
\pmrelated{SolutionsOfXyYx}

% this is the default PlanetMath preamble.  as your knowledge
% of TeX increases, you will probably want to edit this, but
% it should be fine as is for beginners.

% almost certainly you want these
\usepackage{amssymb}
\usepackage{amsmath}
\usepackage{amsfonts}

% used for TeXing text within eps files
%\usepackage{psfrag}
% need this for including graphics (\includegraphics)
%\usepackage{graphicx}
% for neatly defining theorems and propositions
%\usepackage{amsthm}
% making logically defined graphics
%%%\usepackage{xypic}

% there are many more packages, add them here as you need them

% define commands here
\begin{document}
The sum, difference, \PMlinkname{product}{Ring} and quotient of two non-zero real numbers, from which one is rational and the other irrational, is irrational.

{\em Proof.} \,Let $a$ be a rational and $\alpha$ irrational number. \,Here we prove only that $\frac{a}{\alpha}$ is irrational --- the other cases are similar. \,If \,$\frac{a}{\alpha}$ were a rational number \,$r \neq 0$, \,then also \,$\alpha = ar^{-1}$\, would be rational as a product of two rationals. \,This contradiction shows that $\frac{a}{\alpha}$ is irrational.

\textbf{Note.} \,In the result, the words real, rational and irrational may be replaced resp. by the words complex, algebraic and transcendental or resp. by the words complex, real and \PMlinkescapetext{imaginary} (the last \PMlinkescapetext{term} here meaning, as commonly in Continental Europe, a complex number having non-zero imaginary part).
%%%%%
%%%%%
\end{document}
