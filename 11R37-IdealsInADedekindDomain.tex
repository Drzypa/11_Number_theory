\documentclass[12pt]{article}
\usepackage{pmmeta}
\pmcanonicalname{IdealsInADedekindDomain}
\pmcreated{2013-03-22 12:49:50}
\pmmodified{2013-03-22 12:49:50}
\pmowner{yark}{2760}
\pmmodifier{yark}{2760}
\pmtitle{ideals in a Dedekind domain}
\pmrecord{9}{33154}
\pmprivacy{1}
\pmauthor{yark}{2760}
\pmtype{Theorem}
\pmcomment{trigger rebuild}
\pmclassification{msc}{11R37}
\pmclassification{msc}{11R04}
\pmrelated{DivisorAsFactorOfPrincipalDivisor}
\pmrelated{FundamentalTheoremOfIdealTheory}

\usepackage{amssymb}
\usepackage{amsmath}
\usepackage{amsfonts}
\begin{document}
Let $R$ be a Dedekind domain,
and let $\mathfrak{a}$ and $\mathfrak{b}$ be ideals of $R$.
Then there is an element $\omega$ and an ideal $\mathfrak{c}$ of $R$ such that
$$\mathfrak{ac} = (\omega)$$
and
$$\mathfrak{b+c} = R.$$

This result was proved by Steinitz in 1911.
%%%%%
%%%%%
\end{document}
