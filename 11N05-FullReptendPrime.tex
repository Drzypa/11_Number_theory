\documentclass[12pt]{article}
\usepackage{pmmeta}
\pmcanonicalname{FullReptendPrime}
\pmcreated{2013-03-22 16:04:50}
\pmmodified{2013-03-22 16:04:50}
\pmowner{PrimeFan}{13766}
\pmmodifier{PrimeFan}{13766}
\pmtitle{full reptend prime}
\pmrecord{5}{38139}
\pmprivacy{1}
\pmauthor{PrimeFan}{13766}
\pmtype{Definition}
\pmcomment{trigger rebuild}
\pmclassification{msc}{11N05}
\pmsynonym{long prime}{FullReptendPrime}

\endmetadata

% this is the default PlanetMath preamble.  as your knowledge
% of TeX increases, you will probably want to edit this, but
% it should be fine as is for beginners.

% almost certainly you want these
\usepackage{amssymb}
\usepackage{amsmath}
\usepackage{amsfonts}

% used for TeXing text within eps files
%\usepackage{psfrag}
% need this for including graphics (\includegraphics)
%\usepackage{graphicx}
% for neatly defining theorems and propositions
%\usepackage{amsthm}
% making logically defined graphics
%%%\usepackage{xypic}

% there are many more packages, add them here as you need them

% define commands here

\begin{document}
If for a prime number $p$ in a given base $b$ such that $\gcd(p, b) = 1$, the formula $$\frac{b^{p - 1} - 1}{p}$$ gives a cyclic number, then $p$ is a {\em full reptend prime} or {\em long prime}.

The first few base 10 full reptend primes are given by A001913 of Sloane's OEIS: 7, 17, 19, 23, 29, 47, 59, 61, 97, 109, 113, 131, 149, 167.

For example, the case $b = 10$, $p = 7$ gives the cyclic number 142857, thus, 7 is a full reptend prime.

Not all values of $p$ will yield a cyclic number using this formula; for example $p = 13$ gives 076923076923. These failed cases will always contain a repetition of digits (possibly several).

The known pattern to this sequence comes from algebraic number theory, specifically, this sequence is the set of primes $p$ such that 10 is a primitive root modulo $p$. A conjecture of Emil Artin on primitive roots is that this sequence contains about 37 percent of the primes.

The term {\em long prime} was used by John Conway and Richard Guy in their {\it Book of Numbers}. Confusingly, Sloane's OEIS refers to these primes as "cyclic numbers."
%%%%%
%%%%%
\end{document}
