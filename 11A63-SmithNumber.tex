\documentclass[12pt]{article}
\usepackage{pmmeta}
\pmcanonicalname{SmithNumber}
\pmcreated{2013-03-22 16:19:32}
\pmmodified{2013-03-22 16:19:32}
\pmowner{CompositeFan}{12809}
\pmmodifier{CompositeFan}{12809}
\pmtitle{Smith number}
\pmrecord{4}{38454}
\pmprivacy{1}
\pmauthor{CompositeFan}{12809}
\pmtype{Definition}
\pmcomment{trigger rebuild}
\pmclassification{msc}{11A63}

% this is the default PlanetMath preamble.  as your knowledge
% of TeX increases, you will probably want to edit this, but
% it should be fine as is for beginners.

% almost certainly you want these
\usepackage{amssymb}
\usepackage{amsmath}
\usepackage{amsfonts}

% used for TeXing text within eps files
%\usepackage{psfrag}
% need this for including graphics (\includegraphics)
%\usepackage{graphicx}
% for neatly defining theorems and propositions
%\usepackage{amsthm}
% making logically defined graphics
%%%\usepackage{xypic}

% there are many more packages, add them here as you need them

% define commands here

\begin{document}
A {\em Smith number} is a number which in a given base, the sum of its digits is equal to the sum of the digits in its factorization. (In the case of numbers that are not square-free, the factorization is written without exponents, writing the repeated factor as many times as needed). For example, 202 is a Smith number, since 2 + 0 + 2 = 4, and its factorization is $2 \times 101$, and 2 + 1 + 0 + 1 = 4.

Prime numbers are not considered, since it is obvious that all of them satisfy the condition given above.

In base 10, the first few Smith numbers are

4, 22, 27, 58, 85, 94, 121, 166, 202, 265, 274, 319, 346, 355, 378, 382, 391, 438, 454, 483 (sequence A006753 in the OEIS).

There are infinitely many Smith numbers that are also palindromic numbers.

Smith numbers were named by Albert Wilansky of Lehigh University for his brother-in-law Harold Smith whose phone number (493-7775) was the first noticed Smith number.

Smith numbers can be constructed from factored repunits. The largest known Smith number is (as of 2005) $9 \times R_{1031} \times (10^{4594} + 3 \times 10^{2297} + 1)^{1476} \times 10^{3913210}$ where $R_{1031} = 2^{1032} - 1$.
%%%%%
%%%%%
\end{document}
