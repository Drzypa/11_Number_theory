\documentclass[12pt]{article}
\usepackage{pmmeta}
\pmcanonicalname{ExtractingEveryNmathrmthTermOfASeries}
\pmcreated{2013-03-22 16:23:34}
\pmmodified{2013-03-22 16:23:34}
\pmowner{rm50}{10146}
\pmmodifier{rm50}{10146}
\pmtitle{extracting every $n^\mathrm{th}$ term of a series}
\pmrecord{10}{38538}
\pmprivacy{1}
\pmauthor{rm50}{10146}
\pmtype{Theorem}
\pmcomment{trigger rebuild}
\pmclassification{msc}{11-00}

% this is the default PlanetMath preamble.  as your knowledge
% of TeX increases, you will probably want to edit this, but
% it should be fine as is for beginners.

% almost certainly you want these
\usepackage{amssymb}
\usepackage{amsmath}
\usepackage{amsfonts}

% used for TeXing text within eps files
%\usepackage{psfrag}
% need this for including graphics (\includegraphics)
%\usepackage{graphicx}
% for neatly defining theorems and propositions
%\usepackage{amsthm}
% making logically defined graphics
%%%\usepackage{xypic}

% there are many more packages, add them here as you need them

% define commands here

\begin{document}
Roots of unity can be used to extract every $n^\mathrm{th}$ term of a series. This method is due to Simpson [1759]. 

\textbf{Theorem.} Let $\omega=e^{2\pi i/k}$ be a primitive $k^\mathrm{th}$ root of unity. If $f(x)=\sum_{j=0}^{\infty} a_j x^j$ and $n\not\equiv 0\pmod k$, then
\[\sum_{j=0}^{\infty} a_{kj+n}x^{kj+n}=\frac{1}{k}\sum_{j=0}^{k-1}\omega^{-jn}f(\omega^{j}x)\]

\textbf{Proof.}
This is a consequence of the fact that $\sum_{j=0}^{k-1}\omega^{jm}=0$ for $m\not\equiv 0\pmod k$.

Consider the term involving $x^r$ on the right-hand side. It is
\[\frac{1}{k}\sum_{j=0}^{k-1}\omega^{-jn}a_r\omega^{jr}x^r=\frac{1}{k}a_rx^r\sum_{j=0}^{k-1}\omega^{j(r-n)}\]
If $r\not\equiv n\pmod k$, the sum is zero. So the term involving $x^r$ is zero unless $r\equiv n\pmod k$, in which case it is $a_rx^r$ since each element of the sum is $1$.

Note that this method is a generalization of the commonly known trick for extracting alternate terms of a series:
\[\frac{1}{2}(f(x)-f(-x))\]
produces the odd terms of $f$.
%%%%%
%%%%%
\end{document}
