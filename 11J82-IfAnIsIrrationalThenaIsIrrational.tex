\documentclass[12pt]{article}
\usepackage{pmmeta}
\pmcanonicalname{IfAnIsIrrationalThenaIsIrrational}
\pmcreated{2013-03-22 14:18:50}
\pmmodified{2013-03-22 14:18:50}
\pmowner{Wkbj79}{1863}
\pmmodifier{Wkbj79}{1863}
\pmtitle{if $a^n$ is irrational then ${a}$ is irrational}
\pmrecord{13}{35779}
\pmprivacy{1}
\pmauthor{Wkbj79}{1863}
\pmtype{Theorem}
\pmcomment{trigger rebuild}
\pmclassification{msc}{11J82}
\pmclassification{msc}{11J72}

\usepackage{amssymb}
\usepackage{amsmath}
\usepackage{amsfonts}
\usepackage{amsthm}

\newtheorem*{thm*}{Theorem}
\begin{document}
\begin{thm*}
If $a$ be a real number and $n$ is an integer such that $a^n$ is irrational, then $a$ is irrational.
\end{thm*}

\begin{proof}
We show this by way of contrapositive.  In other words, we show that, if $a$ is rational, then $a^n$ is rational.

Let $a$ be rational.  Then there exist integers $b$ and $c$ with $c\neq 0$ such that $\displaystyle a=\frac{b}{c}$.  Thus, $\displaystyle a^n=\frac{b^n}{c^n}$, which is a rational number.
\end{proof}

Note that the converse is not true.  For example, $\sqrt{2}$ is irrational and $\left(\sqrt{2}\right)^2=2$ is rational.
%%%%%
%%%%%
\end{document}
