\documentclass[12pt]{article}
\usepackage{pmmeta}
\pmcanonicalname{Divisibility}
\pmcreated{2013-03-22 11:59:49}
\pmmodified{2013-03-22 11:59:49}
\pmowner{mathcam}{2727}
\pmmodifier{mathcam}{2727}
\pmtitle{divisibility}
\pmrecord{11}{30923}
\pmprivacy{1}
\pmauthor{mathcam}{2727}
\pmtype{Definition}
\pmcomment{trigger rebuild}
\pmclassification{msc}{11A51}
\pmsynonym{divides}{Divisibility}
\pmsynonym{divisor}{Divisibility}
\pmsynonym{factor}{Divisibility}
\pmsynonym{multiple}{Divisibility}
\pmrelated{LeastCommonMultiple}
\pmrelated{ExampleOfGcd}
\pmrelated{TauFunction}
\pmrelated{ExactlyDivides}
\pmrelated{DivisorSumOfAnArithmeticFunction}
\pmrelated{StrictDivisibility}
\pmrelated{FundamentalTheoremOfArithmetic}
\pmrelated{NumberTheory}

\usepackage{amssymb}
\usepackage{amsmath}
\usepackage{amsfonts}
\usepackage{graphicx}
%%%\usepackage{xypic}
\begin{document}
Given integers $a$ and $b$, then we say $a$ \emph{divides} $b$ if and only if there is some $q \in \mathbb{Z}$ such that $b=qa$.

There are many other ways in common use to express this relationship:

\begin{itemize}
\item $a\mid b$ (read ``$a$ divides $b$'').
\item $b$ is divisible by $a$.
\item $a$ is a factor of $b$.
\item $a$ is a divisor of $b$.
\item $b$ is a \emph{multiple} of $a$.
\end{itemize}

The notion of divisibility can apply to other rings (e.g., polynomials).
%%%%%
%%%%%
%%%%%
\end{document}
