\documentclass[12pt]{article}
\usepackage{pmmeta}
\pmcanonicalname{ClassicalGroups}
\pmcreated{2013-03-22 15:50:13}
\pmmodified{2013-03-22 15:50:13}
\pmowner{Algeboy}{12884}
\pmmodifier{Algeboy}{12884}
\pmtitle{classical groups}
\pmrecord{23}{37812}
\pmprivacy{1}
\pmauthor{Algeboy}{12884}
\pmtype{Definition}
\pmcomment{trigger rebuild}
\pmclassification{msc}{11E57}
\pmsynonym{linear algebraic groups}{ClassicalGroups}
\pmrelated{SemilinearTransformation}
\pmrelated{PolaritiesAndForms}
\pmrelated{SesquilinearFormsOverGeneralFields}
\pmdefines{classical group}
\pmdefines{isometry}

\endmetadata

\usepackage{latexsym}
\usepackage{amssymb}
\usepackage{amsmath}
\usepackage{amsfonts}
\usepackage{amsthm}

%%\usepackage{xypic}

%-----------------------------------------------------

%       Standard theoremlike environments.

%       Stolen directly from AMSLaTeX sample

%-----------------------------------------------------

%% \theoremstyle{plain} %% This is the default

\newtheorem{thm}{Theorem}

\newtheorem{coro}[thm]{Corollary}

\newtheorem{lem}[thm]{Lemma}

\newtheorem{lemma}[thm]{Lemma}

\newtheorem{prop}[thm]{Proposition}

\newtheorem{conjecture}[thm]{Conjecture}

\newtheorem{conj}[thm]{Conjecture}

\newtheorem{defn}[thm]{Definition}

\newtheorem{remark}[thm]{Remark}

\newtheorem{ex}[thm]{Example}



%\countstyle[equation]{thm}



%--------------------------------------------------

%       Item references.

%--------------------------------------------------


\newcommand{\exref}[1]{Example-\ref{#1}}

\newcommand{\thmref}[1]{Theorem-\ref{#1}}

\newcommand{\defref}[1]{Definition-\ref{#1}}

\newcommand{\eqnref}[1]{(\ref{#1})}

\newcommand{\secref}[1]{Section-\ref{#1}}

\newcommand{\lemref}[1]{Lemma-\ref{#1}}

\newcommand{\propref}[1]{Prop\-o\-si\-tion-\ref{#1}}

\newcommand{\corref}[1]{Cor\-ol\-lary-\ref{#1}}

\newcommand{\figref}[1]{Fig\-ure-\ref{#1}}

\newcommand{\conjref}[1]{Conjecture-\ref{#1}}


% Normal subgroup or equal.

\providecommand{\normaleq}{\unlhd}

% Normal subgroup.

\providecommand{\normal}{\lhd}

\providecommand{\rnormal}{\rhd}
% Divides, does not divide.

\providecommand{\divides}{\mid}

\providecommand{\ndivides}{\nmid}


\providecommand{\union}{\cup}

\providecommand{\bigunion}{\bigcup}

\providecommand{\intersect}{\cap}

\providecommand{\bigintersect}{\bigcap}
\begin{document}
\section{Classical Groups}

It is commonplace to express the classical groups with explicit matrices; however, the theory and classification of classical groups can benefit from a basis free consideration.

Given a finite dimensional vector space $V$ over any field $k$, the set of all linear transformations on $V$ is denoted $GL(V)$ and called the \emph{general linear group}.  We now define several significant related groups of $GL(V)$.

\begin{thm}[Birkhoff-von Neumann]\label{thm:class}
Given a reflexive non-degenerate sesquilinear form $b:V\times V\rightarrow k$, then 
up to a constant $b$ is one of the following:
\begin{itemize}
\item Alternating: that is $b(v,v)=0$.
\item Hermitian: so $b(v,w)=b(w,v)^{\sigma}$ where $\sigma$ is a field
automorphism of $k$ of order $2$.
\item Symmetric: so $b(v,w)=b(w,v)$.
\end{itemize}
\end{thm}
(Refer to \cite[Theorem 7.1]{Taylor} and \cite[Chapter V]{GW}.)
We prefer the definition $b(v,v)=0$ over $b(v,w)=-b(w,v)$ so that we can accommodate the fields of characteristic 2.  In all other characteristics these two properties are equivalent.

In keeping with tradition for group theory, we let $GL(V)$ act on the vector space $V$ on the right hand side.  This means given $v\in V$ and $f\in GL(V)$, 
$vf$ corresponds the vector in $V$ which $f$ sends $v$ to.  If one thinks of $f$ as a matrix this requires $v$ to be a row vector.  It is also common to consider $f$ as a function and use the notation $f(v)$.   

\begin{defn}
Given a reflexive non-degenerate sesquilinear form $b:V\times V\rightarrow k$
we define
\[Isom(b)=\{f\in GL(V):b(vf,wf)=b(v,w), v,w\in V\}.\]
This is called the isometry group of $b$ in $GL(V)$.
\end{defn}

\begin{prop}
$Isom(b)$ is a subgroup of $GL(V)$.
\end{prop}
\begin{proof}
Given $f,g\in Isom(b)$, $v,w\in V$ then 
\[ b(v(fg),w(fg))=b((vf)g,(wf)g)=b(vf,wf)=b(v,w).\]
Hence $fg\in Isom(b)$.  Clearly $1\in Isom(b)$ as well.  Finally, 
\[ b(vf^{-1},wf^{-1})=b((vf^{-1})f,(wf^{-1})f)=b(v,w).\]
So $f^{-1}\in Isom(b)$ and $Isom(b)$ is a subgroup of $GL(V)$.
\end{proof}


 Now if we return to Theorem \ref{thm:class} we find that there are only three isometry group types, as there are only three types of reflexive non-degenerate sesquilinear forms.  These receive the well-known names:
\begin{itemize}
\item Symplectic group $Sp(V, b)$ if $b$ is alternating.
\item Unitary group $U(V,b)$ if $b$ is hermitian.
\item Orthogonal group $O(V,b)$ if $b$ is symmetric.
\end{itemize}

A vector space $V$ equipped with a reflexive non-degenerated sesquilinear form $b$ is also given the designation symplectic, unitary, and orthogonal based on the classification of the form.

Because symplectic spaces have a standard hyperbolic basis it follows every symplectic group over a vector space of the same dimension is isometric, meaning isomorphic as vector spaces but with an isomorphism which respects the forms.  Thus we can write $Sp(V)$ instead of $Sp(V,b)$.  For unitary and orthogonal groups more care is required.

\begin{defn}
A \emph{classical group} is any one of the family of groups derived from these three and the general linear group.  
\end{defn}

\section{Classical Groups as Matrices}

When expressing these groups with matrices it becomes necessary to establish the bilinear forms with matrices.  Given any $n\times n$-matrix $B$ over some field $k$, and row vectors $v,w\in k^n$ we have a reflexive bilinear form defined by
\[ b(v,w)=vBw^t.\]
Whence $b$ is non-degenerate if and only if $\det B\neq 0$.

The most common example is the identity matrix $B=I_n$.  For then
\[b(v,w)=vw^t\]
is the usual dot product, only perhaps without the positive definite axiom which makes sense only for ordered fields like the rationals $\mathbb{Q}$ and reals 
$\mathbb{R}$.

The isometry group of $I$ is nothing more than the invertible matrices $A$ where
\[AIA^t=I;\qquad AA^t=I.\]
Thus it is common for $O(n)$ to denote the orthogonal group over $\mathbb{R}$ and be given by
\[ O(n)=\{A\in GL(n):AA^t=I\}.\]

For symplectic groups the form is the typical $J=\begin{bmatrix} 0 & I\\ -I & 0\end{bmatrix}$ matrix found in the definition of symplectic matrices.
Hence the isometry condition for an alternating form
\[b(vA,wA)=vAJA^t w^t=vJw^t=b(v,w)\]
show that $AJA^t=J$.  Thus it is common to define 
\[Sp(2m)=\{A\in GL(2m): AJA^t=J\}.\]
Thus the symplectic matrices form a group of isometries.


\section{Special subgroups}

The commutator of $GL(V)$ is the special linear group $SL(V)$ composed of all invertible linear transformations of determinant 1.  Given a reflexive non-degenerate sesquilinear form $b$ on $V$, we can create the groups 
\[SIsom(V,b)=Isom(V,b)\intersect SL(V).\]
These get the names
\[Sp(V),\quad SU(V,b), \textnormal{ and } SO(V,b).\]
Notice that $Sp(V)\leq SL(V)$ so we do not require a new name.

\section{Projective groups}

The projective geometry of a vector space $V$, denoted $PG(V)$ is its lattice of subspaces.  Clearly invertible linear maps act on the projective geometry because they send points (1-dimensional subspaces) to points, and lines (2-dimensional subspaces) to lines, and in general $m$-dimensional subspaces to other $m$-dimensional subspaces.  

However, the scalar transformations, i.e.: those $f\in GL(V)$ such that $vf=v\lambda$ for some fixed $\lambda\in k$, do not move any of the subspaces of $V$ -- they fix $PG(V)$.  Therefore when we consider the action of $GL(V)$ on $PG(V)$ we factor out the kernel of the action -- that is the scalar transforms (matrices.)  We denote this group by $PGL(V)$.  Because scalar matrices commute with all other, and not other matrices do, we notice this is the same as factoring by the center $Z(GL(V))$.

Immediately this gives rise the projective versions of each of the classical groups: Let $Z=Z(GL(V))$ -- the set of scalar transformations (a group isomorphic to $k^\times$.)
\begin{itemize}
\item $PGL(V)=GL(V)/Z$, $PSL(V)=SL(V)/(Z\intersect SL(V))$
\item $PSp(V)=Sp(V)/(Z\intersect Sp(V))$
\item $PU(V)=U(V,b)/(Z\intersect U(V,b))$, $PSU(V,b)=SU(V,b)/(Z\intersect SU(V,b))$
\item $PO(V)=O(V,b)/(Z\intersect O(V,b))$, $PSO(V,b)=SO(V,b)/(Z\intersect SO(V,b))$
\end{itemize}

Most of the time the projective special isometry groups are simple groups.
The exceptions arise for small dimensional vector spaces and/or small fields, or with the orthogonal groups.


\bibliographystyle{amsplain}
\providecommand{\bysame}{\leavevmode\hbox to3em{\hrulefill}\thinspace}
\providecommand{\MR}{\relax\ifhmode\unskip\space\fi MR }
% \MRhref is called by the amsart/book/proc definition of \MR.
\providecommand{\MRhref}[2]{%
\href{http://www.ams.org/mathscinet-getitem?mr=#1}{#2}
}
\providecommand{\href}[2]{#2}
\begin{thebibliography}{10}


\bibitem{GW}
Gruenberg, K. W. and Weir, A.J.
\emph{Linear Geometry 2nd Ed.} (English)
[B] Graduate Texts in Mathematics. 49. New York - Heidelberg - Berlin: Springer-Verlag. (1977), pp. x-198.

\bibitem{Ka}
Kantor, W. M.
\emph{Lectures notes on Classical Groups}.

\bibitem{Taylor}
Taylor, Donald E.
\emph{The geometry of the classical groups}
Sigma Series in Pure Mathematics. 9.
Heldermann Verlag, Berlin, xii+229, (1992), ISBN 3-88538-009-9.

\end{thebibliography}


%%%%%
%%%%%
\end{document}
