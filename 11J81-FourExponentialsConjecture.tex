\documentclass[12pt]{article}
\usepackage{pmmeta}
\pmcanonicalname{FourExponentialsConjecture}
\pmcreated{2013-03-22 13:40:51}
\pmmodified{2013-03-22 13:40:51}
\pmowner{Kevin OBryant}{1315}
\pmmodifier{Kevin OBryant}{1315}
\pmtitle{four exponentials conjecture}
\pmrecord{7}{34349}
\pmprivacy{1}
\pmauthor{Kevin OBryant}{1315}
\pmtype{Conjecture}
\pmcomment{trigger rebuild}
\pmclassification{msc}{11J81}
\pmsynonym{4 exponentials}{FourExponentialsConjecture}
%\pmkeywords{transcendental numbers}
\pmrelated{sixexponentialstheorem}
\pmrelated{exponentialfunction}
\pmrelated{ExponentialFunction}
\pmrelated{SixExponentialsTheorem}

    \usepackage{amsmath,amssymb,amsthm}
    \newenvironment{namedtheorem}[1]{\medskip \noindent {\bf #1: }\begin{em}}{\end{em}\medskip}
\begin{document}
\begin{namedtheorem}{\PMlinkescapetext{Four exponentials conjecture}}
Given four complex numbers $x_1,x_2,y_1,y_2$, either $x_1/x_2$ or $y_1/y_2$ is rational, or one of the four
numbers $\exp(x_i y_j)$ is transcendental.
\end{namedtheorem}

This conjecture is stronger than the six exponentials theorem.

\begin{thebibliography}{9}
    \bibitem[1]{Wal2000} Waldschmidt, Michel, {\em Diophantine approximation on linear algebraic groups.
        Transcendence
        properties of the exponential function in several variables}. Grundlehren der Mathematischen
        Wissenschaften
        [Fundamental Principles of Mathematical Sciences], 326. Springer-Verlag, Berlin, 2000. xxiv+633 pp.
        ISBN 3-540-66785-7.
\end{thebibliography}
%%%%%
%%%%%
\end{document}
