\documentclass[12pt]{article}
\usepackage{pmmeta}
\pmcanonicalname{PerrinPseudoprime}
\pmcreated{2013-03-22 16:11:33}
\pmmodified{2013-03-22 16:11:33}
\pmowner{CompositeFan}{12809}
\pmmodifier{CompositeFan}{12809}
\pmtitle{Perrin pseudoprime}
\pmrecord{5}{38284}
\pmprivacy{1}
\pmauthor{CompositeFan}{12809}
\pmtype{Definition}
\pmcomment{trigger rebuild}
\pmclassification{msc}{11B39}
\pmdefines{strong Perrin pseudoprime}

\endmetadata

% this is the default PlanetMath preamble.  as your knowledge
% of TeX increases, you will probably want to edit this, but
% it should be fine as is for beginners.

% almost certainly you want these
\usepackage{amssymb}
\usepackage{amsmath}
\usepackage{amsfonts}

% used for TeXing text within eps files
%\usepackage{psfrag}
% need this for including graphics (\includegraphics)
%\usepackage{graphicx}
% for neatly defining theorems and propositions
%\usepackage{amsthm}
% making logically defined graphics
%%%\usepackage{xypic}

% there are many more packages, add them here as you need them

% define commands here

\begin{document}
Given the Perrin sequence, $a_0 = 3$, $a_1 = 0$, $a_2 = 2$ and $a_n = a_{n - 3} + a_{n - 2}$ for $n > 2$, if $n$ is not prime yet $n|a_n$, that number is then called a {\em Perrin pseudoprime}. The first few Perrin pseudoprimes are 271441, 904631, 16532714, 24658561, 27422714, 27664033, 46672291 (listed in A013998 of Sloane's OEIS).

By itself this is not a good enough test of primality. Further requiring that $a_n \equiv 0 \mod n$ and $a_{-n} \equiv 1 \mod n$ improves the test but not enough to be preferable to actually factoring out the number. The first few {\em strong Perrin pseudoprimes} are 27664033, 46672291, 102690901, 130944133, 517697641 (listed in A018187 of Sloane's OEIS).

Unlike Perrin pseudoprimes, most other \PMlinkname{pseudoprimes}{PseudoprimeP} are \PMlinkescapetext{pseudoprimes} because of a congruence relation to a given base.
%%%%%
%%%%%
\end{document}
