\documentclass[12pt]{article}
\usepackage{pmmeta}
\pmcanonicalname{EverySufficientlyLargeEvenIntegerCanBeExpressedAsTheSumOfAPairOfAbundantNumbers}
\pmcreated{2013-03-22 16:46:58}
\pmmodified{2013-03-22 16:46:58}
\pmowner{rspuzio}{6075}
\pmmodifier{rspuzio}{6075}
\pmtitle{every sufficiently large even integer can be expressed as the sum of a pair of abundant numbers}
\pmrecord{9}{39014}
\pmprivacy{1}
\pmauthor{rspuzio}{6075}
\pmtype{Proof}
\pmcomment{trigger rebuild}
\pmclassification{msc}{11A05}

% this is the default PlanetMath preamble.  as your knowledge
% of TeX increases, you will probably want to edit this, but
% it should be fine as is for beginners.

% almost certainly you want these
\usepackage{amssymb}
\usepackage{amsmath}
\usepackage{amsfonts}

% used for TeXing text within eps files
%\usepackage{psfrag}
% need this for including graphics (\includegraphics)
%\usepackage{graphicx}
% for neatly defining theorems and propositions
\usepackage{amsthm}
% making logically defined graphics
%%%\usepackage{xypic}

% there are many more packages, add them here as you need them

% define commands here
\newtheorem{thm}{Theorem}
\begin{document}
\begin{thm}
If $n > 1540539$, then $n = a + b$, where $a$ and
$b$ are abundant numbers.
\end{thm}

\begin{proof}
Note that both $20$ and $81081$ are abundant numbers.
Furthermore, we have $81081 = 4054 \cdot 20 + 1$.
If $n$ is a multiple of $20$, then $n-20$ is also 
a multiple of $20$ hence, as a multiple of an abundant
number, is also abundant, so we may choose $a = 20$ 
and $b = n-20$.  Otherwise, write $n = 20 m + k$ where
$m$ and $k$ are positive and $k < 20$.  Note that,
since $n > 1540539$ and $k < 20$, it follows that
$m > 77026 > 4054 k$, hence we have
\[
n = 20 (m - 4054 k) + 81081 k.
\]  
Since positive multiples of abundant numbers are 
abundant, we may set $a = 20 (m - 4054 k)$ and
$b = 81081 k$.
\end{proof}

%%%%%
%%%%%
\end{document}
