\documentclass[12pt]{article}
\usepackage{pmmeta}
\pmcanonicalname{ComplexMultiplication}
\pmcreated{2013-03-22 13:41:35}
\pmmodified{2013-03-22 13:41:35}
\pmowner{alozano}{2414}
\pmmodifier{alozano}{2414}
\pmtitle{complex multiplication}
\pmrecord{15}{34367}
\pmprivacy{1}
\pmauthor{alozano}{2414}
\pmtype{Definition}
\pmcomment{trigger rebuild}
\pmclassification{msc}{11G05}
%\pmkeywords{complex multiplication}
%\pmkeywords{elliptic curve}
%\pmkeywords{endomorphism}
\pmrelated{EllipticCurve}
\pmrelated{KroneckerWeberTheorem}
\pmrelated{OrderInAnAlgebra}
\pmrelated{ArithmeticOfEllipticCurves}
\pmdefines{complex multiplication}
\pmdefines{endomorphism ring}

\endmetadata

% this is the default PlanetMath preamble.  as your knowledge
% of TeX increases, you will probably want to edit this, but
% it should be fine as is for beginners.

% almost certainly you want these
\usepackage{amssymb}
\usepackage{amsmath}
\usepackage{amsfonts}

\newtheorem{thm}{Theorem}
\newtheorem{defn}{Definition}
\newtheorem{prop}{Proposition}
\newtheorem{lemma}{Lemma}
\newtheorem{cor}{Corollary}

% used for TeXing text within eps files
%\usepackage{psfrag}
% need this for including graphics (\includegraphics)
%\usepackage{graphicx}
% for neatly defining theorems and propositions
%\usepackage{amsthm}
% making logically defined graphics
%%%\usepackage{xypic}

% there are many more packages, add them here as you need them

% define commands here
\begin{document}
Let $E$ be an elliptic curve. The {\it endomorphism ring} of $E$,
denoted $\operatorname{End}(E)$, is the set of all regular maps $\phi \colon E
\to E$ such that $\phi(O)=O$, where $O \in E$ is the
identity element for the group structure of $E$. Note that this is
indeed a ring under addition ($(\phi + \psi)(P)=\phi(P) +
\psi(P)$) and composition of maps.

The following theorem implies that every endomorphism is also a
group endomorphism:
\begin{thm}
Let $E_1, E_2$ be elliptic curves, and let $\phi \colon E_1
\to E_2$ be a regular map such that
$\phi(O_{E_1})=O_{E_2}$. Then $\phi$ is also a group homomorphism,
i.e. $$\forall P,Q \in E_1,\ \phi(P +_{E_1}
Q)=\phi(P)+_{E_2}\phi(Q).$$
\end{thm}
[Proof: See $\cite{silverman}$, Theorem 4.8, page 75]
\\

If $\operatorname{End}(E)$ is isomorphic (as a ring) to an \PMlinkname{order}{OrderInAnAlgebra} $R$ in a quadratic imaginary
field $K$ then we say that the elliptic curve E has {\it complex
multiplication} by $K$ (or complex multiplication by $R$).

{\it Note}: $\operatorname{End}(E)$ always contains a subring isomorphic to
$\mathbb{Z}$, formed by the {\it multiplication by n} maps:
$$[n]\colon E \to E,\quad [n]P=n\cdot P$$
and, in general, these are all the maps in the endomorphism ring of $E$.

{\bf Example}: Fix $d\in \mathbb{Z}$. Let $E$ be the elliptic
curve defined by
$$y^2=x^3-dx$$
then this curve has complex multiplication by $\mathbb{Q}(i)$
(more concretely by $\mathbb{Z}(i)$). Besides the multiplication
by $n$ maps, $\operatorname{End}(E)$ contains a genuine new element:
$$[i]\colon E \to E,\quad [i](x,y)=(-x,iy)$$
(the name {\it complex multiplication} comes from the fact that we
are ``multiplying'' the points in the curve by a complex number, $i$
in this case).

\begin{thebibliography}{8}
\bibitem{milne} James Milne, {\em Elliptic Curves}, online course notes. \PMlinkexternal{http://www.jmilne.org/math/CourseNotes/math679.html}{http://www.jmilne.org/math/CourseNotes/math679.html}
\bibitem{silverman} Joseph H. Silverman, {\em The Arithmetic of Elliptic Curves}. Springer-Verlag, New York, 1986.
\bibitem{silverman2} Joseph H. Silverman, {\em Advanced Topics in
the Arithmetic of Elliptic Curves}. Springer-Verlag, New York,
1994.
\bibitem{shimura} Goro Shimura, {\em Introduction to the
Arithmetic Theory of Automorphic Functions}. Princeton University
Press, Princeton, New Jersey, 1971.
\end{thebibliography}
%%%%%
%%%%%
\end{document}
