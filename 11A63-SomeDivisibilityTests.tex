\documentclass[12pt]{article}
\usepackage{pmmeta}
\pmcanonicalname{SomeDivisibilityTests}
\pmcreated{2014-05-29 11:26:48}
\pmmodified{2014-05-29 11:26:48}
\pmowner{pahio}{2872}
\pmmodifier{pahio}{2872}
\pmtitle{some divisibility tests}
\pmrecord{7}{42286}
\pmprivacy{1}
\pmauthor{pahio}{2872}
\pmtype{Derivation}
\pmcomment{trigger rebuild}
\pmclassification{msc}{11A63}
\pmclassification{msc}{11A07}
%\pmkeywords{digit sum}
%\pmkeywords{divisibility by 3}
%\pmkeywords{divisibily by 9}
%\pmkeywords{divisibility by 11}

% this is the default PlanetMath preamble.  as your knowledge
% of TeX increases, you will probably want to edit this, but
% it should be fine as is for beginners.

% almost certainly you want these
\usepackage{amssymb}
\usepackage{amsmath}
\usepackage{amsfonts}

% used for TeXing text within eps files
%\usepackage{psfrag}
% need this for including graphics (\includegraphics)
%\usepackage{graphicx}
% for neatly defining theorems and propositions
 \usepackage{amsthm}
% making logically defined graphics
%%%\usepackage{xypic}

% there are many more packages, add them here as you need them

% define commands here

\theoremstyle{definition}
\newtheorem*{thmplain}{Theorem}

\begin{document}
We want to derive the divisibility tests for 3, 9, and 11.

Let 
$$A \;=\; a_n\!\cdot\!10^n\!+\!a_{n-1}\!\cdot\!10^{n-1}\!+\ldots+\!a_0$$
be the digital representation of the integer $A$ in the decadic system ($0 \le a_i \le 9$).\\

Since\, $10 \equiv 1 \pmod {3,\,9}$,\, we have also\, $10^i \equiv 1 \pmod {3,\,9}$\, for each $i$,\, and hence
$$a_i\!\cdot\!10^i \equiv a_i \!\pmod {3,\,9}.$$
Consequently
$$A \;=\; \sum_{i=0}^na_i\!\cdot\!10^i \;\equiv\; \sum_{i=0}^na_i \!\pmod{3,\,9}.$$
This congruence means that
\begin{align}
\sum_{i=0}^na_i \;\equiv\; 0 \;\pmod{3,\,9} 
\qquad \Leftrightarrow \qquad A \;\equiv\; 0 \;\pmod{3,\,9}.
\end{align}

The fact\, $10 \equiv -1 \pmod {11}$\, similarly yields
$$A \;=\; \sum_{i=0}^na_i\!\cdot\!10^i \;\equiv\; \sum_{i=0}^n(-1)^ia_i \!\pmod{11},$$
whence
\begin{align}
\sum_{i=0}^n(-1)^ia_i \;\equiv\; 0 \;\pmod{11} 
\qquad \Leftrightarrow \qquad A \;\equiv\; 0 \;\pmod{11}.
\end{align}


The results (1) and (2) may be rendered as the following:

\begin{itemize}
\item $A$ is divisible by 3 (resp. 9) if and only if\, $a_0\!+\!a_1\!+\ldots+\!a_n$\, is.\\
\item $A$ is divisible by 11 if and only if\, $a_0\!-\!a_1\!+-\ldots+\!(-1)^na_n$\, is.
\end{itemize}
%%%%%
%%%%%
\end{document}
