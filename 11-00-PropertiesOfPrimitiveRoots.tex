\documentclass[12pt]{article}
\usepackage{pmmeta}
\pmcanonicalname{PropertiesOfPrimitiveRoots}
\pmcreated{2013-03-22 16:20:47}
\pmmodified{2013-03-22 16:20:47}
\pmowner{alozano}{2414}
\pmmodifier{alozano}{2414}
\pmtitle{properties of primitive roots}
\pmrecord{4}{38479}
\pmprivacy{1}
\pmauthor{alozano}{2414}
\pmtype{Theorem}
\pmcomment{trigger rebuild}
\pmclassification{msc}{11-00}

\endmetadata

% this is the default PlanetMath preamble.  as your knowledge
% of TeX increases, you will probably want to edit this, but
% it should be fine as is for beginners.

% almost certainly you want these
\usepackage{amssymb}
\usepackage{amsmath}
\usepackage{amsthm}
\usepackage{amsfonts}

% used for TeXing text within eps files
%\usepackage{psfrag}
% need this for including graphics (\includegraphics)
%\usepackage{graphicx}
% for neatly defining theorems and propositions
%\usepackage{amsthm}
% making logically defined graphics
%%%\usepackage{xypic}

% there are many more packages, add them here as you need them

% define commands here

\newtheorem*{thm}{Theorem}
\newtheorem*{defn}{Definition}
\newtheorem*{prop}{Proposition}
\newtheorem{lemma}{Lemma}
\newtheorem{cor}{Corollary}

\theoremstyle{definition}
\newtheorem{exa}{Example}

% Some sets
\newcommand{\Nats}{\mathbb{N}}
\newcommand{\Ints}{\mathbb{Z}}
\newcommand{\Reals}{\mathbb{R}}
\newcommand{\Complex}{\mathbb{C}}
\newcommand{\Rats}{\mathbb{Q}}
\newcommand{\Gal}{\operatorname{Gal}}
\newcommand{\Cl}{\operatorname{Cl}}
\begin{document}
\begin{defn}
Let $m>1$ be an integer. An integer $g$ is said to be a primitive root of $m$ if $\gcd(g,m)=1$ and the multiplicative order of $g$ is exactly $\phi(m)$, where $\phi$ is the Euler phi function. In other words, $g^{\phi(m)}\equiv 1 \mod m$ and $g^n\neq 1 \mod m$ for any $n<\phi(m)$.
\end{defn}

\begin{thm}
An integer $m>1$ has a primitive root if and only if $m$ is $2,4,p^k$ or $2p^k$ for some $k\geq 1$. In particular, every prime has a primitive root.
\end{thm}

\begin{prop}
Let $m>1$ be an integer.
\begin{enumerate}
\item If $g$ is a primitive root of $m$ then the set $\{1,g,g^2,\ldots,g^{\phi(m)}\}$ is a complete set of representatives for $(\Ints/m\Ints)^\times$.

\item If $\gcd(g,m)=1$ then $g$ is a primitive root of $m$ if and only if $g^{\phi(m)/q}\neq 1 \mod m$ for every prime divisor $q$ of $\phi(m)$.

\item If $g$ is a primitive root of $m$, then $g^s\equiv g^t \mod m$ if and only if $s\equiv t \mod \phi(m)$. Thus $g^s\equiv 1 \mod m$ if and only if $\phi(m)$ divides $s$.

\item If $g$ is a primitive root of $m$, then $g^k$ is a primitive root of $m$ if and only if $\gcd(k,\phi(m))=1$.

\item If $m$ has a primitive root then $m$ has exactly $\phi(\phi(m))$ incongruent primitive roots.
\end{enumerate}
\end{prop}
%%%%%
%%%%%
\end{document}
