\documentclass[12pt]{article}
\usepackage{pmmeta}
\pmcanonicalname{UnitaryPerfectNumber}
\pmcreated{2013-03-22 16:19:29}
\pmmodified{2013-03-22 16:19:29}
\pmowner{CompositeFan}{12809}
\pmmodifier{CompositeFan}{12809}
\pmtitle{unitary perfect number}
\pmrecord{4}{38453}
\pmprivacy{1}
\pmauthor{CompositeFan}{12809}
\pmtype{Definition}
\pmcomment{trigger rebuild}
\pmclassification{msc}{11A05}

% this is the default PlanetMath preamble.  as your knowledge
% of TeX increases, you will probably want to edit this, but
% it should be fine as is for beginners.

% almost certainly you want these
\usepackage{amssymb}
\usepackage{amsmath}
\usepackage{amsfonts}

% used for TeXing text within eps files
%\usepackage{psfrag}
% need this for including graphics (\includegraphics)
%\usepackage{graphicx}
% for neatly defining theorems and propositions
%\usepackage{amsthm}
% making logically defined graphics
%%%\usepackage{xypic}

% there are many more packages, add them here as you need them

% define commands here

\begin{document}
A {\em unitary perfect number} is an integer which is the sum of its positive proper unitary divisors, not including the number itself. (A divisor $d$ of a number $n$ is a unitary divisor if $d$ and $\frac{n}{d}$ share no common factors.) Some perfect numbers are not unitary perfect numbers, and some unitary perfect numbers are not regular perfect numbers.

The first few unitary perfect numbers are 6, 60, 90, 87360, 146361946186458562560000 (sequence A002827 in the OEIS).

There are no odd unitary perfect numbers. This follows since one has $2d*(n)$ dividing the sum of the unitary divisors of an odd number (where $d*(n)$ is the number of distinct prime divisors of $n$). One gets this because the sum of all the unitary divisors is a multiplicative function and one has the sum of the unitary divisors of a power of a prime $p^a$ is $p^a + 1$ which is even for all odd primes $p$. Therefore, an odd unitary perfect number must have only one distinct prime factor, and it is not hard to show that a power of prime cannot be a unitary perfect number, since there are not enough divisors. It's not known whether or not there are infinitely many unitary perfect numbers.

%%%%%
%%%%%
\end{document}
