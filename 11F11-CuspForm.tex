\documentclass[12pt]{article}
\usepackage{pmmeta}
\pmcanonicalname{CuspForm}
\pmcreated{2013-03-22 14:07:43}
\pmmodified{2013-03-22 14:07:43}
\pmowner{olivierfouquetx}{2421}
\pmmodifier{olivierfouquetx}{2421}
\pmtitle{cusp form}
\pmrecord{9}{35536}
\pmprivacy{1}
\pmauthor{olivierfouquetx}{2421}
\pmtype{Definition}
\pmcomment{trigger rebuild}
\pmclassification{msc}{11F11}

% this is the default PlanetMath preamble.  as your knowledge
% of TeX increases, you will probably want to edit this, but
% it should be fine as is for beginners.

% almost certainly you want these
\usepackage{amssymb}
\usepackage{amsmath}
\usepackage{amsfonts}

% used for TeXing text within eps files
%\usepackage{psfrag}
% need this for including graphics (\includegraphics)
%\usepackage{graphicx}
% for neatly defining theorems and propositions
%\usepackage{amsthm}
% making logically defined graphics
%%%\usepackage{xypic}

% there are many more packages, add them here as you need them

% define commands here
\newcommand{\sldeuxz}{\textrm{SL}_{2}(\mathbb Z)}
\begin{document}
A cusp form is a modular form whose first coefficient in any expansion around a cusp is zero. Another more general way to define cusp forms is to consider the forms orthogonal to Eisenstein series with respect to the Petersson scalar product.

The Weierstrass $\Delta$ function, also called modular discriminant is a weight 12 cusp form for the full modular group $\sldeuxz$
\begin{equation}
\Delta(z)=q\underset{n=1}{\overset{\infty}{\prod}}(1-q^n)^{24}
\end{equation}

The vector space of weight $k$ cusp forms for the full modular group is finite dimensionnal and non-trivial for $k$ integral greater than 12 and not 14.
%%%%%
%%%%%
\end{document}
