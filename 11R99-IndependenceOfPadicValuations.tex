\documentclass[12pt]{article}
\usepackage{pmmeta}
\pmcanonicalname{IndependenceOfPadicValuations}
\pmcreated{2013-03-22 14:12:14}
\pmmodified{2013-03-22 14:12:14}
\pmowner{alozano}{2414}
\pmmodifier{alozano}{2414}
\pmtitle{independence of $p$-adic valuations}
\pmrecord{4}{35636}
\pmprivacy{1}
\pmauthor{alozano}{2414}
\pmtype{Corollary}
\pmcomment{trigger rebuild}
\pmclassification{msc}{11R99}
\pmrelated{Valuation}
\pmrelated{PAdicIntegers}
\pmrelated{PAdicValuation}

% this is the default PlanetMath preamble.  as your knowledge
% of TeX increases, you will probably want to edit this, but
% it should be fine as is for beginners.

% almost certainly you want these
\usepackage{amssymb}
\usepackage{amsmath}
\usepackage{amsthm}
\usepackage{amsfonts}

% used for TeXing text within eps files
%\usepackage{psfrag}
% need this for including graphics (\includegraphics)
%\usepackage{graphicx}
% for neatly defining theorems and propositions
%\usepackage{amsthm}
% making logically defined graphics
%%%\usepackage{xypic}

% there are many more packages, add them here as you need them

% define commands here

\newtheorem{thm}{Theorem}
\newtheorem{defn}{Definition}
\newtheorem{prop}{Proposition}
\newtheorem{lemma}{Lemma}
\newtheorem{cor}{Corollary}

% Some sets
\newcommand{\Nats}{\mathbb{N}}
\newcommand{\Ints}{\mathbb{Z}}
\newcommand{\Reals}{\mathbb{R}}
\newcommand{\Complex}{\mathbb{C}}
\newcommand{\Rats}{\mathbb{Q}}
\begin{document}
We prove the following particular case:

\begin{prop}
Let $p_1,\ldots,p_n \in \Ints$ be distinct prime numbers and let $\mid\cdot\mid_{p_i}$ be the corresponding $p$-adic valuations of $\Rats$. Let $a_1,\ldots,a_n\in \Ints$ and let $\epsilon_i$ be arbitrary positive real numbers, then there exists $y\in\Ints$ such that for all $i=1,\ldots,n$:
$$\mid y - a_i \mid_{p_i} < \epsilon_i$$
\end{prop}
\begin{proof}
Let $p$ be an arbitrary prime, and let $\epsilon$ be an arbitrary positive real number. Notice that $\Ints$ injects into $\Ints_p=\varprojlim \Ints/p^n\Ints$, the $p$-adic integers. For any $b\in \Ints$, we also write $b$ for its image in $\Ints_p$, and it can be written as a sequence $b=(b_j)$ with $b\equiv b_j \mod p^j$. Let $n=n_{p,\epsilon}\in \Nats$ be such that $p^{-n} < \epsilon$ (and thus  for any other $c\in \Ints $ such that $c\equiv b_n \mod p^n$ we have $\mid b-c \mid_p \leq p^{-n} < \epsilon$).

Now, for the proof of the proposition, let $n_i=n_{p_i,\epsilon_i}$ and recall that by the Chinese Remainder Theorem we have an isomorphism:

$$\prod_{i=1}^n \Ints/p_i^{n_i}\Ints \equiv \Ints/(\prod p_i^{n_i})\Ints$$

Therefore we can find an element $\tilde{y}$ of $\Ints/(\prod p_i^{n_i})\Ints$ (and thus a lift $y$ of $\tilde{y}$ to $\Ints$) such that $y \equiv a_i \mod p_i^{n_i}$ for all $i=1,\ldots,n$. Hence:
$$\mid y - a_i \mid_{p_i} < \epsilon_i$$
\end{proof}
%%%%%
%%%%%
\end{document}
