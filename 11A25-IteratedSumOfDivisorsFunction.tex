\documentclass[12pt]{article}
\usepackage{pmmeta}
\pmcanonicalname{IteratedSumOfDivisorsFunction}
\pmcreated{2013-03-22 17:03:36}
\pmmodified{2013-03-22 17:03:36}
\pmowner{PrimeFan}{13766}
\pmmodifier{PrimeFan}{13766}
\pmtitle{iterated sum of divisors function}
\pmrecord{4}{39351}
\pmprivacy{1}
\pmauthor{PrimeFan}{13766}
\pmtype{Definition}
\pmcomment{trigger rebuild}
\pmclassification{msc}{11A25}

% this is the default PlanetMath preamble.  as your knowledge
% of TeX increases, you will probably want to edit this, but
% it should be fine as is for beginners.

% almost certainly you want these
\usepackage{amssymb}
\usepackage{amsmath}
\usepackage{amsfonts}

% used for TeXing text within eps files
%\usepackage{psfrag}
% need this for including graphics (\includegraphics)
%\usepackage{graphicx}
% for neatly defining theorems and propositions
%\usepackage{amsthm}
% making logically defined graphics
%%%\usepackage{xypic}

% there are many more packages, add them here as you need them

% define commands here

\begin{document}
The {\em iterated sum of divisors function} $\sigma^k(n)$ is $a_k$ in the recurrence relation $a_0 = n$ and $a_i = \sigma(a_{i - 1})$ for $i > 0$, where $\sigma(x)$ is the sum of divisors function.

Since $n$ itself is included in the set of its divisors, the sequence generated by repeated iterations is an increasing sequence (that is, in ascending order). For example, iterating the sum of divisors function for $n = 2$ gives the sequence 2, 3, 4, 7, 8, 15, etc. Erd\H{o}s conjectured that there is a limit for $(\sigma^k(n))^{\frac{1}{k}}$ as $k$ approaches infinity.

\begin{thebibliography}{1}
\bibitem{rg} R. K. Guy, {\it Unsolved Problems in Number Theory} New York: Springer-Verlag 2004: B9
\end{thebibliography}

%%%%%
%%%%%
\end{document}
