\documentclass[12pt]{article}
\usepackage{pmmeta}
\pmcanonicalname{SuperincreasingSequence}
\pmcreated{2013-03-22 11:55:22}
\pmmodified{2013-03-22 11:55:22}
\pmowner{Wkbj79}{1863}
\pmmodifier{Wkbj79}{1863}
\pmtitle{superincreasing sequence}
\pmrecord{10}{30623}
\pmprivacy{1}
\pmauthor{Wkbj79}{1863}
\pmtype{Definition}
\pmcomment{trigger rebuild}
\pmclassification{msc}{11B83}
\pmsynonym{superincreasing}{SuperincreasingSequence}
\pmrelated{Superconvergence}

\usepackage{amssymb}
\usepackage{amsmath}
\usepackage{amsfonts}
\usepackage{graphicx}
%%%%\usepackage{xypic}

\begin{document}
A sequence $\{s_j\}$ of real numbers is \emph{superincreasing} if $\displaystyle s_{n+1} > \sum_{j=1}^n s_j$ for every positive integer $n$.  That is, any element of the sequence is greater than all of the previous elements added together.

A commonly used superincreasing sequence is that of powers of two ($s_n = 2^n$.)

Suppose that $\displaystyle x=\sum_{j=1}^n a_j s_j$.  If $\{s_j\}$ is a superincreasing sequence and every $a_j \in \{0,1\}$, then we can always determine the $a_j$'s simply by knowing $x$.  This is analogous to the fact that, for any natural number, we can always determine which bits are on and off in the binary bitstring representing the number.
%%%%%
%%%%%
%%%%%
%%%%%
\end{document}
