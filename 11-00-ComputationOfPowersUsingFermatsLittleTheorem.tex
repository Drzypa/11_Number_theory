\documentclass[12pt]{article}
\usepackage{pmmeta}
\pmcanonicalname{ComputationOfPowersUsingFermatsLittleTheorem}
\pmcreated{2013-03-22 13:17:42}
\pmmodified{2013-03-22 13:17:42}
\pmowner{basseykay}{877}
\pmmodifier{basseykay}{877}
\pmtitle{computation of powers using Fermat's little theorem}
\pmrecord{5}{33794}
\pmprivacy{1}
\pmauthor{basseykay}{877}
\pmtype{Example}
\pmcomment{trigger rebuild}
\pmclassification{msc}{11-00}
%\pmkeywords{number theory}
%\pmkeywords{RSA}
%\pmkeywords{cryptography}

\endmetadata

% this is the default PlanetMath preamble.  as your knowledge
% of TeX increases, you will probably want to edit this, but
% it should be fine as is for beginners.

% almost certainly you want these
\usepackage{amssymb}
\usepackage{amsmath}
\usepackage{amsfonts}

% used for TeXing text within eps files
%\usepackage{psfrag}
% need this for including graphics (\includegraphics)
%\usepackage{graphicx}
% for neatly defining theorems and propositions
%\usepackage{amsthm}
% making logically defined graphics
%%%\usepackage{xypic}

% there are many more packages, add them here as you need them

% define commands here
\newcommand{\remainder}{\: \% \:}
\begin{document}
A straightforward application of Fermat's theorem consists of rewriting the power of an integer mod $n$. Suppose we have $x \equiv a^b \pmod{n}$ with $a \in \mathcal{U}(n)$. Then, by Fermat's theorem we have
\begin{equation*}
a^{\phi(n)}\equiv 1 \pmod{n},
\end{equation*}
so
\begin{equation*}
x \equiv a^b (1)^k \equiv a^b (a^{\phi(n)})^k \equiv a^{b+k\phi(n)} \pmod{n}
\end{equation*}
for any integer $k$. This means we can replace $b$ by any integer congruent to it mod $\phi(n)$. In particular we have
\begin{equation*}
x \equiv a^{b \remainder \phi(n)} \pmod{n}
\end{equation*}
 where $b \remainder \phi(n)$ denotes the remainder of $b$ upon division by $\phi(n)$.

This can be used to make the computation of large powers easier. It also allows one to find an easy to compute inverse to $x^b \pmod{n}$ whenever $b \in \mathcal{U}(n)$. In fact, this is just $x^{b^{-1}}$ where $b^{-1}$ is an inverse to $b$ mod $\phi(n)$. This forms the base of the RSA cryptosystem where a message $x$ is encrypted by raising it to the $b$th power, giving $x^b$, and is decrypted by raising it to the $b^{-1}$th power, giving 
\begin{equation*}
(x^b)^{b^{-1}} \equiv x^{b b^{-1}},
\end{equation*}
which, by the above argument, is just
\begin{equation*}
x^{ b b^{-1} \remainder \phi(n)} \equiv x,
\end{equation*}
the original message!
%%%%%
%%%%%
\end{document}
