\documentclass[12pt]{article}
\usepackage{pmmeta}
\pmcanonicalname{RootOfUnity}
\pmcreated{2014-11-06 15:47:15}
\pmmodified{2014-11-06 15:47:15}
\pmowner{alozano}{2414}
\pmmodifier{pahio}{2872}
\pmtitle{root of unity}
\pmrecord{17}{30351}
\pmprivacy{1}
\pmauthor{alozano}{2872}
\pmtype{Definition}
\pmcomment{trigger rebuild}
\pmclassification{msc}{11-00}
\pmclassification{msc}{11-02}
\pmrelated{CyclotomicPolynomial}
\pmrelated{ExamplesOfCyclotomicPolynomials}
\pmrelated{RamanujanSum}
\pmrelated{Unity}
\pmrelated{CriterionForConstructibilityOfRegularPolygon}
\pmrelated{BinomialEquation}
\pmdefines{primitive $n$th root of unity}
\pmdefines{primitive root of unity}

\usepackage{amssymb}
\usepackage{amsmath}
\usepackage{amsfonts}
\usepackage{graphicx}
%%%%\usepackage{xypic}
\begin{document}
A \emph{root of unity} is a number $\omega$ such that some power
$\omega^n$, where $n$ is a positive integer, equals to $1$.

Specifically, if $K$ is a field, then the $n$th roots of unity in $K$
are the numbers $\omega$ in $K$ such that $\omega^n = 1$.
Equivalently, they are all the roots of the polynomial $X^n-1$. No
matter what field $K$ is, the polynomial can never have more than $n$
roots.  Clearly $1$ is an example; if $n$ is even, then $-1$ will also
be an example. Beyond this, the list of possibilities depends on $K$.

\begin{itemize} 
\item If $K$ is the set of real numbers, then $1$ and $-1$ are the
only possibilities.

\item If $K$ is the field of the complex numbers, the fundamental
theorem of algebra assures us that the polynomial $X^n-1$ has exactly
$n$ roots (counting multiplicities).  Comparing $X^n-1$ with its
formal \PMlinkname{derivative}{derivativeofpolynomial}, $nX^{n-1}$, we see that they are coprime, and
therefore all the roots of $X^n-1$ are distinct.  That is, there exist
$n$ distinct complex numbers $\omega$ such that $\omega^n=1$.

If $\zeta=e^{2\pi i/n}=\cos(2\pi/n)+i \sin(2\pi/n)$, then all the
$n$th roots of unity are: $\zeta^k=e^{2\pi ki/n}=\cos(2\pi k/n)+i
\sin(2\pi k/n)$ for $k=1,2,\ldots,n$.\medskip

If drawn on the complex plane, the $n$th roots of unity are the
vertices of a regular $n$-gon centered at the origin and with a vertex
at $1$.

\item If $K$ is a finite field having $p^a$ elements, for $p$ a prime,
then \emph{every} nonzero element is a $p^a-1$th root of unity (in
fact this characterizes them completely; this is the role of the
Frobenius operator).  For other $n$, the answer is more complicated.
For example, if $n$ is divisible by $p$, the formal derivative of
$X^n-1$ is $nX^{n-1}$, which is zero since the
\PMlinkid{characteristic}{1160} of $K$ is $p$ and $n$ is zero modulo
$p$.  So one is not guaranteed that the roots of unity will be
distinct.  For example, in the field of two elements, $1=-1$, so there
is only one square root of $1$.
\end{itemize}

If an element $\omega$ is an $n$th root of unity but is not an $m$th
root of unity for any $0<m<n$, then $\omega$ is called a
\emph{\PMlinkescapetext{primitive} $n$th root of unity}.  For example,
the number $\zeta$ defined above is a \PMlinkescapetext{primitive}
$n$th root of unity.  If $\omega \in \mathbb{C}$ is a primitive $n$th
root of unity, then all of the primitive $n$th roots of unity have the
form $\omega^m$ for some $m \in \mathbb{Z}$ with $\gcd(m,n)=1$.

The roots of unity in any field have many special relationships to one
another, some of which are true in general and some of which depend on
the field.  It is upon these relationships that the various algorithms
for computing fast Fourier transforms are based.

Finally, one could ask about similar situations where $K$ is not a
field but some more general object.  Here, things are much more
complicated.  For example, in the ring of endomorphisms of a vector
space, the unipotent linear transformations are the closest analogue
to roots of unity.  They still form a group, but there may be many
more of them than $n$.  In a finite group, every element $g$ has a
power $n$ such that $g^n=1$.

\PMlinkescapeword{primitive}
\PMlinkescapeword{term}
%%%%%
%%%%%
%%%%%
%%%%%
\end{document}
