\documentclass[12pt]{article}
\usepackage{pmmeta}
\pmcanonicalname{GregorySeries}
\pmcreated{2013-03-22 17:35:33}
\pmmodified{2013-03-22 17:35:33}
\pmowner{PrimeFan}{13766}
\pmmodifier{PrimeFan}{13766}
\pmtitle{Gregory series}
\pmrecord{4}{40005}
\pmprivacy{1}
\pmauthor{PrimeFan}{13766}
\pmtype{Definition}
\pmcomment{trigger rebuild}
\pmclassification{msc}{11-00}
\pmclassification{msc}{51-00}
\pmclassification{msc}{01A16}
\pmclassification{msc}{01A20}
\pmclassification{msc}{01A25}
\pmclassification{msc}{01A32}
\pmclassification{msc}{01A40}
\pmrelated{TaylorSeriesOfArcusTangent}

\endmetadata

% this is the default PlanetMath preamble.  as your knowledge
% of TeX increases, you will probably want to edit this, but
% it should be fine as is for beginners.

% almost certainly you want these
\usepackage{amssymb}
\usepackage{amsmath}
\usepackage{amsfonts}

% used for TeXing text within eps files
%\usepackage{psfrag}
% need this for including graphics (\includegraphics)
%\usepackage{graphicx}
% for neatly defining theorems and propositions
%\usepackage{amsthm}
% making logically defined graphics
%%%\usepackage{xypic}

% there are many more packages, add them here as you need them

% define commands here

\begin{document}
The {\em Gregory series} is an alternating sum whose value is a quarter that of $\pi$: $$\frac{\pi}{4} = \sum_{i = 0}^\infty (-1)^i \frac{1}{2i + 1} = 1 - \frac{1}{3} + \frac{1}{5} - \frac{1}{7} + \frac{1}{9} - \ldots$$
(The approximate decimal value of this expression is 0.7853981633974483...)

More generally, a Gregory series for a given $n$ is $$\sum_{i = 0}^\infty (-1)^i \frac{n^{2i + 1}}{2i + 1}.$$

The Gregory series is named after the Scottish astronomer and astrologer James Gregory.
%%%%%
%%%%%
\end{document}
