\documentclass[12pt]{article}
\usepackage{pmmeta}
\pmcanonicalname{AgohGiugaConjecture}
\pmcreated{2013-03-22 16:17:55}
\pmmodified{2013-03-22 16:17:55}
\pmowner{PrimeFan}{13766}
\pmmodifier{PrimeFan}{13766}
\pmtitle{Agoh-Giuga conjecture}
\pmrecord{6}{38419}
\pmprivacy{1}
\pmauthor{PrimeFan}{13766}
\pmtype{Conjecture}
\pmcomment{trigger rebuild}
\pmclassification{msc}{11D85}
\pmsynonym{Giuga conjecture}{AgohGiugaConjecture}
\pmsynonym{Giuga's conjecture}{AgohGiugaConjecture}
\pmsynonym{Agoh conjecture}{AgohGiugaConjecture}
\pmsynonym{Agoh's conjecture}{AgohGiugaConjecture}

% this is the default PlanetMath preamble.  as your knowledge
% of TeX increases, you will probably want to edit this, but
% it should be fine as is for beginners.

% almost certainly you want these
\usepackage{amssymb}
\usepackage{amsmath}
\usepackage{amsfonts}

% used for TeXing text within eps files
%\usepackage{psfrag}
% need this for including graphics (\includegraphics)
%\usepackage{graphicx}
% for neatly defining theorems and propositions
%\usepackage{amsthm}
% making logically defined graphics
%%%\usepackage{xypic}

% there are many more packages, add them here as you need them

% define commands here

\begin{document}
In 1950, Giuseppe Giuga conjectured that if and only if an integer $p$ is prime then it will satisfy the congruence $$\sum_{i = 1}^{p - 1} i^{p - 1} \equiv -1 \mod p.$$ This is sometimes called the \emph{Giuga conjecture}. Takashi Agoh rephrased the conjecture as $pB_{p - 1} \equiv -1 \mod p$, where $B$ is a Bernoulli number; this is called the \emph{Agoh-Giuga conjecture}. In 2003 Simon Plouffe performed an exhaustive search for a counterexample below 50000 but came up empty.
%%%%%
%%%%%
\end{document}
