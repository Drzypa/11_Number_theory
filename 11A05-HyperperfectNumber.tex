\documentclass[12pt]{article}
\usepackage{pmmeta}
\pmcanonicalname{HyperperfectNumber}
\pmcreated{2013-03-22 17:49:38}
\pmmodified{2013-03-22 17:49:38}
\pmowner{CompositeFan}{12809}
\pmmodifier{CompositeFan}{12809}
\pmtitle{hyperperfect number}
\pmrecord{4}{40291}
\pmprivacy{1}
\pmauthor{CompositeFan}{12809}
\pmtype{Definition}
\pmcomment{trigger rebuild}
\pmclassification{msc}{11A05}

\endmetadata

% this is the default PlanetMath preamble.  as your knowledge
% of TeX increases, you will probably want to edit this, but
% it should be fine as is for beginners.

% almost certainly you want these
\usepackage{amssymb}
\usepackage{amsmath}
\usepackage{amsfonts}

% used for TeXing text within eps files
%\usepackage{psfrag}
% need this for including graphics (\includegraphics)
%\usepackage{graphicx}
% for neatly defining theorems and propositions
%\usepackage{amsthm}
% making logically defined graphics
%%%\usepackage{xypic}

% there are many more packages, add them here as you need them

% define commands here

\begin{document}
A {\em hyperperfect number} $n$ for a given $k$ is an integer such that $k \sigma(n) = -1 + k + (k + 1)n$, where $\sigma(x)$ is the sum of divisors function. $n$ is then called $k$-hyperperfect. For example, 325 is 3-hyperperfect since its divisors (1, 5, 13, 25, 65, 325) add up to 434, and $3 \times 434 = -1 + 3 + (3 + 1)325 = 1302$. Numbers that are 1-hyperperfect are by default called perfect numbers, since $1 \sigma(n) = -1 + 1 + (1 + 1)n = 2n$.

The 2-hyperperfect numbers are listed in A007593 of Sloane's OEIS. As of 2007, the only known 3-hyperperfect number is 325. The two known 4-hyperperfect numbers are 1950625 and 1220640625, a sequence too short to list in the OEIS, and no 5-hyperperfect numbers are known to exist. The 6-hyperperfect numbers are listed in A028499.

\begin{thebibliography}{1}
\bibitem{abeiler} Judson S. McCrainie, ``A Study of Hyperperfect Numbers'' {\it Journal of Integer Sequences} {\bf 3} (2000): 00.1.3
\end{thebibliography}

%%%%%
%%%%%
\end{document}
