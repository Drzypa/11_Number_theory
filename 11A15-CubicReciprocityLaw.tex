\documentclass[12pt]{article}
\usepackage{pmmeta}
\pmcanonicalname{CubicReciprocityLaw}
\pmcreated{2013-03-22 13:41:26}
\pmmodified{2013-03-22 13:41:26}
\pmowner{mathcam}{2727}
\pmmodifier{mathcam}{2727}
\pmtitle{cubic reciprocity law}
\pmrecord{7}{34363}
\pmprivacy{1}
\pmauthor{mathcam}{2727}
\pmtype{Topic}
\pmcomment{trigger rebuild}
\pmclassification{msc}{11A15}
%\pmkeywords{reciprocity character}
\pmrelated{QuadraticReciprocityRule}
\pmdefines{cubic residue}
\pmdefines{cubic residue character}

\usepackage{amssymb}
\usepackage{amsmath}
\usepackage{amsfonts}
\newcommand{\legsymp}[1]{\left(\frac{#1}{p}\right)}
\newcommand{\Z}{\mathbb{Z}}
\begin{document}
\PMlinkescapeword{order}
In a ring $\Z/n\Z$, a cubic residue is just a value of the function
$x^3$ for some invertible element $x$ of the ring. Cubic residues display
a reciprocity phenomenon similar to that seen with quadratic
residues. But we need some preparation in order to state the cubic
reciprocity law.

$\omega$ will denote $\frac{-1+i\sqrt{3}}{2}$, which is one of the
complex cube roots of $1$.
$K$ will denote the ring $K=\mathbb{Z}[\omega]$. The elements of
$K$ are the complex numbers $a+b\omega$
where $a$ and $b$ are integers. We define the norm $N:K\to \Z$ by
$$N(a+b\omega)=a^2-ab+b^2$$
or equivalently
$$N(z)=z\overline{z}\;.$$

Whereas $\Z$ has only two units (meaning invertible elements), namely
$\pm 1$, $K$ has six, namely all the sixth roots of 1:
$$\pm 1\qquad\pm\omega\qquad\pm\omega^2$$
and we know $\omega^2=-1-\omega$. Two nonzero elements $\alpha$
and $\beta$ of $K$ are said
to be associates if $\alpha=\beta\mu$ for some unit $\mu$. This
is an equivalence relation, and any nonzero element has six associates.

$K$ is a principal ring, hence has unique factorization. Let us call
$\rho\in K$ ``irreducible'' if the condition $\rho=\alpha\beta$ implies
that $\alpha$ or $\beta$, but not both, is a unit.
It turns out that the irreducible elements of $K$ are (up to multiplication
by units):

-- the number $1-\omega$, which has norm 3. We will denote it by $\pi$.

-- positive real integers $q\equiv 2\pmod{3}$ which are prime in $\Z$.
Such integers are called rational primes in $K$.

-- complex numbers $q=a+b\omega$ where $N(q)$ is a prime in $Z$ and
$N(q)\equiv 1\pmod{3}$.

For example, $3+2\omega$ is a prime in $K$ because its norm, 7, is
prime in $\Z$ and is 1 mod 3; but 7 is not a prime in $K$.

Now we need some convention whereby at most one of any six associates
is called a prime. By convention, the following numbers are nominated:

-- the number $\pi$.

-- rational primes (rather than their negative or complex associates).

-- complex numbers $q=a+b\omega$ where $N(q)\equiv 1\pmod{3}$ is
prime in $\Z$ and
\begin{eqnarray*}
a&\equiv& 2\pmod{3} \\
b&\equiv& 0\pmod{3}\;.
\end{eqnarray*}
One can verify that this selection exists and is unambigous.

Next, we seek a three-valued function analogous to the
two-valued quadratic residue character $x\mapsto\legsymp{x}$.
Let $\rho$ be a prime in $K$, with $\rho\ne\pi$. If $\alpha$ is any
element of $K$ such that $\rho\nmid\alpha$, then
$$\alpha^{N(\rho)-1}\equiv 1\pmod{\rho}\;.$$
Since $N(\rho)-1$ is a multiple of 3, we can define a function
$$\chi_\rho:K\to\{1,\omega,\omega^2\}$$
by
\begin{eqnarray*}
\chi_\rho(\alpha)&\equiv&\alpha^{(N(\rho)-1)/3}\text{ if }\rho\nmid\alpha
\\
\chi_\rho(\alpha)&=&0\text{ if }\rho\mid\alpha\;.
\end{eqnarray*}
$\chi_\rho$ is a character, called the cubic residue character mod $\rho$.
We have $\chi_\rho(\alpha)=1$ if and only if $\alpha$ is a nonzero cube
mod $\rho$. (Compare Euler's criterion.)

At last we can state this famous result of Eisenstein and Jacobi:

\noindent
\textbf{Theorem (Cubic Reciprocity Law): }
If $\rho$ and $\sigma$ are any two distinct
primes in $K$, neither of them $\pi$, then
$$\chi_\rho(\sigma)=\chi_\sigma(\rho)\;.$$

The quadratic reciprocity law has two ``supplements'' which describe
$\legsymp{-1}$ and $\legsymp{2}$. Likewise the cubic law has this supplement,
due to Eisenstein:

\noindent
\textbf{Theorem:} For any prime $\rho$ in $K$, other than $\pi$,
$$\chi_\rho(\pi)=\omega^{2m}$$
where
\begin{eqnarray*}
m&=&(\rho+1)/3\qquad\text{ if $\rho$ is a rational prime}
\\
m&=&(a+1)/3\qquad\text{ if $\rho=a+b\omega$ is a complex prime.}
\end{eqnarray*}

\textbf{Remarks: }Some writers refer to our ``irreducible'' elements
as ``primes'' in $K$; what we have called primes, they call ``primary primes''.

The quadratic reciprocity law would take a simpler form if we were to
make a different convention on what is a prime in $\Z$, a convention
similar to the one in $K$: a prime in $\Z$ is either 2 or an
irreducible element $x$ of $\Z$ such that $x\equiv 1\pmod 4$.
The primes would then be 2, -3, 5, -7, -11, 13, \ldots and the QRL
would say simply
$$\left(\frac{p}{q}\right)\left(\frac{q}{p}\right)=1$$
for any two distinct odd primes $p$ and $q$.
%%%%%
%%%%%
\end{document}
