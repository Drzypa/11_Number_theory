\documentclass[12pt]{article}
\usepackage{pmmeta}
\pmcanonicalname{WeylsCriterion}
\pmcreated{2013-03-22 14:17:31}
\pmmodified{2013-03-22 14:17:31}
\pmowner{bbukh}{348}
\pmmodifier{bbukh}{348}
\pmtitle{Weyl's criterion}
\pmrecord{7}{35747}
\pmprivacy{1}
\pmauthor{bbukh}{348}
\pmtype{Theorem}
\pmcomment{trigger rebuild}
\pmclassification{msc}{11K06}
\pmclassification{msc}{11K38}
\pmclassification{msc}{11L03}
%\pmkeywords{uniform distribution}
%\pmkeywords{exponential sums}
\pmrelated{UniformlyDistributed}

\usepackage{amssymb}
\usepackage{amsmath}
\usepackage{amsfonts}

\newcommand*{\abs}[1]{\left\lvert #1\right\rvert}

\usepackage[T2A]{fontenc}
\usepackage[russian,english]{babel}


\makeatletter
\@ifundefined{bibname}{}{\renewcommand{\bibname}{References}}
\makeatother
\begin{document}
Let $\{u_n\}$ be a sequence of real numbers. Then $\{u_n\}$ is
uniformly distributed modulo $1$ if and only if
\begin{equation*}
\lim_{N\to\infty} \frac{1}{N} \sum_{n=1}^N e(k u_n)=0
\end{equation*}
for every nonzero integer $k$, where $e(x)=\exp(2\pi i x)$.

Weyl's criterion reduces the problem of uniform distribution of
sequences to the problem of estimating certain exponential sums.
Whereas the problem of estimating a family of exponential sums
might seem harder at first, the exponential map has the
multiplicative property which often makes the problem easier.

\emph{Example:} If $x$ is irrational, then the sequence $\{nx\}$
is uniformly distributed modulo $1$. Proof:
\begin{equation*}
\abs{\sum_{n=1}^{N} e(k n
x)}=\abs{\frac{e(k(N+1)x)-e(kx)}{e(kx)-1}}\leq
\frac{2}{\abs{\,e(kx)-1}}=O_k(1)
\end{equation*}
because the irrationality of $x$ implies $e(kx)\neq 1$.

\begin{thebibliography}{1}

\bibitem{cite:karatsuba_ant}
Ð?.~Ð?. Карацуба.
\newblock {\em ОÑ?новы аналитичеÑ?кой теории чиÑ?ел}.
\newblock Ð?аука, 1983.
\newblock \PMlinkexternal{Zbl 0428.10019}{http://www.emis.de/cgi-bin/zmen/ZMATH/en/quick.html?type=html&an=0428.10019}.
\newblock For English translation see \cite{cite:karatsuba_ant_eng}.

\bibitem{cite:karatsuba_ant_eng}
A.~A. Karatsuba.
\newblock {\em Basic analytic number theory}.
\newblock Springer-Verlag, 1993.
\newblock \PMlinkexternal{Zbl 0767.11001}{http://www.emis.de/cgi-bin/zmen/ZMATH/en/quick.html?type=html&an=0767.11001}.
%\newblock This is a translation of \cite{cite:karatsuba_ant}.

\bibitem{cite:montgomery_tenlect}
Hugh~L. Montgomery.
\newblock {\em Ten Lectures on the Interface Between Analytic Number Theory and
  Harmonic Analysis}, volume~84 of {\em Regional Conference Series in
  Mathematics}.
\newblock AMS, 1994.
\newblock \PMlinkexternal{Zbl 0814.11001}{http://www.emis.de/cgi-bin/zmen/ZMATH/en/quick.html?type=html&an=0814.11001}.

\end{thebibliography}
%%%%%
%%%%%
\end{document}
