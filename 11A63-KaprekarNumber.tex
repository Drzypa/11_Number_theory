\documentclass[12pt]{article}
\usepackage{pmmeta}
\pmcanonicalname{KaprekarNumber}
\pmcreated{2013-03-22 16:00:17}
\pmmodified{2013-03-22 16:00:17}
\pmowner{PrimeFan}{13766}
\pmmodifier{PrimeFan}{13766}
\pmtitle{Kaprekar number}
\pmrecord{7}{38034}
\pmprivacy{1}
\pmauthor{PrimeFan}{13766}
\pmtype{Definition}
\pmcomment{trigger rebuild}
\pmclassification{msc}{11A63}

% this is the default PlanetMath preamble.  as your knowledge
% of TeX increases, you will probably want to edit this, but
% it should be fine as is for beginners.

% almost certainly you want these
\usepackage{amssymb}
\usepackage{amsmath}
\usepackage{amsfonts}

% used for TeXing text within eps files
%\usepackage{psfrag}
% need this for including graphics (\includegraphics)
%\usepackage{graphicx}
% for neatly defining theorems and propositions
%\usepackage{amsthm}
% making logically defined graphics
%%%\usepackage{xypic}

% there are many more packages, add them here as you need them

% define commands here

\begin{document}
Let $n$ be a $k$-digit integer in base $b$. Then $n$ is said to be a {\em Kaprekar number} in base $b$ if $n^2$ has the following property: when you add the number formed by its right hand digits to that formed by its left hand digits, you get $n$.

Or to put it algebraically, an integer $n$ such that in a given base $b$ has $$n^2 = \sum_{i = 0}^{k - 1} d_ib^i$$ (where $d_x$ are digits, with $d_0$ the least significant digit and $d_{k - 1}$ the most significant) such that $$\sum_{i = {k \over 2} + 1}^k d_ib^{i - {k \over 2} - 1} + \sum_{i = 1}^{k \over 2} d_ib^{i - 1} = n$$ if $k$ is even or $$\sum_{i = \lceil {k \over 2} \rceil}^k d_ib^{i - \lfloor {k \over 2} \rfloor - 1} + \sum_{i = 1}^{k \over 2} d_ib^{i - 1} = n$$ if $k$ is odd.

$b^x - 1$ for a natural $x$ is always a Kaprekar number in base $b$.

\begin{thebibliography}{1}
\bibitem{dk} D. R. Kaprekar, ``On Kaprekar numbers" {\it J. Rec. Math.} 13 (1980-1981), 81 - 82.
\end{thebibliography}
%%%%%
%%%%%
\end{document}
