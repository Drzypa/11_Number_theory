\documentclass[12pt]{article}
\usepackage{pmmeta}
\pmcanonicalname{FourHundredNinetysix}
\pmcreated{2013-03-22 17:10:51}
\pmmodified{2013-03-22 17:10:51}
\pmowner{CompositeFan}{12809}
\pmmodifier{CompositeFan}{12809}
\pmtitle{four hundred ninety-six}
\pmrecord{4}{39496}
\pmprivacy{1}
\pmauthor{CompositeFan}{12809}
\pmtype{Feature}
\pmcomment{trigger rebuild}
\pmclassification{msc}{11A99}
\pmsynonym{four hundred and ninety-six}{FourHundredNinetysix}

\endmetadata

% this is the default PlanetMath preamble.  as your knowledge
% of TeX increases, you will probably want to edit this, but
% it should be fine as is for beginners.

% almost certainly you want these
\usepackage{amssymb}
\usepackage{amsmath}
\usepackage{amsfonts}

% used for TeXing text within eps files
%\usepackage{psfrag}
% need this for including graphics (\includegraphics)
%\usepackage{graphicx}
% for neatly defining theorems and propositions
%\usepackage{amsthm}
% making logically defined graphics
%%%\usepackage{xypic}

% there are many more packages, add them here as you need them

% define commands here

\begin{document}
The third perfect number, {\em four hundred ninety-six} (496) has been known since antiquity. With just one larger perfect number known to him, Euclid was able to prove that all even perfect numbers are the product of a Mersenne prime and the nearest smaller power of two. In the case of 496, these are 31 and 16.

As a counterexample, 496 disproves Thomas Greenwood's conjecture that an even triangular number with a prime index is one less than a prime, since although 496 is the 31st triangular number, 497 is not a prime.

496 is an important number in physics, and specifically string theory. ``The massless bosonic states in this theory consist of a symmetric rank two field, an anti-symmetric rank two field, a scalar field known as the dilaton and a set of 496 gauge fields filling up the adjoint representation of the gauge group $E_8 \times E_8$.'' (Sen, 1998) This discovery of the importance of 496, by Michael Green and John Schwartz is credited with ushering in an era of important revelations in string theory.

\begin{thebibliography}{2}
\bibitem{dw} D. Wells {\it The Dictionary of Curious and Interesting Numbers} Suffolk: Penguin Books (1987): 155
\bibitem{as} A. Sen ``An Introduction to Non-perturbative String Theory'' \PMlinkexternal{ArXiv preprint}{http://arxiv.org/abs/hep-th/9802051v1}
\end{thebibliography}

%%%%%
%%%%%
\end{document}
