\documentclass[12pt]{article}
\usepackage{pmmeta}
\pmcanonicalname{SiegelsTheorem}
\pmcreated{2013-03-22 15:57:24}
\pmmodified{2013-03-22 15:57:24}
\pmowner{alozano}{2414}
\pmmodifier{alozano}{2414}
\pmtitle{Siegel's theorem}
\pmrecord{6}{37969}
\pmprivacy{1}
\pmauthor{alozano}{2414}
\pmtype{Theorem}
\pmcomment{trigger rebuild}
\pmclassification{msc}{11G05}
\pmrelated{FaltingsTheorem}

% this is the default PlanetMath preamble.  as your knowledge
% of TeX increases, you will probably want to edit this, but
% it should be fine as is for beginners.

% almost certainly you want these
\usepackage{amssymb}
\usepackage{amsmath}
\usepackage{amsthm}
\usepackage{amsfonts}

% used for TeXing text within eps files
%\usepackage{psfrag}
% need this for including graphics (\includegraphics)
%\usepackage{graphicx}
% for neatly defining theorems and propositions
%\usepackage{amsthm}
% making logically defined graphics
%%%\usepackage{xypic}

% there are many more packages, add them here as you need them

% define commands here

\newtheorem*{thm}{Theorem}
\newtheorem{defn}{Definition}
\newtheorem{prop}{Proposition}
\newtheorem{lemma}{Lemma}
\newtheorem{cor}{Corollary}

\theoremstyle{definition}
\newtheorem{exa}{Example}

% Some sets
\newcommand{\Nats}{\mathbb{N}}
\newcommand{\Ints}{\mathbb{Z}}
\newcommand{\Reals}{\mathbb{R}}
\newcommand{\Complex}{\mathbb{C}}
\newcommand{\Rats}{\mathbb{Q}}
\newcommand{\Gal}{\operatorname{Gal}}
\newcommand{\Cl}{\operatorname{Cl}}
\begin{document}
The following theorem is a very deep application of Roth's theorem. Let $K$ be a number field and let $S$ be a finite set of places of $K$. Let $R_S$ be the \PMlinkid{ring of $S$-integers}{RingOfSIntegers} in $K$. Let $C/K$ be a smooth projective curve of genus $g$ defined over $K$ and let $f$ be a non-constant function in the function field of $C/K$, i.e. $f\in K(C)$. 

\begin{thm}[Siegel's Theorem]
Assume that $C/K$ has genus $g\geq 1$. Then the set $\{P\in C(K) : f(P) \in R_S\}$ is finite.
\end{thm} 

In particular, when $f$ is the coordinate functions $x(P)$ and $y(P)$, Siegel's theorem implies that a curve of genus $\geq 1$ has only finitely many integral points. For example, this shows that an elliptic curve defined over $\Rats$ can only have finitely many points defined over $\Ints$.
%%%%%
%%%%%
\end{document}
