\documentclass[12pt]{article}
\usepackage{pmmeta}
\pmcanonicalname{QuadraticMap}
\pmcreated{2013-03-22 16:27:55}
\pmmodified{2013-03-22 16:27:55}
\pmowner{Algeboy}{12884}
\pmmodifier{Algeboy}{12884}
\pmtitle{quadratic map}
\pmrecord{9}{38625}
\pmprivacy{1}
\pmauthor{Algeboy}{12884}
\pmtype{Derivation}
\pmcomment{trigger rebuild}
\pmclassification{msc}{11E08}
\pmclassification{msc}{11E04}
\pmclassification{msc}{15A63}
\pmrelated{QuadraticJordanAlgebra}
\pmrelated{IsotropicQuadraticSpace}
\pmdefines{quadratic map}
\pmdefines{totally singular}
\pmdefines{totally isotropic}
\pmdefines{polarization formula}
\pmdefines{polarization identity}

\endmetadata

\usepackage{latexsym}
\usepackage{amssymb}
\usepackage{amsmath}
\usepackage{amsfonts}
\usepackage{amsthm}

%%\usepackage{xypic}

%-----------------------------------------------------

%       Standard theoremlike environments.

%       Stolen directly from AMSLaTeX sample

%-----------------------------------------------------

%% \theoremstyle{plain} %% This is the default

\newtheorem{thm}{Theorem}

\newtheorem{coro}[thm]{Corollary}

\newtheorem{lem}[thm]{Lemma}

\newtheorem{lemma}[thm]{Lemma}

\newtheorem{prop}[thm]{Proposition}

\newtheorem{conjecture}[thm]{Conjecture}

\newtheorem{conj}[thm]{Conjecture}

\newtheorem{defn}[thm]{Definition}

\newtheorem{remark}[thm]{Remark}

\newtheorem{ex}[thm]{Example}



%\countstyle[equation]{thm}



%--------------------------------------------------

%       Item references.

%--------------------------------------------------


\newcommand{\exref}[1]{Example-\ref{#1}}

\newcommand{\thmref}[1]{Theorem-\ref{#1}}

\newcommand{\defref}[1]{Definition-\ref{#1}}

\newcommand{\eqnref}[1]{(\ref{#1})}

\newcommand{\secref}[1]{Section-\ref{#1}}

\newcommand{\lemref}[1]{Lemma-\ref{#1}}

\newcommand{\propref}[1]{Prop\-o\-si\-tion-\ref{#1}}

\newcommand{\corref}[1]{Cor\-ol\-lary-\ref{#1}}

\newcommand{\figref}[1]{Fig\-ure-\ref{#1}}

\newcommand{\conjref}[1]{Conjecture-\ref{#1}}


% Normal subgroup or equal.

\providecommand{\normaleq}{\unlhd}

% Normal subgroup.

\providecommand{\normal}{\lhd}

\providecommand{\rnormal}{\rhd}
% Divides, does not divide.

\providecommand{\divides}{\mid}

\providecommand{\ndivides}{\nmid}


\providecommand{\union}{\cup}

\providecommand{\bigunion}{\bigcup}

\providecommand{\intersect}{\cap}

\providecommand{\bigintersect}{\bigcap}










\begin{document}
Given a commutative ring $K$ and two $K$-modules $M$ and $N$ then a map
$q:M\rightarrow N$ is called \emph{quadratic} if
\begin{enumerate}
\item $q(\alpha x)=\alpha^2 q(x)$ for all $x\in M$ and $\alpha\in K$.
\item $b(x,y):=q(x+y)-q(x)-q(y)$, for $x,y\in M$, is a bilinear map.
\end{enumerate}


The only difference between quadratic maps and quadratic forms is the insistence on the codomain $N$ instead of a $K$.  So in this way every quadratic form is a special case of a quadratic map.  Most of the properties for quadratic forms apply to quadratic maps as well.  For instance, if $K$ has no 2-torsion ($2x=0$ implies $x=0$) then 
    \[2c(x,y)=q(x+y)-q(x)-q(y).\]
defines a symmetric $K$-bilinear map $c:M\times M\to N$ with $c(x,x)=q(x)$.  In particular if $1/2\in K$ then 
$c(x,y)=\frac{1}{2}b(x,y)$.
This definition is one instance of a polarization (i.e.: substituting a single
variable in a formula with $x+y$ and comparing the result with the formula over $x$ and $y$ separately.)  Continuing
without $2$-torsion, if $b$ is a symmetric $K$-bilinear map (perhaps not a form) then defining
$q_b(x)=b(x,x)$ determines a quadratic map since
   \[q_b(\alpha x)=b(\alpha x,\alpha x)=\alpha^2 b(x,x)=\alpha^2 q(x)\]
and
\[q_b(x+y)-q_b(x)-q_b(y)
     =b(x+y,x+y)-b(x,x)-b(y,y)=b(x,y)+b(y,x)=2 b(x,y).\]
Have have no $2$-torsion we can recover $b$ form $q_b$.  So in odd and 0 characteristic rings we find symmetric 
bilinear maps and quadratic maps are in 1-1 correspondence.

An alternative understanding of $b$ is to treat this as the obstruction to
$q$ being an additive homomorphism.  Thus a submodule $T$ of $M$ for which $b(T,T)=0$ is a submodule of $M$ on which $q|_T$ is an additive homomorphism.
Of course because of the first condition, $q$ is semi-linear on $T$ only when $\alpha\mapsto \alpha^2$ is an automorphism of $K$, in particular, if $K$ has characteristic 2.  When the characteristic of $K$ is odd or 0 then $q(T)=0$ 
if and only if $b(T,T)=0$ simply because $q(x)=b(x,x)$ (or up to a $1/2$
multiple depending on conventions).  However, in characteristic 2 it is 
possible for $b(T,T)=0$ yet $q(T)\neq 0$.  For instance, we can have 
$q(x)\neq 0$ yet $b(x,x)=q(2x)-q(x)-q(x)=0$.  This is summed up in the following
definition:

A subspace $T$ of $M$ is called \emph{totally singular} if $q(T)=0$ and
totally isotropic if $b(T,T)=0$.  In odd or 0 characteristic, totally singular
subspaces are precisely totally isotropic subspaces.




%%%%%
%%%%%
\end{document}
