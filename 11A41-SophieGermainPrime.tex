\documentclass[12pt]{article}
\usepackage{pmmeta}
\pmcanonicalname{SophieGermainPrime}
\pmcreated{2013-03-22 14:34:23}
\pmmodified{2013-03-22 14:34:23}
\pmowner{yark}{2760}
\pmmodifier{yark}{2760}
\pmtitle{Sophie Germain prime}
\pmrecord{8}{36131}
\pmprivacy{1}
\pmauthor{yark}{2760}
\pmtype{Definition}
\pmcomment{trigger rebuild}
\pmclassification{msc}{11A41}
\pmsynonym{Germain prime}{SophieGermainPrime}

\usepackage{amssymb}
\usepackage{amsmath}
\usepackage{amsfonts}

\begin{document}
\PMlinkescapeword{estimate}

A prime number $p$ is called \emph{Sophie Germain prime} if $2p+1$ is also prime.

The first few Sophie Germain primes are:
$2, 3, 5, 11, 23, 29, 41, 53, 83, 89, 113, 131, 173, 179, 191, 233,\dots$

It is conjectured that there are infinitely many Sophie Germain primes,
but (like the Twin Prime Conjecture) this has not been proven.
A heuristic estimate for the number of Sophie Germain primes less than $n$ is $\frac{2cn}{\ln^2{n}}$, where $c$ is the twin prime constant.

%%%%%
%%%%%
\end{document}
