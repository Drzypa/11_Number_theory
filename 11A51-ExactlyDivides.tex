\documentclass[12pt]{article}
\usepackage{pmmeta}
\pmcanonicalname{ExactlyDivides}
\pmcreated{2013-03-22 16:10:44}
\pmmodified{2013-03-22 16:10:44}
\pmowner{Wkbj79}{1863}
\pmmodifier{Wkbj79}{1863}
\pmtitle{exactly divides}
\pmrecord{7}{38266}
\pmprivacy{1}
\pmauthor{Wkbj79}{1863}
\pmtype{Definition}
\pmcomment{trigger rebuild}
\pmclassification{msc}{11A51}
\pmrelated{Divides}
\pmrelated{Divisibility}
\pmrelated{DivisibilityInRings}

\endmetadata

\usepackage{amssymb}
\usepackage{amsmath}
\usepackage{amsfonts}

\usepackage{psfrag}
\usepackage{graphicx}
\usepackage{amsthm}
%%\usepackage{xypic}

\begin{document}
\PMlinkescapeword{similar} \PMlinkescapeword{expression}

Let $a$ and $b$ be integers and $n$ a positive integer.\,  Then $a^m$ \emph{exactly divides} $b$ (denoted as $a^n \parallel n$) if $a^n$ divides $b$ but $a^{n+1}$ does not divide $b$.\, For example,\, $2^4 \parallel 48$.

One can, of course, use the similar expression and notation for the elements $a$, $b$ of any commutative ring or monoid (cf. e.g. divisor as factor of principal divisor).

%%%%%
%%%%%
\end{document}
