\documentclass[12pt]{article}
\usepackage{pmmeta}
\pmcanonicalname{RecurrenceFormulaForBernoulliNumbers}
\pmcreated{2013-03-22 17:46:19}
\pmmodified{2013-03-22 17:46:19}
\pmowner{rm50}{10146}
\pmmodifier{rm50}{10146}
\pmtitle{recurrence formula for Bernoulli numbers}
\pmrecord{4}{40228}
\pmprivacy{1}
\pmauthor{rm50}{10146}
\pmtype{Derivation}
\pmcomment{trigger rebuild}
\pmclassification{msc}{11B68}

\endmetadata

% this is the default PlanetMath preamble.  as your knowledge
% of TeX increases, you will probably want to edit this, but
% it should be fine as is for beginners.

% almost certainly you want these
\usepackage{amssymb}
\usepackage{amsmath}
\usepackage{amsfonts}

% used for TeXing text within eps files
%\usepackage{psfrag}
% need this for including graphics (\includegraphics)
%\usepackage{graphicx}
% for neatly defining theorems and propositions
%\usepackage{amsthm}
% making logically defined graphics
%%%\usepackage{xypic}

% there are many more packages, add them here as you need them

% define commands here

\begin{document}
This article establishes a well-known recurrence formula for the Bernoulli numbers.

The Bernoulli polynomials $b_r(x), r\geq 1$ can be written explicitly as
\[b_r(x) = \sum_{k=1}^r \binom{r}{k}B_{r-k}x^k + B_r\]
(see \PMlinkname{this article}{CoefficientsOfBernoulliPolynomials}).

For $r\geq 2$, we have
\[0=\int_0^1 b_{r-1}(x)dx = \frac{1}{r}b_r(x)\big\lvert_0^1=\frac{1}{r}(b_r(1)-b_r(0))\]
and thus
\[B_r=b_r(0)=b_r(1)=\sum_{k=1}^r \binom{r}{k}B_{r-k} + B_r\]

It follows that (still when $r\geq 2$)
\[\sum_{k=1}^r \binom{r}{k}B_{r-k}=0\]
so that
\[\binom{r}{1}B_{r-1} = -\sum_{k=2}^r \binom{r}{k}B_{r-k}\]
Replacing $r$ by $r+1$ and simplifying, we see that for $r\geq 1$,
\[B_r = \frac{-1}{r+1}\sum_{k=2}^{r+1}\binom{r+1}{k}B_{r+1-k} = \frac{-1}{r+1}\sum_{k=1}^r\binom{r+1}{k+1}B_{r-k}\]

%%%%%
%%%%%
\end{document}
