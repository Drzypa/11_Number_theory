\documentclass[12pt]{article}
\usepackage{pmmeta}
\pmcanonicalname{Duodecimal}
\pmcreated{2013-03-22 16:20:55}
\pmmodified{2013-03-22 16:20:55}
\pmowner{PrimeFan}{13766}
\pmmodifier{PrimeFan}{13766}
\pmtitle{duodecimal}
\pmrecord{5}{38482}
\pmprivacy{1}
\pmauthor{PrimeFan}{13766}
\pmtype{Definition}
\pmcomment{trigger rebuild}
\pmclassification{msc}{11A63}

% this is the default PlanetMath preamble.  as your knowledge
% of TeX increases, you will probably want to edit this, but
% it should be fine as is for beginners.

% almost certainly you want these
\usepackage{amssymb}
\usepackage{amsmath}
\usepackage{amsfonts}

% used for TeXing text within eps files
%\usepackage{psfrag}
% need this for including graphics (\includegraphics)
%\usepackage{graphicx}
% for neatly defining theorems and propositions
%\usepackage{amsthm}
% making logically defined graphics
%%%\usepackage{xypic}

% there are many more packages, add them here as you need them

% define commands here

\begin{document}
The {\em duodecimal system} is a positional number system with base 12, using the digits 0, 1, 2, 3, 4, 5, 6, 7, 8, 9, A and B.

The \PMlinkescapetext{term} duodecimal comes fro Latin.\, There is also a less used synonym {\em duodecadic} mixed of a Latin begin and a Greek end and still a purely Greek \PMlinkescapetext{formation} {\em dodecadic}.

Some divisibility tests in duodecimal are:

$n$ is divisible by 2 if its least significant digit is 0, 2, 4, 6, 8 or A.

$n$ is divisible by 3 if its least significant digit is 0, 3, 6 or 9.

$n$ is divisible by 4 if its least significant digit is 0, 4 or 8.

$n$ is divisible by 6 if its least significant digit is 0 or 6.

$n$ is divisible by 11 if it has digital root B.

$n$ is of course divisible by 12 if it ends in a 0.

$n$ is divisible by 13 if the difference of the odd placed digits and the even place digits of $n$ is a multiple of 13.
%%%%%
%%%%%
\end{document}
