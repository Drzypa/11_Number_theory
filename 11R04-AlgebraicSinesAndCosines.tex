\documentclass[12pt]{article}
\usepackage{pmmeta}
\pmcanonicalname{AlgebraicSinesAndCosines}
\pmcreated{2013-03-22 18:51:27}
\pmmodified{2013-03-22 18:51:27}
\pmowner{pahio}{2872}
\pmmodifier{pahio}{2872}
\pmtitle{algebraic sines and cosines}
\pmrecord{7}{41667}
\pmprivacy{1}
\pmauthor{pahio}{2872}
\pmtype{Corollary}
\pmcomment{trigger rebuild}
\pmclassification{msc}{11R04}
\pmclassification{msc}{11C08}
\pmrelated{RationalSineAndCosine}
\pmrelated{MultiplesOfAnAlgebraicNumber}

% this is the default PlanetMath preamble.  as your knowledge
% of TeX increases, you will probably want to edit this, but
% it should be fine as is for beginners.

% almost certainly you want these
\usepackage{amssymb}
\usepackage{amsmath}
\usepackage{amsfonts}

% used for TeXing text within eps files
%\usepackage{psfrag}
% need this for including graphics (\includegraphics)
%\usepackage{graphicx}
% for neatly defining theorems and propositions
 \usepackage{amsthm}
% making logically defined graphics
%%%\usepackage{xypic}

% there are many more packages, add them here as you need them

% define commands here

\theoremstyle{definition}
\newtheorem*{thmplain}{Theorem}

\begin{document}
For any rational number $r$, the sine and the cosine of the number $r\pi$ are algebraic numbers.\\

\emph{Proof.}\, According to the \PMlinkid{parent entry}{11664}, $\sin{n\varphi}$ and $\cos{n\varphi}$ can be expressed as polynomials with integer coefficients of $\sin\varphi$ or $\cos\varphi$, respectively, when $n$ is an integer.\, Thus we can write 
$$\sin{n\varphi} \;=\; P(\sin\varphi), \quad \cos{n\varphi} \;=\; Q(\cos\varphi),$$
where\, $P(x),\,Q(x) \in \mathbb{Z}[x]$.\, If\, $\displaystyle r = \frac{m}{n}$\, where $m,\,n$ are integers and\, 
$n \neq 0$,\, we have
$$P(\sin{r\pi}) \;=\; \sin{nr\pi} \;=\; \sin{m\pi} \;=\; 0, \quad
  Q(\cos{r\pi}) \;=\; \cos{nr\pi} \;=\; \cos{m\pi} \;=\; \pm1,$$
i.e. both $\sin{r\pi}$ and $\cos{r\pi}$ satisfy an algebraic equation.\, Q.E.D.\\


For example,
$$\cos{7\varphi} \;=\; 64\cos^7\varphi-112\cos^5\varphi+56\cos^3\varphi-7\cos\varphi,$$
whence we have the identity
$$64\cos^7\frac{\pi}{7}-112\cos^5\frac{\pi}{7}+56\cos^3\frac{\pi}{7}-7\cos\frac{\pi}{7}+1 \;=\; 0,$$
and therefore $\cos\frac{\pi}{7}$ is algebraic over $\mathbb{Z}$.

%%%%%
%%%%%
\end{document}
