\documentclass[12pt]{article}
\usepackage{pmmeta}
\pmcanonicalname{BibliographyForNumberTheory}
\pmcreated{2013-03-22 14:14:19}
\pmmodified{2013-03-22 14:14:19}
\pmowner{alozano}{2414}
\pmmodifier{alozano}{2414}
\pmtitle{bibliography for number theory}
\pmrecord{13}{35681}
\pmprivacy{1}
\pmauthor{alozano}{2414}
\pmtype{Bibliography}
\pmcomment{trigger rebuild}
\pmclassification{msc}{11-00}
\pmrelated{NumberTheory}
\pmrelated{ClassNumbersAndDiscriminantsTopicsOnClassGroups}
\pmrelated{AlgebraicNumberTheory}

% this is the default PlanetMath preamble.  as your knowledge
% of TeX increases, you will probably want to edit this, but
% it should be fine as is for beginners.

% almost certainly you want these
\usepackage{amssymb}
\usepackage{amsmath}
\usepackage{amsthm}
\usepackage{amsfonts}

% used for TeXing text within eps files
%\usepackage{psfrag}
% need this for including graphics (\includegraphics)
%\usepackage{graphicx}
% for neatly defining theorems and propositions
%\usepackage{amsthm}
% making logically defined graphics
%%%\usepackage{xypic}

% there are many more packages, add them here as you need them

% define commands here

\newtheorem{thm}{Theorem}
\newtheorem{defn}{Definition}
\newtheorem{prop}{Proposition}
\newtheorem{lemma}{Lemma}
\newtheorem{cor}{Corollary}

% Some sets
\newcommand{\Nats}{\mathbb{N}}
\newcommand{\Ints}{\mathbb{Z}}
\newcommand{\Reals}{\mathbb{R}}
\newcommand{\Complex}{\mathbb{C}}
\newcommand{\Rats}{\mathbb{Q}}

% Annotation of a book
\begin{document}
\PMlinkescapeword{sources}
\PMlinkescapeword{areas}
\section*{References for Number Theory, MSC 11}
The following are excellent sources for the indicated areas in Number Theory.

\subsection*{Elementary Number Theory, MSC 11A}
\begin{enumerate}
\item G.~H. Hardy, E.~M. Wright, {\em An Introduction To The Theory Of Numbers}, Oxford University Press, London.
\begin{quote}
An introductory book which is both comprehensive and comprehensible.
\end{quote}
\item K. Ireland, M. Rosen, {\em A Classical Introduction to Modern Number Theory}, Springer-Verlag, 1998.
\end{enumerate}

\subsection*{Sequences and sets, MSC 11B}
\begin{enumerate}
\item Halberstam and Roth, {\em Sequences}, Oxford Clarendon Press
\begin{quote}
This well-written book is somewhat outdated by now, but it is an excellent source to learn the basics from.
\end{quote}
\item Nathanson, {\em Inverse Problems and Geometry of Sumsets}, Springer
\begin{quote}
The inverse problem in additive number theory is the problem of inferring the structure of summands from the structure of the sumset. The book is the most complete source for information on such problems.
\end{quote}
\end{enumerate}

\subsection*{Diophantine equations, MSC 11D}
\begin{enumerate}
\item K. Ireland, M. Rosen, {\em A Classical Introduction to Modern Number Theory}, Springer-Verlag, 1998.
\end{enumerate}

\subsection*{Arithmetic algebraic geometry (Diophantine geometry), MSC 11G}
\begin{enumerate}
\item K. Ireland, M. Rosen, {\em A Classical Introduction to Modern Number Theory}, Springer-Verlag, 1998.
\end{enumerate}

\subsection*{Exponential sums and character sums, MSC 11L}
\begin{enumerate}
\item E.~C. Titchmarsh, {\em The Theory of the Riemann Zeta-function}. second ed. Oxford Science Pub. 1986
\begin{quote}
The book covers the classical methods of Weyl, van der Corput/Phillips as well as mean-value method of Vinogradov.
\end{quote}
\end{enumerate}

\subsection*{Multiplicative number theory, MSC 11N}
\begin{enumerate}
\item Davenport, {\em Multiplicative number theory}. Markham Publishing Comp., Chicago.
\begin{quote}
Carefully written and motivated introduction to the multiplicative number theory.
\end{quote}
\end{enumerate}

\subsection*{Algebraic Number Theory: Global Fields, MSC 11R}
\begin{enumerate}
\item Serge Lang, {\em Algebraic Number Theory}. Springer-Verlag, New York.
\item Daniel A. Marcus, {\em Number Fields}, Springer, New York.
\item K. Ireland, M. Rosen, {\em A Classical Introduction to Modern Number Theory}, Springer-Verlag, 1998.
\end{enumerate}

\subsubsection*{Cyclotomic Extensions, MSC 11R18}
\begin{enumerate}
\item Lawrence C. Washington, {\em Introduction to Cyclotomic Fields}, Springer-Verlag, New York.
\end{enumerate}

\subsubsection*{Galois Cohomology, MSC 11R34}
\begin{enumerate}
\item J. P. Serre, {\em Galois Cohomology}, Springer-Verlag, New York.
\end{enumerate}

\subsection*{Algebraic Number Theory: Local Fields and $p$-adic Fields, MSC 11S}
\begin{enumerate}
\item Serge Lang, {\em Algebraic Number Theory}. Springer-Verlag, New York.

\item Jean Pierre Serre, {\em Local Fields}, Springer-Verlag, New York. 

\item Senon I. Borewicz, Igor R. \v{S}afarevi\v{c}, {\em Zahlentheorie}, Birkh\"auser Verlag, Basel und Stuttgart (1966).
\end{enumerate}

\subsection*{Finite fields and finite commutative rings (number-theoretic), MSC 11T}
\begin{enumerate}
\item K. Ireland, M. Rosen, {\em A Classical Introduction to Modern Number Theory}, Springer-Verlag, 1998.
\end{enumerate}
%%%%%
%%%%%
\end{document}
