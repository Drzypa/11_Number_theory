\documentclass[12pt]{article}
\usepackage{pmmeta}
\pmcanonicalname{ThePrimePowerDividingAFactorial}
\pmcreated{2013-03-22 13:22:34}
\pmmodified{2013-03-22 13:22:34}
\pmowner{Thomas Heye}{1234}
\pmmodifier{Thomas Heye}{1234}
\pmtitle{the prime power dividing a factorial}
\pmrecord{10}{33908}
\pmprivacy{1}
\pmauthor{Thomas Heye}{1234}
\pmtype{Theorem}
\pmcomment{trigger rebuild}
\pmclassification{msc}{11A51}

% this is the default PlanetMath preamble.  as your knowledge
% of TeX increases, you will probably want to edit this, but
% it should be fine as is for beginners.

% almost certainly you want these
\usepackage{amssymb}
\usepackage{amsmath}
\usepackage{amsfonts}

% used for TeXing text within eps files
%\usepackage{psfrag}
% need this for including graphics (\includegraphics)
%\usepackage{graphicx}
% for neatly defining theorems and propositions
\usepackage{amsthm}
% making logically defined graphics
%%%\usepackage{xypic}

% there are many more packages, add them here as you need them

% define commands here
\newtheorem*{cor}{Corollary}
\begin{document}
In 1808, Legendre showed that the exact power of a prime $p$ dividing
$n!$ is
\[\sum_{i=1}^K \left\lfloor \frac{n}{p^i} \right\rfloor\]
where $K$ is the largest power of $p$ being $\leq n$.

\begin{proof}
If $p>n$ then $p$ doesn't divide $n!$, and its power is 0, and the sum
above is
empty. So let the prime $p \le n$. \\
For each $1 \le i \le K$, there are $\left\lfloor \frac{n}{p^i}
\right\rfloor
-\left\lfloor \frac{n}{p^{i+1}} \right\rfloor$ numbers between 1 and
$n$ with
$i$ being the greatest power of $p$ dividing each. So the power of $p$
dividing $n!$ is
\[\sum_{i=1}^K i\cdot \left(\left\lfloor \frac{n}{p^i} \right\rfloor
-\left\lfloor \frac{n}{p^{i+1}} \right\rfloor\right).\]
But each $\left\lfloor \frac{n}{p^i} \right\rfloor, i \ge 2$ in the
sum appears with factors $i$
and $i-1$, so the above sum equals
\[\sum{i=1}^K \left\lfloor \frac{n}{p^i} \right\rfloor.\]
\end{proof}
\begin{cor}
\[\sum_{k=1}^K \left\lfloor \frac{n}{p^K} \right\rfloor =\frac{n
-\delta_p(n)}{p-1},\]
where $\delta_p$ denotes the sums of digits function in base $p$.
\end{cor}
\begin{proof}
If $n < p$, then $\delta_p(n) =n$ and $\left\lfloor \frac{n}{p} \right\rfloor$
is $0 =\frac{n -\delta_p(n)}{p-1}$. So we assume $p \leq n$.

Let $n_Kn_{K-1}\cdots n_0$ be the $p$-adic representation of $n$. Then
\begin{eqnarray*}
\frac{n -\delta_p(n)}{p-1} &= \sum_{k=1}^K n_k\left(\sum_{j=0}^{k-1} p^j\right)
\\
&=\sum_{0 \le j < k \le K} p^j.n_k  \\
&=\left(n_1 +n_2p +\ldots +n_Kp^{K-1}\right)
\\
&+\left(n_2 +n_3p +\ldots +n_Kp^{K-2}\right) \\
& \ddots \\
& +n_K \\
&&= \sum_{k=1}^K \left\lfloor \frac{n}{p^k}\right\rfloor.
\end{eqnarray*}
\end{proof}
%%%%%
%%%%%
\end{document}
