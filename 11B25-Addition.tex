\documentclass[12pt]{article}
\usepackage{pmmeta}
\pmcanonicalname{Addition}
\pmcreated{2013-03-22 16:35:28}
\pmmodified{2013-03-22 16:35:28}
\pmowner{PrimeFan}{13766}
\pmmodifier{PrimeFan}{13766}
\pmtitle{addition}
\pmrecord{8}{38786}
\pmprivacy{1}
\pmauthor{PrimeFan}{13766}
\pmtype{Definition}
\pmcomment{trigger rebuild}
\pmclassification{msc}{11B25}
\pmclassification{msc}{00A06}
\pmclassification{msc}{00A05}
\pmrelated{Summation}

% this is the default PlanetMath preamble.  as your knowledge
% of TeX increases, you will probably want to edit this, but
% it should be fine as is for beginners.

% almost certainly you want these
\usepackage{amssymb}
\usepackage{amsmath}
\usepackage{amsfonts}

% used for TeXing text within eps files
%\usepackage{psfrag}
% need this for including graphics (\includegraphics)
%\usepackage{graphicx}
% for neatly defining theorems and propositions
%\usepackage{amsthm}
% making logically defined graphics
%%%\usepackage{xypic}

% there are many more packages, add them here as you need them

% define commands here

\begin{document}
{\em Addition} is a mathematical operation in which two or more numbers are added up. The numbers being added are usually called the \emph{addends}, while the result is usually called the \emph{sum}.  The numbers may be real, imaginary or complex. Two examples: 2 + 2 = 4; $4 + \pi \approx 7.14159$. In the realm of real numbers, adding up positive numbers yields a result that is greater than any of the summands.

The usual symbol for addition is the cross with its four arms of equal length pointing north, east, west and south: +. This operator is used in standard infix notation as well as in Polish notation and reverse Polish notation.  The operation of addition is commutative, so $x + y = y + z$.  Moreover, it is associative, so $x + (y + z) = (x + y) + z$.  
Combining these two facts leads to the observation that if $S$ is a finite set of numbers, then we can add them up one by one in any order without any potential ambiguity in the resulting sum.  This allows us to define the sigma notation 
\[
\sum_{x\in S} x
\] 
for the sum of finitely many numbers.  For example, the sum of the first $n$ positive perfect cubes can be written as
\[
\sum_{i\in\{1,\dots,n\}} i^3.
\]
An alternate way of writing the same sum is
\[
\sum_{i=1}^n i^3.
\]
Here we might imagine the sum as being produced by an iterative process.  But because of the aforementioned commutativity and associativity, it doesn't matter if the iterator is started at the smallest value or the largest value.

A notable property of the iterative sum is that 
\[
\sum_{i = 1}^n i^3 = \biggl(\sum_{i = 1}^n i\biggr)^2.
\]

(See Nicomachus' theorem).

Besides the possibility of overflow, addition presents no problems for fixed point arithmetic provided the operands are representable in fixed point to begin with. In floating point there exists the possibility of loss of precision (if for example we were to add up several different irrational numbers).

\PMlinkescapeword{observation}
\PMlinkescapeword{order}
\PMlinkescapeword{perfect}
\PMlinkescapeword{potential}
\PMlinkescapeword{property}
%%%%%
%%%%%
\end{document}
