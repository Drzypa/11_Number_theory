\documentclass[12pt]{article}
\usepackage{pmmeta}
\pmcanonicalname{KroneckersJugendtraum}
\pmcreated{2013-03-22 15:01:08}
\pmmodified{2013-03-22 15:01:08}
\pmowner{mathcam}{2727}
\pmmodifier{mathcam}{2727}
\pmtitle{Kronecker's Jugendtraum}
\pmrecord{7}{36726}
\pmprivacy{1}
\pmauthor{mathcam}{2727}
\pmtype{Definition}
\pmcomment{trigger rebuild}
\pmclassification{msc}{11R37}

% this is the default PlanetMath preamble.  as your knowledge
% of TeX increases, you will probably want to edit this, but
% it should be fine as is for beginners.

% almost certainly you want these
\usepackage{amssymb}
\usepackage{amsmath}
\usepackage{amsfonts}
\usepackage{amsthm}

% used for TeXing text within eps files
%\usepackage{psfrag}
% need this for including graphics (\includegraphics)
%\usepackage{graphicx}
% for neatly defining theorems and propositions
%\usepackage{amsthm}
% making logically defined graphics
%%%\usepackage{xypic}

% there are many more packages, add them here as you need them

% define commands here

\newcommand{\mc}{\mathcal}
\newcommand{\mb}{\mathbb}
\newcommand{\mf}{\mathfrak}
\newcommand{\ol}{\overline}
\newcommand{\ra}{\rightarrow}
\newcommand{\la}{\leftarrow}
\newcommand{\La}{\Leftarrow}
\newcommand{\Ra}{\Rightarrow}
\newcommand{\nor}{\vartriangleleft}
\newcommand{\Gal}{\text{Gal}}
\newcommand{\GL}{\text{GL}}
\newcommand{\Z}{\mb{Z}}
\newcommand{\R}{\mb{R}}
\newcommand{\Q}{\mb{Q}}
\newcommand{\C}{\mb{C}}
\newcommand{\<}{\langle}
\renewcommand{\>}{\rangle}
\begin{document}
Kronecker's Jugendtraum (Jugendtraum is German for ``youthful dream'') describes a central problem in class field theory, to explicitly describe the abelian extensions of an arbitrary number field $K$ in \PMlinkescapetext{terms} of values of transcendental functions.

Class field theory gives a \PMlinkescapetext{complete} solution to this problem in the case where $K=\mathbb{Q}$, the field of rational numbers.  Specifically, the Kronecker-Weber theorem gives that any \PMlinkescapetext{abelian} number field sits inside one of the cyclotomic fields $\mathbb{Q}(\zeta_n)$ for some $n$.  Refining this only slightly gives that we can explicitly generate all abelian extensions of $\mathbb{Q}$ by adjoining values of the transcendental function $e^{2\pi iz}$ for certain points $z\in \Q/\Z$.

A slightly more complicated example is when $K$ is a quadratic imaginary extension of $\mathbb{Q}$, in which case Kronecker's Jugendtraum has been solved by the theory of ``complex multiplication'' (see CM-field). The specific transcendental functions which generate all these abelian extensions are the $j$-function (as in elliptic curves) and Weber's $w$-function.

Though there are partial results in the cases of CM-fields or real quadratic fields, the problem is largely still \PMlinkname{open}{OpenQuestion}, and earned great prestige by being included as Hilbert's twelfth problem.
%%%%%
%%%%%
\end{document}
