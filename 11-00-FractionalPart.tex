\documentclass[12pt]{article}
\usepackage{pmmeta}
\pmcanonicalname{FractionalPart}
\pmcreated{2013-03-22 16:14:13}
\pmmodified{2013-03-22 16:14:13}
\pmowner{Wkbj79}{1863}
\pmmodifier{Wkbj79}{1863}
\pmtitle{fractional part}
\pmrecord{7}{38338}
\pmprivacy{1}
\pmauthor{Wkbj79}{1863}
\pmtype{Definition}
\pmcomment{trigger rebuild}
\pmclassification{msc}{11-00}
\pmclassification{msc}{26A09}
\pmrelated{IntegerPart}

\usepackage{amssymb}
\usepackage{amsmath}
\usepackage{amsfonts}

\usepackage{psfrag}
\usepackage{graphicx}
\usepackage{amsthm}
%%\usepackage{xypic}

\begin{document}
The \emph{fractional part} of a nonnegative real number is the part of the number that appears after the decimal \PMlinkescapetext{point}.  For example, the fractional part of $\frac{7}{3}$ is $\frac{1}{3}$.

To be more precise, for $x \in \mathbb{R}$ with $x \ge 0$, the fractional part of $x$, denoted as $\{x\}$, is given by $\{x\}=x-[x]$, where $[x]$ denotes the integer part of $x$.

The name ``fractional part'' is somewhat of a misnomer:  To the novice, the name may seem to imply that the result must be a fraction (and therefore rational), which is not the case.  For example, $\{\pi\}=\pi-3$, which is not rational.
%%%%%
%%%%%
\end{document}
