\documentclass[12pt]{article}
\usepackage{pmmeta}
\pmcanonicalname{GreatestCommonDivisor}
\pmcreated{2013-03-22 11:46:50}
\pmmodified{2013-03-22 11:46:50}
\pmowner{CWoo}{3771}
\pmmodifier{CWoo}{3771}
\pmtitle{greatest common divisor}
\pmrecord{22}{30248}
\pmprivacy{1}
\pmauthor{CWoo}{3771}
\pmtype{Definition}
\pmcomment{trigger rebuild}
\pmclassification{msc}{11-00}
\pmclassification{msc}{03E20}
\pmclassification{msc}{06F25}
\pmsynonym{gcd}{GreatestCommonDivisor}
\pmsynonym{greatest common factor}{GreatestCommonDivisor}
\pmsynonym{highest common factor}{GreatestCommonDivisor}
\pmsynonym{hcf}{GreatestCommonDivisor}
%\pmkeywords{number theory}
\pmrelated{GcdDomain}
\pmrelated{CorollaryOfBezoutsLemma}
\pmrelated{ExampleOfGcd}
\pmrelated{ExistenceAndUniquenessOfTheGcdOfTwoIntegers}
\pmrelated{DivisionOfCongruence}
\pmrelated{Congruences}
\pmrelated{RationalSineAndCosine}
\pmrelated{IntegralBasisOfQuadraticField}
\pmrelated{SquareRootsOfRationals}
\pmrelated{IntegerContraharmonicMeans}
\pmrelated{SubgroupsWithCoprim}

\usepackage{amssymb}
\usepackage{amsmath}
\usepackage{amsfonts}
\usepackage{graphicx}
%%%%\usepackage{xypic}
\begin{document}
Let $a$ and $b$ be given integers, with at least one of them different from zero. The \emph{greatest common divisor} of $a$ and $b$, denoted by $\gcd(a,b)$, is the positive integer $d$ satisfying
\begin{enumerate}
\item $d\mid a$ and $d\mid b$,
\item if $c\mid a$ and $c\mid b$, then $c\mid d$.
\end{enumerate}
More intuitively, the greatest common divisor is the largest integer dividing both $a$ and $b$.

For example, $\gcd(345,135)=15$, $\gcd(-7,-21)=7$, $\gcd(18,0)=18$, and $\gcd(44,97)=1$

Given two (rational) integers, one can construct their gcd via Euclidean algorithm.  Once the gcd is found, it can be written as a linear combination of the two integers.  That is, for any two integers $a$ and $b$, we have $\operatorname{gcd}(a,b)=ra+sb$ for some integers $r$ and $s$.  This expression is known as the Bezout identity.

One can generalize the notion of a gcd of two integers into a gcd of two elements in a commutative ring.  However, given two elements in a general commutative ring, even an integral domain, a gcd may not exist.  For example, in $\mathbb{Z}[x^2,x^3]$, a gcd for the elements $x^5$ and $x^6$ does not exist, for $x^2$ and $x^3$ are both common divisors of $x^5$ and $x^6$, but neither one divides another.

The idea of the gcd of two integers can be generalized in another direction: to the gcd of a non-empty set of integers.  If $S$ is a non-empty set of integers, then the $\operatorname{gcd}$ of $S$, is a positive integer $d$ such that 
\begin{enumerate}
\item $d\mid a$ for all $a\in S$,
\item if $c\mid a, \forall a\in S$, then $c\mid d$.
\end{enumerate}
We denote $d=\operatorname{gcd}(S)$.

\textbf{Remarks}.
\begin{itemize}
\item $\gcd(a,b,c)=\gcd(\gcd(a,b),c)$.
\item If $\varnothing\ne T\subseteq S\subseteq \mathbb{Z}$, then $\gcd(S)\mid \gcd(T)$.
\item For any $\varnothing\ne S\subseteq\mathbb{Z}$, there is a finite subset $T\subseteq S$ such that $\gcd(T)=\gcd(S)$.
\end{itemize}
%%%%%
%%%%%
%%%%%
%%%%%
\end{document}
