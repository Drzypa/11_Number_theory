\documentclass[12pt]{article}
\usepackage{pmmeta}
\pmcanonicalname{displaystylesumnleXYomeganOyxlogXy1ForYge0}
\pmcreated{2013-03-22 16:11:22}
\pmmodified{2013-03-22 16:11:22}
\pmowner{Wkbj79}{1863}
\pmmodifier{Wkbj79}{1863}
\pmtitle{$\displaystyle \sum_{n \le x} y^{\omega(n)}=O_y(x(\log x)^{y-1})$ for $y \ge 0$}
\pmrecord{9}{38279}
\pmprivacy{1}
\pmauthor{Wkbj79}{1863}
\pmtype{Theorem}
\pmcomment{trigger rebuild}
\pmclassification{msc}{11N37}
\pmrelated{AsymptoticEstimate}
\pmrelated{DisplaystyleYOmeganOleftFracxlogXy12YRightFor1LeY2}
\pmrelated{WirsingCondition}

\endmetadata

\usepackage{amssymb}
\usepackage{amsmath}
\usepackage{amsfonts}

\usepackage{psfrag}
\usepackage{graphicx}
\usepackage{amsthm}
%%\usepackage{xypic}

\newtheorem{thm*}{Theorem}

\begin{document}
Within this entry, $\omega$ refers to the number of distinct prime factors function, $\lfloor \, \cdot \, \rfloor$ refers to the floor function, $\log$ refers to the natural logarithm, $p$ refers to a prime, and $k$ and $n$ refer to positive integers.

\begin{thm*} For $y \ge 0$, $\displaystyle \sum_{n \le x} y^{\omega(n)}=O_y(x(\log x)^{y-1})$.
\end{thm*}

\begin{proof}
Since $y^{\omega(p^k)}=y$ for all $p$ and $k$, the real-valued nonnegative multiplicative function $y^{\omega(n)}$ \PMlinkescapetext{satisfies} the Wirsing condition with $c=y$ and $\lambda=1$.  Thus:

\begin{center}
\begin{tabular}{ll}
$\displaystyle \sum_{n \le x} y^{\omega(n)}$ & $\displaystyle =O_y \left( \frac{x}{\log x} \sum_{n \le x} \frac{y^{\omega(n)}}{n} \right)$ \\
& $\displaystyle =O_y \left( \frac{x}{\log x} \prod_{p \le x} \left( 1+\sum_{k=1}^{\left\lfloor \frac{\log x}{\log p} \right\rfloor } \frac{y^{\omega(p^k)}}{p^k} \right) \right)$ \\
& $\displaystyle =O_y \left( \frac{x}{\log x} \left( \exp \left( \sum_{p \le x} \sum_{k=1}^{\left\lfloor \frac{\log x}{\log p} \right\rfloor } \frac{y}{p^k} \right) \right) \right)$ \\
& $\displaystyle =O_y \left( \frac{x}{\log x} \left( \exp \left( y \sum_{p \le x} \sum_{k=1}^{\left\lfloor \frac{\log x}{\log p} \right\rfloor } \frac{1}{p^k} \right) \right) \right)$ \\
& $\displaystyle =O_y \left( \frac{x}{\log x} ( \exp (y(\log(\log x)+O(1)))) \right)$ \\
& $\displaystyle =O_y \left( \frac{x}{\log x} ( \exp (\log (\log x)^y)) \right)$ \\
& $\displaystyle =O_y \left( \frac{x}{\log x} (\log x)^y \right)$ \\
& $\displaystyle =O_y(x(\log x)^{y-1})$ \end{tabular}
\end{center}
\end{proof}
%%%%%
%%%%%
\end{document}
