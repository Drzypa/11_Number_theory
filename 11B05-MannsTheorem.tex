\documentclass[12pt]{article}
\usepackage{pmmeta}
\pmcanonicalname{MannsTheorem}
\pmcreated{2013-03-22 13:20:32}
\pmmodified{2013-03-22 13:20:32}
\pmowner{bbukh}{348}
\pmmodifier{bbukh}{348}
\pmtitle{Mann's theorem}
\pmrecord{5}{33857}
\pmprivacy{1}
\pmauthor{bbukh}{348}
\pmtype{Theorem}
\pmcomment{trigger rebuild}
\pmclassification{msc}{11B05}
\pmclassification{msc}{11B13}
\pmsynonym{$(\alpha+\beta)$-conjecture}{MannsTheorem}
\pmrelated{SchnirlemannDensity}

\usepackage{amssymb}
\usepackage{amsmath}
\usepackage{amsfonts}
\begin{document}
Let $A$ and $B$ be subsets of $\mathbb{Z}$. If $0 \in A \cap B$,
\begin{equation*}
\sigma(A+B)\geq \min(1,\sigma A + \sigma B),
\end{equation*}
where $\sigma$ denotes Schnirelmann density.

This statement was known also as $(\alpha+\beta)$-conjecture until H.~B. Mann proved it in 1942.
%%%%%
%%%%%
\end{document}
