\documentclass[12pt]{article}
\usepackage{pmmeta}
\pmcanonicalname{CongruenceOfArbitraryDegree}
\pmcreated{2013-03-22 18:52:29}
\pmmodified{2013-03-22 18:52:29}
\pmowner{pahio}{2872}
\pmmodifier{pahio}{2872}
\pmtitle{congruence of arbitrary degree}
\pmrecord{10}{41722}
\pmprivacy{1}
\pmauthor{pahio}{2872}
\pmtype{Theorem}
\pmcomment{trigger rebuild}
\pmclassification{msc}{11A05}
\pmclassification{msc}{11A07}
\pmrelated{SufficientConditionOfPolynomialCongruence}
\pmrelated{APolynomialOfDegreeNOverAFieldHasAtMostNRoots}

% this is the default PlanetMath preamble.  as your knowledge
% of TeX increases, you will probably want to edit this, but
% it should be fine as is for beginners.

% almost certainly you want these
\usepackage{amssymb}
\usepackage{amsmath}
\usepackage{amsfonts}

% used for TeXing text within eps files
%\usepackage{psfrag}
% need this for including graphics (\includegraphics)
%\usepackage{graphicx}
% for neatly defining theorems and propositions
 \usepackage{amsthm}
% making logically defined graphics
%%%\usepackage{xypic}

% there are many more packages, add them here as you need them

% define commands here

\theoremstyle{definition}
\newtheorem*{thmplain}{Theorem}

\begin{document}
\textbf{Theorem.}\; A congruence of $n$th degree and modulo a prime number has at most $n$ incongruent roots.\\

\emph{Proof.}\, In the case \,$n = 1$,\, the assertion turns out from the entry linear congruence.\, We make the induction hypothesis, that the assertion is true for congruences of degree less than $n$.\, 

We suppose now that the congruence
\begin{align}
f(x) \;:=\; a_nx^n+a_{n-1}x^{n-1}+\ldots+a_0 \;\equiv\; 0 \pmod{p},
\end{align}
where\, $p \nmid a_n$,\, has at least $n$ incongruent roots $x_1,\,x_2,\,\ldots,\,x_n$.\, Form the congruence
\begin{align}
f(x) \;\equiv\; a_n(x-x_1)(x-x_2)\cdots(x-x_n) \pmod{p}.
\end{align}
Both sides have the same term $a_nx^n$ of the highest degree, whence they may be cancelled from the congruence and the degree of (2) has a lower degree than $n$.\, Because (2), however, clearly has $n$ incongruent roots 
$x_1,\,x_2,\,\ldots,\,x_n$,\, it must by the induction hypothesis be simplifiable to the form\, $0 \equiv 0 \pmod{p}$\, and thus be an identical congruence.

Now, if the congruence (1) had an additional incongruent root $x_{n+1}$, i.e.\, $P(x_{n+1}) \equiv 0 \pmod{p}$, then the identical congruence (2) would imply
$$a_n(x_{n+1}-x_1)(x_{n+1}-x_2)\cdots(x_{n+1}-x_n) \;\equiv\; 0 \pmod{p}.$$
Yet, this is impossible, since no one of the \PMlinkname{factors}{Product} of the left hand side is divisible by $p$.\, This settles the induction proof.\\

Cf. \PMlinkexternal{SpringerLink}{http://eom.springer.de/c/c024860.htm}.\\

\textbf{Example.}\, When\, $f(x) := x^5\!+\!x\!+\!1 \equiv 0 \pmod{7}$,\, we have\\
$f(0) \equiv 1 \pmod{7}$,\\
$f(1) \equiv 3 \pmod{7}$,\\
$f(2) \equiv 32+2+1 \equiv 0 \pmod{7}$,\\
$f(3) \equiv 27\cdot9+3+1 \equiv -1\cdot2+4 \equiv 2 \pmod{7}$,\\
$f(4) \equiv (-3)^5+4+1 \equiv +2+5 \equiv 0 \pmod{7}$,\\
$f(5) \equiv (-2)^5+5+1 \equiv -32+6 \equiv -26 \equiv 2 \pmod{7}$,\\
$f(6) \equiv (-1)^5+6+1 \equiv 6 \pmod{7}$.\\
Thus only the representants 2 and 4 of a complete residue system modulo 7 (see conditional congruences) are roots of the given congruense.\, A congruence needs not have the maximal amount of incongruent roots mentionned in the theorem.






\begin{thebibliography}{9}
\bibitem{K.V.} {\sc K. V\"ais\"al\"a}: {\em Lukuteorian ja korkeamman algebran alkeet}.\, Tiedekirjasto No. 17.\quad  Kustannusosakeyhti\"o Otava, Helsinki (1950).
\end{thebibliography}



%%%%%
%%%%%
\end{document}
