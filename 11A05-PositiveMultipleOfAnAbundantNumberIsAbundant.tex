\documentclass[12pt]{article}
\usepackage{pmmeta}
\pmcanonicalname{PositiveMultipleOfAnAbundantNumberIsAbundant}
\pmcreated{2013-03-22 16:17:07}
\pmmodified{2013-03-22 16:17:07}
\pmowner{Mathprof}{13753}
\pmmodifier{Mathprof}{13753}
\pmtitle{positive multiple of an abundant number is abundant}
\pmrecord{13}{38400}
\pmprivacy{1}
\pmauthor{Mathprof}{13753}
\pmtype{Theorem}
\pmcomment{trigger rebuild}
\pmclassification{msc}{11A05}
\pmrelated{TheoremOnMultiplesOfAbundantNumbers}

% this is the default PlanetMath preamble.  as your knowledge
% of TeX increases, you will probably want to edit this, but
% it should be fine as is for beginners.

% almost certainly you want these
\usepackage{amssymb}
\usepackage{amsmath}
\usepackage{amsfonts}

% used for TeXing text within eps files
%\usepackage{psfrag}
% need this for including graphics (\includegraphics)
%\usepackage{graphicx}
% for neatly defining theorems and propositions
%\usepackage{amsthm}
% making logically defined graphics
%%%\usepackage{xypic}

% there are many more packages, add them here as you need them

% define commands here

\begin{document}
{\bf Theorem.} A positive multiple of an abundant number is abundant. \\
\emph{Proof.} Let $n$ be abundant and $m >0$ be an integer. We have to show
that $\sigma (mn) > 2mn$,  where $\sigma(n)$ is the sum of the positive divisors of $n$.
 Let $d_1, \ldots,  d_k$ be the positive divisors of $n$. Then
certainly $md_1, \ldots, md_k$ are distinct divisors of $mn$. The result is
clear if $m=1$, so assume $m>1$. Then
\begin{eqnarray*}
\sigma(mn) &>& 1+ \sum_{i=1}^k md_i \\
&>& m \sum_{i=1}^k d_i \\
&>& m(2n) \\
&=& 2mn.
\end{eqnarray*}

As a corollary, the positive abundant numbers form a semigroup.


 
%%%%%
%%%%%
\end{document}
