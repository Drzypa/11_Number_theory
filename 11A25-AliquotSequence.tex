\documentclass[12pt]{article}
\usepackage{pmmeta}
\pmcanonicalname{AliquotSequence}
\pmcreated{2013-03-22 16:07:14}
\pmmodified{2013-03-22 16:07:14}
\pmowner{PrimeFan}{13766}
\pmmodifier{PrimeFan}{13766}
\pmtitle{aliquot sequence}
\pmrecord{5}{38189}
\pmprivacy{1}
\pmauthor{PrimeFan}{13766}
\pmtype{Definition}
\pmcomment{trigger rebuild}
\pmclassification{msc}{11A25}

% this is the default PlanetMath preamble.  as your knowledge
% of TeX increases, you will probably want to edit this, but
% it should be fine as is for beginners.

% almost certainly you want these
\usepackage{amssymb}
\usepackage{amsmath}
\usepackage{amsfonts}

% used for TeXing text within eps files
%\usepackage{psfrag}
% need this for including graphics (\includegraphics)
%\usepackage{graphicx}
% for neatly defining theorems and propositions
%\usepackage{amsthm}
% making logically defined graphics
%%%\usepackage{xypic}

% there are many more packages, add them here as you need them

% define commands here

\begin{document}
For a given $m$, define the recurrence relation $a_1 = m$, $a_n = \sigma(a_{n - 1}) - a_{n - 1}$, where $\sigma(x)$ is the sum of divisors function. $a$ is then the {\em aliquot sequence} of $m$.

If $m$ is an amicable number, its aliquot sequence is periodic, alternating between the abundant and deficient member of the amicable pair. For a prime number $p$, its aliquot sequence is $p, 1, 0$. In other cases, the aliquot sequence reaches a fixed point upon 0, or on a perfect number.
%%%%%
%%%%%
\end{document}
