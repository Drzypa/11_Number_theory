\documentclass[12pt]{article}
\usepackage{pmmeta}
\pmcanonicalname{SieveOfEratosthenes1}
\pmcreated{2013-03-22 14:07:40}
\pmmodified{2013-03-22 14:07:40}
\pmowner{bbukh}{348}
\pmmodifier{bbukh}{348}
\pmtitle{sieve of Eratosthenes}
\pmrecord{10}{35535}
\pmprivacy{1}
\pmauthor{bbukh}{348}
\pmtype{Definition}
\pmcomment{trigger rebuild}
\pmclassification{msc}{11N35}
\pmrelated{PrincipleOfInclusionExclusion}
\pmrelated{BrunsPureSieve}

\endmetadata

\usepackage{amssymb}
\usepackage{amsmath}
\usepackage{amsfonts}

%%%\usepackage{xypic}

\makeatletter
\@ifundefined{bibname}{}{\renewcommand{\bibname}{References}}
\makeatother

\newcommand*{\abs}[1]{\left\lvert #1\right\rvert}
\begin{document}
\PMlinkescapeword{order}
\PMlinkescapeword{term}
\PMlinkescapeword{estimate}

The \emph{sieve of Eratosthenes} is the number-theoretic version of the
principle of inclusion-exclusion. In the typical application of
the sieve of Eratosthenes one is concerned with estimating the
number of elements of a set $\mathcal{A}$ that are not divisible
by any of the primes in the set $\mathcal{P}$.

\PMlinkname{M\"obius inversion formula}{MobiusInversionFormula} can be written as
\begin{equation}\label{eq:mobius}
\sum_{\substack{d\mid n\\d\mid P}}\mu(d)=\sum_{d\mid \gcd(n,P)}
\mu(d)=\begin{cases}1,&\text{if }\gcd(n,P)=1,\\0,&\text{if
}\gcd(n,P)>1.\end{cases}
\end{equation}

If we set $P=\prod_{p\in\mathcal{P}} p$ then \eqref{eq:mobius} is
the \PMlinkname{characteristic function}{CharacteristicFunction} of the set of numbers that are not
divisible by any of the primes in $\mathcal{P}$. If we set $A_d$
to be the number of elements of $A$ that are divisible by $d$,
then the number of elements of $A$ that are not divisible by any
primes in $\mathcal{P}$ can be expressed as
\begin{equation}\label{eq:sieve}
\sum_{n\in\mathcal{A}}\sum_{\substack{d\mid n\\d\mid
P}}\mu(d)=\sum_{d\mid
P}\mu(d)\sum_{\substack{n\in\mathcal{A}\\d\mid n}}1=\sum_{d\mid
P}\mu(d)A_d.
\end{equation}

\emph{Example.} To establish an upper bound on the number of primes
less than $x$ we set $\mathcal{A}=\{y,y+1,\dotsc,x\}$ and $P=\{p :
p \text{ is a prime less than }y\}$. Then \eqref{eq:sieve} counts
the number of integers between $y$ and $x$ that are not divisible
by any prime less than $y$. Hence the number of primes not
exceeding $x$ is
\begin{align*}
\pi(x)&\leq \sum_{d\mid P}\mu(d) \bigl((x-y)/d+O(1)\bigr)\\
      &\leq x\sum_{d\mid P}\frac{\mu(d)}{d}+O(\abs{P})\\
      &= x\prod_{p\leq y} (1-1/p)+O(2^y).
\end{align*}
By Mertens' estimate for $\prod (1-1/p)$ the main term is
\begin{equation*}
x\frac{e^{-\gamma}+o(1)}{\log y},
\end{equation*}
where $\gamma$ is Euler's constant. Setting $y=\tfrac{1}{2}\log x$
we obtain
\begin{equation*}
\pi(x)\leq \bigl(e^{-\gamma}+o(1)\bigr)\frac{x}{\log\log x}.
\end{equation*}
The obtained bound is clearly inferior to the prime number
theorem, but the method applies to problems that are not tractable
by the other techniques.

The error term of the order $O(2^y)$ can be significantly reduced
by using more robust sifting procedures such as \PMlinkname{Brun's}{BrunsPureSieve} and Selberg's sieves. For detailed exposition
of various sieve methods see monographs
\cite{cite:greaves_sieves,cite:halberstam_richert_sieves,cite:hooley_appsievemethods}.

\begin{thebibliography}{1}

\bibitem{cite:greaves_sieves}
George Greaves.
\newblock {\em Sieves in number theory}.
\newblock Springer, 2001.
\newblock \PMlinkexternal{Zbl
  1003.11044}{http://www.emis.de/cgi-bin/zmen/ZMATH/en/quick.html?type=html&an=1003.11044}.

\bibitem{cite:halberstam_richert_sieves}
H.~Halberstam and H.-E. Richert.
\newblock {\em Sieve methods}, volume~4 of {\em London Mathematical Society
  Monographs}.
\newblock 1974.
\newblock \PMlinkexternal{Zbl
  0298.10026}{http://www.emis.de/cgi-bin/zmen/ZMATH/en/quick.html?type=html&an=0298.10026}.

\bibitem{cite:hooley_appsievemethods}
C.~Hooley.
\newblock {\em Application of sieve methods to the theory of numbers},
  volume~70 of {\em Cambridge Tracts in Mathematics}.
\newblock Cambridge Univ. Press, 1976.
\newblock \PMlinkexternal{Zbl
  0327.10044}{http://www.emis.de/cgi-bin/zmen/ZMATH/en/quick.html?type=html&an=0327.10044}.

\end{thebibliography}
%%%%%
%%%%%
\end{document}
