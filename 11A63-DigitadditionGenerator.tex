\documentclass[12pt]{article}
\usepackage{pmmeta}
\pmcanonicalname{DigitadditionGenerator}
\pmcreated{2013-03-22 15:56:12}
\pmmodified{2013-03-22 15:56:12}
\pmowner{PrimeFan}{13766}
\pmmodifier{PrimeFan}{13766}
\pmtitle{digitaddition generator}
\pmrecord{5}{37945}
\pmprivacy{1}
\pmauthor{PrimeFan}{13766}
\pmtype{Definition}
\pmcomment{trigger rebuild}
\pmclassification{msc}{11A63}
\pmsynonym{digit addition generator}{DigitadditionGenerator}
\pmsynonym{digit-addition generator}{DigitadditionGenerator}

% this is the default PlanetMath preamble.  as your knowledge
% of TeX increases, you will probably want to edit this, but
% it should be fine as is for beginners.

% almost certainly you want these
\usepackage{amssymb}
\usepackage{amsmath}
\usepackage{amsfonts}

% used for TeXing text within eps files
%\usepackage{psfrag}
% need this for including graphics (\includegraphics)
%\usepackage{graphicx}
% for neatly defining theorems and propositions
%\usepackage{amsthm}
% making logically defined graphics
%%%\usepackage{xypic}

% there are many more packages, add them here as you need them

% define commands here

\begin{document}
Given an integer $m$ consisting of $k$ digits $d_x$ in base $b$, it follows that $$m + \sum_{i = 0}^{k - 1} d_{i + 1}b^i = n$$, another integer. Then $m$ is said to be the {\em digitaddition generator} of $n$.

In a randomly chosen range of $2b$ consecutive integers, most will have a digitaddition generator and one or two might have none (such integers are called self numbers). If the range falls near a multiple of $b^2$, it might contain a few numbers with two digitaddition generators. If the range includes $0 < n < b$ and $2|b$, the $n \not\vert 2$ will lack digitaddition generators.
%%%%%
%%%%%
\end{document}
