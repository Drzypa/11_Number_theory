\documentclass[12pt]{article}
\usepackage{pmmeta}
\pmcanonicalname{Googolplex}
\pmcreated{2013-03-22 12:25:33}
\pmmodified{2013-03-22 12:25:33}
\pmowner{PrimeFan}{13766}
\pmmodifier{PrimeFan}{13766}
\pmtitle{googolplex}
\pmrecord{7}{32457}
\pmprivacy{1}
\pmauthor{PrimeFan}{13766}
\pmtype{Definition}
\pmcomment{trigger rebuild}
\pmclassification{msc}{11-00}
\pmrelated{Googol}

\endmetadata

%\usepackage{graphicx}
%%%\usepackage{xypic} 
\usepackage{bbm}
\newcommand{\Z}{\mathbbmss{Z}}
\newcommand{\C}{\mathbbmss{C}}
\newcommand{\R}{\mathbbmss{R}}
\newcommand{\Q}{\mathbbmss{Q}}
\newcommand{\mathbb}[1]{\mathbbmss{#1}}
\begin{document}
A \emph{googolplex} is $10^{10^{100}}$; that is, 10 raised to the googol-th power. This can also be viewed as a one followed by a googol of zeros at its right. A googoplex is much larger than a googol. In fact, since a googolplex has 
a googol number of zeros in its decimal representation, a googoplex has more digits than there are atoms in our universe. Thus, even if all matter in the universe were at one's disposal, it would not be possible to write down the decimal~representation of a googolplex.
%%%%%
%%%%%
\end{document}
