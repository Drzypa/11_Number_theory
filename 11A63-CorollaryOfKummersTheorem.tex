\documentclass[12pt]{article}
\usepackage{pmmeta}
\pmcanonicalname{CorollaryOfKummersTheorem}
\pmcreated{2013-03-22 13:23:07}
\pmmodified{2013-03-22 13:23:07}
\pmowner{Thomas Heye}{1234}
\pmmodifier{Thomas Heye}{1234}
\pmtitle{corollary of Kummer's theorem}
\pmrecord{7}{33921}
\pmprivacy{1}
\pmauthor{Thomas Heye}{1234}
\pmtype{Corollary}
\pmcomment{trigger rebuild}
\pmclassification{msc}{11A63}

\endmetadata

% this is the default PlanetMath preamble.  as your knowledge
% of TeX increases, you will probably want to edit this, but
% it should be fine as is for beginners.

% almost certainly you want these
\usepackage{amssymb}
\usepackage{amsmath}
\usepackage{amsfonts}

% used for TeXing text within eps files
%\usepackage{psfrag}
% need this for including graphics (\includegraphics)
%\usepackage{graphicx}
% for neatly defining theorems and propositions
%\usepackage{amsthm}
% making logically defined graphics
%%%\usepackage{xypic}

% there are many more packages, add them here as you need them

% define commands here
\begin{document}
As shown in Kummer's theorem, the power of a prime number $p$ dividing
$\binom{n}{m}, n \ge m \in \mathbb{N}$, was the total number of carries when
adding $m$ and $n-m$ in base $p$. We'll give a recurrence relation for the carry
indicator.

Given integers $n \ge m \ge 0$ and a prime number $p$, let $n_i, m_i, r_i$ be
the $i$-th digit of $n,m$, and $r:=n-m$, respectively.

Define $c_{-1} =0$, and
\begin{displaymath}
c_i =\begin{cases}
1 & \text{if $m_i +r_i \ge p$,}\\
0 & \text{otherwise}
\end{cases}
\end{displaymath}
for each $i\ge 0$ up to the number of digits of $n$.

For each $i \ge 0$ we have
\begin{displaymath}
n_i =m_i +r_i +c_{i-1} -p.c_i.
\end{displaymath}
Starting with the $i$-th digit of $n$, we multiply with increasing powers of $p$
to get
\begin{displaymath}
\sum_{k=i}^d n_kp^{k-i} =\left(\sum_{k=i}^d p^{k-i}(m_k +r_k)\right)
+\sum_{k=i}^d \left(p^{k-1-(i-1)}c_{k-1} -p^{k-(i-1)}c_k\right).
\end{displaymath}
The last sum in the above equation leaves only the values for indices $i$ and
$d$, and we get
\begin{equation}
\label{L1}
\left\lfloor \frac{n}{p^i}\right\rfloor =\left\lfloor \frac{m}{p^i}\right\rfloor
+\left\lfloor \frac{r}{p^i}\right\rfloor +c_{i-1}
\end{equation}
for all $i \ge 0$.
%%%%%
%%%%%
\end{document}
