\documentclass[12pt]{article}
\usepackage{pmmeta}
\pmcanonicalname{AdditionChain}
\pmcreated{2013-03-22 18:27:27}
\pmmodified{2013-03-22 18:27:27}
\pmowner{PrimeFan}{13766}
\pmmodifier{PrimeFan}{13766}
\pmtitle{addition chain}
\pmrecord{4}{41121}
\pmprivacy{1}
\pmauthor{PrimeFan}{13766}
\pmtype{Definition}
\pmcomment{trigger rebuild}
\pmclassification{msc}{11B13}

\endmetadata

% this is the default PlanetMath preamble.  as your knowledge
% of TeX increases, you will probably want to edit this, but
% it should be fine as is for beginners.

% almost certainly you want these
\usepackage{amssymb}
\usepackage{amsmath}
\usepackage{amsfonts}

% used for TeXing text within eps files
%\usepackage{psfrag}
% need this for including graphics (\includegraphics)
%\usepackage{graphicx}
% for neatly defining theorems and propositions
%\usepackage{amsthm}
% making logically defined graphics
%%%\usepackage{xypic}

% there are many more packages, add them here as you need them

% define commands here

\begin{document}
An {\em addition chain} $a$ is a sequence of integers of length $k$ such that each term $a_i$ for $0 < i \leq k$ (with $a_0 = 1$) is the sum of two previous terms in at least one way. In the sum $a_m + a_n$ it is not required that $m \neq n$. For example, 1, 2, 3, 5, 10, 20, 40, 80, is an addition chain of length 7: 3 is 1 + 2, 5 = 2 + 3, 10 = 5 + 5, and the rest have $m = n$.

There are various subclassifications of addition chains, such as the Lucas chains. A Mian-Chowla sequence is an addition chain with the restriction that each term is the sum of two previous terms in only one way. The length may be infinite, and thus the Fibonacci sequence is an addition chain.
%%%%%
%%%%%
\end{document}
