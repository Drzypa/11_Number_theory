\documentclass[12pt]{article}
\usepackage{pmmeta}
\pmcanonicalname{NumberOfTheBeast}
\pmcreated{2013-03-22 16:29:33}
\pmmodified{2013-03-22 16:29:33}
\pmowner{CompositeFan}{12809}
\pmmodifier{CompositeFan}{12809}
\pmtitle{number of the beast}
\pmrecord{5}{38666}
\pmprivacy{1}
\pmauthor{CompositeFan}{12809}
\pmtype{Definition}
\pmcomment{trigger rebuild}
\pmclassification{msc}{11A99}
\pmsynonym{beast number}{NumberOfTheBeast}

\endmetadata

% this is the default PlanetMath preamble.  as your knowledge
% of TeX increases, you will probably want to edit this, but
% it should be fine as is for beginners.

% almost certainly you want these
\usepackage{amssymb}
\usepackage{amsmath}
\usepackage{amsfonts}

% used for TeXing text within eps files
%\usepackage{psfrag}
% need this for including graphics (\includegraphics)
%\usepackage{graphicx}
% for neatly defining theorems and propositions
%\usepackage{amsthm}
% making logically defined graphics
%%%\usepackage{xypic}

% there are many more packages, add them here as you need them

% define commands here

\begin{document}
In the Book of Revelation of the {\it Holy Bible}, verse 13:18 in the King James Version reads ``Let him that hath understanding count the number of the beast: for it is the number of a man; and his number is Six hundred threescore and six.'' The translation by Ulrich Zwingli gives the number of the beast as 616. Theologians have given various different explanations for these numbers.

If 666 appears more mathematically interesting than 616, it could be because professional mathematicians and amateurs alike have studied it more. Thus it has been noticed that the sum of the squares of the first seven primes is 666, that 666 is a doubly triangular number, that the 666th member of the Padovan sequence is a prime, etc. 666 would be a pandigital number in Roman numerals if it weren't for the fact that it lacks the 'digit' M. In base 10, the number 666 is a Smith number and also a repdigit (and a palidromic number as a trivial consequence of that).

Whereas 666 points to a prime in the Padovan sequence, 616 is itself a member of that sequence. Other than that, there doesn't seem to be much about 616 that is interesting. The T-shirt that jokes ``333: I'm Only Half Evil'' would be a much less funny 308.
%%%%%
%%%%%
\end{document}
