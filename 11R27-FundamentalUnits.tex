\documentclass[12pt]{article}
\usepackage{pmmeta}
\pmcanonicalname{FundamentalUnits}
\pmcreated{2014-11-24 16:38:36}
\pmmodified{2014-11-24 16:38:36}
\pmowner{pahio}{2872}
\pmmodifier{pahio}{2872}
\pmtitle{fundamental units}
\pmrecord{22}{36080}
\pmprivacy{1}
\pmauthor{pahio}{2872}
\pmtype{Definition}
\pmcomment{trigger rebuild}
\pmclassification{msc}{11R27}
\pmclassification{msc}{11R04}
%\pmkeywords{Dirichlet's unit theorem}
\pmrelated{NumberField}
\pmrelated{AlgebraicInteger}

% this is the default PlanetMath preamble.  as your knowledge
% of TeX increases, you will probably want to edit this, but
% it should be fine as is for beginners.

% almost certainly you want these
\usepackage{amssymb}
\usepackage{amsmath}
\usepackage{amsfonts}

% used for TeXing text within eps files
%\usepackage{psfrag}
% need this for including graphics (\includegraphics)
%\usepackage{graphicx}
% for neatly defining theorems and propositions
%\usepackage{amsthm}
% making logically defined graphics
%%%\usepackage{xypic}

% there are many more packages, add them here as you need them

% define commands here
\begin{document}
The ring $R$ of algebraic integers of any algebraic number 
field contains a finite set 
$H = \{\eta_1,\, \eta_2,\, \ldots,\, \eta_t\}$ of so-called 
{\em fundamental units} such that every unit $\varepsilon$ of 
$R$ is a \PMlinkname{power}{GeneralAssociativity} product of 
these, multiplied by a root of unity:
   $$\varepsilon = \zeta\!\cdot\!\eta_1^{k_1}\eta_2^{k_2}\ldots\eta_t^{k_t}$$
Conversely, every such element $\varepsilon$ of the field is a 
unit of $R$.

Examples:\, units of quadratic fields,\, \PMlinkname{units of certain cubic fields}{UnitsOfRealCubicFieldsWithExactlyOneRealEmbedding}


For some algebraic number fields, such as all imaginary 
quadratic fields, the set $H$ may be empty ($t = 0$).\, In the 
case of a single fundamental unit ($t = 1$), which occurs e.g. 
in all 
\PMlinkname{real quadratic fields}{ImaginaryQuadraticField}, 
there are two alternative units 
$\eta$ and its conjugate $\overline{\eta}$ which one can use as 
fundamental unit; then we can speak of {\em the} uniquely 
determined fundamental unit $\eta_1$ which is greater than 1.
%%%%%
%%%%%
\end{document}
