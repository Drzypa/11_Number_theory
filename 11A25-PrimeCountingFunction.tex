\documentclass[12pt]{article}
\usepackage{pmmeta}
\pmcanonicalname{PrimeCountingFunction}
\pmcreated{2013-03-22 12:49:00}
\pmmodified{2013-03-22 12:49:00}
\pmowner{XJamRastafire}{349}
\pmmodifier{XJamRastafire}{349}
\pmtitle{prime counting function}
\pmrecord{13}{33138}
\pmprivacy{1}
\pmauthor{XJamRastafire}{349}
\pmtype{Definition}
\pmcomment{trigger rebuild}
\pmclassification{msc}{11A25}
\pmclassification{msc}{11A41}
\pmclassification{msc}{11N05}
%\pmkeywords{number theory}
\pmrelated{LogarithmicIntegral2}

% this is the default PlanetMath preamble.  as your knowledge
% of TeX increases, you will probably want to edit this, but
% it should be fine as is for beginners.

% almost certainly you want these
\usepackage{amssymb}
\usepackage{amsmath}
\usepackage{amsfonts}

% used for TeXing text within eps files
%\usepackage{psfrag}
% need this for including graphics (\includegraphics)
%\usepackage{graphicx}
% for neatly defining theorems and propositions
%\usepackage{amsthm}
% making logically defined graphics
%%%\usepackage{xypic}

% there are many more packages, add them here as you need them

% define commands here
\begin{document}
The {\it prime counting function} is a non-multiplicative function for any positive real number $x$, denoted as $\pi(x)$ and gives the number of primes not exceeding $x$. It usually takes a positive integer $n$ for an argument. The first few values of $\pi(n)$ for $n = 1, 2, 3, \ldots $ are $0, 1, 2, 2, 3, 3, 4, 4, 4, 4, 5, 5, 6, 6, 6, 6, 7, 7, 8, 8 \ldots $ (\PMlinkexternal{OEIS  A000720}{http://www.research.att.com/~njas/sequences/eisA.cgi?Anum=000720}
).

The asymptotic behavior of $\pi(x) \sim x/\ln x$ is given by the prime number theorem. This function is closely related with {\it Chebyshev's functions} $\vartheta(x)$ and $\psi(x)$.
%%%%%
%%%%%
\end{document}
