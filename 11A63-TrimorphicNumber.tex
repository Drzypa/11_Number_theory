\documentclass[12pt]{article}
\usepackage{pmmeta}
\pmcanonicalname{TrimorphicNumber}
\pmcreated{2013-03-22 16:21:32}
\pmmodified{2013-03-22 16:21:32}
\pmowner{CompositeFan}{12809}
\pmmodifier{CompositeFan}{12809}
\pmtitle{trimorphic number}
\pmrecord{5}{38495}
\pmprivacy{1}
\pmauthor{CompositeFan}{12809}
\pmtype{Definition}
\pmcomment{trigger rebuild}
\pmclassification{msc}{11A63}

\endmetadata

% this is the default PlanetMath preamble.  as your knowledge
% of TeX increases, you will probably want to edit this, but
% it should be fine as is for beginners.

% almost certainly you want these
\usepackage{amssymb}
\usepackage{amsmath}
\usepackage{amsfonts}

% used for TeXing text within eps files
%\usepackage{psfrag}
% need this for including graphics (\includegraphics)
%\usepackage{graphicx}
% for neatly defining theorems and propositions
%\usepackage{amsthm}
% making logically defined graphics
%%%\usepackage{xypic}

% there are many more packages, add them here as you need them

% define commands here

\begin{document}
Given a base $b$ integer $$n = \sum_{i = 1}^k d_ib^{i - 1}$$ where $d_1$ is the least significant digit and $d_k$ is the most significant, if it's also the case that the $k$ least significant digits of $n^3$ are the same of those of $n$, then $n$ is called a {\em trimorphic number}.

All automorphic numbers (with $m = 1$) are also trimorphic numbers, but not all trimorphic numbers are 1-automorphic.
%%%%%
%%%%%
\end{document}
