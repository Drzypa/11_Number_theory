\documentclass[12pt]{article}
\usepackage{pmmeta}
\pmcanonicalname{StrictlyNonpalindromicNumber}
\pmcreated{2013-03-22 16:25:12}
\pmmodified{2013-03-22 16:25:12}
\pmowner{PrimeFan}{13766}
\pmmodifier{PrimeFan}{13766}
\pmtitle{strictly non-palindromic number}
\pmrecord{5}{38570}
\pmprivacy{1}
\pmauthor{PrimeFan}{13766}
\pmtype{Definition}
\pmcomment{trigger rebuild}
\pmclassification{msc}{11A63}

\endmetadata

% this is the default PlanetMath preamble.  as your knowledge
% of TeX increases, you will probably want to edit this, but
% it should be fine as is for beginners.

% almost certainly you want these
\usepackage{amssymb}
\usepackage{amsmath}
\usepackage{amsfonts}

% used for TeXing text within eps files
%\usepackage{psfrag}
% need this for including graphics (\includegraphics)
%\usepackage{graphicx}
% for neatly defining theorems and propositions
%\usepackage{amsthm}
% making logically defined graphics
%%%\usepackage{xypic}

% there are many more packages, add them here as you need them

% define commands here

\begin{document}
If for a given integer $n > 0$ there is no base $1 < b < (n - 1)$ such that each digit $d_i = d_{k + 1 - i}$ of $n$ (where $k$ is the number of significant digits of $n$ in base $b$ and $i$ is a simple iterator in the range $0 < i < (k + 1)$), meaning that $n$ is not a palindromic number in any of these bases, then $n$ is called a {\em strictly non-palindromic number}.

Clearly $n > 2$ will be palindromic for $b = n - 1$, and though trivially, this is also true for $b > n$.

6 is the largest composite strictly non-palindromic number. For any other $2|n$, it is easy to find a base in which $n$ is written $22_b$ by simply computing $b = \frac{n}{2} - 1$. For odd composites $n = mp$, where $p$ is an odd prime and $m \ge p$ we can almost always either find that for $b = p - 1$, $n = b^2 + 2b + 1$, or for $b = m - 1$ then $n = pb + p$ and written with two instances of the digit corresponding to $p$ in that base. The one odd case of $n = 9$ is quickly dismissed with $b = 2$.
%%%%%
%%%%%
\end{document}
