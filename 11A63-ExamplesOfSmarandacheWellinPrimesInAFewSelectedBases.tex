\documentclass[12pt]{article}
\usepackage{pmmeta}
\pmcanonicalname{ExamplesOfSmarandacheWellinPrimesInAFewSelectedBases}
\pmcreated{2013-03-22 15:54:43}
\pmmodified{2013-03-22 15:54:43}
\pmowner{PrimeFan}{13766}
\pmmodifier{PrimeFan}{13766}
\pmtitle{examples of Smarandache-Wellin primes in a few selected bases}
\pmrecord{6}{37915}
\pmprivacy{1}
\pmauthor{PrimeFan}{13766}
\pmtype{Example}
\pmcomment{trigger rebuild}
\pmclassification{msc}{11A63}

% this is the default PlanetMath preamble.  as your knowledge
% of TeX increases, you will probably want to edit this, but
% it should be fine as is for beginners.

% almost certainly you want these
\usepackage{amssymb}
\usepackage{amsmath}
\usepackage{amsfonts}

% used for TeXing text within eps files
%\usepackage{psfrag}
% need this for including graphics (\includegraphics)
%\usepackage{graphicx}
% for neatly defining theorems and propositions
%\usepackage{amsthm}
% making logically defined graphics
%%%\usepackage{xypic}

% there are many more packages, add them here as you need them

% define commands here

\begin{document}
In base 10, there are eight known Smarandache-Wellin primes. The first three are 2, 23 and 2357; these are listed in A069151 of Sloane's OEIS. The fourth is approximately $2.357111317192329313741434753 \times 10^{355}$, and the other four are much larger.

The first three binary Smarandache-Wellin primes are 2, 11 and 751, while in hexadecimal the first two are 2 and 144763. There are slightly more in octal: 2, 19, 157 and 440152287140627310757403395684438501207352921. By now it should be quite obvious that 2 is a Smarandache-Wellin prime regardless of the base.

The first few Smarandache-Wellin primes in factorial base are 2, ...

This concept cannot be extended to Roman numerals, as the concatenations yield invalid combinations like IIIIIV and IIIIIVVII.
%%%%%
%%%%%
\end{document}
