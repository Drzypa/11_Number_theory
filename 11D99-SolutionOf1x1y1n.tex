\documentclass[12pt]{article}
\usepackage{pmmeta}
\pmcanonicalname{SolutionOf1x1y1n}
\pmcreated{2013-03-22 16:30:44}
\pmmodified{2013-03-22 16:30:44}
\pmowner{rspuzio}{6075}
\pmmodifier{rspuzio}{6075}
\pmtitle{solution of $1/x + 1/y = 1/n$}
\pmrecord{13}{38689}
\pmprivacy{1}
\pmauthor{rspuzio}{6075}
\pmtype{Theorem}
\pmcomment{trigger rebuild}
\pmclassification{msc}{11D99}

% this is the default PlanetMath preamble.  as your knowledge
% of TeX increases, you will probably want to edit this, but
% it should be fine as is for beginners.

% almost certainly you want these
\usepackage{amssymb}
\usepackage{amsmath}
\usepackage{amsfonts}

% used for TeXing text within eps files
%\usepackage{psfrag}
% need this for including graphics (\includegraphics)
%\usepackage{graphicx}
% for neatly defining theorems and propositions
\usepackage{amsthm}
% making logically defined graphics
%%%\usepackage{xypic}

% there are many more packages, add them here as you need them

% define commands here

\newtheorem{theorem}{Theorem}
\begin{document}
\begin{theorem}
Given an integer $n$, if there exist integers $x$ and $y$ such that
\[ 
\frac{1}{x} + \frac{1}{y} = \frac{1}{n} ,
\]
then one has
\begin{eqnarray*}
x &=& \frac{n(u+v)}{u} \\
y &=& \frac{n(u+v)}{v}
\end{eqnarray*}
where $u$ and $v$ are integers such that $uv$ divides $n$.
\end{theorem}

\begin{proof}
To begin, cross multiply to obtain
\[
xy = n(x+y).
\]

Since this involves setting a product equal to another
product, we can think in terms of factorization. To
clarify things, let us pull out a common factor and
write $x = kv$ and $y = ku$, where $k$ is the greatest
common factor and $u$ is relatively prime to $v$. Then,
cancelling a common factor of $k$, our equation becomes
the following:

\[
kuv = n(u+v)
\]

This is equivalent to

\[
uv \mid n(u+v)
\]

Since $u$ and $v$ are relatively prime, it follows that $u$ is
relatively prime to $u+v$ and that $v$ is relatively prime to
$u+v$ as well. Hence, we must have that $uv$ divides $n$,

Now we can obtain the general solution to the equation.
Write $n = muv$ with $u$ and $v$ relatively prime. Then,
substituting into our equation and cancelling a $u$ and a
$v$, we obtain

\[
k = m(u+v),
\]

so the solution to the original equation is

\begin{eqnarray*}
x &=& mv(u+v) \\
y &=& mu(u+v)
\end{eqnarray*}
Using the definition of $m$, this can be rewritten as
\begin{eqnarray*}
x &=& \frac{n(u+v)}{u} \\
y &=& \frac{n(u+v)}{v}.
\end{eqnarray*}
\end{proof}
%%%%%
%%%%%
\end{document}
