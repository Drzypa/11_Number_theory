\documentclass[12pt]{article}
\usepackage{pmmeta}
\pmcanonicalname{KummersTheorem}
\pmcreated{2013-03-22 13:22:37}
\pmmodified{2013-03-22 13:22:37}
\pmowner{Thomas Heye}{1234}
\pmmodifier{Thomas Heye}{1234}
\pmtitle{Kummer's theorem}
\pmrecord{14}{33909}
\pmprivacy{1}
\pmauthor{Thomas Heye}{1234}
\pmtype{Theorem}
\pmcomment{trigger rebuild}
\pmclassification{msc}{11A63}

% this is the default PlanetMath preamble.  as your knowledge
% of TeX increases, you will probably want to edit this, but
% it should be fine as is for beginners.

% almost certainly you want these
\usepackage{amssymb}
\usepackage{amsmath}
\usepackage{amsfonts}

% used for TeXing text within eps files
%\usepackage{psfrag}
% need this for including graphics (\includegraphics)
%\usepackage{graphicx}
% for neatly defining theorems and propositions
\usepackage{amsthm}
% making logically defined graphics
%%%\usepackage{xypic}

% there are many more packages, add them here as you need them

% define commands here
\begin{document}
Given integers $n \ge m \ge 0$ and a prime number $p$, then the power of $p$
dividing $\binom{n}{m}$ is equal to the number of carries when adding $m$ and
$n -m$ in base $p$.
\begin{proof}
For the proof we can allow \PMlinkescapeword{representations} of numbers in base $p$ with leading
zeros. So let
\begin{eqnarray*}
n_dn_{d-1}\cdots n_0 & := n, \\
m_dm_{d-1}\cdots m_0 & := m,
\end{eqnarray*}
all in base $p$. We set $r=n-m$ and denote the $p$-adic representation of $r$
with $r_dr_{d-1}\cdots r_0$.

We define $c_{-1} =0$, and for each $0 \le j \le d$
\begin{equation}
c_j =\begin{cases}
1 & \text{for $m_j +r_j \ge p$} \\
0 &\text{otherwise.}
\end{cases}
\end{equation}

Finally, we introduce $\delta_p(n)$ as the sum of digits in the $p$-adic
\PMlinkescapeword{representation} of $n$. Then it follows that the power of $p$ dividing $\binom{n}{m}$ is
\[\frac{\delta_p(m) +\delta_p(r) -\delta_p(n)}{p-1}.\]
For each $j \ge 0$, we have
\[n_j =m_j +r_j +c_{j-1} -p.c_j.\]
Then
\begin{eqnarray*}
\delta_p(m) +\delta_p(r) -\delta_p(n) &=\sum_{k=0}^d \left(m_k +r_k -n_k\right)
\\
&=\sum_{k=0}^d \left((p-1)c_j\right)
&+\sum_{k=0}^d \left(c_j
-c_{j-1}\right) \\
&=\sum_{k=0}^d (p-1)c_j&+c_d -c_{-1}
\end{eqnarray*}
This gives us
\[\frac{\delta_p(m) +\delta_p(r) -\delta_p(n)}{p-1} =\sum_{k=0}^d c_k,\]
the total number of carries.
\end{proof}
%%%%%
%%%%%
\end{document}
