\documentclass[12pt]{article}
\usepackage{pmmeta}
\pmcanonicalname{RiemannZetaFunctionHasNoZerosOnReS01}
\pmcreated{2013-03-22 17:54:37}
\pmmodified{2013-03-22 17:54:37}
\pmowner{rm50}{10146}
\pmmodifier{rm50}{10146}
\pmtitle{Riemann zeta function has no zeros on $\Re s=0,1$}
\pmrecord{5}{40404}
\pmprivacy{1}
\pmauthor{rm50}{10146}
\pmtype{Theorem}
\pmcomment{trigger rebuild}
\pmclassification{msc}{11M06}

% this is the default PlanetMath preamble.  as your knowledge
% of TeX increases, you will probably want to edit this, but
% it should be fine as is for beginners.

% almost certainly you want these
\usepackage{amssymb}
\usepackage{amsmath}
\usepackage{amsfonts}

% used for TeXing text within eps files
%\usepackage{psfrag}
% need this for including graphics (\includegraphics)
%\usepackage{graphicx}
% for neatly defining theorems and propositions
%\usepackage{amsthm}
% making logically defined graphics
%%%\usepackage{xypic}

% there are many more packages, add them here as you need them

% define commands here
\newcommand{\Reals}{\mathbb{R}}
\newcommand{\Complex}{\mathbb{C}}
\newtheorem{thm}{Theorem}
\newtheorem{cor}{Corollary}
\begin{document}
\PMlinkescapeword{point}
\PMlinkescapeword{points}
This article shows that the Riemann zeta function $\zeta(s)$ has no zeros along the lines $\Re s=0$ or $\Re s=1$. That implies that all nontrivial zeros of $\zeta(s)$ lie strictly within the critical strip $0<\Re s<1$. As the \PMlinkescapetext{parent} article points out, this is known to be equivalent to one version of the prime number theorem. 

It can in fact be shown that $\zeta(s)\neq 0$ for any $s=\sigma+it$ with $0<\sigma<1$ if
\[\sigma\geq 1-\frac{c}{\log(\lvert t\rvert+1)}\]
for some constant $c$. By using the functional equation
\[
  \pi^{\frac{-s}{2}}\Gamma\left(\frac{s}{2}\right)\zeta(s) =
     \pi^{-\frac{1-s}{2}}\Gamma\left(\frac{1-s}{2}\right)\zeta(1-s)
\]
we have also that $\zeta(\sigma+it)\neq 0$ if
\[\sigma \leq \frac{c}{\log(\lvert t\rvert+1)}\]
Bounding the zeros of $\zeta(s)$ away from $\Re s = 0$, $1$ leads to a version of the prime number theorem with more precise error terms.

\begin{thm} $\zeta(1+it)\neq 0$ for $t\in\Reals$.
\end{thm}

\textbf{Proof.}\ 
Notice that for $\theta\in\Complex$
\begin{equation}\label{eqn:costheta}
0\leq 2(1+\cos\theta)^2 = 2\cos^2\theta + 4\cos \theta + 2 = 3+4\cos\theta+\cos(2\theta)
\end{equation}
If $\sigma=\Re s>1$, then $\zeta(\sigma+it)=\prod_{p\text{ prime}} (1-p^{-\sigma-it})^{-1}$, so that
\[\log\zeta(\sigma+it)=-\sum_{p\text{ prime}} \log(1-p^{-\sigma-it}) = \sum_{p\text{ prime}} \sum_{m=1}^{\infty} \frac{1}{m}p^{-m\sigma-imt}\]
and thus
\[
\log\lvert\zeta(\sigma+it)\rvert = \sum_{p\text{ prime}}\sum_{m=1}^{\infty}\frac{1}{mp^{m\sigma}}\cos(mt\log p)
\]
since the log of the absolute value is the real part of the log.

Using equation \eqref{eqn:costheta}, we then have
\begin{align*}
3\log \zeta(\sigma) + & 4\log\lvert\zeta(\sigma+it)\rvert + \log\lvert\zeta(\sigma+i2t)\rvert \\
&= \sum_{p\text{ prime}}\sum_{m=1}^{\infty}\frac{1}{mp^{m\sigma}}(3+4\cos(mt\log p)+\cos(2mt\log p))\geq 0\end{align*}
so that
\begin{equation}\label{eq:f}\zeta(\sigma)^3\lvert\zeta(\sigma+it)\rvert^4\lvert\zeta(\sigma+it\cdot 2)\rvert\geq 1\ \text{ for all }\ \sigma>1, t\in\Reals\end{equation}
But if $\zeta$ has a zero at $\sigma+it_0$, then
\[\lim_{\sigma\to 1^+}\zeta(\sigma)^3\lvert\zeta(\sigma+it_0)\rvert^4\lvert\zeta(\sigma+i2t)\rvert=0\]
since the first factor gives a \PMlinkname{pole}{Pole} of order 3 at $1$ and the second factor gives a zero of order at least 4 at $1+it_0$. This contradicts equation \eqref{eq:f}.


\begin{cor} $\zeta(it)\neq 0$ for $t\in\Reals$.
\end{cor}

\textbf{Proof. }\ Use the functional equation
\[\pi^{\frac{-s}{2}}\Gamma\left(\frac{s}{2}\right)\zeta(s)=\pi^{-\frac{1-s}{2}}\Gamma\left(\frac{1-s}{2}\right)\zeta(1-s)\]
and set $s=it$. The theorem implies that the RHS is nonzero, so the LHS is as well. Thus $\zeta(s)\ne 0$.


%%%%%
%%%%%
\end{document}
