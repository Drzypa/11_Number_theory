\documentclass[12pt]{article}
\usepackage{pmmeta}
\pmcanonicalname{ExistenceAndUniquenessOfDecimalExpansion}
\pmcreated{2013-03-22 15:42:12}
\pmmodified{2013-03-22 15:42:12}
\pmowner{stevecheng}{10074}
\pmmodifier{stevecheng}{10074}
\pmtitle{existence and uniqueness of decimal expansion}
\pmrecord{8}{37648}
\pmprivacy{1}
\pmauthor{stevecheng}{10074}
\pmtype{Theorem}
\pmcomment{trigger rebuild}
\pmclassification{msc}{11A99}
\pmsynonym{every decimal expansion represents a real number}{ExistenceAndUniquenessOfDecimalExpansion}
\pmrelated{CantorsDiagonalArgument}
\pmrelated{DecimalExpansion}

\usepackage{amssymb}
\usepackage{amsmath}
\usepackage{amsfonts}
%\usepackage{amsthm}
\usepackage{enumerate}
%\usepackage{graphicx}
%\usepackage{psfrag}
%%%\usepackage{xypic}

% define commands here
\newcommand{\complex}{\mathbb{C}}
\newcommand{\real}{\mathbb{R}}
\newcommand{\rat}{\mathbb{Q}}
\newcommand{\nat}{\mathbb{N}}

\providecommand{\abs}[1]{\lvert#1\rvert}
\providecommand{\absW}[1]{\left\lvert#1\right\rvert}
\providecommand{\absB}[1]{\Bigl\lvert#1\Bigr\rvert}
\providecommand{\norm}[1]{\lVert#1\rVert}
\providecommand{\normW}[1]{\left\lVert#1\right\rVert}
\providecommand{\normB}[1]{\Bigl\lVert#1\Bigr\rVert}
\providecommand{\defnterm}[1]{\emph{#1}}

\DeclareMathOperator{\D}{D}
\DeclareMathOperator{\linspan}{span}
\begin{document}
The existence and uniqueness of decimal expansions 
(or more generally, base-$b$ expansions)
is taken for granted
by most grade school students, 
but they are facts that need to be rigorously proven if one wants to understand 
the real numbers thoroughly. 

We mention the following fact about natural numbers $n,m \in \nat_0$,
which we will use many times implicitly:
\[
n < m \Leftrightarrow n \leq m-1\,
\]
This fact can be proven by mathematical induction on $m$.

\tableofcontents

\section{Proof of Existence}
Let $x$ be a number for which we want to write a base-$b$ expansion
for any natural number $b$ greater than one.

We shall assume $x \geq 0$, since the base-$b$ 
expansion of a negative number is by definition
the negative of the expansion for its absolute value.

\subsection{Expansions for non-negative integers}
First we prove the existence of expansions of the form
\[
x = \sum_{i=0}^k a_i b^i\,, \quad 0 \leq a_i < b\,, a_i \in \nat_0
\]
for non-negative integers $x$,
using mathematical induction.  (This proof is essentially the formal
statement of how to do addition by base-$b$ digits.)

The number $x = 0$ obviously has the expansion $0$.

Suppose that we know the existence of expansions for a number $x-1$.
We prove the existence of an expansion for $x$.

Let $x-1$ be expanded as
\[
x - 1 = \sum_{i=0}^k a_i b^i\,, \quad 0 \leq a_i < b\,, a_i \in \nat_0\,, a_{k+1} = 0\,.
\]
From the above equation, add $1$ to both sides:
\[
x = (a_0+1) + \sum_{i=1}^k a_i b^i \,.
\]
If $a_0 < b-1$, then we are done.  Otherwise,
$(a_0+1) = b$, and therefore we may write instead
\[
x = 0 + (a_1+1) b + \sum_{i=2}^k a_i b^i\,.
\]
If $a_1 < b-1$, then we can stop.  Otherwise,
repeat the process and continue carrying digits
until we reach some $i$ for which $a_i < b-1$.
Since $a_{k+1} = 0$, this process is guaranteed to stop.
At the end we will have expressed $x$ in base $b$.

\subsection{Reduction to numbers in $[0,1)$}
Let $\lfloor x \rfloor$ be the greatest integer less than or equal to $x$,
otherwise known as the floor of $x$.  
We prove that the floor of $x$ exists.

The set
\[
A = \{ n \in \nat_0 \colon n \leq x \}
\]
is bounded above by $x$.
However, by the Archimedean 
property, the set of natural numbers is \emph{not} bounded
above, so $\nat_0 \setminus A$ must be non-empty,
and have a smallest element $u$
(formally, by the well-ordering principle).
For every $n \in A$, we have
$n \leq x < u$.
The latter condition is equivalent to $n \leq u-1 < x$,
so $u-1$ is the maximum element of $A$.
In other words, $\lfloor x \rfloor = u-1$.

Since $u-1 \leq x < u$, we have
$0 \leq x - \lfloor x \rfloor < 1$.
We shall obtain the base $b$ expansion of $x$
as the sum of the expansion of $\lfloor x \rfloor$ and
$x - \lfloor x \rfloor$.

\subsection{Expansion of numbers in $[0,1)$}
Given $x \in [0,1)$,
let $a_1 = \lfloor bx \rfloor$.
Then $0 \leq a_1 \leq bx < b$, so we can take $a_1$ as the first
digit of the base-$b$ expansion of $x$.
Next, write
\[
x = a_1 b^{-1} + yb^{-1}\,,
\]
and observe that $0 \leq x - a_1 b^{-1} = b^{-1} (bx - \lfloor bx \rfloor) < b^{-1}$,
so it is possible to get the next digit of the expansion by
expanding $y$.
We do this recursively, leading to these recursive relations:
\[
y_{i} = a_i b^{-1} + y_{i+1} b^{-1} \,, \quad 0 \leq a_i = \lfloor by_i \rfloor < b\,, \quad 0 \leq y_i < 1\,, \quad y_1 = x\,.
\]
More explicitly, we have
\begin{align*}
x - (a_1 b^{-1} + a_2 b^{-2} + \dotsb + a_k b^{-k}) &= 
b^{-1} \bigl( y_2 - (a_2 b^{-1} + \dotsb + a_k b^{-k+1}) \bigr) \\ &=
b^{-2} \bigl( y_3 - (a_3 b^{-1} + \dotsb + a_k b^{-k+2}) \bigr) \\
&= \dotsb \\
&= b^{-k+1} (y_{k} - a_k b^{-1}) \\
&= b^{-k} y_{k+1}\,.
\end{align*}

It is easy to prove that the expansion
\[
a_1 b^{-1} + a_2 b^{-2} + \dotsb + a_k b^{-k} + \dotsb
\]
converges to $x$:
\[
0 \leq x - \sum_{i=1}^k a_i b^{-i} = b^{-k} y_{k+1} < \frac{1}{b^k} \to 0\,, \quad
\text{as $k \to \infty$.}
\]
(Formally, the ``$\to 0$'' part appeals to the Archimedean property.)




\section{Proof of uniqueness}

\subsection{Uniqueness for non-negative integers}
Suppose that
\[
x = \sum_{i=0}^k a_i b^i\,, \quad 0 \leq a_i < b\,.
\]
Now 
\[
a_k b^k \leq \sum_{i=0}^k a_i b^i \leq a_k b^k + \sum_{i=0}^{k-1} (b-1) b^i = a_k b^k + (b^k - 1) < (a_k + 1)b^k\,,
\]
and the intervals $[a_k b^k, (a_k+1)b^k)$ are disjoint for each
value of $a_k$, so $a_k$ is uniquely determined by where $x$ lies
in amongst these intervals.

Then we can consider
\[
x - a_k b^k = \sum_{i=0}^{k-1} a_i b^i\,.
\]
Repeating the previous argument with $k$ replaced by $k-1$,
we see that $a_{k-1}$ is uniquely determined.
Then we can consider $x - a_k b^k - a_{k-1} b^{k-1}$ and so on.
Continuing this way, we see that all the digits $a_i$ are uniquely
determined.

\subsection{Near-uniqueness for non-negative numbers}
If
\[
x = a_k b^k + \dotsb + a_1 b + a_0 + a_{-1} b^{-1} + a_{-2} b^{-2} + \dotsb
\]
then $a_0, \dotsc, a_k$ are uniquely determined,
since $a_k b^k + \dotsb + a_1 + a_0$ 
is the expansion for the non-negative integer $\lfloor x \rfloor$.

The argument to prove that $a_{-i}$ are uniquely determined proceeds
similarly as before.
We have
\begin{align*}
a_{-1} b^{-1} &\leq a_{-1} b^{-1} + a_{-2} b^{-2} + \dotsb \\
&\leq a_{-1} b^{-1} + \sum_{i=2}^\infty (b-1) b^{-i} 
& \text{(geometric series)}
\\
&= a_{-1} b^{-1} + \frac{(b-1)b^2}{1-b^{-1}} \\
&= (a_{-1}+1)\, b^{-1}\,,
\end{align*}
where equality on the second line occurs if and only if 
$a_{-i} = b-1$ for every $i \geq 2$.
If we insist that $a_{-i}$ is never eventually the same
digit $b-1$,
then this shows that the digit $a_{-1}$ is uniquely determined by 
where the original number $x$ in the disjoint intervals $[a_{-1} b^{-1}, (a_{-1} + 1) b^{-1})$.

This argument may be repeated, to show that $a_{-i}$ are uniquely
determined, under the assumption that the expansion
does not end in all digits being $b-1$.

If the assumption is not made, then
numbers which have an expansion ending in all digits $0$
have an alternate expansion ending in all digits $b-1$,
but other numbers still have unique base-$b$ expansions.

\section{Every base-$b$ expansion represents a real number}

We also want to prove that for every sequence of digits $a_k, a_{k-1}, \dotsc, a_1, a_0, a_{-1}, a_{-2}, \dotsc$
there exists a real number $x$
with the base-$b$ expansion
\[
x = \sum_{i=0}^k a_i b^i + \sum_{i=0}^\infty a_{-i} b^{-i}\,.
\]

This is the \PMlinkescapetext{place} where we use the least upper bounds property of the 
real numbers.  (So far we have only used the Archimedean property,
so what we have done so far is also valid for $\rat$.)

Consider the sequence $\{ s_n \}$ with the \PMlinkescapetext{terms}
\[
s_n = \sum_{i=0}^k a_i b^i + \sum_{i=0}^n a_{-i} b^{-i}\,.
\]
This sequence, considered as a set,
is bounded above, for $s_n \leq a_k b^k + \dotsb + a_0 + 1$.
So it has a least upper bound $x$. 
Since the sequence $\{ s_n \}$ is also \emph{increasing},
its least upper bound is the same as its
limit.
%%%%%
%%%%%
\end{document}
