\documentclass[12pt]{article}
\usepackage{pmmeta}
\pmcanonicalname{BrunsConstant}
\pmcreated{2013-03-22 13:20:01}
\pmmodified{2013-03-22 13:20:01}
\pmowner{bbukh}{348}
\pmmodifier{bbukh}{348}
\pmtitle{Brun's constant}
\pmrecord{8}{33847}
\pmprivacy{1}
\pmauthor{bbukh}{348}
\pmtype{Definition}
\pmcomment{trigger rebuild}
\pmclassification{msc}{11N36}
\pmclassification{msc}{11N05}
%\pmkeywords{Brun's sieve}
%\pmkeywords{twin primes}
\pmrelated{BrunsPureSieve}

\usepackage{amssymb}
\usepackage{amsmath}
\usepackage{amsfonts}
\begin{document}
\emph{Brun's constant} is the sum of the reciprocals of all twin primes
\begin{equation*}
B=\sum_{\substack{p\\p+2 \text{ is prime}}} \left(\frac{1}{p}+\frac{1}{p+2}\right)\approx 1.9216058.
\end{equation*}
Viggo Brun proved that the constant exists by using a new sieving method, which later became known as \PMlinkname{Brun's sieve}{BrunsPureSieve}.
%%%%%
%%%%%
\end{document}
