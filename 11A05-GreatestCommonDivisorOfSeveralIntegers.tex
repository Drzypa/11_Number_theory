\documentclass[12pt]{article}
\usepackage{pmmeta}
\pmcanonicalname{GreatestCommonDivisorOfSeveralIntegers}
\pmcreated{2013-03-22 19:20:14}
\pmmodified{2013-03-22 19:20:14}
\pmowner{pahio}{2872}
\pmmodifier{pahio}{2872}
\pmtitle{greatest common divisor of several integers}
\pmrecord{5}{42285}
\pmprivacy{1}
\pmauthor{pahio}{2872}
\pmtype{Theorem}
\pmcomment{trigger rebuild}
\pmclassification{msc}{11A05}
\pmsynonym{GCD of several integers}{GreatestCommonDivisorOfSeveralIntegers}
%\pmkeywords{linear combination}
\pmrelated{BezoutsLemma}
\pmrelated{IdealDecompositionInDedekindDomain}

% this is the default PlanetMath preamble.  as your knowledge
% of TeX increases, you will probably want to edit this, but
% it should be fine as is for beginners.

% almost certainly you want these
\usepackage{amssymb}
\usepackage{amsmath}
\usepackage{amsfonts}

% used for TeXing text within eps files
%\usepackage{psfrag}
% need this for including graphics (\includegraphics)
%\usepackage{graphicx}
% for neatly defining theorems and propositions
 \usepackage{amsthm}
% making logically defined graphics
%%%\usepackage{xypic}

% there are many more packages, add them here as you need them

% define commands here

\theoremstyle{definition}
\newtheorem*{thmplain}{Theorem}

\begin{document}
\textbf{Theorem.}\, If the greatest common divisor of the nonzero integers $a_1,\,a_2,\,\ldots,\,a_n$ is $d$, then there exist the integers $x_1,\,x_2,\,\ldots,\,x_n$ such that
\begin{align}
d \;=\; x_1a_1\!+\!x_2a_2\!+\ldots+\!x_na_n.
\end{align}

\emph{Proof.}\, In the case \,$n = 2$\, the Euclidean algorithm for two nonzero integers $a,\,b$ always guarantees the integers $x,\,y$ such that
\begin{align}
\gcd(a,\,b) \;=\; xa\!+\!yb.
\end{align}
Make the induction hypothesis that the theorem is true whenever\, $n < k$.\\
Since obviously
$$d \;=\; \gcd(a_1,\,a_2,\,\ldots,\,a_k) \;=\; \gcd(\gcd(a_1,\,a_2,\,\ldots,\,a_{k-1}),\,a_k),$$
we may write by (2) that
$$d \;=\; x\gcd(a_1,\,a_2,\,\ldots,\,a_{k-1})+ya_k$$
and thus by the induction hypothesis that
$$d \;=\; x(x_1a_1\!+\!x_2a_2\!+\ldots+\!x_{k-1}a_{k-1})\!+\!ya_k.$$
Consequently, we have gotten an equation of the form (1), Q.E.D.\\


\textbf{Note.}\, An analogous theorem concerns elements of any \PMlinkname{B\'ezout domain}{BezoutDomain}.
%%%%%
%%%%%
\end{document}
