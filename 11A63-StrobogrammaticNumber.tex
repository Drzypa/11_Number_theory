\documentclass[12pt]{article}
\usepackage{pmmeta}
\pmcanonicalname{StrobogrammaticNumber}
\pmcreated{2013-03-22 16:46:35}
\pmmodified{2013-03-22 16:46:35}
\pmowner{CompositeFan}{12809}
\pmmodifier{CompositeFan}{12809}
\pmtitle{strobogrammatic number}
\pmrecord{4}{39006}
\pmprivacy{1}
\pmauthor{CompositeFan}{12809}
\pmtype{Definition}
\pmcomment{trigger rebuild}
\pmclassification{msc}{11A63}

% this is the default PlanetMath preamble.  as your knowledge
% of TeX increases, you will probably want to edit this, but
% it should be fine as is for beginners.

% almost certainly you want these
\usepackage{amssymb}
\usepackage{amsmath}
\usepackage{amsfonts}

% used for TeXing text within eps files
%\usepackage{psfrag}
% need this for including graphics (\includegraphics)
%\usepackage{graphicx}
% for neatly defining theorems and propositions
%\usepackage{amsthm}
% making logically defined graphics
%%%\usepackage{xypic}

% there are many more packages, add them here as you need them

% define commands here

\begin{document}
A {\em strobogrammatic number} is a number that, given a base and given a set of glyphs, appears the same whether viewed normally or upside down. In base 10, given a set of glyphs where 0, 1 and 8 are symmetrical around the horizontal axis, and 6 and 9 are the same as each other upside down (such as the digit characters in ASCII using the font Stylus BT), the first few strobogrammatic numbers are:

1, 8, 11, 69, 88, 96, 101, 111, 181, 609, 619, 689, 808, 818, 888, 906, 916, 986, 1001 (sequence A000787 in OEIS)

Like the concept of repunits and palindromic numbers, the concept of strobogrammatic numbers is base-dependent. Unlike palindromicity it is also font dependent. But the concept of strobogrammatic numbers is not neatly expressible algebraically, the way that the concept of repunits is, or even the concept of palindromic numbers.

There are sets of glyphs for writing numbers in base 10, such as the Devanagari and Gurmukhi of India in which the numbers listed above are not strobogrammatic at all.

In binary, given a glyph for 1 consisting of a single line without hooks or serifs, all palindromic numbers are strobogrammatic, which means that all Mersenne numbers are strobogrammatic.
%%%%%
%%%%%
\end{document}
