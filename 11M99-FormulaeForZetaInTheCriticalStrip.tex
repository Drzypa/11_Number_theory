\documentclass[12pt]{article}
\usepackage{pmmeta}
\pmcanonicalname{FormulaeForZetaInTheCriticalStrip}
\pmcreated{2013-03-22 13:28:14}
\pmmodified{2013-03-22 13:28:14}
\pmowner{mathcam}{2727}
\pmmodifier{mathcam}{2727}
\pmtitle{formulae for zeta in the critical strip}
\pmrecord{11}{34040}
\pmprivacy{1}
\pmauthor{mathcam}{2727}
\pmtype{Theorem}
\pmcomment{trigger rebuild}
\pmclassification{msc}{11M99}
\pmrelated{CriticalStrip}
\pmrelated{ValueOfTheRiemannZetaFunctionAtS0}
\pmrelated{AnalyticContinuationOfRiemannZeta}

\endmetadata

\usepackage{amssymb}
\usepackage{amsmath}
\usepackage{amsfonts}
\begin{document}
\PMlinkescapeword{theory}
\PMlinkescapeword{terms}
\PMlinkescapeword{formula}
\PMlinkescapeword{simple}
\PMlinkescapeword{one way} %Gawd.
Let us use the traditional notation $s=\sigma+it$ for the complex variable,
where $\sigma$ and $t$ are real numbers.

\begin{eqnarray}
\zeta(s)&=&
\frac{1}{1-2^{1-s}}\sum_{n=1}^\infty (-1)^{n+1}n^{-s}
\qquad\sigma>0
\label{eq:one}
\\
\zeta(s)&=&
\frac{1}{s-1}+1-s\int_1^\infty \frac{x-[x]}{x^{s+1}}dx
\qquad\sigma>0
\label{eq:two}
\\
\zeta(s)&=&
\frac{1}{s-1}+\frac{1}{2}-s\int_1^\infty \frac{((x))}{x^{s+1}}dx
\quad\sigma>-1
\label{eq:three}
\end{eqnarray}
where $[x]$ denotes the largest integer $\le x$,
and $((x))$ denotes $x-[x]-\frac{1}{2}$.

We will prove \eqref{eq:two} and \eqref{eq:three} with the help of this
useful lemma:

\textbf{Lemma:} For integers $u$ and $v$ such that $0<u<v$:
\begin{equation*}
\sum_{n=u+1}^v n^{-s} =
  -s\int_u^v \frac{x-[x]}{x^{s+1}}dx
  +\frac{v^{1-s}-u^{1-s}}{1-s}
\end{equation*}

\textbf{Proof:} If we can prove the special case $v=u+1$, namely
\begin{equation}
(u+1)^{-s} = -s\int_u^{u+1} \frac{x-[x]}{x^{s+1}}dx
+\frac{(u+1)^{1-s}-u^{1-s}}{1-s}
\label{eq:four}
\end{equation}
then the lemma will follow by summing a finite sequence of cases of
\eqref{eq:four}.
The integral in \eqref{eq:four} is
\begin{eqnarray*}
\int_0^1 \frac{tdt}{(u+t)^{s+1}}
     &=& \int_0^1 (u+t)^{-s}dt - \int_0^1 u(u+t)^{-s-1}dt \\
     &=& \frac{(u+1)^{1-s}-u^{1-s}}{1-s}
         +\frac{u\left[(u+1)^{-s}-u^{-s}\right]}{s}
\end{eqnarray*}
so the right side of \eqref{eq:four} is
$$\frac{-s}{1-s}\left[(u+1)^{1-s}-u^{1-s}\right]
         -u\left[(u+1)^{-s}-u^{-s}\right]
         -\frac{u^{1-s}}{1-s} +\frac{(u+1)^{1-s}}{1-s}
$$
$$=(u+1)^{-s}\left[\frac{-s(u+1)}{1-s}-u+\frac{u+1}{1-s}\right]
+u^{-s}\left[\frac{us}{1-s}+u-\frac{u}{1-s}\right]
$$
$$=(u+1)^{-s}\cdot 1+u^{-s}\cdot 0
$$
and the lemma is proved.

Now take $u=1$ and let $v\to \infty$ in the lemma, showing that
\eqref{eq:two} holds for $\sigma>1$.
By the principle of analytic continuation, if
the integral in \eqref{eq:two} is analytic for $\sigma>0$,
then \eqref{eq:two} holds for $\sigma>0$.
But $x-[x]$ is bounded, so the integral converges
uniformly on $\sigma\ge\epsilon$ for any $\epsilon>0$, and the claim
\eqref{eq:two} follows.

We have
\begin{equation*}
\frac{1}{2}s\int_1^\infty x^{-1-s}dx=\frac{1}{2}
\end{equation*}
Adding and subtracting this quantity from \eqref{eq:two},
we get \eqref{eq:three} for $\sigma>0$.
We need to show that
$$\int_1^\infty \frac{((x))}{x^{s+1}}dx$$
is analytic on $\sigma>-1$. Write
$$f(y)=\int_1^y ((x))dx$$
and integrate by parts:
$$\int_1^\infty \frac{((x))}{x^{s+1}}dx
=\lim_{x\to\infty}f(x)x^{-1-s} - f(1)x^{-1-1}+(s+1)
\int_1^\infty\frac{f(x)}{x^{s+2}}dx
$$
The first two terms on the right are zero, and the integral
converges for $\sigma>-1$ because $f$ is bounded.

\textbf{Remarks:}
We will prove \eqref{eq:one} in a later version of this entry.

Using formula \eqref{eq:three}, one can verify Riemann's
functional equation in the strip $-1<\sigma<2$.
By analytic continuation, it follows that the functional
equation holds everywhere.
One way to prove it in the strip is to decompose the
sawtooth function $((x))$ into a Fourier series, and
do a termwise integration.
But the proof gets rather technical, because that
series does not converge uniformly.
%%%%%
%%%%%
\end{document}
