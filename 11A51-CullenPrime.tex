\documentclass[12pt]{article}
\usepackage{pmmeta}
\pmcanonicalname{CullenPrime}
\pmcreated{2013-03-22 17:21:37}
\pmmodified{2013-03-22 17:21:37}
\pmowner{PrimeFan}{13766}
\pmmodifier{PrimeFan}{13766}
\pmtitle{Cullen prime}
\pmrecord{6}{39720}
\pmprivacy{1}
\pmauthor{PrimeFan}{13766}
\pmtype{Definition}
\pmcomment{trigger rebuild}
\pmclassification{msc}{11A51}

\endmetadata

% this is the default PlanetMath preamble.  as your knowledge
% of TeX increases, you will probably want to edit this, but
% it should be fine as is for beginners.

% almost certainly you want these
\usepackage{amssymb}
\usepackage{amsmath}
\usepackage{amsfonts}

% used for TeXing text within eps files
%\usepackage{psfrag}
% need this for including graphics (\includegraphics)
%\usepackage{graphicx}
% for neatly defining theorems and propositions
%\usepackage{amsthm}
% making logically defined graphics
%%%\usepackage{xypic}

% there are many more packages, add them here as you need them

% define commands here

\begin{document}
A {\em Cullen prime} is a Cullen number $C_m$ that is also a prime number. The first two Cullen primes are 3 and 393050634124102232869567034555427371542904833, corresponding to $m = 1$ and 141 respectively; other $m$ to give Cullen primes are 4713, 5795, 6611, 18496, 32292, 32469, 59656, 90825, etc. listed in A005849 of Sloane's OEIS (those primes being too large to write here). None of these indexes are prime, and it is a matter of conjecture whether there exists a Cullen prime of the form $p2^p + 1$ with $p$ also prime. It is not even known whether there are infinitely many Cullen primes (a possibility despite their rarity).
%%%%%
%%%%%
\end{document}
