\documentclass[12pt]{article}
\usepackage{pmmeta}
\pmcanonicalname{ExampleOfArithmeticDerivative}
\pmcreated{2013-03-22 13:35:12}
\pmmodified{2013-03-22 13:35:12}
\pmowner{Johan}{1032}
\pmmodifier{Johan}{1032}
\pmtitle{example of arithmetic derivative}
\pmrecord{7}{34209}
\pmprivacy{1}
\pmauthor{Johan}{1032}
\pmtype{Example}
\pmcomment{trigger rebuild}
\pmclassification{msc}{11Z05}

\endmetadata

% this is the default PlanetMath preamble.  as your knowledge
% of TeX increases, you will probably want to edit this, but
% it should be fine as is for beginners.

% almost certainly you want these
\usepackage{amssymb}
\usepackage{amsmath}
\usepackage{amsfonts}

% used for TeXing text within eps files
%\usepackage{psfrag}
% need this for including graphics (\includegraphics)
%\usepackage{graphicx}
% for neatly defining theorems and propositions
%\usepackage{amsthm}
% making logically defined graphics
%%%\usepackage{xypic}

% there are many more packages, add them here as you need them

% define commands here
\begin{document}
Consider the natural number $6$. Using the rules of the arithmetic derivative we get: \\ $6'=(2\cdot3)'=2'\cdot 3+2\cdot 3'=1\cdot 3+2\cdot 1=5$ \\
\\ Below is a list of the 10 first natural numbers and their first and second arithmetic derivatives:\\
\begin{tabular}{|c|cccccccccc|}
\hline
$n$&$1$&$2$&$3$&$4$&$5$&$6$&$7$&$8$&$9$&$10$ \\
\hline
$n'$&$0$&$1$&$1$&$4$&$1$&$5$&$1$&$12$&$6$&$7$ \\ 
$n''$&$0$&$0$&$0$&$4$&$0$&$1$&$0$&$16$&$5$&$1$ \\ 
\hline
\end{tabular}
%%%%%
%%%%%
\end{document}
