\documentclass[12pt]{article}
\usepackage{pmmeta}
\pmcanonicalname{ValuesOfTheLegendreSymbol}
\pmcreated{2013-03-22 16:18:13}
\pmmodified{2013-03-22 16:18:13}
\pmowner{alozano}{2414}
\pmmodifier{alozano}{2414}
\pmtitle{values of the Legendre symbol}
\pmrecord{5}{38425}
\pmprivacy{1}
\pmauthor{alozano}{2414}
\pmtype{Theorem}
\pmcomment{trigger rebuild}
\pmclassification{msc}{11-00}
\pmrelated{1IsQuadraticResidueIfAndOnlyIfPequiv1Mod4}
\pmrelated{QuadraticCharacterOf2}

% this is the default PlanetMath preamble.  as your knowledge
% of TeX increases, you will probably want to edit this, but
% it should be fine as is for beginners.

% almost certainly you want these
\usepackage{amssymb}
\usepackage{amsmath}
\usepackage{amsthm}
\usepackage{amsfonts}

% used for TeXing text within eps files
%\usepackage{psfrag}
% need this for including graphics (\includegraphics)
%\usepackage{graphicx}
% for neatly defining theorems and propositions
%\usepackage{amsthm}
% making logically defined graphics
%%%\usepackage{xypic}

% there are many more packages, add them here as you need them

% define commands here

\newtheorem*{thm}{Theorem}
\newtheorem{defn}{Definition}
\newtheorem{prop}{Proposition}
\newtheorem{lemma}{Lemma}
\newtheorem{cor}{Corollary}

\theoremstyle{definition}
\newtheorem{exa}{Example}

% Some sets
\newcommand{\Nats}{\mathbb{N}}
\newcommand{\Ints}{\mathbb{Z}}
\newcommand{\Reals}{\mathbb{R}}
\newcommand{\Complex}{\mathbb{C}}
\newcommand{\Rats}{\mathbb{Q}}
\newcommand{\Gal}{\operatorname{Gal}}
\newcommand{\Cl}{\operatorname{Cl}}
\begin{document}
For an integer $a$ and an odd prime $p$, let $\displaystyle \left(\frac{a}{p}\right)$ be the Legendre symbol.

\begin{thm}
Let $p$ be an odd prime. The Legendre symbol takes the following values:
\begin{enumerate}
\item \[
\left(\frac{-1}{p}\right) =
\begin{cases}
1 &\text{if }p \equiv 1 \mod 4\\
-1 &\text{if }p \equiv 3 \mod 4.
\end{cases}
\]
\item \[
\left(\frac{2}{p}\right) =
\begin{cases}
1 &\text{if }p\equiv \pm 1 \mod 8\\
-1 &\text{if }p\equiv 3,5 \mod 8.
\end{cases}
\]
\item \[
\left(\frac{3}{p}\right) =
\begin{cases}
1 &\text{if }p\equiv \pm 1 \mod 12\\
-1 &\text{otherwise.}
\end{cases}
\]
\item \[
\left(\frac{5}{p}\right) =
\begin{cases}
1 &\text{if }p\equiv \pm 1 \mod 5\\
-1 &\text{if }p\equiv 2,3 \mod 5.
\end{cases}
\]
\end{enumerate}
\end{thm}
\begin{proof}
For a proof of (1), see \PMlinkid{this entry}{1IsQuadraticResidueIfAndOnlyIfPequiv1Mod4}. Part (2) is proved in \PMlinkid{this entry}{QuadraticCharacterOf2}. For parts (3), (4) and (5), we use quadratic reciprocity. For example, 
$$\left(\frac{5}{p}\right)=\left(\frac{p}{5}\right)$$
and the only quadratic residues modulo $5$ are $\pm 1 \mod 5$.
\end{proof}
%%%%%
%%%%%
\end{document}
