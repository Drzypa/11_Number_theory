\documentclass[12pt]{article}
\usepackage{pmmeta}
\pmcanonicalname{Pseudoprime}
\pmcreated{2013-03-22 16:49:44}
\pmmodified{2013-03-22 16:49:44}
\pmowner{CompositeFan}{12809}
\pmmodifier{CompositeFan}{12809}
\pmtitle{pseudoprime}
\pmrecord{6}{39068}
\pmprivacy{1}
\pmauthor{CompositeFan}{12809}
\pmtype{Definition}
\pmcomment{trigger rebuild}
\pmclassification{msc}{11A51}

% this is the default PlanetMath preamble.  as your knowledge
% of TeX increases, you will probably want to edit this, but
% it should be fine as is for beginners.

% almost certainly you want these
\usepackage{amssymb}
\usepackage{amsmath}
\usepackage{amsfonts}

% used for TeXing text within eps files
%\usepackage{psfrag}
% need this for including graphics (\includegraphics)
%\usepackage{graphicx}
% for neatly defining theorems and propositions
%\usepackage{amsthm}
% making logically defined graphics
%%%\usepackage{xypic}

% there are many more packages, add them here as you need them

% define commands here

\begin{document}
A {\em pseudoprime} is a composite number $p$ that passes a given primality test. In other words, a pseudoprime is a false positive on a primality test.

The primality test is usually a power congruence to a given base $b$ that is coprime to $p$, such as $b^{p - 1} \equiv 1 \mod p$ (from Fermat's little theorem). For example, if $p$ is an odd prime, then the congruence $2^{p - 1} \equiv 1 \mod p$ will be true, but in reverse, $p$ can satisfy the congruence but turn out to be an odd composite number. Such pseudoprimes are then called {\em pseudoprime to base $b$}, and in our example with $b = 2$, the first few are 341, 561, 645, 1105, 1387, 1729, 1905 (A001567 in Sloane's OEIS). According to Neil Sloane, writing in the OEIS, the term ``pseudoprime'' is sometimes used specifically to refer to pseudoprimes to base 2. These are sometimes called Sarrus numbers.

Another kind of pseudoprimes are those in which a composite $p$ divides an element it indexes in a Fibonacci-like sequence. For example, a Perrin pseudoprime $p$ divides $a_p$, where $a_n$ is the $n$-th element in the Perrin sequence.

\begin{thebibliography}{1}
\bibitem{rc} R. Crandall \& C. Pomerance, {\it Prime Numbers: A Computational Perspective}, Springer, NY, 2001: 5.1
\end{thebibliography}
%%%%%
%%%%%
\end{document}
