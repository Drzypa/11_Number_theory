\documentclass[12pt]{article}
\usepackage{pmmeta}
\pmcanonicalname{EulerNumbers}
\pmcreated{2014-12-02 17:43:40}
\pmmodified{2014-12-02 17:43:40}
\pmowner{pahio}{2872}
\pmmodifier{pahio}{2872}
\pmtitle{Euler numbers}
\pmrecord{11}{42002}
\pmprivacy{1}
\pmauthor{pahio}{2872}
\pmtype{Definition}
\pmcomment{trigger rebuild}
\pmclassification{msc}{11B68}
\pmrelated{GudermannianFunction}
\pmrelated{BernoulliNumber}
\pmrelated{InverseGudermannianFunction}
\pmrelated{HermiteNumbers}

\endmetadata

% this is the default PlanetMath preamble.  as your knowledge
% of TeX increases, you will probably want to edit this, but
% it should be fine as is for beginners.

% almost certainly you want these
\usepackage{amssymb}
\usepackage{amsmath}
\usepackage{amsfonts}

% used for TeXing text within eps files
%\usepackage{psfrag}
% need this for including graphics (\includegraphics)
%\usepackage{graphicx}
% for neatly defining theorems and propositions
 \usepackage{amsthm}
% making logically defined graphics
%%%\usepackage{xypic}

% there are many more packages, add them here as you need them

% define commands here

\theoremstyle{definition}
\newtheorem*{thmplain}{Theorem}

\begin{document}
\PMlinkescapeword{index}

\emph{Euler numbers} $E_n$ have the generating function $\displaystyle\frac{1}{\cosh{x}}$ such that
$$\frac{1}{\cosh{x}} \;=:\; \sum_{n=0}^\infty\frac{E_n}{n!}\,x^n.$$
They are integers but have no \PMlinkescapetext{simple} expression for calculating them.\, Their only \PMlinkescapetext{regularities} are that the numbers with odd \PMlinkname{index}{IndexingSet} are all 0 and that 
$$\mbox{sgn}(E_{2m}) \;=\; (-1)^m \qquad \mbox{for} \quad m = 0,\,1,\,2,\,\ldots$$
The Euler number have intimate relation to the Bernoulli numbers.\, The first Euler numbers with even index are
$$E_0 = 1,\quad E_2 = -1,\quad E_4 = 5,\quad E_6 = -61,
\quad E_8 = 1385,\quad E_{10} = -50521.$$

\begin{itemize}

\item One can by hand determine Euler numbers by performing the 
division of 1 by the Taylor series of hyperbolic cosine 
(cf. Taylor series via division and Taylor series of hyperbolic 
functions).\, 
Since\, $\cosh{ix} = \cos{x}$,\, the division $1:\cos{x}$ 
correspondingly gives only terms with plus sign, i.e. it shows the 
absolute values of the Euler numbers.\\

\item The Euler numbers may also be obtained by using the Euler polynomials $E_n(x)$:
$$E_n \;=\; 2^nE_n\!\!\left(\!\frac{1}{2}\!\right)$$

\item If the Euler numbers $E_k$ are denoted as symbolic powers $E^k$, then one may write the equation
$$(E\!+\!1)^n+(E\!-\!1)^n \;=\; 0,$$
which can be used as a recurrence relation for computing the values of the even-indexed Euler numbers.\, Cf. the Leibniz rule for derivatives of product $fg$.

\end{itemize}



%%%%%
%%%%%
\end{document}
