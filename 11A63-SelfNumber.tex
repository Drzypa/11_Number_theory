\documentclass[12pt]{article}
\usepackage{pmmeta}
\pmcanonicalname{SelfNumber}
\pmcreated{2013-03-22 15:56:15}
\pmmodified{2013-03-22 15:56:15}
\pmowner{CompositeFan}{12809}
\pmmodifier{CompositeFan}{12809}
\pmtitle{self number}
\pmrecord{6}{37946}
\pmprivacy{1}
\pmauthor{CompositeFan}{12809}
\pmtype{Definition}
\pmcomment{trigger rebuild}
\pmclassification{msc}{11A63}
\pmsynonym{Columbian number}{SelfNumber}
\pmsynonym{Colombian number}{SelfNumber}
\pmsynonym{self-number}{SelfNumber}

\endmetadata

% this is the default PlanetMath preamble.  as your knowledge
% of TeX increases, you will probably want to edit this, but
% it should be fine as is for beginners.

% almost certainly you want these
\usepackage{amssymb}
\usepackage{amsmath}
\usepackage{amsfonts}

% used for TeXing text within eps files
%\usepackage{psfrag}
% need this for including graphics (\includegraphics)
%\usepackage{graphicx}
% for neatly defining theorems and propositions
%\usepackage{amsthm}
% making logically defined graphics
%%%\usepackage{xypic}

% there are many more packages, add them here as you need them

% define commands here

\begin{document}
An integer $n$ that in a given base $b$ lacks a digitaddition generator.

Consider, for example, the integer 41 in base 10. It can be expressed as 34 + 3 + 4. For 42, however, there is no such digitaddition, hence it is a self number.

If $2|b$, all odd $n < b$ will be self numbers.

Though self numbers form a small proportion of most ranges of $2b$ consecutive integers, there are infinitely many of them: The recurrence relation $S_i = (b - 2)b^{i - 1} + S_{i - 1} + (b - 2)$ (with $S_1 = b - 1$ if $2|b$ and $S_1 = b - 2$ otherwise) will give an incomplete though infinite list of self numbers.

Reference

Kaprekar, D. R. {\em The Mathematics of New Self-Numbers}. Devaiali, 1963: 19 - 20
%%%%%
%%%%%
\end{document}
