\documentclass[12pt]{article}
\usepackage{pmmeta}
\pmcanonicalname{TchebotarevDensityTheorem}
\pmcreated{2013-03-22 12:46:49}
\pmmodified{2013-03-22 12:46:49}
\pmowner{djao}{24}
\pmmodifier{djao}{24}
\pmtitle{Tchebotarev density theorem}
\pmrecord{5}{33093}
\pmprivacy{1}
\pmauthor{djao}{24}
\pmtype{Theorem}
\pmcomment{trigger rebuild}
\pmclassification{msc}{11R37}
\pmclassification{msc}{11R44}
\pmclassification{msc}{11R45}
\pmsynonym{Chebotarev density theorem}{TchebotarevDensityTheorem}

\endmetadata

% this is the default PlanetMath preamble.  as your knowledge
% of TeX increases, you will probably want to edit this, but
% it should be fine as is for beginners.

% almost certainly you want these
\usepackage{amssymb}
\usepackage{amsmath}
\usepackage{amsfonts}

% used for TeXing text within eps files
%\usepackage{psfrag}
% need this for including graphics (\includegraphics)
%\usepackage{graphicx}
% for neatly defining theorems and propositions
%\usepackage{amsthm}
% making logically defined graphics
%%%\usepackage{xypic} 

% there are many more packages, add them here as you need them

% define commands here
\begin{document}
Let $L/K$ be any finite Galois extension of number fields with Galois group $G$. For any conjugacy class $C \subset G$, the subset of prime ideals $\mathfrak{p} \subset K$ which are unramified in $L$ and satisfy the property
$$
[L/K,\mathfrak{P}] \in C\ \text{for any prime }\ \mathfrak{P} \subset L\ \text{containing }\ \mathfrak{p}
$$
has analytic density $\frac{|C|}{|G|}$, where $[L/K,\mathfrak{P}]$ denotes the Artin symbol at $\mathfrak{P}$.

Note that the conjugacy class of $[L/K,\mathfrak{P}]$ is independent of the choice of prime $\mathfrak{P}$ lying over $\mathfrak{p}$, since any two such choices of primes are related by a Galois automorphism and their corresponding Artin symbols are conjugate by this same automorphism.
%%%%%
%%%%%
\end{document}
