\documentclass[12pt]{article}
\usepackage{pmmeta}
\pmcanonicalname{TableOfContinuedFractionsOfsqrtnFor1N102}
\pmcreated{2013-03-22 17:30:25}
\pmmodified{2013-03-22 17:30:25}
\pmowner{PrimeFan}{13766}
\pmmodifier{PrimeFan}{13766}
\pmtitle{table of continued fractions of $\sqrt{n}$ for $1 < n < 102$}
\pmrecord{4}{39896}
\pmprivacy{1}
\pmauthor{PrimeFan}{13766}
\pmtype{Data Structure}
\pmcomment{trigger rebuild}
\pmclassification{msc}{11A25}

% this is the default PlanetMath preamble.  as your knowledge
% of TeX increases, you will probably want to edit this, but
% it should be fine as is for beginners.

% almost certainly you want these
\usepackage{amssymb}
\usepackage{amsmath}
\usepackage{amsfonts}

% used for TeXing text within eps files
%\usepackage{psfrag}
% need this for including graphics (\includegraphics)
%\usepackage{graphicx}
% for neatly defining theorems and propositions
%\usepackage{amsthm}
% making logically defined graphics
%%%\usepackage{xypic}

% there are many more packages, add them here as you need them

% define commands here

\begin{document}
The simple continued fractions for the square roots of positive integers (which aren't perfect powers) are non-terminating but they are periodic. In the following table, the square roots of the integers from 2 to 101 (excluding perfect powers) are listed in compact form: first the integer part followed by semicolon, then the periodic part stated once, its individual terms separated by commas. For example, the notation ``14; 14, 28'' for 198 means $$\sqrt{198} = 14 + \frac{1}{14 + \frac{1}{28 + \frac{1}{14 + \frac{1}{28 + \ldots}}}},$$ where the dots mean a periodic repetition of 14 and 28 in the denominators.

\begin{tabular}{|r|l|}
$n$ & Continued fraction of $\sqrt{n}$ \\
2 & 1; 2 \\
3 & 1; 1, 2 \\
5 & 2; 4 \\
6 & 2; 2, 4 \\
7 & 2; 1, 1, 1, 4 \\
8 & 2; 1, 4 \\
10 & 3; 6 \\
11 & 3; 3, 6 \\
12 & 3; 2, 6 \\
13 & 3; 1, 1, 1, 1, 6 \\
14 & 3; 1, 2, 1, 6 \\
15 & 3; 1, 6 \\
17 & 4; 8 \\
18 & 4; 4, 8 \\
19 & 4; 2, 1, 3, 1, 2, 8 \\
20 & 4; 2, 8 \\
21 & 4; 1, 1, 2, 1, 1, 8 \\
22 & 4; 1, 2, 4, 2, 1, 8 \\
23 & 4; 1, 3, 1, 8 \\
24 & 4; 1, 8 \\
26 & 5; 10 \\
27 & 5; 5, 10 \\
28 & 5; 3, 2, 3, 10 \\
29 & 5; 2, 1, 1, 2, 10 \\
30 & 5; 2, 10 \\
31 & 5; 1, 1, 3, 5, 3, 1, 1, 10 \\
32 & 5; 1, 1, 1, 10 \\
33 & 5; 1, 2, 1, 10 \\
34 & 5; 1, 4, 1, 10 \\
35 & 5; 1, 10 \\
37 & 6; 12 \\
38 & 6; 6, 12 \\
39 & 6; 4, 12 \\
40 & 6; 3, 12 \\
41 & 6; 2, 2, 12 \\
42 & 6; 2, 12 \\
43 & 6; 1, 1, 3, 1, 5, 1, 3, 1, 1, 12 \\
44 & 6; 1, 1, 1, 2, 1, 1, 1, 12 \\
45 & 6; 1, 2, 2, 2, 1, 12 \\
46 & 6; 1, 3, 1, 1, 2, 6, 2, 1, 1, 3, 1, 12 \\
47 & 6; 1, 5, 1, 12 \\
48 & 6; 1, 12 \\
50 & 7; 14 \\
51 & 7; 7, 14 \\
52 & 7; 4, 1, 2, 1, 4, 14 \\
53 & 7; 3, 1, 1, 3, 14 \\
54 & 7; 2, 1, 6, 1, 2, 14 \\
55 & 7; 2, 2, 2, 14 \\
56 & 7; 2, 14 \\
57 & 7; 1, 1, 4, 1, 1, 14 \\
58 & 7; 1, 1, 1, 1, 1, 1, 14 \\
59 & 7; 1, 2, 7, 2, 1, 14 \\
60 & 7; 1, 2, 1, 14 \\
61 & 7; 1, 4, 3, 1, 2, 2, 1, 3, 4, 1, 14 \\
62 & 7; 1, 6, 1, 14 \\
63 & 7; 1, 14 \\
65 & 8; 16 \\
66 & 8; 8, 16 \\
67 & 8; 5, 2, 1, 1, 7, 1, 1, 2, 5, 16 \\
68 & 8; 4, 16 \\
69 & 8; 3, 3, 1, 4, 1, 3, 3, 16 \\
70 & 8; 2, 1, 2, 1, 2, 16 \\
71 & 8; 2, 2, 1, 7, 1, 2, 2, 16 \\
72 & 8; 2, 16 \\
73 & 8; 1, 1, 5, 5, 1, 1, 16 \\
74 & 8; 1, 1, 1, 1, 16 \\
75 & 8; 1, 1, 1, 16 \\
76 & 8; 1, 2, 1, 1, 5, 4, 5, 1, 1, 2, 1, 16 \\
77 & 8; 1, 3, 2, 3, 1, 16 \\
78 & 8; 1, 4, 1, 16 \\
79 & 8; 1, 7, 1, 16 \\
80 & 8; 1, 16 \\
82 & 9; 18 \\
83 & 9; 9, 18 \\
84 & 9; 6, 18 \\
85 & 9; 4, 1, 1, 4, 18 \\
86 & 9; 3, 1, 1, 1, 8, 1, 1, 1, 3, 18 \\
87 & 9; 3, 18 \\
88 & 9; 2, 1, 1, 1, 2, 18 \\
89 & 9; 2, 3, 3, 2, 18 \\
90 & 9; 2, 18 \\
91 & 9; 1, 1, 5, 1, 5, 1, 1, 18 \\
92 & 9; 1, 1, 2, 4, 2, 1, 1, 18 \\
93 & 9; 1, 1, 1, 4, 6, 4, 1, 1, 1, 18 \\
94 & 9; 1, 2, 3, 1, 1, 5, 1, 8, 1, 5, 1, 1, 3, 2, 1, 18 \\
95 & 9; 1, 2, 1, 18 \\
96 & 9; 1, 3, 1, 18 \\
97 & 9; 1, 5, 1, 1, 1, 1, 1, 1, 5, 1, 18 \\
98 & 9; 1, 8, 1, 18 \\
99 & 9; 1, 18 \\
101 & 10; 20 \\
\end{tabular}

As the table shows, the periodic part ends with $2 \lfloor \sqrt{n} \rfloor$.
%%%%%
%%%%%
\end{document}
