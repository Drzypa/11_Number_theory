\documentclass[12pt]{article}
\usepackage{pmmeta}
\pmcanonicalname{ErdHosTuranConjecture}
\pmcreated{2013-03-22 13:27:11}
\pmmodified{2013-03-22 13:27:11}
\pmowner{bbukh}{348}
\pmmodifier{bbukh}{348}
\pmtitle{Erd\H{o}s-Turan conjecture}
\pmrecord{8}{34018}
\pmprivacy{1}
\pmauthor{bbukh}{348}
\pmtype{Conjecture}
\pmcomment{trigger rebuild}
\pmclassification{msc}{11B13}
\pmclassification{msc}{11B34}
\pmclassification{msc}{11B05}
%\pmkeywords{Sidon set}
%\pmkeywords{thin bases}
\pmrelated{SidonSet}

\usepackage{amssymb}
\usepackage{amsmath}
\usepackage{amsfonts}

\newcommand*{\naturals}{\mathbb{N}}
\newcommand*{\integers}{\mathbb{Z}}

\makeatletter
\@ifundefined{bibname}{}{\renewcommand{\bibname}{References}}
\makeatother
\begin{document}
Erd\H{o}s-Turan conjecture asserts there exist no \PMlinkname{asymptotic basis}{AsymptoticBasis} $A\subset \naturals_0$ of order $2$ such that its representation function
\begin{equation*}
r'_{A,2}(n)=\sum_{\substack{a_1+a_2=n\\a_1\leq a_2}} 1
\end{equation*}
is bounded.

Alternatively, the question can be phrased as whether there exists a power series $F$ with coefficients $0$ and $1$ such that all coefficients of $F^2$ are greater than $0$, but are bounded.

If we replace set of nonnegative integers by the set of all integers, then the question was settled by Nathanson\cite{cite:nathanson_everyrepr} in negative, that is, there exists a set $A\subset \integers$ such that $r'_{A,2}(n)=1$.

\begin{thebibliography}{1}

\bibitem{cite:halberstam_sequences}
Heini Halberstam and Klaus~Friedrich Roth.
\newblock {\em Sequences}.
\newblock Springer-Verlag, second edition, 1983.
\newblock \PMlinkexternal{Zbl 0498.10001}{http://www.emis.de/cgi-bin/zmen/ZMATH/en/quick.html?type=html&an=0498.10001}.

\bibitem{cite:nathanson_everyrepr}
Melvyn~B. Nathanson.
\newblock Every function is the representation function of an additive basis
  for the integers.
\newblock \PMlinkexternal{arXiv:math.NT/0302091}{http://front.math.ucdavis.edu/math.NT/0302091}.

\end{thebibliography}
%
%@BOOK{cite:halberstam_sequences,
% author    = {Heini Halberstam and Klaus Friedrich Roth},
% title     = "Sequences",
% publisher = {Springer-Verlag},
% year      = {1983},
% edition   = {Second}
%}
%
%@MISC{cite:nathanson_everyrepr,
%    title = {Every function is the representation function of an additive
%        basis for the integers},
%    author = {Melvyn B. Nathanson},
%    note = {arXiv:math.NT/0302091}
%}
%%%%%
%%%%%
\end{document}
