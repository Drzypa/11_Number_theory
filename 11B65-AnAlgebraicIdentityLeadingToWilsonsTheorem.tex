\documentclass[12pt]{article}
\usepackage{pmmeta}
\pmcanonicalname{AnAlgebraicIdentityLeadingToWilsonsTheorem}
\pmcreated{2013-03-22 14:31:38}
\pmmodified{2013-03-22 14:31:38}
\pmowner{GeraW}{6138}
\pmmodifier{GeraW}{6138}
\pmtitle{an algebraic identity leading to Wilson's theorem}
\pmrecord{16}{36069}
\pmprivacy{1}
\pmauthor{GeraW}{6138}
\pmtype{Result}
\pmcomment{trigger rebuild}
\pmclassification{msc}{11B65}
\pmclassification{msc}{05A10}
\pmrelated{Factorial}
\pmrelated{WilsonsTheorem}

\endmetadata

% this is the default PlanetMath preamble.  as your knowledge
% of TeX increases, you will probably want to edit this, but
% it should be fine as is for beginners.

% almost certainly you want these
\usepackage{amssymb}
\usepackage{amsmath}
\usepackage{amsfonts}

% used for TeXing text within eps files
%\usepackage{psfrag}
% need this for including graphics (\includegraphics)
%\usepackage{graphicx}
% for neatly defining theorems and propositions
%\usepackage{amsthm}
% making logically defined graphics
%%%\usepackage{xypic}

% there are many more packages, add them here as you need them

% define commands here
\begin{document}
For any positive integer $n$ and any real or complex number $x$,

\[ \sum_{k=0}^{n} (-1)^k {n \choose k}(x-k)^n = n! \]

Furthermore, if $n>m$ then

\[ \sum_{k=0}^{n} (-1)^k {n \choose k}(x-k)^m = 0 \]


\begin{thebibliography}{1}

\bibitem{Rui96}
S.~M Ruiz.
\newblock An algebraic identity leading to Wilson's theorem.
\newblock {\em The Mathemtical Gazette}, 80(489):579--582, November 1996.
\newblock \PMlinkexternal{math.GM/0406086}{http://arxiv.org/abs/math.GM/0406086}.


\end{thebibliography}
%%%%%
%%%%%
\end{document}
