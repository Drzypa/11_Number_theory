\documentclass[12pt]{article}
\usepackage{pmmeta}
\pmcanonicalname{RamificationIndex}
\pmcreated{2013-03-22 12:36:36}
\pmmodified{2013-03-22 12:36:36}
\pmowner{djao}{24}
\pmmodifier{djao}{24}
\pmtitle{ramification index}
\pmrecord{17}{32868}
\pmprivacy{1}
\pmauthor{djao}{24}
\pmtype{Definition}
\pmcomment{trigger rebuild}
\pmclassification{msc}{11S15}
\pmclassification{msc}{30F99}
\pmclassification{msc}{30F99}
\pmclassification{msc}{12F99}
\pmclassification{msc}{13B02}
\pmclassification{msc}{14E22}
\pmsynonym{ramify}{RamificationIndex}
\pmsynonym{ramified}{RamificationIndex}
\pmsynonym{unramified}{RamificationIndex}
\pmsynonym{ramification degree}{RamificationIndex}
\pmsynonym{ramification}{RamificationIndex}
\pmrelated{NumberField}
\pmrelated{DecompositionGroup}
\pmrelated{UnramifiedExtensionsAndClassNumberDivisibility}
\pmrelated{SplittingAndRamificationInNumberFieldsAndGaloisExtensions}
\pmdefines{branch point}
\pmdefines{ramification point}

\endmetadata

% this is the default PlanetMath preamble.  as your knowledge
% of TeX increases, you will probably want to edit this, but
% it should be fine as is for beginners.

% almost certainly you want these
\usepackage{amssymb}
\usepackage{amsmath}
\usepackage{amsfonts}

% used for TeXing text within eps files
%\usepackage{psfrag}
% need this for including graphics (\includegraphics)
%\usepackage{graphicx}
% for neatly defining theorems and propositions
\usepackage{amsthm}
% making logically defined graphics
%%%\usepackage{xypic}
\usepackage{epsfig}

% there are many more packages, add them here as you need them

% define commands here

\newcommand{\p}{{\mathfrak{p}}}
\newcommand{\m}{{\mathfrak{m}}}
\renewcommand{\P}{{\mathfrak{P}}}
\newcommand{\C}{\mathbb{C}}
\newcommand{\R}{\mathbb{R}}
\newcommand{\Z}{\mathbb{Z}}
\newcommand{\N}{\mathbb{N}}
\renewcommand{\H}{\mathcal{H}}
\newcommand{\A}{\mathbb{A}}
\renewcommand{\c}{\mathcal{C}}
\renewcommand{\O}{\mathcal{O}}
\newcommand{\D}{\mathcal{D}}
\newcommand{\lra}{\longrightarrow}
\renewcommand{\div}{\mid}
\newcommand{\res}{\operatorname{res}}
\newcommand{\Spec}{\operatorname{Spec}}
\newcommand{\id}{\operatorname{id}}
\newcommand{\diff}{\operatorname{diff}}
\newcommand{\incl}{\operatorname{incl}}
\newcommand{\Hom}{\operatorname{Hom}}
\renewcommand{\Re}{\operatorname{Re}}
\newcommand{\intersect}{\cap}
\newcommand{\union}{\cup}
\newcommand{\bigintersect}{\bigcap}
\newcommand{\bigunion}{\bigcup}
\newcommand{\ilim}{\,\underset{\longleftarrow}{\lim}\,}

\newtheorem{theorem}{Theorem}
\newtheorem{proposition}[theorem]{Proposition}
\newtheorem{lemma}[theorem]{Lemma}
\newtheorem{corollary}[theorem]{Corollary}

\theoremstyle{definition}
\newtheorem{definition}[theorem]{Definition}
\newtheorem{example}[theorem]{Example}
\begin{document}
\section{Ramification in number fields}

\begin{definition}[First definition]\label{definition-1}
Let $L/K$ be an extension of number fields. Let $\p$ be a nonzero
prime ideal in the ring of integers $\O_K$ of $K$, and suppose the
ideal $\p \O_L \subset \O_L$ factors as
$$
\p \O_L = \prod_{i=1}^n \P_i^{e_i}
$$
for some prime ideals $\P_i \subset \O_L$ and exponents $e_i \in
\N$. The natural number $e_i$ is called the {\em ramification index}
of $\P_i$ over $\p$. It is often denoted $e(\P_i/\p)$. If $e_i > 1$
for any $i$, then we say the ideal $\p$ {\em ramifies} in $L$.

Likewise, if $\P$ is a nonzero prime ideal in $\O_L$, and $\p := \P
\intersect \O_K$, then we say $\P$ {\em ramifies} over $K$ if the
ramification index $e(\P/\p)$ of $\P$ in the factorization of the
ideal $\p \O_L \subset \O_L$ is greater than 1. That is, a prime $\p$
in $\O_K$ ramifies in $L$ if at least one prime $\P$ dividing $\p
\O_L$ ramifies over $K$. If $L/K$ is a Galois extension, then the
ramification indices of all the primes dividing $\p \O_L$ are equal,
since the Galois group is transitive on this set of primes.
\end{definition}

\subsection{The local view}

The phenomenon of ramification has an equivalent interpretation in
terms of local rings. With $L/K$ as before, let $\P$ be a prime in
$\O_L$ with $\p := \P \intersect \O_K$. Then the induced map of
localizations $(\O_K)_\p \hookrightarrow (\O_L)_\P$ is a local
homomorphism of local rings (in fact, of discrete valuation rings),
and the ramification index of $\P$ over $\p$ is the unique natural
number $e$ such that
$$
\p (\O_L)_\P = (\P (\O_L)_\P)^e \subset (\O_L)_\P.
$$

An astute reader may notice that this formulation of ramification
index does not require that $L$ and $K$ be number fields, or even that
they play any role at all. We take advantage of this fact here to give
a second, more general definition.

\begin{definition}[Second definition]\label{definition-2}
Let $\iota: A \lra B$ be any ring homomorphism. Suppose $\P \subset B$
is a prime ideal such that the localization $B_\P$ of $B$ at $\P$ is a
discrete valuation ring. Let $\p$ be the prime ideal $\iota^{-1}(\P)
\subset A$, so that $\iota$ induces a local homomorphism $\iota_\P:
A_\p \lra B_\P$. Then the {\em ramification index} $e(\P/\p)$ is
defined to be the unique natural number such that
$$
\iota(\p)B_\P = (\P B_\P)^{e(\P/\p)} \subset B_\P,
$$
or $\infty$ if $\iota(\p)B_\P = (0)$.
\end{definition}

The reader who is not interested in local rings may assume that $A$
and $B$ are unique factorization domains, in which case $e(\P/\p)$ is
the exponent of $\P$ in the factorization of the ideal $\iota(\p) B$,
just as in our first definition (but without the requirement that the
rings $A$ and $B$ originate from number fields).

There is of course much more that can be said about ramification
indices even in this purely algebraic setting, but we limit ourselves
to the following remarks:

\begin{enumerate}
\item Suppose $A$ and $B$ are themselves discrete valuation rings,
with respective maximal ideals $\p$ and $\P$. Let $\hat{A} := \ilim
A/\p^n$ and $\hat{B} := \ilim B/\P^n$ be the completions of $A$
and $B$ with respect to $\p$ and $\P$. Then
\begin{equation}\label{completion}
e(\P/\p) = e(\P\hat{B}/\p\hat{A}).
\end{equation}
In other words, the ramification index of $\P$ over $\p$ in the
$A$--algebra $B$ equals the ramification index in the completions of
$A$ and $B$ with respect to $\p$ and $\P$.
\item Suppose $A$ and $B$ are Dedekind domains, with respective
fraction fields $K$ and $L$. If $B$ equals the integral closure of $A$
in $L$, then
\begin{equation}\label{sum-formula}
\sum_{\P \div \p} e(\P/\p) f(\P/\p) \leq [L:K],
\end{equation}
where $\P$ ranges over all prime ideals in $B$ that divide $\p B$, and
$f(\P/\p) := \dim_{A/\p}(B/\P)$ is the inertial degree of $\P$ over
$\p$. Equality holds in Equation~\eqref{sum-formula} whenever $B$ is
finitely generated as an $A$--module.
\end{enumerate}

\section{Ramification in algebraic geometry}

The word ``ramify'' in English means ``to divide into two or more
branches,'' and we will show in this section that the mathematical term
lives up to its common English meaning.

\begin{definition}[Algebraic version]\label{algebraic}
Let $f: C_1 \lra C_2$ be a non--constant regular morphism of curves
(by which we mean one dimensional nonsingular irreducible algebraic
varieties) over an algebraically closed field $k$. Then $f$ has a
nonzero degree $n := \deg f$, which can be defined in any of the
following ways:
\begin{itemize}
\item The number of points in a generic fiber $f^{-1}(p)$, for $p \in
C_2$
\item The maximum number of points in $f^{-1}(p)$, for $p \in C_2$
\item The degree of the extension $k(C_1)/f^* k(C_2)$ of function
fields
\end{itemize}

There is a finite set of points $p \in C_2$ for which the inverse
image $f^{-1}(p)$ does not have size $n$, and we call these points the
{\em branch points} or {\em ramification points} of $f$. If $P \in
C_1$ with $f(P) = p$, then the {\em ramification index} $e(P/p)$ of
$f$ at $P$ is the ramification index obtained algebraically from
Definition~\ref{definition-2} by taking
\begin{itemize}
\item $A = k[C_2]_p$, the local ring consisting of all rational
functions in the function field $k(C_2)$ which are regular at $p$.
\item $B = k[C_1]_P$, the local ring consisting of all rational
functions in the function field $k(C_1)$ which are regular at $P$.
\item $\p = \m_p$, the maximal ideal in $A$ consisting of all
functions which vanish at $p$.
\item $\P = \m_P$, the maximal ideal in $B$ consisting of all
functions which vanish at $P$.
\item $\iota = f^*_p: k[C_2]_p \hookrightarrow k[C_1]_P$, the map on
the function fields induced by the morphism $f$.
\end{itemize}
\end{definition}

\begin{example}\label{example-1}
The picture in Figure~\ref{parabola-figure} may be worth a thousand words. Let $k = \C$ and
$C_1 = C_2 = \C = \A^1_\C$. Take the map $f: \C \lra \C$ given by
$f(y) = y^2$. Then $f$ is plainly a map of degree 2, and every point
in $C_2$ except for 0 has two preimages in $C_1$. The point 0 is thus
a ramification point of $f$ of index 2, and we have drawn the graph of $f$ near $0$.
\begin{figure}
\epsfig{file=curve.eps}
\caption{The function $f(y) = y^2$ near $y=0$.}
\label{parabola-figure}
\end{figure}

Note that we have only drawn the real locus of $f$ because that is all
that can fit into two dimensions. We see from the figure that a
typical point on $C_2$ such as the point $x = 1$ has two points in
$C_1$ which map to it, but that the point $x = 0$ has only one
corresponding point of $C_1$ which ``branches'' or ``ramifies'' into two
distinct points of $C_1$ whenever one moves away from 0.
\end{example}

\subsection{Relation to the number field case}

The relationship between Definition~\ref{definition-2} and
Definition~\ref{algebraic} is easiest to explain in the case where $f$
is a map between affine varieties. When $C_1$ and $C_2$ are affine,
then their coordinate rings $k[C_1]$ and $k[C_2]$ are Dedekind
domains, and the points of the curve $C_1$ (respectively, $C_2$)
correspond naturally with the maximal ideals of the ring $k[C_1]$
(respectively, $k[C_2]$). The ramification points of the curve $C_1$
are then exactly the points of $C_1$ which correspond to maximal
ideals of $k[C_1]$ that ramify in the algebraic sense, with respect to
the map $f^*: k[C_2] \lra k[C_1]$ of coordinate rings.

Equation~\eqref{sum-formula} in this case says
$$
\sum_{P \in f^{-1}(p)} e(P/p) = n,
$$
and we see that the well known formula~\eqref{sum-formula} in number
theory is simply the algebraic analogue of the geometric fact that the
number of points in the fiber of $f$, counting multiplicities, is
always $n$.

\begin{example}
Let $f: \C \lra \C$ be given by $f(y) = y^2$ as in
Example~\ref{example-1}. Since $C_2$ is just the affine line, the
coordinate ring $\C[C_2]$ is equal to $\C[X]$, the polynomial ring in
one variable over $\C$. Likewise, $\C[C_1] = \C[Y]$, and the induced
map $f^*: \C[X] \lra \C[Y]$ is naturally given by $f^*(X) = Y^2$. We
may accordingly identify the coordinate ring $\C[C_2]$ with the
subring $\C[X^2]$ of $\C[X] = \C[C_1]$.

Now, the ring $\C[X]$ is a principal ideal domain, and the maximal
ideals in $\C[X]$ are exactly the principal ideals of the form $(X -
a)$ for any $a \in \C$. Hence the nonzero prime ideals in $\C[X^2]$
are of the form $(X^2 - a)$, and these factor in $\C[X]$ as
$$
(X^2 - a) = (X - \sqrt{a}) (X + \sqrt{a}) \subset \C[X].
$$
Note that the two prime ideals $(X - \sqrt{a})$ and $(X + \sqrt{a})$
of $\C[X]$ are equal only when $a = 0$, so we see that the ideal $(X^2
- a)$ in $\C[X^2]$, corresponding to the point $a \in C_2$, ramifies
in $C_1$ exactly when $a = 0$. We have therefore recovered our
previous geometric characterization of the ramified points of $f$,
solely in terms of the algebraic factorizations of ideals in $\C[X]$.
\end{example}

In the case where $f$ is a map between projective varieties,
Definition~\ref{definition-2} does not directly apply to the
coordinate rings of $C_1$ and $C_2$, but only to those of open covers
of $C_1$ and $C_2$ by affine varieties. Thus we do have an instance of
yet another new phenomenon here, and rather than keep the reader in
suspense we jump straight to the final, most general definition of
ramification that we will give.

\begin{definition}[Final form]\label{final}
Let $f: (X,\O_X) \lra (Y,\O_Y)$ be a morphism of locally ringed
spaces. Let $p \in X$ and suppose that the stalk $(\O_X)_p$ is a
discrete valuation ring. Write $\phi_p: (\O_Y)_{f(p)} \lra (\O_X)_p$
for the induced map of $f$ on stalks at $p$. Then the {\em
ramification index} of $p$ over $Y$ is the unique natural number $e$,
if it exists (or $\infty$ if it does not exist), such that
$$
\phi_p(\m_{f(p)})(\O_X)_p = \m_p^e,
$$
where $\m_p$ and $\m_{f(p)}$ are the respective maximal ideals of
$(\O_X)_p$ and $(\O_Y)_{f(p)}$. We say $p$ is {\em ramified} in $Y$ if
$e > 1$.
\end{definition}

\begin{example}
A ring homomorphism $\iota: A \lra B$ corresponds functorially to a
morphism $\Spec(B) \lra \Spec(A)$ of locally ringed spaces from the
prime spectrum of $B$ to that of $A$, and the algebraic notion of
ramification from Definition~\ref{definition-2} equals the
sheaf--theoretic notion of ramification from Definition~\ref{final}.
\end{example}

\begin{example}
For any morphism of varieties $f: C_1 \lra C_2$, there is an induced
morphism $f^\#$ on the structure sheaves of $C_1$ and $C_2$, which are
locally ringed spaces. If $C_1$ and $C_2$ are curves, then the stalks are one dimensional regular local rings and therefore discrete valuation rings, so in this way we recover the algebraic
geometric definition (Definition~\ref{algebraic}) from the sheaf
definition (Definition~\ref{final}).
\end{example}

\section{Ramification in complex analysis}

Ramification points or branch points in complex geometry are merely a
special case of the high--flown terminology of
Definition~\ref{final}. However, they are important enough to merit a
separate mention here.

\begin{definition}[Analytic version]\label{analytic}
Let $f: M \lra N$ be a holomorphic map of Riemann surfaces. For any $p
\in M$, there exists local coordinate charts $U$ and $V$ around $p$
and $f(p)$ such that $f$ is locally the map $z \mapsto z^e$ from $U$
to $V$. The natural number $e$ is called the {\em ramification index}
of $f$ at $p$, and $p$ is said to be a {\em branch point} or {\em
ramification point} of $f$ if $e > 1$.
\end{definition}

\begin{example}
Take the map $f: \C \lra \C$, $f(y) = y^2$ of
Example~\ref{example-1}. We study the behavior of $f$ near the
unramified point $y=1$ and near the ramified point $y=0$. Near $y=1$,
take the coordinate $w = y-1$ on the domain and $v = x-1$ on the
range. Then $f$ maps $w+1$ to $(w+1)^2$, which in the $v$ coordinate
is $(w+1)^2 - 1 = 2w + w^2$. If we change coordinates to $z = 2w +
w^2$ on the domain, keeping $v$ on the range, then $f(z) = z$, so the
ramification index of $f$ at $y=1$ is equal to 1.

Near $y=0$, the function $f(y) = y^2$ is already in the form $z
\mapsto z^e$ with $e=2$, so the ramification index of $f$ at $y=0$ is
equal to 2.
\end{example}

\subsection{Algebraic--analytic correspondence}

Of course, the analytic notion of ramification given in
Definition~\ref{analytic} can be couched in terms of locally ringed
spaces as well. Any Riemann surface together with its sheaf of
holomorphic functions is a locally ringed space. Furthermore the stalk
at any point is always a discrete valuation ring, because germs of
holomorphic functions have Taylor expansions making the stalk
isomorphic to the power series ring $\C[[z]]$. We can therefore apply
Definition~\ref{final} to any holomorphic map of Riemann surfaces, and
it is not surprising that this process yields the same results as
Definition~\ref{analytic}.

More generally, every map of algebraic varieties $f: V \lra W$ can be
interpreted as a holomorphic map of Riemann surfaces in the usual way,
and the ramification points on $V$ and $W$ under $f$ as algebraic
varieties are identical to their ramification points as Riemann
surfaces. It turns out that the analytic structure may be regarded in
a certain sense as the ``completion'' of the algebraic structure, and in
this sense the algebraic--analytic correspondence between the
ramification points may be regarded as the geometric version of the
equality~\eqref{completion} in number theory.

The algebraic--analytic correspondence of ramification points is
itself only one manifestation of the wide ranging identification
between algebraic geometry and analytic geometry which is explained to
great effect in the seminal paper of Serre~\cite{gaga}.

\begin{thebibliography}{9}
\bibitem{hartshorne} Robin Hartshorne, {\em Algebraic
Geometry}, Springer--Verlag, 1977 (GTM {\bf 52}).
\bibitem{janusz} Gerald Janusz, {\em Algebraic Number Fields, Second
Edition}, American Mathematical Society, 1996 (GSM {\bf 7}).
\bibitem{jost} J\"urgen Jost, {\em Compact Riemann Surfaces},
Springer--Verlag, 1997.
\bibitem{lorenzini} Dino Lorenzini, {\em An Invitation to Arithmetic Geometry}, American Mathematical Society, 1996 (GSM {\bf 9}).
\bibitem{serre} Jean--Pierre Serre, {\em Local Fields},
Springer--Verlag, 1979 (GTM {\bf 67}).
\bibitem{gaga} Jean--Pierre Serre, ``G\'eom\'etrie alg\'ebraique et
g\'eom\'etrie analytique,'' Ann. de L'Inst. Fourier {\bf 6} pp. 1--42,
1955--56.
\bibitem{silverman} Joseph Silverman, {The Arithmetic of Elliptic
Curves}, Springer--Verlag, 1986 (GTM {\bf 106}).
\end{thebibliography}
%%%%%
%%%%%
\end{document}
