\documentclass[12pt]{article}
\usepackage{pmmeta}
\pmcanonicalname{Deficiency}
\pmcreated{2013-03-22 16:46:50}
\pmmodified{2013-03-22 16:46:50}
\pmowner{PrimeFan}{13766}
\pmmodifier{PrimeFan}{13766}
\pmtitle{deficiency}
\pmrecord{6}{39011}
\pmprivacy{1}
\pmauthor{PrimeFan}{13766}
\pmtype{Definition}
\pmcomment{trigger rebuild}
\pmclassification{msc}{11A05}
\pmrelated{Abundance}

% this is the default PlanetMath preamble.  as your knowledge
% of TeX increases, you will probably want to edit this, but
% it should be fine as is for beginners.

% almost certainly you want these
\usepackage{amssymb}
\usepackage{amsmath}
\usepackage{amsfonts}

% used for TeXing text within eps files
%\usepackage{psfrag}
% need this for including graphics (\includegraphics)
%\usepackage{graphicx}
% for neatly defining theorems and propositions
%\usepackage{amsthm}
% making logically defined graphics
%%%\usepackage{xypic}

% there are many more packages, add them here as you need them

% define commands here

\begin{document}
Given an integer $n$ with divisors $d_1, \ldots , d_k$ (where the divisors are in ascending order and $d_1 = 1$, $d_k = n$) the difference $$2n - \left( \sum_{i = 1}^k d_i \right)$$ is the {\em deficiency} of $n$. Or if one prefers, $$n - \left( \sum_{i = 1}^{k - 1} d_i \right).$$ The deficiency is essentially the same thing as the abundance multiplied by $-1$. Thus, the deficiency is positive for deficient numbers, 0 for perfect numbers and negative for abundant numbers.

For example, the divisors of 13 add up to 14, which is 12 less than 26. Therefore, 12 has an deficiency of 12. Another example: the divisors of 14 add up to 24, which is 4 less than 28. The deficiency of the first 72 integers is listed in A033879 of Sloane's OEIS.
%%%%%
%%%%%
\end{document}
