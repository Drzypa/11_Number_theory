\documentclass[12pt]{article}
\usepackage{pmmeta}
\pmcanonicalname{BeprisqueNumber}
\pmcreated{2013-03-22 19:10:11}
\pmmodified{2013-03-22 19:10:11}
\pmowner{PrimeFan}{13766}
\pmmodifier{PrimeFan}{13766}
\pmtitle{beprisque number}
\pmrecord{5}{42076}
\pmprivacy{1}
\pmauthor{PrimeFan}{13766}
\pmtype{Definition}
\pmcomment{trigger rebuild}
\pmclassification{msc}{11A51}

% this is the default PlanetMath preamble.  as your knowledge
% of TeX increases, you will probably want to edit this, but
% it should be fine as is for beginners.

% almost certainly you want these
\usepackage{amssymb}
\usepackage{amsmath}
\usepackage{amsfonts}

% used for TeXing text within eps files
%\usepackage{psfrag}
% need this for including graphics (\includegraphics)
%\usepackage{graphicx}
% for neatly defining theorems and propositions
%\usepackage{amsthm}
% making logically defined graphics
%%%\usepackage{xypic}

% there are many more packages, add them here as you need them

% define commands here

\begin{document}
A {\em beprisque number} $n$ is an integer which is either one more than a prime number and one less than a perfect square, or one more than a square and one less than a prime. That is, either $(n - 1) \in \mathbb{P}$ and $\sqrt{n + 1} \in \mathbb{Z}$ or $\sqrt{n - 1} \in \mathbb{Z}$ and $(n + 1) \in \mathbb{P}$.

The beprisque numbers below a thousand are 1, 2, 3, 8, 10, 24, 48, 80, 82, 168, 224, 226, 360, 440, 442, 728, 840. The listing on the OEIS, A163492, goes further but omits 1. It could be argued that a beprisque number should itself not be a square, but then it becomes difficult to justify why beprisque numbers that are themselves primes are allowed.

The only prime beprisque numbers are 2 and 3. For a prime number to be a beprisque number, it has to neighbor another prime number; the only two primes fitting that bill are of course 2 and 3.

The only odd beprisque numbers are 1 and 3. For an odd number to be a beprisque number, it has to neighbor an even prime, and since 2 is the only even prime, only 1 and 3 can be beprisque numbers.
%%%%%
%%%%%
\end{document}
