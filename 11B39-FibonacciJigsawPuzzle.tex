\documentclass[12pt]{article}
\usepackage{pmmeta}
\pmcanonicalname{FibonacciJigsawPuzzle}
\pmcreated{2013-03-22 18:03:42}
\pmmodified{2013-03-22 18:03:42}
\pmowner{PrimeFan}{13766}
\pmmodifier{PrimeFan}{13766}
\pmtitle{Fibonacci jigsaw puzzle}
\pmrecord{4}{40593}
\pmprivacy{1}
\pmauthor{PrimeFan}{13766}
\pmtype{Result}
\pmcomment{trigger rebuild}
\pmclassification{msc}{11B39}

\endmetadata

% this is the default PlanetMath preamble.  as your knowledge
% of TeX increases, you will probably want to edit this, but
% it should be fine as is for beginners.

% almost certainly you want these
\usepackage{amssymb}
\usepackage{amsmath}
\usepackage{amsfonts}

% used for TeXing text within eps files
%\usepackage{psfrag}
% need this for including graphics (\includegraphics)
\usepackage{graphicx}
% for neatly defining theorems and propositions
%\usepackage{amsthm}
% making logically defined graphics
%%%\usepackage{xypic}

% there are many more packages, add them here as you need them

% define commands here

\begin{document}
In the \emph{Fibonacci jigsaw puzzle}, one is asked to take a square of given area $x^2$ and rearrange it into a rectangle of greater or lesser area by making the lengths correspond to Fibonacci numbers. Of course there is no solution, but the appearance of a valid solution can be created. The square with sides 8 units long (with an area of 64 square units) is the one most commonly used for presenting this purpose.

\begin{center}
\includegraphics{FiboPuzz1}
\end{center}

By cutting the plywood or paper (or whatever material one has used to make the square) as indicated by the green lines, one now has four pieces. Rotate the bottom two pieces 90 degrees in opposite directions. Slide the topmost piece to the right of one of the former bottom pieces, and the other former top piece to the left of one of the former bottom pieces. Most presentations of the puzzle then state that the rectangle has an area of 65 square units and ask how this is possible. The diagram normally shown has a central diagonal that looks misleadingly straight. However, by making this puzzle out of an actual material and placing it on a surface with a differently-patterned surface, it becomes obvious that  there is a gap in the middle of the rectangle.

\begin{center}
\includegraphics{FiboPuzz4}
\end{center}

The area of this gap turns out to be the 1 square unit that was seemingly created by rearranging the square into a rectangle.
%%%%%
%%%%%
\end{document}
