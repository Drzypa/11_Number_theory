\documentclass[12pt]{article}
\usepackage{pmmeta}
\pmcanonicalname{LindelofHypothesis}
\pmcreated{2015-08-22 13:08:30}
\pmmodified{2015-08-22 13:08:30}
\pmowner{pahio}{2872}
\pmmodifier{pahio}{2872}
\pmtitle{Lindel\"of hypothesis}
\pmrecord{8}{42187}
\pmprivacy{1}
\pmauthor{pahio}{2872}
\pmtype{Conjecture}
\pmcomment{trigger rebuild}
\pmclassification{msc}{11M06}

% this is the default PlanetMath preamble.  as your knowledge
% of TeX increases, you will probably want to edit this, but
% it should be fine as is for beginners.

% almost certainly you want these
\usepackage{amssymb}
\usepackage{amsmath}
\usepackage{amsfonts}

% used for TeXing text within eps files
%\usepackage{psfrag}
% need this for including graphics (\includegraphics)
%\usepackage{graphicx}
% for neatly defining theorems and propositions
 \usepackage{amsthm}
% making logically defined graphics
%%%\usepackage{xypic}

% there are many more packages, add them here as you need them

% define commands here

\theoremstyle{definition}
\newtheorem*{thmplain}{Theorem}

\begin{document}
\PMlinkname{Ernst Lindel\"of}{ernstlindelof} presented in 1908 a
conjecture known as \emph{Lindel\"of hypothesis} which concerns 
the \PMlinkescapetext{order} of the Riemann zeta function on the 
``critical line''\, $x = \frac{1}{2}$.\, Up to now (2015), this 
hypothesis has not been proved.\, It is weaker than the Riemann hypothesis such that this latter implies it but not conversely.\\

\textbf{Lindel\"of hypothesis.}\; $\zeta(\frac{1}{2}\!+\!it) 
\;=\; O(t^\varepsilon)$\; for every\, $\varepsilon > 0$\; when\; 
$t \to \infty$.

Here $O$ is the Landau 
\PMlinkname{big ordo}{formaldefinitionoflandaunotation} notation.



\begin{thebibliography}{8}
\bibitem{lindelof}{\sc Ernst Lindel\"of}: ``Quelques remarques sur 
la croissance de la fonction $\zeta(s)$''. --\emph{Bull. Sci. Math.} \textbf{32} (1908).
\end{thebibliography}
%%%%%
%%%%%
\end{document}
