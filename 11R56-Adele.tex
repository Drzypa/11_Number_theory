\documentclass[12pt]{article}
\usepackage{pmmeta}
\pmcanonicalname{Adele}
\pmcreated{2013-03-22 12:39:31}
\pmmodified{2013-03-22 12:39:31}
\pmowner{djao}{24}
\pmmodifier{djao}{24}
\pmtitle{ad\`{e}le}
\pmrecord{5}{32927}
\pmprivacy{1}
\pmauthor{djao}{24}
\pmtype{Definition}
\pmcomment{trigger rebuild}
\pmclassification{msc}{11R56}
\pmrelated{Idele}
\pmdefines{ad\`{e}le group}
\pmdefines{group of ad\`{e}les}

% this is the default PlanetMath preamble.  as your knowledge
% of TeX increases, you will probably want to edit this, but
% it should be fine as is for beginners.

% almost certainly you want these
\usepackage{amssymb}
\usepackage{amsmath}
\usepackage{amsfonts}

% used for TeXing text within eps files
%\usepackage{psfrag}
% need this for including graphics (\includegraphics)
%\usepackage{graphicx}
% for neatly defining theorems and propositions
%\usepackage{amsthm}
% making logically defined graphics
%%%\usepackage{xypic} 

% there are many more packages, add them here as you need them

% define commands here
\newcommand{\p}{{\mathfrak{p}}}
\newcommand{\m}{{\mathfrak{m}}}
\newcommand{\M}{{\mathfrak{M}}}
\renewcommand{\P}{{\mathfrak{P}}}
\newcommand{\C}{\mathbb{C}}
\newcommand{\R}{\mathbb{R}}
\newcommand{\Z}{\mathbb{Z}}
\newcommand{\Q}{\mathbb{Q}}
\newcommand{\N}{\mathbb{N}}
\renewcommand{\H}{\mathcal{H}}
\newcommand{\A}{\mathbb{A}}
\renewcommand{\c}{\mathcal{C}}
\renewcommand{\O}{\mathcal{O}}
\renewcommand{\o}{\mathfrak{o}}
\newcommand{\D}{\mathcal{D}}
\newcommand{\lra}{\longrightarrow}
\renewcommand{\div}{\mid}
\newcommand{\res}{\operatorname{res}}
\newcommand{\Spec}{\operatorname{Spec}}
\newcommand{\Gal}{\operatorname{Gal}}
\newcommand{\id}{\operatorname{id}}
\newcommand{\diff}{\operatorname{diff}}
\newcommand{\incl}{\operatorname{incl}}
\newcommand{\Hom}{\operatorname{Hom}}
\renewcommand{\Re}{\operatorname{Re}}
\newcommand{\intersect}{\cap}
\newcommand{\union}{\cup}
\newcommand{\bigintersect}{\bigcap}
\newcommand{\bigunion}{\bigcup}
\newcommand{\ilim}{\,\underset{\longleftarrow}{\lim}\,}
\begin{document}
Let $K$ be a number field. For each finite prime $v$ of $K$, let
$\o_v$ denote the valuation ring of the completion $K_v$ of $K$ at
$v$. The {\em ad\`ele group} $\A_K$ of $K$ is defined to be the
restricted direct product of the collection of locally compact
additive groups $\{K_v\}$ over all primes $v$ of $K$ (both finite
primes and infinite primes), with respect to the collection of compact
open subgroups $\{\o_v\}$ defined for all finite primes $v$.

The set $\A_K$ inherits addition and multiplication operations
(defined pointwise) which make it into a topological ring. The
original field $K$ embeds as a ring into $\A_K$ via the map
$$
x \mapsto \prod_v x_v.
$$
defined for $x \in K$, where $x_v$ denotes the image of $x$ in $K_v$
under the embedding $K \hookrightarrow K_v$. Note that $x_v \in \o_v$
for all but finitely many $v$, so that the element $x$ is sent under
the above definition into the restricted direct product as claimed.

It turns out that the image of $K$ in $\A_K$ is a discrete set and the
quotient group $\A_K/K$ is a compact space in the quotient topology.
%%%%%
%%%%%
\end{document}
