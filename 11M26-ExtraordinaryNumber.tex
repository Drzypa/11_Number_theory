\documentclass[12pt]{article}
\usepackage{pmmeta}
\pmcanonicalname{ExtraordinaryNumber}
\pmcreated{2013-03-22 19:33:41}
\pmmodified{2013-03-22 19:33:41}
\pmowner{pahio}{2872}
\pmmodifier{pahio}{2872}
\pmtitle{extraordinary number}
\pmrecord{14}{42547}
\pmprivacy{1}
\pmauthor{pahio}{2872}
\pmtype{Definition}
\pmcomment{trigger rebuild}
\pmclassification{msc}{11M26}
\pmclassification{msc}{11A25}
%\pmkeywords{sum of divisors}
%\pmkeywords{Riemann Hypothesis}
\pmrelated{PropertiesOfXiFunction}

\endmetadata

% this is the default PlanetMath preamble.  as your knowledge
% of TeX increases, you will probably want to edit this, but
% it should be fine as is for beginners.

% almost certainly you want these
\usepackage{amssymb}
\usepackage{amsmath}
\usepackage{amsfonts}

% used for TeXing text within eps files
%\usepackage{psfrag}
% need this for including graphics (\includegraphics)
%\usepackage{graphicx}
% for neatly defining theorems and propositions
 \usepackage{amsthm}
% making logically defined graphics
%%%\usepackage{xypic}

% there are many more packages, add them here as you need them

% define commands here

\theoremstyle{definition}
\newtheorem*{thmplain}{Theorem}

\begin{document}
Define the function $G$ for integers $n>1$ by
        $$G(n) \; :=\; \frac{\sigma(n)}{n\ln(\ln{n})},$$
where $\sigma(n)$ is the sum of the positive divisors of $n$.\, A positive integer $N$ is said to be an {\it extraordinary number} if it is composite and
          $$G(N) \;\ge\; \max\{G(N/p),\,G(aN)\}$$
for any prime factor $p$ of $N$ and any multiple $aN$ of $N$.\\

It has been proved in [1] that the Riemann Hypothesis is true iff 4 is the only extraordinary number.\, The proof is based on Gronwall's theorem and Robin's theorem.


\begin{thebibliography}{8}
\bibitem{CNS}{\sc Geoffrey Caveney, Jean-Louis Nicolas, Jonathan Sondow}:\, Robin's theorem, primes, and a new  elementary reformulation of the Riemann Hypothesis.\, $-$ \emph{Integers} \textbf{11} (2011) article A33;\; available directly at \PMlinkexternal{arXiv}{http://arxiv.org/pdf/1110.5078.pdf}.
\end{thebibliography}

%%%%%
%%%%%
\end{document}
