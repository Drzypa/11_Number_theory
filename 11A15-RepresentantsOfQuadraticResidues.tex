\documentclass[12pt]{article}
\usepackage{pmmeta}
\pmcanonicalname{RepresentantsOfQuadraticResidues}
\pmcreated{2013-03-22 19:00:35}
\pmmodified{2013-03-22 19:00:35}
\pmowner{pahio}{2872}
\pmmodifier{pahio}{2872}
\pmtitle{representants of quadratic residues}
\pmrecord{7}{41879}
\pmprivacy{1}
\pmauthor{pahio}{2872}
\pmtype{Theorem}
\pmcomment{trigger rebuild}
\pmclassification{msc}{11A15}
\pmsynonym{representant system of quadratic residues}{RepresentantsOfQuadraticResidues}
\pmrelated{GaussianSum}
\pmrelated{DifferenceOfSquares}
\pmrelated{DivisibilityByPrimeNumber}

\endmetadata

% this is the default PlanetMath preamble.  as your knowledge
% of TeX increases, you will probably want to edit this, but
% it should be fine as is for beginners.

% almost certainly you want these
\usepackage{amssymb}
\usepackage{amsmath}
\usepackage{amsfonts}

% used for TeXing text within eps files
%\usepackage{psfrag}
% need this for including graphics (\includegraphics)
%\usepackage{graphicx}
% for neatly defining theorems and propositions
 \usepackage{amsthm}
% making logically defined graphics
%%%\usepackage{xypic}

% there are many more packages, add them here as you need them

% define commands here

\theoremstyle{definition}
\newtheorem*{thmplain}{Theorem}

\begin{document}
\PMlinkescapeword{complete}
\textbf{Theorem.}\, Let $p$ be a positive odd prime number.\, Then the integers
\begin{align}
1^2,\,2^2,\,\ldots,\,\left(\frac{p\!-\!1}{2}\right)^{\!2}
\end{align}
constitute a \PMlinkescapetext{complete} representant system of incongruent quadratic residues modulo $p$.\, Accordingly, there are $\frac{p\!-\!1}{2}$ quadratic residues and equally many nonresidues modulo $p$.\\

\emph{Proof.}\, Firstly, the numbers (1), being squares, are quadratic residues modulo $p$.\, Secondly, they are incongruent, because a congruence \,$a^2 \equiv b^2 \pmod{p}$\, would imply
$$p \mid a\!+\!b \quad \mbox{or} \quad p \mid a\!-\!b,$$
which is impossible when $a$ and $b$ are different integers among $1,\,2,\,\ldots,\,\frac{p\!-\!1}{2}$.\, Third, if $c$ is any quadratic residue modulo $p$, and therefore the congruence \,$x^2 \equiv c \pmod{p}$\, has a solution $x$, then $x$ is congruent with one of the numbers
$$\pm1,\,\pm2,\,\ldots,\,\pm\frac{p\!-\!1}{2}$$
which form a reduced residue system modulo $p$ (see absolutely least remainders).\, Then $x^2$ and $c$ are congruent with one of the numbers (1).
%%%%%
%%%%%
\end{document}
