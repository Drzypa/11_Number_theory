\documentclass[12pt]{article}
\usepackage{pmmeta}
\pmcanonicalname{CahensConstant}
\pmcreated{2013-03-22 16:24:04}
\pmmodified{2013-03-22 16:24:04}
\pmowner{PrimeFan}{13766}
\pmmodifier{PrimeFan}{13766}
\pmtitle{Cahen's constant}
\pmrecord{5}{38549}
\pmprivacy{1}
\pmauthor{PrimeFan}{13766}
\pmtype{Definition}
\pmcomment{trigger rebuild}
\pmclassification{msc}{11A55}
\pmsynonym{Cahen constant}{CahensConstant}

\endmetadata

% this is the default PlanetMath preamble.  as your knowledge
% of TeX increases, you will probably want to edit this, but
% it should be fine as is for beginners.

% almost certainly you want these
\usepackage{amssymb}
\usepackage{amsmath}
\usepackage{amsfonts}

% used for TeXing text within eps files
%\usepackage{psfrag}
% need this for including graphics (\includegraphics)
%\usepackage{graphicx}
% for neatly defining theorems and propositions
%\usepackage{amsthm}
% making logically defined graphics
%%%\usepackage{xypic}

% there are many more packages, add them here as you need them

% define commands here

\begin{document}
Whereas a simple addition of unit fractions with the terms of Sylvester's sequence as denominators gives as a result the integer 1, an alternating sum $$\sum_{i = 0}^\infty \frac{(-1)^i}{a_i - 1}$$ (where $a_i$ is the $i$th term of Sylvester's sequence) gives the transcendental number known as \emph{Cahen's constant} (after Eug\`ene Cahen) with an approximate decimal value of 0.643410546288338026182254307757564763286587860268239505987 (see A118227 in Sloane's OEIS). Alternatively, we can express Cahen's constant as $$\sum_{j = 0}^\infty \frac{1}{a_{2j}}.$$ The recurrence relation $b_{n + 2} = {b_n}^2b_{n + 1} + b_n$ gives us the terms for the continued fraction representation of this constant: $$1 + \frac{1}{{b_0}^2 + \frac{1}{{b_1}^2 + \frac{1}{{b_3}^2 + \, \cdots}}}$$
%%%%%
%%%%%
\end{document}
