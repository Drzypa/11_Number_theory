\documentclass[12pt]{article}
\usepackage{pmmeta}
\pmcanonicalname{ProofOfWaringsFormula}
\pmcreated{2013-03-22 15:34:29}
\pmmodified{2013-03-22 15:34:29}
\pmowner{kshum}{5987}
\pmmodifier{kshum}{5987}
\pmtitle{proof of Waring's formula}
\pmrecord{7}{37482}
\pmprivacy{1}
\pmauthor{kshum}{5987}
\pmtype{Proof}
\pmcomment{trigger rebuild}
\pmclassification{msc}{11C08}

\endmetadata

% this is the default PlanetMath preamble.  as your knowledge
% of TeX increases, you will probably want to edit this, but
% it should be fine as is for beginners.

% almost certainly you want these
\usepackage{amssymb}
\usepackage{amsmath}
\usepackage{amsfonts}

% used for TeXing text within eps files
%\usepackage{psfrag}
% need this for including graphics (\includegraphics)
%\usepackage{graphicx}
% for neatly defining theorems and propositions
%\usepackage{amsthm}
% making logically defined graphics
%%%\usepackage{xypic}

% there are many more packages, add them here as you need them

% define commands here
\begin{document}
The following is a proof of the Waring's formula using formal
power series. We will work with formal power series in
indeterminate $z$ with coefficients in the ring
$\mathbb{Q}[x_1,\ldots,x_n]$. We also need the following equality
\[
  -\log(1-z) = \sum_{j=1}^\infty \frac{z^j}{j}.
\]

Taking log on both sides of
\[
  1 - \sigma_1z+\ldots + (-1)^n \sigma_n z^n =
  \prod_{m=1}^n(1-x_mz),
\]
we get
\begin{equation}
  \log(1 - \sigma_1z+\ldots + (-1)^n \sigma_n z^n) =
  \sum_{m=1}^n \log(1-x_mz), \label{eq}
\end{equation}
Waring's formula will follow by comparing the coefficients on both
sides.


The right hand side of the above equation equals
\[
  \sum_{m=1}^n \sum_{j=1}^\infty \frac{x_m^j}{j}z^j
\]
or
\[
  \sum_{j=1}^\infty \left( \sum_{m=1}^n  x_m^j \right)
  \frac{z^j}{j}
\]
The  coefficient of $z^k$ is equal to $S_k/k$.

On the other hand, the left hand side of \eqref{eq} can be written
as
\[
 \sum_{j=1}^\infty
 \frac{1}{j}(\sigma_1z-\sigma_2z^2+\ldots+(-1)^{n-1} \sigma_n
 z^n)^j.
\]
For each $j$, the coefficient of $z^k$ in
\[(\sigma_1z-\sigma_2z^2+\ldots+(-1)^{n-1} \sigma_n
 z^n)^j
\]
is
\[\sum_{i_1,\ldots,i_n} (-1)^{i_2+i_4+i_6+\ldots} \frac{j!}{i_1!\cdots
i_n!}\sigma_1^{i_1} \cdots \sigma_n^{i_n},
\]
where the summation is extended over all $n$-tuple
$(i_1,\ldots,i_n)$ whose entries are non-negative integers, such
that
\begin{gather*}
i_1+i_2+\ldots+i_n = j \\
i_1+2i_2+\ldots +ni_n = k.
\end{gather*}
So the coefficient of $z^k$ in the left hand side of \eqref{eq} is
\[
\sum_{j=1}^\infty \sum_{i_1,\ldots,i_n} (-1)^{i_2+i_4+i_6+\ldots}
\frac{(j-1)!}{i_1!\cdots i_n!}\sigma_1^{i_1} \cdots
\sigma_n^{i_n},
\]
or
\[\sum (-1)^{i_2+i_4+i_6+\ldots} \frac{(i_1+\ldots+i_n-1)!}{i_1!\cdots
i_n!}\sigma_1^{i_1} \cdots \sigma_n^{i_n}.
\]
The last summation is over all $(i_1,\ldots, i_n)\in \mathbb{Z}^n$
with non-negative entries such that $i_1+2i_2+\ldots+ni_n=k$.
%%%%%
%%%%%
\end{document}
