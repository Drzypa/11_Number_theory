\documentclass[12pt]{article}
\usepackage{pmmeta}
\pmcanonicalname{MorphicNumber}
\pmcreated{2013-03-22 19:09:51}
\pmmodified{2013-03-22 19:09:51}
\pmowner{pahio}{2872}
\pmmodifier{pahio}{2872}
\pmtitle{morphic number}
\pmrecord{6}{42070}
\pmprivacy{1}
\pmauthor{pahio}{2872}
\pmtype{Definition}
\pmcomment{trigger rebuild}
\pmclassification{msc}{11B39}

% this is the default PlanetMath preamble.  as your knowledge
% of TeX increases, you will probably want to edit this, but
% it should be fine as is for beginners.

% almost certainly you want these
\usepackage{amssymb}
\usepackage{amsmath}
\usepackage{amsfonts}

% used for TeXing text within eps files
%\usepackage{psfrag}
% need this for including graphics (\includegraphics)
%\usepackage{graphicx}
% for neatly defining theorems and propositions
 \usepackage{amsthm}
% making logically defined graphics
%%%\usepackage{xypic}

% there are many more packages, add them here as you need them

% define commands here

\theoremstyle{definition}
\newtheorem*{thmplain}{Theorem}

\begin{document}
The golden ratio \,$\varphi = \frac{1+\sqrt{5}}{2}$\, satisfies the equations
\begin{align}
\begin{cases}
\varphi\!+\!1 \;=\; \varphi^2, \\
\varphi\!-\!1 \;=\; \varphi^{-1}
\end{cases}
\end{align}
from which the latter is obained from the former by dividing by $\varphi$.\,
There is a \PMlinkescapetext{similar} pair of equations satisfied by the plastic number $P$:
\begin{align}
\begin{cases}
P\!+\!1 \;=\; P^3, \\
P\!-\!1 \;=\; P^{-4}
\end{cases}
\end{align}
Here, the latter equation is justified by
$$P^5\!-\!P^4\!-\!1 \;\equiv\; (\underbrace{P^3\!-\!P\!-\!1}_{=\;0})(P^2\!-\!P\!+\!1) $$
when this is divided by $P^4$.\\

An algebraic integer is called a \emph{morphic number}, iff it satisfies a pair of equations
\begin{align}
\begin{cases}
x\!+\!1 \;=\; x^m, \\
x\!-\!1 \;=\; x^{-n}
\end{cases}
\end{align}
for some positive integers $m$ and $n$.\\

Accordingly, the golden ratio and the plastic number are morphic numbers.\, It can be shown that there are no other real morphic numbers greater than 1.

\begin{thebibliography}{8}
\bibitem{naw}{\sc J. Aarts, R. Fokkink, G. Kruijtzer}: Morphic numbers.\, -- \emph{Nieuw  Archief voor Wiskunde} 5/2 (2001).
\end{thebibliography}
%%%%%
%%%%%
\end{document}
