\documentclass[12pt]{article}
\usepackage{pmmeta}
\pmcanonicalname{AnExactSequenceForRayClassGroups}
\pmcreated{2013-03-22 15:42:46}
\pmmodified{2013-03-22 15:42:46}
\pmowner{alozano}{2414}
\pmmodifier{alozano}{2414}
\pmtitle{an exact sequence for ray class groups}
\pmrecord{6}{37660}
\pmprivacy{1}
\pmauthor{alozano}{2414}
\pmtype{Result}
\pmcomment{trigger rebuild}
\pmclassification{msc}{11R29}
\pmrelated{Modulus}
\pmrelated{RayClassField}
\pmrelated{ClassNumbersAndDiscriminantsTopicsOnClassGroups}

\endmetadata

% this is the default PlanetMath preamble.  as your knowledge
% of TeX increases, you will probably want to edit this, but
% it should be fine as is for beginners.

% almost certainly you want these
\usepackage{amssymb}
\usepackage{amsmath}
\usepackage{amsthm}
\usepackage{amsfonts}

% used for TeXing text within eps files
%\usepackage{psfrag}
% need this for including graphics (\includegraphics)
%\usepackage{graphicx}
% for neatly defining theorems and propositions
%\usepackage{amsthm}
% making logically defined graphics
%%%\usepackage{xypic}

% there are many more packages, add them here as you need them

% define commands here

\newtheorem{thm}{Theorem}
\newtheorem{defn}{Definition}
\newtheorem*{prop}{Proposition}
\newtheorem{lemma}{Lemma}
\newtheorem{cor}{Corollary}

\theoremstyle{definition}
\newtheorem{exa}{Example}

% Some sets
\newcommand{\Nats}{\mathbb{N}}
\newcommand{\Ints}{\mathbb{Z}}
\newcommand{\Reals}{\mathbb{R}}
\newcommand{\Complex}{\mathbb{C}}
\newcommand{\Rats}{\mathbb{Q}}
\newcommand{\Gal}{\operatorname{Gal}}
\newcommand{\Cl}{\operatorname{Cl}}
\begin{document}
Let $K$ be a number field, let $\mathcal{O}_K$ be its ring of integers and let $\mathfrak{m}$ be a modulus in $K$, i.e. 
$$\mathfrak{m}=\mathfrak{m}_0\mathfrak{m}_\infty$$
where $\mathfrak{m}_0$ is an integral ideal in $\mathcal{O}_K$ and $\mathfrak{m}_\infty$ is a product of real infinite places (i.e. real archimedean primes). Let $\Cl(K)$ be the ideal class group of $K$ and let $\Cl(K,\mathfrak{m})$ be the ray class group of $K$ of conductor $\mathfrak{m}$. Also, define
$$(\mathcal{O}_K/\mathfrak{m})^\times=(\mathcal{O}_K/\mathfrak{m}_0)^\times \times (\Ints/2\Ints)^{|\mathfrak{m}_\infty|}$$
where $|\mathfrak{m}_\infty|$ denotes the number of real places in $\mathfrak{m}$. Finally, let $U=\mathcal{O}_K^\times$ be the unit group of $K$.

\begin{prop}
The elements above fit in the following exact sequence:
$$U\longrightarrow (\mathcal{O}_K/\mathfrak{m})^\times\longrightarrow \Cl(K,\mathfrak{m})\longrightarrow \Cl(K)\longrightarrow 1.$$
\end{prop}

\begin{exa}
Let $K=\Rats$. Thus, $\Cl(\Rats)$ is trivial and $U=\{ \pm 1\}\cong \Ints/2\Ints$. Let $\mathfrak{m}=p\infty$ where $p>2$ is any prime. Then:
$$(\Ints/\mathfrak{m})^\times=(\Ints/p\Ints)^\times \times (\Ints/2\Ints).$$
The exact sequence now reads:
$$\Ints/2\Ints\longrightarrow (\Ints/p\Ints)^\times \times (\Ints/2\Ints)\longrightarrow \Cl(\Rats,p\infty)\longrightarrow 1.$$
Therefore, $\Cl(\Rats,p\infty)\cong (\Ints/p\Ints)^\times$. In fact, as we know, the \PMlinkid{ray class field}{RayClassField} of $\Rats$ of conductor $\mathfrak{m}=p\infty$ is the cyclotomic field $\Rats(\zeta_p)$ where $\zeta_p$ is any primitive $p$th root of unity. Moreover $$\Gal(\Rats(\zeta_p)/\Rats)\cong \Cl(\Rats,p\infty)\cong (\Ints/p\Ints)^\times.$$ 
Finally notice that the ray class group of $\Rats$ of conductor $\mathfrak{m}=p$ is simply $(\Ints/p\Ints)^\times/\{\pm 1 \}$ which corresponds to the ray class field $\Rats(\zeta_p)^+=\Rats(\zeta_p+\zeta_p^{-1})$, the maximal real subfield of $\Rats(\zeta_p)$.
\end{exa}
%%%%%
%%%%%
\end{document}
