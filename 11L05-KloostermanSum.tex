\documentclass[12pt]{article}
\usepackage{pmmeta}
\pmcanonicalname{KloostermanSum}
\pmcreated{2013-03-22 13:59:33}
\pmmodified{2013-03-22 13:59:33}
\pmowner{mathcam}{2727}
\pmmodifier{mathcam}{2727}
\pmtitle{Kloosterman sum}
\pmrecord{7}{34769}
\pmprivacy{1}
\pmauthor{mathcam}{2727}
\pmtype{Definition}
\pmcomment{trigger rebuild}
\pmclassification{msc}{11L05}
\pmclassification{msc}{43A25}
\pmrelated{GaussSum}

\endmetadata

\usepackage{amssymb}
\usepackage{amsmath}
\usepackage{amsfonts}
\newcommand{\C}{\mathbb{C}}
\newcommand{\Fpstar}{\mathbb{F}_p^*}
\newcommand{\Fqstar}{\mathbb{F}_q^*}
\newcommand{\Fp}{\mathbb{F}_p}
\newcommand{\Fq}{\mathbb{F}_q}
\begin{document}
\PMlinkescapeword{theory}
\PMlinkescapeword{finite}

The Kloosterman sum is one of various trigonometric sums that are
useful in number theory and, more generally, in finite harmonic
analysis.
The original Kloosterman sum is
$$K_p(a,b)=\sum_{x\in\Fpstar}
\exp \left( \frac{2\pi i(ax+bx^{-1})}{p} \right) $$
where $\Fp$ is the field of prime order $p$.
Such sums have been generalized
in a few different ways since their introduction in 1926.
For instance, let $q$ be a prime power, $\Fq$ the field
of $q$ elements, $\chi:\Fqstar\to\C$ a character, and
$\psi:\Fq\to\C$
a mapping such that $\psi(x+y)=\psi(x)\psi(y)$ identically.
The sums
$$K_\psi(\chi|a,b)=\sum_{x\in\Fqstar}\chi(x)\psi(ax+bx^{-1})$$
are of interest, because they come up as Fourier coefficients
of modular forms.

Kloosterman sums are finite analogs of the $K$-Bessel
functions of this kind:
$$K_s(a)=\frac{1}{2}
\int_0^\infty
x^{s-1}\exp\left(\frac{-a(x+x^{-1})}{2}\right) dx$$
where $\Re(a)>0$.
%%%%%
%%%%%
\end{document}
