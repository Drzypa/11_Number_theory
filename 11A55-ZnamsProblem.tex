\documentclass[12pt]{article}
\usepackage{pmmeta}
\pmcanonicalname{ZnamsProblem}
\pmcreated{2013-03-22 15:47:39}
\pmmodified{2013-03-22 15:47:39}
\pmowner{Mravinci}{12996}
\pmmodifier{Mravinci}{12996}
\pmtitle{Zn\'am's problem}
\pmrecord{8}{37755}
\pmprivacy{1}
\pmauthor{Mravinci}{12996}
\pmtype{Definition}
\pmcomment{trigger rebuild}
\pmclassification{msc}{11A55}
\pmsynonym{Znam's problem}{ZnamsProblem}

\endmetadata

% this is the default PlanetMath preamble.  as your knowledge
% of TeX increases, you will probably want to edit this, but
% it should be fine as is for beginners.

% almost certainly you want these
\usepackage{amssymb}
\usepackage{amsmath}
\usepackage{amsfonts}

% used for TeXing text within eps files
%\usepackage{psfrag}
% need this for including graphics (\includegraphics)
%\usepackage{graphicx}
% for neatly defining theorems and propositions
%\usepackage{amsthm}
% making logically defined graphics
%%%\usepackage{xypic}

% there are many more packages, add them here as you need them

% define commands here
\begin{document}
Given a length $k$, is it possible to construct a set of integers $n_1, \ldots , n_k$ such that each

$$n_i |(1 + \prod_{j \ne i}^n n_j)$$

as a proper divisor? This is \emph{Zn\'am's problem}.

This problem has solutions for $k > 4$, and all solutions for $4 < k < 9$ have been found, and a few for higher $k$ are known. The Sylvester sequence provides many of the solutions. At Wayne \PMlinkescapetext{State} University in 2001, Brenton and Vasiliu devised an algorithm to exhaustively search for solutions for a given length, and thus they found all solutions for $k = 8$. Their algorithm, though smarter than a brute force search, is still computationally intense the larger $k$ gets.

Solutions to the problem have applications in continued fractions and perfectly weighted graphs.

The problem is believed to have been first posed by \PMlinkname{\v{S}tefan Zn\'am}{VStefanZnam} in 1972. Qi Sun proved in 1983 that there are solutions for all $k > 4$.

References

Brenton, L, and Vasiliu, A. ``Znam's Problem." Math. Mag. {\bf 75}, 3-11, 2002.
%%%%%
%%%%%
\end{document}
