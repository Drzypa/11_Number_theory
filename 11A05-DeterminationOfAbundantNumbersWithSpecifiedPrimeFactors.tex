\documentclass[12pt]{article}
\usepackage{pmmeta}
\pmcanonicalname{DeterminationOfAbundantNumbersWithSpecifiedPrimeFactors}
\pmcreated{2013-03-22 16:47:41}
\pmmodified{2013-03-22 16:47:41}
\pmowner{rspuzio}{6075}
\pmmodifier{rspuzio}{6075}
\pmtitle{determination of abundant numbers with specified prime factors}
\pmrecord{12}{39028}
\pmprivacy{1}
\pmauthor{rspuzio}{6075}
\pmtype{Theorem}
\pmcomment{trigger rebuild}
\pmclassification{msc}{11A05}

\endmetadata

% this is the default PlanetMath preamble.  as your knowledge
% of TeX increases, you will probably want to edit this, but
% it should be fine as is for beginners.

% almost certainly you want these
\usepackage{amssymb}
\usepackage{amsmath}
\usepackage{amsfonts}

% used for TeXing text within eps files
%\usepackage{psfrag}
% need this for including graphics (\includegraphics)
%\usepackage{graphicx}
% for neatly defining theorems and propositions
\usepackage{amsthm}
% making logically defined graphics
%%%\usepackage{xypic}

% there are many more packages, add them here as you need them

% define commands here

\newtheorem{thm}{Theorem}
\begin{document}
The formula for sums of factors may be used to
find all abundant numbers with a specified set
of prime factors or that no such numbers 
exist.  To accomplish this, we first do a 
little algebraic manipulation to our formula.

\begin{thm}
A number $n$ whose factorization
into prime numbers is $\prod_{i=1}^k p_i^{m_1}$
is abundant if and only if
\[
\prod_{i=1}^k
\left( {
1 - p_i^{-m_i - 1}  \over
1 - p_i^{-1}
} \right)
> 2.
\]
\end{thm}

\begin{proof}
By definition $n$ is abundant, if the sum of
the proper divisors of $n$ is greater than $n$.  
Using our formula, this is equivalent to the condition
\[
\prod_{i=1}^k
\left( {
p_i^{m_i + 1} - 1 \over p_i - 1
} \right) >
2 \prod_{i=1}^k p_i^{m_i}.
\]
Dividing the $k$-th term in the product on
the left-hand side by the $k$-th term on the
right-hand side,
\[
{1 \over p_i^{m_i}}
{p_i^{m_i + 1} - 1 \over p_i - 1} =
{p_i^{-m_i - 1} \over p_i^{-1}}
{p_i - p_i^{-m_i} \over p_i - 1} =
{1 - p_i^{-m_i - 1} \over 1 - p_i^{-1}},
\]
so the condition becomes
\[
\prod_{i=1}^k
\left( {
1 - p_i^{-m_i - 1} \over 1 - p_i^{-1}
} \right)
> 2
\]
\end{proof}

Note that each of the terms in the product is bigger than 1.
Furthemore, the $k$-th term is bounded by
\[
{1 \over 1 - p_i^{-1}} = 
{p_i \over p_i - 1}.
\]
This means that it is only possible to have an abundant
number whose prime factors are $\{p_i \mid 1 \le i \le k\}$
if
\[
\prod_{i=1}^k
\left( {p_i \over p_i - 1} \right)
> 2.
\]
As it turns out, the convers also holds, so we have a 
nice criterion for determining when a set of prime
numbers happens to be the set of prime divisors of 
an abundant number.

\begin{thm}
A finite set $S$ of prime numbers is the set of
prime divisors of an abundant number if and only if
\[
\prod_{p \in S} \left( {p \over p - 1} \right) > 2.
\]
\end{thm}

\begin{proof}
As described above, if $S$ is a set of prime factors
of an abundant number, then we may bound each term 
in the inequality of the previous theorem to obtain
the inequality in the current theorem.  Assume, then
that $S$ is a finite set of prime numbers which
satisfies said inequality.  Then, by continuity,
there must exist a real number $\epsilon > 0$ such
that
\[
\prod_{p \in S} 
\left( {p \over p - 1} - x \right) 
> 2
\]
whenver $0 < x < \epsilon$.  Since $\lim_{k \to \infty} 
n^{-k} = 0$ when $n > 1$, we can, for every $p \in S$, 
find an $m(p)$ such that
\[
{p^{m(p)} \over p - 1} < \epsilon.
\]
Hence, 
\[
\prod_{p \in S}
\left( {p \over p - 1} - 
{p^{m(p)} \over p - 1} \right) =
\prod_{p \in S}
\left( {1 - p^{-m(p)-1} \over 1 - p^{-1}} \right) > 2
\]
so, by the previous theorem, $\prod_{p \in S} p^{m(p)}$
is an abundant number.
\end{proof}


%%%%%
%%%%%
\end{document}
