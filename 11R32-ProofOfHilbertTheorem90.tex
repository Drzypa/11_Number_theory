\documentclass[12pt]{article}
\usepackage{pmmeta}
\pmcanonicalname{ProofOfHilbertTheorem90}
\pmcreated{2013-03-22 15:19:27}
\pmmodified{2013-03-22 15:19:27}
\pmowner{mathcam}{2727}
\pmmodifier{mathcam}{2727}
\pmtitle{proof of Hilbert Theorem 90}
\pmrecord{8}{37134}
\pmprivacy{1}
\pmauthor{mathcam}{2727}
\pmtype{Proof}
\pmcomment{trigger rebuild}
\pmclassification{msc}{11R32}
\pmclassification{msc}{11S25}
\pmclassification{msc}{11R34}

% this is the default PlanetMath preamble.  as your knowledge
% of TeX increases, you will probably want to edit this, but
% it should be fine as is for beginners.

% almost certainly you want these
\usepackage{amssymb}
\usepackage{amsmath}
\usepackage{amsfonts}

% used for TeXing text within eps files
%\usepackage{psfrag}
% need this for including graphics (\includegraphics)
%\usepackage{graphicx}
% for neatly defining theorems and propositions
%\usepackage{amsthm}
% making logically defined graphics
%%%\usepackage{xypic}

% there are many more packages, add them here as you need them

% define commands here
\begin{document}
Remember that two cocycles $a, a^\prime\colon G\to L^*$ are called cohomologous, denoted by $a\sim a^\prime$, if there exists $b\in L^*$, such that $a^\prime(\tau)=ba(\tau)\tau(b^{-1})$ for all $\tau\in G$. Then 
$$H^1(G,L^*)=\{a\colon G\to L^*|a\text{ is a cocycle}\}/\sim.$$
Now let $a\colon G\to L^*$ be a cocycle. Then consider the map 
$$\alpha\colon L\to L, c\mapsto\sum_{\tau\in G}a(\tau)\tau(c).$$
Since elements of the Galois group are linearly independent, $\alpha$ is not $0$. So we can choose $c\in L$, such that $b=\alpha(c)\neq 0$. Then for $\sigma\in G$ we have
\begin{align*}
\sigma(b)&=\sum_{\tau\in G}\sigma(a(\tau)\tau(c))\\
&=\sum_{\tau\in G}\sigma(a(\tau))(\sigma\tau)(c)\\
&=\sum_{\tau\in G}a(\sigma)^{-1}a(\sigma\tau)(\sigma\tau)(c),
\end{align*}
since $a$ is a cocycle, i.e. $a(\sigma\tau)=a(\sigma)\sigma(a(\tau))$. Then we get
\begin{align*}
\sigma(b)&=a(\sigma)^{-1}\sum_{\tau\in G}a(\sigma\tau)(\sigma\tau)(c)\\
&=a(\sigma)^{-1}b.
\end{align*}
Thus we have that $a(\sigma)=b\sigma(b)^{-1}$ is a 1-coboundary.

Now we prove the corollary. Denote the norm by $N$. Now if $x=\frac{y}{\sigma(y)}$, we have
$$N(x)=N\left(\frac{y}{\sigma(y)}\right)=\prod_{\tau\in G}\frac{\tau(y)}{\tau(\sigma(y))}=1.$$
Now let $N(x)=1$, $n=|G|$. Since $G$ is assumed cyclic, let $\sigma$ be a generator of $G$. $G$ is isomorphic to $\mathbb{Z}/n\mathbb{Z}$. We define the map $\tilde{x}\colon \mathbb{Z}/n\mathbb{Z}\to L^*$ by
$$\tilde{x}([i])=\prod_{0\leq j\leq i-1}\sigma^j(x),$$
where $[i]$ denotes the class of $i\in\mathbb{Z}$ in $\mathbb{Z}/n\mathbb{Z}$. Since $N(x)=1$, $\tilde{x}$ is well defined. We have
\begin{align*}
\tilde{x}([i+k])&=\prod_{0\leq j\leq i+k-1}\sigma^j(x)\\
&=\left(\prod_{0\leq j\leq i-1}\sigma^j(x)\right)\sigma^i\left(\prod_{0\leq j\leq k-1}\sigma^j(x)\right)\\
&=\tilde{x}([i])\sigma^i(\tilde{x}([j])).
\end{align*}
Therefore $\tilde{x}$ is a cocycle. Because of Hilberts Theorem 90, there exists $y\in L^*$, such that $x=\tilde{x}([1])=y\sigma(y)^{-1}$.
%%%%%
%%%%%
\end{document}
