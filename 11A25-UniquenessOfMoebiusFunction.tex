\documentclass[12pt]{article}
\usepackage{pmmeta}
\pmcanonicalname{UniquenessOfMoebiusFunction}
\pmcreated{2013-03-22 14:17:25}
\pmmodified{2013-03-22 14:17:25}
\pmowner{mathcam}{2727}
\pmmodifier{mathcam}{2727}
\pmtitle{uniqueness of Moebius function}
\pmrecord{6}{35745}
\pmprivacy{1}
\pmauthor{mathcam}{2727}
\pmtype{Definition}
\pmcomment{trigger rebuild}
\pmclassification{msc}{11A25}

\endmetadata

% this is the default PlanetMath preamble.  as your knowledge
% of TeX increases, you will probably want to edit this, but
% it should be fine as is for beginners.

% almost certainly you want these
\usepackage{amssymb}
\usepackage{amsmath}
\usepackage{amsfonts}
\usepackage{amsthm}

% used for TeXing text within eps files
%\usepackage{psfrag}
% need this for including graphics (\includegraphics)
%\usepackage{graphicx}
% for neatly defining theorems and propositions
%\usepackage{amsthm}
% making logically defined graphics
%%%\usepackage{xypic}

% there are many more packages, add them here as you need them

% define commands here

\newcommand{\mc}{\mathcal}
\newcommand{\mb}{\mathbb}
\newcommand{\mf}{\mathfrak}
\newcommand{\ol}{\overline}
\newcommand{\ra}{\rightarrow}
\newcommand{\la}{\leftarrow}
\newcommand{\La}{\Leftarrow}
\newcommand{\Ra}{\Rightarrow}
\newcommand{\nor}{\vartriangleleft}
\newcommand{\Gal}{\text{Gal}}
\newcommand{\GL}{\text{GL}}
\newcommand{\Z}{\mb{Z}}
\newcommand{\R}{\mb{R}}
\newcommand{\Q}{\mb{Q}}
\newcommand{\C}{\mb{C}}
\newcommand{\<}{\langle}
\renewcommand{\>}{\rangle}
\newcommand{\Nstar}{\mathbb{N}^{*}}
\begin{document}
\PMlinkescapeword{proposition} \PMlinkescapeword{clearly}
\PMlinkescapeword{fix} \PMlinkescapeword{connection}

Here is a sample result for the function, essentially classifying its uniqueness :

\noindent
\textbf{Proposition 1:} $\mu$ is the unique mapping $\Nstar\to\Z$ such that
\begin{eqnarray}
\mu(1)&=&1 \\
\sum_{d|n}\mu(d)&=&0\textrm{ for all }n>1
\end{eqnarray}

\noindent
\textbf{Proof:} By induction, there can only be one function with these
properties. $\mu$ clearly satisfies (1), so take some $n>1$. Let $p$ be
some prime factor of $n$, and let $m$ be the product of all the prime
factors of $n$.
\begin{eqnarray*}
\sum_{d|n}\mu(d)&=&\sum_{d|m}\mu(d) \\
&=&\sum_{\substack{d|m\\p\nmid d}}\mu(d) +\sum_{\substack{d|m\\p\mid d}}\mu(d) \\
&=&\sum_{d|m/p}\mu(d)-\sum_{d|m/p}\mu(d) \\
&=&0
\end{eqnarray*}
%%%%%
%%%%%
\end{document}
