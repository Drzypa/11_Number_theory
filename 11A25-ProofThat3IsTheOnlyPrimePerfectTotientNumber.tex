\documentclass[12pt]{article}
\usepackage{pmmeta}
\pmcanonicalname{ProofThat3IsTheOnlyPrimePerfectTotientNumber}
\pmcreated{2013-03-22 16:34:29}
\pmmodified{2013-03-22 16:34:29}
\pmowner{PrimeFan}{13766}
\pmmodifier{PrimeFan}{13766}
\pmtitle{proof that 3 is the only prime perfect totient number}
\pmrecord{5}{38765}
\pmprivacy{1}
\pmauthor{PrimeFan}{13766}
\pmtype{Proof}
\pmcomment{trigger rebuild}
\pmclassification{msc}{11A25}

% this is the default PlanetMath preamble.  as your knowledge
% of TeX increases, you will probably want to edit this, but
% it should be fine as is for beginners.

% almost certainly you want these
\usepackage{amssymb}
\usepackage{amsmath}
\usepackage{amsfonts}

% used for TeXing text within eps files
%\usepackage{psfrag}
% need this for including graphics (\includegraphics)
%\usepackage{graphicx}
% for neatly defining theorems and propositions
%\usepackage{amsthm}
% making logically defined graphics
%%%\usepackage{xypic}

% there are many more packages, add them here as you need them

% define commands here

\begin{document}
Given a prime number $p$, only $p = 3$ satisfies the equation $$p = \sum_{i = 1}^{c + 1} \phi^i(n),$$ where $\phi^i(x)$ is the iterated totient function and $c$ is the integer such that $\phi^c(n) = 2$. That is, 3 is the only perfect totient number that is prime.

The first four primes are most easily examined empirically. Since $\phi(2) = 1$, 2 is deficient totient number. $\phi(3) = 2$, so, per the previous remark, it is a perfect totient number. For 5, the iterates are 4, 2 and 1, adding up to 7, hence 5 is an abundant totient number. The same goes for 7, with its iterates being 7, 6, 2, 1.

It is for $p > 7$ that we can avail ourselves of the inequality $\phi(n) > \sqrt{n}$ (true for all $n > 6$). It is obvious that $\phi(p) = p - 1$, and by the foregoing, $\phi^2(p) > 3.162278$ (that is, it is sure to be more than the square root of 10), so it follows that $\phi(p) + \phi^2(p) > p + 2.162278$ and thus it is not necessary to examine any further iterates to see that all such primes are abundant totient numbers.
%%%%%
%%%%%
\end{document}
