\documentclass[12pt]{article}
\usepackage{pmmeta}
\pmcanonicalname{IntegralBasis}
\pmcreated{2013-03-22 12:36:03}
\pmmodified{2013-03-22 12:36:03}
\pmowner{rspuzio}{6075}
\pmmodifier{rspuzio}{6075}
\pmtitle{integral basis}
\pmrecord{12}{32853}
\pmprivacy{1}
\pmauthor{rspuzio}{6075}
\pmtype{Definition}
\pmcomment{trigger rebuild}
\pmclassification{msc}{11R04}
\pmsynonym{minimal basis}{IntegralBasis}
\pmsynonym{minimal bases}{IntegralBasis}
%\pmkeywords{integral basis}
\pmrelated{AlgebraicInteger}
\pmrelated{Integral}
\pmrelated{Basis}
\pmrelated{DiscriminantOfANumberField}
\pmrelated{ConditionForPowerBasis}
\pmrelated{BasisOfIdealInAlgebraicNumberField}
\pmrelated{CanonicalFormOfElementOfNumberField}
\pmdefines{integral bases}

\endmetadata

% this is the default PlanetMath preamble.  as your knowledge
% of TeX increases, you will probably want to edit this, but
% it should be fine as is for beginners.

% almost certainly you want these
\usepackage{amssymb}
\usepackage{amsmath}
\usepackage{amsfonts}

% used for TeXing text within eps files
%\usepackage{psfrag}
% need this for including graphics (\includegraphics)
%\usepackage{graphicx}
% for neatly defining theorems and propositions
%\usepackage{amsthm}
% making logically defined graphics
%%%\usepackage{xypic}

% there are many more packages, add them here as you need them

% define commands here
\begin{document}
Let $K$ be a number field.  A set of algebraic integers
$\{\alpha_1,\ldots,\alpha_s\}$ is said to be an {\em integral basis} for $K$
if every $\gamma$ in $\mathcal{O}_K$ can be represented uniquely 
as an integer linear combination of $\{\alpha_1,\ldots,\alpha_s\}$ (i.e. one can write $\gamma = m_1 \alpha_1 + \cdots + m_s \alpha_s$ with $m_1,\ldots,m_s$
(rational) integers).

If $I$ is an ideal of $\mathcal{O}_K$, then $\{\alpha_1,\ldots,\alpha_s\} \in I$ is said to be an {\em integral basis} for $I$ if every element of $I$ can be represented uniquely as an integer linear combination of $\{\alpha_1,\ldots,\alpha_s\}$.

(In the above, $\mathcal{O}_K$ denotes the ring of algebraic integers of $K$.)

An integral basis for $K$ over $\mathbb{Q}$ is a basis for $K$ over $\mathbb{Q}$.
%%%%%
%%%%%
\end{document}
