\documentclass[12pt]{article}
\usepackage{pmmeta}
\pmcanonicalname{NearsquarePrime}
\pmcreated{2013-03-22 18:57:37}
\pmmodified{2013-03-22 18:57:37}
\pmowner{PrimeFan}{13766}
\pmmodifier{PrimeFan}{13766}
\pmtitle{near-square prime}
\pmrecord{6}{41818}
\pmprivacy{1}
\pmauthor{PrimeFan}{13766}
\pmtype{Definition}
\pmcomment{trigger rebuild}
\pmclassification{msc}{11A41}

% this is the default PlanetMath preamble.  as your knowledge
% of TeX increases, you will probably want to edit this, but
% it should be fine as is for beginners.

% almost certainly you want these
\usepackage{amssymb}
\usepackage{amsmath}
\usepackage{amsfonts}

% used for TeXing text within eps files
%\usepackage{psfrag}
% need this for including graphics (\includegraphics)
%\usepackage{graphicx}
% for neatly defining theorems and propositions
%\usepackage{amsthm}
% making logically defined graphics
%%%\usepackage{xypic}

% there are many more packages, add them here as you need them

% define commands here

\begin{document}
A {\em near-square prime} is a prime number $p$ of the form $n^2 + k$, with $n$ being any integer and $0 < |k| < |n|$ also an integer. Since for any nonzero real number $x$ it is always the case that $x^2 \geq 0$, it doesn't matter if $n$ is negative.

\begin{tabular}{|r|r|r|r|r|r|r|r|r|r|r|r|r|}
   5  &    &    &     &     &    &    &    &    &    &      &     & 149 \\
   4  &    &    &     &     & 29 &    & 53 &    &    &      &     &     \\
   3  &    &    &     &     &    &    &    & 67 &    &  103 &     &     \\
   2  &    &    &  11 &     &    &    &    &    & 83 &      &     &     \\
   1  &    &  5 &     &  17 &    & 37 &    &    &    &  101 &     &     \\
   0  &  1 &  4 &   9 &  16 & 25 & 36 & 49 & 64 & 81 &  100 & 121 & 144 \\
 $-1$ &    &  3 &     &     &    &    &    &    &    &      &     &     \\
 $-2$ &    &    &   7 &     & 23 &    & 47 &    & 79 &      &     &     \\
 $-3$ &    &    &     &     &    &    &    &    &    &   97 &     &     \\
 $-4$ &    &    &     &     &    &    &    &    &    &      &     &     \\
 $-5$ &    &    &     &     &    & 31 &    & 59 &    &      &     & 139 \\
\end{tabular}

Fermat primes are near-square primes for $k = 1$ with the additional requirement that $n = 2^{2^m - 1}$, while Carol primes are near-square primes for $k = -2$ with the additional requirement that $n = 2^m - 1$.

For $k = -1$, only $n = 2$ gives a prime, namely 3.
%%%%%
%%%%%
\end{document}
