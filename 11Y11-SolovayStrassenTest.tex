\documentclass[12pt]{article}
\usepackage{pmmeta}
\pmcanonicalname{SolovayStrassenTest}
\pmcreated{2013-03-22 14:27:33}
\pmmodified{2013-03-22 14:27:33}
\pmowner{mathwizard}{128}
\pmmodifier{mathwizard}{128}
\pmtitle{Solovay-Strassen test}
\pmrecord{7}{35978}
\pmprivacy{1}
\pmauthor{mathwizard}{128}
\pmtype{Algorithm}
\pmcomment{trigger rebuild}
\pmclassification{msc}{11Y11}
\pmrelated{MillerRabinPrimeTest}

\endmetadata

% this is the default PlanetMath preamble.  as your knowledge
% of TeX increases, you will probably want to edit this, but
% it should be fine as is for beginners.

% almost certainly you want these
\usepackage{amssymb}
\usepackage{amsmath}
\usepackage{amsfonts}

% used for TeXing text within eps files
%\usepackage{psfrag}
% need this for including graphics (\includegraphics)
%\usepackage{graphicx}
% for neatly defining theorems and propositions
\usepackage{amsthm}
% making logically defined graphics
%%%\usepackage{xypic}

% there are many more packages, add them here as you need them

% define commands here
\newcommand{\jacobi}[2]{{\genfrac{(}{)}{}{}{#1}{#2}}}
\newcommand{\Z}{\mathbb{Z}}

\newtheorem{theorem}{Theorem}
\newtheorem{proposition}[theorem]{Proposition}
\newtheorem{lemma}[theorem]{Lemma}
\newtheorem{corollary}[theorem]{Corollary}
\newtheorem{conjecture}[theorem]{Conjecture}
\theoremstyle{definition}
\newtheorem{definition}[theorem]{Definition}
\newtheorem{protocol}[theorem]{Protocol}
\newtheorem{example}[theorem]{Example}
\newcommand{\ndiv}{\nmid}
\renewcommand{\div}{\mid}


\begin{document}
It is known that an odd number $n$ is prime if and only if for every $1<a<n$ such that $(a,n)=1$ we have
\begin{equation}\label{eq:criterion}
\jacobi{a}{n}\equiv a^{\frac{n-1}{2}}\pmod n
\end{equation}
where $\jacobi{a}{n}$ is the Jacobi symbol. (The only if part is obvious; the if part follows from Theorem~\ref{ss-half}.) From this we can derive the following algorithm.
\begin{enumerate}
\item Choose a random number $a$ between $1$ and $n-1$.
\item Check if $(a,n)=1$ (for example using the Euclidean algorithm). If it is not, then $n$ is not prime and $(a,n)$ is a divisor of $n$.
\item Check if Equation~\eqref{eq:criterion} holds. If it does not, then $n$ is not prime. Otherwise $n$ is a candidate for primality.
\end{enumerate}
By repeating this algorithm we can increase the chance that the result is correct. In order to estimate the probability of error, we make use of Theorem~\ref{ss-half}, which says that every independent iteration of the algorithm has a chance of at most $50\%$ of being wrong. Hence, after $t$ iterations there is at most a $2^{-t}$ chance of getting a wrong result.

\begin{theorem}\label{ss-half}
Let $n \geq 3$ be an odd composite integer. Then at least half of the
elements of $\Z_n^*$ do not satisfy Equation~\eqref{eq:criterion}.
\end{theorem}
\begin{proof}
It suffices to exhibit one element $a \in \Z_n^*$ which does not satisfy Equation~\eqref{eq:criterion}. Indeed, if there exists one such element, then the set of
all elements which do satisfy Equation~\eqref{eq:criterion} forms a proper subgroup of $\Z_n^*$, from which we
conclude that the elements satisfying Equation~\eqref{eq:criterion} number no more than half of the elements
of $\Z_n^*$.

We consider separately the cases where $n$ is squarefree and not
squarefree.  If $n$ is squarefree, let $p$ be a prime dividing $n$ and
let $b$ be a quadratic non-residue mod $n$. Using the Chinese
Remainder Theorem, choose an integer $a \in \Z_n^*$ such that:
\begin{align*}
a &\equiv b \pmod{p}, \\
a &\equiv 1 \pmod{\frac{n}{p}}.
\end{align*}
The Jacobi symbol $\jacobi{a}{n}$ is given by
\[
\jacobi{a}{n} = \jacobi{a}{p} \jacobi{a}{n/p} = \jacobi{b}{p}
\jacobi{1}{n/p} = (-1) \cdot 1 = -1.
\]
We will assume that $a^{\frac{n-1}{2}} \equiv -1 \pmod{n}$ and derive
a contradiction. The equation $a^{\frac{n-1}{2}} \equiv -1 \pmod{n}$
implies that $a^{\frac{n-1}{2}} \equiv -1 \pmod{\frac{n}{p}}$.
However, since $a \equiv 1 \pmod{\frac{n}{p}}$, we must have
$a^{\frac{n-1}{2}} \equiv 1 \pmod{\frac{n}{p}}$, so that $1 \equiv -1
\pmod{\frac{n}{p}}$, which is a contradiction.

Now suppose that $n$ is not squarefree. Let $p$ be a prime such that
$p^2 \div n$, and set $a = 1 + \frac{n}{p}$. By the binomial theorem,
we have
\[
a^p = \left(1+\frac{n}{p}\right)^p = 1 +
\binom{p}{1}\left(n/p\right) +
\binom{p}{2}\left(n/p\right)^2 + \cdots +
\binom{p}{p}\left(n/p\right)^p
\equiv 1 \pmod{n},
\]
so the multiplicative order of $a \bmod n$ is equal to $p$, and hence
in particular $a^{n-1} \neq 1 \pmod{n}$, since $p \ndiv n-1$. On the
other hand,
\[
\jacobi{a}{n} = \jacobi{1+\frac{n}{p}}{n} = \jacobi{1+\frac{n}{p}}{p}
\jacobi{1+\frac{n}{p}}{n/p} = \jacobi{1}{p} \jacobi{1}{n/p} = 1,
\]
so $a$ does not satisfy Equation~\eqref{eq:criterion}.
\end{proof}

%%%%%
%%%%%
\end{document}
