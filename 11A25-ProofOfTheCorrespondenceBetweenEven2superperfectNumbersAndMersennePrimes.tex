\documentclass[12pt]{article}
\usepackage{pmmeta}
\pmcanonicalname{ProofOfTheCorrespondenceBetweenEven2superperfectNumbersAndMersennePrimes}
\pmcreated{2013-03-22 17:03:48}
\pmmodified{2013-03-22 17:03:48}
\pmowner{PrimeFan}{13766}
\pmmodifier{PrimeFan}{13766}
\pmtitle{proof of the correspondence between even 2-superperfect numbers and Mersenne primes}
\pmrecord{5}{39355}
\pmprivacy{1}
\pmauthor{PrimeFan}{13766}
\pmtype{Proof}
\pmcomment{trigger rebuild}
\pmclassification{msc}{11A25}

% this is the default PlanetMath preamble.  as your knowledge
% of TeX increases, you will probably want to edit this, but
% it should be fine as is for beginners.

% almost certainly you want these
\usepackage{amssymb}
\usepackage{amsmath}
\usepackage{amsfonts}

% used for TeXing text within eps files
%\usepackage{psfrag}
% need this for including graphics (\includegraphics)
%\usepackage{graphicx}
% for neatly defining theorems and propositions
%\usepackage{amsthm}
% making logically defined graphics
%%%\usepackage{xypic}

% there are many more packages, add them here as you need them

% define commands here

\begin{document}
Statement. Among the even numbers, only powers of two $2^x$ (with $x$ being a nonnegative integer) can be 2-\PMlinkname{superperfect numbers}{SuperperfectNumber}, and then if and only if $2^{x + 1} - 1$ is a Mersenne prime. (The default multiplier $m = 2$ is tacitly assumed from this point forward).

Proof. The only divisors of $n = 2^x$ are smaller powers of 2 and itself, $1, 2, \ldots , 2^{x - 1}, 2^x$. Therefore, the first iteration of the sum of divisors function is $$\sigma(n) = \sum_{i = 0}^x 2^i = 2^{x + 1} - 1 = 2n - 1.$$ If $2n - 1$ is prime, that means its only other divisor is 1, and thus for the second iteration $\sigma(2n - 1) = 2n$, and is thus a 2-superperfect number. But if $2n - 1$ is composite then it is clear that $\sigma(2n - 1) > 2n$ by at least 2. So, for example, $\sigma(8) = 15$ and $\sigma^2(8) = 24$, so 8 is not 2-superperfect. One more example: $\sigma(16) = 31$ and since 31 is prime, $\sigma^2(16) = 32$.

Now it only remains to prove that no other even number $n$ can be 2-superperfect. Any other even number can of course still be divisible by one or more powers of two, but it also must be divisible by some odd prime $p > 2$. Since the sum of divisors function is a multiplicative function, it follows that if $n = 2^xp$ then $\sigma(n) = \sigma(2^x)\sigma(p)$. So, if, say, $p = 3$, it is clear that $(2^{x + 3} - 4) > 2^{x + 1}3$, and that on the second iteration this value that already exceeded twice the original value will be even greater. For example, $12 = 2^2 3$, and $\sigma(12) = 2^5 - 4$ which is greater than $2^3 3$ by 4. With any larger $p$ the excess will be much greater.
%%%%%
%%%%%
\end{document}
