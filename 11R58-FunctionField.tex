\documentclass[12pt]{article}
\usepackage{pmmeta}
\pmcanonicalname{FunctionField}
\pmcreated{2013-03-22 15:34:35}
\pmmodified{2013-03-22 15:34:35}
\pmowner{alozano}{2414}
\pmmodifier{alozano}{2414}
\pmtitle{function field}
\pmrecord{8}{37484}
\pmprivacy{1}
\pmauthor{alozano}{2414}
\pmtype{Definition}
\pmcomment{trigger rebuild}
\pmclassification{msc}{11R58}
\pmsynonym{algebraic function field}{FunctionField}
\pmrelated{SimpleTranscendentalFieldExtension}
\pmdefines{rational function field}
\pmdefines{geometric extension}
\pmdefines{genus of a function field}
\pmdefines{degree of a prime}

\endmetadata

% this is the default PlanetMath preamble.  as your knowledge
% of TeX increases, you will probably want to edit this, but
% it should be fine as is for beginners.

% almost certainly you want these
\usepackage{amssymb}
\usepackage{amsmath}
\usepackage{amsthm}
\usepackage{amsfonts}

% used for TeXing text within eps files
%\usepackage{psfrag}
% need this for including graphics (\includegraphics)
%\usepackage{graphicx}
% for neatly defining theorems and propositions
%\usepackage{amsthm}
% making logically defined graphics
%%%\usepackage{xypic}

% there are many more packages, add them here as you need them

% define commands here

\newtheorem{thm}{Theorem}
\newtheorem{defn}{Definition}
\newtheorem{prop}{Proposition}
\newtheorem{lemma}{Lemma}
\newtheorem{cor}{Corollary}

\theoremstyle{definition}
\newtheorem{exa}{Example}

% Some sets
\newcommand{\Nats}{\mathbb{N}}
\newcommand{\Ints}{\mathbb{Z}}
\newcommand{\Reals}{\mathbb{R}}
\newcommand{\Complex}{\mathbb{C}}
\newcommand{\Rats}{\mathbb{Q}}
\newcommand{\Gal}{\operatorname{Gal}}
\newcommand{\Cl}{\operatorname{Cl}}
\begin{document}
Let $F$ be a field.

\begin{defn}
The rational function field over $F$ in one variable ($x$), denoted by $F(x)$, is the field of all rational functions $p(x)/q(x)$ with polynomials $p,q\in F[x]$ and $q(x)$ not identically zero.
\end{defn}

\begin{defn}
A function field (in one variable) over $F$ is a field $K$, containing $F$ and at least one element $x$, transcendental over $F$, such that $K/F(x)$ is a \PMlinkid{finite}{FiniteExtension} algebraic extension.
\end{defn}

Let $\overline{F}$ be a fixed algebraic closure of $F$.

\begin{defn}
Let $K$ be a function field over $F$ and let $L$ be a finite extension of $K$. The extension $L/K$ of function fields is said to be geometric if $L\cap \overline{F}=F$.
\end{defn}

\begin{exa}
The extension $\Rats(\sqrt{x})/\Rats(x)$ is geometric, but $\Rats(\sqrt{2})(x)/\Rats(x)$ is not geometric.
\end{exa}

\begin{thm}[Thm. I.6.9 of \cite{hart}] Let $K$ be a function field over an algebraically closed field $F$. There exists a nonsingular projective curve $C_K$ such that the function field of $C_K$ is isomorphic to $K$.
\end{thm}

\begin{defn}
Let $K$ be a function field over a field $F$. Let $K'=K\overline{F}$ which is a function field over $\overline{F}$, a fixed algebraic closure of $F$, and let $C_{K'}$ be the curve given by the previous theorem. The genus of $K$ is, by definition, the genus of $C_{K'}$.
\end{defn}

\begin{defn}
Let $K$ be a function field over a field $F$. A prime in $K$ is by definition a discrete valuation ring $R$ with maximal $P$ such that $F\subset R$ and the quotient field of $R$ is equal to $K$. The prime is usually denoted $P$ after the maximal ideal of $R$. The degree of $P$, denoted by $\deg P$, is defined to be the dimension of $R/P$ over $F$.
\end{defn}

\begin{exa}
Let $K=F(x)$ be the rational function field over $F$ and let $\mathcal{O}=F[x]$. The prime ideals of $\mathcal{O}$ are generated by monic irreducible polynomials in $F[x]$. Let $P=(f(x))$ be such a prime. Then $R_P=\mathcal{O}_P$, the localization of $\mathcal{O}$ at the prime $P$ is a discrete valuation ring with $F\subset \mathcal{O}_P$ and the quotient field of $R_P$ is $K$. Thus $R_P=\mathcal{O}_P$ is a prime of $K$.

One can define an `extra' prime in the following way. Let $R_\infty=\mathcal{O}_\infty=F[\frac{1}{x}]$ and let $P_\infty=(\frac{1}{x})$ be the prime ideal of $R_\infty$ generated by $\frac{1}{x}$. The localization ring $(R_\infty)_{P_\infty}$ is a prime of $K$, called the prime at infinity.
\end{exa}

\begin{thebibliography}{00}
\bibitem{hart} R. Hartshorne, {\em Algebraic Geometry}, Springer-Verlag, New York.
\bibitem{rosen} M. Rosen, {\em Number Theory in Function Fields}, Springer-Verlag, New York.
\end{thebibliography}
%%%%%
%%%%%
\end{document}
