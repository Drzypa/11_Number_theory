\documentclass[12pt]{article}
\usepackage{pmmeta}
\pmcanonicalname{GaussianPrime}
\pmcreated{2013-03-22 16:54:13}
\pmmodified{2013-03-22 16:54:13}
\pmowner{PrimeFan}{13766}
\pmmodifier{PrimeFan}{13766}
\pmtitle{Gaussian prime}
\pmrecord{7}{39161}
\pmprivacy{1}
\pmauthor{PrimeFan}{13766}
\pmtype{Definition}
\pmcomment{trigger rebuild}
\pmclassification{msc}{11R04}
\pmclassification{msc}{11A41}

% this is the default PlanetMath preamble.  as your knowledge
% of TeX increases, you will probably want to edit this, but
% it should be fine as is for beginners.

% almost certainly you want these
\usepackage{amssymb}
\usepackage{amsmath}
\usepackage{amsfonts}

% used for TeXing text within eps files
%\usepackage{psfrag}
% need this for including graphics (\includegraphics)
%\usepackage{graphicx}
% for neatly defining theorems and propositions
%\usepackage{amsthm}
% making logically defined graphics
%%%\usepackage{xypic}

% there are many more packages, add them here as you need them

% define commands here

\begin{document}
A {\em Gaussian prime} $p$ is a Gaussian integer $a + bi$ (where $i$ is the imaginary unit and $a$ and $b$ are real integers) that is divisible only by the units 1, $-1$, $i$ and $-i$, itself, its associates and no others. For example, $3 + 20i$ is a Gaussian prime because there is no pair of Gaussian integers (besides the units and associates) that multiply to $3 + 20i$. But $3 + 21i$ is not a Gaussian prime because $3(-i)(1 + i)(1 + 2i)^2 = 3 + 21i$. If $a + bi$ is prime then so are $a - bi$, $-a + bi$ and $-a - bi$, as well as the associates $b + ai$, $b - ai$, $b - ai$ and $-b - ai$.

The real and the imaginary parts must be of different parity. For a real prime to be a Gaussian prime of the form $p + 0i$, the real part has to be of the form $p = 4n - 1$; the same goes for the associates $0 + pi$. It follows from \PMlinkname{Fermat's theorem on sums of two squares}{RepresentingPrimesAsX2ny2} that since real primes of the form $p = 4n + 1$ can be represented as $x^2 + y^2$, then in the complex plane they have the factorization $(x + yi)(x - yi)$. For example, $17 = 4^2 + 1^2$, so $(4 + i)(4 - i) = 17$.

Sometimes Gaussian primes are simply called ``complex primes,'' which is an incorrect term found in some of the older literature.

\begin{thebibliography}{1}
\bibitem{ek} Kogbetliantz, Ervand George {\it Handbook of first complex prime numbers} London: Gordon and Breach Science Publishers (1971)
\end{thebibliography}
%%%%%
%%%%%
\end{document}
