\documentclass[12pt]{article}
\usepackage{pmmeta}
\pmcanonicalname{MinkowskisConstant}
\pmcreated{2013-03-22 15:05:33}
\pmmodified{2013-03-22 15:05:33}
\pmowner{alozano}{2414}
\pmmodifier{alozano}{2414}
\pmtitle{Minkowski's constant}
\pmrecord{4}{36819}
\pmprivacy{1}
\pmauthor{alozano}{2414}
\pmtype{Corollary}
\pmcomment{trigger rebuild}
\pmclassification{msc}{11H06}
\pmclassification{msc}{11R29}
%\pmkeywords{ideal class group}
%\pmkeywords{discriminant}
\pmrelated{IdealClass}
\pmrelated{StirlingsApproximation}
\pmrelated{DiscriminantOfANumberField}
\pmrelated{ClassNumbersAndDiscriminantsTopicsOnClassGroups}
\pmrelated{ProofOfMinkowskisBound}
\pmdefines{Minkowski's theorem on ideal classes}

\endmetadata

% this is the default PlanetMath preamble.  as your knowledge
% of TeX increases, you will probably want to edit this, but
% it should be fine as is for beginners.

% almost certainly you want these
\usepackage{amssymb}
\usepackage{amsmath}
\usepackage{amsthm}
\usepackage{amsfonts}

% used for TeXing text within eps files
%\usepackage{psfrag}
% need this for including graphics (\includegraphics)
%\usepackage{graphicx}
% for neatly defining theorems and propositions
%\usepackage{amsthm}
% making logically defined graphics
%%%\usepackage{xypic}

% there are many more packages, add them here as you need them

% define commands here

\newtheorem{thm}{Theorem}
\newtheorem{defn}{Definition}
\newtheorem{prop}{Proposition}
\newtheorem{lemma}{Lemma}
\newtheorem{cor}{Corollary}

% Some sets
\newcommand{\Nats}{\mathbb{N}}
\newcommand{\Ints}{\mathbb{Z}}
\newcommand{\Reals}{\mathbb{R}}
\newcommand{\Complex}{\mathbb{C}}
\newcommand{\Rats}{\mathbb{Q}}
\newcommand{\Cl}{\operatorname{Cl}}
\begin{document}
The following is a corollary to the famous Minkowski's theorem on lattices and convex regions. It was also found by Minkowski and sometimes also called Minkowski's theorem.

\begin{thm}[Minkowski's Theorem]
\label{thm1}
Let $K$ be a number field and let $D_K$ be its discriminant. Let $n=r_1+2r_2$ be the degree of $K$ over $\Rats$, where $r_1$ and $r_2$ are the number of real and complex embeddings, respectively. The class group of $K$ is denoted by $\Cl(K)$. In any ideal class $C\in \Cl(K)$, there exists an ideal $\mathfrak{A}\in C$ such that:
$$|{\bf N}(\mathfrak{A})| \leq M_K \sqrt{|D_K|}$$
where ${\bf N}(\mathfrak{A})$ denotes the absolute norm of $\mathfrak{A}$ and 
$$M_K=\frac{n!}{n^n} \left(\frac{4}{\pi}\right)^{r_2}.$$
\end{thm}

\begin{defn}
The constant $M_K$, as in the theorem, is usually called the Minkowski's constant.
\end{defn}

In the applications, one uses Stirling's formula to find approximations of Minkowski's constant. The following is an immediate corollary of Theorem \ref{thm1}.

\begin{cor}
Let $K$ be an arbitrary number field. Then the absolute value of the discriminant of $K$, $D_K$, is greater than $1$, i.e. $|D_K|>1$. In particular, there is at least one rational prime $p\in \Ints$ which ramifies in $K$. 
\end{cor}

See the entry \PMlinkname{on discriminants}{DiscriminantOfANumberField} for the relationship between $D_K$ and the ramification of primes.
%%%%%
%%%%%
\end{document}
