\documentclass[12pt]{article}
\usepackage{pmmeta}
\pmcanonicalname{CollatzProblem}
\pmcreated{2013-03-22 11:42:43}
\pmmodified{2013-03-22 11:42:43}
\pmowner{akrowne}{2}
\pmmodifier{akrowne}{2}
\pmtitle{Collatz problem}
\pmrecord{32}{30029}
\pmprivacy{1}
\pmauthor{akrowne}{2}
\pmtype{Conjecture}
\pmcomment{trigger rebuild}
\pmclassification{msc}{11B37}
\pmsynonym{Ulam's Problem}{CollatzProblem}
\pmsynonym{1-4-2 Problem}{CollatzProblem}
\pmsynonym{Syracuse problem}{CollatzProblem}
\pmsynonym{Thwaites conjecture}{CollatzProblem}
\pmsynonym{Kakutani's problem}{CollatzProblem}
\pmsynonym{3n+1 problem}{CollatzProblem}
%\pmkeywords{Collatz}
%\pmkeywords{Ulam}

\endmetadata

\usepackage{amssymb}
\usepackage{amsmath}
\usepackage{amsfonts}
\usepackage{graphicx}
%%%%%%%%%%%%%%%\usepackage{xypic}
\begin{document}
We define the function $f : \mathbb{N} \longrightarrow \mathbb{N} $ (where $\mathbb{N}$ excludes zero) such that

$$ f(a) =  \left\{
\begin{array}{rl}
3a+1  & \text{ if } a \text{ is odd }   \\
 a/2  & \text{ if } a \text{ is even.}  
\end{array}
\right. $$

Then let the sequence $c_n$ be defined as $c_i = f(c_{i-1})$, with $c_0$ an arbitrary natural seed value.

It is conjectured that the sequence $c_0, c_1, c_2, \ldots$ will always end in  $1,4,2$, repeating infinitely.  This has been verified by computer up to very large values of $c_0$, but is unproven in general. It is also not known whether this problem is decideable.  This is generally called the \emph{Collatz problem}.

The sequence $c_n$ is sometimes called the ``hailstone sequence''.  This is because it behaves analogously to a hailstone in a cloud which falls by gravity and is tossed up again repeatedly.  The sequence similarly ends in an eternal oscillation.
%%%%%
%%%%%
%%%%%
%%%%%
%%%%%
%%%%%
%%%%%
%%%%%
%%%%%
%%%%%
%%%%%
%%%%%
%%%%%
%%%%%
%%%%%
\end{document}
