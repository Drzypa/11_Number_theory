\documentclass[12pt]{article}
\usepackage{pmmeta}
\pmcanonicalname{QuotientRuleForArithmeticDerivative}
\pmcreated{2013-03-22 17:04:44}
\pmmodified{2013-03-22 17:04:44}
\pmowner{Wkbj79}{1863}
\pmmodifier{Wkbj79}{1863}
\pmtitle{quotient rule for arithmetic derivative}
\pmrecord{4}{39372}
\pmprivacy{1}
\pmauthor{Wkbj79}{1863}
\pmtype{Theorem}
\pmcomment{trigger rebuild}
\pmclassification{msc}{11Z05}

\endmetadata

\usepackage{amssymb}
\usepackage{amsmath}
\usepackage{amsfonts}

\usepackage{psfrag}
\usepackage{graphicx}
\usepackage{amsthm}
%%\usepackage{xypic}

\newtheorem*{thm}{Theorem}
\begin{document}
\begin{thm}
If the notion of arithmetic derivative is extended to rational numbers, then we have that, for every $a,b \in \mathbb{Z}$ with $b \neq 0$:

$$\left(\frac{a}{b}\right)'=\frac{a'b-b'a}{b^2}$$
\end{thm}

\begin{proof}

Note that

\begin{center}
\begin{tabular}{rl}
$a'$ & $= \displaystyle \left( b \cdot \frac{a}{b} \right)'$ \\
& \\
& $= \displaystyle b \cdot \left( \frac{a}{b} \right)' +b' \cdot \frac{a}{b}$ by the Leibniz rule.
\end{tabular}
\end{center}

Thus,

\begin{center}
$\begin{array}{rl}
\displaystyle b \cdot \left( \frac{a}{b} \right)' & = \displaystyle a'-b' \cdot \frac{a}{b} \\
& \\
& = \displaystyle \frac{a'b-b'a}{b}. \end{array}$
\end{center}

It follows that

$$\left(\frac{a}{b}\right)'=\frac{a'b-b'a}{b^2}.$$
\end{proof}
%%%%%
%%%%%
\end{document}
