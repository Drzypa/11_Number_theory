\documentclass[12pt]{article}
\usepackage{pmmeta}
\pmcanonicalname{SzemeredisTheorem}
\pmcreated{2013-03-22 13:19:40}
\pmmodified{2013-03-22 13:19:40}
\pmowner{bbukh}{348}
\pmmodifier{bbukh}{348}
\pmtitle{Szemer\'edi's theorem}
\pmrecord{17}{33839}
\pmprivacy{1}
\pmauthor{bbukh}{348}
\pmtype{Theorem}
\pmcomment{trigger rebuild}
\pmclassification{msc}{11B25}
\pmclassification{msc}{05D10}

\endmetadata

\usepackage{amssymb}
\usepackage{amsmath}
\usepackage{amsfonts}

\makeatletter
%\@ifundefined{url}{\usepackage{url}}{}
\@ifundefined{bibname}{}{\renewcommand{\bibname}{References}}
\makeatother
\begin{document}
Let $k$ be a positive integer and let $\delta>0$. There exists a positive integer $N=N(k,\delta)$ such that every subset of $\{1,2,\dotsc,N\}$ of \PMlinkname{size}{Cardinality} $\delta N$ contains an arithmetic progression of length $k$.

The case $k=3$ was first proved by Roth\cite{cite:roth_szem_three}. His method did not seem to extend to the case $k>3$. Using completely different ideas Szemer\'edi proved the case $k=4$ \cite{cite:szemeredi_szemth_four}, and the general case of an arbitrary $k$ \cite{cite:szemeredi_szemth_gen}.

The best known bounds for $N(k,\delta)$ are 
\begin{equation*}
e^{c (\log \frac{1}{\delta})^{k-1}}\leq N(k,\delta)\leq 2^{2^{\delta^{-2^{2^{k+9}}}}},
\end{equation*}
where the lower bound is due to Behrend\cite{cite:behrend_szem_low} (for $k=3$) and Rankin\cite{cite:rankin_behrgen}, and the upper bound is due to Gowers\cite{cite:gowers_szem_new_gen}.

For $k=3$ a better upper bound was obtained by Bourgain
\begin{equation*}
N(3,\delta)\leq c \delta^{-2} e^{2^{56} \delta^{-2}}.
\end{equation*}

\begin{thebibliography}{1}

\bibitem{cite:behrend_szem_low}
Felix~A. Behrend.
\newblock On the sets of integers which contain no three in arithmetic
  progression.
\newblock {\em Proc. Nat. Acad. Sci.}, 23:331--332, 1946.
\newblock \PMlinkexternal{Zbl 0060.10302}{http://www.emis.de/cgi-bin/zmen/ZMATH/en/quick.html?type=html&an=0060.10302}.

\bibitem{cite:gowers_szem_new_gen}
Timothy Gowers.
\newblock A new proof of {S}zemer{\'e}di's theorem.
\newblock {\em Geom. Funct. Anal.}, 11(3):465--588, 2001.
\newblock Preprint available at
  \PMlinkexternal{http://www.dpmms.cam.ac.uk/~wtg10/papers.html}{http://www.dpmms.cam.ac.uk/~wtg10/papers.html}.

\bibitem{cite:rankin_behrgen}
Robert~A. Rankin.
\newblock Sets of integers containing not more than a given number of terms in
  arithmetical progression.
\newblock {\em Proc. Roy. Soc. Edinburgh Sect. A}, 65:332--344, 1962.
\newblock \PMlinkexternal{Zbl 0104.03705}{http://www.emis.de/cgi-bin/zmen/ZMATH/en/quick.html?type=html&an=0104.03705}.

\bibitem{cite:roth_szem_three}
Klaus~Friedrich Roth.
\newblock On certain sets of integers.
\newblock {\em J. London Math. Soc.}, 28:245--252, 1953.
\newblock \PMlinkexternal{Zbl 0050.04002}{http://www.emis.de/cgi-bin/zmen/ZMATH/en/quick.html?type=html&an=0050.04002}.

\bibitem{cite:szemeredi_szemth_four}
Endre Szemer{\'e}di.
\newblock On sets of integers containing no four elements in arithmetic
  progression.
\newblock {\em Acta Math. Acad. Sci. Hung.}, 20:89--104, 1969.
\newblock \PMlinkexternal{Zbl 0175.04301}{http://www.emis.de/cgi-bin/zmen/ZMATH/en/quick.html?type=html&an=0175.04301}.


\bibitem{cite:szemeredi_szemth_gen}
Endre Szemer{\'e}di.
\newblock On sets of integers containing no $k$ elements in arithmetic
  progression.
\newblock {\em Acta. Arith.}, 27:299--345, 1975.
\newblock \PMlinkexternal{Zbl 0303.10056}{http://www.emis.de/cgi-bin/zmen/ZMATH/en/quick.html?type=html&an=0303.10056}.

\end{thebibliography}
%
%
%@ARTICLE{cite:gowers_szem_new_gen,
% author    = {Timothy Gowers},
% title     = {A new proof of {S}zemer{\'e}di's theorem},
% journal   = {Geom. Funct. Anal.},
% volume    = 11,
% number    = 3,
% year      = 2001,
% pages     = {465--588},
% note      = {Preprint available at %\href{http://www.dpmms.cam.ac.uk/~wtg10/papers.html}{http://www.dpmms.cam.ac.uk/~wtg10/papers.html}}
%}
%
%@ARTICLE{cite:behrend_szem_low,
% author    = {Felix A. Behrend},
% title     = {On the sets of integers which contain no three in arithmetic %progression},
% journal   = {Proc. Nat. Acad. Sci.},
% volume    = 23,
% year      = 1946,
% pages     = {331--332}
%}
%
%@ARTICLE{cite:roth_szem_three,
% author    = {Klaus Friedrich Roth},
% title     = {On certain sets of integers},
% journal   = {J. London Math. Soc.},
% volume    = 28,
% year      = 1953,
% pages      = {245--252}
%}
%
%@ARTICLE{cite:szemeredi_szemth_four,
% author    = {Endre Szemer{\'e}di},
% title     = {On sets of integers containing no four elements in arithmetic %progression},
% journal   = {Acta Math. Acad. Sci. Hung.},
% volume    = 20,
% year      = 1969,
% pages     = {89--104}
%}
%
%@ARTICLE{cite:szemeredi_szemth_gen,
% author    = {Endre Szemer{\'e}di},
% title     = {On sets of integers containing no $k$ elements in arithmetic %progression},
% journal   = {Acta. Arith.},
% volume    = 27,
% year      = 1975,
% pages     = {299--345}
%}
%
%@ARTICLE{cite:rankin_behrgen,
% author    = {Rankin, Robert A.},
% title     = {Sets of integers containing not more than a given number of terms %in arithmetical progression},
% journal   = {Proc. Roy. Soc. Edinburgh Sect. A},
% volume    = 65,
% year      = 1962,
% pages     = {332--344}
%}
%%%%%
%%%%%
\end{document}
