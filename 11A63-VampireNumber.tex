\documentclass[12pt]{article}
\usepackage{pmmeta}
\pmcanonicalname{VampireNumber}
\pmcreated{2013-03-22 15:45:10}
\pmmodified{2013-03-22 15:45:10}
\pmowner{CompositeFan}{12809}
\pmmodifier{CompositeFan}{12809}
\pmtitle{vampire number}
\pmrecord{8}{37706}
\pmprivacy{1}
\pmauthor{CompositeFan}{12809}
\pmtype{Definition}
\pmcomment{trigger rebuild}
\pmclassification{msc}{11A63}
\pmdefines{vampire number}
\pmdefines{fang}

\endmetadata

% this is the default PlanetMath preamble.  as your knowledge
% of TeX increases, you will probably want to edit this, but
% it should be fine as is for beginners.

% almost certainly you want these
\usepackage{amssymb}
\usepackage{amsmath}
\usepackage{amsfonts}

% used for TeXing text within eps files
%\usepackage{psfrag}
% need this for including graphics (\includegraphics)
%\usepackage{graphicx}
% for neatly defining theorems and propositions
%\usepackage{amsthm}
% making logically defined graphics
%%%\usepackage{xypic}

% there are many more packages, add them here as you need them

% define commands here
\begin{document}
Consider the integer 1395. In the equation $$1395 = 15 \cdot 93,$$ expressed in base 10, both \PMlinkname{sides}{Equation} use the same digits.

When a number with an even number of digits is also the product of two multiplicands having half as many digits as the product, and together having the same digits, the product is called a {\em vampire number}. The multiplicands are called {\em fangs}.

By definition, a vampire number can't be a prime number. But if both of its fangs are prime numbers, then it might be referred to as a ``prime vampire number.''

This concept can be applied to any positional base, and \PMlinkescapetext{even} to Roman numerals. For example, $$VIII = II \cdot IV.$$

A vampire number is automatically a Friedman number also.
%%%%%
%%%%%
\end{document}
