\documentclass[12pt]{article}
\usepackage{pmmeta}
\pmcanonicalname{DiscreteLogarithm}
\pmcreated{2013-03-22 14:54:27}
\pmmodified{2013-03-22 14:54:27}
\pmowner{mathwizard}{128}
\pmmodifier{mathwizard}{128}
\pmtitle{discrete logarithm}
\pmrecord{7}{36592}
\pmprivacy{1}
\pmauthor{mathwizard}{128}
\pmtype{Definition}
\pmcomment{trigger rebuild}
\pmclassification{msc}{11A15}
\pmsynonym{index}{DiscreteLogarithm}

\endmetadata

% this is the default PlanetMath preamble.  as your knowledge
% of TeX increases, you will probably want to edit this, but
% it should be fine as is for beginners.

% almost certainly you want these
\usepackage{amssymb}
\usepackage{amsmath}
\usepackage{amsfonts}

% used for TeXing text within eps files
%\usepackage{psfrag}
% need this for including graphics (\includegraphics)
%\usepackage{graphicx}
% for neatly defining theorems and propositions
%\usepackage{amsthm}
% making logically defined graphics
%%%\usepackage{xypic}

% there are many more packages, add them here as you need them

% define commands here
\begin{document}
Let $p$ be a prime. We know that the group $G:=(\mathbb{Z}/p\mathbb{Z})^*$ is cyclic. Let $g$ be a primitive root of $G$, i.e. $G=\{1,g,g^2,\dots,g^{p-2}\}$.
For a number $x\in G$ we want to know the unique number $0\leq n\leq p-2$ with
$$x=g^n.$$
This number $n$ is called the \textit{discrete logarithm} or \textit{index} of $x$ to the basis $g$ and is denoted as $\operatorname{ind}_g(x)$. For $x,y\in G$ it satisfies the following properties:
\begin{align*}
\operatorname{ind}_g(xy)&=\operatorname{ind}_g(x)+\operatorname{ind}_g(y);\\
\operatorname{ind}_g(x^{-1})&=-\operatorname{ind}_g(x);\\
\operatorname{ind}_g(x^k)&=k\cdot\operatorname{ind}_g(x).
\end{align*}
Furthermore, for a pair $g,h$ of distinct primitive roots, we also have, for any $x\in G$:
\begin{align*}
\operatorname{ind}_h(x)&=\operatorname{ind}_h(g)\cdot \operatorname{ind}_g(x);\\
1 &=\operatorname{ind}_g(h)\cdot \operatorname{ind}_h(g);\\
\operatorname{ind}_g(-1)&=\frac{p-1}{2}.
\end{align*}
It is a difficult problem to compute the discrete logarithm, while powering is very easy. Therefore this is of some interest to cryptography.
%%%%%
%%%%%
\end{document}
