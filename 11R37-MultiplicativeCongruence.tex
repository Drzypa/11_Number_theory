\documentclass[12pt]{article}
\usepackage{pmmeta}
\pmcanonicalname{MultiplicativeCongruence}
\pmcreated{2013-03-22 12:50:16}
\pmmodified{2013-03-22 12:50:16}
\pmowner{djao}{24}
\pmmodifier{djao}{24}
\pmtitle{multiplicative congruence}
\pmrecord{4}{33163}
\pmprivacy{1}
\pmauthor{djao}{24}
\pmtype{Definition}
\pmcomment{trigger rebuild}
\pmclassification{msc}{11R37}
\pmsynonym{multiplicatively congruent}{MultiplicativeCongruence}
\pmrelated{Congruence2}

% this is the default PlanetMath preamble.  as your knowledge
% of TeX increases, you will probably want to edit this, but
% it should be fine as is for beginners.

% almost certainly you want these
\usepackage{amssymb}
\usepackage{amsmath}
\usepackage{amsfonts}

% used for TeXing text within eps files
%\usepackage{psfrag}
% need this for including graphics (\includegraphics)
%\usepackage{graphicx}
% for neatly defining theorems and propositions
%\usepackage{amsthm}
% making logically defined graphics
%%%\usepackage{xypic} 

% there are many more packages, add them here as you need them

% define commands here
\newcommand{\p}{{\mathfrak{p}}}
\newcommand{\m}{{\mathfrak{m}}}
\renewcommand{\P}{{\mathfrak{P}}}
\newcommand{\C}{\mathbb{C}}
\newcommand{\R}{\mathbb{R}}
\newcommand{\Z}{\mathbb{Z}}
\newcommand{\N}{\mathbb{N}}
\renewcommand{\H}{\mathcal{H}}
\newcommand{\A}{\mathbb{A}}
\renewcommand{\c}{\mathcal{C}}
\renewcommand{\O}{\mathcal{O}}
\newcommand{\D}{\mathcal{D}}
\newcommand{\lra}{\longrightarrow}
\renewcommand{\div}{\mid}
\newcommand{\res}{\operatorname{res}}
\newcommand{\Spec}{\operatorname{Spec}}
\newcommand{\id}{\operatorname{id}}
\newcommand{\diff}{\operatorname{diff}}
\newcommand{\incl}{\operatorname{incl}}
\newcommand{\Hom}{\operatorname{Hom}}
\renewcommand{\Re}{\operatorname{Re}}
\newcommand{\intersect}{\cap}
\newcommand{\union}{\cup}
\newcommand{\bigintersect}{\bigcap}
\newcommand{\bigunion}{\bigcup}
\newcommand{\ilim}{\,\underset{\longleftarrow}{\lim}\,}
\begin{document}
Let $\p$ be any real prime of a number field $K$, and write $i: K \lra \R$ for the corresponding real embedding of $K$. We say two elements $\alpha, \beta \in K$ are {\em multiplicatively congruent} mod $\p$ if the real numbers $i(\alpha)$ and $i(\beta)$ are either both positive or both negative.

Now let $\p$ be a finite prime of $K$, and write $(\O_K)_\p$ for the localization of the ring of integers $\O_K$ of $K$ at $\p$. For any natural number $n$, we say $\alpha$ and $\beta$ are {\em multiplicatively congruent} mod $\p^n$ if they are members of the same coset of the subgroup $1+\p^n(\O_K)_\p$ of the multiplicative group $K^\times$ of $K$.

If $\m$ is any modulus for $K$, with factorization
$$
\m = \prod_{\p} \p^{n_\p},
$$
then we say $\alpha$ and $\beta$ are {\em multiplicatively congruent} mod $\m$ if they are multiplicatively congruent mod $\p^{n_\p}$ for every prime $\p$ appearing in the factorization of $\m$.

Multiplicative congruence of $\alpha$ and $\beta$ mod $\m$ is commonly denoted using the notation
$$
\alpha \equiv^* \beta \pmod{\m}.
$$
%%%%%
%%%%%
\end{document}
