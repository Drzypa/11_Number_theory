\documentclass[12pt]{article}
\usepackage{pmmeta}
\pmcanonicalname{ThabitNumber}
\pmcreated{2013-03-22 15:52:58}
\pmmodified{2013-03-22 15:52:58}
\pmowner{Mravinci}{12996}
\pmmodifier{Mravinci}{12996}
\pmtitle{Thabit number}
\pmrecord{5}{37882}
\pmprivacy{1}
\pmauthor{Mravinci}{12996}
\pmtype{Definition}
\pmcomment{trigger rebuild}
\pmclassification{msc}{11A05}
\pmsynonym{Thabit ibn Kurra number}{ThabitNumber}
\pmsynonym{Thabit ibn Kurrah number}{ThabitNumber}
\pmsynonym{Thabit ibn Qurra number}{ThabitNumber}
\pmsynonym{Thabit ibn Qurrah number}{ThabitNumber}
\pmsynonym{Thabit bin Kurra number}{ThabitNumber}
\pmsynonym{Thabit bin Kurrah number}{ThabitNumber}
\pmsynonym{Thabit bin Qurra number}{ThabitNumber}
\pmsynonym{Thabit bin Qurrah number}{ThabitNumber}
\pmrelated{AFormulaForAmicablePairs}

% this is the default PlanetMath preamble.  as your knowledge
% of TeX increases, you will probably want to edit this, but
% it should be fine as is for beginners.

% almost certainly you want these
\usepackage{amssymb}
\usepackage{amsmath}
\usepackage{amsfonts}

% used for TeXing text within eps files
%\usepackage{psfrag}
% need this for including graphics (\includegraphics)
%\usepackage{graphicx}
% for neatly defining theorems and propositions
%\usepackage{amsthm}
% making logically defined graphics
%%%\usepackage{xypic}

% there are many more packages, add them here as you need them

% define commands here
\begin{document}
An integer of the form $3 \cdot 2^n - 1$, or $2^{n + 1} + 2^n - 1$. They are listed in A055010 of Sloane's OEIS. The Thabit numbers are a subset of the Proth numbers.

The mathematician and astronomer Thabit ibn Qurra studied these numbers in search of a formula for amicable pairs. He found that when two consecutive Thabit numbers are  also prime numbers (corresponding to indices $n$ and $n - 1$) and $9 \cdot 2^{2n - 1} - 1$ is a prime number, too, then these numbers multiplied by $2^n$ will reveal an amicable pair. The only $n$ known to fit these criteria are 2, 4 and 7. The largest Thabit number known to be prime corresponds to index 2312734, its immediate lower neighbor is composite.

It is conjectured that the nimfactorial of a Thabit number is always 2.
%%%%%
%%%%%
\end{document}
