\documentclass[12pt]{article}
\usepackage{pmmeta}
\pmcanonicalname{CompositionOfMultiplicativeFunctions}
\pmcreated{2013-03-22 16:09:50}
\pmmodified{2013-03-22 16:09:50}
\pmowner{Wkbj79}{1863}
\pmmodifier{Wkbj79}{1863}
\pmtitle{composition of multiplicative functions}
\pmrecord{6}{38247}
\pmprivacy{1}
\pmauthor{Wkbj79}{1863}
\pmtype{Theorem}
\pmcomment{trigger rebuild}
\pmclassification{msc}{11A25}

\usepackage{amssymb}
\usepackage{amsmath}
\usepackage{amsfonts}

\usepackage{psfrag}
\usepackage{graphicx}
\usepackage{amsthm}
%%\usepackage{xypic}

\newtheorem*{thm*}{Theorem}

\begin{document}
\begin{thm*}
If $f$ is a completely multiplicative function and $g$ is a multiplicative function, then $f \circ g$ is a multiplicative function.
\end{thm*}

\begin{proof}
First note that $(f \circ g)(1)=f(g(1))=f(1)=1$ since both $f$ and $g$ are multiplicative.

Let $a$ and $b$ be relatively prime positive integers.  Then

\begin{center}
\begin{tabular}{ll}
$(f \circ g)(ab)$ & $=f(g(ab))$ \\
& $=f(g(a)\cdot g(b))$ since $g$ is multiplicative \\
& $=f(g(a))f(g(b))$ since $f$ is completely multiplicative \\
& $=(f \circ g)(a)(f \circ g)(b)$. \end{tabular}
\end{center}
\end{proof}

Note that the assumption that $f$ is \emph{completely} multiplicative (as opposed to merely multiplicative) is essential in proving that $f \circ g$ is multiplicative.  For instance, $\tau \circ \tau$, where $\tau$ denotes the divisor function, is not multiplicative:

$$(\tau \circ \tau)(2 \cdot 3)=(\tau \circ \tau)(6)=\tau(\tau(6))=\tau(4)=3$$

$$(\tau \circ \tau)(2) \cdot (\tau \circ \tau)(3)=\tau(\tau(2)) \cdot \tau(\tau(3))=\tau(2) \cdot \tau(2)=2\cdot 2=4$$
%%%%%
%%%%%
\end{document}
