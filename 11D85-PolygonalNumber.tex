\documentclass[12pt]{article}
\usepackage{pmmeta}
\pmcanonicalname{PolygonalNumber}
\pmcreated{2013-03-22 13:55:38}
\pmmodified{2013-03-22 13:55:38}
\pmowner{mathcam}{2727}
\pmmodifier{mathcam}{2727}
\pmtitle{polygonal number}
\pmrecord{5}{34685}
\pmprivacy{1}
\pmauthor{mathcam}{2727}
\pmtype{Definition}
\pmcomment{trigger rebuild}
\pmclassification{msc}{11D85}
\pmclassification{msc}{11D09}
\pmsynonym{figurate number}{PolygonalNumber}
\pmdefines{pentagonal number}

\endmetadata

\usepackage{amssymb}
\usepackage{amsmath}
\usepackage{amsfonts}
\usepackage{graphicx}
\begin{document}
\PMlinkescapeword{connection}
\PMlinkescapeword{fixed}
\PMlinkescapeword{formulas}
A polygonal number, or figurate number, is any value of the function
$$P_d(n)=\frac{(d-2)n^2+(4-d)n}{2}$$
for integers $n\ge 0$ and $d\ge 3$.
A ``generalized polygonal number''
is any value of $P_d(n)$ for some integer $d\ge 3$ and any $n\in\mathbb{Z}$.
For fixed $d$, $P_d(n)$ is called a $d$-gonal or $d$-polygonal number.
For $d=3,4,5,\ldots$, we speak of a triangular number, a square
number or a square, a pentagonal number, and so on.

An equivalent definition of $P_d$, by induction on $n$, is:
$$P_d(0)=0$$
$$P_d(n)=P_d(n-1)+(d-2)(n-1)+1\qquad\text{ for all }n\ge 1$$
$$P_d(n-1)=P_d(n)+(d-2)(1-n)-1\qquad\text{ for all }n<0\;.$$
From these equations, we can deduce that all generalized polygonal
numbers are nonnegative integers.
The first two formulas show that $P_d(n)$ points can be arranged in a
set of $n$ nested $d$-gons, as in this diagram of
$P_3(5)=15$ and $P_5(5)=35$.
\begin{center}\includegraphics{dgon}\end{center}

Polygonal numbers were studied somewhat by the ancients, as far
back as the Pythagoreans, but nowadays their interest
is mostly historical, in connection with this famous result:

\textbf{Theorem: }For any $d\ge 3$, any integer $n\ge 0$ is the
sum of some $d$ $d$-gonal numbers.

In other words, any nonnegative integer is a sum of three
triangular numbers, four squares, five pentagonal numbers,
and so on.
Fermat made this remarkable statement in a letter to Mersenne.
Regrettably, he never revealed the argument or proof that he
had in mind. More than a century passed before Lagrange proved
the easiest case: Lagrange's four-square theorem. The
case $d=3$ was demonstrated by Gauss around 1797, and the
general case by Cauchy in 1813.
%%%%%
%%%%%
\end{document}
