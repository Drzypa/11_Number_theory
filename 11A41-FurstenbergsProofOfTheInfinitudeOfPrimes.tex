\documentclass[12pt]{article}
\usepackage{pmmeta}
\pmcanonicalname{FurstenbergsProofOfTheInfinitudeOfPrimes}
\pmcreated{2013-03-22 14:42:10}
\pmmodified{2013-03-22 14:42:10}
\pmowner{mathcam}{2727}
\pmmodifier{mathcam}{2727}
\pmtitle{F\"urstenberg's proof of the infinitude of primes}
\pmrecord{8}{36315}
\pmprivacy{1}
\pmauthor{mathcam}{2727}
\pmtype{Proof}
\pmcomment{trigger rebuild}
\pmclassification{msc}{11A41}
\pmrelated{HausdorffSpaceNotCompletelyHausdorff}

% this is the default PlanetMath preamble.  as your knowledge
% of TeX increases, you will probably want to edit this, but
% it should be fine as is for beginners.

% almost certainly you want these
\usepackage{amssymb}
\usepackage{amsmath}
\usepackage{amsfonts}
\usepackage{amsthm}

% used for TeXing text within eps files
%\usepackage{psfrag}
% need this for including graphics (\includegraphics)
%\usepackage{graphicx}
% for neatly defining theorems and propositions
%\usepackage{amsthm}
% making logically defined graphics
%%%\usepackage{xypic}

% there are many more packages, add them here as you need them

% define commands here

\newcommand{\mc}{\mathcal}
\newcommand{\mb}{\mathbb}
\newcommand{\mf}{\mathfrak}
\newcommand{\ol}{\overline}
\newcommand{\ra}{\rightarrow}
\newcommand{\la}{\leftarrow}
\newcommand{\La}{\Leftarrow}
\newcommand{\Ra}{\Rightarrow}
\newcommand{\nor}{\vartriangleleft}
\newcommand{\Gal}{\text{Gal}}
\newcommand{\GL}{\text{GL}}
\newcommand{\Z}{\mb{Z}}
\newcommand{\R}{\mb{R}}
\newcommand{\Q}{\mb{Q}}
\newcommand{\C}{\mb{C}}
\newcommand{\<}{\langle}
\renewcommand{\>}{\rangle}
\begin{document}
F\"urstenberg's proof (\cite{fberg}, \cite{priben}) that there are infinitely many primes is an amusing and beautiful blend of elementary number theory and point-set topology.

Consider the arithmetic progression topology on the positive integers, where a basis of open sets is given by subsets of the form $U_{a,b}=\{n\in\Z^+|n\equiv b\mod a\}$.  Arithmetic progressions themselves are by definition open, and in fact clopen, since

\begin{align*}
U_{a,b}^c=\bigcup_{c\neq b} U_{a,c}
\end{align*}

where the union is taken over a set of distinct residue classes modulo $a$.  Hence the complement of $U_{a,b}$ is a union of open sets and so is open, so $U_{a,b}$ itself is closed (and hence clopen).

Consider the set $U=\cup_p U_{p,0}$, where the union runs over all primes $p$.  Then the complement of $U$ in $\Z^+$ is the single element $\{1\}$, which is clearly not an open set (every open set is infinite in this topology).  Thus $U$ is not closed, but since we have written $U$ as a union of closed sets and a \emph{\PMlinkescapetext{finite}} union of closed sets is again closed, this implies that there must be infinitely many terms appearing in that union, i.e. that there must be infinitely many distinct primes.

\begin{thebibliography}{9}
\bibitem{fberg} 
Furstenberg, Harry, 
\emph{On the infinitude of primes}, 
American Mathematical Monthly, Vol. 62, 1955, p. 353. 
\bibitem{priben}
Ribenboim, Paulo.  \emph{The New Book of Prime Number Records}.  Springer, 1996.  p. 10
\end{thebibliography}
%%%%%
%%%%%
\end{document}
