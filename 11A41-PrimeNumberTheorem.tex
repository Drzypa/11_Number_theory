\documentclass[12pt]{article}
\usepackage{pmmeta}
\pmcanonicalname{PrimeNumberTheorem}
\pmcreated{2013-03-22 11:45:18}
\pmmodified{2013-03-22 11:45:18}
\pmowner{bbukh}{348}
\pmmodifier{bbukh}{348}
\pmtitle{prime number theorem}
\pmrecord{21}{30199}
\pmprivacy{1}
\pmauthor{bbukh}{348}
\pmtype{Theorem}
\pmcomment{trigger rebuild}
\pmclassification{msc}{11A41}
\pmclassification{msc}{55U10}
\pmclassification{msc}{55U30}
\pmclassification{msc}{55T25}
\pmclassification{msc}{55M05}
\pmclassification{msc}{55U15}
\pmclassification{msc}{81T25}
\pmclassification{msc}{81-XX}
\pmclassification{msc}{20G42}
\pmclassification{msc}{81R50}
\pmclassification{msc}{17B37}
\pmclassification{msc}{81Q60}
\pmclassification{msc}{81V05}
\pmclassification{msc}{81T05}
\pmclassification{msc}{55R40}
%\pmkeywords{number theory}
\pmdefines{pi(x)}
\pmdefines{logarithmic integral}

\endmetadata

\usepackage{amsmath}
\usepackage{amssymb}
\usepackage{amsfonts}
%\usepackage{graphicx}
%%%%%\usepackage{xypic}
\DeclareMathOperator{\li}{li}
\begin{document}
\PMlinkescapeword{term}
\PMlinkescapeword{behavior}

Define $\pi (x)$ as the number of primes less than or equal to $x$. The prime number theorem asserts that \[ \pi (x) \sim \frac{x}{\log x} \] as $x \rightarrow \infty$, that is, $\pi(x) / \frac{x}{\log x}$ tends to 1 as $x$ increases. Here ${\log x}$ is the natural logarithm.

There is a sharper statement that is also known as the prime number
theorem:
\begin{equation*}
\pi(x)=\li x+R(x),
\end{equation*}
where $\li$ is the logarithmic integral defined as
\begin{equation*}
\li x=\int_2^x \frac{dt}{\log t}=\frac{x}{\log x}+\frac{1!x}{(\log
x)^2}+\dotsb+\frac{(k-1)!x}{(\log x)^{k}}+O\left\{\frac{x}{(\log
x)^{k+1}}\right\},\qquad\text{for any fixed $k$}
\end{equation*}
and $R(x)$ is the error term whose behavior is still not fully
known. From the work of Korobov and Vinogradov on zeroes of
Riemann zeta-function it is known that
\begin{equation*}
R(x)=O\left\{x \exp(-c(\theta)(\log x)^\theta)\right\}
\end{equation*}
for every $\theta>\tfrac{3}{5}$. The unproven Riemann hypothesis
is equivalent to the statement that $R(x)=O(x^{1/2}\log x)$.

There exist a number of proofs of the prime number theorem. The
original proofs by Hadamard \cite{cite:hadamard_pnt} and de~la
Vall\'ee~Poussin\cite{cite:poussin_pnt} called on analysis of
behavior of the Riemann zeta function $\zeta(s)$ near the line $\Re s=1$
to deduce the estimates for $R(x)$. For a long time it was an open
problem to find an elementary proof of the prime number theorem
(``elementary'' meaning ``not involving complex analysis'').
Finally Erd\H{o}s and Selberg
\cite{cite:erdos_elempnt,cite:selberg_elempnt} found such a proof.
Nowadays there are some very short proofs of the prime number
theorem (for example, see~\cite{cite:newman_shortpnt}).

\begin{thebibliography}{1}

\bibitem{cite:apostol_introanalytic}
Tom~M. Apostol.
\newblock {\em Introduction to Analytic Number Theory}.
\newblock Narosa Publishing House, second edition, 1990.
\newblock \PMlinkexternal{Zbl
  0335.10001}{http://www.emis.de/cgi-bin/zmen/ZMATH/en/quick.html?type=html&an=0335.10001}.

\bibitem{cite:davenport_multnumtheory}
Harold Davenport.
\newblock {\em Multiplicative Number Theory}.
\newblock Markham Pub.\ Co., 1967.
\newblock \PMlinkexternal{Zbl
  0159.06303}{http://www.emis.de/cgi-bin/zmen/ZMATH/en/quick.html?type=html&an=0159.06303}.

\bibitem{cite:erdos_elempnt}
Paul Erd{\H{o}}s.
\newblock On a new method in elementary number theory.
\newblock {\em Proc. Nat. Acad. Sci. U.S.A.}, 35:374--384, 1949.
\newblock \PMlinkexternal{Zbl
  0034.31403}{http://www.emis.de/cgi-bin/zmen/ZMATH/en/quick.html?type=html&an=0034.31403}.

\bibitem{cite:hadamard_pnt}
Jacques Hadamard.
\newblock Sur la distribution des z{\'e}ros de la fonction $\zeta(s)$ et ses
  cons{\'e}quences arithm{\'e}tiques.
\newblock {\em Bull. Soc. Math. France}, 24:199--220.
\newblock 
\PMlinkexternal{JFM 27.0154.01}{http://www.emis.de/cgi-bin/zmen/ZMATH/en/quick.html?type=html&an=27.0154.01}.

\bibitem{cite:newman_shortpnt}
Donald~J. Newman.
\newblock Simple analytic proof of the prime number theorem.
\newblock {\em Amer. Math. Monthly}, 87:693--696, 1980.
\newblock 
\PMlinkexternal{Available online}{http://links.jstor.org/sici?sici=0002-9890\%28198011\%2987\%3A9\%3C693\%3ASAPOTP\%3E2.0.CO\%3B2-U}
 at \PMlinkexternal{JSTOR}{http://www.jstor.org}.

\bibitem{cite:selberg_elempnt}
Atle Selberg.
\newblock An elementary proof of the prime number theorem.
\newblock {\em Ann. Math. (2)}, 50:305--311, 1949.
\newblock \PMlinkexternal{Zbl
  0036.30604}{http://www.emis.de/cgi-bin/zmen/ZMATH/en/quick.html?type=html&an=0036.30604}.

\bibitem{cite:poussin_pnt}
Charles~{de~la} {Vall\'ee~Poussin}.
\newblock Recherces analytiques sur la th{\'e}orie des nombres premiers.
\newblock {\em Ann. Soc. Sci. Bruxells}, 1897.

\end{thebibliography}
%%%%%
%%%%%
%%%%%
%%%%%
\end{document}
