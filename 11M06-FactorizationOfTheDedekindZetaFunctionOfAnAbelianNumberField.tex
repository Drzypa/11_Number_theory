\documentclass[12pt]{article}
\usepackage{pmmeta}
\pmcanonicalname{FactorizationOfTheDedekindZetaFunctionOfAnAbelianNumberField}
\pmcreated{2013-03-22 16:01:21}
\pmmodified{2013-03-22 16:01:21}
\pmowner{alozano}{2414}
\pmmodifier{alozano}{2414}
\pmtitle{Factorization of the Dedekind zeta function of an abelian number field}
\pmrecord{4}{38062}
\pmprivacy{1}
\pmauthor{alozano}{2414}
\pmtype{Theorem}
\pmcomment{trigger rebuild}
\pmclassification{msc}{11M06}
\pmclassification{msc}{11R42}
\pmrelated{ValuesOfDedekindZetaFunctionsOfRealQuadraticNumberFieldsAtNegativeIntegers}

\endmetadata

% this is the default PlanetMath preamble.  as your knowledge
% of TeX increases, you will probably want to edit this, but
% it should be fine as is for beginners.

% almost certainly you want these
\usepackage{amssymb}
\usepackage{amsmath}
\usepackage{amsthm}
\usepackage{amsfonts}

% used for TeXing text within eps files
%\usepackage{psfrag}
% need this for including graphics (\includegraphics)
%\usepackage{graphicx}
% for neatly defining theorems and propositions
%\usepackage{amsthm}
% making logically defined graphics
%%%\usepackage{xypic}

% there are many more packages, add them here as you need them

% define commands here

\newtheorem*{thm}{Theorem}
\newtheorem{defn}{Definition}
\newtheorem{prop}{Proposition}
\newtheorem{lemma}{Lemma}
\newtheorem{cor}{Corollary}

\theoremstyle{definition}
\newtheorem{exa}{Example}

% Some sets
\newcommand{\Nats}{\mathbb{N}}
\newcommand{\Ints}{\mathbb{Z}}
\newcommand{\Reals}{\mathbb{R}}
\newcommand{\Complex}{\mathbb{C}}
\newcommand{\Rats}{\mathbb{Q}}
\newcommand{\Gal}{\operatorname{Gal}}
\newcommand{\Cl}{\operatorname{Cl}}
\begin{document}
The Dedekind zeta function of an abelian number field
factors as a product of Dirichlet L-functions as follows. Let $K$
be an abelian number field, i.e. $K/\Rats$ is Galois and
$\Gal(K/\Rats)$ is abelian. Then, by the Kronecker-Weber theorem,
there is an integer $n$ (which we choose to be minimal) such that
$K\subseteq \Rats(\zeta_n)$ where $\zeta_n$ is a primitive $n$th
root of unity. Let $G=\Gal(\Rats(\zeta_n)/\Rats)\cong
(\Ints/n\Ints)^\times$ and let $\chi:G\to \Complex^\times$ be a
Dirichlet character. Then the kernel of $\chi$ determines a fixed
field of $\Rats(\zeta_n)$. Further, for any field $K$ as before,
there exists a group $X$ of Dirichlet characters of $G$ such that
$K$ is equal to the intersection of the fixed fields by the
kernels of all $\chi\in X$. The order of $X$ is $[K:\Rats]$ and
$X\cong \Gal(K/\Rats)$.

\begin{thm}[\cite{wash}, Thm. 4.3]
\label{factor} Let $K$ be an abelian number field and let $X$ be
the associated group of Dirichlet characters. The Dedekind zeta
function of $K$ factors as follows:
$$\zeta_K(s)=\prod_{\chi \in X} L(s,\chi).$$
\end{thm}
Notice that for the trivial character $\chi_0$ one has
$L(s,\chi_0)=\zeta(s)$, the Riemann zeta function, which has a
simple pole at $s=1$ with residue $1$. Thus, for an arbitrary
abelian number field $K$:
$$\zeta_K(s)=\prod_{\chi \in X}L(s,\chi)=\zeta(s)\cdot \prod_{\chi_0\neq \chi \in
X} L(s,\chi)$$ where the last product is taken over all
non-trivial characters $\chi \in X$.

\begin{thebibliography}{99}
\bibitem{wash} L. C. Washington, {\em Introduction to Cyclotomic Fields}, Springer-Verlag, New York.
\end{thebibliography}
%%%%%
%%%%%
\end{document}
