\documentclass[12pt]{article}
\usepackage{pmmeta}
\pmcanonicalname{ContinuedFraction}
\pmcreated{2013-03-22 12:47:12}
\pmmodified{2013-03-22 12:47:12}
\pmowner{PrimeFan}{13766}
\pmmodifier{PrimeFan}{13766}
\pmtitle{continued fraction}
\pmrecord{29}{33101}
\pmprivacy{1}
\pmauthor{PrimeFan}{13766}
\pmtype{Definition}
\pmcomment{trigger rebuild}
\pmclassification{msc}{11Y65}
\pmclassification{msc}{11J70}
\pmclassification{msc}{11A55}
\pmsynonym{chain fraction}{ContinuedFraction}
%\pmkeywords{number theory}
%\pmkeywords{rational number}
%\pmkeywords{irrational number}
%\pmkeywords{Pell}
\pmrelated{FareySequence}
\pmrelated{AdjacentFraction}
\pmrelated{SternBrocotTree}
\pmrelated{FareyPair}
\pmdefines{anthyphairetic ratio}
\pmdefines{simple continued fraction}

\usepackage{amsmath}
\usepackage{amssymb}
\usepackage{amsfonts}
\makeatletter
\@ifundefined{url}{\newcommand*{\sfrac}{\cfrac}}{\newcommand*{\sfrac}{\frac}}
\makeatother
% @ifundefined tests whether LaTeX2HTML or page image mode
% is used because LaTeX2HTML has a bug, which prevents 
% from rendering \cfrac
\begin{document}
\PMlinkescapeword{distribution}
\PMlinkescapeword{simple}
\PMlinkescapeword{terms}
\PMlinkescapeword{cycle}
Given a sequence of positive real numbers $(a_n)_{n \ge 1}$, with $a_0$ any real number. Consider the sequence

\begin{align*}
c_1&=a_0+\sfrac{1}{a_1}\\
c_2&=a_0+\sfrac{1}{a_1+\sfrac{1}{a_2}}\\
c_3&=a_0+\sfrac{1}{a_1+\sfrac{1}{a_2+\sfrac{1}{a_3}}}\\
c_4&=\ldots
\end{align*}

The limit $c$ of this sequence, if it exists, is called the value or limit of the \emph{infinite continued fraction} with \emph{convergents} $(c_n)$, and is denoted by $$a_0 + \sfrac{1}{a_1 + \sfrac{1}{a_2 + \sfrac{1}{a_3 + \ldots}}}$$ or by $$a_{0} + \frac{1}{a_{1} + }\frac{1}{a_{2} + }\frac{1}{a_{3} + }\ldots$$

In the same way, a finite sequence $$(a_n)_{1\le n\le k}$$ defines a finite sequence $$(c_n)_{1\le n\le k}\;.$$ We then speak of a finite continued fraction with value $c_k$.

An archaic word for a continued fraction is \emph{anthyphairetic ratio}.

If the denominators $a_n$ are all (positive) integers, we speak of a \emph{simple} continued fraction. We then use the notation $q = \langle a_0; a_1, a_2, a_3,\ldots \rangle $ or, in the finite case, $q = \langle a_0; a_1, a_2, a_3, \ldots , a_n \rangle \;.$

It is not hard to prove that any irrational number $c$ is the value of a unique infinite simple continued fraction. Moreover, if $c_n$ denotes its $n$th convergent, then $c-c_n$ is an alternating sequence and $|c - c_n|$ is decreasing (as well as convergent to zero). Also, the value of an infinite simple continued fraction is perforce
irrational.

Any rational number is the value of two and only two finite continued fractions; in one of them, the last denominator is 1. E.g. $$\frac{43}{30} = \langle 1;2,3,4 \rangle = \langle 1;2,3,3,1 \rangle \;.$$

These two conditions on a real number $c$ are equivalent:

\noindent
1. $c$ is a root of an irreducible quadratic polynomial with
integer coefficients.

\noindent
2. $c$ is irrational and its simple
continued fraction is ``eventually periodic''; i.e.
$$c=\langle a_0;a_1,a_2,\ldots\rangle$$
and, for some integer $m$ and some integer $k>0$, we have $a_n=a_{n+k}$
for all $n\ge m$.

For example, consider the quadratic equation for the golden ratio: $$x^2 = x + 1$$ or equivalently $$x = 1 + \frac{1}{x}\;.$$ We get

\begin{eqnarray*}
x & = & 1 + \sfrac{1}{1+\sfrac{1}{x}} \\
  & = & 1 + \sfrac{1}{1+\sfrac{1}{1+\sfrac{1}{x}}}
\end{eqnarray*}

and so on.
If $x > 0$, we therefore expect $$x = \langle 1; 1, 1, 1, \ldots \rangle$$ which indeed can be proved. As an exercise, you might like to look for a continued fraction expansion of the \emph{other} solution of $x^2 = x + 1$.

Although $e$ is transcendental, there is a surprising pattern in its simple continued fraction expansion. 
$$e=\langle 2; 1, 2, 1, 1, 4, 1, 1, 6, 1, 1, 8, 1, 1, 10, \ldots \rangle$$
No pattern is apparent in the expansions some other well-known transcendental constants, such as $\pi$ and Ap\'ery's constant $\zeta(3)$.

Owing to a kinship with the Euclidean division algorithm, continued fractions arise naturally in number theory. An
interesting example is the Pell diophantine equation $$x^2 - Dy^2 = 1$$ where $D$ is a nonsquare integer $> 0$. It turns out that if $(x, y)$ is any solution of the Pell equation other than $(\pm 1, 0)$, then $| \frac{x}{y}|$ is a convergent to $\sqrt{D}$.

$\frac{22}{7}$ and $\frac{355}{113}$ are well-known rational approximations to $\pi$, and indeed both are convergents to $\pi$:

\begin{eqnarray*}
3.14159265\ldots & = & \pi = \langle 3;7,15,1,292,...\rangle \\
3.14285714\ldots & = & \frac{22}{7} = \langle 3;7\rangle \\
3.14159292\ldots & = & \frac{355}{113} = \langle 3;7,15,1\rangle=\langle 3;7,16\rangle
\end{eqnarray*}

For one more example, the distribution of leap years in the 4800-month cycle of the Gregorian calendar can be interpreted (loosely speaking) in terms of the continued fraction expansion of the number of days
in a solar year.
%%%%%
%%%%%
\end{document}
