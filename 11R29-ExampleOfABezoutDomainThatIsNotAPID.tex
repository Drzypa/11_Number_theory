\documentclass[12pt]{article}
\usepackage{pmmeta}
\pmcanonicalname{ExampleOfABezoutDomainThatIsNotAPID}
\pmcreated{2013-03-22 16:57:04}
\pmmodified{2013-03-22 16:57:04}
\pmowner{Wkbj79}{1863}
\pmmodifier{Wkbj79}{1863}
\pmtitle{example of a Bezout domain that is not a PID}
\pmrecord{13}{39220}
\pmprivacy{1}
\pmauthor{Wkbj79}{1863}
\pmtype{Example}
\pmcomment{trigger rebuild}
\pmclassification{msc}{11R29}
\pmclassification{msc}{11R04}
\pmclassification{msc}{13G05}

\usepackage{amssymb}
\usepackage{amsmath}
\usepackage{amsfonts}

\usepackage{psfrag}
\usepackage{graphicx}
\usepackage{amsthm}
%%\usepackage{xypic}

\begin{document}
Let $\mathbb{A}$ be the ring of all algebraic numbers whose minimal polynomials are in $\mathbb{Z}[x]$; \PMlinkname{i.e.}{Ie}, every element of $\mathbb{A}$ is an algebraic integer.

In the following example, ideals are considered to be of $\mathbb{A}$ unless indicated otherwise via intersection with a subring of $\mathbb{A}$.

Let $I$ be a \PMlinkescapetext{finitely generated} ideal of $\mathbb{A}$.  Then there exists a positive integer $n$ and $\alpha_1, \dots , \alpha_n \in \mathbb{A}$ with $I=\langle \alpha_1, \dots , \alpha_n \rangle$.  Let $K=\mathbb{Q}(\alpha_1, \dots , \alpha_n)$, and let $\mathcal{O}_K$ denote the ring of integers of $K$.  Then $\alpha_1, \dots , \alpha_n \in \mathcal{O}_K$ and $I \cap \mathcal{O}_K$ is an ideal of $\mathcal{O}_K$.  Let $h$ denote the class number of $K$.  Then $(I \cap \mathcal{O}_K)^h=\langle \beta \rangle \cap \mathcal{O}_K$ for some $\beta \in \mathcal{O}_K$.  Let $L=K(\sqrt[h]{\beta})$, and let $\mathcal{O}_L$ denote the ring of integers of $L$.  Then

\begin{center}
$\begin{array}{rl}
(I \cap \mathcal{O}_L)^h & = [(I \cap \mathcal{O}_K)\mathcal{O}_L]^h \\
& =(I \cap \mathcal{O}_K)^h(\mathcal{O}_L)^h \\
& =(\langle \beta \rangle \cap \mathcal{O}_K) \mathcal{O}_L \\
& =\langle \beta \rangle \cap \mathcal{O}_L \\
& =(\langle \sqrt[h]{\beta} \rangle \cap \mathcal{O}_L)^h \end{array}$
\end{center}

Since unique factorization of ideals holds in $\mathcal{O}_L$, $I \cap \mathcal{O}_L=\langle \sqrt[h]{\beta} \rangle \cap \mathcal{O}_L$.  Since $\mathcal{O}_K \subseteq \mathcal{O}_L$ and $\alpha_1, \dots , \alpha_n \in I \cap \mathcal{O}_K \subseteq I \cap \mathcal{O}_L=\langle \sqrt[h]{\beta} \rangle \cap \mathcal{O}_L$, there exist $\gamma_1, \dots , \gamma_n \in \mathcal{O}_L$ with $\alpha_j = \gamma_j \sqrt[h]{\beta}$ for all positive integers $j$ with $j \le n$.  Thus, $I=\langle \alpha_1, \dots , \alpha_n \rangle = \langle \gamma_1 \sqrt[h]{\beta}, \dots , \gamma_n \sqrt[h]{\beta} \rangle \subseteq \langle \sqrt[h]{\beta} \rangle$.  Since $I \subseteq \langle \sqrt[h]{\beta} \rangle$ and $I \cap \mathcal{O}_L =\langle \sqrt[h]{\beta} \rangle \cap \mathcal{O}_L$, $I=\langle \sqrt[h]{\beta} \rangle$.  Hence, $I$ is principal.  It follows that $\mathbb{A}$ is a Bezout domain.

On the other hand, $\mathbb{A}$ is not a principal ideal domain (PID).  For example, the ideal \PMlinkescapetext{generated by} all of the \PMlinkname{$n$th roots}{NthRoot} of $2$, $J=\langle 2, \sqrt{2}, \sqrt[3]{2}, \dots \rangle$, is an ideal of $\mathbb{A}$ that is not principal.
%%%%%
%%%%%
\end{document}
