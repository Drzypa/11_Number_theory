\documentclass[12pt]{article}
\usepackage{pmmeta}
\pmcanonicalname{Fortytwo}
\pmcreated{2013-03-22 16:28:37}
\pmmodified{2013-03-22 16:28:37}
\pmowner{CompositeFan}{12809}
\pmmodifier{CompositeFan}{12809}
\pmtitle{forty-two}
\pmrecord{8}{38642}
\pmprivacy{1}
\pmauthor{CompositeFan}{12809}
\pmtype{Feature}
\pmcomment{trigger rebuild}
\pmclassification{msc}{11A99}

\endmetadata

% this is the default PlanetMath preamble.  as your knowledge
% of TeX increases, you will probably want to edit this, but
% it should be fine as is for beginners.

% almost certainly you want these
\usepackage{amssymb}
\usepackage{amsmath}
\usepackage{amsfonts}

% used for TeXing text within eps files
%\usepackage{psfrag}
% need this for including graphics (\includegraphics)
%\usepackage{graphicx}
% for neatly defining theorems and propositions
%\usepackage{amsthm}
% making logically defined graphics
%%%\usepackage{xypic}

% there are many more packages, add them here as you need them

% define commands here

\begin{document}
\PMlinkescapeword{core}

Ever since Douglas Adams's {\it Hitchhiker's Guide to the Galaxy} was first published, amateur mathematicians have sought to give explanations, with varying degrees of seriousness, as to why {\em forty-two} is {\em The Answer to Life, the Universe, and Everything}.

In the Adams book, 42 is the answer given by the computer Deep Thought, which then tells the people waiting to hear the answer that they don't know what the question is. Arthur Dent tries to assemble the question from Scrabble tiles, coming up with the question "What do you get if you multiply six by nine?" The answer is $54 = 42_{13}$. Adams says he did not intend to reference base 13 at all, and thus 54 would be a red herring in a search for the question.

From a purely mathematical point of view, the most interesting thing about 42 could be that it could be the third moment of the Riemann zeta function, the leading coefficient of $${1 \over T}\int_0^T \left| \zeta\left({1 \over 2} + it\right) \right|^6\,dt$$ expanded in powers of $\log(T)$.

The third primary pseudoperfect number and the product of the first three members of Sylvester's sequence, 42 is the second sphenic number, but doesn't seem to be the first of any interesting integer sequence (at least any sequence with the keywords "core" or "nice" in the OEIS). It does belong to several other interesting sequences, however, that of the Catalan numbers to name just one more.

When the 42nd Mersenne prime was discovered in 2005, Weisstein jokingly remarked that the answer "is somehow contained in the 7.8 million decimal digits of that" prime. Berggren, Borwein and Borwein (authors of {\it Pi: A Source Book}) have also looked for 42 in the decimal expansions of $\pi$ and ${1 \over \pi}$, finding "042" at the fifty billionth position in both.

\begin{thebibliography}{3}
\bibitem{jk} J. P. Keating and N. C. Snaith, ``Random Matrix Theory and $\theta(\frac{1}{2} + it)$", {\it Commun. Math. Phys.} {\bf 214}, (2000): 57 - 89
\bibitem{lb} L. Berggren, J. Borwein, and P. Borwein, {\it Pi: A Source Book}, New York: Springer-Verlag (1997)
\bibitem{ew} E. Weisstein, ``\PMlinkexternal{42}{http://mathworld.wolfram.com/42.html}" {\it Mathworld}
\end{thebibliography}
%%%%%
%%%%%
\end{document}
