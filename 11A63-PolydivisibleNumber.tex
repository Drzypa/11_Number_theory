\documentclass[12pt]{article}
\usepackage{pmmeta}
\pmcanonicalname{PolydivisibleNumber}
\pmcreated{2013-03-22 16:22:20}
\pmmodified{2013-03-22 16:22:20}
\pmowner{CompositeFan}{12809}
\pmmodifier{CompositeFan}{12809}
\pmtitle{polydivisible number}
\pmrecord{5}{38511}
\pmprivacy{1}
\pmauthor{CompositeFan}{12809}
\pmtype{Definition}
\pmcomment{trigger rebuild}
\pmclassification{msc}{11A63}

% this is the default PlanetMath preamble.  as your knowledge
% of TeX increases, you will probably want to edit this, but
% it should be fine as is for beginners.

% almost certainly you want these
\usepackage{amssymb}
\usepackage{amsmath}
\usepackage{amsfonts}

% used for TeXing text within eps files
%\usepackage{psfrag}
% need this for including graphics (\includegraphics)
%\usepackage{graphicx}
% for neatly defining theorems and propositions
%\usepackage{amsthm}
% making logically defined graphics
%%%\usepackage{xypic}

% there are many more packages, add them here as you need them

% define commands here

\begin{document}
Given a base $b$ integer $n$ with $k$ digits $d_1, \ldots, d_k$, consider $d_k$ the least significant digit and $d_1$, to suit our purpose in this case. If for each $1 < j < k$ it is the case that $$(\sum_{i = 1}^j d_ib^{k - j - i}) | j,$$ then $n$ is said to be a {\em polydivisible number}.

A reasonably good estimate of how many polydivisible numbers base $b$ has is $$\sum_{i = 2}^{b - 1} \frac{(b - 1)b^{i - 1}}{i!}.$$ In any given base, there is only one polydivisible number that is also a pandigital number.
%%%%%
%%%%%
\end{document}
