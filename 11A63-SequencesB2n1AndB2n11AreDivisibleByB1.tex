\documentclass[12pt]{article}
\usepackage{pmmeta}
\pmcanonicalname{SequencesB2n1AndB2n11AreDivisibleByB1}
\pmcreated{2013-03-22 16:14:19}
\pmmodified{2013-03-22 16:14:19}
\pmowner{perucho}{2192}
\pmmodifier{perucho}{2192}
\pmtitle{sequences $b^{2n}-1$ and $b^{2n-1}+1$ are divisible by $b+1$}
\pmrecord{6}{38340}
\pmprivacy{1}
\pmauthor{perucho}{2192}
\pmtype{Derivation}
\pmcomment{trigger rebuild}
\pmclassification{msc}{11A63}

% this is the default PlanetMath preamble.  as your knowledge
% of TeX increases, you will probably want to edit this, but
% it should be fine as is for beginners.

% almost certainly you want these
\usepackage{amssymb}
\usepackage{amsmath}
\usepackage{amsfonts}

% used for TeXing text within eps files
%\usepackage{psfrag}
% need this for including graphics (\includegraphics)
%\usepackage{graphicx}
% for neatly defining theorems and propositions
%\usepackage{amsthm}
% making logically defined graphics
%%%\usepackage{xypic}

% there are many more packages, add them here as you need them

% define commands here

\begin{document}
Consider the alternating geometric finite series
\begin{align}
S_{m+1}(\mu)=\sum_{i=0}^m(-1)^{i+\mu}b^i,
\end{align} 
where $\mu=1,2$ and $b\geq 2$ an integer. Multiplying (1) by $-b$ and subtracting from it
\begin{align*}
(b+1)S_{m+1}(\mu)=\sum_{i=0}^m(-1)^{i+\mu}b^i-\sum_{i=0}^m(-1)^{i+1+\mu}b^{i+1}
\end{align*}
and by elemental manipulations, we obtain
\begin{align}
S_{m+1}(\mu)=\frac{(-1)^\mu[1-(-1)^{m+1}b^{m+1}]}{b+1}=
\sum_{i=0}^m(-1)^{i+\mu}b^i.
\end{align}
Let $\mu=1$, $m=2n-1$. Then
\begin{align}
\frac{b^{2n}-1}{b+1}=-\sum_{i=0}^{2n-1}(-1)^ib^i.
\end{align}
Likewise, for $\mu=2$, $m=2n-2$
\begin{align}
\frac{b^{2n-1}+1}{b+1}=\sum_{i=0}^{2n-2}(-1)^ib^i,
\end{align}
as desired.
\paragraph{Palindromic numbers of even length}
As an application of above sequences, let us consider an even palindromic number (EPN) of arbitrary length $2n$ which can be expressed in any base $b$ as
\begin{align}
(EPN)_n=\sum_{k=0}^{n-1}b_k(b^{2n-1-k}+b^k)=
\sum_{k=0}^{n-1}\frac{b_k}{b^k}[(b^{2n-1}+1)+(b^{2k}-1)],
\end{align}
where $0\leq b_k\leq b-1$.It is clear, from (3) and (4), that $(EPN)_n$ is divisible by $b+1$. Indeed this one can be given by
\begin{align}
(EPN)_n=(b+1)\sum_{k=0}^{n-1}\sum_{j=0}^{2(n-1-k)}b_k(-1)^jb^{k+j}.
\end{align} 


%%%%%
%%%%%
\end{document}
