\documentclass[12pt]{article}
\usepackage{pmmeta}
\pmcanonicalname{RiemannHurwitzTheorem}
\pmcreated{2013-03-22 15:34:40}
\pmmodified{2013-03-22 15:34:40}
\pmowner{alozano}{2414}
\pmmodifier{alozano}{2414}
\pmtitle{Riemann-Hurwitz theorem}
\pmrecord{4}{37486}
\pmprivacy{1}
\pmauthor{alozano}{2414}
\pmtype{Theorem}
\pmcomment{trigger rebuild}
\pmclassification{msc}{11R58}
\pmdefines{different divisor of an extension of function fields}

% this is the default PlanetMath preamble.  as your knowledge
% of TeX increases, you will probably want to edit this, but
% it should be fine as is for beginners.

% almost certainly you want these
\usepackage{amssymb}
\usepackage{amsmath}
\usepackage{amsthm}
\usepackage{amsfonts}

% used for TeXing text within eps files
%\usepackage{psfrag}
% need this for including graphics (\includegraphics)
%\usepackage{graphicx}
% for neatly defining theorems and propositions
%\usepackage{amsthm}
% making logically defined graphics
%%%\usepackage{xypic}

% there are many more packages, add them here as you need them

% define commands here

\newtheorem*{thm}{Theorem}
\newtheorem{defn}{Definition}
\newtheorem{prop}{Proposition}
\newtheorem{lemma}{Lemma}
\newtheorem{cor}{Corollary}

\theoremstyle{definition}
\newtheorem{exa}{Example}

% Some sets
\newcommand{\Nats}{\mathbb{N}}
\newcommand{\Ints}{\mathbb{Z}}
\newcommand{\Reals}{\mathbb{R}}
\newcommand{\Complex}{\mathbb{C}}
\newcommand{\Rats}{\mathbb{Q}}
\newcommand{\Gal}{\operatorname{Gal}}
\newcommand{\Cl}{\operatorname{Cl}}
\begin{document}
First we define the different divisor of an extension of function fields. Let $K$ be a function field over a field $F$ and let $L$ be a finite separable extension of $K$. Let $\mathcal{O}_P$ be a prime of $K$, i.e. a discrete valuation ring with $F\subset \mathcal{O}_P$, maximal ideal $P$ and quotient field equal to $K$. Let $R_P$ be the integral closure of $\mathcal{O}_P$ in $L$. Notice that if $\mathfrak{p}$ is a prime ideal of $R_P$, then the localization $\mathcal{O}_{\mathfrak{p}}=(R_P)_{\mathfrak{p}}$ is a prime of $L$ (which is said to be lying over $\mathcal{O}_P$). The maximal ideal of $\mathcal{O}_{\mathfrak{p}}$ is $\mathfrak{p}(R_P)_{\mathfrak{p}}$.

Let $\mathcal{O}_\mathfrak{P}$ be any prime of $L$, then it lays over some prime ideal $P$ of $K$ and in fact, if $\mathfrak{p}=R_P\cap \mathfrak{P}$ then $\mathcal{O}_\mathfrak{p}\cong \mathcal{O}_\mathfrak{P}$. Let $\delta(\mathfrak{P})$ be the exact power of $\mathfrak{p}$ dividing the different  of $R_P$ over $\mathcal{O}_P$ (the different of an extension of Dedekind domains is a fractional ideal). We define the different divisor of $L/K$ as follows:
$$D_{L/K}=\sum_\mathfrak{P} \delta({\mathfrak{P}})\mathfrak{P}$$
as an element of the free abelian group generated by the prime ideals of $L$.

\begin{thm}[Riemann-Hurwitz]
Let $L/K$ be a finite, separable, geometric extension of function fields and suppose the genus of $K$ is $g_K$. Then the genus of $L$ is given by the formula:
$$2g_L-2=[L:K](2g_K-2)+\deg_L D_{L/K}.$$
\end{thm}
%%%%%
%%%%%
\end{document}
