\documentclass[12pt]{article}
\usepackage{pmmeta}
\pmcanonicalname{SumOfReciprocalsOfSylvestersSequence}
\pmcreated{2013-03-22 15:48:33}
\pmmodified{2013-03-22 15:48:33}
\pmowner{rspuzio}{6075}
\pmmodifier{rspuzio}{6075}
\pmtitle{sum of reciprocals of Sylvester's sequence}
\pmrecord{7}{37773}
\pmprivacy{1}
\pmauthor{rspuzio}{6075}
\pmtype{Proof}
\pmcomment{trigger rebuild}
\pmclassification{msc}{11A55}

% this is the default PlanetMath preamble.  as your knowledge
% of TeX increases, you will probably want to edit this, but
% it should be fine as is for beginners.

% almost certainly you want these
\usepackage{amssymb}
\usepackage{amsmath}
\usepackage{amsfonts}

% used for TeXing text within eps files
%\usepackage{psfrag}
% need this for including graphics (\includegraphics)
%\usepackage{graphicx}
% for neatly defining theorems and propositions
%\usepackage{amsthm}
% making logically defined graphics
%%%\usepackage{xypic}

% there are many more packages, add them here as you need them

% define commands here
\begin{document}
We will show that the sum of the reciprocals of the Sylvester numbers indeed
converges to 1.  

Let $s_n$ denote a partial sum of the series of reciprocals: 
\[ s_n = \sum_{i_0}^{n-1} {1 \over a_i} \] 
We would like to show that $\lim_{n \to \infty} s_n = 1$.  Putting
over a common denominator, we obtain 
\[s_n = {\sum_{j=0}^{n-1} \prod\limits_{i \neq j \atop 0 \le i < n}
a_i \over \prod_{i=0}^{n-1} a_i}. \]
Define $b_n$ as follows:
\[ b_n = 1 + \sum_{j=0}^{n-1} \prod_{i \neq j \atop 0 \le i < n} a_i
\]
Using this new definition and the definition of the Sylvester numbers,
we can rewrite the expression for $s_n$ as follows:
\[ s_n = {b_n + 1 \over a_n - 1}\]

Let us now consider this sequence $b_n$.  We will start by deriving a
recurrence relation:
\begin{eqnarray*}
b_{n+1} - 1 &=&  \sum_{j=0}^n \prod_{i \neq j \atop 0 \le i < n+1} a_i =
\prod_{i=0}^{n-1} a_i + a_n \sum_{j=0}^{n-1} \prod_{i \neq j \atop 0 \le
i < n} a_i \\
&=& (a_n - 1) + a_n (b_n - 1) \\
\end{eqnarray*}
Simplifying, we have $b_{n+1} = a_n b_n$.  Now, $b_2 = 1 + a_0 + a_1 =
6$, hence we can solve the recursion with a product:
\begin{eqnarray*}
b_n &=& b_2 \prod_{i=2}^{n-1} a_i \\
&=& {b_2 \over a_0 a_1} \prod_{1=0}^{n-1} a_i \\
&=& \prod_{1=0}^{n-1} a_i \\
&=& a_n - 1
\end{eqnarray*}

Substituting this in the expression for $s_n$ yields
\[ s_n = {a_n \over a_n - 1}. \]
Since $\lim_{n \to \infty} a_n = \infty$,
it follows that $\lim_{n \to \infty} s_n = 1$.
%%%%%
%%%%%
\end{document}
