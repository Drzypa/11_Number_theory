\documentclass[12pt]{article}
\usepackage{pmmeta}
\pmcanonicalname{ProofOfNewtonGirardFormulaForSymmetricPolynomials}
\pmcreated{2013-03-22 15:34:37}
\pmmodified{2013-03-22 15:34:37}
\pmowner{kshum}{5987}
\pmmodifier{kshum}{5987}
\pmtitle{proof of Newton-Girard formula for symmetric polynomials}
\pmrecord{4}{37485}
\pmprivacy{1}
\pmauthor{kshum}{5987}
\pmtype{Proof}
\pmcomment{trigger rebuild}
\pmclassification{msc}{11C08}

\endmetadata

% this is the default PlanetMath preamble.  as your knowledge
% of TeX increases, you will probably want to edit this, but
% it should be fine as is for beginners.

% almost certainly you want these
\usepackage{amssymb}
\usepackage{amsmath}
\usepackage{amsfonts}

% used for TeXing text within eps files
%\usepackage{psfrag}
% need this for including graphics (\includegraphics)
%\usepackage{graphicx}
% for neatly defining theorems and propositions
%\usepackage{amsthm}
% making logically defined graphics
%%%\usepackage{xypic}

% there are many more packages, add them here as you need them

% define commands here
\begin{document}
The following is a proof of Newton-Girard formula using formal
power series. Let $z$ be an indeterminate and $f(z)$ be the
polynomial
\[1-E_1z+\ldots+(-1)^nE_nz^n.\]
Take log and differentiate both sides of the equation
\[ f(z) = \prod_{i=1}^n (1-x_iz).\]

We obtain
\begin{equation} f'(z)/f(z) = \sum_{i=1}^n
\frac{-x_i}{1-x_iz}, \label{eq2}
\end{equation}
where $f'(z)$ is the  derivative of $f(z)$
\[
 f'(z) =  -E_1 + 2E_2z - \ldots +(-1)^{n}nE_n z^{n-1}.
\]

The right hand side of \eqref{eq2} is equal to
\[
  -\sum_{i=1}^n \sum_{k=0}^\infty x_i^{k+1} z^{k} =
  -\sum_{k=0}^\infty S_{k+1} z^{k}.
\]

By equating coefficients of
\[
 f'(z) = -f(z)(S_1+S_2z+S_3z^2+\ldots)
\]
we get the Newton-Girard formula.
%%%%%
%%%%%
\end{document}
