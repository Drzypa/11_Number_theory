\documentclass[12pt]{article}
\usepackage{pmmeta}
\pmcanonicalname{PrimeQuadrupletConjecture}
\pmcreated{2013-03-22 19:00:38}
\pmmodified{2013-03-22 19:00:38}
\pmowner{PrimeFan}{13766}
\pmmodifier{PrimeFan}{13766}
\pmtitle{prime quadruplet conjecture}
\pmrecord{5}{41880}
\pmprivacy{1}
\pmauthor{PrimeFan}{13766}
\pmtype{Conjecture}
\pmcomment{trigger rebuild}
\pmclassification{msc}{11N05}

% this is the default PlanetMath preamble.  as your knowledge
% of TeX increases, you will probably want to edit this, but
% it should be fine as is for beginners.

% almost certainly you want these
\usepackage{amssymb}
\usepackage{amsmath}
\usepackage{amsfonts}

% used for TeXing text within eps files
%\usepackage{psfrag}
% need this for including graphics (\includegraphics)
%\usepackage{graphicx}
% for neatly defining theorems and propositions
%\usepackage{amsthm}
% making logically defined graphics
%%%\usepackage{xypic}

% there are many more packages, add them here as you need them

% define commands here

\begin{document}
\PMlinkescapeword{terms}

{\bf Conjecture.} (Hardy \& Littlewood) There are infinitely many prime quadruplets.

As with twin primes, prime quadruplets generally become scarcer the higher one looks for them with the aid of the computer, yet they also display the same unevenness of distribution: there is only one prime quadruplet between 40000 and 50000, yet there are three between 70000 and 80000. While Euclid proved long ago that there are infinitely many primes, it is still not known whether there are also infinitely many prime quadruplets.

The question is related to the twin prime conjecture: proving the prime quadruplet conjecture would automatically prove the twin prime conjecture as well. However, disproving the prime quadruplet conjecture might not necessarily disprove the twin prime conjecture as well.

Hardy and Littlewood stated their conjecture in more general terms as the prime $k$-tuple conjecture, with the case of the prime quadruplets being for $k = 4$.
%%%%%
%%%%%
\end{document}
