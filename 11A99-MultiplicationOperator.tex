\documentclass[12pt]{article}
\usepackage{pmmeta}
\pmcanonicalname{MultiplicationOperator}
\pmcreated{2013-03-22 16:46:32}
\pmmodified{2013-03-22 16:46:32}
\pmowner{PrimeFan}{13766}
\pmmodifier{PrimeFan}{13766}
\pmtitle{multiplication operator}
\pmrecord{6}{39005}
\pmprivacy{1}
\pmauthor{PrimeFan}{13766}
\pmtype{Definition}
\pmcomment{trigger rebuild}
\pmclassification{msc}{11A99}

\endmetadata

% this is the default PlanetMath preamble.  as your knowledge
% of TeX increases, you will probably want to edit this, but
% it should be fine as is for beginners.

% almost certainly you want these
\usepackage{amssymb}
\usepackage{amsmath}
\usepackage{amsfonts}

% used for TeXing text within eps files
%\usepackage{psfrag}
% need this for including graphics (\includegraphics)
%\usepackage{graphicx}
% for neatly defining theorems and propositions
%\usepackage{amsthm}
% making logically defined graphics
%%%\usepackage{xypic}

% there are many more packages, add them here as you need them

% define commands here

\begin{document}
\PMlinkescapeword{multiplication operator}
\PMlinkescapeword{multiplication operators}

A {\em multiplication operator} is an operator signifying multiplication among two or more operands. Common multiplication operators are $\times$, $\cdot$ and $\ast$ as well as the tacit multiplication operator. The iterated multiplication operator is the Greek capital letter $\Pi$. Most of these can be used in Polish notation or reverse Polish notation just as easily as in standard infix notation. For example: $2 \cdot 3 \cdot 7$ or $2 \quad 3 \quad 7 \times$.

Most computer programming languages (such as FORTRAN and \PMlinkname{C++}{C}) use the asterisk ($\ast$), which is available as Shift-6 on almost all American typewriters and computer keyboards. For arithmetic computations by hand on paper, the cross ($\times$) is prefered and this is the symbol that appears on the multiplication key of most calculators. The tacit multiplication operator is mostly used in algebra to multiply named variables or constants, or a single literal by single or multiple named variables or constants.
%%%%%
%%%%%
\end{document}
