\documentclass[12pt]{article}
\usepackage{pmmeta}
\pmcanonicalname{PerrinSequence}
\pmcreated{2013-03-22 16:05:19}
\pmmodified{2013-03-22 16:05:19}
\pmowner{Mravinci}{12996}
\pmmodifier{Mravinci}{12996}
\pmtitle{Perrin sequence}
\pmrecord{5}{38149}
\pmprivacy{1}
\pmauthor{Mravinci}{12996}
\pmtype{Definition}
\pmcomment{trigger rebuild}
\pmclassification{msc}{11B39}
\pmdefines{Perrin number}

\endmetadata

% this is the default PlanetMath preamble.  as your knowledge
% of TeX increases, you will probably want to edit this, but
% it should be fine as is for beginners.

% almost certainly you want these
\usepackage{amssymb}
\usepackage{amsmath}
\usepackage{amsfonts}

% used for TeXing text within eps files
%\usepackage{psfrag}
% need this for including graphics (\includegraphics)
%\usepackage{graphicx}
% for neatly defining theorems and propositions
%\usepackage{amsthm}
% making logically defined graphics
%%%\usepackage{xypic}

% there are many more packages, add them here as you need them

% define commands here

\begin{document}
Construct a recurrence relation with initial terms $a_0 = 3$, $a_1 = 0$, $a_2 = 2$ and $a_n = a_{n - 3} + a_{n - 2}$ for $n > 2$. The first few terms of the sequence defined by this recurrence relation are: 3, 0, 2, 3, 2, 5, 5, 7, 10, 12, 17, 22, 29, 39, 51, 68, 90, 119, 158, 209, 277, 367 (listed in A001608 of Sloane's OEIS). This is the {\em Perrin sequence}, sometimes called the {\em Ondrej Such sequence}. Its generating function is $$G(a(n);x)=\frac{3-x^2}{1-x^2-x^3}.$$ A number in the Perrin sequence is called a \emph{Perrin number}.

It has been observed that if $n|a_n$, then $n$ is a prime number, at least among the first hundred thousand integers or so. However, the square of 521 passes this test.

The $n$th Perrin number asymptotically matches the $n$th power of the plastic constant.

\begin{thebibliography}{1}
\bibitem{wa} W. W. Adams and D. Shanks, ``Strong primality tests that are not sufficient" {\it Math. Comp.} {\bf 39}, pp. 255 - 300 (1982)
\end{thebibliography}
%%%%%
%%%%%
\end{document}
