\documentclass[12pt]{article}
\usepackage{pmmeta}
\pmcanonicalname{WilsonsTheoremForPrimePowers}
\pmcreated{2013-03-22 13:22:14}
\pmmodified{2013-03-22 13:22:14}
\pmowner{Thomas Heye}{1234}
\pmmodifier{Thomas Heye}{1234}
\pmtitle{Wilson's theorem for prime powers}
\pmrecord{8}{33899}
\pmprivacy{1}
\pmauthor{Thomas Heye}{1234}
\pmtype{Theorem}
\pmcomment{trigger rebuild}
\pmclassification{msc}{11A07}
\pmclassification{msc}{11A41}

\endmetadata

% this is the default PlanetMath preamble.  as your knowledge
% of TeX increases, you will probably want to edit this, but
% it should be fine as is for beginners.

% almost certainly you want these
\usepackage{amssymb}
\usepackage{amsmath}
\usepackage{amsfonts}

% used for TeXing text within eps files
%\usepackage{psfrag}
% need this for including graphics (\includegraphics)
%\usepackage{graphicx}
% for neatly defining theorems and propositions
%\usepackage{amsthm}
% making logically defined graphics
%%%\usepackage{xypic}

% there are many more packages, add them here as you need them

\newcommand{\pfac}[1]{\left(#1\underline{!}\right)_p}
\begin{document}
For every natural number $n$, let $\pfac{n}$ denote the product of numbers $1
\le m \le n$ with $gcd(m,p) =1$.

For prime $p$ and $s \in \mathbb{N}$
\[\pfac{p^s} \equiv \left(
\begin{array}{ll}
1 & \mbox{for } p=2, s \ge 3 \\
-1 & \mbox{otherwise}
\end{array}\right. \pmod{p^s}.\]

\textbf{Proof:} We pair up all factors of the product $\pfac{p^s}$ into those
numbers $m$ where $m \not\equiv m^{-1} \pmod{p^s}$ and those where this is not
the case. So $\pfac{p^s}$ is congruent (modulo $p^s$) to the product of those
numbers $m$ where $m \equiv m{-1} \pmod{p^s} \leftrightarrow m^2 \equiv 1
\pmod{p^s}$.

Let $p$ be an odd prime and $s \in \mathbb{N}$. Since $2 \not\vert p^s$, $p^s
\vert (m^2 -1)$ implies $p^s \vert (m +1)$ either or $p^s \vert (m-1)$. This
leads to
\[\pfac{p^s} \equiv -1 \pmod{p^s}\]
for odd prime $p$ and any $s \in \mathbb{N}$.

Now let $p=2$ and $s \ge 2$. Then
\[\left(1 +t.2^{s-1}\right)^2 \equiv 1 \pmod{2^s}, t =\stackrel{-}{+}1.\]
Since
\[\left(2^{s -1} +1\right)\left(2^{s-1} -1\right) \equiv -1 \pmod{2^s},\]
we have
\[\pfac{p^s} \equiv (-1).(-1) =1 \pmod{p^s}\]
For $p=2, s \ge 3$, but $-1$ for $s=1,2. \square$
%%%%%
%%%%%
\end{document}
