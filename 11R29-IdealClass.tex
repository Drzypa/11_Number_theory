\documentclass[12pt]{article}
\usepackage{pmmeta}
\pmcanonicalname{IdealClass}
\pmcreated{2013-03-22 12:36:42}
\pmmodified{2013-03-22 12:36:42}
\pmowner{mathcam}{2727}
\pmmodifier{mathcam}{2727}
\pmtitle{ideal class}
\pmrecord{22}{32869}
\pmprivacy{1}
\pmauthor{mathcam}{2727}
\pmtype{Definition}
\pmcomment{trigger rebuild}
\pmclassification{msc}{11R29}
\pmsynonym{ideal classes}{IdealClass}
\pmrelated{ExistenceOfHilbertClassField}
\pmrelated{FractionalIdeal}
\pmrelated{NumberField}
\pmrelated{UnramifiedExtensionsAndClassNumberDivisibility}
\pmrelated{ClassNumberDivisibilityInExtensions}
\pmrelated{PushDownTheoremOnClassNumbers}
\pmrelated{MinkowskisConstant}
\pmrelated{ExtensionsWithoutUnramifiedSubextensionsAndClassNumberDivis}
\pmdefines{class number}
\pmdefines{ideal class group}
\pmdefines{class group}

% this is the default PlanetMath preamble.  as your knowledge
% of TeX increases, you will probably want to edit this, but
% it should be fine as is for beginners.

% almost certainly you want these
\usepackage{amssymb}
\usepackage{amsmath}
\usepackage{amsfonts}

% used for TeXing text within eps files
%\usepackage{psfrag}
% need this for including graphics (\includegraphics)
%\usepackage{graphicx}
% for neatly defining theorems and propositions
%\usepackage{amsthm}
% making logically defined graphics
%%%\usepackage{xypic}

% there are many more packages, add them here as you need them

% define commands here
\begin{document}
Let $K$ be a number field.  Let $\mathfrak{a}$ and $\mathfrak{b}$ be ideals in $\mathcal{O}_{K}$ (the ring of algebraic integers of $K$).  Define a relation $\sim$ on the ideals of $\mathcal{O}_{K}$ in the following way: write $\mathfrak{a} \sim \mathfrak{b}$ if there exist nonzero elements $\alpha$ and $\beta$ of $\mathcal{O}_K$ such that $(\alpha)\mathfrak{a}=(\beta)\mathfrak{b}$.

The relation $\sim$ is an equivalence relation, and the equivalence classes under $\sim$ are known as {\em ideal classes}.

The number of equivalence classes, denoted by $h$ or $h_K$, is called the {\em class number} of $K$.

Note that the set of ideals of any ring $R$ forms an abelian semigroup with the product of ideals as the semigroup operation.  By replacing ideals by ideal classes, it is possible to define a group on the ideal classes of $\mathcal{O}_{K}$ in the following way.

Let $\mathfrak{a}$, $\mathfrak{b}$ be ideals of $\mathcal{O}_{K}$.  Denote the ideal classes of which $\mathfrak{a}$ and $\mathfrak{b}$ are representatives by $[\mathfrak{a}]$ and $[\mathfrak{b}]$ respectively.  Then define $\cdot$ by \[ [\mathfrak{a}] \cdot [\mathfrak{b}]=[\mathfrak{a} \mathfrak{b}] \]

Let ${\cal C} = \{ [\mathfrak{a}] \mid \mathfrak{a} \neq (0), \mathfrak{a} \text{ an ideal of } \mathcal{O}_{K} \}$.
With the above definition of multiplication, $\cal C$ is an abelian group, called the {\em ideal class group} (or frequently just the \emph{class group}) of $K$.

Note that the ideal class group of $K$ is simply the quotient group of the ideal group of $K$ by the subgroup of principal fractional ideals.
%%%%%
%%%%%
\end{document}
