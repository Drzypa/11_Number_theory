\documentclass[12pt]{article}
\usepackage{pmmeta}
\pmcanonicalname{CriterionForAMultiplicativeFunctionToBeCompletelyMultiplicative}
\pmcreated{2013-03-22 15:58:44}
\pmmodified{2013-03-22 15:58:44}
\pmowner{Wkbj79}{1863}
\pmmodifier{Wkbj79}{1863}
\pmtitle{criterion for a multiplicative function to be completely multiplicative}
\pmrecord{9}{37995}
\pmprivacy{1}
\pmauthor{Wkbj79}{1863}
\pmtype{Theorem}
\pmcomment{trigger rebuild}
\pmclassification{msc}{11A25}
\pmrelated{FormulaForTheConvolutionInverseOfACompletelyMultiplicativeFunction}

\endmetadata

\usepackage{amssymb}
\usepackage{amsmath}
\usepackage{amsfonts}

\usepackage{psfrag}
\usepackage{graphicx}
\usepackage{amsthm}
%%\usepackage{xypic}

\newtheorem*{thm*}{Theorem}
\begin{document}
\begin{thm*}
Let $f$ be a multiplicative function with convolution inverse $g$.  Then $f$ is completely multiplicative if and only if $g(p^k)=0$ for all primes $p$ and for all $k \in \mathbb{N}$ with $k>1$.
\end{thm*}

\begin{proof}
Note first that, since $f(1)=1$ and $f*g=\varepsilon$, where $\varepsilon$ denotes the convolution identity function, then $g(1)=1$.  Let $p$ be any prime.  Then

$$0=\varepsilon(p)=(f*g)(p)=f(1)g(p)+f(p)g(1)=g(p)+f(p).$$

Thus, $g(p)=-f(p)$.

Assume that $f$ is completely multiplicative.  The statement about $g$ will be proven by induction on $k$.  Note that:

\begin{center}
$\begin{array}{ll}
0 & =\varepsilon(p^2) \\
& =(f*g)(p^2) \\
& =f(1)g(p^2)+f(p)g(p)+f(p^2)g(1) \\
& =g(p^2)+f(p)(-f(p))+(f(p))^2 \\
& =g(p^2) \end{array}$
\end{center}

Let $m \in \mathbb{N}$ with $m>2$ such that, for all $k \in \mathbb{N}$ with $1<k<m$, $g(p^k)=0$.  Then:

\begin{center}
$\begin{array}{ll}
0 & =\varepsilon(p^m) \\
& =(f*g)(p^m) \\
& =f(1)g(p^m)+f(p^{m-1})g(p)+f(p^m)g(1) \\
& =g(p^m)+(f(p))^{m-1}(-f(p))+(f(p))^m \\
& =g(p^m) \end{array}$
\end{center}

Conversely, assume that $g(p^k)=0$ for all $k \in \mathbb{N}$ with $k>1$.  The statement $f(p^k)=(f(p))^k$ will be proven by induction on $k$.  The statement is obvious for $k=1$.  Let $m \in \mathbb{N}$ such that $f(p^{m-1})=(f(p))^{m-1}$.  Then:

\begin{center}
$\begin{array}{ll}
0 & =\varepsilon(p^m) \\
& =(f*g)(p^m) \\
& =f(p^{m-1})g(p)+f(p^m)g(1) \\
& =(f(p))^{m-1}(-f(p))+f(p^m) \\
& =-(f(p))^m+f(p^m) \end{array}$
\end{center}

Thus, $f(p^m)=(f(p))^m$.  It follows that $f$ is completely multiplicative.
\end{proof}
%%%%%
%%%%%
\end{document}
