\documentclass[12pt]{article}
\usepackage{pmmeta}
\pmcanonicalname{ExamplesOfCharacteristicFunctions}
\pmcreated{2013-03-22 17:54:28}
\pmmodified{2013-03-22 17:54:28}
\pmowner{Wkbj79}{1863}
\pmmodifier{Wkbj79}{1863}
\pmtitle{examples of characteristic functions}
\pmrecord{11}{40401}
\pmprivacy{1}
\pmauthor{Wkbj79}{1863}
\pmtype{Example}
\pmcomment{trigger rebuild}
\pmclassification{msc}{11R04}
\pmclassification{msc}{11A25}
\pmclassification{msc}{26A09}
\pmclassification{msc}{26-00}

\endmetadata

\usepackage{amssymb}
\usepackage{amsmath}
\usepackage{amsfonts}
\usepackage{pstricks}
\usepackage{psfrag}
\usepackage{graphicx}
\usepackage{amsthm}
%%\usepackage{xypic}
\usepackage{tabls}
\newcommand{\ds}{\displaystyle}
\newcommand{\ipart}{\operatorname{Im}}
\newcommand{\norm}{\operatorname{N}}
\newcommand{\order}{\mathcal{O}}
\newcommand{\real}{\operatorname{Re}}
\begin{document}
\PMlinkescapephrase{closed form}

In this entry, we give some examples of characteristic functions written in closed form (\PMlinkname{i.e.}{Ie}, not piecewise defined).

We use the following conventions:

\begin{itemize}
\item $X$ denotes the \PMlinkname{domain}{Domain} of the characteristic function
\item $A$ denotes a subset of $X$
\item $\chi_A$ denotes the characteristic function of $A$
\item $|\cdot|$ denotes distance from $0$ (\PMlinkname{e.g.}{Eg} absolute value or \PMlinkname{modulus}{Modulus2})
\item $\lfloor\cdot\rfloor$ denotes the floor function
\item $\lceil\cdot\rceil$ denotes the ceiling function
\item $\real$ denotes the real part of a complex number
\item $\ipart$ denotes the imaginary part of a complex number
\item $\mathbb{N}$ denotes the positive integers
\item $\mu$ denotes the M\"{o}bius function
\item $\varphi$ denotes the Euler totient function
\item $\order_K$ denotes the ring of integers of a number field $K$
\item $\norm_K$ denotes the norm on a number field $K$
\end{itemize}

Below is a \PMlinkescapetext{chart} of characteristic functions:

\begin{center}
\begin{tabular}{|c|c|c|}
\hline
$X$ & $A$ & $\chi_A(x)$ \\
\hline \hline
$\mathbb{Z}$ & evens & $\ds \frac{1+(-1)^x}{2}$ \\
\hline
$\mathbb{Z}$ & \PMlinkname{odds}{OddNumber} & $\ds \frac{1-(-1)^x}{2}$ \\
\hline
$\mathbb{R}$ & positive reals & $\ds \frac{|x|+x}{2}$ \\
\hline
$\mathbb{R}$ & negative reals & $\ds \frac{|x|-x}{2}$ \\
\hline
$\mathbb{R}$ & $\mathbb{Z}$ & $\lfloor \cos^2(\pi x) \rfloor$ \\
\hline
$\mathbb{R}$ & $\mathbb{R}\setminus\mathbb{Z}$ & $\lceil \sin^2(\pi x) \rceil$ \\
\hline
$\mathbb{C}\setminus\{0\}$ & $\mathbb{R}\setminus\{0\}$ & $\ds \left\lfloor \frac{\real{x}}{|x|} \right\rfloor$ \\
\hline
$\mathbb{C}\setminus\{0\}$ & $\mathbb{C}\setminus\mathbb{R}$ & $\ds \left\lceil \frac{\ipart{x}}{|x|} \right\rceil$ \\
\hline
$\mathbb{N}$ & positive squarefree integers & $\mu^2(x)$ \\
\hline
$\mathbb{N}\setminus\{1\}$ & primes & $\ds \left\lfloor \frac{\varphi(x)+1}{x} \right\rfloor$ \\
\hline
$\order_K\setminus\{0\}$ & units of $\order_K$ & $\ds \left\lfloor \frac{1}{|\norm_K(x)|} \right\rfloor$ \\
\hline
\end{tabular}
\end{center}
%%%%%
%%%%%
\end{document}
