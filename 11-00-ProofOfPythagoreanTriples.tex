\documentclass[12pt]{article}
\usepackage{pmmeta}
\pmcanonicalname{ProofOfPythagoreanTriples}
\pmcreated{2013-03-22 14:28:05}
\pmmodified{2013-03-22 14:28:05}
\pmowner{fredlb}{5992}
\pmmodifier{fredlb}{5992}
\pmtitle{proof of Pythagorean triples}
\pmrecord{9}{35989}
\pmprivacy{1}
\pmauthor{fredlb}{5992}
\pmtype{Proof}
\pmcomment{trigger rebuild}
\pmclassification{msc}{11-00}

% this is the default PlanetMath preamble.  as your knowledge
% of TeX increases, you will probably want to edit this, but
% it should be fine as is for beginners.

% almost certainly you want these
\usepackage{amssymb}
\usepackage{amsmath}
\usepackage{amsfonts}

% used for TeXing text within eps files
%\usepackage{psfrag}
% need this for including graphics (\includegraphics)
%\usepackage{graphicx}
% for neatly defining theorems and propositions
%\usepackage{amsthm}
% making logically defined graphics
%%%\usepackage{xypic}

% there are many more packages, add them here as you need them

% define commands here
\begin{document}
If $a,\,b$, and $c$ are positive integers such that
\begin{equation}
\label{eq:p1}
    a^2+b^2=c^2
\end{equation}
then $(a,b,c)$ is a Pythagorean triple. If $a,\,b$, and $c$ are
relatively prime in pairs then $(a,b,c)$ is a primitive
Pythagorean triple. Clearly, if $k$ divides any two of $a,\,b$,
and $c$ it divides all three. And if $a^2+b^2=c^2$ then
$k^2a^2+k^2b^2=k^2c^2$. That is, for a positive integer $k$, if
$(a,b,c)$ is a Pythagorean triple then so is $(ka,kb,kc)$.
Hence, to find all Pythagorean triples, it's sufficient to find
all primitive Pythagorean triples.

Let $a, b$, and $c$ be relatively prime positive integers such
that $a^2+b^2=c^2$. Set

$$\frac{m}{n}=\frac{a+c}{b}$$ reduced to
lowest terms, That is, $\gcd(m,n)=1$. From the triangle inequality
$m > n$. Then
\begin{equation}
    \frac{m}{n}\,b-a=c.
\label{eq:p2}
\end{equation}
Squaring both sides of (\ref{eq:p2}) and multiplying through by $n^2$
we get
\[
    m^2b^2-2mnab+n^2a^2=n^2a^2+n^2b^2;
\]
which, after cancelling and rearranging terms, becomes
\begin{equation}
    b\left(m^2-n^2\right)=a(2mn).
\label{eq:p3}
\end{equation}
There are two cases, either $m$ and $n$ are of opposite parity, or
they or both odd. Since $\gcd(m,n)=1$, they can not both be
even.\\[3pt]


 \textbf{Case 1.}     $m$ and $n$ of opposite parity, i.e., $m
\not\equiv \pm n (\, mod \,\,2\, )$. So 2 divides b since
$m^2-n^2$ is odd. From equation (\ref{eq:p2}), $n$ divides $b$.
Since $\gcd(m,n)=1$ then $\gcd(m,m^2-n^2)=1$, therefore $m$ also
divides $b$. And since $\gcd(a,b)=1$, $b$ divides $2mn$. Therefore
$b=2mn$. Then

\begin{equation}
    a=m^2-n^2,  \quad b=2mn,\quad \mbox{and from
    (\ref{eq:p2})},\,\, c=\frac{m}{n}\,2mn-(m^2-n^2)=m^2+n^2.
\label{eq:p4}
\end{equation}

\textbf{Case 2.} $m$ and $n$ both odd, i.e., $m \equiv \pm n (\,
mod \,\,2\, )$. So 2 divides $m^2-n^2$. Then by the same process
as in the first case we have
\begin{equation}
    a=\frac{m^2-n^2}{2}, \quad b=mn,\quad and \quad
    c=\frac{m^2+n^2}{2}.
\label{eq:p5}
\end{equation}

The parametric equations in (\ref{eq:p4}) and (\ref{eq:p5}) appear
to be different but they generate the same solutions. To show
this, let
\[
    u=\frac{m+n}{2}\,\, \mbox{ and } \,\,v=\frac{m-n}{2}\,.
\]
Then $m=u+v$, and $n=u-v$. Substituting those values for $m$ and
$n$ into (\ref{eq:p5}) we get
\begin{equation}
  a=2uv, \,\,\,  b=u^2-v^2, \quad \mbox{and} \quad c=u^2+v^2
\label{eq:p6}
\end{equation}
where $u > v$, $gcd(u,v)=1$, and $u$ and $v$ are of opposite
parity. Therefore (\ref{eq:p6}), with  a and b
interchanged, is identical to (\ref{eq:p4}). Thus since
$\left(m^2-n^2,2mn,m^2+n^2\right)$, as in (\ref{eq:p4}), is a
primitive Pythagorean triple, we can say that $(a,b,c)$ is a
primitive pythagorean triple if and only if there exists
relatively prime, positive integers $m$ and $n$, $m>n$, such that
$a=m^2-n^2,\,\, b=2mn,\,\,\mbox{ and } \,\,\, c=m^2+n^2$\,.
%%%%%
%%%%%
\end{document}
