\documentclass[12pt]{article}
\usepackage{pmmeta}
\pmcanonicalname{Totient}
\pmcreated{2013-03-22 13:38:35}
\pmmodified{2013-03-22 13:38:35}
\pmowner{mathcam}{2727}
\pmmodifier{mathcam}{2727}
\pmtitle{totient}
\pmrecord{5}{34293}
\pmprivacy{1}
\pmauthor{mathcam}{2727}
\pmtype{Definition}
\pmcomment{trigger rebuild}
\pmclassification{msc}{11A25}
\pmdefines{totient}
\pmdefines{Jordan totient}

\endmetadata

\usepackage{amssymb}
\usepackage{amsmath}
\usepackage{amsfonts}
\begin{document}
\PMlinkescapeword{term}
\PMlinkescapeword{convolution}
A \emph{totient} is a sequence $f:{\{1,2,3,\ldots\}}\to {\mathbb C}$ such
that $$g\ast f=h$$
for some two completely multiplicative sequences $g$ and $h$, where $\ast$
denotes the convolution product (or Dirichlet product; see multiplicative function).

The term `totient' was introduced by Sylvester in the 1880's, but is
seldom used nowadays except in two cases. The Euler totient $\phi$
satisfies
$$\iota_0\ast\phi = \iota_1$$
where $\iota_k$ denotes the function $n\mapsto n^k$ (which is completely
multiplicative). The more general \emph{Jordan totient} $J_k$ is defined by $$\iota_0\ast J_k=\iota_k.$$
%%%%%
%%%%%
\end{document}
