\documentclass[12pt]{article}
\usepackage{pmmeta}
\pmcanonicalname{ExamplesOf1automorphicNumbers}
\pmcreated{2013-03-22 16:20:20}
\pmmodified{2013-03-22 16:20:20}
\pmowner{PrimeFan}{13766}
\pmmodifier{PrimeFan}{13766}
\pmtitle{examples of 1-automorphic numbers}
\pmrecord{5}{38470}
\pmprivacy{1}
\pmauthor{PrimeFan}{13766}
\pmtype{Example}
\pmcomment{trigger rebuild}
\pmclassification{msc}{11A63}

% this is the default PlanetMath preamble.  as your knowledge
% of TeX increases, you will probably want to edit this, but
% it should be fine as is for beginners.

% almost certainly you want these
\usepackage{amssymb}
\usepackage{amsmath}
\usepackage{amsfonts}

% used for TeXing text within eps files
%\usepackage{psfrag}
% need this for including graphics (\includegraphics)
%\usepackage{graphicx}
% for neatly defining theorems and propositions
%\usepackage{amsthm}
% making logically defined graphics
%%%\usepackage{xypic}

% there are many more packages, add them here as you need them

% define commands here

\begin{document}
Concerning ourselves only with searching for automorphic numbers $n$ in bases $1 < b < 17$ (binary to hexadecimal) and the ranges given by the iterator $0 < i < b^3 - 1$, and limiting to $m = 1$ we find the following results:

First, it is obvious that 1 is a 1-automorphic number regardless of the base.

For the range and limit specified, there are no other 1-automorphic numbers in binary through quinary, bases 7 through 9, 11, 13 and 16.

In base 6, there are 1, 3, 4, 9, 28, 81, 136, and it is easy to verify that ${3_6}^2 = 13_6$, ${4_6}^2 = 24_6$, ${13_6}^2 = 213_6$, ${44_6}^2 = 3344_6$, etc.

In base 10, these ought to look familiar: 1, 5, 6, 25, 76, 376, 625.

Duodecimal: 1, 4, 9, 64, 81, 513, 1216. Noticing that 4 also appears in the list for base 6, we might wonder if 4 is always 1-automorphic when $6|b$? The question is moot because the next multiple of 6 is $18 > 4^2$, thus in base 18 and any other higher bases, the square of 4 is also a 1-digit number.

Base 14: 1, 7, 8, 49, 148, 344, 2401.

Base 15: 1, 6, 10, 100, 126, 1000, 2376. Base 15 is the smallest odd base $b$ to have 1-automorphic numbers in the range specified. This should not be taken to imply that it is the smallest odd base to have automorphic numbers at all.
%%%%%
%%%%%
\end{document}
