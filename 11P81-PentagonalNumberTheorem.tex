\documentclass[12pt]{article}
\usepackage{pmmeta}
\pmcanonicalname{PentagonalNumberTheorem}
\pmcreated{2013-03-22 13:57:51}
\pmmodified{2013-03-22 13:57:51}
\pmowner{bbukh}{348}
\pmmodifier{bbukh}{348}
\pmtitle{pentagonal number theorem}
\pmrecord{7}{34733}
\pmprivacy{1}
\pmauthor{bbukh}{348}
\pmtype{Theorem}
\pmcomment{trigger rebuild}
\pmclassification{msc}{11P81}
\pmclassification{msc}{14K25}

\usepackage{amssymb}
\usepackage{amsmath}
\usepackage{amsfonts}
\begin{document}
\PMlinkescapeword{name}
\PMlinkescapeword{theory}
\PMlinkescapeword{connection}
\PMlinkescapeword{maps}
\PMlinkescapeword{fix}
\newcommand{\Z}{\mathbb{Z}}
\newcommand{\Nplus}{\mathbb{N}^+}
\newcommand{\gbar}{\overline{g}}
\textbf{Theorem :}
\begin{equation} \label{eq:a}
\prod_{k=1}^{\infty}(1-x^k)=\sum_{n=-\infty}^{\infty}(-1)^nx^{n(3n+1)/2}
\end{equation}
where the two sides are regarded as formal power series over $\Z$.

\textbf{Proof: }For $n\ge 0$, denote by $f(n)$ the coefficient of
$x^n$ in the product on the left, i.e. write
$$\prod_{k=1}^{\infty}(1-x^k)=\sum_{n=0}^{\infty}f(n)x^n\;.$$
By this definition, we have for all $n$
$$f(n)=e(n)-d(n)$$
where $e(n)$ (resp. $d(n)$) is the number of partitions of $n$ as
a sum of an even (resp. odd) number of distinct summands.
To fix the notation, let $P(n)$ be set of pairs $(s,g)$ where $s$ is
a natural number $>0$ and $g$ is a decreasing mapping
$\{1,2,\ldots,s\}\to\Nplus$ such that $\sum_xg(x)=n$.
The cardinal of $P(n)$ is thus $f(n)$, and $P(n)$ is the union of
these two disjoint sets:
$$E(n)=\{(s,g)\in P(n)\mid s\text{ is even}\},$$
$$D(n)=\{(s,g)\in P(n)\mid s\text{ is odd}\}.$$

Now on the right side of \eqref{eq:a} we have
$$1+\sum_{n=1}^{\infty}(-1)^nx^{n(3n+1)/2}+
     \sum_{n=1}^{\infty}(-1)^nx^{n(3n-1)/2}\;.$$
Therefore what we want to prove is
\begin{eqnarray}
\label{eq:special}
e(n)&=&d(n)+(-1)^m\qquad\text{if }n=m(3m\pm 1)/2\text{ for some }m \\
\label{eq:normal}
e(n)&=&d(n)\qquad\text{otherwise.}
\end{eqnarray}
For $m\ge 1$ we have
\begin{eqnarray}
\label{eq:highpent}
m(3m+1)/2 &=& 2m+(2m-1)+\ldots+(m+1) \\
\label{eq:lowpent}
m(3m-1)/2 &=& (2m-1)+(2m-2)+\ldots+m
\end{eqnarray}
Take some $(s,g)\in P(n)$, and suppose first that $n$ is not
of the form~\eqref{eq:highpent} nor~\eqref{eq:lowpent}.
Since $g$ is decreasing, there is a unique integer
$k\in[1,s]$ such that
\begin{align*}
g(j)&=g(1)-j+1\qquad\text{for }j\in[1,k],
g(j)&<g(1)-j+1\qquad\text{for }j\in[k+1,s].
\end{align*}
If $g(s)\le k$, define $\gbar\colon [1,s-1]\to\Nplus$ by
$$\gbar(x)=
\begin{cases}
g(x)+1, &\text{if }x\in[1,g(s)], \\
g(x),   &\text{if }x\in[g(s)+1,s-1].
\end{cases}$$
If $g(s)>k$, define $\gbar\colon [1,s+1]\to\Nplus$ by
$$\gbar(x)=
\begin{cases}
g(x)-1, &\text{if }x\in[1,k], \\
g(x),   &\text{if }x\in[k+1,s],\\
k,      &\text{ if }x=s+1.
\end{cases}$$
In both cases, $\gbar$ is decreasing and $\sum_x\gbar(x)=n$.
The mapping $g\to\gbar$ maps takes an element having odd $s$
to an element having even $s$, and vice versa.
Finally, the reader can verify that $\overline{\overline{g}}=g$.
Thus we have constructed a bijection $E(n)\to D(n)$,
proving \eqref{eq:normal}.

Now suppose that $n=m(3m+1)/2$ for some (perforce unique) $m$.
The above construction still yields a bijection between
$E(n)$ and $D(n)$ excluding (from one set or the other) the single
element $(m,g_0)$:
$$g_0(x)=2m+1-x\qquad\text{for }x\in[1,m]$$
as in \eqref{eq:highpent}.
Likewise if $n=m(3m-1)/2$, only this element $(m,g_1)$ is excluded:
$$g_1(x)=2m-x\qquad\text{for }x\in[1,m]$$
as in \eqref{eq:lowpent}.
In both cases we deduce \eqref{eq:special}, completing the proof.

\textbf{Remarks: }The name of the theorem derives from the fact that
the exponents $n(3n+1)/2$ are the generalized pentagonal numbers.

The theorem was discovered and proved by Euler around 1750.
This was one of the first results about what are now called theta
functions, and was also one of the earliest applications of
the formalism of generating functions.

The above proof is due to F. Franklin,
(\emph{Comptes Rendus de l'Acad. des Sciences},
\textbf{92}, 1881, pp. 448-450).
%%%%%
%%%%%
\end{document}
