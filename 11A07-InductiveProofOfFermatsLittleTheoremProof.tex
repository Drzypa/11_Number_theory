\documentclass[12pt]{article}
\usepackage{pmmeta}
\pmcanonicalname{InductiveProofOfFermatsLittleTheoremProof}
\pmcreated{2013-03-22 11:47:46}
\pmmodified{2013-03-22 11:47:46}
\pmowner{mathcam}{2727}
\pmmodifier{mathcam}{2727}
\pmtitle{inductive proof of Fermat's little theorem proof}
\pmrecord{17}{30312}
\pmprivacy{1}
\pmauthor{mathcam}{2727}
\pmtype{Proof}
\pmcomment{trigger rebuild}
\pmclassification{msc}{11A07}
\pmrelated{FermatsTheoremProof}

\usepackage{amssymb}
\usepackage{amsmath}
\usepackage{amsfonts}
\usepackage{graphicx}
%%%%\usepackage{xypic}
\newcommand{\binomial}[2]{{{#1}\choose{#2}}}
\begin{document}
We will show $$a^{p} \equiv a \pmod{p}$$ with $p$ prime. The equivalent statement $$a^{p-1}\equiv 1 \pmod{p}$$ when $p$ does not divide $a$ follows by cancelling $a$ both sides (which can be done since then $a,p$ are coprime).

When $a=1$, we have $$ 1^{p} \equiv 1 \pmod{p}$$  

Now assume the theorem holds for some positive $a$ and we want to prove the statement for $a+1$. We will have as a direct consequence that 
$$a^p \equiv a \pmod{p}$$

Let's examine $a+1$.  By the binomial theorem, we have
\begin{eqnarray*}
(a+1)^{p} & \equiv & \binomial{p}{0}a^p + \binomial{p}{1}a^{p-1} + \cdots + \binomial{p}{p-1} a + 1  \\
& \equiv & a + pa^{p-1} + p\frac{(p-1)}{2}a^{p-2} + \cdots + p a + 1 \\
& \equiv & (a + 1) + [ pa^{p-1} + p \frac{(p-1)}{2}a^{p-2} + \cdots + p a ] 
\end{eqnarray*}

However, note that the entire bracketed term is divisible by $p$, since each element of it is divsible by $p$.  Hence
$$(a+1)^p \equiv (a+1) \pmod{p}$$

Therefore by induction it follows that
\[a^p\equiv a \pmod{p}\]
for all positive integers $a$.

It is easy to show that it also holds for $-a$ whenever it holds for $a$, so the statement works for all integers $a$.
%%%%%
%%%%%
%%%%%
%%%%%
\end{document}
