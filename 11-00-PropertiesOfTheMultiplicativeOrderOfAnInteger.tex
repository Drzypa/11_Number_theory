\documentclass[12pt]{article}
\usepackage{pmmeta}
\pmcanonicalname{PropertiesOfTheMultiplicativeOrderOfAnInteger}
\pmcreated{2013-03-22 16:20:44}
\pmmodified{2013-03-22 16:20:44}
\pmowner{alozano}{2414}
\pmmodifier{alozano}{2414}
\pmtitle{properties of the multiplicative order of an integer}
\pmrecord{4}{38478}
\pmprivacy{1}
\pmauthor{alozano}{2414}
\pmtype{Theorem}
\pmcomment{trigger rebuild}
\pmclassification{msc}{11-00}
\pmclassification{msc}{13M05}
\pmclassification{msc}{13-00}

\endmetadata

% this is the default PlanetMath preamble.  as your knowledge
% of TeX increases, you will probably want to edit this, but
% it should be fine as is for beginners.

% almost certainly you want these
\usepackage{amssymb}
\usepackage{amsmath}
\usepackage{amsthm}
\usepackage{amsfonts}

% used for TeXing text within eps files
%\usepackage{psfrag}
% need this for including graphics (\includegraphics)
%\usepackage{graphicx}
% for neatly defining theorems and propositions
%\usepackage{amsthm}
% making logically defined graphics
%%%\usepackage{xypic}

% there are many more packages, add them here as you need them

% define commands here

\newtheorem{thm}{Theorem}
\newtheorem*{defn}{Definition}
\newtheorem*{prop}{Proposition}
\newtheorem{lemma}{Lemma}
\newtheorem{cor}{Corollary}

\theoremstyle{definition}
\newtheorem{exa}{Example}

% Some sets
\newcommand{\Nats}{\mathbb{N}}
\newcommand{\Ints}{\mathbb{Z}}
\newcommand{\Reals}{\mathbb{R}}
\newcommand{\Complex}{\mathbb{C}}
\newcommand{\Rats}{\mathbb{Q}}
\newcommand{\Gal}{\operatorname{Gal}}
\newcommand{\Cl}{\operatorname{Cl}}
\newcommand{\ord}{\operatorname{ord}}
\begin{document}
\begin{defn}
Let $m>1$ be an integer and let $a$ be another integer relatively prime to $m$. The order of $a$ modulo $m$ (or the multiplicative order of $a \mod m$) is the smallest positive integer $n$ such that $a^n\equiv 1 \mod m$. The order is sometimes denoted by $\operatorname{ord} a$ or $\operatorname{ord}_m a$.
\end{defn}

\begin{prop}
Let $m$ be a positive integer and suppose that $(a,m)=1$.
\begin{enumerate}
\item $a^s\equiv 1 \mod m$ if and only if $\ord a$ divides $s$. In particular, $\ord a$ divides $\phi(m)$, where $\phi$ is the Euler phi function.

\item $a^s\equiv a^t \mod m$ if and only if $s\equiv t \mod \ord a$.

\item If $\ord a =d$ then $\displaystyle \ord a^k =\frac{d}{\gcd(k,d)}$ for any $k\geq 1$.

\item If $\ord a =d$ and $e$ is a positive divisor of $d$ then $a^{d/e}$ has exact order $e$. 

\item Suppose $\ord a=h$ and $\ord b = k$ with $\gcd(h,k)=1$. Then $\ord (ab)=hk$.
\end{enumerate}
\end{prop}
%%%%%
%%%%%
\end{document}
