\documentclass[12pt]{article}
\usepackage{pmmeta}
\pmcanonicalname{NonmultiplicativeFunction}
\pmcreated{2013-03-22 12:47:04}
\pmmodified{2013-03-22 12:47:04}
\pmowner{Mathprof}{13753}
\pmmodifier{Mathprof}{13753}
\pmtitle{non-multiplicative function}
\pmrecord{16}{33098}
\pmprivacy{1}
\pmauthor{Mathprof}{13753}
\pmtype{Example}
\pmcomment{trigger rebuild}
\pmclassification{msc}{11A25}
%\pmkeywords{number theory}
%\pmkeywords{arithmetic function}
%\pmkeywords{multiplicative function}
\pmrelated{PartitionFunction2}
\pmdefines{partition function}

\endmetadata

% this is the default PlanetMath preamble.  as your knowledge
% of TeX increases, you will probably want to edit this, but
% it should be fine as is for beginners.

% almost certainly you want these
\usepackage{amssymb}
\usepackage{amsmath}
\usepackage{amsfonts}

% used for TeXing text within eps files
%\usepackage{psfrag}
% need this for including graphics (\includegraphics)
%\usepackage{graphicx}
% for neatly defining theorems and propositions
%\usepackage{amsthm}
% making logically defined graphics
%%%\usepackage{xypic}

% there are many more packages, add them here as you need them

% define commands here
\begin{document}
\PMlinkescapeword{between}
\PMlinkescapeword{order}
\PMlinkescapeword{orders}
\PMlinkescapeword{representations}

In  number theory, a {\it non-multiplicative function} is an arithmetic function  which is not multiplicative.

\paragraph{Examples}

Some examples of a non-multiplicative functions are the arithmetic functions:

\begin{itemize}
\item $r_{2}(n)$ - the number of unordered representations of $n$ as a sum of squares of two integers, positive, negative or zero,
\item $c_{4}(n)$ - the number of ways that $n$ can be expressed as the sum of four squares of nonnegative integers, where we distinguish between different orders of the summands. For example:
$$ 1 = 1^{2}+0^{2}+0^{2}+0^{2} = 0^{2}+1^{2}+0^{2}+0^{2}+0^{2} = 0^{2}+0^{2}+1^{2}+0^{2} = 0^{2} + 0^{2} + 0^{2} + 1^{2} \; ,$$
hence $c_{4}(1)=4 \ne 1 \; .$
\item The {\it partition function} $P(n)$ - the number of ordered representations of $n$ as a sum of positive integers. For instance:
$$ P(2 \cdot 5) = P(10) = 42 \quad \hbox{and} $$
$$ P(2) P(5) = 2 \cdot 7 = 14 \ne 42 \; . $$
\item The prime counting function $\pi(n)$. Here we first have $\pi(1) = 0 \ne 1$ and then we have as yet for example: 
$$ \pi(2 \cdot 5) = \pi(10) = 4 \quad \hbox{and} $$
$$ \pi(2) \pi(5) = 1 \cdot 3 = 3 \ne 4 \; . $$
\item The Mangoldt function $\Lambda(n)$. $\Lambda(1) = \ln 1 \ne 1$ and for example: 
$$ \Lambda(2 \cdot 5) = \Lambda(10) = 0 \quad \hbox{and} $$
$$ \Lambda(2) \Lambda(5) = \ln 2 \cdot \ln 5 \ne 0 \; . $$
We would think that for some $n$ multiplicativity of $\Lambda(n)$ would be true as in:
$$ \Lambda(2 \cdot 6) = \Lambda(12) = 0 \quad \hbox{and} $$
$$ \Lambda(2) \Lambda(6) = \ln 2 \cdot 0 = 0 \; , $$
but we have to write:
$$ \Lambda(2^{2}) \Lambda(3) = \ln 2 \cdot \ln 3 \ne 0 \; . $$
\end{itemize}
%%%%%
%%%%%
\end{document}
