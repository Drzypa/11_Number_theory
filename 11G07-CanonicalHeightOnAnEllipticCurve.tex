\documentclass[12pt]{article}
\usepackage{pmmeta}
\pmcanonicalname{CanonicalHeightOnAnEllipticCurve}
\pmcreated{2013-03-22 16:23:20}
\pmmodified{2013-03-22 16:23:20}
\pmowner{alozano}{2414}
\pmmodifier{alozano}{2414}
\pmtitle{canonical height on an elliptic curve}
\pmrecord{6}{38534}
\pmprivacy{1}
\pmauthor{alozano}{2414}
\pmtype{Definition}
\pmcomment{trigger rebuild}
\pmclassification{msc}{11G07}
\pmclassification{msc}{11G05}
\pmclassification{msc}{14H52}
\pmsynonym{Neron-Tate height}{CanonicalHeightOnAnEllipticCurve}
\pmrelated{HeightFunction}
\pmrelated{RegulatorOfAnEllipticCurve}
\pmdefines{canonical height}

% this is the default PlanetMath preamble.  as your knowledge
% of TeX increases, you will probably want to edit this, but
% it should be fine as is for beginners.

% almost certainly you want these
\usepackage{amssymb}
\usepackage{amsmath}
\usepackage{amsthm}
\usepackage{amsfonts}

% used for TeXing text within eps files
%\usepackage{psfrag}
% need this for including graphics (\includegraphics)
%\usepackage{graphicx}
% for neatly defining theorems and propositions
%\usepackage{amsthm}
% making logically defined graphics
%%%\usepackage{xypic}

% there are many more packages, add them here as you need them

% define commands here

\newtheorem*{thm}{Theorem}
\newtheorem*{defn}{Definition}
\newtheorem{prop}{Proposition}
\newtheorem{lemma}{Lemma}
\newtheorem{cor}{Corollary}

\theoremstyle{definition}
\newtheorem{exa}{Example}

% Some sets
\newcommand{\Nats}{\mathbb{N}}
\newcommand{\Ints}{\mathbb{Z}}
\newcommand{\Reals}{\mathbb{R}}
\newcommand{\Complex}{\mathbb{C}}
\newcommand{\Rats}{\mathbb{Q}}
\newcommand{\Gal}{\operatorname{Gal}}
\newcommand{\Cl}{\operatorname{Cl}}
\begin{document}
Let $E/\Rats$ be an elliptic curve. It is often useful to have a notion of {\it height} of a point, in order to talk about the arithmetic complexity of a point $P$ in $E(\Rats)$. For this, one defines height functions. For example, in $\Rats$ one can define a height by 
$$H(p/q)=max(|p|,|q|).$$
Following the example of $\Rats$, one may define a height on $E/\Rats$ by
$$h_x(P)=\begin{cases}
\log H(x(P)) & \text{if } P\neq O\\
0 & \text{if } P=O.
\end{cases}
$$
In fact, given any even function $f:E(\Rats)\to \Reals$ on $E(\Rats)$ (i.e. $f(P)=f(-P)$ for any $P\in E(\Rats)$) one can define a height by:
$$h_f(P)=\log H(f(P)).$$
However, one can refine this definition so that the height function satisfies some very nice properties (see below).

\begin{defn}
Let $\Rats$ be a number field and let $E$ be an elliptic curve defined over $\Rats$. The canonical height (or N\'eron-Tate height) on $E/\Rats$, denoted by $\hat{h}$, is the function on $E(\Rats)$ (with real values) defined by:
$$\hat{h}(P)=\frac{1}{\deg f} \lim_{N\to \infty} \frac{h_f([2^N]P)}{4^N}$$
for any even function $f:E(\Rats)\to \Reals$.
\end{defn}

The fact that the definition does not depend on the choice of even function $f$ is due to J. Tate. In particular, one can simply choose $f$ to be the $x$-function, whose degree is $2$. The canonical height satisfies the following properties:

\begin{thm}
Let $E/\Rats$ and let $\hat{h}$ be the canonical height on $E$. Then:
\begin{enumerate}
\item The height $\hat{h}$ satisfies the parallelogram law:
$$\hat{h}(P+Q)+\hat{h}(P-Q)=2\hat{h}(P)+2\hat{h}(Q)$$
for all $P,Q \in E(\overline{\Rats})$.

\item For all $m\in \Ints$ and all $P\in E(\overline{\Rats})$:
$$\hat{h}([m]P)=m^2\hat{h}(P).$$

\item The height $\hat{h}$ is even and the pairing:
$$\langle \cdot, \cdot \rangle : E(\overline{\Rats})\times E(\overline{\Rats}) \to \Reals,\quad \langle P,Q \rangle = \hat{h}(P+Q)-\hat{h}(P)-\hat{h}(Q)$$
is bilinear (usually called the N\'eron-Tate pairing on $E/\Rats$).

\item For all $P\in E(\overline{\Rats})$ one has $\hat{h}(P)\geq 0$ and $\hat{h}(P)=0$ if and only if $P$ is a torsion point.
\end{enumerate}
\end{thm}

%%%%%
%%%%%
\end{document}
