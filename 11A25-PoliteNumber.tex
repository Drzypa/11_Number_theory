\documentclass[12pt]{article}
\usepackage{pmmeta}
\pmcanonicalname{PoliteNumber}
\pmcreated{2013-03-22 18:09:54}
\pmmodified{2013-03-22 18:09:54}
\pmowner{PrimeFan}{13766}
\pmmodifier{PrimeFan}{13766}
\pmtitle{polite number}
\pmrecord{6}{40725}
\pmprivacy{1}
\pmauthor{PrimeFan}{13766}
\pmtype{Definition}
\pmcomment{trigger rebuild}
\pmclassification{msc}{11A25}

\endmetadata

% this is the default PlanetMath preamble.  as your knowledge
% of TeX increases, you will probably want to edit this, but
% it should be fine as is for beginners.

% almost certainly you want these
\usepackage{amssymb}
\usepackage{amsmath}
\usepackage{amsfonts}

% used for TeXing text within eps files
%\usepackage{psfrag}
% need this for including graphics (\includegraphics)
%\usepackage{graphicx}
% for neatly defining theorems and propositions
%\usepackage{amsthm}
% making logically defined graphics
%%%\usepackage{xypic}

% there are many more packages, add them here as you need them

% define commands here

\begin{document}
A {\em polite number} $n$ is an integer that is the sum of two or more consecutive nonnegative integers in at least one way. To put it algebraically, if $n$ is polite then there is a solution to $$n = \sum_{i = a}^b i$$ with $b > a$ and $a > -1$. For example, 42 is a polite number since it is the sum of the integers from 3 to 9. The first few polite numbers are 3, 5, 6, 7, 9, 10, 11, 12, 13, 14, 15, 17, 18, 19, 20, 21, 22, 23, 24, 25, 26, 27, 28, 29, 30, 31, 33, 34, 35, 36, 37, 38, 39, 40, etc.

Obviously all triangular numbers are polite numbers. So are all odd numbers. In fact, the numbers that are not polite are the powers of 2.
%%%%%
%%%%%
\end{document}
