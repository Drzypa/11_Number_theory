\documentclass[12pt]{article}
\usepackage{pmmeta}
\pmcanonicalname{UntouchableNumber}
\pmcreated{2013-03-22 15:51:52}
\pmmodified{2013-03-22 15:51:52}
\pmowner{PrimeFan}{13766}
\pmmodifier{PrimeFan}{13766}
\pmtitle{untouchable number}
\pmrecord{8}{37855}
\pmprivacy{1}
\pmauthor{PrimeFan}{13766}
\pmtype{Definition}
\pmcomment{trigger rebuild}
\pmclassification{msc}{11A05}

\endmetadata

% this is the default PlanetMath preamble.  as your knowledge
% of TeX increases, you will probably want to edit this, but
% it should be fine as is for beginners.

% almost certainly you want these
\usepackage{amssymb}
\usepackage{amsmath}
\usepackage{amsfonts}

% used for TeXing text within eps files
%\usepackage{psfrag}
% need this for including graphics (\includegraphics)
%\usepackage{graphicx}
% for neatly defining theorems and propositions
%\usepackage{amsthm}
% making logically defined graphics
%%%\usepackage{xypic}

% there are many more packages, add them here as you need them

% define commands here
\begin{document}
An {\em untouchable number} is an integer $n$ for which there exists no integer $m$ such that \[ \left( \sum_{d\vert m} d \right) - m = n, \], thus $n$ can't be "touched" by the sum of proper divisors of any other integer. Paul Erd\H{o}s proved that there are infinitely many untouchable numbers.

Obviously no perfect number can be an untouchable number. Neither can any integer of the form $p + 1$, where $p$ is a prime number. What is not so obvious is whether 5 is the only odd untouchable number, and the related question of whether 2 and 5 are the only prime untouchable numbers.

\begin{thebibliography}{1}
\bibitem{cite:E}
P. Erd\H{o}s, \"Uber die Zahlen der Form $\sigma(n)-n$ und $n-\phi(n)$.
\emph{Elem. Math.} {\bf 28} (1973), 83--86.
\end{thebibliography}
%%%%%
%%%%%
\end{document}
