\documentclass[12pt]{article}
\usepackage{pmmeta}
\pmcanonicalname{HexagonalNumber}
\pmcreated{2013-03-22 17:50:51}
\pmmodified{2013-03-22 17:50:51}
\pmowner{PrimeFan}{13766}
\pmmodifier{PrimeFan}{13766}
\pmtitle{hexagonal number}
\pmrecord{4}{40319}
\pmprivacy{1}
\pmauthor{PrimeFan}{13766}
\pmtype{Definition}
\pmcomment{trigger rebuild}
\pmclassification{msc}{11D09}

\endmetadata

% this is the default PlanetMath preamble.  as your knowledge
% of TeX increases, you will probably want to edit this, but
% it should be fine as is for beginners.

% almost certainly you want these
\usepackage{amssymb}
\usepackage{amsmath}
\usepackage{amsfonts}

% used for TeXing text within eps files
%\usepackage{psfrag}
% need this for including graphics (\includegraphics)
%\usepackage{graphicx}
% for neatly defining theorems and propositions
%\usepackage{amsthm}
% making logically defined graphics
%%%\usepackage{xypic}

% there are many more packages, add them here as you need them

% define commands here

\begin{document}
A {\em hexagonal number} is a figurate number that represents a hexagon. The hexagonal number for $n$ is given by the formula $n(2n - 1)$. The first few hexagonal numbers are
1, 6, 15, 28, 45, 66, 91, 120, 153, 190, 231, 276, 325, 378, 435, 496, 561, 630, 703, 780, 861, 946, etc., listed in A000384 of Sloane's OEIS. Like a triangular number, the base 10 digital root of a hexagonal number can only be 1, 3, 6 or 9. Every hexagonal number is a triangular number, but not every triangular number is a hexagonal number.

Any integer greater than 1791 can be expressed as a sum of at most four hexagonal numbers, a fact proven by Adrien-Marie Legendre in 1830.

Hexagonal numbers should not be confused with centered hexagonal numbers, which model the standard packaging of Vienna sausages.

%%%%%
%%%%%
\end{document}
