\documentclass[12pt]{article}
\usepackage{pmmeta}
\pmcanonicalname{ExamplesOfRegularPrimes}
\pmcreated{2013-03-22 14:05:58}
\pmmodified{2013-03-22 14:05:58}
\pmowner{alozano}{2414}
\pmmodifier{alozano}{2414}
\pmtitle{examples of regular primes}
\pmrecord{10}{35494}
\pmprivacy{1}
\pmauthor{alozano}{2414}
\pmtype{Example}
\pmcomment{trigger rebuild}
\pmclassification{msc}{11R18}
\pmclassification{msc}{11R29}
%\pmkeywords{class number}
%\pmkeywords{cyclotomic field}
\pmrelated{ClassNumbersAndDiscriminantsTopicsOnClassGroups}

% this is the default PlanetMath preamble.  as your knowledge
% of TeX increases, you will probably want to edit this, but
% it should be fine as is for beginners.

% almost certainly you want these
\usepackage{amssymb}
\usepackage{amsmath}
\usepackage{amsthm}
\usepackage{amsfonts}

% used for TeXing text within eps files
%\usepackage{psfrag}
% need this for including graphics (\includegraphics)
%\usepackage{graphicx}
% for neatly defining theorems and propositions
%\usepackage{amsthm}
% making logically defined graphics
%%%\usepackage{xypic}

% there are many more packages, add them here as you need them

% define commands here

\newtheorem{thm}{Theorem}
\newtheorem{defn}{Definition}
\newtheorem{prop}{Proposition}
\newtheorem{lemma}{Lemma}
\newtheorem{cor}{Corollary}

% Some sets
\newcommand{\Nats}{\mathbb{N}}
\newcommand{\Ints}{\mathbb{Z}}
\newcommand{\Reals}{\mathbb{R}}
\newcommand{\Complex}{\mathbb{C}}
\newcommand{\Rats}{\mathbb{Q}}
\begin{document}
{\bf Examples}:
\begin{enumerate}
\item These are all the irregular primes up to $1061$:

37, 59, 67, 101, 103, 131, 149, 157, 233, 257, 263, 271,\\
283, 293, 307, 311, 347, 353, 379, 389, 401,\\
409, 421, 433, 461, 463, 467, 491, 523, 541,\\
547, 557, 577, 587, 593, 607, 613, 617, 619,\\
631, 647, 653, 659, 673, 677, 683, 691, 727,\\ 
751, 757, 761, 773, 797, 809, 811, 821, 827,\\
839, 877, 881, 887, 929, 953, 971, 1061.

(for this, see the \PMlinkexternal{On-Line Encyclopedia of Integer Sequences}{http://www.research.att.com/~njas/sequences/Seis.html},  \PMlinkexternal{
sequence A000928}{http://www.research.att.com/cgi-bin/access.cgi/as/njas/sequences/eisA.c
gi?Anum=A000928})

\item The following are the first few class numbers of the cyclotomic fields $\Rats(\zeta_p)$, where $\zeta_p$ is a primitive $p$-th root of unity:

\begin{tabular}{|c|c|}
  \hline
  % after \\: \hline or \cline{col1-col2} \cline{col3-col4} ...
  $p$ & Class Number \\
  \hline
  3 & 1 \\
  5 & 1 \\
  7 & 1 \\
  11 & 1 \\
  13 & 1 \\
  17 & 1 \\
  19 & 1 \\
  23 & 3 \\
  29 & 8 \\
  31 & 9 \\
  37 & 37 \\
  41 & 121 \\
  43 & 211 \\
  47 & 695 \\
  53 & 4889 \\
  59 & 41241 \\
  61 & 76301 \\
  \hline
\end{tabular}

An excellent reference for this is $\cite{wash}$.

{\bf Remarks}:
\begin{itemize}
\item Notice that $37$ divides $37$, and $59$ divides $41241=3\cdot 59\cdot 233$, thus $37,\ 59$ are irregular primes (see above).  

\item The class number of the cyclotomic fields grows very quickly with $p$. For example, $p=19$ is the last cyclotomic field of class number 1.

\end{itemize}

\end{enumerate}

\begin{thebibliography}{9}
\bibitem{wash} L. C. Washington, {\em Introduction to Cyclotomic Fields},
Springer-Verlag, New York.
\end{thebibliography}
%%%%%
%%%%%
\end{document}
