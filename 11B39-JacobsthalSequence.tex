\documentclass[12pt]{article}
\usepackage{pmmeta}
\pmcanonicalname{JacobsthalSequence}
\pmcreated{2013-03-22 18:09:40}
\pmmodified{2013-03-22 18:09:40}
\pmowner{PrimeFan}{13766}
\pmmodifier{PrimeFan}{13766}
\pmtitle{Jacobsthal sequence}
\pmrecord{6}{40720}
\pmprivacy{1}
\pmauthor{PrimeFan}{13766}
\pmtype{Definition}
\pmcomment{trigger rebuild}
\pmclassification{msc}{11B39}
\pmdefines{Jacobsthal number}

\endmetadata

% this is the default PlanetMath preamble.  as your knowledge
% of TeX increases, you will probably want to edit this, but
% it should be fine as is for beginners.

% almost certainly you want these
\usepackage{amssymb}
\usepackage{amsmath}
\usepackage{amsfonts}

% used for TeXing text within eps files
%\usepackage{psfrag}
% need this for including graphics (\includegraphics)
%\usepackage{graphicx}
% for neatly defining theorems and propositions
%\usepackage{amsthm}
% making logically defined graphics
%%%\usepackage{xypic}

% there are many more packages, add them here as you need them

% define commands here

\begin{document}
The {\em Jacobsthal sequence} is an additive sequence similar to the Fibonacci sequence, defined by the recurrence relation $J_n = J_{n - 1} + 2J_{n - 2}$, with initial terms $J_0 = 0$ and $J_1 = 1$. A number in the sequence is called a {\em Jacobsthal number}. The first few are 0, 1, 1, 3, 5, 11, 21, 43, 85, 171, 341, etc., listed in A001045 of Sloane's OEIS.

The $n$th Jacobsthal number is the numerator of the alternating sum $$\sum_{i = 1}^n (-1)^{i - 1} \frac{1}{2^i}$$ (the denominators are powers of two). This suggests a closed form: by putting the series solution over a common denominator and summing the geometric series in the numerator, we obtain two equations, one for even-indexed terms of the sequence, $$J_{2n} = \frac{2^{2n} - 1}{3}$$ and the other one for the odd-indexed terms, $$J_{2n + 1} = \frac{2^{2n + 1} - 2}{3} + 1.$$ These equations can be further generalized to $$J_n = \frac{(-1)^{n - 1} + 2^n}{3}.$$

The Jacobsthal numbers are named after the German mathematician Ernst Jacobsthal.


%%%%%
%%%%%
\end{document}
