\documentclass[12pt]{article}
\usepackage{pmmeta}
\pmcanonicalname{PrimeIdealDecompositionInQuadraticExtensionsOfmathbbQ}
\pmcreated{2013-03-22 13:53:46}
\pmmodified{2013-03-22 13:53:46}
\pmowner{alozano}{2414}
\pmmodifier{alozano}{2414}
\pmtitle{prime ideal decomposition in quadratic extensions of $\mathbb{Q}$}
\pmrecord{7}{34643}
\pmprivacy{1}
\pmauthor{alozano}{2414}
\pmtype{Theorem}
\pmcomment{trigger rebuild}
\pmclassification{msc}{11R11}
%\pmkeywords{quadratic field}
%\pmkeywords{prime decomposition}
%\pmkeywords{splitting}
\pmrelated{CalculatingTheSplittingOfPrimes}
\pmrelated{ExamplesOfPrimeIdealDecompositionInNumberFields}
\pmrelated{PrimeIdealDecompositionInCyclotomicExtensionsOfMathbbQ}
\pmrelated{NumberField}
\pmrelated{SplittingAndRamificationInNumberFieldsAndGaloisExtensions}

\endmetadata

% this is the default PlanetMath preamble.  as your knowledge
% of TeX increases, you will probably want to edit this, but
% it should be fine as is for beginners.

% almost certainly you want these
\usepackage{amssymb}
\usepackage{amsmath}
\usepackage{amsthm}
\usepackage{amsfonts}

% used for TeXing text within eps files
%\usepackage{psfrag}
% need this for including graphics (\includegraphics)
%\usepackage{graphicx}
% for neatly defining theorems and propositions
%\usepackage{amsthm}
% making logically defined graphics
%%%\usepackage{xypic}

% there are many more packages, add them here as you need them

% define commands here

\newtheorem{thm}{Theorem}
\newtheorem{defn}{Definition}
\newtheorem{prop}{Proposition}
\newtheorem{lemma}{Lemma}
\newtheorem{cor}{Corollary}

% Some sets
\newcommand{\Nats}{\mathbb{N}}
\newcommand{\Ints}{\mathbb{Z}}
\newcommand{\Reals}{\mathbb{R}}
\newcommand{\Complex}{\mathbb{C}}
\newcommand{\Rats}{\mathbb{Q}}
\begin{document}
Let $K$ be a quadratic number field, i.e. $K=\Rats(\sqrt{d})$ for
some square-free integer $d$. The discriminant of the extension is
\begin{equation*}
D_{K/\Rats}=\begin{cases} d, & \text{ if } d\equiv 1 \ \operatorname{mod}\ 4,\\
4d, & \text{ if } d\equiv 2,3 \operatorname{mod}\ 4.\end{cases}
\end{equation*}
Let $\mathcal{O}_K$ denote the ring of integers of $K$. We have:
\begin{equation*}\mathcal{O}_K\cong \begin{cases}
\Ints\oplus \frac{1+\sqrt{d}}{2}\Ints, & \text{ if } d\equiv 1 \ \operatorname{mod}\ 4,\\
\Ints\oplus \sqrt{d}\Ints, & \text{ if } d\equiv 2,3
\operatorname{mod}\ 4. \end{cases}
\end{equation*}
Prime ideals of $\Ints$ decompose as follows in $\mathcal{O}_K$:

\begin{thm} Let $p\in \Ints$ be a prime.
\begin{enumerate}
\item If $p\mid d$ (divides), then
$p\mathcal{O}_K=(p,\sqrt{d})^2$;

\item If $d$ is odd, then
\begin{equation*}2\mathcal{O}_K=\begin{cases}
(2,1+\sqrt{d})^2, & \text{ if } d\equiv 3\ \operatorname{mod}\ 4,\\
\left(2,\frac{1+\sqrt{d}}{2}\right)\left(2,\frac{1-\sqrt{d}}{2}\right), & \text{ if } d\equiv 1\ \operatorname{mod}\ 8,\\
\text{prime}, & \text{ if } d\equiv 5\ \operatorname{mod}\ 8.
\end{cases} \end{equation*}

\item If $p\neq 2$, $p$ does not divide $d$, then
\begin{equation*}
p\mathcal{O}_K=\begin{cases} (p,n+\sqrt{d})(p,n-\sqrt{d}), & \text{ if } d\equiv n^2\ \operatorname{mod}\ p,\\
\text{prime}, & \text{ if $d$ is not a square } \operatorname{mod}\ p.
\end{cases}
\end{equation*}
\end{enumerate}
\end{thm}

\begin{thebibliography}{9}
\bibitem{marcus} Daniel A.Marcus, {\em Number Fields}. Springer, New York.
\end{thebibliography}
%%%%%
%%%%%
\end{document}
