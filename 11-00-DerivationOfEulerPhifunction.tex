\documentclass[12pt]{article}
\usepackage{pmmeta}
\pmcanonicalname{DerivationOfEulerPhifunction}
\pmcreated{2013-03-22 17:42:52}
\pmmodified{2013-03-22 17:42:52}
\pmowner{jwaixs}{18148}
\pmmodifier{jwaixs}{18148}
\pmtitle{derivation of Euler phi-function}
\pmrecord{22}{40159}
\pmprivacy{1}
\pmauthor{jwaixs}{18148}
\pmtype{Derivation}
\pmcomment{trigger rebuild}
\pmclassification{msc}{11-00}

% this is the default PlanetMath preamble.  as your knowledge
% of TeX increases, you will probably want to edit this, but
% it should be fine as is for beginners.

% almost certainly you want these
\usepackage{amssymb}
\usepackage{amsmath}
\usepackage{amsfonts}

% used for TeXing text within eps files
%\usepackage{psfrag}
% need this for including graphics (\includegraphics)
%\usepackage{graphicx}
% for neatly defining theorems and propositions
%\usepackage{amsthm}
% making logically defined graphics
%%%\usepackage{xypic}

% there are many more packages, add them here as you need them

% define commands here

\begin{document}
In this ``proof" we will construct the solution for the Euler phi-function, $ \phi(n) = n \prod_{ p | n } ( 1 - \frac{1}{n} ) $.

We will do this for the natural number $ n > 0 $.
Keep in mind that $ \gcd(a,n) = 1 \Longleftrightarrow $ $ a $ is not divisible by $ p $ for all primes $ p $ dividing $ n $. 

Let $ n \ge 2 $ and $ p_1,p_2,\cdots,p_r $ be all prime divisors of n.
Let $ N=\{a \mid 0 \leq a < n, \gcd(a,n) = 1\} $ and $ A_i := \{ a \mid 0 \leq a < n, p_i | a \} $.
If $ J \subset \{ 1,2,\cdots,r \} $ than $ p_J := \prod_{j \in J} p_i $. 

Thus, $ \# ( A_J ) = \# ( \bigcap_{ j \in J } A_j ) = \# ( \{ a \in A : p_J | a \} ) = \frac{ n }{ p_J } $ 

Using inclusion-exclusion, 
\begin{align*}
\phi(n) 
= \# ( N )
= \sum_{ J \subset \{ 1,2,\cdots,r \} } (-1)^{ \# ( J ) } \# ( A_J )
= \sum_{ J \subset \{ 1,2,\cdots,r \}} (-1)^{ \# ( J ) } \frac{ n }{ p_J } 
= n \sum_{ J \subset \{ 1,2,\cdots,r \}} (-1)^{ \# ( J ) } \frac{ 1 }{ p_J }
& \\ = n (1 - (\frac{1}{p_1} + \frac{1}{p_2} + \cdots + \frac{1}{p_r}) + \cdots (-1)^r \frac{1}{p_1 p_2 \cdots p_r})
= n \prod_{ p | n } ( 1 - \frac{1}{p} ). 
\end{align*} $ \Box $
%%%%%
%%%%%
\end{document}
