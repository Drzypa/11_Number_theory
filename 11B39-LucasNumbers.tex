\documentclass[12pt]{article}
\usepackage{pmmeta}
\pmcanonicalname{LucasNumbers}
\pmcreated{2013-03-22 12:19:58}
\pmmodified{2013-03-22 12:19:58}
\pmowner{Mathprof}{13753}
\pmmodifier{Mathprof}{13753}
\pmtitle{Lucas numbers}
\pmrecord{8}{31963}
\pmprivacy{1}
\pmauthor{Mathprof}{13753}
\pmtype{Definition}
\pmcomment{trigger rebuild}
\pmclassification{msc}{11B39}
\pmsynonym{Lucas sequence}{LucasNumbers}
\pmrelated{FibonacciSequence}


\begin{document}
The \emph{Lucas numbers} are a slight variation of Fibonacci numbers. 
These numbers follow the same recursion:
$$l_{n+1}=l_n + l_{n-1}$$
but having different initial conditions: $l_1=1, l_2=3$ leading to the sequence
$1, 3, 4, 7, 11, 18, 29, 47, 76, 123,\ldots$.

Lucas numbers have the following property: $l_n=f_{n-1}+f_{n+1}$ where $f_n$ is the $n^{th}$ Fibonacci number.
%%%%%
%%%%%
\end{document}
