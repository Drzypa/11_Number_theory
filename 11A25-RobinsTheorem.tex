\documentclass[12pt]{article}
\usepackage{pmmeta}
\pmcanonicalname{RobinsTheorem}
\pmcreated{2013-03-22 19:33:47}
\pmmodified{2013-03-22 19:33:47}
\pmowner{pahio}{2872}
\pmmodifier{pahio}{2872}
\pmtitle{Robin's theorem}
\pmrecord{7}{42549}
\pmprivacy{1}
\pmauthor{pahio}{2872}
\pmtype{Theorem}
\pmcomment{trigger rebuild}
\pmclassification{msc}{11A25}
\pmclassification{msc}{11M26}
\pmrelated{GronwallsTheorem}
\pmrelated{PropertiesOfXiFunction}

% this is the default PlanetMath preamble.  as your knowledge
% of TeX increases, you will probably want to edit this, but
% it should be fine as is for beginners.

% almost certainly you want these
\usepackage{amssymb}
\usepackage{amsmath}
\usepackage{amsfonts}

% used for TeXing text within eps files
%\usepackage{psfrag}
% need this for including graphics (\includegraphics)
%\usepackage{graphicx}
% for neatly defining theorems and propositions
 \usepackage{amsthm}
% making logically defined graphics
%%%\usepackage{xypic}

% there are many more packages, add them here as you need them

% define commands here

\theoremstyle{definition}
\newtheorem*{thmplain}{Theorem}

\begin{document}
Let $\sigma(n)$ be the sum of the positive divisors of an integer $n$ and
$$G(n) \;:=\; \frac{\sigma(n)}{n\ln(\ln{n})}  \qquad(n \;=\; 2,\,3,\,4,\,\ldots).$$
The Riemann Hypothesis is true if and only if 
$$G(n) \;<\; e^\gamma \quad \mbox{for all }\, n \;>\; 7!$$
where $\gamma$ is the Euler--Mascheroni constant.

\begin{thebibliography}{8}
\bibitem{G}{\sc G. Robin}: Grandes valeurs de la fonction somme des diviseurs et hypoth\`ese de
Riemann.\, $-$ \emph{J. Math. Pures Appl.} \textbf{14} (1984) 187--213.
\bibitem{L}{\sc J. C. Lagarias}: An elementary problem equivalent to the Riemann hypothesis.\, $-$ {\it Amer. Math. Monthly} \textbf{109} (2002), 534--543. Available at http://arxiv.org/abs/math/0008177
\end{thebibliography}

%%%%%
%%%%%
\end{document}
