\documentclass[12pt]{article}
\usepackage{pmmeta}
\pmcanonicalname{CenteredHexagonalNumber}
\pmcreated{2013-03-22 16:34:35}
\pmmodified{2013-03-22 16:34:35}
\pmowner{PrimeFan}{13766}
\pmmodifier{PrimeFan}{13766}
\pmtitle{centered hexagonal number}
\pmrecord{5}{38767}
\pmprivacy{1}
\pmauthor{PrimeFan}{13766}
\pmtype{Definition}
\pmcomment{trigger rebuild}
\pmclassification{msc}{11D09}
\pmsynonym{hex number}{CenteredHexagonalNumber}

% this is the default PlanetMath preamble.  as your knowledge
% of TeX increases, you will probably want to edit this, but
% it should be fine as is for beginners.

% almost certainly you want these
\usepackage{amssymb}
\usepackage{amsmath}
\usepackage{amsfonts}

% used for TeXing text within eps files
%\usepackage{psfrag}
% need this for including graphics (\includegraphics)
%\usepackage{graphicx}
% for neatly defining theorems and propositions
%\usepackage{amsthm}
% making logically defined graphics
%%%\usepackage{xypic}

% there are many more packages, add them here as you need them

% define commands here

\begin{document}
A {\em centered hexagonal number}, or {\em hex number} is a figurate number that represents a hexagon with a dot in the center and all other dots surrounding the center dot equidistantly. The centered hexagonal number for $n$ is given by the formula $1 + 6\left({1 \over 2} n(n + 1 ) \right)$. In other words, the centered hexagonal number for $n$ is the triangular number for $n$ multiplied by 6, then add 1.

The first few centered hexagonal numbers are: 1, 7, 19, 37, 61, 91, 127, 169, 217, 271, 331, 397, 469, 547, 631, 721, 817, 919, ... listed in A003215 of Sloane's OEIS.

To find centered hexagonal numbers besides 1 that are also triangular numbers or squares, it is necessary to solve Diophantine equations. By solving the Diophantine equation ${1 \over 2} m(m + 1) = 3n^2 + 3n + 1,$ we learn that 91, 8911 and 873181 are numbers that are both centered hexagonal numbers and triangular numbers (they grow very large after that), while solving the Diophantine equation $m^2 = 3n^2 + 3n + 1,$ we learn that 169 and 32761 are centered hexagonal numbers that are also squares.

The sum of the first $n$ centered hexagonal numbers is $n^3$. The difference between $(2n)^2$ and the $n$th centered hexagonal number is a number of the form $n^2 + 3n - 1$, while the difference between $(2n - 1))^2$ and the $n$th centered hexagonal number is an oblong number.
%%%%%
%%%%%
\end{document}
