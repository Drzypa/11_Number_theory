\documentclass[12pt]{article}
\usepackage{pmmeta}
\pmcanonicalname{Modulus}
\pmcreated{2013-03-22 12:35:26}
\pmmodified{2013-03-22 12:35:26}
\pmowner{djao}{24}
\pmmodifier{djao}{24}
\pmtitle{modulus}
\pmrecord{4}{32841}
\pmprivacy{1}
\pmauthor{djao}{24}
\pmtype{Definition}
\pmcomment{trigger rebuild}
\pmclassification{msc}{11R37}

% this is the default PlanetMath preamble.  as your knowledge
% of TeX increases, you will probably want to edit this, but
% it should be fine as is for beginners.

% almost certainly you want these
\usepackage{amssymb}
\usepackage{amsmath}
\usepackage{amsfonts}

% used for TeXing text within eps files
%\usepackage{psfrag}
% need this for including graphics (\includegraphics)
%\usepackage{graphicx}
% for neatly defining theorems and propositions
%\usepackage{amsthm}
% making logically defined graphics
%%%\usepackage{xypic} 

% there are many more packages, add them here as you need them

% define commands here
\newcommand{\p}{\mathfrak{p}}
\begin{document}
A {\em modulus} for a number field $K$ is a formal product
$$
\prod_{\p} \p^{n_\p}
$$
where
\begin{itemize}
\item The product is taken over all finite primes and infinite primes of $K$
\item The exponents $n_\p$ are nonnegative integers
\item All but finitely many of the $n_\p$ are zero
\item For every real prime $\p$, the exponent $n_\p$ is either 0 or 1
\item For every complex prime $\p$, the exponent $n_\p$ is 0
\end{itemize}
A modulus can be written as a product of its finite part
$$
\prod_{\p \text{ finite}} \p^{n_\p}
$$
and its infinite part
$$
\prod_{\p \text{ real}} \p^{n_\p},
$$
with the finite part equal to some ideal in the ring of integers $\mathcal{O}_K$ of $K$, and the infinite part equal to the product of some subcollection of the real primes of $K$.
%%%%%
%%%%%
\end{document}
