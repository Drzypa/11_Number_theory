\documentclass[12pt]{article}
\usepackage{pmmeta}
\pmcanonicalname{Emirp}
\pmcreated{2013-03-22 17:06:54}
\pmmodified{2013-03-22 17:06:54}
\pmowner{PrimeFan}{13766}
\pmmodifier{PrimeFan}{13766}
\pmtitle{emirp}
\pmrecord{4}{39415}
\pmprivacy{1}
\pmauthor{PrimeFan}{13766}
\pmtype{Definition}
\pmcomment{trigger rebuild}
\pmclassification{msc}{11A41}
\pmclassification{msc}{11A63}
\pmdefines{semirp}

\endmetadata

% this is the default PlanetMath preamble.  as your knowledge
% of TeX increases, you will probably want to edit this, but
% it should be fine as is for beginners.

% almost certainly you want these
\usepackage{amssymb}
\usepackage{amsmath}
\usepackage{amsfonts}

% used for TeXing text within eps files
%\usepackage{psfrag}
% need this for including graphics (\includegraphics)
%\usepackage{graphicx}
% for neatly defining theorems and propositions
%\usepackage{amsthm}
% making logically defined graphics
%%%\usepackage{xypic}

% there are many more packages, add them here as you need them

% define commands here

\begin{document}
An {\em emirp} is a prime number that in a given base $b$ reads like a different prime number backwards. The plural of emirp is {\em emirps}, but {\em semirp} is occasionally used by amateurs.

For example, in base 10, the number 37 read backwards is 73, both of these are prime. The backwards number must be different. A006567 of Sloane's OEIS lists the first fifty base 10 emirps.

A palindromic prime is not an emirp, since it reads the same number both ways. Thus, in binary, Mersenne primes can not be emirps.
%%%%%
%%%%%
\end{document}
