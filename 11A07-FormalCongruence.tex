\documentclass[12pt]{article}
\usepackage{pmmeta}
\pmcanonicalname{FormalCongruence}
\pmcreated{2013-03-22 14:23:50}
\pmmodified{2013-03-22 14:23:50}
\pmowner{pahio}{2872}
\pmmodifier{pahio}{2872}
\pmtitle{formal congruence}
\pmrecord{32}{35894}
\pmprivacy{1}
\pmauthor{pahio}{2872}
\pmtype{Definition}
\pmcomment{trigger rebuild}
\pmclassification{msc}{11A07}
\pmclassification{msc}{11C08}
\pmsynonym{polynomial congruence}{FormalCongruence}
%\pmkeywords{congruence}
%\pmkeywords{congruent}
\pmrelated{CongruenceRelationOnAnAlgebraicSystem}
\pmrelated{APolynomialOfDegreeNOverAFieldHasAtMostNRoots}
\pmrelated{IdealDecompositionInDedekindDomain}
\pmrelated{ExampleOfGcd}
\pmrelated{FermatsLittleTheorem}
\pmrelated{WilsonsTheorem}
\pmrelated{FreshmansDream}
\pmrelated{UsingPrimitiveRootsAndIndexToSolveCongruences}
\pmrelated{SufficientConditio}
\pmdefines{formally congruent}
\pmdefines{identical congruence}

\endmetadata

% this is the default PlanetMath preamble.  as your knowledge
% of TeX increases, you will probably want to edit this, but
% it should be fine as is for beginners.

% almost certainly you want these
\usepackage{amssymb}
\usepackage{amsmath}
\usepackage{amsfonts}

% used for TeXing text within eps files
%\usepackage{psfrag}
% need this for including graphics (\includegraphics)
%\usepackage{graphicx}
% for neatly defining theorems and propositions
%\usepackage{amsthm}
% making logically defined graphics
%%%\usepackage{xypic}

% there are many more packages, add them here as you need them

% define commands here
\begin{document}
Let $m$ be a positive integer.\, Two polynomials \,$f(X_1,\,X_2,\,\ldots,\,X_n)$\, and\, $g(X_1,\,X_2,\,\ldots,\,X_n)$\, with integer coefficients are said to be \emph{formally congruent} modulo $m$, denoted
\begin{align}
f(X_1,\,X_2,\,\ldots,\,X_n) \;\underline{\equiv}\; g(X_1,\,X_2,\,\ldots,\,X_n) \pmod{m},
\end{align}
iff all coefficients of the difference polynomial\, $f\!-\!g$\, are divisible by $m$.\\

\textbf{Remark 1.}\, The formal congruence of polynomials is an equivalence relation in the set\, 
$\mathbb{Z}[X_1,\,X_2,\,\ldots,\,X_n]$.\\

\textbf{Remark 2.}\, The formal congruence (1) implies that all integers substituted for the variables satisfy it, in other words, one can speak of an \emph{identical congruence}.\, However, there are identical congruences that are not formal congruences; e.g.
$$X^{p} \;\equiv\; X \pmod{p}$$
where $p$ is a positive prime.\\

\textbf{Examples}
\begin{enumerate}

 \item $(X\!+\!Y)^p \;\underline{\equiv}\; X^p\!+\!Y^p \pmod{p}$\, when $p$ is a positive prime number.\, This result is easily proved with the binomial theorem (cp. the freshman's dream) and may be by induction generalized to
   $$(X_1\!+\!X_2\!+\ldots+\!X_n)^p \;\underline{\equiv}\; X_1^p\!+\!X_2^p\!+\ldots+\!X_n^p \pmod{p}.$$
If one substitutes in this formal congruence 1 for all $X_1$, $X_2$, \ldots, $X_n$, one obtains the congruence
           $$n^p \;\equiv\; n \pmod p;$$
\PMlinkescapetext{similar} substitution of $-1$ shows that the last congruence is valid also for negative integers $n$.\, If it is supposed that $p$ is not factor of $n$, we have got the Fermat's little theorem.

 \item Let $p$ be a positive prime number.\, It is possible that the $n^\mathrm{th}$ degree congruence
   $$P(x) \;:=\; a_0x^n\!+\!a_1x^{n-1}\!+\ldots+\!a_n \;\equiv\; 0 \pmod{p},$$
where\, $a_i \in \mathbb{Z} \,\, \forall i$\, and\, $p \nmid a_0$,\, has $n$ modulo $p$ incongruent roots\, $x = x_1,\,x_2,\,\ldots,\,x_n$.\, We then have the formal congruence
  $$P(x) \;\underline{\equiv}\; a_0(x\!-\!x_1)(x\!-\!x_2)\cdots(x\!-\!x_n) \pmod{p}.$$
Especially, we have
  $$x^{p-1}\!-\!1 \;\underline{\equiv}\; (x\!-\!1)(x\!-\!2)\cdots(x\!-\!p\!+\!1) \pmod{p},$$
because Fermat's little theorem guarantees the roots\, $x = 1,\,2,\,\ldots,\,p\!-\!1$\, for the congruence\, 
$x^{p-1}\!-\!1 \equiv 0 \pmod{p}$.\, If the value\, $x = 0$\, is substituted in the previous formal congruence, it gives
  $$-1 \equiv (-1)^{p-1}(p\!-\!1)! \pmod{p}$$
or 
  $$(p\!-\!1)! \;\equiv\; -1 \pmod{p},$$
which is half of Wilson's theorem.\, The reverse part of this theorem follows from the last congruence so that if $p$ were not prime, then the number $-1$ would be divisible by any proper divisor of $p$.
\end{enumerate}

\begin{thebibliography}{8}
\bibitem{FS}{\sc F. St\"owener}: ``Simultanbeweis des Fermatschen und Wilsonschen Satzes''.\, -- \emph{Elemente der Mathematik} \textbf{30} (1975).
\end{thebibliography}

%%%%%
%%%%%
\end{document}
