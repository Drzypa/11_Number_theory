\documentclass[12pt]{article}
\usepackage{pmmeta}
\pmcanonicalname{ProofOfDivisionAlgorithmForIntegers}
\pmcreated{2013-03-22 13:01:11}
\pmmodified{2013-03-22 13:01:11}
\pmowner{drini}{3}
\pmmodifier{drini}{3}
\pmtitle{proof of division algorithm for integers}
\pmrecord{4}{33404}
\pmprivacy{1}
\pmauthor{drini}{3}
\pmtype{Proof}
\pmcomment{trigger rebuild}
\pmclassification{msc}{11A51}

\usepackage{graphicx}
%%%\usepackage{xypic} 
\usepackage{bbm}
\newcommand{\Z}{\mathbbmss{Z}}
\newcommand{\C}{\mathbbmss{C}}
\newcommand{\R}{\mathbbmss{R}}
\newcommand{\Q}{\mathbbmss{Q}}
\newcommand{\mathbb}[1]{\mathbbmss{#1}}
\newcommand{\figura}[1]{\begin{center}\includegraphics{#1}\end{center}}
\newcommand{\figuraex}[2]{\begin{center}\includegraphics[#2]{#1}\end{center}}
\begin{document}
Let $a,b$ integers ($b>0$). We want to express $a=bq+r$ for some integers $q,r$ with $0\leq r < b$ and that such expression is unique.

Consider the numbers
$$\ldots, a-3b, a-2b, a-b,a,a+b,a+2b,a+3b,\ldots$$
From all these numbers, there has to be a smallest non negative one. Let it be $r$. 
Since $r=a-qb$ for some $q$,\footnote{For example, if $r =a+5b$ then $q=-5$.} we have $a=bq+r$. And, if $r\geq b$ then $r$ wasn't the smallest non-negative number on the list, since the previous (equal to $r-b$) would also be non-negative. Thus $0\leq r<b$.

So far, we have proved that we can express $a$ as$bq+r$ for some pair of integers $q,r$ such that $0\leq r<b$. Now we  will prove the uniqueness of such expression.


Let $q'$ and $r'$ another pair of integers holding $a=bq'+r'$ and $0\leq r' <b$. Suppose $r\neq r'$. Since $r' = a-bq'$ is a number on the list, cannot be smaller or equal than $r$ and thus $r<r'$. Notice that 
$$0<r'-r =(a-bq') - (a-bq) = b(q-q')$$
so $b$ divides $r'-r$ which is impossible since $0<r'-r < b$. We conclude that $r'=r$. 
Finally, if $r=r'$ then $a-bq = a-bq'$ and therefore $q=q'$. This concludes the proof of the uniqueness part.
%%%%%
%%%%%
\end{document}
