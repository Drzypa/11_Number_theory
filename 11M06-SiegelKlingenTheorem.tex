\documentclass[12pt]{article}
\usepackage{pmmeta}
\pmcanonicalname{SiegelKlingenTheorem}
\pmcreated{2013-03-22 16:01:19}
\pmmodified{2013-03-22 16:01:19}
\pmowner{alozano}{2414}
\pmmodifier{alozano}{2414}
\pmtitle{Siegel-Klingen Theorem}
\pmrecord{4}{38061}
\pmprivacy{1}
\pmauthor{alozano}{2414}
\pmtype{Theorem}
\pmcomment{trigger rebuild}
\pmclassification{msc}{11M06}
\pmclassification{msc}{11R42}

% this is the default PlanetMath preamble.  as your knowledge
% of TeX increases, you will probably want to edit this, but
% it should be fine as is for beginners.

% almost certainly you want these
\usepackage{amssymb}
\usepackage{amsmath}
\usepackage{amsthm}
\usepackage{amsfonts}

% used for TeXing text within eps files
%\usepackage{psfrag}
% need this for including graphics (\includegraphics)
%\usepackage{graphicx}
% for neatly defining theorems and propositions
%\usepackage{amsthm}
% making logically defined graphics
%%%\usepackage{xypic}

% there are many more packages, add them here as you need them

% define commands here

\newtheorem*{thm}{Theorem}
\newtheorem{defn}{Definition}
\newtheorem{prop}{Proposition}
\newtheorem{lemma}{Lemma}
\newtheorem{cor}{Corollary}

\theoremstyle{definition}
\newtheorem{exa}{Example}

% Some sets
\newcommand{\Nats}{\mathbb{N}}
\newcommand{\Ints}{\mathbb{Z}}
\newcommand{\Reals}{\mathbb{R}}
\newcommand{\Complex}{\mathbb{C}}
\newcommand{\Rats}{\mathbb{Q}}
\newcommand{\Gal}{\operatorname{Gal}}
\newcommand{\Cl}{\operatorname{Cl}}
\begin{document}
\begin{thm}[Siegel-Klingen Theorem, \cite{klingen},\cite{siegel}]
Let $K$ be a totally real number field and let $\zeta(s,K)$ be the Dedekind zeta function of $K$. If $n\geq 1$ is an integer then $\zeta(-n,K)$ is a rational number (i.e. $\zeta(-n,K)\in \Rats$).
\end{thm}

\begin{thebibliography}{99}
\bibitem{klingen}  Klingen, Helmut, {\em \"Uber die Werte der Dedekindschen Zetafunktion}. (German)  Math. Ann.  145  1961/1962 265--272.

\bibitem{siegel} Siegel, Carl Ludwig, {\em  \"Uber die analytische Theorie der quadratischen Formen. III}. (German)  Ann. of Math. (2)  38  (1937),  no. 1, 212--291.
\end{thebibliography}
%%%%%
%%%%%
\end{document}
