\documentclass[12pt]{article}
\usepackage{pmmeta}
\pmcanonicalname{DivisibilityByProduct}
\pmcreated{2013-03-22 14:50:37}
\pmmodified{2013-03-22 14:50:37}
\pmowner{pahio}{2872}
\pmmodifier{pahio}{2872}
\pmtitle{divisibility by product}
\pmrecord{13}{36513}
\pmprivacy{1}
\pmauthor{pahio}{2872}
\pmtype{Theorem}
\pmcomment{trigger rebuild}
\pmclassification{msc}{11A51}
\pmclassification{msc}{13A05}
%\pmkeywords{B\'ezout ring}
\pmrelated{BezoutDomain}
\pmrelated{ProductDivisibleButFactorCoprime}
\pmrelated{CorollaryOfBezoutsLemma}
\pmdefines{B\'ezout ring}

\endmetadata

% this is the default PlanetMath preamble.  as your knowledge
% of TeX increases, you will probably want to edit this, but
% it should be fine as is for beginners.

% almost certainly you want these
\usepackage{amssymb}
\usepackage{amsmath}
\usepackage{amsfonts}

% used for TeXing text within eps files
%\usepackage{psfrag}
% need this for including graphics (\includegraphics)
%\usepackage{graphicx}
% for neatly defining theorems and propositions
 \usepackage{amsthm}
% making logically defined graphics
%%%\usepackage{xypic}

% there are many more packages, add them here as you need them

% define commands here
\theoremstyle{definition}
\newtheorem*{thmplain}{Theorem}
\begin{document}
\begin{thmplain}
\,Let $R$ be a {\em B\'ezout ring}, i.e. a commutative ring with non-zero unity where every finitely generated ideal is a principal ideal.  If $a,\,b,\,c$ are three elements of $R$ such that $a$ and $b$ divide $c$ and\, $\gcd(a,\,b) = 1$, \,then also $ab$ divides $c$.
\end{thmplain}

{\em Proof.}  The divisibility assumptions \PMlinkescapetext{mean} that\, $c = aa_1 = bb_1$\, where $a_1$ and $b_1$ are some elements of $R$.\, Because $R$ is a B\'ezout ring, there exist such elements $x$ and $y$ of $R$ that \,$\gcd(a,\,b) = 1 = xa+yb$. This implies the equation \,$a_1 = xaa_1+yba_1 = xbb_1+yba_1$\, which shows that $a_1$ is divisible by $b$, i.e.\, $a_1 = bb_2$,\, $b_2\in R$.  Consequently,\, $c = aa_1 = abb_2$,\, or\, $ab \mid c$\; Q.E.D.

\textbf{Note 1.}  The theorem may by induction be generalized for several \PMlinkname{factors}{Divisibility} of $c$.

\textbf{Note 2.}  The theorem holds e.g. in all B\'ezout domains, especially in principal ideal domains, such as $\mathbb{Z}$ and polynomial rings over a field.
%%%%%
%%%%%
\end{document}
