\documentclass[12pt]{article}
\usepackage{pmmeta}
\pmcanonicalname{StickelbergersTheorem}
\pmcreated{2013-03-22 14:12:31}
\pmmodified{2013-03-22 14:12:31}
\pmowner{mathcam}{2727}
\pmmodifier{mathcam}{2727}
\pmtitle{Stickelberger's theorem}
\pmrecord{6}{35642}
\pmprivacy{1}
\pmauthor{mathcam}{2727}
\pmtype{Theorem}
\pmcomment{trigger rebuild}
\pmclassification{msc}{11R29}
\pmdefines{Stickelberger element}

% this is the default PlanetMath preamble.  as your knowledge
% of TeX increases, you will probably want to edit this, but
% it should be fine as is for beginners.

% almost certainly you want these
\usepackage{amssymb}
\usepackage{amsmath}
\usepackage{amsfonts}
\usepackage{amsthm}

% used for TeXing text within eps files
%\usepackage{psfrag}
% need this for including graphics (\includegraphics)
%\usepackage{graphicx}
% for neatly defining theorems and propositions
%\usepackage{amsthm}
% making logically defined graphics
%%%\usepackage{xypic}

% there are many more packages, add them here as you need them

% define commands here
\newtheorem{Theo}{Theorem}
\newcommand{\mc}{\mathcal}
\newcommand{\mb}{\mathbb}
\newcommand{\mf}{\mathfrak}
\newcommand{\ol}{\overline}
\newcommand{\ra}{\rightarrow}
\newcommand{\la}{\leftarrow}
\newcommand{\La}{\Leftarrow}
\newcommand{\Ra}{\Rightarrow}
\newcommand{\nor}{\vartriangleleft}
\newcommand{\Gal}{\text{Gal}}
\newcommand{\GL}{\text{GL}}
\newcommand{\Z}{\mb{Z}}
\newcommand{\R}{\mb{R}}
\newcommand{\Q}{\mb{Q}}
\newcommand{\C}{\mb{C}}
\newcommand{\<}{\langle}
\renewcommand{\>}{\rangle}
\begin{document}
\begin{Theo}[Stickelberger]
Let $L=\mathbb{Q}(\zeta_m)$ be a cyclotomic field extension of $\Q$ with Galois group $G=\{\sigma_a\}_{a\in(\Z/m\Z)^\times}$, and consider the group ring $\Q[G]$.  Define the Stickelberger element $\theta\in\Q[G]$ by
\begin{align*}
\theta=\frac{1}{m}\sum_{1\leq a\leq m, (a,m)=1}a\sigma_a^{-1},
\end{align*}
and take $\beta\in\Z[G]$ such that $\beta\theta\in\Z[G]$ as well.  Then $\beta\theta$ is an annihilator for the ideal class group of $\Q(\zeta_m)$.
\end{Theo}

Note that $\theta$ itself need not be an annihilator, just that any multiple of it in $\Z[G]$ is.

This result allows for the most basic \PMlinkescapetext{connections} between the (otherwise hard to determine) \PMlinkescapetext{structure} of a cyclotomic ideal class group and its \PMlinkescapetext{collection} of annihilators.  For an application of Stickelberger's theorem, see Herbrand's theorem.
%%%%%
%%%%%
\end{document}
