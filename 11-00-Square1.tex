\documentclass[12pt]{article}
\usepackage{pmmeta}
\pmcanonicalname{Square1}
\pmcreated{2013-03-22 12:02:35}
\pmmodified{2013-03-22 12:02:35}
\pmowner{drini}{3}
\pmmodifier{drini}{3}
\pmtitle{square}
\pmrecord{10}{31087}
\pmprivacy{1}
\pmauthor{drini}{3}
\pmtype{Definition}
\pmcomment{trigger rebuild}
\pmclassification{msc}{11-00}
\pmclassification{msc}{26-00}
\pmclassification{msc}{30-00}
\pmsynonym{second power}{Square1}
\pmrelated{CubeOfANumber}
\pmrelated{PerfectPower}
\pmdefines{perfect square}

\endmetadata

\usepackage{amssymb}
\usepackage{amsmath}
\usepackage{amsfonts}
\usepackage{graphicx}
%%%\usepackage{xypic}
\begin{document}
The \emph{square} of a number $x$ is the number obtained multiplying $x$ by itself.\, It's denoted as $x^2$.

Some examples:
\begin{eqnarray*}
5^2&=&25\\
\left(\frac{1}{3}\right)^2&=&\frac{1}{9}\\
0^2&=&0\\
(-0.5)^2 &=& 0.25
\end{eqnarray*}

A \emph{perfect square} is the square of an integer.  The first few perfect squares are 0, 1, 4, 9, 16, 25, $\ldots$.
%%%%%
%%%%%
%%%%%
\end{document}
