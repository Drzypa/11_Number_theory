\documentclass[12pt]{article}
\usepackage{pmmeta}
\pmcanonicalname{ErdHosHeilbronnConjecture}
\pmcreated{2013-03-22 13:38:15}
\pmmodified{2013-03-22 13:38:15}
\pmowner{bbukh}{348}
\pmmodifier{bbukh}{348}
\pmtitle{Erd\H{o}s-Heilbronn conjecture}
\pmrecord{10}{34287}
\pmprivacy{1}
\pmauthor{bbukh}{348}
\pmtype{Theorem}
\pmcomment{trigger rebuild}
\pmclassification{msc}{11B75}
\pmrelated{CauchyDavenportTheorem}

\usepackage{amssymb}
\usepackage{amsmath}
\usepackage{amsfonts}

% used for TeXing text within eps files
%\usepackage{psfrag}
% need this for including graphics (\includegraphics)
%\usepackage{graphicx}
% for neatly defining theorems and propositions
%\usepackage{amsthm}
% making logically defined graphics
%%%\usepackage{xypic}
\newcommand*{\integers}{\ensuremath{{\mathbb{Z}}}}

\makeatletter
\@ifundefined{bibname}{}{\renewcommand{\bibname}{References}}
\makeatother
\begin{document}
Let $A\subset \integers_p$ be a set of residues modulo $p$, and let $h$ be a positive integer, then
\begin{equation*}
h^\wedge\!A=\{\,a_1+a_2+\dotsb+a_h \mid a_1,a_2,\dotsc,a_h\text{ are distinct elements of }A\,\}
\end{equation*}
has cardinality at least $\min(p,hk-h^2+1)$. This was conjectured by Erd\H{o}s and Heilbronn in 1964\cite{cite:erdos_heilbronn_conj}. The first proof was given by Dias da Silva and Hamidoune in 1994.

\begin{thebibliography}{1}

\bibitem{cite:erdos_heilbronn_conj}
Paul Erd{\H{o}}s and Hans Heilbronn.
\newblock On the addition of residue classes $\mod p$.
\newblock {\em Acta Arith.}, 9:149--159, 1964.
\newblock \PMlinkexternal{Zbl 0156.04801}{http://www.emis.de/cgi-bin/zmen/ZMATH/en/quick.html?type=html&an=0156.04801}.

\bibitem{cite:nathanson_classicalbases}
Melvyn~B. Nathanson.
\newblock {\em Additive Number Theory: Inverse Problems and Geometry of
Sumsets}, volume 165 of {\em GTM}.
\newblock Springer, 1996.
\newblock \PMlinkexternal{Zbl 0859.11003}{http://www.emis.de/cgi-bin/zmen/ZMATH/en/quick.html?type=html&an=0859.11003}.

\end{thebibliography}

%@BOOK{cite:nathanson_inverseprob,
% author = {Melvyn B. Nathanson},
% title = {Additive Number Theory: Inverse Problems and Geometry of Sumsets},
% series = {GTM},
% volume = 165,
% year = 1996,
% publisher = {Springer},
% note      = {\PMlinkexternal{Zbl 0859.11003}{http://www.emis.de/cgi-bin/zmen/ZMATH/en/quick.html?type=html&an=0859.11003}}
%}
%
%@ARTICLE{cite:erdos_heilbronn_conj,
% author    = {Paul Erd{\H{o}}s and Hans Heilbronn},
% title     = {On the addition of residue classes $\mod p$},
% journal   = {Acta Arith.},
% volume    = 9,
% year      = 1964,
% pages     = {149--159},
% note      = {\PMlinkexternal{Zbl 0156.04801}{http://www.emis.de/cgi-bin/zmen/ZMATH/en/quick.html?type=html&an=0156.04801}}
%}
%%%%%
%%%%%
\end{document}
