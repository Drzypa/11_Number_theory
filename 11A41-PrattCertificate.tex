\documentclass[12pt]{article}
\usepackage{pmmeta}
\pmcanonicalname{PrattCertificate}
\pmcreated{2013-03-22 18:53:06}
\pmmodified{2013-03-22 18:53:06}
\pmowner{PrimeFan}{13766}
\pmmodifier{PrimeFan}{13766}
\pmtitle{Pratt certificate}
\pmrecord{4}{41733}
\pmprivacy{1}
\pmauthor{PrimeFan}{13766}
\pmtype{Definition}
\pmcomment{trigger rebuild}
\pmclassification{msc}{11A41}

% this is the default PlanetMath preamble.  as your knowledge
% of TeX increases, you will probably want to edit this, but
% it should be fine as is for beginners.

% almost certainly you want these
\usepackage{amssymb}
\usepackage{amsmath}
\usepackage{amsfonts}

% used for TeXing text within eps files
%\usepackage{psfrag}
% need this for including graphics (\includegraphics)
%\usepackage{graphicx}
% for neatly defining theorems and propositions
%\usepackage{amsthm}
% making logically defined graphics
%%%\usepackage{xypic}

% there are many more packages, add them here as you need them

% define commands here

\begin{document}
A {\em Pratt certificate} for a given integer $n$ is a primality certificate in which the numbers allow verification of primality by using the converse of Fermat's little theorem (or Lehmer's theorem). Generating a Pratt certificate requires knowledge of the prime factorization of $n - 1$ (the primes $p_i$). Then, one must find a witness $w$ such that $w^{n - 1} \equiv 1 \mod n$ but not $$w^{\frac{n - 1}{p_i}} \equiv 1 \mod n$$ for any $i \leq \omega(n - 1)$ (with $\omega(x)$ being the number of distinct prime factors function). Pratt certificates typically include witnesses not just for $n$ but also for the prime factors of $n - 1$.

Because a Pratt certificate requires the factorization of $n - 1$, it is generally only used for small numbers, with ``small'' being roughly defined as being less than about a billion. We'll use a much smaller number for our example, one for which it would actually be faster to just perform trial division: $n = 127$. We then have to factor 126, which gives us 2, 3, 3, 7. Choosing our witness $w = 12$, we then see that $12^{126} \equiv 1 \mod 127$ but $12^{63} \equiv -1 \mod 127$, $12^{42} \equiv 107 \mod 127$ and $12^{18} \equiv 8 \mod 127$. Most algorithms for the Pratt certificate generally hard-code 2 as a prime number, but provide certificates for the other primes in the factorization. For this example we won't bother to give certificates for 3 and 7.
%%%%%
%%%%%
\end{document}
