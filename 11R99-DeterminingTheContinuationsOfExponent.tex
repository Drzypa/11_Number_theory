\documentclass[12pt]{article}
\usepackage{pmmeta}
\pmcanonicalname{DeterminingTheContinuationsOfExponent}
\pmcreated{2013-03-22 18:00:16}
\pmmodified{2013-03-22 18:00:16}
\pmowner{pahio}{2872}
\pmmodifier{pahio}{2872}
\pmtitle{determining the continuations of exponent}
\pmrecord{7}{40518}
\pmprivacy{1}
\pmauthor{pahio}{2872}
\pmtype{Example}
\pmcomment{trigger rebuild}
\pmclassification{msc}{11R99}
\pmclassification{msc}{13A18}
\pmclassification{msc}{12J20}
\pmclassification{msc}{13F30}
\pmrelated{ExampleOfRingWhichIsNotAUFD}

\endmetadata

% this is the default PlanetMath preamble.  as your knowledge
% of TeX increases, you will probably want to edit this, but
% it should be fine as is for beginners.

% almost certainly you want these
\usepackage{amssymb}
\usepackage{amsmath}
\usepackage{amsfonts}

% used for TeXing text within eps files
%\usepackage{psfrag}
% need this for including graphics (\includegraphics)
%\usepackage{graphicx}
% for neatly defining theorems and propositions
 \usepackage{amsthm}
% making logically defined graphics
%%%\usepackage{xypic}

% there are many more packages, add them here as you need them

% define commands here

\theoremstyle{definition}
\newtheorem*{thmplain}{Theorem}

\begin{document}
\PMlinkescapeword{exponent} \PMlinkescapeword{prime number}
\textbf{Task.}\, Let $\nu_0$ be the \PMlinkname{3-adic (triadic)}{PAdicValuation}\, exponent valuation of the field $\mathbb{Q}$ of the rational numbers and let $\mathfrak{o}$ be the ring of the exponent.\, Determine the integral closure $\mathfrak{O}$ of $\mathfrak{o}$ in the extension field $\mathbb{Q}(\sqrt{-5})$ and the continuations of $\nu_0$ to this field.\\

The triadic \PMlinkname{exponent}{ExponentValuation} of $\mathbb{Q}$ at any non-zero rational number $\displaystyle\frac{3^nu}{v}$, where $u$ and $v$ are integers not divisible by 3, is defined as
$$\nu_0\left(\frac{3^nu}{v}\right) \,:=\, n.$$
Any number of the quadratic field $\mathbb{Q}(\sqrt{-5})$ is of the form
$$r+s\sqrt{-5}$$
with $r$ and $s$ rational numbers.\, When\, $\alpha = r+s\sqrt{-5}$\, belongs to $\mathfrak{O}$, the rational coefficients of the quadratic equation
$$x^2-2rx+(r^2+5s^2) = 0,$$
satisfied by $\alpha$, belong to the ring $\mathfrak{o}$, whence one has
$$\nu_0(-2r) \geqq 0, \quad \nu_0(r^2+5c^2) \geqq 0.$$
The first of these inequalities implies that\, $\nu_0(r) \geqq 0$\, since $-2$ is a unit of $\mathfrak{o}$.\, As for $s$, if one had\, $\nu_0(s) < 0$,\, then\, $\nu_0(5s^2) = 2\nu_0(s) < 0$,\, and therefore one had
$$\nu_0(r^2+5s^2) \;=\; \min\{\nu_0(r^2),\,\nu_0(5s^2)\} < 0.$$
Thus we have to have \,$\nu_0(s) \geqq 0$,\, too.\, So we have seen that for\, $r+s\sqrt{-5} \in \mathfrak{O}$,\, it's necessary that\, $r,\,s \in \mathfrak{o}$.\, The last condition is, apparently, also sufficient.\, Accordingly, we have obtained the result
$$\mathfrak{O} = \{r\!+\!s\sqrt{-5}\,\vdots\;\;\; r,\,s \in \mathfrak{o}\}.$$

Since the \PMlinkname{degree}{Degree} of the field extension $\mathbb{Q}(\sqrt{-5})/\mathbb{Q}$ is 2, the exponent $\nu_0$ has, by the theorem in the \PMlinkname{parent entry}{TheoremsOnContinuation}, at most two continuations to $\mathbb{Q}(\sqrt{-5})$.\, Moreover, the \PMlinkname{same entry}{TheoremsOnContinuation} implies that the intersection of the rings of those continuations coincides with $\mathfrak{O}$, whose \PMlinkname{non-associated}{Associate} prime elements determine the continuations in question.

We will show that there are exactly two of those continuations and that one may choose e.g. the conjugate numbers
$$\pi_1 := 1+\sqrt{-5}, \quad \pi_2 := 1-\sqrt{-5}$$
for such prime elements.

Suppose that $\pi_1$ splits in $\mathfrak{O}$ into \PMlinkname{factors}{DivisibilityInRings} as
$$\pi_1 = \alpha\beta$$
where\, $\alpha = a_0+a_1\sqrt{-1}$,\, $\beta = b_0+b_1\sqrt{-5}$\; ($a_i,\,b_i \in \mathfrak{o}$).\, Then also
$$\pi_2 = \alpha'\beta'$$
where\, $\alpha' = a_0-a_1\sqrt{-1}$,\, $\beta' = b_0-b_1\sqrt{-5}$.\, We perceive that 
$$\pi_1\pi_2 = 6 = \alpha\alpha' \cdot \beta\beta' = (a_0^2+5a_1^2)(b_0^2+5b_1^2),$$
but according to the entry ring of exponent, the only prime numbers of $\mathfrak{o}$ are the associates of 3.\, Now we have factorised the prime number $6$ of $\mathfrak{o}$ into a product of two \PMlinkname{factors}{Product} $\alpha\alpha'$ and $\beta\beta'$, and consequently, e.g. $\alpha\alpha'$ is a unit of $\mathfrak{o}$ and hence of $\mathfrak{O}$, too.\, Thus $\alpha$ and $\alpha'$ are units of $\mathfrak{O}$, which means that $\pi_1$ and $\pi_2$ have only trivial factors.\, The numbers $\pi_1$ and $\pi_2$ themselves are not units, because\, $\frac{1}{1\pm\sqrt{-5}} = 
\frac{1}{6}\mp\frac{1}{6}\sqrt{-5} \not\in \mathfrak{O}$; $\pi_1$ and $\pi_2$ are not associates of each other, since\, $\frac{\pi_1}{\pi_2} = 1+\frac{1}{3}\sqrt{-5} \not\in \mathfrak{O}$.\, So $\pi_1$ and $\pi_2$ are non-associated prime elements of $\mathfrak{O}$.\, This ring has no other prime elements non-associated with both $\pi_1$ and $\pi_2$, because otherwise $\nu_0$ would have more than two continuations.

According to the entry \PMlinkname{ring of exponent}{RingOfExponent}, any non-zero element of the field $\mathbb{Q}(\sqrt{-5})$ is uniquely \PMlinkescapetext{expressible} in the form
                    $$\xi = \varepsilon\pi_1^m\pi_2^n,$$
with $\varepsilon$ a unit of $\mathfrak{O}$ and  $m,\,n$ integers.\, The both continuations $\nu_1$ and $\nu_2$ of the triadic exponent $\nu_0$ are then determined as follows:
                  $$\nu_1(\xi) = m, \quad \nu_2(\xi) = n.$$

%%%%%
%%%%%
\end{document}
