\documentclass[12pt]{article}
\usepackage{pmmeta}
\pmcanonicalname{ChineseRemainderTheorem1}
\pmcreated{2013-03-22 12:16:43}
\pmmodified{2013-03-22 12:16:43}
\pmowner{bwebste}{988}
\pmmodifier{bwebste}{988}
\pmtitle{Chinese remainder theorem}
\pmrecord{7}{31729}
\pmprivacy{1}
\pmauthor{bwebste}{988}
\pmtype{Theorem}
\pmcomment{trigger rebuild}
\pmclassification{msc}{11N99}
\pmclassification{msc}{11A05}
\pmclassification{msc}{13A15}
\pmrelated{ChineseRemainderTheoremInTermsOfDivisorTheory}

% this is the default PlanetMath preamble.  as your knowledge
% of TeX increases, you will probably want to edit this, but
% it should be fine as is for beginners.

% almost certainly you want these
\usepackage{amssymb}
\usepackage{amsmath}
\usepackage{amsfonts}

% used for TeXing text within eps files
%\usepackage{psfrag}
% need this for including graphics (\includegraphics)
%\usepackage{graphicx}
% for neatly defining theorems and propositions
%\usepackage{amsthm}
% making logically defined graphics
%%%\usepackage{xypic} 

% there are many more packages, add them here as you need them

% define commands here
\begin{document}
Let $R$ be a commutative ring with identity.  If $I_1,\ldots,I_n$ are ideals of $R$ such that $I_i + I_j = R$ whenever $i\neq j$, then let $$I=\cap_{i=1}^n I_i = \prod_{i=1}^n I_i.$$ The sum of quotient maps $R/I\to R/I_i$ gives an isomorphism
$$R/I\cong \prod_{i=1}^n {R}/{I_i}.$$  This has the slightly weaker consequence that given a system of congruences $x\cong a_i\pmod{I_i}$, there is a solution in $R$ which is unique mod $I$, as the theorem is usually stated for the integers.
%%%%%
%%%%%
\end{document}
