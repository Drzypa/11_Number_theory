\documentclass[12pt]{article}
\usepackage{pmmeta}
\pmcanonicalname{ArtinMap}
\pmcreated{2013-03-22 12:34:55}
\pmmodified{2013-03-22 12:34:55}
\pmowner{djao}{24}
\pmmodifier{djao}{24}
\pmtitle{Artin map}
\pmrecord{9}{32831}
\pmprivacy{1}
\pmauthor{djao}{24}
\pmtype{Definition}
\pmcomment{trigger rebuild}
\pmclassification{msc}{11R37}
\pmrelated{RayClassField}
\pmdefines{Artin symbol}
\pmdefines{Frobenius automorphism}

\endmetadata

% this is the default PlanetMath preamble.  as your knowledge
% of TeX increases, you will probably want to edit this, but
% it should be fine as is for beginners.

% almost certainly you want these
\usepackage{amssymb}
\usepackage{amsmath}
\usepackage{amsfonts}

% used for TeXing text within eps files
%\usepackage{psfrag}
% need this for including graphics (\includegraphics)
%\usepackage{graphicx}
% for neatly defining theorems and propositions
\usepackage{amsthm}
% making logically defined graphics
%%%\usepackage{xypic} 

% there are many more packages, add them here as you need them

% define commands here

\newcommand{\p}{\mathfrak{p}}
\renewcommand{\P}{\mathfrak{P}}
\renewcommand{\O}{\mathcal{O}}
\newcommand{\lra}{\longrightarrow}
\newcommand{\Gal}{\operatorname{Gal}}
\newcommand{\Frob}{\operatorname{Frob}}

\newtheorem{theorem}{Theorem}
\newtheorem{proposition}[theorem]{Proposition}
\newtheorem{lemma}[theorem]{Lemma}
\newtheorem{corollary}[theorem]{Corollary}

\theoremstyle{definition}
\newtheorem{definition}[theorem]{Definition}
\begin{document}
Let $L/K$ be a Galois extension of number fields, with rings of integers $\O_L$ and $\O_K$. For any finite prime $\P \subset L$ lying over a prime $\p \in K$, let $D(\P)$ denote the decomposition group of $\P$, let $T(\P)$ denote the inertia group of $\P$, and let $l := \O_L/\P$ and $k := \O_K/\p$ be the residue fields. The exact sequence
$$
1 \lra T(\P) \lra D(\P) \lra \Gal(l/k) \lra 1
$$
yields an isomorphism $D(\P)/T(\P) \cong \Gal(l/k)$. In particular, there is a unique element in $D(\P)/T(\P)$, denoted $[L/K,\P]$, which maps to the $q^{\rm th}$ power Frobenius map $\Frob_q \in \Gal(l/k)$ under this isomorphism (where $q$ is the number of elements in $k$). The notation $[L/K,\P]$ is referred to as the {\em Artin symbol} of the extension $L/K$ at $\P$.

If we add the additional assumption that $\p$ is unramified, then $T(\P)$ is the trivial group, and $[L/K,\P]$ in this situation is an element of $D(\P) \subset \Gal(L/K)$, called the {\em Frobenius automorphism} of $\P$.

If, furthermore, $L/K$ is an abelian extension (that is, $\Gal(L/K)$ is an abelian group), then $[L/K,\P] = [L/K,\P']$ for any other prime $\P' \subset L$ lying over $\p$. In this case, the Frobenius automorphism $[L/K,\P]$ is denoted $(L/K,\p)$; the change in notation from $\P$ to $\p$ reflects the fact that the automorphism is determined by $\p \in K$ independent of which prime $\P$ of $L$ above it is chosen for use in the above construction.

\begin{definition}
Let $S$ be a finite set of primes of $K$, containing all the primes that ramify in $L$. Let $I_K^S$ denote the subgroup of the group $I_K$ of fractional ideals of $K$ which is generated by all the primes in $K$ that are not in $S$. The {\em Artin map}
$$
\phi_{L/K}: I_K^S \lra \Gal(L/K)
$$
is the map given by $\phi_{L/K}(\p) := (L/K,\p)$ for all primes $\p \notin S$, extended linearly to $I_K^S$.
\end{definition}
%%%%%
%%%%%
\end{document}
