\documentclass[12pt]{article}
\usepackage{pmmeta}
\pmcanonicalname{FermatMethod}
\pmcreated{2013-03-22 16:39:03}
\pmmodified{2013-03-22 16:39:03}
\pmowner{PrimeFan}{13766}
\pmmodifier{PrimeFan}{13766}
\pmtitle{Fermat method}
\pmrecord{6}{38855}
\pmprivacy{1}
\pmauthor{PrimeFan}{13766}
\pmtype{Algorithm}
\pmcomment{trigger rebuild}
\pmclassification{msc}{11A41}
\pmsynonym{Fermat's method}{FermatMethod}
\pmsynonym{Fermat factorization method}{FermatMethod}
\pmsynonym{Fermat's factorization method}{FermatMethod}

\endmetadata

% this is the default PlanetMath preamble.  as your knowledge
% of TeX increases, you will probably want to edit this, but
% it should be fine as is for beginners.

% almost certainly you want these
\usepackage{amssymb}
\usepackage{amsmath}
\usepackage{amsfonts}

% used for TeXing text within eps files
%\usepackage{psfrag}
% need this for including graphics (\includegraphics)
%\usepackage{graphicx}
% for neatly defining theorems and propositions
%\usepackage{amsthm}
% making logically defined graphics
%%%\usepackage{xypic}

% there are many more packages, add them here as you need them

% define commands here

\begin{document}
The {\em Fermat method} is a factorization algorithm for odd integers that tries to represent them as the difference of two square numbers.

Call the Fermat method algorithm with an integer $n = 2x + 1$.

\begin{enumerate}
\item Initialize the iterator $i = \lceil \sqrt{n} \rceil$ and test cap at $\sqrt{i^2 - n}$.
\item Compute $j = \sqrt{i^2 - n}$.
\item Test if $j$ is an integer. If it is, break out of subroutine and return $i - j$.
\item Increment iterator $i$ once. If $i < \frac{n + 9}{6}$, go to step 2.
\item Return $n$.
\end{enumerate}

For example, $n = 221$. The square root is approximately 15, so that's what we set our iterator's initial state to. The test cap is 39.

At $i = 15$, we find that $\sqrt{15^2 - 221} = 2$, clearly an integer.

Then, $15 - 2 = 13$ and $15 + 2 = 17$. By multiplication, we verify that $221 = 13 \cdot 17$, indeed. By trial division, this would have taken 15 test divisions.

However, there are integers for which trial division performs much better than the Fermat method, such as numbers of the form $3p$ where $p$ is an odd prime greater than 3.

\begin{thebibliography}{1}
\bibitem{rc} R. Crandall \& C. Pomerance, {\it Prime Numbers: A Computational Perspective}, Springer, NY, 2001: 5.1
\end{thebibliography}


%%%%%
%%%%%
\end{document}
