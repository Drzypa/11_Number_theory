\documentclass[12pt]{article}
\usepackage{pmmeta}
\pmcanonicalname{ProofOfLucassTheoremByBinomialExpansion}
\pmcreated{2013-03-22 18:19:59}
\pmmodified{2013-03-22 18:19:59}
\pmowner{whm22}{2009}
\pmmodifier{whm22}{2009}
\pmtitle{proof of Lucas's theorem by binomial expansion}
\pmrecord{4}{40965}
\pmprivacy{1}
\pmauthor{whm22}{2009}
\pmtype{Proof}
\pmcomment{trigger rebuild}
\pmclassification{msc}{11B65}

\endmetadata

% this is the default PlanetMath preamble.  as your knowledge
% of TeX increases, you will probably want to edit this, but
% it should be fine as is for beginners.

% almost certainly you want these
\usepackage{amssymb}
\usepackage{amsmath}
\usepackage{amsfonts}

% used for TeXing text within eps files
%\usepackage{psfrag}
% need this for including graphics (\includegraphics)
%\usepackage{graphicx}
% for neatly defining theorems and propositions
%\usepackage{amsthm}
% making logically defined graphics
%%%\usepackage{xypic}

% there are many more packages, add them here as you need them

% define commands here

\begin{document}
We work with polynomials in $x$ over the integers modulo $p$.  \newline By
the binomial theorem we have $(1+x)^p = 1 + x^p$.   More
generally, by induction on $i$ we have $(1+x)^{p^i} = 1+x^{p^i}$.


Hence the following holds:

$$
(1+x)^n=(1+x)^{\left[\sum_{i=0}^{k}a_i p^i \right ]} =
\prod_{i=0}^k (1+x^{p^i})^{a_i} = \prod_{i=0}^k \sum_{b=0}^{a_i}
{{a_i}\choose {b}} x^{bp^i}
$$

Then the coefficient on $x^m$ on the left hand side is $n \choose
m$.

As $m$ is uniquely \PMlinkescapetext{expressible} base $p$, the coefficient on $x^m$
on the right hand side is $\prod_{i=0}^k {a_i \choose b_i}$.

Equating the coefficients on $x^m$ on either \PMlinkescapetext{side} therefore yields
the result.

%%%%%
%%%%%
\end{document}
