\documentclass[12pt]{article}
\usepackage{pmmeta}
\pmcanonicalname{CalculatingTheJacobiSymbol}
\pmcreated{2013-03-22 14:55:05}
\pmmodified{2013-03-22 14:55:05}
\pmowner{mathwizard}{128}
\pmmodifier{mathwizard}{128}
\pmtitle{calculating the Jacobi symbol}
\pmrecord{4}{36604}
\pmprivacy{1}
\pmauthor{mathwizard}{128}
\pmtype{Algorithm}
\pmcomment{trigger rebuild}
\pmclassification{msc}{11A07}
\pmclassification{msc}{11A15}
\pmclassification{msc}{11Y99}

\endmetadata

% this is the default PlanetMath preamble.  as your knowledge
% of TeX increases, you will probably want to edit this, but
% it should be fine as is for beginners.

% almost certainly you want these
\usepackage{amssymb}
\usepackage{amsmath}
\usepackage{amsfonts}

% used for TeXing text within eps files
%\usepackage{psfrag}
% need this for including graphics (\includegraphics)
%\usepackage{graphicx}
% for neatly defining theorems and propositions
%\usepackage{amsthm}
% making logically defined graphics
%%%\usepackage{xypic}

% there are many more packages, add them here as you need them

% define commands here
\begin{document}
To calculate the Jacobi symbol $\left(\frac{a}{m}\right)$ for positive integers $a,m$, $m$ odd, we apply the quadratic reciprocity law and the fact that
$$\left(\frac{a}{m}\right)=\left(\frac{b}{m}\right)$$
if $a\equiv b\mod m$. So if $\gcd(a,m)=1$ (otherwise the Jacobi symbol is $0$) and $a>m$ choose $b\equiv a\mod m$ with $b<m$. Then $\left(\frac{a}{m}\right)=\left(\frac{b}{m}\right)$. To apply the quadratic reciprocity law, which basically allows us to exchange $b$ and $m$, we need to make $b$ odd if it is even. Writing $b=2^sc$ with odd $c$ does the trick since
$$\left(\frac{b}{m}\right)=\left(\frac{2}{m}\right)^s\left(\frac{c}{m}\right).$$
Using the fact that 
\begin{equation}\label{eq:suppl}
\left(\frac{2}{m}\right)=(-1)^\frac{m^2-1}{8}
\end{equation}
we compute $\left(\frac{2}{m}\right)$ to be $1$ if $m\equiv\pm 1\mod 8$ and $-1$ if $m\equiv\pm 3\mod 8$.
Now we apply the quadratic reciprocity law:
$$\left(\frac{c}{m}\right)=(-1)^\frac{(c-1)(m-1)}{4}\left(\frac{m}{c}\right).$$
The factor $(-1)^\frac{(c-1)(m-1)}{4}$ is equal to $-1$ if $c,m\equiv 3\mod 4$ and $1$ if at least one of the numbers $c,m$ is congruent to $1$ modulo $4$. At this point we can start thw whole procedure over again for $\left(\frac{m}{c}\right)$. Eventually we can apply the equation $\left(\frac{-1}{m}\right)=(-1)^\frac{m-1}{2}$, which is equal to $1$ if $m\equiv 1\mod 4$ and $-1$ otherwise.

\textbf{Example:} We try to calculate $\left(\frac{107}{23}\right)$. Since $\gcd(107,23)=1$ and $107\equiv15\mod23$ we find $\left(\frac{107}{23}\right)=\left(\frac{15}{23}\right)$. Since both $15$ and $23$ are congruent $3$ modulo $4$ we have 
$$\left(\frac{15}{23}\right)=-\left(\frac{23}{15}\right)=-\left(\frac{8}{15}\right)=-\left(\frac{2}{15}\right).$$ 
This can be evaluated using equation (\ref{eq:suppl}) and we find
$$\left(\frac{107}{23}\right)=-1.$$
%%%%%
%%%%%
\end{document}
