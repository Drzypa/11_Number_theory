\documentclass[12pt]{article}
\usepackage{pmmeta}
\pmcanonicalname{MersenneNumbersTwoSmallResultsOn}
\pmcreated{2013-03-22 15:07:53}
\pmmodified{2013-03-22 15:07:53}
\pmowner{CWoo}{3771}
\pmmodifier{CWoo}{3771}
\pmtitle{Mersenne numbers, two small results on}
\pmrecord{12}{36874}
\pmprivacy{1}
\pmauthor{CWoo}{3771}
\pmtype{Result}
\pmcomment{trigger rebuild}
\pmclassification{msc}{11A41}
%\pmkeywords{Mersenne number}
%\pmkeywords{prime factor}
%\pmkeywords{coprime}
\pmrelated{MersenneNumbers}

% This is Cosmin's PlanetMath preamble.

% Packages
\usepackage{amssymb}
\usepackage{amsmath}
\usepackage{amsfonts}
\usepackage{graphicx}
\usepackage{amsthm}
\usepackage{mathrsfs}
%%%\usepackage{xypic}

% Theorem Environments
\newtheorem*{thm}{Theorem}
\newtheorem*{lem}{Lemma}
\newtheorem*{cor}{Corollary}

% New Commands
  %Sets
    \newcommand{\bbP}{\mathbb{P}}
    \newcommand{\bbN}{\mathbb{N}}
    \newcommand{\bbZ}{\mathbb{Z}}
    \newcommand{\bbQ}{\mathbb{Q}}
    \newcommand{\bbR}{\mathbb{R}}
    \newcommand{\bbC}{\mathbb{C}}
    \newcommand{\bbK}{\mathbb{K}}
    \newcommand{\bbB}{\mathbb{B}}
    \newcommand{\bbS}{\mathbb{S}}
    \newcommand{\bbA}{\mathbb{A}}
    \newcommand{\bbT}{\mathbb{T}}
  %Script and Cal Letters
    \newcommand{\scP}{\mathscr{P}}
    \newcommand{\scF}{\mathscr{F}}
    \newcommand{\scC}{\mathscr{C}}
    \newcommand{\scL}{\mathscr{L}}
    \newcommand{\scS}{\mathscr{S}}
    \newcommand{\caA}{\mathcal{A}}
    \newcommand{\caB}{\mathcal{B}}
    \newcommand{\caC}{\mathcal{C}}
    \newcommand{\caD}{\mathcal{D}}
    \newcommand{\caE}{\mathcal{E}}
    \newcommand{\caF}{\mathcal{F}}
    \newcommand{\caR}{\mathcal{R}}
    \newcommand{\caP}{\mathcal{P}}
    \newcommand{\caM}{\mathcal{M}}
    \newcommand{\caS}{\mathcal{S}}
    \newcommand{\caH}{\mathcal{H}}
    \newcommand{\caT}{\mathcal{T}}
    \newcommand{\caU}{\mathcal{U}}
    \newcommand{\caX}{\mathcal{X}}
    \newcommand{\caY}{\mathcal{Y}}
    \newcommand{\caZ}{\mathcal{Z}}
  %Other Commands
    \newcommand{\vect}[1]{\boldsymbol{#1}}
    \renewcommand{\geq}{\geqslant}
    \renewcommand{\leq}{\leqslant}
    \renewcommand{\div}{\!\mid\!}
\begin{document}
\PMlinkescapeword{simple}
\PMlinkescapeword{gcd}
\PMlinkescapeword{infinity}
\PMlinkescapeword{odd}
\PMlinkescapeword{order}
This entry presents two simple results on Mersenne numbers\footnote{In this entry, the Mersenne numbers are indexed by the primes.}, namely that any two Mersenne numbers are relatively prime and that any prime dividing a Mersenne number \(M_p\) is greater than \(p\). We prove something slightly stronger for both these results:
\begin{thm} If \(q\) is a prime such that \(q\div M_p\), then \(p \div (q-1)\). \end{thm}
\begin{proof}
By definition of \(q\), we have \(2^p \equiv 1 \pmod q\). Since \(p\) is prime, this implies that \(2\) has order \(p\) in the multiplicative group \(\bbZ_q\mathbin{\setminus}\{0\}\) and, by Lagrange's Theorem, it divides the order of this \PMlinkname{group}{Group}, which is \(q-1\).
\end{proof}
\begin{thm} If \(m\) and \(n\) are relatively prime positive integers, then \(2^m-1\) and \(2^n-1\) are also relatively prime. \end{thm}
\begin{proof}
Let \(d := \gcd(2^n-1,2^m-1)\). Since \(d\) is odd, \(2\) is a unit in \(\bbZ_d\) and, since \(2^n \equiv 1 \pmod d\) and \(2^m \equiv 1 \pmod d\), the order of \(2\) divides both \(m\) and \(n\): it is \(1\). Thus \(2 \equiv 1 \pmod d\) and \(d=1\).
\end{proof}
Note that these two facts can be easily converted into proofs of the infinity of primes: indeed, the first one constructs a prime bigger than any prime \(p\) and the second easily implies that, if there were finitely many primes, every \(M_p\) (since there would be as many Mersenne numbers as primes) is a prime power, which is clearly false (consider \(M_{11} = 23\cdot89\)).
%%%%%
%%%%%
\end{document}
