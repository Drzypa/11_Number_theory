\documentclass[12pt]{article}
\usepackage{pmmeta}
\pmcanonicalname{Factorion}
\pmcreated{2013-03-22 17:43:52}
\pmmodified{2013-03-22 17:43:52}
\pmowner{CompositeFan}{12809}
\pmmodifier{CompositeFan}{12809}
\pmtitle{factorion}
\pmrecord{7}{40180}
\pmprivacy{1}
\pmauthor{CompositeFan}{12809}
\pmtype{Definition}
\pmcomment{trigger rebuild}
\pmclassification{msc}{11A63}
\pmclassification{msc}{05A10}

\endmetadata

% this is the default PlanetMath preamble.  as your knowledge
% of TeX increases, you will probably want to edit this, but
% it should be fine as is for beginners.

% almost certainly you want these
\usepackage{amssymb}
\usepackage{amsmath}
\usepackage{amsfonts}

% used for TeXing text within eps files
%\usepackage{psfrag}
% need this for including graphics (\includegraphics)
%\usepackage{graphicx}
% for neatly defining theorems and propositions
%\usepackage{amsthm}
% making logically defined graphics
%%%\usepackage{xypic}

% there are many more packages, add them here as you need them

% define commands here

\begin{document}
Given a base $b$ integer $$n = \sum_{i = 1}^k d_ib^{i - 1}$$ where $d_1$ is the least significant digit and $d_k$ is the most significant, if it is also the case that $$n = \sum_{i = 1}^k d_i!$$ then $n$ is a \emph{factorion}. In other words, the sum of the factorials of the digits in a standard positional integer base $b$ (such as base 10) gives the same number as multiplying the digits by the appropriate power of that base. With the exception of 1, the factorial base representation of a factorion is always different from that in the integer base. Obviously, all numbers are factorions in factorial base.

1 is a factorion in any integer base. 2 is a factorion in all integer bases except binary. In base 10, there are only four factorions: 1, 2, 145 and 40585. For example, $4 \times 10^4 + 0 \times 10^3 + 5 \times 10^2 + 8 \times 10^1 + 5 \times 10^0 = 4! + 0! + 5! + 8! + 5! = 40585$. (The factorial base representation of 40585 is 10021001).

\begin{thebibliography}{1}
\bibitem{dw} D. Wells, {\it The Penguin Dictionary of Curious and Interesting Numbers} London: Penguin Group. (1987): 125
\end{thebibliography}
%%%%%
%%%%%
\end{document}
