\documentclass[12pt]{article}
\usepackage{pmmeta}
\pmcanonicalname{ValueOfRiemannZetaFunctionAtS4}
\pmcreated{2013-03-22 18:22:06}
\pmmodified{2013-03-22 18:22:06}
\pmowner{pahio}{2872}
\pmmodifier{pahio}{2872}
\pmtitle{value of Riemann zeta function at $s = 4$}
\pmrecord{7}{41009}
\pmprivacy{1}
\pmauthor{pahio}{2872}
\pmtype{Example}
\pmcomment{trigger rebuild}
\pmclassification{msc}{11M06}
\pmrelated{SubstitutionNotation}
\pmrelated{CosineAtMultiplesOfStraightAngle}
\pmrelated{ValueOfTheRiemannZetaFunctionAtS2}

\endmetadata

% this is the default PlanetMath preamble.  as your knowledge
% of TeX increases, you will probably want to edit this, but
% it should be fine as is for beginners.

% almost certainly you want these
\usepackage{amssymb}
\usepackage{amsmath}
\usepackage{amsfonts}

% used for TeXing text within eps files
%\usepackage{psfrag}
% need this for including graphics (\includegraphics)
%\usepackage{graphicx}
% for neatly defining theorems and propositions
%\usepackage{amsthm}
% making logically defined graphics
%%%\usepackage{xypic}

% there are many more packages, add them here as you need them

% define commands here
\newcommand{\sijoitus}[2]%
{\operatornamewithlimits{\Big/}_{\!\!\!#1}^{\,#2}}
\begin{document}
By applying Parseval's identity (\PMlinkname{Lyapunov equation}{PersevalEquality}) to the Fourier series
$$\frac{a_0}{2}+(a_1\cos{x}+b_1\sin{x})+(a_2\cos{2x}+b_2\sin{2x})+\ldots$$
of $x^2$ on the interval \,$[-\pi,\,\pi]$,\, one may derive the value of Riemann zeta function at\, $s = 4$.\\

Let us first find the needed Fourier coefficients $a_n$ and $b_n$.\, Since $x^2$ defines an even function, we have
$$b_n = 0 \quad \forall\, n = 1,\,2,\,3,\,\ldots.$$
Then
$$a_0 = \frac{1}{\pi}\int_{-\pi}^\pi x^2\,dx = \frac{2}{\pi}\int_0^\pi x^2\,dx \,=\, \frac{2\pi^2}{3}.$$
For other coefficients $a_n$, we must perform twice integrations by parts:
\begin{align*}
a_n = \frac{1}{\pi}\int_{-\pi}^\pi x^2\cos{nx}\,dx & = \frac{2}{\pi}\int_0^\pi x^2\cos{nx}\,dx\\ 
& = \frac{2}{\pi}\left(\!\sijoitus{0}{\quad\pi}x^2\cdot\frac{\sin{nx}}{n}-\int_0^\pi2x\cdot\frac{\sin{nx}}{n}\,dx\right)\\
& = -\frac{4}{n\pi}\int_o^\pi x\sin{nx}\,dx\\
& = -\frac{4}{n\pi}\left(\sijoitus{0}{\quad\pi}x\cdot\frac{-\cos{nx}}{n}
-\int_0^\pi 1\cdot\frac{-\cos{nx}}{n}\,dx\right)\\
& = -\frac{4}{n\pi}\sijoitus{0}{\quad\pi}\left(\frac{-x\cos{nx}}{n}-\frac{\sin{nx}}{n^2}\right)\\
& = \frac{4\cos{n\pi}}{n^2} \;=\; \frac{4(-1)^n}{n^2} \quad \forall\, n = 1,\,2,\,3,\,\ldots
\end{align*}
Thus
$$x^2 \;=\; \frac{\pi^2}{3}+\sum_{n=1}^\infty\frac{4(-1)^n}{n^2}\cos{nx}\quad \mbox{for}\;\; -\pi \leqq x \leqq \pi.$$
The left hand side of Parseval's identity 
$$\frac{1}{2\pi}\int_{-\pi}^\pi(f(x))^2\,dx = \frac{a_0^2}{4}+\frac{1}{2}\sum_{n=1}^\infty(a_n^2+b_n^2)$$
reads now
$$\frac{1}{\pi}\int_0^\pi(x^2)^2\,dx = \frac{1}{\pi}\!\sijoitus{0}{\quad\pi}\frac{x^5}{5} = \frac{\pi^4}{5}$$
and its right hand side
$$\frac{1}{4}\!\left(\frac{2\pi^2}{3}\right)^2+\frac{1}{2}\sum_{n=1}^\infty\left(\frac{4}{n^2}\right)^2
= \frac{\pi^4}{9}+8\sum_{n=1}^\infty\frac{1}{n^4} = \frac{\pi^4}{9}+8\zeta(4).$$
Accordingly, we obtain the result
\begin{align}
\zeta(4) \;=\; 1+\frac{1}{2^4}+\frac{1}{3^4}+\ldots \;=\; \frac{\pi^4}{90}.
\end{align}



%%%%%
%%%%%
\end{document}
