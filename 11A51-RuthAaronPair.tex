\documentclass[12pt]{article}
\usepackage{pmmeta}
\pmcanonicalname{RuthAaronPair}
\pmcreated{2013-03-22 16:07:09}
\pmmodified{2013-03-22 16:07:09}
\pmowner{CompositeFan}{12809}
\pmmodifier{CompositeFan}{12809}
\pmtitle{Ruth-Aaron pair}
\pmrecord{5}{38187}
\pmprivacy{1}
\pmauthor{CompositeFan}{12809}
\pmtype{Definition}
\pmcomment{trigger rebuild}
\pmclassification{msc}{11A51}
\pmsynonym{Ruth Aaron pair}{RuthAaronPair}
\pmsynonym{Aaron-Ruth pair}{RuthAaronPair}
\pmsynonym{Aaron Ruth pair}{RuthAaronPair}
\pmdefines{Ruth-Aaron number}

% this is the default PlanetMath preamble.  as your knowledge
% of TeX increases, you will probably want to edit this, but
% it should be fine as is for beginners.

% almost certainly you want these
\usepackage{amssymb}
\usepackage{amsmath}
\usepackage{amsfonts}

% used for TeXing text within eps files
%\usepackage{psfrag}
% need this for including graphics (\includegraphics)
%\usepackage{graphicx}
% for neatly defining theorems and propositions
%\usepackage{amsthm}
% making logically defined graphics
%%%\usepackage{xypic}

% there are many more packages, add them here as you need them

% define commands here

\begin{document}
After Hank Aaron broke Babe Ruth's home run record of 714, Carl Pomerance noticed that the prime factors of 714 and 715 both add up to 29. Since then, a pair of integers $n$ and $m = n + 1$ is called a {\em Ruth-Aaron pair} if the prime factors add up to the same number, and individually $n$ is called a {\em Ruth-Aaron number} and so is $m$.

Given the factorizations of the numbers $$n = \prod_{i = 1}^{\omega(n)} {p_i}^{a_i}, m = \prod_{j = 1}^{\omega(m)} {q_j}^{b_j},$$ where the $p_i$ and $q_j$ are all distinct primes, and the $a_i$ and $b_j$ are positive integers (not necessarily distinct), and $\omega(x)$ is the number of distinct prime factors function, it becomes apparent that there are at least two different ways to sum up the prime factors.

The most obvious way is to simply test $$\sum_{i = 1}^{\omega(n)} p_i = \sum_{j = 1}^{\omega(m)} q_j,$$ in which case the first few Ruth-Aaron pairs are (5, 6), (24, 25), (49, 50), (77, 78), (104, 105), (153, 154), (369, 370), (492, 493), (714, 715), ... (see A006145 in the OEIS).

The second way is to count repeated prime factors as they occur (e.g., the sum of prime factors of $72 = 2^33^2$ is 12). Thus the test becomes $$\sum_{i = 1}^{\omega(n)} p_ia_i = \sum_{j = 1}^{\omega(m)} q_jb_j,$$ in which case the first few Ruth-Aaron pairs are (5, 6), (8, 9), (15, 16), (77, 78), (125, 126), (714, 715), (948, 949), ... (see A039752 in the OEIS).

The pairs (5, 6), (77, 78), (714, 715), ... work under either definition, and it can be observed that $|\mu(n)| = |\mu(m)| = 1$, where $\mu(x)$ is the M\"obius function (at least for $n < 1777028$).
%%%%%
%%%%%
\end{document}
